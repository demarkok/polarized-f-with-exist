\PassOptionsToPackage{prologue,dvipsnames}{xcolor}

% \usepackage{jfp-natbib-hack}
\documentclass{jfp}

\usepackage{lscape}
\usepackage{amsmath}
\usepackage{amsthm}
\usepackage{amssymb}
\usepackage{booktabs}
\usepackage{multicol}
\usepackage{supertabular}
\usepackage[inline]{enumitem}
\usepackage{cleveref}
\usepackage{proof}

\usepackage{stackengine}

\usepackage{mathabx}
\usepackage[dvipsnames]{xcolor}
\usepackage{scalerel}


\usepackage{todonotes}

\usepackage{enumitem}
\usepackage{xparse}
\usepackage{casenum}

\usepackage{braket}

\newcommand{\niton}{\not\owns}

\newcommand{\ilyam}[1]{{\color{red} \texttt{Ilya:  #1}}}

\newtheorem{definition}{Definition}
\newtheorem{theorem}{Theorem}
\newtheorem{lemma}{Lemma}
\newtheorem{corollary}{Corollary}
\newtheorem{observation}{Observation}
\newtheorem*{assertion*}{Assertion}

% https://tex.stackexchange.com/questions/85033/colored-symbols/85035#85035
\newcommand*{\mathcolor}{}
\def\mathcolor#1#{ \mathcoloraux{#1} }
\newcommand*{\mathcoloraux}[3]{%
  \protect\leavevmode
  \begingroup
  \color#1{#2}#3%
  \endgroup
}

\newcommand{\UB}[0]{\mathsf{UB}}
\newcommand{\NFUB}[0]{\mathsf{NFUB}}


% \newcounter{casenum}

% \newenvironment{caseof}
% {%
%   \par
%   \setlength{\parskip}{6pt}%
%   % \setlength{\parindent}{0pt}%
%   \everypar{\setlength{\hangindent}{17pt}}%
%   \setcounter{casenum}{0}%
% }
% {\par\vskip.5\baselineskip}

% \NewDocumentCommand{\case}{omm}{%
%   % \vskip.5\baselineskip\par%
%   \itemindent\parindent
%   \refstepcounter{casenum}%
%   {\bfseries Case} {\bfseries \arabic{casenum}}%
%   \IfNoValueF{#1}{\label{#1}}%
%   {\bfseries:} #2\\#3 %
% }


\newcommand{\ottDir}{../ott/_gen}
\newcommand{\genDir}{../paper/_gen}
% generated by Ott 0.32 from: grammar.ott rules.ott antiunification.ott
\documentclass[11pt]{article}
\usepackage{amsmath,amssymb}
\usepackage{supertabular}
\usepackage{geometry}
\usepackage{ifthen}
\usepackage{alltt}%hack
\geometry{a4paper,dvips,twoside,left=22.5mm,right=22.5mm,top=20mm,bottom=30mm}
\usepackage{color}
\newcommand{\ottdrule}[4][]{{\displaystyle\frac{\begin{array}{l}#2\end{array}}{#3}\quad\ottdrulename{#4}}}
\newcommand{\ottusedrule}[1]{\[#1\]}
\newcommand{\ottpremise}[1]{ #1 \\}
\newenvironment{ottdefnblock}[3][]{ \framebox{\mbox{#2}} \quad #3 \\[0pt]}{}
\newenvironment{ottfundefnblock}[3][]{ \framebox{\mbox{#2}} \quad #3 \\[0pt]\begin{displaymath}\begin{array}{l}}{\end{array}\end{displaymath}}
\newcommand{\ottfunclause}[2]{ #1 \equiv #2 \\}
\newcommand{\ottnt}[1]{\mathit{#1}}
\newcommand{\ottmv}[1]{\mathit{#1}}
\newcommand{\ottkw}[1]{\mathbf{#1}}
\newcommand{\ottsym}[1]{#1}
\newcommand{\ottcom}[1]{\text{#1}}
\newcommand{\ottdrulename}[1]{\textsc{#1}}
\newcommand{\ottcomplu}[5]{\overline{#1}^{\,#2\in #3 #4 #5}}
\newcommand{\ottcompu}[3]{\overline{#1}^{\,#2<#3}}
\newcommand{\ottcomp}[2]{\overline{#1}^{\,#2}}
\newcommand{\ottgrammartabular}[1]{\begin{supertabular}{llcllllll}#1\end{supertabular}}
\newcommand{\ottmetavartabular}[1]{\begin{supertabular}{ll}#1\end{supertabular}}
\newcommand{\ottrulehead}[3]{$#1$ & & $#2$ & & & \multicolumn{2}{l}{#3}}
\newcommand{\ottprodline}[6]{& & $#1$ & $#2$ & $#3 #4$ & $#5$ & $#6$}
\newcommand{\ottfirstprodline}[6]{\ottprodline{#1}{#2}{#3}{#4}{#5}{#6}}
\newcommand{\ottlongprodline}[2]{& & $#1$ & \multicolumn{4}{l}{$#2$}}
\newcommand{\ottfirstlongprodline}[2]{\ottlongprodline{#1}{#2}}
\newcommand{\ottbindspecprodline}[6]{\ottprodline{#1}{#2}{#3}{#4}{#5}{#6}}
\newcommand{\ottprodnewline}{\\}
\newcommand{\ottinterrule}{\\[5.0mm]}
\newcommand{\ottafterlastrule}{\\}

\newcommand{\appRightarrow}{ \mathcolor{OliveGreen}{\Rightarrow \hspace{-7pt} \Rightarrow} }

\newcommand{\tripprox}{\setbox0\hbox{$\approx$} \mbox{\makebox[0pt][l]{\raisebox{0.48\ht0}{$\approx$} }$\approx$} }

\newcommand{\approxRight}{ \mathrel{ \tripprox \hspace{-2.3pt}  \raisebox{0.24\ht0}{$>$} } }
\newcommand{\appBull}{ \mathcolor{OliveGreen}{\bullet} }
\newcommand{\rcolor}{blue}
\newcommand{\ccolor}{purple}

\usepackage{mathabx}
\usepackage{color}
\usepackage[dvipsnames,usenames]{xcolor}

% https://tex.stackexchange.com/questions/33401/a-version-of-colorbox-that-works-inside-math-environments
\setlength{\fboxsep}{1pt}
\newcommand{\ngbox}[1]{\mathchoice%
  {\colorbox{black!8}{$\displaystyle      \mathit{ #1 } $} }%
  {\colorbox{black!8}{$\textstyle         \mathit{ #1 } $} }%
  {\colorbox{black!8}{$\scriptstyle       \mathit{ #1 } $} }%
  {\colorbox{black!8}{$\scriptscriptstyle \mathit{ #1 } $} } }%

% https://tex.stackexchange.com/questions/85033/colored-symbols/85035#85035
\newcommand*{\mathcolor}{}
\def\mathcolor#1#{ \mathcoloraux{#1} }
\newcommand*{\mathcoloraux}[3]{%
  \protect\leavevmode
  \begingroup
    \color#1{#2}#3%
  \endgroup
}

\newcommand{\ottmetavars}{
\ottmetavartabular{
 $ \ottmv{x} ,\, \ottmv{y} $ & \ottcom{term variable} \\
 $ \ottmv{f} ,\, \ottmv{g} $ & \ottcom{constructors} \\
 $ \widehat{\alpha} ,\, \widehat{\beta} ,\, \widehat{\gamma} ,\, \widehat{\delta} $ & \ottcom{unification variable} \\
 $ \vec{x} ,\, \vec{y} ,\, \vec{z} ,\, \vec{t} $ & \ottcom{variable list} \\
 $ \ottmv{n} ,\, \ottmv{m} ,\, \ottmv{i} ,\, \ottmv{j} $ & \ottcom{index variables} \\
}}

\newcommand{\ottarn}{
\ottrulehead{n  ,\ k}{::=}{\ottcom{arity}}\ottprodnewline
\ottfirstprodline{|}{\ottsym{0}}{}{}{}{}\ottprodnewline
\ottprodline{|}{\ottsym{1}}{}{}{}{}\ottprodnewline
\ottprodline{|}{\ottsym{2}}{}{}{}{}\ottprodnewline
\ottprodline{|}{n_{{\mathrm{1}}}  \ottsym{+}  n_{{\mathrm{2}}}}{}{}{}{}\ottprodnewline
\ottprodline{|}{\ottsym{\mbox{$\mid$}}  \ottnt{vars}  \ottsym{\mbox{$\mid$}}}{}{}{}{}}

\newcommand{\ottvars}{
\ottrulehead{\ottnt{vars}}{::=}{\ottcom{variable list}}\ottprodnewline
\ottfirstprodline{|}{\vec{x}}{}{}{}{}\ottprodnewline
\ottprodline{|}{\ottmv{x_{{\mathrm{1}}}}  \ottsym{,} \, .. \, \ottsym{,}  \ottmv{x_{\ottmv{n}}}}{}{}{}{}\ottprodnewline
\ottprodline{|}{\ottnt{vars_{{\mathrm{1}}}}  \cap  \ottnt{vars_{{\mathrm{2}}}}}{}{}{}{}\ottprodnewline
\ottprodline{|}{\ottnt{vars_{{\mathrm{1}}}}  \sqcap  \ottnt{vars_{{\mathrm{2}}}}}{}{}{}{}\ottprodnewline
\ottprodline{|}{\ottcomp{\ottnt{vars_{\ottmv{i}}}}{\ottmv{i}}}{}{}{}{}\ottprodnewline
\ottprodline{|}{\ottkw{UVARGS} \, \ottnt{t}} {\textsf{M}}{}{}{\ottcom{arguments of the unification variables of the term}}}

\newcommand{\ottt}{
\ottrulehead{\ottnt{t}  ,\ \ottnt{v}  ,\ \ottnt{w}  ,\ \ottnt{h}  ,\ \ottnt{d}}{::=}{\ottcom{terms}}\ottprodnewline
\ottfirstprodline{|}{\ottmv{x}}{}{}{}{}\ottprodnewline
\ottprodline{|}{\ottmv{x}  \ottsym{.}  \ottnt{t}}{}{\textsf{bind}\; \ottmv{x}\; \textsf{in}\; \ottnt{t}}{}{}\ottprodnewline
\ottprodline{|}{\ottnt{vars}  \ottsym{.}  \ottnt{t}}{}{}{}{}\ottprodnewline
\ottprodline{|}{\widehat{\alpha}  \ottsym{[}  \ottnt{vars}  \ottsym{]}}{}{}{}{}\ottprodnewline
\ottprodline{|}{\ottmv{f}  \ottsym{(}  \ottnt{t_{{\mathrm{1}}}}  \ottsym{,..,}  \ottnt{t_{\ottmv{n}}}  \ottsym{)}}{}{}{}{}\ottprodnewline
\ottprodline{|}{\ottsym{[}  \Theta  \ottsym{]}  \ottnt{v}} {\textsf{M}}{}{}{}\ottprodnewline
\ottprodline{|}{\ottsym{(}  \ottnt{v}  \ottsym{)}} {\textsf{S}}{}{}{}\ottprodnewline
\ottprodline{|}{\ottsym{\{}  \widehat{\alpha}_{{\mathrm{1}}}  \ottsym{[}  \ottnt{vars_{{\mathrm{1}}}}  \ottsym{]}  \ottsym{/}  \widehat{\alpha}_{{\mathrm{2}}}  \ottsym{[}  \ottnt{vars_{{\mathrm{2}}}}  \ottsym{]}  \ottsym{\}}  \ottnt{t}}{}{}{}{}}

\newcommand{\ottterminals}{
\ottrulehead{\ottnt{terminals}}{::=}{}\ottprodnewline
\ottfirstprodline{|}{ \in }{}{}{}{}\ottprodnewline
\ottprodline{|}{ \notin }{}{}{}{}\ottprodnewline
\ottprodline{|}{ \cdot }{}{}{}{}\ottprodnewline
\ottprodline{|}{ \vdash }{}{}{}{}\ottprodnewline
\ottprodline{|}{ \mathcolor{\rcolor}{\vDash} }{}{}{}{}\ottprodnewline
\ottprodline{|}{ \mathcolor{\rcolor}{\Dashv} }{}{}{}{}\ottprodnewline
\ottprodline{|}{ \mathcolor{\ccolor}{\VDash} }{}{}{}{}\ottprodnewline
\ottprodline{|}{ \mathcolor{\ccolor}{\DashV} }{}{}{}{}\ottprodnewline
\ottprodline{|}{ \neq }{}{}{}{}\ottprodnewline
\ottprodline{|}{ \appRightarrow }{}{}{}{}\ottprodnewline
\ottprodline{|}{ \appBull }{}{}{}{}\ottprodnewline
\ottprodline{|}{ \mathcolor{\rcolor}{\equiv} }{}{}{}{}\ottprodnewline
\ottprodline{|}{ \equiv_{n} }{}{}{}{}\ottprodnewline
\ottprodline{|}{ \searrow }{}{}{}{}\ottprodnewline
\ottprodline{|}{ \unlhd }{}{}{}{}\ottprodnewline
\ottprodline{|}{ \cap }{}{}{}{}\ottprodnewline
\ottprodline{|}{ \sqcap }{}{}{}{}\ottprodnewline
\ottprodline{|}{ \subseteq }{}{}{}{}\ottprodnewline
\ottprodline{|}{ \emptyset }{}{}{}{}\ottprodnewline
\ottprodline{|}{ \approxRight }{}{}{}{}}

\newcommand{\ottT}{
\ottrulehead{\Theta}{::=}{\ottcom{computational variable context}}\ottprodnewline
\ottfirstprodline{|}{\ottmv{x}}{}{}{}{\ottcom{a variable}}\ottprodnewline
\ottprodline{|}{\vec{x}} {\textsf{S}}{}{}{\ottcom{variables}}\ottprodnewline
\ottprodline{|}{\widehat{\alpha}  \ottsym{:}  n}{}{}{}{\ottcom{a unification variable}}\ottprodnewline
\ottprodline{|}{\widehat{\alpha}  \ottsym{:}  n  \ottsym{=}  \ottnt{t}}{}{}{}{\ottcom{instantiate a unification variable}}\ottprodnewline
\ottprodline{|}{\ottcomp{\Theta_{\ottmv{i}}}{\ottmv{i}}}{}{}{}{\ottcom{concatenate contexts}}\ottprodnewline
\ottprodline{|}{\cdot}{}{}{}{\ottcom{empty context}}\ottprodnewline
\ottprodline{|}{\Theta_{{\mathrm{1}}}  \ottsym{\{}  \Theta_{{\mathrm{2}}}  \ottsym{\}}} {\textsf{S}}{}{}{\ottcom{surgery}}\ottprodnewline
\ottprodline{|}{\ottsym{(}  \Theta  \ottsym{)}} {\textsf{S}}{}{}{}\ottprodnewline
\ottprodline{|}{\Theta_{{\mathrm{1}}}  \ottsym{\mbox{$\backslash{}$}}  \ottsym{(}  \widehat{\alpha}_{{\mathrm{1}}}  \ottsym{,..,}  \widehat{\alpha}_{\ottmv{n}}  \ottsym{)}} {\textsf{S}}{}{}{\ottcom{context subtraction}}\ottprodnewline
\ottprodline{|}{ \Theta' ^{\color{red}\star} } {\textsf{M}}{}{}{\ottcom{context self-application}}}

\newcommand{\ottformula}{
\ottrulehead{\ottnt{formula}}{::=}{}\ottprodnewline
\ottfirstprodline{|}{\ottnt{judgement}}{}{}{}{}\ottprodnewline
\ottprodline{|}{\ottmv{x}  \in  \Theta}{}{}{}{\ottcom{lookup $\ottmv{x}$ in context $\Theta$}}\ottprodnewline
\ottprodline{|}{\widehat{\alpha}  \notin  \ottnt{t}}{}{}{}{}\ottprodnewline
\ottprodline{|}{\vec{x}  \subseteq  \Theta}{}{}{}{}\ottprodnewline
\ottprodline{|}{\ottkw{let} \, \Theta_{{\mathrm{1}}}  \ottsym{=}  \Theta_{{\mathrm{2}}}}{}{}{}{}\ottprodnewline
\ottprodline{|}{\ottkw{let} \, \vec{x}  \ottsym{=}  \ottnt{vars}}{}{}{}{}\ottprodnewline
\ottprodline{|}{\ottnt{vars}  \cap  \Theta  \ottsym{=}  \emptyset}{}{}{}{}\ottprodnewline
\ottprodline{|}{\ottnt{vars_{{\mathrm{1}}}}  \cap  \ottnt{vars_{{\mathrm{2}}}}  \ottsym{=}  \emptyset}{}{}{}{}\ottprodnewline
\ottprodline{|}{\ottkw{UV} \, \ottsym{(}  \ottnt{t}  \ottsym{)}  \ottsym{=}  \widehat{\alpha}_{{\mathrm{1}}}  \ottsym{[}  \ottnt{vars_{{\mathrm{1}}}}  \ottsym{]}  \ottsym{,..,}  \widehat{\alpha}_{\ottmv{n}}  \ottsym{[}  \ottnt{vars_{\ottmv{n}}}  \ottsym{]}}{}{}{}{}\ottprodnewline
\ottprodline{|}{\ottkw{ux} \, \ottsym{:}  n  \in  \Theta}{}{}{}{\ottcom{lookupof $\ottkw{ux}$ in context $\Theta$}}\ottprodnewline
\ottprodline{|}{\ottkw{ux} \, \ottsym{:}  n  \ottsym{=}  \ottnt{t}  \in  \Theta}{}{}{}{\ottcom{lookup type of $\ottkw{ux}$  instantiation in context $\Theta$}}\ottprodnewline
\ottprodline{|}{\ottnt{v}  \neq  \ottnt{w}}{}{}{}{}\ottprodnewline
\ottprodline{|}{\vec{x}  \ottsym{=}  \ottnt{vars}}{}{}{}{}\ottprodnewline
\ottprodline{|}{\ottkw{arity} \, \ottmv{f}  \ottsym{=}  \ottsym{[}  n_{{\mathrm{1}}}  \ottsym{,..,}  n_{\ottmv{n}}  \ottsym{]}}{}{}{}{}\ottprodnewline
\ottprodline{|}{\ottnt{formula_{{\mathrm{1}}}} \quad .. \quad \ottnt{formula_{\ottmv{n}}}}{}{}{}{}\ottprodnewline
\ottprodline{|}{ \cdots }{}{}{}{}}

\newcommand{\ottFoo}{
\ottrulehead{\ottnt{Foo}}{::=}{}\ottprodnewline
\ottfirstprodline{|}{ \Theta' ^{\color{red}\star} }{}{}{}{\ottcom{context self-application}}\ottprodnewline
\ottprodline{|}{\ottkw{UVARGS} \, \ottnt{t}  \ottsym{===}  \ottnt{vars}}{}{}{}{\ottcom{arguments of the unification variables of the term}}}

\newcommand{\ottAOne}{
\ottrulehead{\ottnt{A1}}{::=}{}\ottprodnewline
\ottfirstprodline{|}{\Theta_{{\mathrm{1}}}  \mathcolor{\rcolor}{\vDash}  \ottnt{v}  \mathcolor{\rcolor}{\equiv}  \ottnt{w}  \ottsym{:}  n  \mathcolor{\rcolor}{\Dashv}  \Theta_{{\mathrm{2}}}}{}{}{}{\ottcom{The unification}}}

\newcommand{\ottBOne}{
\ottrulehead{\ottnt{B1}}{::=}{}\ottprodnewline
\ottfirstprodline{|}{\Theta_{{\mathrm{1}}}  \mathcolor{\ccolor}{\VDash}  \ottnt{v}  \cap  \ottsym{[}  \ottnt{vars}  \ottsym{]}  \approxRight  \ottnt{w}  \mathcolor{\ccolor}{\DashV}  \Theta_{{\mathrm{2}}}}{}{}{}{\ottcom{The prunning phase}}\ottprodnewline
\ottprodline{|}{\Theta_{{\mathrm{1}}}  \mathcolor{\ccolor}{\VDash}  \ottnt{v}  \mathcolor{\rcolor}{\equiv}  \ottnt{w}  \mathcolor{\ccolor}{\DashV}  \Theta_{{\mathrm{2}}}}{}{}{}{\ottcom{The alternative unification}}\ottprodnewline
\ottprodline{|}{\ottnt{v} \, \ottkw{ext}}{}{}{}{\ottcom{The external term}}\ottprodnewline
\ottprodline{|}{\Theta \, \ottkw{ext}}{}{}{}{\ottcom{The external environment}}}

\newcommand{\ottjudgement}{
\ottrulehead{\ottnt{judgement}}{::=}{}\ottprodnewline
\ottfirstprodline{|}{\ottnt{A1}}{}{}{}{}\ottprodnewline
\ottprodline{|}{\ottnt{B1}}{}{}{}{}}

\newcommand{\ottuserXXsyntax}{
\ottrulehead{\ottnt{user\_syntax}}{::=}{}\ottprodnewline
\ottfirstprodline{|}{\ottmv{x}}{}{}{}{}\ottprodnewline
\ottprodline{|}{\ottmv{f}}{}{}{}{}\ottprodnewline
\ottprodline{|}{\widehat{\alpha}}{}{}{}{}\ottprodnewline
\ottprodline{|}{\vec{x}}{}{}{}{}\ottprodnewline
\ottprodline{|}{\ottmv{n}}{}{}{}{}\ottprodnewline
\ottprodline{|}{n}{}{}{}{}\ottprodnewline
\ottprodline{|}{\ottnt{vars}}{}{}{}{}\ottprodnewline
\ottprodline{|}{\ottnt{t}}{}{}{}{}\ottprodnewline
\ottprodline{|}{\ottnt{terminals}}{}{}{}{}\ottprodnewline
\ottprodline{|}{\Theta}{}{}{}{}\ottprodnewline
\ottprodline{|}{\ottnt{formula}}{}{}{}{}}

\newcommand{\ottgrammar}{\ottgrammartabular{
\ottarn\ottinterrule
\ottvars\ottinterrule
\ottt\ottinterrule
\ottterminals\ottinterrule
\ottT\ottinterrule
\ottformula\ottinterrule
\ottFoo\ottinterrule
\ottAOne\ottinterrule
\ottBOne\ottinterrule
\ottjudgement\ottinterrule
\ottuserXXsyntax\ottafterlastrule
}}

% defnss
% fundefns Foo
% fundefn simpl

\newcommand{\ottfundefnsimpl}[1]{\begin{ottfundefnblock}[#1]{$ \Theta' ^{\color{red}\star} $}{\ottcom{context self-application}}
\ottfunclause{ \cdot ^{\color{red}\star} }{\cdot}%
\ottfunclause{ \ottsym{(}  \Theta  \ottsym{,}  \ottmv{x}  \ottsym{)} ^{\color{red}\star} }{ \Theta ^{\color{red}\star}   \ottsym{,}  \ottmv{x}}%
\ottfunclause{ \ottsym{(}  \Theta  \ottsym{,}  \widehat{\alpha}  \ottsym{:}  n  \ottsym{)} ^{\color{red}\star} }{ \Theta ^{\color{red}\star}   \ottsym{,}  \widehat{\alpha}  \ottsym{:}  n}%
\ottfunclause{ \ottsym{(}  \Theta  \ottsym{,}  \widehat{\alpha}  \ottsym{:}  n  \ottsym{=}  \ottnt{t}  \ottsym{)} ^{\color{red}\star} }{ \Theta ^{\color{red}\star}   \ottsym{,}  \widehat{\alpha}  \ottsym{:}  n  \ottsym{=}  \ottsym{[}   \Theta ^{\color{red}\star}   \ottsym{]}  \ottnt{t}}%
\end{ottfundefnblock}}


% fundefn uvarargs

\newcommand{\ottfundefnuvarargs}[1]{\begin{ottfundefnblock}[#1]{$\ottkw{UVARGS} \, \ottnt{t}$}{\ottcom{arguments of the unification variables of the term}}
\end{ottfundefnblock}}


\newcommand{\ottfundefnsFoo}{
\ottfundefnsimpl{}
\ottfundefnuvarargs{}}

% defns A1
%% defn un
\newcommand{\ottdruleVXXV}[1]{\ottdrule[#1]{%
\ottpremise{\ottmv{x}  \in  \Theta}%
}{
\Theta  \mathcolor{\rcolor}{\vDash}  \ottmv{x}  \mathcolor{\rcolor}{\equiv}  \ottmv{x}  \ottsym{:}  \ottsym{0}  \mathcolor{\rcolor}{\Dashv}  \Theta}{%
{\ottdrulename{V\_V}}{}%
}}


\newcommand{\ottdruleBXXB}[1]{\ottdrule[#1]{%
\ottpremise{\Theta_{{\mathrm{1}}}  \ottsym{,}  \ottmv{x}  \mathcolor{\rcolor}{\vDash}  \ottnt{t}  \mathcolor{\rcolor}{\equiv}  \ottnt{t}  \ottsym{:}  n  \mathcolor{\rcolor}{\Dashv}  \Theta_{{\mathrm{2}}}  \ottsym{,}  \ottmv{x}}%
}{
\Theta_{{\mathrm{1}}}  \mathcolor{\rcolor}{\vDash}  \ottmv{x}  \ottsym{.}  \ottnt{t}  \mathcolor{\rcolor}{\equiv}  \ottmv{x}  \ottsym{.}  \ottnt{t}  \ottsym{:}  n  \ottsym{+}  \ottsym{1}  \mathcolor{\rcolor}{\Dashv}  \Theta_{{\mathrm{2}}}}{%
{\ottdrulename{B\_B}}{}%
}}


\newcommand{\ottdruleFXXF}[1]{\ottdrule[#1]{%
\ottpremise{\ottkw{arity} \, \ottmv{f}  \ottsym{=}  \ottsym{[}  k_{{\mathrm{1}}}  \ottsym{,..,}  k_{\ottmv{n}}  \ottsym{]}}%
\ottpremise{\Theta_{{\mathrm{0}}}  \mathcolor{\rcolor}{\vDash}  \ottnt{v_{{\mathrm{1}}}}  \mathcolor{\rcolor}{\equiv}  \ottnt{w_{{\mathrm{1}}}}  \ottsym{:}  k_{{\mathrm{1}}}  \mathcolor{\rcolor}{\Dashv}  \Theta_{{\mathrm{1}}}}%
\ottpremise{\Theta_{{\mathrm{1}}}  \mathcolor{\rcolor}{\vDash}  \ottsym{[}  \Theta_{{\mathrm{1}}}  \ottsym{]}  \ottnt{v_{{\mathrm{2}}}}  \mathcolor{\rcolor}{\equiv}  \ottsym{[}  \Theta_{{\mathrm{1}}}  \ottsym{]}  \ottnt{w_{{\mathrm{2}}}}  \ottsym{:}  k_{{\mathrm{2}}}  \mathcolor{\rcolor}{\Dashv}  \Theta_{{\mathrm{2}}}}%
\ottpremise{ \cdots }%
\ottpremise{\Theta_{{\ottmv{n}-1}}  \mathcolor{\rcolor}{\vDash}  \ottsym{[}  \Theta_{{\ottmv{n}-1}}  \ottsym{]}  \ottnt{v_{\ottmv{n}}}  \mathcolor{\rcolor}{\equiv}  \ottsym{[}  \Theta_{{\ottmv{n}-1}}  \ottsym{]}  \ottnt{w_{\ottmv{n}}}  \ottsym{:}  k_{\ottmv{n}}  \mathcolor{\rcolor}{\Dashv}  \Theta_{\ottmv{n}}}%
}{
\Theta_{{\mathrm{0}}}  \mathcolor{\rcolor}{\vDash}  \ottmv{f}  \ottsym{(}  \ottnt{v_{{\mathrm{1}}}}  \ottsym{,..,}  \ottnt{v_{\ottmv{n}}}  \ottsym{)}  \mathcolor{\rcolor}{\equiv}  \ottmv{f}  \ottsym{(}  \ottnt{w_{{\mathrm{1}}}}  \ottsym{,..,}  \ottnt{w_{\ottmv{n}}}  \ottsym{)}  \ottsym{:}  \ottsym{0}  \mathcolor{\rcolor}{\Dashv}  \Theta_{\ottmv{n}}}{%
{\ottdrulename{F\_F}}{}%
}}


\newcommand{\ottdruleUVXXV}[1]{\ottdrule[#1]{%
}{
\Theta  \ottsym{\{}  \widehat{\alpha}  \ottsym{:}  n  \ottsym{\}}  \mathcolor{\rcolor}{\vDash}  \widehat{\alpha}  \ottsym{[}  \vec{x}  \ottsym{]}  \mathcolor{\rcolor}{\equiv}  \ottmv{x_{\ottmv{i}}}  \ottsym{:}  \ottsym{0}  \mathcolor{\rcolor}{\Dashv}   \ottsym{(}  \Theta  \ottsym{\{}  \widehat{\alpha}  \ottsym{:}  n  \ottsym{=}  \vec{x}  \ottsym{.}  \ottmv{x_{\ottmv{i}}}  \ottsym{\}}  \ottsym{)} ^{\color{red}\star} }{%
{\ottdrulename{UV\_V}}{}%
}}


\newcommand{\ottdruleUVXXUV}[1]{\ottdrule[#1]{%
\ottpremise{\vec{z}  \ottsym{=}  \vec{x}  \sqcap  \vec{y}}%
}{
\Theta  \ottsym{\{}  \widehat{\alpha}  \ottsym{:}  n  \ottsym{\}}  \mathcolor{\rcolor}{\vDash}  \widehat{\alpha}  \ottsym{[}  \vec{x}  \ottsym{]}  \mathcolor{\rcolor}{\equiv}  \widehat{\alpha}  \ottsym{[}  \vec{y}  \ottsym{]}  \ottsym{:}  \ottsym{0}  \mathcolor{\rcolor}{\Dashv}   \ottsym{(}  \Theta  \ottsym{\{}  \widehat{\beta}  \ottsym{:}  \ottsym{\mbox{$\mid$}}  \vec{z}  \ottsym{\mbox{$\mid$}}  \ottsym{,}  \widehat{\alpha}  \ottsym{:}  n  \ottsym{=}  \vec{x}  \ottsym{.}  \widehat{\beta}  \ottsym{[}  \vec{z}  \ottsym{]}  \ottsym{\}}  \ottsym{)} ^{\color{red}\star} }{%
{\ottdrulename{UV\_UV}}{}%
}}


\newcommand{\ottdruleUVXXUVTwo}[1]{\ottdrule[#1]{%
\ottpremise{\vec{z}  \ottsym{=}  \vec{x}  \cap  \vec{y}}%
}{
\Theta_{{\mathrm{0}}}  \ottsym{,}  \widehat{\alpha}  \ottsym{:}  n  \ottsym{,}  \Theta_{{\mathrm{1}}}  \ottsym{,}  \widehat{\beta}  \ottsym{:}  k  \ottsym{,}  \Theta_{{\mathrm{2}}}  \mathcolor{\rcolor}{\vDash}  \widehat{\alpha}  \ottsym{[}  \vec{x}  \ottsym{]}  \mathcolor{\rcolor}{\equiv}  \widehat{\beta}  \ottsym{[}  \vec{y}  \ottsym{]}  \ottsym{:}  \ottsym{0}  \mathcolor{\rcolor}{\Dashv}   \ottsym{(}  \Theta_{{\mathrm{0}}}  \ottsym{,}  \ottsym{(}  \widehat{\gamma}  \ottsym{:}  \ottsym{\mbox{$\mid$}}  \vec{z}  \ottsym{\mbox{$\mid$}}  \ottsym{)}  \ottsym{,}  \ottsym{(}  \widehat{\alpha}  \ottsym{:}  n  \ottsym{=}  \vec{x}  \ottsym{.}  \widehat{\gamma}  \ottsym{[}  \vec{z}  \ottsym{]}  \ottsym{)}  \ottsym{,}  \Theta_{{\mathrm{1}}}  \ottsym{,}  \ottsym{(}  \widehat{\beta}  \ottsym{:}  k  \ottsym{=}  \vec{y}  \ottsym{.}  \widehat{\gamma}  \ottsym{[}  \vec{z}  \ottsym{]}  \ottsym{)}  \ottsym{,}  \Theta_{{\mathrm{2}}}  \ottsym{)} ^{\color{red}\star} }{%
{\ottdrulename{UV\_UV2}}{}%
}}


\newcommand{\ottdruleUVXXF}[1]{\ottdrule[#1]{%
\ottpremise{\widehat{\alpha}  \notin  \ottmv{f}  \ottsym{(}  \ottnt{t_{{\mathrm{1}}}}  \ottsym{,..,}  \ottnt{t_{\ottmv{m}}}  \ottsym{)}}%
\ottpremise{\ottkw{arity} \, \ottmv{f}  \ottsym{=}  \ottsym{[}  k_{{\mathrm{1}}}  \ottsym{,..,}  k_{\ottmv{n}}  \ottsym{]}}%
\ottpremise{\Theta_{{\mathrm{0}}}  \ottsym{\{}  \widehat{\beta}_{{\mathrm{1}}}  \ottsym{:}  n  \ottsym{+}  k_{{\mathrm{1}}}  \ottsym{,}  \widehat{\alpha}  \ottsym{:}  n  \ottsym{\}}  \mathcolor{\rcolor}{\vDash}  \vec{y}_{{\mathrm{1}}}  \ottsym{.}  \widehat{\beta}_{{\mathrm{1}}}  \ottsym{[}  \vec{x}  \ottsym{,}  \vec{y}_{{\mathrm{1}}}  \ottsym{]}  \mathcolor{\rcolor}{\equiv}  \ottnt{t_{{\mathrm{1}}}}  \ottsym{:}  k_{{\mathrm{1}}}  \mathcolor{\rcolor}{\Dashv}  \Theta_{{\mathrm{1}}}  \ottsym{\{}  \widehat{\alpha}  \ottsym{:}  n  \ottsym{\}}}%
\ottpremise{\Theta_{{\mathrm{1}}}  \ottsym{\{}  \widehat{\beta}_{{\mathrm{2}}}  \ottsym{:}  n  \ottsym{+}  k_{{\mathrm{2}}}  \ottsym{,}  \widehat{\alpha}  \ottsym{:}  n  \ottsym{\}}  \mathcolor{\rcolor}{\vDash}  \vec{y}_{{\mathrm{2}}}  \ottsym{.}  \widehat{\beta}_{{\mathrm{2}}}  \ottsym{[}  \vec{x}  \ottsym{,}  \vec{y}_{{\mathrm{2}}}  \ottsym{]}  \mathcolor{\rcolor}{\equiv}  \ottsym{[}  \Theta_{{\mathrm{1}}}  \ottsym{]}  \ottnt{t_{{\mathrm{2}}}}  \ottsym{:}  k_{{\mathrm{2}}}  \mathcolor{\rcolor}{\Dashv}  \Theta_{{\mathrm{2}}}  \ottsym{\{}  \widehat{\alpha}  \ottsym{:}  n  \ottsym{\}}}%
\ottpremise{ \cdots }%
\ottpremise{\Theta_{{\ottmv{n}-1}}  \ottsym{\{}  \widehat{\beta}_{\ottmv{m}}  \ottsym{:}  n  \ottsym{+}  k_{{\mathrm{2}}}  \ottsym{,}  \widehat{\alpha}  \ottsym{:}  n  \ottsym{\}}  \mathcolor{\rcolor}{\vDash}  \vec{y}_{\ottmv{m}}  \ottsym{.}  \widehat{\beta}_{\ottmv{m}}  \ottsym{[}  \vec{x}  \ottsym{,}  \vec{y}_{\ottmv{m}}  \ottsym{]}  \mathcolor{\rcolor}{\equiv}  \ottsym{[}  \Theta_{{\ottmv{m}-1}}  \ottsym{]}  \ottnt{t_{\ottmv{m}}}  \ottsym{:}  k_{\ottmv{m}}  \mathcolor{\rcolor}{\Dashv}  \Theta_{\ottmv{m}}  \ottsym{\{}  \widehat{\alpha}  \ottsym{:}  n  \ottsym{\}}}%
}{
\Theta_{{\mathrm{0}}}  \ottsym{\{}  \widehat{\alpha}  \ottsym{:}  n  \ottsym{\}}  \mathcolor{\rcolor}{\vDash}  \widehat{\alpha}  \ottsym{[}  \vec{x}  \ottsym{]}  \mathcolor{\rcolor}{\equiv}  \ottmv{f}  \ottsym{(}  \ottnt{t_{{\mathrm{1}}}}  \ottsym{,..,}  \ottnt{t_{\ottmv{m}}}  \ottsym{)}  \ottsym{:}  \ottsym{0}  \mathcolor{\rcolor}{\Dashv}   \ottsym{(}  \Theta_{\ottmv{m}}  \ottsym{\{}  \widehat{\alpha}  \ottsym{:}  n  \ottsym{=}  \vec{x}  \ottsym{.}  \ottmv{f}  \ottsym{(}  \vec{y}_{{\mathrm{1}}}  \ottsym{.}  \widehat{\beta}_{{\mathrm{1}}}  \ottsym{[}  \vec{x}  \ottsym{,}  \vec{y}_{{\mathrm{1}}}  \ottsym{]}  \ottsym{,..,}  \vec{y}_{\ottmv{n}}  \ottsym{.}  \widehat{\beta}_{\ottmv{n}}  \ottsym{[}  \vec{x}  \ottsym{,}  \vec{y}_{\ottmv{n}}  \ottsym{]}  \ottsym{)}  \ottsym{\}}  \ottsym{)} ^{\color{red}\star}   \ottsym{\mbox{$\backslash{}$}}  \ottsym{(}  \widehat{\beta}_{{\mathrm{1}}}  \ottsym{,..,}  \widehat{\beta}_{\ottmv{m}}  \ottsym{)}}{%
{\ottdrulename{UV\_F}}{}%
}}

\newcommand{\ottdefnun}[1]{\begin{ottdefnblock}[#1]{$\Theta_{{\mathrm{1}}}  \mathcolor{\rcolor}{\vDash}  \ottnt{v}  \mathcolor{\rcolor}{\equiv}  \ottnt{w}  \ottsym{:}  n  \mathcolor{\rcolor}{\Dashv}  \Theta_{{\mathrm{2}}}$}{\ottcom{The unification}}
\ottusedrule{\ottdruleVXXV{}}
\ottusedrule{\ottdruleBXXB{}}
\ottusedrule{\ottdruleFXXF{}}
\ottusedrule{\ottdruleUVXXV{}}
\ottusedrule{\ottdruleUVXXUV{}}
\ottusedrule{\ottdruleUVXXUVTwo{}}
\ottusedrule{\ottdruleUVXXF{}}
\end{ottdefnblock}}


\newcommand{\ottdefnsAOne}{
\ottdefnun{}}

% defns B1
%% defn prun
\newcommand{\ottdruleaprUV}[1]{\ottdrule[#1]{%
\ottpremise{\vec{z}  \ottsym{=}  \vec{y}  \cap  \vec{x}}%
}{
\Theta  \ottsym{\{}  \,  \ottsym{\}}  \ottsym{\{}  \widehat{\beta}  \ottsym{:}  n  \ottsym{\}}  \mathcolor{\ccolor}{\VDash}  \widehat{\beta}  \ottsym{[}  \vec{y}  \ottsym{]}  \cap  \ottsym{[}  \vec{x}  \ottsym{]}  \approxRight  \widehat{\beta}'  \ottsym{[}  \vec{z}  \ottsym{]}  \mathcolor{\ccolor}{\DashV}  \Theta  \ottsym{\{}  \widehat{\beta}'  \ottsym{:}  \ottsym{\mbox{$\mid$}}  \vec{z}  \ottsym{\mbox{$\mid$}}  \ottsym{\}}  \ottsym{\{}  \widehat{\beta}  \ottsym{:}  n  \ottsym{=}  \vec{y}  \ottsym{.}  \widehat{\beta}'  \ottsym{[}  \vec{z}  \ottsym{]}  \ottsym{\}}}{%
{\ottdrulename{aprUV}}{}%
}}


\newcommand{\ottdruleaprF}[1]{\ottdrule[#1]{%
}{
\Theta_{{\mathrm{0}}}  \ottsym{\{}  \,  \ottsym{\}}  \mathcolor{\ccolor}{\VDash}  \ottmv{f}  \ottsym{(}  \vec{y}_{{\mathrm{1}}}  \ottsym{.}  \ottnt{v_{{\mathrm{1}}}}  \ottsym{,..,}  \vec{y}_{\ottmv{n}}  \ottsym{.}  \ottnt{v_{\ottmv{n}}}  \ottsym{)}  \cap  \ottsym{[}  \vec{x}  \ottsym{]}  \approxRight  \widehat{\beta}'  \ottsym{[}  \vec{z}  \ottsym{]}  \mathcolor{\ccolor}{\DashV}  \Theta  \ottsym{\{}  \widehat{\beta}'  \ottsym{:}  \ottsym{\mbox{$\mid$}}  \vec{z}  \ottsym{\mbox{$\mid$}}  \ottsym{\}}  \ottsym{\{}  \widehat{\beta}  \ottsym{:}  n  \ottsym{=}  \vec{y}  \ottsym{.}  \widehat{\beta}'  \ottsym{[}  \vec{z}  \ottsym{]}  \ottsym{\}}}{%
{\ottdrulename{aprF}}{}%
}}

\newcommand{\ottdefnaprun}[1]{\begin{ottdefnblock}[#1]{$\Theta_{{\mathrm{1}}}  \mathcolor{\ccolor}{\VDash}  \ottnt{v}  \cap  \ottsym{[}  \ottnt{vars}  \ottsym{]}  \approxRight  \ottnt{w}  \mathcolor{\ccolor}{\DashV}  \Theta_{{\mathrm{2}}}$}{\ottcom{The prunning phase}}
\ottusedrule{\ottdruleaprUV{}}
\ottusedrule{\ottdruleaprF{}}
\end{ottdefnblock}}

%% defn un2
\newcommand{\ottdruleaVXXV}[1]{\ottdrule[#1]{%
\ottpremise{\ottmv{x}  \in  \Theta}%
}{
\Theta  \mathcolor{\ccolor}{\VDash}  \ottmv{x}  \mathcolor{\rcolor}{\equiv}  \ottmv{x}  \mathcolor{\ccolor}{\DashV}  \Theta}{%
{\ottdrulename{aV\_V}}{}%
}}


\newcommand{\ottdruleaFXXF}[1]{\ottdrule[#1]{%
\ottpremise{\ottkw{arity} \, \ottmv{f}  \ottsym{=}  \ottsym{[}  k_{{\mathrm{1}}}  \ottsym{,..,}  k_{\ottmv{n}}  \ottsym{]}}%
\ottpremise{\Theta_{{\mathrm{0}}}  \ottsym{,}  \vec{x}_{{\mathrm{1}}}  \mathcolor{\ccolor}{\VDash}  \ottnt{v_{{\mathrm{1}}}}  \mathcolor{\rcolor}{\equiv}  \ottnt{w_{{\mathrm{1}}}}  \mathcolor{\ccolor}{\DashV}  \Theta_{{\mathrm{1}}}  \ottsym{,}  \vec{x}_{{\mathrm{1}}}}%
\ottpremise{\Theta_{{\mathrm{1}}}  \ottsym{,}  \vec{x}_{{\mathrm{2}}}  \mathcolor{\ccolor}{\VDash}  \ottsym{[}  \Theta_{{\mathrm{1}}}  \ottsym{]}  \ottnt{v_{{\mathrm{2}}}}  \mathcolor{\rcolor}{\equiv}  \ottsym{[}  \Theta_{{\mathrm{1}}}  \ottsym{]}  \ottnt{w_{{\mathrm{2}}}}  \mathcolor{\ccolor}{\DashV}  \Theta_{{\mathrm{2}}}  \ottsym{,}  \vec{x}_{{\mathrm{2}}}}%
\ottpremise{ \cdots }%
\ottpremise{\Theta_{{\ottmv{n}-1}}  \ottsym{,}  \vec{x}_{\ottmv{n}}  \mathcolor{\ccolor}{\VDash}  \ottsym{[}  \Theta_{{\ottmv{n}-1}}  \ottsym{]}  \ottnt{v_{\ottmv{n}}}  \mathcolor{\rcolor}{\equiv}  \ottsym{[}  \Theta_{{\ottmv{n}-1}}  \ottsym{]}  \ottnt{w_{\ottmv{n}}}  \mathcolor{\ccolor}{\DashV}  \Theta_{\ottmv{n}}  \ottsym{,}  \vec{x}_{\ottmv{n}}}%
}{
\Theta_{{\mathrm{0}}}  \mathcolor{\ccolor}{\VDash}  \ottmv{f}  \ottsym{(}  \vec{x}_{{\mathrm{1}}}  \ottsym{.}  \ottnt{v_{{\mathrm{1}}}}  \ottsym{,..,}  \ottmv{x_{\ottmv{n}}}  \ottsym{.}  \ottnt{v_{\ottmv{n}}}  \ottsym{)}  \mathcolor{\rcolor}{\equiv}  \ottmv{f}  \ottsym{(}  \vec{x}_{{\mathrm{1}}}  \ottsym{.}  \ottnt{w_{{\mathrm{1}}}}  \ottsym{,..,}  \vec{x}_{\ottmv{n}}  \ottsym{.}  \ottnt{w_{\ottmv{n}}}  \ottsym{)}  \mathcolor{\ccolor}{\DashV}  \Theta_{\ottmv{n}}}{%
{\ottdrulename{aF\_F}}{}%
}}


\newcommand{\ottdruleaUVXXUV}[1]{\ottdrule[#1]{%
\ottpremise{\vec{z}  \ottsym{=}  \vec{x}  \sqcap  \vec{y}}%
}{
\Theta  \ottsym{\{}  \widehat{\alpha}  \ottsym{:}  n  \ottsym{\}}  \mathcolor{\ccolor}{\VDash}  \widehat{\alpha}  \ottsym{[}  \vec{x}  \ottsym{]}  \mathcolor{\rcolor}{\equiv}  \widehat{\alpha}  \ottsym{[}  \vec{y}  \ottsym{]}  \mathcolor{\ccolor}{\DashV}   \ottsym{(}  \Theta  \ottsym{\{}  \widehat{\beta}  \ottsym{:}  \ottsym{\mbox{$\mid$}}  \vec{z}  \ottsym{\mbox{$\mid$}}  \ottsym{,}  \widehat{\alpha}  \ottsym{:}  n  \ottsym{=}  \vec{x}  \ottsym{.}  \widehat{\beta}  \ottsym{[}  \vec{z}  \ottsym{]}  \ottsym{\}}  \ottsym{)} ^{\color{red}\star} }{%
{\ottdrulename{aUV\_UV}}{}%
}}


\newcommand{\ottdruleaUVXXF}[1]{\ottdrule[#1]{%
\ottpremise{\widehat{\alpha}  \notin  \ottnt{t}}%
\ottpremise{\Theta  \ottsym{\{}  \,  \ottsym{\}}  \mathcolor{\ccolor}{\VDash}  \ottnt{t}  \cap  \ottsym{[}  \vec{x}  \ottsym{]}  \approxRight  \ottnt{t'}  \mathcolor{\ccolor}{\DashV}  \Theta'  \ottsym{\{}  \widehat{\alpha}  \ottsym{:}  n  \ottsym{\}}}%
}{
\Theta  \mathcolor{\ccolor}{\VDash}  \widehat{\alpha}  \ottsym{[}  \vec{x}  \ottsym{]}  \mathcolor{\rcolor}{\equiv}  \ottnt{t}  \mathcolor{\ccolor}{\DashV}   \ottsym{(}  \Theta'  \ottsym{\{}  \widehat{\alpha}  \ottsym{:}  n  \ottsym{=}  \vec{x}  \ottsym{.}  \ottnt{t'}  \ottsym{\}}  \ottsym{)} ^{\color{red}\star} }{%
{\ottdrulename{aUV\_F}}{}%
}}

\newcommand{\ottdefnaunTwo}[1]{\begin{ottdefnblock}[#1]{$\Theta_{{\mathrm{1}}}  \mathcolor{\ccolor}{\VDash}  \ottnt{v}  \mathcolor{\rcolor}{\equiv}  \ottnt{w}  \mathcolor{\ccolor}{\DashV}  \Theta_{{\mathrm{2}}}$}{\ottcom{The alternative unification}}
\ottusedrule{\ottdruleaVXXV{}}
\ottusedrule{\ottdruleaFXXF{}}
\ottusedrule{\ottdruleaUVXXUV{}}
\ottusedrule{\ottdruleaUVXXF{}}
\end{ottdefnblock}}

%% defn ext
\newcommand{\ottdruleaV}[1]{\ottdrule[#1]{%
}{
\ottmv{x} \, \ottkw{ext}}{%
{\ottdrulename{aV}}{}%
}}


\newcommand{\ottdruleaUV}[1]{\ottdrule[#1]{%
}{
\widehat{\alpha}  \ottsym{[}  \vec{x}  \ottsym{]} \, \ottkw{ext}}{%
{\ottdrulename{aUV}}{}%
}}


\newcommand{\ottdruleaBind}[1]{\ottdrule[#1]{%
\ottpremise{\ottnt{t} \, \ottkw{ext}}%
}{
\ottmv{x}  \ottsym{.}  \ottnt{t} \, \ottkw{ext}}{%
{\ottdrulename{aBind}}{}%
}}


\newcommand{\ottdruleaConstr}[1]{\ottdrule[#1]{%
\ottpremise{\vec{x}_{{\mathrm{1}}}  \cap  \ottkw{UVARGS} \, \ottnt{t_{{\mathrm{1}}}}  \ottsym{=}  \emptyset \quad \ottnt{t_{{\mathrm{1}}}} \, \ottkw{ext}}%
\ottpremise{ \cdots }%
\ottpremise{\vec{x}_{\ottmv{n}}  \cap  \ottkw{UVARGS} \, \ottnt{t_{\ottmv{n}}}  \ottsym{=}  \emptyset \quad \ottnt{t_{\ottmv{n}}} \, \ottkw{ext}}%
}{
\ottmv{f}  \ottsym{(}  \vec{x}_{{\mathrm{1}}}  \ottsym{.}  \ottnt{t_{{\mathrm{1}}}}  \ottsym{,..,}  \vec{x}_{\ottmv{n}}  \ottsym{.}  \ottnt{t_{\ottmv{n}}}  \ottsym{)} \, \ottkw{ext}}{%
{\ottdrulename{aConstr}}{}%
}}

\newcommand{\ottdefnaext}[1]{\begin{ottdefnblock}[#1]{$\ottnt{v} \, \ottkw{ext}$}{\ottcom{The external term}}
\ottusedrule{\ottdruleaV{}}
\ottusedrule{\ottdruleaUV{}}
\ottusedrule{\ottdruleaBind{}}
\ottusedrule{\ottdruleaConstr{}}
\end{ottdefnblock}}

%% defn extC
\newcommand{\ottdruleaEmpty}[1]{\ottdrule[#1]{%
}{
\cdot \, \ottkw{ext}}{%
{\ottdrulename{aEmpty}}{}%
}}


\newcommand{\ottdruleaVar}[1]{\ottdrule[#1]{%
}{
\Theta  \ottsym{,}  \ottmv{x} \, \ottkw{ext}}{%
{\ottdrulename{aVar}}{}%
}}


\newcommand{\ottdruleaUVar}[1]{\ottdrule[#1]{%
}{
\Theta  \ottsym{,}  \widehat{\alpha}  \ottsym{:}  n \, \ottkw{ext}}{%
{\ottdrulename{aUVar}}{}%
}}


\newcommand{\ottdruleaUVarInst}[1]{\ottdrule[#1]{%
\ottpremise{\ottnt{t} \, \ottkw{ext}}%
}{
\Theta  \ottsym{,}  \widehat{\alpha}  \ottsym{:}  n  \ottsym{=}  \ottnt{t} \, \ottkw{ext}}{%
{\ottdrulename{aUVarInst}}{}%
}}

\newcommand{\ottdefnaextC}[1]{\begin{ottdefnblock}[#1]{$\Theta \, \ottkw{ext}$}{\ottcom{The external environment}}
\ottusedrule{\ottdruleaEmpty{}}
\ottusedrule{\ottdruleaVar{}}
\ottusedrule{\ottdruleaUVar{}}
\ottusedrule{\ottdruleaUVarInst{}}
\end{ottdefnblock}}


\newcommand{\ottdefnsBOne}{
\ottdefnaprun{}\ottdefnaunTwo{}\ottdefnaext{}\ottdefnaextC{}}

\newcommand{\ottdefnss}{
\ottfundefnsFoo
\ottdefnsAOne
\ottdefnsBOne
}

\newcommand{\ottall}{\ottmetavars\\[0pt]
\ottgrammar\\[5.0mm]
\ottdefnss}

\begin{document}
\ottall

\begin{verbatim}
Definition rules:        21 good    0 bad
Definition rule clauses: 51 good    0 bad
\end{verbatim}
\end{document}

\renewcommand{\ottrulehead}[3]{\multicolumn{4}{l}{#3}\ottprodnewline$#1$ & & $#2$ & & }
% \renewenvironment{ottdefnblock}[3][]{\noindent#3 \\\framebox{\mbox{#2}}\\[0pt]}{}



% ord varset in uN = varset'

\renewcommand{\ottdruleONVarInName}[0]{(Var$_{-\in}^{\text{Ord}}$)}
\renewcommand{\ottdruleONVarNInName}[0]{(Var$_{-\notin}^{\text{Ord}}$)}
\renewcommand{\ottdruleONUVarName}[0]{(UVar$_{-} ^{\text{Ord}}$)}
\renewcommand{\ottdruleOShiftUName}[0]{($\uparrow^{\text{Ord}}$)}
\renewcommand{\ottdruleOArrowName}[0]{($\rightarrow^{\text{Ord}}$)}
\renewcommand{\ottdruleOForallName}[0]{($\forall^{\text{Ord}}$)}


% ord varset in uP = varset'

\renewcommand{\ottdruleOPVarInName}[0]{(Var$_{+\in}^{\text{Ord}}$)}
\renewcommand{\ottdruleOPVarNInName}[0]{(Var$_{+\notin}^{\text{Ord}}$)}
\renewcommand{\ottdruleOPUVarName}[0]{(UVar$_{+}^{\text{Ord}}$)}
\renewcommand{\ottdruleOShiftDName}[0]{($\downarrow^{\text{Ord}}$)}
\renewcommand{\ottdruleOExistsName}[0]{($\exists^{\text{Ord}}$)}


% nf(N) = M
\renewcommand{\ottdruleNrmNVarName}[0]{(Var$_{-}^{\text{nf}}$)}
\renewcommand{\ottdruleNrmNUVarName}[0]{(UVar$_{-}^{\text{nf}}$)}
\renewcommand{\ottdruleNrmShiftUName}[0]{($\uparrow^{\text{nf}}$)}
\renewcommand{\ottdruleNrmArrowName}[0]{($\rightarrow^{\text{nf}}$)}
\renewcommand{\ottdruleNrmForallName}[0]{($\forall^{\text{nf}}$)}

% nf(P) = Q
\renewcommand{\ottdruleNrmPVarName}[0]{(Var$_{-}^{\text{nf}}$)}
\renewcommand{\ottdruleNrmPUVarName}[0]{(UVar$_{-}^{\text{nf}}$)}
\renewcommand{\ottdruleNrmShiftDName}[0]{($\downarrow^{\text{nf}}$)}
\renewcommand{\ottdruleNrmExistsName}[0]{($\exists^{\text{nf}}$)}


% N ≈ M

\renewcommand{\ottdruleEOneNVarName}[0]{(Var$_-^{\eqEOne}$)}
\renewcommand{\ottdruleEOneShiftUName}[0]{($\uparrow^{\eqEOne}$)}
\renewcommand{\ottdruleEOneArrowName}[0]{($\rightarrow^{\eqEOne}$)}
\renewcommand{\ottdruleEOneForallName}[0]{($\forall^{\eqEOne}$)}

% P ≈ Q
\renewcommand{\ottdruleEOnePVarName}[0]{(Var$_+^{\eqEOne}$)}
\renewcommand{\ottdruleEOneShiftDName}[0]{($\downarrow^{\eqEOne}$)}
\renewcommand{\ottdruleEOneExistsName}[0]{($\exists^{\eqEOne}$)}


% G ⊢ N ≤1 M

\renewcommand{\ottdruleDOneNVarName}[0]{(Var$_-^{\subDOne}$)}
\renewcommand{\ottdruleDOneShiftUName}[0]{($\uparrow^{\subDOne}$)}
\renewcommand{\ottdruleDOneArrowName}[0]{($\rightarrow^{\subDOne}$)}
\renewcommand{\ottdruleDOneForallName}[0]{($\forall^{\subDOne}$)}

% G ⊢ P ≥1 Q
\renewcommand{\ottdruleDOnePVarName}[0]{(Var$_+^{\supDOne}$)}
\renewcommand{\ottdruleDOneShiftDName}[0]{($\downarrow^{\supDOne}$)}
\renewcommand{\ottdruleDOneExistsName}[0]{($\exists^{\supDOne}$)}


% G ⊢ N ≈1 M
\renewcommand{\ottdruleDOneNDefName}[0]{($\eqDOneNeg$)}

% G ⊢ P ≈1 Q
\renewcommand{\ottdruleDOnePDefName}[0]{($\eqDOnePos$)}



% G ⊨ iP1 ∨ iP2 = iQ
\renewcommand{\ottdruleLUBVarName}[0]{(Var$^{\vee}$)}
\renewcommand{\ottdruleLUBShiftName}[0]{($\downarrow^{\vee}$)}
\renewcommand{\ottdruleLUBExistsName}[0]{($\exists^{\vee}$)}
\renewcommand{\ottdruleLUBUpgradeName}[0]{(Upg)}


% G ; Θ ⊨ uN ≤ iM ⫤ us
\renewcommand{\ottdruleANVarName}[0]{(Var$_-^{\subA}$)}
\renewcommand{\ottdruleAShiftUName}[0]{($\uparrow^{\subA}$)}
\renewcommand{\ottdruleAArrowName}[0]{($\rightarrow^{\subA}$)}
\renewcommand{\ottdruleAForallName}[0]{($\forall^{\subA}$)}

% G ; Θ ⊨ iP ≥ uQ ⫤ us
\renewcommand{\ottdruleAPVarName}[0]{(Var$_+^{\supA}$)}
\renewcommand{\ottdruleAShiftDName}[0]{($\downarrow^{\supA}$)}
\renewcommand{\ottdruleAExistsName}[0]{($\exists^{\supA}$)}
\renewcommand{\ottdruleAPUVarName}[0]{(UVar$^{\supA}$)}


% Γ ⊢ scE1 & scE2 = scE3 :: :: E :: 'E'
\renewcommand{\ottdruleSCMESupSupName}[0]{$([[≥]]\&^{+}[[≥]])$}
\renewcommand{\ottdruleSCMEEqSupName}[0]{$([[≈]]\&^{+}[[≥]])$}
\renewcommand{\ottdruleSCMESupEqName}[0]{$([[≥]]\&^{+}[[≈]])$}
\renewcommand{\ottdruleSCMEPEqEqName}[0]{$([[≈]]\&^{+}[[≈]])$}
\renewcommand{\ottdruleSCMENEqEqName}[0]{$([[≈]]\&^{-}[[≈]])$}

% Γ ; Θ ⊨ uN ≈u iM ⫤ UC 
\renewcommand{\ottdruleUNVarName}[0]{(Var$_{-}^{[[≈u]]}$)}
\renewcommand{\ottdruleUShiftUName}[0]{($\uparrow^{[[≈u]]}$)}
\renewcommand{\ottdruleUArrowName}[0]{($\rightarrow^{[[≈u]]}$)}
\renewcommand{\ottdruleUForallName}[0]{($\forall^{[[≈u]]}$)}
\renewcommand{\ottdruleUNUVarName}[0]{(UVar$_{-}^{[[≈u]]}$)}

% Γ ; Θ ⊨ uP ≈u iQ ⫤ UC 
\renewcommand{\ottdruleUPVarName}[0]{(Var$_{+}^{[[≈u]]}$)}
\renewcommand{\ottdruleUShiftDName}[0]{($\downarrow^{[[≈u]]}$)}
\renewcommand{\ottdruleUExistsName}[0]{($\exists^{[[≈u]]}$)}
\renewcommand{\ottdruleUPUVarName}[0]{(UVar$_{+}^{[[≈u]]}$)}

% G ⊨ iP1 ≈au iP2 ⫤ ( Ξ , uQ , aus1 , aus2 )
\renewcommand{\ottdruleAUPVarName}[0]{(Var$_{+}^{[[≈au]]}$)}
\renewcommand{\ottdruleAUShiftDName}[0]{($\downarrow^{[[≈au]]}$)}
\renewcommand{\ottdruleAUExistsName}[0]{($\exists^{[[≈au]]}$)}

% G ⊨ iN1 ≈au iN2 ⫤ ( Ξ , uM , aus1 , aus2 )
\renewcommand{\ottdruleAUNVarName}[0]{(Var$_{-}^{[[≈au]]}$)}
\renewcommand{\ottdruleAUShiftUName}[0]{($\uparrow^{[[≈au]]}$)}
\renewcommand{\ottdruleAUForallName}[0]{($\forall^{[[≈au]]}$)}
\renewcommand{\ottdruleAUArrowName}[0]{($\rightarrow^{[[≈au]]}$)}
\renewcommand{\ottdruleAUAUName}[0]{(AU)}

% Γ ⊢ iN 
\renewcommand{\ottdruleWFTNVarName}[0]{(Var$_{-}^{\text{WF}}$)}
\renewcommand{\ottdruleWFTShiftUName}[0]{($\uparrow^{\text{WF}}$)}
\renewcommand{\ottdruleWFTArrowName}[0]{($\rightarrow^{\text{WF}}$)}
\renewcommand{\ottdruleWFTForallName}[0]{($\forall^{\text{WF}}$)}

% Γ ⊢ iP
\renewcommand{\ottdruleWFTPVarName}[0]{(Var$_{+}^{\text{WF}}$)}
\renewcommand{\ottdruleWFTShiftDName}[0]{($\downarrow^{\text{WF}}$)}
\renewcommand{\ottdruleWFTExistsName}[0]{($\exists^{\text{WF}}$)}

% Γ ; Ξ ⊢ uN
\renewcommand{\ottdruleWFATNVarName}[0]{(Var$_{-}^{\text{WF}}$)}
\renewcommand{\ottdruleWFATNUVarName}[0]{(UVar$_{-}^{\text{WF}}$)}
\renewcommand{\ottdruleWFATShiftUName}[0]{($\uparrow^{\text{WF}}$)}
\renewcommand{\ottdruleWFATArrowName}[0]{($\rightarrow^{\text{WF}}$)}
\renewcommand{\ottdruleWFATForallName}[0]{($\forall^{\text{WF}}$)}

% Γ ; Ξ ⊢ uP
\renewcommand{\ottdruleWFATPVarName}[0]{(Var$_{+}^{\text{WF}}$)}
\renewcommand{\ottdruleWFATPUVarName}[0]{(UVar$_{+}^{\text{WF}}$)}
\renewcommand{\ottdruleWFATShiftDName}[0]{($\downarrow^{\text{WF}}$)}
\renewcommand{\ottdruleWFATExistsName}[0]{($\exists^{\text{WF}}$)}

% Γ ⊢ iP : scE
\renewcommand{\ottdruleSATSCESupName}[0]{($[[:≥]]_{+}^\text{sat}$)}
\renewcommand{\ottdruleSATSCEPEqName}[0]{($[[:≈]]_{+}^\text{sat}$)}

% Γ ⊢ iN : scE
\renewcommand{\ottdruleSATSCENEqName}[0]{($[[:≈]]_{-}^\text{sat}$)}

% Γ ; Φ ⊢ v : iP 
\renewcommand{\ottdruleDTVarName}[0]{(Var$^{\text{inf}}$)}
\renewcommand{\ottdruleDTThunkName}[0]{($\{\}^{\text{inf}}$)}
\renewcommand{\ottdruleDTPAnnotName}[0]{(ann$_+^{\text{inf}}$)}
\renewcommand{\ottdruleDTPEquivName}[0]{($[[≈]]_+^{\text{inf}}$)}

% Γ ; Φ ⊢ c : iN 
\renewcommand{\ottdruleDTtLamName}[0]{($\lambda^{\text{inf}}$)}
\renewcommand{\ottdruleDTTLamName}[0]{($\Lambda^{\text{inf}}$)}
\renewcommand{\ottdruleDTReturnName}[0]{(ret$^{\text{inf}}$)}
\renewcommand{\ottdruleDTVarLetName}[0]{(let$^{\text{inf}}$)}
\renewcommand{\ottdruleDTAppLetName}[0]{(let$_@^{\text{inf}}$)}
\renewcommand{\ottdruleDTAppLetAnnName}[0]{(let$_{:@}^{\text{inf}}$)}
\renewcommand{\ottdruleDTUnpackName}[0]{(let$_{\exists}^{\text{inf}}$)}
\renewcommand{\ottdruleDTNAnnotName}[0]{(ann$_-^{\text{inf}}$)}
\renewcommand{\ottdruleDTNEquivName}[0]{($[[≈]]_-^{\text{inf}}$)}

% Γ ; Φ ⊢ iN ● args ⇒> iM 
\renewcommand{\ottdruleDTEmptyAppName}[0]{($\emptyset_{[[●]][[⇒>]]}^{\text{inf}}$)}
\renewcommand{\ottdruleDTArrowAppName}[0]{($\rightarrow_{[[●]][[⇒>]]}^{\text{inf}}$)}
\renewcommand{\ottdruleDTForallAppName}[0]{($\forall_{[[●]][[⇒>]]}^{\text{inf}}$)}



% Γ ; Φ ⊨ v : iP  
\renewcommand{\ottdruleATVarName}[0]{(Var$^{\text{inf}}$)}
\renewcommand{\ottdruleATThunkName}[0]{($\{\}^{\text{inf}}$)}
\renewcommand{\ottdruleATPAnnotName}[0]{(ann$_+^{\text{inf}}$)}

% Γ ; Φ ⊨ c : iN 
\renewcommand{\ottdruleATtLamName}[0]{($\lambda^{\text{inf}}$)}
\renewcommand{\ottdruleATTLamName}[0]{($\Lambda^{\text{inf}}$)}
\renewcommand{\ottdruleATReturnName}[0]{(ret$^{\text{inf}}$)}
\renewcommand{\ottdruleATVarLetName}[0]{(let$^{\text{inf}}$)}
\renewcommand{\ottdruleATAppLetName}[0]{(let$_@^{\text{inf}}$)}
\renewcommand{\ottdruleATAppLetAnnName}[0]{(let$_{:@}^{\text{inf}}$)}
\renewcommand{\ottdruleATUnpackName}[0]{(let$_{\exists}^{\text{inf}}$)}
\renewcommand{\ottdruleATNAnnotName}[0]{(ann$_-^{\text{inf}}$)}

% Γ ; Φ ; Θ1 ⊨ uN ● args ⇒> uM ⫤ Θ2 ; SC 
\renewcommand{\ottdruleATEmptyAppName}[0]{($\emptyset_{[[●]][[⇒>]]}^{\text{inf}}$)}
\renewcommand{\ottdruleATArrowAppName}[0]{($\rightarrow_{[[●]][[⇒>]]}^{\text{inf}}$)}
\renewcommand{\ottdruleATForallAppName}[0]{($\forall_{[[●]][[⇒>]]}^{\text{inf}}$)}

% scE1 singular with iP
\renewcommand{\ottdruleSINGPEqName}[0]{($[[≈]]_{+}^{\text{sing}}$)}
\renewcommand{\ottdruleSINGSupVarName}[0]{($[[:≥]]\alpha^{\text{sing}}$)}
\renewcommand{\ottdruleSINGSupShiftName}[0]{($[[:≥]][[↓]]^{\text{sing}}$)}

% scE1 singular with iN
\renewcommand{\ottdruleSINGNEqName}[0]{($[[≈]]_{-}^{\text{sing}}$)}




% \usepackage{../bibliography}

\bibliographystyle{jfplike}

\begin{document}

\journaltitle{JFP}
\cpr{Cambridge University Press}
\doival{10.1017/xxxxx}

\lefttitle{Local Type Inference for Polarised System F with Existentials}
\righttitle{Journal of Functional Programming}

\totalpg{\pageref{lastpage01}}
\jnlDoiYr{2024}

% \title{Journal of Functional Programming: \LaTeX\ Guidelines for~authors}
\title{Local Type Inference for Polarised System F with Existentials}

\begin{authgrp}

\author{Ilya Kaysin}
% \orcid{0000-0002-6301-152X}
\affiliation{University of Cambridge \email{ik404@cam.ac.uk}}

\author{Neel Krishnaswami}
% \orcid{0000-0003-2838-5865}
\affiliation{University of Cambridge \email{nk480@cl.cam.ac.uk}}
\end{authgrp}

% \renewcommand{\shortauthors}{Kaysin et al.}

\begin{abstract}
    We study type inference for the calculus \fexists, which is a version of
    \CBPV extended with support for fully impredicative existential and universal 
    quantifiers. We give a local type inference algorithm for this calculus
    as well as a declarative type system acting as a specification for the algorithm
    and show that the algorithm is sound and complete with respect to the declarative
    specification. The inclusion of existential quantifiers is unusual and supporting it
    required us to use a number of novel techniques including the
    combination of unification and anti-unification as part of the inference algorithm.
\end{abstract}



%%
%% The code below is generated by the tool at http://dl.acm.org/ccs.cfm.
%% Please copy and paste the code instead of the example below.
%%
% \begin{CCSXML}
%   <ccs2012>
%      <concept>
%          <concept_id>10003752.10003790.10011740</concept_id>
%          <concept_desc>Theory of computation~Type theory</concept_desc>
%          <concept_significance>500</concept_significance>
%          </concept>
%      <concept>
%          <concept_id>10011007.10011006.10011008.10011009.10011012</concept_id>
%          <concept_desc>Software and its engineering~Functional languages</concept_desc>
%          <concept_significance>300</concept_significance>
%          </concept>
%      <concept>
%          <concept_id>10011007.10011006.10011008.10011024.10011025</concept_id>
%          <concept_desc>Software and its engineering~Polymorphism</concept_desc>
%          <concept_significance>500</concept_significance>
%          </concept>
%      <concept>
%          <concept_id>10011007.10011006.10011008.10011009.10011015</concept_id>
%          <concept_desc>Software and its engineering~Constraint and logic languages</concept_desc>
%          <concept_significance>100</concept_significance>
%          </concept>
%      <concept>
%          <concept_id>10011007.10011006.10011008.10011009.10011011</concept_id>
%          <concept_desc>Software and its engineering~Object oriented languages</concept_desc>
%          <concept_significance>100</concept_significance>
%          </concept>
%      <concept>
%          <concept_id>10011007.10011006.10011008.10011024.10011032</concept_id>
%          <concept_desc>Software and its engineering~Constraints</concept_desc>
%          <concept_significance>300</concept_significance>
%          </concept>
%      <concept>
%          <concept_id>10011007.10011006.10011008.10011024.10011031</concept_id>
%          <concept_desc>Software and its engineering~Modules / packages</concept_desc>
%          <concept_significance>300</concept_significance>
%          </concept>
%      <concept>
%          <concept_id>10011007.10011006.10011039.10011311</concept_id>
%          <concept_desc>Software and its engineering~Semantics</concept_desc>
%          <concept_significance>300</concept_significance>
%          </concept>
%      <concept>
%          <concept_id>10003752.10003753.10003754.10003733</concept_id>
%          <concept_desc>Theory of computation~Lambda calculus</concept_desc>
%          <concept_significance>500</concept_significance>
%          </concept>
%      <concept>
%          <concept_id>10003752.10003766.10003767</concept_id>
%          <concept_desc>Theory of computation~Formalisms</concept_desc>
%          <concept_significance>500</concept_significance>
%          </concept>
%      <concept>
%          <concept_id>10003752.10003809.10010031.10010032</concept_id>
%          <concept_desc>Theory of computation~Pattern matching</concept_desc>
%          <concept_significance>300</concept_significance>
%          </concept>
%      <concept>
%          <concept_id>10003752.10010124.10010125.10010130</concept_id>
%          <concept_desc>Theory of computation~Type structures</concept_desc>
%          <concept_significance>500</concept_significance>
%          </concept>
%      <concept>
%          <concept_id>10003752.10010070.10010071.10010078</concept_id>
%          <concept_desc>Theory of computation~Inductive inference</concept_desc>
%          <concept_significance>500</concept_significance>
%          </concept>
%    </ccs2012>
%   \end{CCSXML}
  
  % \ccsdesc[500]{Theory of computation~Type theory}
  % \ccsdesc[300]{Software and its engineering~Functional languages}
  % \ccsdesc[500]{Software and its engineering~Polymorphism}
  % \ccsdesc[100]{Software and its engineering~Constraint and logic languages}
  % \ccsdesc[100]{Software and its engineering~Object oriented languages}
  % \ccsdesc[300]{Software and its engineering~Constraints}
  % \ccsdesc[300]{Software and its engineering~Modules / packages}
  % \ccsdesc[300]{Software and its engineering~Semantics}
  % \ccsdesc[500]{Theory of computation~Lambda calculus}
  % \ccsdesc[500]{Theory of computation~Formalisms}
  % \ccsdesc[300]{Theory of computation~Pattern matching}
  % \ccsdesc[500]{Theory of computation~Type structures}
  % \ccsdesc[500]{Theory of computation~Inductive inference}

%%
%% Keywords. The author(s) should pick words that accurately describe
%% the work being presented. Separate the keywords with commas.
% \keywords{Type Inference, System F, Call-by-Push-Value, Polarized Typing, Focalisation, Subtyping}



%%
%% This command processes the author and affiliation and title
%% information and builds the first part of the formatted document.
\maketitle

\section{Introduction}
Over the last half-century, there has been considerable work on developing type
inference algorithms for programming languages,  mostly aimed at solving the
problem of \emph{type polymorphism}.

That is, in pure polymorphic lambda
calculus---\systemf\cite{girard-system-f,jcr-system-f}---the polymorphic type
$[[∀a.fA]]$ has a big lambda $[[Λa.fe]]$ as an introduction form, and an
explicit type application $[[ fe[fA] ]]$ as an elimination form. This is an
extremely simple and compact type system, whose rules fit on a single page, but
whose semantics are sophisticated enough to model things like parametricity and
representation independence. However, System F by itself is unwieldy as a
programming language. The fact that the universal type $[[∀a.fA]]$ has explicit
eliminations means that programs written using polymorphic types will need to be
stuffed to the gills with type annotations explaining how and when to
instantiate the quantifiers.

Therefore, most work on type inference has been aimed at handling type
instantiations implicitly---we want to be able to use a polymorphic function
like $[[len : ∀a.List a → Int]]$ at any concrete list type without explicitly
instantiating the quantifier in $[[len]]$'s type. That is, we want to write
$[[ len [one,two,three] ]]$ instead of writing
$[[ len [Int] [one,two,three] ]]$.

The most famous of the algorithms for solving these constraints is the
Damas-Hindley-Milner (HDM) algorithm
\cite{hindley69:principal,milner78:theory,damas82:principal}. The idea is that
type instantiation induces a subtype ordering: the type $[[∀a.fA]]$ is a subtype
of all of its instantiations. So we wish to be free to use the same function
$[[len]]$ at many different types such as 
$[[List Int → Int]]$, $[[List Bool → Int]]$,
$[[List (Int × Bool) → Int]]$, and so on.
However, the subtype relation is nondeterministic: it tells us that whenever we
see a polymorphic type $[[∀a.fA]]$, we know it is a subtype of \emph{any}
of its instantiations. To turn this into an algorithm, we have to actually make
some choices, and DHM works by using \emph{unification}. Whenever we would have
had to introduce a particular concrete type in the specification, the DHM
algorithm introduces a unification variable, and incrementally instantiates this
variable as more and more type constraints are observed.


                 


However, the universal quantifier is not the only quantifier! Dual to the
universal quantifier $\forall$ is the existential quantifier $\exists$. Even
though existential types have an equal logical status to universal types, they
have been studied much less frequently in the literature. The most widely-used
algorithm for handling existentials is the algorithm of Odersky and Laufer
\cite{laufer94:polymorphic}. This algorithm treats existentials in a
second-class fashion: they are not explicit connectives of the logic but rather
are tied to datatype declarations. As a result, packing and unpacking
existential types is tied to constructor introductions and constructor
eliminations in pattern matches. This allows Damas-Milner inference to continue
to work almost unchanged, but it does come at the cost of losing first-class
existentials and also of restricting the witness types to monomorphic types.

There has been a limited amount of work on support for existential types as a
first-class concept. In an unpublished draft, \citet{leijen06:first-class}
studied an extension of Damas-Milner inference in which type schemes contain
alternating universal and existential quantifiers. Quantifiers still range over
monotypes, and higher-rank polymorphism is not permitted. More recently,
\citet{dunfield16:existential} studied type inference for existential types in
the context of GADT type inference, which, while still predicative, supported
higher-rank types (i.e., quantifiers can occur anywhere inside of a type
scheme). \citet{eisenberg21:existential} propose a system which only permits
types of the form forall-exists, but which permits projective elimination in the
style of ML modules. 

All of these papers are restricted to \emph{predicative} quantification, where
quantifiers can only be instantiated with monotypes (i.e., types without any
occurences of quantifiers). However, existential types in full System F are
\emph{impredicative}---that is, quantifiers can be instantiated with arbitrary
types, specifically including types containing quantifiers.

Historically, inference for impredicative quantification has been neglected, due
to results by \citet{tiuryn-urzczyn-96} and \citet{chrzaszcz-98} which show that
not only is full type inference for \systemf undecidable, but even that the
subtyping relation induced by type instantiation is undecidable. However, in
recent years, interest in impredicative inference has revived (for example, the
work of \citet{serrano-2020}), with a focus on avoiding the undecidability
barriers by doing \emph{partial} type inference. That is, we no longer try to do
full type inference, but rather accept that the programmer will need to write
annotations in cases where inference would be too difficult. Unfortunately, it
is often difficult to give a specification for what the partial algorithm
does--for example, \citet{eisenberg21:existential} observe that their algorithm
lacks a declarative specification, and explain why existential types make this
particularly difficult to do. 

One especially well-behaved form of partial type inference is \emph{local type
inference}, introduced by \citet{pierce2000:local}. In this approach,
quantifiers in a polymorphic function are instantiated using only the type
information available in the arguments at each call site. While this infers
fewer quantifiers than global algorithms such as Damas-Milner can, it has the
twin benefits that it is easy to give a mathematical specification to, and that
failures of type inference are easily explained in terms of local features of
the call site. In fact, many production programming languages such as C\# and
Java use a form of local type inference, due to the ease of extending this style
of inference to support OO features. 

In this paper, we extend the local type inference algorithm to a language with
both universal and existential quantifiers, which can both be instantiated
impredicatively. This combination of features broke a number of the invariants
which traditional type inference algorithms depend on, and required us to invent
new algorithms which combine both unification and (surprisingly)
anti-unification. 

In this paper, we extend the local type inference algorithm to a language with
both universal and existential quantifiers, which can both be instantiated
impredicatively. This extended system is referred to as \fexists. The
combination of features in \fexists  broke a number of the invariants which
traditional type inference algorithms depend on and required us to invent
new algorithms which combine both unification and (surprisingly)
anti-unification.

To research the type inference in impredicative polymorphic systems
in the presence of existential types, we make the following \emph{contributions}.

\paragraph{Type Inference Algorithm} We give a local type inference algorithm
    which supports both first-class existential and universal quantifiers, both
    of which can be instantiated impredicatively. To evade the undecidability
    results surrounding type inference for \systemf, we introduce \fexists---a
    variant of \CBPV \cite{levy2006:cbpv}, which lets us formulate a subtyping
    relation which is still decidable. 

\paragraph{Declarative Specification}
    We give a declarative type system (again, based on \CBPV) to
    serve as a specification of our algorithm, and we prove our algorithm is
    sound and complete with respect to the declarative type system. The
    specification makes it easy to see that all type applications (for
    $\forall$-elimination) and all packs (for $\exists$-introdution) are
    inferred. 

\paragraph{Employment of the Anti-unification}
    The introduction of impredicative existentials breaks some of the
    fundamental invariants of HM-style type inference. As a result, it needs to
    mix unification with \emph{anti-unification}---a dual operation which has
    been used in the literature on term rewriting systems, but which has never
    appeared in the context of polymorphic type inference.

\paragraph{Extrapolation of the Inference Framework to Other Type Systems}
    We explore the design space further to show how the
    same type inference scheme could be applied to
    work with different type systems. In particular,
    \begin{enumerate*}
      \item[(i)] in combination with \emph{bidirectional} typechecking---
        the approach used in the original local type inference paper of \citet{pierce2000:local}
        that minimizes the number of needed annotations;
      \item[(ii)] in systems with different subtyping rules, such as  
        the one used in \citet{zhao22:elementary}---a system that
        permits explicit type applications;
      \item[(iii)] in systems with \emph{bounded} polymorphic quantifiers.
        That is, one can only instantiate a quantifier with a 
        type that is a subtype of a given bound.
    \end{enumerate*}


\section{Overview}
% Imprediicative
% CBPV, monadic control
% Local Inference
% Two ways to translate

The language of \fexists is different from standard \systemf:
every term and type has either positive or negative polarity.
The type variables are annotated with their polarities
(e.g. $[[α⁺]]$ or $[[β⁻]]$), and the types can change polarity
via the shift operators $[[↑]]$ (positive to negative) and
$[[↓]]$ (negative to positive).

Positive expressions correspond to values in the \CBPV system
\cite{levy2006:cbpv}, and negative expressions correspond to computations.
The difference between the computations and values is intuitively described 
in the slogan of \CBPV: ``a value is, a computation does''. 
In particular, an argument of a function is always a value (\ie positive), 
and the resulting type, as well as the function itself, 
is a computation (\ie negative).

\subsection{Examples}

In general, any term and type of \systemf can be embedded into \fexists.
This embedding---polarization---will be covered in more details in \cref{sec:rel-to-systemf}.
Now, we discuss a number of examples of standard \systemf terms and types
and their polarization in \fexists.

\begin{figure}[h]
  \begin{align*}
    [[map]] &: [[↓∀α⁺.∀β⁺.↓(α⁺ → ↑β⁺) → List α⁺ → ↑List β⁺]] \\
    [[len]] &: [[↓∀α⁺.List α⁺ → ↑Int]] \\
    [[choose]] &: [[↓∀α⁺.α⁺ → α⁺ → ↑α⁺]] \\
    [[id]] &: [[↓∀α⁺.α⁺ → ↑α⁺]] \\
    [[auto]] &: [[↓(∀α⁺.α⁺ → ↑α⁺) → (∀α⁺.α⁺ → ↑α⁺)]]\\
    [[autoP]] &: [[↓(∀α⁺.α⁺ → ↑α⁺) → (∀α⁺.α⁺ → ↑α⁺)]]\\
  \end{align*}
  \caption{Polarization of standard \systemf terms}
  \label{fig:polarization-examples}
\end{figure}

Let us assume that $[[Φ]]$ is a context containing the terms from \cref{fig:polarization-examples}.
Then the following types are inferrable in \fexists:

\begin{align*}
  % &[[· ; Φ ⊢ [two, three, nine] : List Int]]\\
  &[[· ; Φ, x1:↓(↓iN → iN), x2:↓(↓iM → iM) ⊢ let x = choose (x1, x2); return x : ↑ ∃γ⁻.↓(↓γ⁻ → γ⁻)]]\\
  &[[· ; Φ ⊢ let x = choose (id, auto); return x : ↑ ∃γ⁻.↓γ⁻]]\\
  &[[· ; Φ ⊢ let x = choose (id, autoP); return x : ↑ ∃γ⁻.↓γ⁻]]\\
\end{align*}

The focus of our system is the \emph{local} type inference \cite{pierce2000:local},
which has certain limitations.
In particular, if an argument of a function is polymorphic,
its polymorphic variables are not instantiated by the inference algorithm. 
For example, one would expect the following to be inferrable:
$[[· ; Φ ⊢ let x = map(id, [two, three, nine]); return x : ↑ List Int]]$.
However, to infer this, the $[[id]]$ must be instantiated to $[[Int → ↑Int]]$, 
which requires the information to be propagated from the neighboring branch 
of the syntax tree ($[[ [two, three, nine] ]]$), \ie bypass the locality. 

However, the polymorphic arguments of the function are instantiated, 
and thus, the annotated version of the same term infers the type successfully:
$[[· ; Φ ⊢ let x = map((id : ↓(Int → ↑Int)), [two, three, nine]); return x : ↑ List Int]]$.


\subsection{The Language of Types}

The types of \fexists are given in \cref{fig:declarative-types}.
They are stratified into two syntactic 
categories (polarities): positive and negative,  
similar to the values and computations in the \CBPV system \cite{levy2006:cbpv}.
\begin{itemize}
\item [$-$] $[[na]]$ is a negative type variable, which can be taken from a context or introduced by $[[∃]]$.
\item [$-$] a function $[[iP → iN]]$ takes a value as input and returns a computation; 
\item [$-$] a polymorphic abstraction $[[∀pas.iN]]$ quantifies a computation over
  a list of positive type variables 
  $[[pas]]$. The polarities are chosen to follow the definition of functions.
\item [$-$] a shift $[[↑iP]]$ allows a value to be used as a computation, 
  which at the term level corresponds to a pure computation $[[return v]]$.
\item [$+$] $[[pa]]$ is a positive type variable, taken from a context or introduced by $[[∀]]$.
\item [$+$] $[[∃nas.iP]]$, symmetrically to $[[∀]]$, 
  binds negative variables in a positive type $[[iP]]$. 
\item [$+$] a shift $[[↓iN]]$, symmetrically to the up-shift, 
  thunks a computation, which at the term level corresponds to $[[ {c} ]]$.
\end{itemize}

\begin{figure}[h]
  \begin{multicols}{2}
    \ottgrammartabular{
      \ottiN\ottinterrule
    }\\
    \ottgrammartabular{
      \ottiP\ottinterrule
    }
    \columnbreak
  \end{multicols}
  \caption{Declarative Types of \fexists}
  \label{fig:declarative-types}
\end{figure}

\paragraph{Definitional Equalities}
For simplicity, we assume that alpha-equivalent terms are equal.
This way, we assume that substitutions do not capture bound variables.
Besides, we equate
$[[∀pas.∀pbs.iN]]$ with $[[∀pas,pbs.iN]]$, 
as well as $[[∃nas.∃nbs.iP]]$ with $[[∃nas,nbs.iP]]$,
and lift these equations transitively and congruently 
to the whole system.

\paragraph{Type Context and Type Well-formedness}

In the construction of \fexists, a type context (denoted as $[[Γ]]$) is
represented as a \emph{set} of positive and negative type variables and it is used to
assert the well-formedness of types. The well-formedness of a type is denoted as
$[[Γ ⊢ iP]]$ and $[[Γ ⊢ iN]]$ and it asserts that all type variables are either
bound by a quantifier ($[[∀]]$ and $[[∃]]$) or declared in the context $[[Γ]]$.
The well-formedness checking is an \emph{algorithmic} procedure. As commonly
done, we represent it as a system of inference rules, that correspond to a
recursive algorithm taking the context and the type as input. 

\subsection{The Language of Terms}

In \cref{fig:declarative-terms}, we define the language of terms of 
\fexists. The language combines \systemf with the \CBPV approach.

\begin{itemize}
    \item [$+$] $[[x]]$ denotes a term variable.
      Following the \CBPV stratification, we only have \emph{positive} (value)
      term variables;
    \item [$+$] $[[{c}]]$ is a value corresponding to a thunked 
        or suspended computation;
    \item [$\pm$] $[[(c : iN)]]$ and $[[(v : iP)]]$ allow one to annotate 
        positive and negative terms;
    \item [$-$] $[[return v]]$ is a pure computation, returning a value;
    \item [$-$] $[[λ x : iP . c]]$ and $[[Λ pa . c]]$
        are standard lambda abstractions. Notice that we require
        the type annotation for the argument of $[[λ]]$;
    \item [$-$] $[[ let x = v ; c]]$ is a standard let, binding
        a value $[[v]]$ to a variable $[[x]]$ in a computation $[[c]]$;
    \item [$-$] Applicative let forms $[[let x : iP = v ( args ) ; c]]$ and
        $[[let x = v ( args ) ; c]]$ operate similarly to 
        the bind of a monad: they take a suspended computation $[[v]]$,
        apply it to a list of arguments, bind the result 
        (which is expected to be pure) to a variable $[[x]]$,
        and continue with a computation $[[c]]$.
        If the resulting type of the application is unique, 
        one can omit the type annotation, as in the second form:
        it will be inferred by the algorithm;
    \item [$-$] $[[let∃ ( nas , x ) = v ; c]]$
        is the standard unpack of an existential type:
        expecting $[[v]]$ to be an existential type,
        it binds the packed negative types to a list of 
        variables $[[nas]]$, binds the body of the existential
        to $[[x]]$, and continues with a computation $[[c]]$.
\end{itemize}

\paragraph{Missing constructors}
Notice that the language does not have first-class applications: 
their role is played by the applicative let forms, binding 
the result of a \emph{fully applied} function to a variable.
Also notice that the language does not have a type application (i.e. the eliminator of $[[∀]]$) and dually, it does not have \pack (i.e. the constructor of $[[∃]]$).
This is because the instantiation of polymorphic and existential types is inferred by the algorithm. 
In \cref{sec:extensions}, we discuss the way to modify the system to introduce \emph{explicit} type applications.


\begin{figure}[h]
  \begin{multicols}{2}
    \ottgrammartabular{
      \ottc\ottinterrule
    }

    \ottgrammartabular{
      \ottv\ottinterrule
    }
  \end{multicols}
  \caption{Declarative Terms of \fexists}
  \label{fig:declarative-terms}
\end{figure}

\subsection{The key ideas of the algorithm}

The inference algorithm for \fexists is \emph{local}: it has a limited scope of
the inference information and does not propagate it between far-apart branches
of the syntax tree. Nevertheless, many difficulties appear in the inference of
polymorphic and existential types. Next, we discuss the most challenging of
these difficulties and how they are addressed in the algorithm.

\paragraph{Subtyping} One of the hardest parts of the inference is subtyping,
because this is where the polymorphic instantiation happens. In particular, it
is the subtyping relation that allows us to use the polymorphic terms in the
context where their concrete instantiations are expected: 
$[[Γ ⊢ ∀α⁺.↑α⁺ ≤ ↑Int]]$.

We are able to concentrate most of the difficulty in the subtyping relation 
and then use the polarization to make it decidable. 

\paragraph{Impredicativity}
Impredicativity is another difficulty. Even with forall, the natural subtyping is undecidable. We have to find the restrictions that make it decidable. Unification allows us to solve equality constraints, and the LUB has to be solved with anti-unification. We have to design the system carefully to maintain decidability.

\paragraph{Least Upper Bound}
Now the LUB (Least Upper Bound) is not necessarily the syntactic equality. Examples, types don't have a LUB. How do we decide? The answer is anti-unification, which we explain at a high level. Now we have both forall and exists, and mixing unification and anti-unification.



\section{Declarative System}
\label{sec:declarative-system}

In this section, we present the declarative system of \fexists,
which serves as a specification of the type inference algorithm.
The declarative system is represented as a set of inference rules
and consists of two main subsystems: subtyping and type inference.
First, we present the declarative subtyping rules specifying 
when one type is a subtype of another. Next, we discuss 
the equivalence relation induced by mutual subtyping.
Finally, we present the type inference rules,
that refer to the subtyping and equivalence relation.
We conclude this section by discussing the relation between the
proposed type system and the standard \systemf.

\subsection{Subtyping}

\begin{figure}
  \begin{minipage}[t]{0.49\textwidth}
    \ottdefnDOneNsubLabeled{}
  \end{minipage}%
  \begin{minipage}[t]{0.49\textwidth}
    \ottdefnDOnePsupLabeled{}
  \end{minipage}
  \hfill\\
  \begin{minipage}[t]{0.49\textwidth}
    \ottdefnDOneNeqLabeled{}
  \end{minipage}%
  \begin{minipage}[t]{0.49\textwidth}
    \ottdefnDOnePeqLabeled{}
  \end{minipage}
  \caption{Declarative Subtyping}
  \label{fig:declarative-subtyping}
\end{figure}

The inference rules representing declarative subtyping are shown in
\cref{fig:declarative-subtyping}. Let us discuss them in more detail.

\paragraph*{Quantifiers}  
Symmetric rules \ruleref{\ottdruleDOneForallLabel} and 
\ruleref{\ottdruleDOneExistsLabel} specify 
subtyping between top-level quantified types.
Usually, the polymorphic subtyping is represented by two rules
introducing quantifiers to the left and to the right-hand side of subtyping.
For conciseness of representation, we compose these rules into one.
First, our rule extends context $[[Γ]]$ with the quantified variables 
from the right-hand side ($[[pbs]]$ or $[[nbs]]$), 
as these variables must remain abstract.
Second, it verifies that the left-hand side quantifiers
($[[pas]]$ or $[[nas]]$) can be instantiated 
with an arbitrary substitution to continue subtyping recursively,
which introduces non-determinism. 

The instantiation of quantifiers is modeled by a substitution $[[σ]]$.
The notation $[[Γ2 ⊢ σ : Γ1]]$ specifies its domain and range.
For instance, $[[Γ, pbs ⊢ σ : {pas}]]$ means that 
$[[σ]]$ maps the variables from $[[pas]]$ to (positive) types
well-formed in $[[Γ, pbs]]$.
This way, application $[[ [σ]iN ]]$ instantiates (replaces) every
$[[αi⁻]]$ in $[[iN]]$ with $[[σ]]([[αi⁻]])$.

\paragraph*{Invariant shifts}
As mentioned above, an important restriction that we put on the subtyping system
is that subtyping of shifted types requires their equivalence, as shown in
\ruleref{\ottdruleDOneShiftDLabel} and \ruleref{\ottdruleDOneShiftULabel}. The
reason for this is that if both of these rules were relaxed to the covariant
form, the subtyping relation would become equivalent to the standard subtyping
of \systemf, which is undecidable \cite{tiuryn-urzczyn-96}. 
Relaxations of this condition are discussed in \cref{sec:weakening-invariant}.

% However, it is
% suggested after certain changes \ruleref{\ottdruleDOneShiftULabel} can be
% relaxed to the covariant form, thereby increasing the expressiveness of the
% system.  These changes are discussed in \cref{sec:weakening-invariant}.

\paragraph*{Functions}
Standardly, subtyping of function types is covariant in the return type
and contravariant in the argument type.

\paragraph*{Variables}
Subtyping of variables is defined reflexively, which is enough to ensure the
reflexivity of subtyping in general. The algorithm---specifically the least
upper bound procedure---will use the fact that the subtypes of a variable
coincide with its supertypes (\cref{prop:var-no-subtypes}),
which however does not hold for arbitrary types.

\subsection{Properties of Declarative Subtyping}
\label{sec:decl-subtyping-properties}

A property that is important for the subtyping algorithm, 
in particular for the type \emph{upgrade} procedure (\cref{sec:lub}),
is the preservation of free variables by subtyping.
Informally, it says that the free variables
of a positive type cannot disappear in its subtypes,
and the free variables of a negative type
cannot disappear in its supertypes.

\begin{property}[Subtyping preserves free variables]
  \label{prop:subtyping-preserves-fv}
  Let us assume that all the mentioned types are well-formed in $[[Γ]]$. Then
  $[[Γ ⊢ iN1 ≤ iN2]]$ implies $[[fv(iN1) ⊆ fv(iN2)]]$,
  and $[[Γ ⊢ iP1 ≥ iP2]]$ implies $[[fv(iP1) ⊆ fv(iP2)]]$.
\end{property}

\begin{property}[Variable subtyping is trivial]
  \label{prop:var-no-subtypes}
  A subtype or a supertype of a variable is equivalent to the variable itself:

  \begin{tabular}{@{}llclcl@{}}
    $-$ & $[[Γ ⊢ α⁻ ≤ iN]]$ & $\iff$ & $[[Γ ⊢ iN ≤ α⁻]]$ & $\iff$ & $[[iN]] = [[∀pbs.α⁻]]$ \\
    $+$ & $[[Γ ⊢ α⁺ ≥ iP]]$ & $\iff$ & $[[Γ ⊢ iP ≥ α⁺]]$ & $\iff$ & $[[iP]] = [[∃nbs.α⁺]]$
  \end{tabular}
\end{property}

Another property that we extensively use is that subtyping is reflexive and transitive,
and agrees with substitution.

\begin{property}[Subtyping forms a preorder]
  For a fixed context $[[Γ]]$, the negative subtyping relation 
  $[[Γ ⊢ iN1 ≤ iN2]]$ and the positive subtyping relation 
  $[[Γ ⊢ iP1 ≥ iP2]]$
  are reflexive and transitive on types well-formed in $[[Γ]]$.
\end{property}

\begin{property}[Subtyping agrees with substitution]
  Suppose that  $[[σ]]$ is a substitution such that $[[Γ2 ⊢ σ : Γ1]]$. 
  Then
  \begin{enumerate}
    \item [$-$] $[[Γ1 ⊢ iN ≤ iM]]$ implies $[[Γ2 ⊢ [σ]iN ≤ [σ]iM]]$, and
    \item [$+$] $[[Γ1 ⊢ iP ≥ iQ]]$ implies $[[Γ2 ⊢ [σ]iP ≥ [σ]iQ]]$.
  \end{enumerate}
\end{property}

Moreover, any two \emph{positive} types have a least upper bound, which makes
positive subtyping a semilattice. The positive least upper bound can be found
algorithmically, which we will discuss in the next section.

\begin{property}[Positive least upper bound exists]
  Suppose that $[[iP1]]$ and $[[iP2]]$ are positive types
  well-formed in $[[Γ]]$.
  Then there exists a least common supertype---a type $[[iP]]$ such that
  \begin{itemize}
    \item $[[Γ ⊢ iP ≥ iP1]]$ and $[[Γ ⊢ iP ≥ iP2]]$, and 
    \item for any $[[iQ]]$ such that $[[Γ ⊢ iQ ≥ iP1]]$ and $[[Γ ⊢ iQ ≥ iP2]]$,
      $[[Γ ⊢ iQ ≥ iP]]$.
  \end{itemize}
\end{property}

\paragraph*{Negative greatest lower bound might not exist}
The symmetric construction---the greatest lower bound of two negative types---does 
not always exist, as the following counterexample shows.
Consider the following types: 
\begin{itemize}
  \item $[[iN]]$ and $[[iQ]]$ are arbitrary closed types, 
  \item $[[iP]]$, $[[iP1]]$, and $[[iP2]]$ are non-equivalent closed types 
    such that $[[· ⊢ iP1 ≥ iP]]$ and $[[· ⊢ iP2 ≥ iP]]$, and 
    none of the types is equivalent to $[[iQ]]$.
\end{itemize}
What is the greatest common subtype of  
$[[iQ → ↓↑iQ → ↓↑iQ → iN]]$ and $[[iP → ↓↑iP1 → ↓↑iP2 → iN]]$?
The type $[[∀α⁺,β⁺,γ⁺. α⁺ → ↓↑β⁺ → ↓↑γ⁺ → iN]]$ is a common subtype,
however, it is not the greatest one, as it is too general.

One can find two greater candidates:
$[[iM1 = ∀α⁺,β⁺. α⁺ → ↓↑α⁺ → ↓↑β⁺ → iN]]$ and $[[iM2 = ∀α⁺,β⁺. β⁺ → ↓↑α⁺ → ↓↑β⁺ → iN]]$.
Instantiating $[[α⁺]]$ and $[[β⁺]]$ with $[[iQ]]$ ensures 
that both of these types are subtypes of $[[iQ → ↓↑iQ → ↓↑iQ → iN]]$;
instantiating $[[α⁺]]$ with $[[iP1]]$ and $[[β⁺]]$ with $[[iP2]]$
demonstrates the subtyping with $[[iP → ↓↑iP1 → ↓↑iP2 → iN]]$,
as $[[iP]]$ is a subtype of both $[[iP1]]$ and $[[iP2]]$.

By analyzing the inference rules, we can prove that
both $[[iM1]]$ and $[[iM2]]$ are \emph{maximal} common 
subtypes, \ie there is no common subtype that is greater than them.
However, $[[iM1]]$ and $[[iM2]]$ are not equivalent,
which means that none of them is the greatest.

\subsection{Equivalence and Normalization}
\label{sec:decl-equivalence}

The subtyping relation forms a preorder on types,
and thus, it induces an equivalence relation \aka bicoercibility 
\cite{tiuryn1995:bicoercibility}.
The declarative specification of subtyping must be defined up to this equivalence.
Moreover, the algorithms we use must withstand changes in input types within the equivalence class.
To deal with non-trivial equivalence, 
we use normalization---a function that uniformly selects a representative of the equivalence class.

Using normalization gives us two benefits:
\begin{enumerate*}
  \item [(i)] we do not need to modify significantly standard operations such as unification to withstand non-trivial equivalence, and
  \item [(ii)] if the subtyping (and thus, the equivalence) changes, we only need to modify the normalization function, 
    while the rest of the algorithm remains the same.
\end{enumerate*}

In our system, equivalence is richer than equality. Specifically,
while staying within one equivalence class, one can 
change the type:
\begin{enumerate}
  \item[(i)] introduce and remove redundant quantifiers. For example, $[[∀α⁺,β⁺.↑α⁺]]$ is equivalent
  but not equal to $[[∀α⁺.↑α⁺]]$;
  \item[(ii)] reorder adjacent quantifiers. For example, $[[∀α⁺,β⁺.α⁺ → β⁺ → γ⁻]]$ is equivalent but not equal to
  $[[∀α⁺,β⁺.β⁺ → α⁺ → γ⁻]]$; 
  \item[(iii)] make the transformations (i) and (ii) at any position in the type.
\end{enumerate}

It turns out that the transformations (i-iii) are complete, 
in the sense that they generate the whole equivalence class.
This way, to normalize the type, one must
\begin{enumerate}
  \item [(i)] remove the redundant quantifiers,
  \item [(ii)] reorder the quantifiers to the canonical order,
  \item [(iii)] do the procedures (i) and (ii) recursively on the subterms.
\end{enumerate}

The normalization algorithm is shown in \cref{fig:type-nf}. The steps (i-ii) are
implemented by the \emph{ordering} function `$[[ord varset in iN]]$' and `$[[ord
varset in iP]]$'. For a set of variables $[[varset]]$ and a type, it returns a
list of variables from $[[varset]]$ that occur in the type in the \emph{order of
their first occurrence}. For brevity, we omit the formal definition of ordering
referring the reader to the appendix (\cref{alg:var-ordering}).

\begin{figure}[t]
  
  \begin{multicols}{2}
  \ottdefnNrmNNormLabeled{}
  \\
  $[[ord varset in iN]]$ returns a list of variables from 
  $[[varset ∩ fv(iN)]]$ in the order of their first occurrence in $[[iN]]$
  \columnbreak\\
  \ottdefnNrmPNormLabeled{}
  \\
  $[[ord varset in iP]]$ returns a list of variables from 
  $[[varset ∩ fv(iP)]]$ in the order of their first occurrence in $[[iP]]$
  \end{multicols}


  \caption{Type Normalization Procedure} 
  \label{fig:type-nf}
\end{figure}


For the normalization procedure, 
we prove soundness and completeness \wrt the equivalence relation.
\begin{property}[Correctness of normalization]
  Assuming all types are well-formed in $[[Γ]]$,
  \begin{itemize}
    \item[$-$] $[[Γ ⊢ iN ≈ iM]]$ is equivalent to $[[nf(iN) = nf(iM)]]$, and
    \item[$+$] $[[Γ ⊢ iP ≈ iQ]]$ is equivalent to $[[nf(iP) = nf(iQ)]]$.
  \end{itemize}
\end{property}


\subsection{Typing}

The declarative specification of the type inference is given in 
\cref{fig:declarative-inference}.
It consists of three main judgments: the positive typing judgment,
the negative typing judgment, and the application typing judgment.
The positive typing judgment $[[Γ ; Φ ⊢ v : iP]]$ is read as 
``under the type context $[[Γ]]$ and variable context $[[Φ]]$,
the term $[[v]]$ is allowed to infer type $[[iP]]$'',
where $[[Φ]]$---the variable context---is defined standardly as
a set of pairs of the form $[[x : iP]]$. 
The negative typing judgment is read similarly.
The \emph{Application typing} judgment
infers the type of the application of a function to a list of arguments.
It has form $[[Γ ; Φ ⊢ iN ● args ⇒> iM]]$, 
which reads ``under the type context $[[Γ]]$ and variable context $[[Φ]]$,
the application of a function of type $[[iN]]$ to the list of arguments $[[args]]$
is allowed to infer type $[[iM]]$''.

\begin{figure}[t!]
  \ottdefnsDTLabeled
  \caption{Declarative Inference}
  \label{fig:declarative-inference}
\end{figure}

Let us discuss the rules of the declarative system in more detail.

\paragraph*{Variables}
    Rule \ruleref{\ottdruleDTVarLabel} allows us to infer
    the type of a variable from the context. 
    In literature, one can find another version of this rule,
    that enables inferring a type \emph{equivalent}
    to the type from the context. 
    In our case, the inference of equivalent types is admissible by \ruleref{\ottdruleDTPEquivLabel}.


\paragraph*{Annotations}
  The annotation rules \ruleref{\ottdruleDTNAnnotLabel}
  and \ruleref{\ottdruleDTPAnnotLabel} use subtyping.
  The annotation is only valid if the
  inferred type is a subtype of the annotation type.

\paragraph*{Abstractions}
  The typing of lambda abstraction is standard. 
  Rule \ruleref{\ottdruleDTtLamLabel} first checks
  that the given type $[[iP]]$ annotating the argument is well-formed,
  and then infers the type $[[iN]]$ of the body in the extended context.
  As a result, it returns an arrow type $[[iP → iN]]$.
  Rule \ruleref{\ottdruleDTTLamLabel} infers polymorphic $[[∀]]$-type. 
  It extends the type context with the quantifying variable $[[pa]]$ and 
  infers the type $[[iN]]$ of the body. As a result, it returns a polymorphic type
  $[[∀pa.iN]]$.

\paragraph*{Return and thunk}
  Rules \ruleref{\ottdruleDTReturnLabel} 
  and \ruleref{\ottdruleDTThunkLabel}
  simply add the corresponding shift constructors to the type of the body.

\paragraph*{Unpack}
  Rule \ruleref{\ottdruleDTUnpackLabel} types elimination of $[[∃]]$.
  First, it infers the normalized type of the existential package.
  The normalization is required to fix the order of the quantifying variables
  to bind them. Alternative approaches that do not require normalization
  will be discussed in \cref{sec:explicit-type-application}.
  After the bind, the rule infers the type of the body 
  and ensures that the bound variables do not escape the scope.

\paragraph*{Let binders}
  Rule \ruleref{\ottdruleDTVarLetLabel} is a standard rule for typing let
  binders: we infer the type of the bound value and continue the typing of the
  computation in the extended context.

  Rule \ruleref{\ottdruleDTCVarLetLabel} associates a variable with a computation.
  The inferred type for this computation ($[[iM]]$) must be castable to a
  shifted positive type $[[↑iP]]$, with $[[iP]]$ then assigned to the bound
  variable $[[x]]$ to type the let binder's body. Like all annotated
  constructors, we also verify the annotation type $[[iP]]$'s well-formedness. 
  
\paragraph*{Applicative let binders}
  Rules \ruleref{\ottdruleDTAppLetLabel} and \ruleref{\ottdruleDTAppLetAnnLabel}
  infer the type of the applicative let binders.
  Both of them infer the type of the head $[[v]]$ 
  and invoke the application typing to infer the type of the application 
  before recursing on the body of the let binder.

  The former rule infers the type of an \emph{unannotated} let binder, and thus
  it requires the resulting type of application to be \emph{the principal type},
  so that the type we assign to the bound variable $[[x]]$ is determined.
  In this context, principality means minimality. In other words, 
  $[[Γ ; Φ ⊢ iM ● args ⇒> ↑iQ principal]]$ means that
  any other type $[[iQ']]$ inferrable for the application (\ie $[[Γ ; Φ ⊢ iM ● args ⇒> ↑iQ']]$)
  is greater than the principal type $[[iQ]]$, \ie $[[Γ ⊢ iQ' ≥ iQ]]$.

  The latter rule is for the \emph{annotated} binder,
  and thus, the type of the bound $[[x]]$ is given, 
  however, the rule must check that this type is a
  a supertype of the inferred type of the application. 
  This check is done by invoking the subtyping judgment
  $[[Γ ⊢ iM' ≤ ↑iP]]$.
  % This judgment is more restrictive than checking bare 
  % $[[Γ ⊢ iP ≥ iQ]]$, however, it is necessary
  % to make the algorithm complete as it allows us to preserve
  % certain invariants throughout the algorithm to ensure
  % the decidability of the produced unification task.
  % In \cref{sec:proof-subtyping}, we discuss these invariants in
  % more detail, and in \cref{sec:weakening-invariant} we suggest 
  % a way to relax them.

\paragraph*{Typing up to equivalence}
  As discussed in \cref{sec:decl-equivalence}, the subtyping, as a preorder, 
  induces a non-trivial equivalence relation on types. 
  The system must not distinguish between equivalent types,
  and thus, type inference must be defined up to equivalence. 
  For this purpose, we use rules \ruleref{\ottdruleDTPEquivLabel}  
  and \ruleref{\ottdruleDTNEquivLabel}.
  They allow one to replace the inferred type with an equivalent one.  

\paragraph*{Application to an empty list of arguments}
  The base case of the application type inference is 
  represented by rule \ruleref{\ottdruleDTEmptyAppLabel}.
  If the head of the type $[[iN]]$ is applied to no arguments, 
  the type of the result is allowed to be $[[iN]]$ or any 
  equivalent type. We need to relax this rule up to equivalence
  to ensure the corresponding property globally:
  the inferred application type can be replaced with an equivalent one.
  Alternatively, we could have added a separate rule similar to 
  \ruleref{\ottdruleDTPEquivLabel}, however, 
  the local relaxation is sufficient to prove the global property.

\paragraph*{Application of a polymorphic type $[[∀]]$}
  The complexity of the system lives in the rules whose output type is not
  immediately defined by their input and the output of their premises (\aka not
  mode-correct \cite{dunfield2020:bidirectional}). In our typing system,
  \ruleref{\ottdruleDTForallAppLabel} is such a rule: the instantiation of the
  quantifying variables is not known a priori. The algorithm we present in
  \cref{sec:algorithm} delays this instantiation until more information about it
  (in particular, the set of typing constraints) is collected.

  To ensure the priority of application between this rule and 
  \ruleref{\ottdruleDTEmptyAppLabel}, we also check that 
  the list of arguments is not empty.

\paragraph*{Application of an arrow type}
  Another application rule is \ruleref{\ottdruleDTArrowAppLabel}.
  This is where the subtyping is used to check that the type of the argument
  is convertible to (a subtype of) the type of the function parameter.
  In the algorithm (\cref{sec:algorithm}), this subtyping check will provide the constraints
  we need to resolve the delayed instantiations of the quantifying variables.

  % \vspace{\baselineskip}
% \subsubsection{Properties of Declarative Typing}
  An important property that the declarative system has is
  that the declarative specification is correctly defined for
  equivalence classes.

\begin{property}[Declarative typing is defined up to equivalence]
  Let us assume that $[[Γ ⊢ Φ1 ≈ Φ2]]$, \ie  
  the corresponding types assigned by $[[Φ1]]$ and
  $[[Φ2]]$ are equivalent in $[[Γ]]$.
  Also, let us assume that 
  $[[Γ ⊢ iN1 ≈ iN2]]$, $[[Γ ⊢ iP1 ≈ iP2]]$,
  and $[[Γ ⊢ iM1 ≈ iM2]]$. Then
  \begin{itemize}
    \item [$-$] $[[Γ ; Φ1 ⊢ c : iN1]]$ holds if and only if $[[Γ ; Φ2 ⊢ c : iN2]]$,
    \item [$+$] $[[Γ ; Φ1 ⊢ v : iP1]]$ holds if and only if $[[Γ ; Φ2 ⊢ v : iP2]]$, and
    \item [$\bullet$] $[[Γ; Φ1 ⊢ iN1 ● args ⇒> iM1]]$ holds if and only if $[[Γ; Φ2 ⊢ iN2 ● args ⇒> iM2]]$.
  \end{itemize}
\end{property}

\subsection{Relation to \systemf}
\label{sec:rel-to-systemf}

To establish a correspondence between \fexists and standard unpolarized \systemf,
we present a translation in both ways: the polarization from \systemf to \fexists
and the depolarization from \fexists to \systemf.



\paragraph*{Type-level translation}

The type depolarization (\cref{fig:depolarization}) simply forgets the polarization structure of types:
it removes the shift operators and the polarities of the free type variables.

The type polarization (\cref{fig:polarization}) is more involved, as there are
multiple ways to polarize a type: for instance, a variable $[[α]]$ can be
polarized to $[[α⁺]]$ or $[[α⁻]]$. The choice of polarization affects the
execution strategy of the program. 
In particular, a functional type can be represented in either positive (thunked, call-by-value) way:
$[[↓(iP → ↑iQ)]]$ or negative (call-by-name): $[[↓iN → iM]]$.
We chose the positive polarization: every \systemf type is translated to a
\emph{positive} type of \fexists. This choice makes it smoother to lift the
translation to the term level, which will be discussed next.

The type-level translation is naturally lifted to the contexts:
$[[|Γ|]]$ is the context $[[Γ]]$ with every type depolarized,
and symmetrically, $[[+|fΓ|+]]$ is the context $[[fΓ]]$ with every type polarized.
The context translation is used to formulate the soundness of the
 term-level translation, which we discuss next.


\begin{figure}[t]
  \begin{minipage}[t]{0.60\linewidth}
    \centering
      \begin{minipage}[t]{0.30\linewidth}
      \ottfundefnunpolP{}
      \end{minipage}
      \hspace{0.5cm}
      \begin{minipage}[t]{0.30\linewidth}
      \ottfundefnunpolN{}
      \end{minipage}
    \caption{Type Depolarization \protect\footnotemark}
    \label{fig:depolarization}
  \end{minipage}
  \hspace{0.3cm}
  \begin{minipage}[t]{0.3\linewidth}
    \centering
    \ottfundefnpolarP{}
    \caption{Type Polarization}
    \label{fig:polarization}
  \end{minipage}
\end{figure}


\footnotetext{Here the \systemf existential $[[∃ as . fT]]$ is a syntax sugar for its standard encoding $[[∀b.(∀as.(fT → b)) → b]]$}

\paragraph*{Term-level translation}

The term-level translations between the systems are more complex. They are
defined as \emph{elaborations} of the typing derivations. In other words, the
input of the translation is not a term, but a whole typing derivation in the
corresponding system. For example, the subtyping elaboration $[[Γ ⊢ iN ≤ iM ⤳
ft]]$ constructs a \systemf function $[[ft]]$ of type $[[|iN| → |iM|]]$ that
witnesses the given \fexists-subtyping $[[Γ ⊢ iN ≤ iM]]$. A typing elaboration
$[[Γ; Φ ⊢ v : iP ⤳ ft]]$ builds a term $[[ft]]$---a \systemf counterpart of
$[[v]]$---of type $[[|iP|]]$.

For brevity, we omit the formal definition of the term-level elaboration,
referring the reader to the appendix (\cref{sec:term-level-translation}). We define
the elaboration in such a way that it preserves the soundness invariants given
in \cref{tab:elaboration-soundness}. 

\begin{figure}[t]
\centering
\begin{tabular}{rll}
\hline
\multicolumn{2}{c}{\textbf{Elaboration Judgment}} & \textbf{The Soundness Property} \\
\hline
Negative Subtyping & $[[Γ ⊢ iN ≤ iM ⤳ ft]]$ & $[[|Γ| ; · ⊢ ft : |iN| → |iM| ]]$ \\
Positive Subtyping & $[[Γ ⊢ iP ≥ iQ ⤳ ft]]$ & $[[|Γ| ; · ⊢ ft : |iQ| → |iP| ]]$ \\
Negative Typing & $[[Γ; Φ ⊢ c : iN ⤳ ft]]$ & $[[|Γ| ; |Φ| ⊢ ft : |iN|]]$ \\
Positive Typing & $[[Γ; Φ ⊢ v : iP ⤳ ft]]$ & $[[|Γ| ; |Φ| ⊢ ft : |iP|]]$ \\
Application Typing & $[[Γ ; Φ ⊢ iN ● args ⇒> iM ⤳ fe; fargs]]$ & $[[|Γ| ; |Φ|, x:|iN| ⊢ fe (x ● fargs) : |iM|]]$ \\
\hline
\systemf Typing & $[[fΓ ; fΦ ⊢ ft : fT ⤳ c]]$ & $[[+|fΓ|+ ; +|fΦ|+ ⊢ c : ↑+|fT|+]]$ \\
\hline
\end{tabular}
\caption{Soundness of the Elaboration:
  the elaboration judgment on the left implies the typing judgment on the right}
  \label{tab:elaboration-soundness}
\end{figure}

The least obvious aspect of the elaboration soundness is the application typing.
The elaboration judgment $[[Γ ; Φ ⊢ iN ● args ⇒> iM ⤳ fe; fargs]]$ 
outputs two things: 
\begin{enumerate*}
  \item [(i)] $[[fargs]]$---the \systemf counterpart of the argument list $[[args]]$, and 
  \item [(ii)] $[[fe]]$---a \systemf term that is used for the final cast 
    of the application $[[(x ● ft *)]]$ to the resulting type $[[|iM|]]$.
\end{enumerate*}
This final cast is needed to 
reflect the \fexists feature of \ruleref{\ottdruleDTEmptyAppLabel}
that permits the equivalent conversion of the resulting type $[[Γ ⊢ iN ≈ iN']]$,
as \fexists-equivalent types
might have different \systemf representations (\ie $[[|iN|]] \neq [[|iN'|]]$).


% \begin{figure}[h]
%   \begin{equation*}
%   \begin{aligned}
%     [[|↓iN|]] &= [[|iN|]] \\
%     [[|↑iP|]] &= [[|iP|]] \\
%   \end{aligned}
%   \qquad
%   \begin{aligned}
%     [[|α±|]] &= [[a]] \\
%     [[|iP → iN|]] &= [[|iP| → |iN|]] \\
%   \end{aligned}
%   \qquad
%   \begin{aligned}
%     [[|∀α⁺.iN|]] &= [[∀α.|iN|]] \\
%     [[|∃α⁻.iP|]] &= [[∃α.|iP|]] \\
%   \end{aligned}
%   \end{equation*}
%   \caption{Type depolarization}
%   \label{fig:depolarization}
% \end{figure}

% The translation from \systemf to \fexists can be done in two canonical ways
% depending on the desired polarity. The polarity will also determine the 
% execution strategy \ilyam{how?}.
% Moreover, the translation procedure depends on the \emph{polarity} of the
% free type variables. Thus, the translation functions 
% $[[-| fT |-^Γ]]$ and $[[+| fT |+^Γ]]$ are indexed by a context $[[Γ]]$, 
% containing either $[[α⁺]]$ or $[[α⁻]]$ for each variable $[[α]]$ free in $[[fT]]$.

% \begin{figure}[h]
%   \begin{equation*}
%   \begin{aligned}
%     [[-| a |-^Γ]] &= [[α⁻]]  \text{,~if $[[α⁻ ∊ Γ]]$}\\
%     [[-| a |-^Γ]] &= [[↑α⁺]] \text{,~if $[[α⁺ ∊ Γ]]$}\\
%     [[-| fA → fB |-^Γ]] &= [[ +|fA|+^Γ → -|fB|-^Γ  ]] \\
%     [[-| ∀α.fT |-^Γ]] &= [[ ∀α⁺.-|fT|-^Γ,α⁺ ]] \\
%     [[-| ∃α.fT |-^Γ]] &= [[ ↑ ∃α⁻.+|fT|+^Γ,α⁻ ]] \\
%   \end{aligned}
%   \qquad
%   \begin{aligned}
%     [[+| a |+^Γ]] &= [[α⁺]]  \text{,~if $[[α⁺ ∊ Γ]]$}\\
%     [[+| a |+^Γ]] &= [[↓α⁻]] \text{,~if $[[α⁻ ∊ Γ]]$}\\
%     [[+| fA → fB |+^Γ]] &= [[ ↓ +|fA|+^Γ → -|fB|-^Γ  ]] \\
%     [[+| ∀α.fT |+^Γ]] &= [[ ↓∀α⁺.-|fT|-^Γ,α⁺ ]] \\
%     [[+| ∃α.fT |+^Γ]] &= [[ ∃α⁻.+|fT|+^Γ,α⁻ ]] \\
%   \end{aligned}
%   \end{equation*}
%   \caption{Type polarization}
%   \label{fig:polarization}
% \end{figure}




\section{The Algorithm}
\label{sec:algorithm}

In this section, we present the algorithmization of the declarative system described above.
The algorithmic system follows the structure of the declarative specification closely.
First, it is also given by a set of inference rules, which, however,
are mode-correct (\cite{dunfield2020:bidirectional}), \ie
the output of each rule is always uniquely defined by its input.
And second, the rules of the algorithmic system mirrors the declarative rules 
(except for the rules \ruleref{\ottdruleDTPEquivLabel} and \ruleref{\ottdruleDTNEquivLabel}), 
which simplifies the correctness proof. 


\subsection{Algorithmic Syntax}
\label{sec:algo-syntax}

First, let us discuss the syntax of the algorithmic system (\cref{fig:algo-syntax}).

\begin{figure}[t]

  \begin{minipage}[t]{0.49\textwidth}
      Negative Algorithmic Variables\\
      $[[α̂⁻]]$, $[[β̂⁻]]$, $[[γ̂⁻]]$, \dots\\
  \end{minipage}%
  \begin{minipage}[t]{0.49\textwidth}
      Positive Algorithmic Variables\\
      $[[α̂⁺]]$, $[[β̂⁺]]$, $[[γ̂⁺]]$, \dots\\
  \end{minipage}

  \hfill\\
  \begin{minipage}[t]{0.49\textwidth}
      \ottuNShort
  \end{minipage}
  \begin{minipage}[t]{0.49\textwidth}
      \ottuPShort
  \end{minipage}
  \hfill\\
  Algorithmic Type Context\\
   $[[Ξ]] \Coloneqq \{[[α1̂±]], \dots, [[αn̂±]]\}$ where $[[α1̂±]], \dots, [[αn̂±]]$ are pairwise distinct \\
  \hfill\\
  Instantiation Context\\
   $[[Θ]] \Coloneqq \{[[ α1̂±[Γ1] ]], \dots, [[ αn̂±[Γn] ]]\}$ where $[[α1̂±]], \dots, [[αn̂±]]$ are pairwise distinct \\
  \caption{Algorithmic Syntax}
  \label{fig:algo-syntax}
\end{figure}

\paragraph*{Algorithmic Variables}
To design a mode-correct inference system, we slightly modify the language we operate on.
The entities (terms, types, contexts) that the algorithm manipulates we call \emph{algorithmic}. 
They extend the previously defined declarative terms and types by adding 
\emph{algorithmic type variables} (\aka unification variables). 
The algorithmic variables represent unknown types, 
which cannot be inferred immediately but are promised to be instantiated
as the algorithm proceeds.

We denote algorithmic variables as $[[α̂⁺]]$, $[[β̂⁻]]$, \dots to distinguish
them from normal variables $[[α⁺]]$, $[[β⁻]]$. In a few places, we replace the
quantified variables $[[pas]]$ with their algorithmic counterpart $[[puas]]$.
The procedure of replacing declarative variables with algorithmic ones we call
\emph{algorithmization} and denote as $[[ nuas/nas ]]$ and $[[ puas/pas ]]$. The
converse operation---\emph{dealgorithmization}---is denoted as $[[ nas/nuas ]]$ and 
is used in the least upper bound procedure (\cref{sec:lub}).

\paragraph*{Algorithmic Types}
The syntax of algorithmic types extends the declarative syntax by adding
algorithmic variables as new terminals. We add positive algorithmic variables $[[α̂⁺]]$ 
to the syntax of positive types, and negative algorithmic variables $[[α̂⁻]]$ to the 
syntax of negative types. All the constructors of the system can be applied 
to \emph{algorithmic} types, however, algorithmic variables cannot be abstracted by the
quantifiers $[[∀]]$ and $[[∃]]$.

\paragraph*{Algorithmic Contexts $[[Ξ]]$ and Well-formedness}
To specify when algorithmic types are well-formed, we define algorithmic
contexts $[[Ξ]]$ as sets of algorithmic variables. Then
$[[Γ ; Ξ ⊢ uP]]$ and $[[Γ ; Ξ ⊢ uN]]$ represent the well-formedness judgment of
algorithmic terms defined as expected. Informally, they check that all free
declarative variables are in $[[Γ]]$, and all free algorithmic variables are in
$[[Ξ]]$. Most of the rules are inherited from the well-formedness of
\emph{declarative} types: the declarative variables are checked to belong to the
context $[[Γ]]$, the quantifiers extend the context $[[Γ]]$, type constructors are
well-formed congruently. As we extend the syntax with algorithmic variables, we
also add two base-case rules for them: \ruleref{\ottdruleWFATPUVarLabel} and
\ruleref{\ottdruleWFATNUVarLabel} (see \cref{fig:algo-wf}).

\begin{figure}[t]
  \begin{minipage}{0.49\textwidth}
  $$\vcenter{\hbox{\vdots}}$$
  \ottusedrule{\ottdruleWFATPUVarLabeled{}}
  \end{minipage}
  \begin{minipage}{0.49\textwidth}
  $$\vcenter{\hbox{\vdots}}$$
  \ottusedrule{\ottdruleWFATNUVarLabeled{}}
  \end{minipage}
\caption{Well-formedness of Algorithmic Types extends Declarative Well-formedness}
\label{fig:algo-wf}
\end{figure}

\paragraph*{Instantiation Context $[[Θ]]$}
When one instantiates an algorithmic variable, one may only use type variables
\emph{available in its scope}. As such, each algorithmic variable must remember
the context at the moment when it was introduced. In our algorithm, this
information is represented by an \emph{instantiation context} $[[Θ]]$---a set of
pairs associating algorithmic variables and declarative contexts.

\paragraph*{Algorithmic Substitution}
We define the algorithmic substitution $[[uσ]]$ as a mapping from algorithmic
variables to \emph{declarative} types. The signature $[[Θ ⊢ uσ : Ξ]]$
specifies the domain and the range of $[[uσ]]$: for each variable $[[α̂±]]$
in $[[Ξ]]$, there exists an corresponding entry in $[[Θ]]$ associating 
$[[α̂±]]$ with a declarative context $[[Γ]]$ such that $[[ [uσ]α̂± ]]$
is well-formed in $[[Γ]]$. In addition, we assume that $[[uσ]]$
acts as the identity on the variables not in $[[Ξ]]$.


\paragraph*{Algorithmic Normalization}
Similarly to well-formedness, the normalization of algorithmic types is defined
by extending the declarative definition (\cref{fig:type-nf}) with the
algorithmic variables. To the rules repeating the declarative normalization, we
add rules saying that normalization is trivial on algorithmic variables.

\begin{figure}[t]
\begin{minipage}{0.49\textwidth}
  $$\vcenter{\hbox{\vdots}}$$
  \ottusedrule{\ottdruleNrmPUVar{}}
\end{minipage}
\hfill
\begin{minipage}{0.49\textwidth}
  $$\vcenter{\hbox{\vdots}}$$
  \ottusedrule{\ottdruleNrmNUVar{}}
\end{minipage}
\caption{Normalization of Algorithmic Types extends Declarative Normalization}
\label{fig:algo-nf}
\end{figure}

\subsection{Type Constraints}
Throughout the algorithm's operation, it gathers information about the
algorithmic type variables, represented as \emph{constraints}. These constraints
in our system can be either \emph{subtyping constraints} or \emph{unification
constraints}. As preserved by the algorithm, each subtyping constraint has a
positive shape $[[α̂⁺ :≥ iP]]$, meaning it confines a positive algorithmic
variable to be the supertype of a positive declarative type. Unification
constraints can take a positive $[[α̂⁺ :≈ iP]]$ or negative $[[α̂⁻ :≈ iN]]$
form. Algorithmic variables cannot occur in the right-hand side of the
constraints.  The constraint \emph{set} is denoted as $[[SC]]$, and we presume that each
algorithmic variable can be restricted by at most one constraint.

We define $[[UC]]$ separately as a set solely containing \emph{unification}
constraints for simpler algorithm representation. The unification algorithm,
which we use as a subroutine of the subtyping algorithm, can only produce
unification constraints. The resolution of unification constraints is simpler than
that of a general constraint set. This way, this separation allows us to better
decompose the algorithm's structure, thus simplifying the inductive proofs.

  \begin{figure}[t]
    \begin{minipage}{0.49\textwidth}
      \ottgrammartabular { 
      \ottscE 
      \ottinterrule
      \ottruleheadOneLine
        {[[SC]]}{\Coloneqq} {\ottcom{Constraint Set}}
        {\{[[scE1]], \dots, [[scEn]]\}}\ottprodnewline
      }
    \end{minipage}
    \begin{minipage}{0.49\textwidth}
      \ottgrammartabular { 
      \ottucE 
      \ottprodnewline
      \ottinterrule
      \ottruleheadOneLine
        {[[UC]]}{\Coloneqq} {\ottcom{Unification Constraint Set}}
        {\{[[ucE1]], \dots, [[ucEn]]\}}\ottprodnewline
      }
    \end{minipage}
    \label{fig:syntax-e-sc}
    \caption{Constraint Entries and Sets}
  \end{figure}


\paragraph*{Auxiliary Functions}
\begin{itemize}
  \item
    We define $[[dom(SC)]]$---the domain of constraint set $[[SC]]$---as a set of
    algorithmic variables that it restricts. Similarly, we define $[[dom(Θ)]]$---the
    domain of instantiation context $[[Θ]]$---as a set of algorithmic variables that
    $[[Θ]]$ associates with their contexts.  
  \item We write $[[Θ(α̂±)]]$ to denote the
    declarative context associated with $[[α̂±]]$ in $[[Θ]]$. 
  \item $[[uv(uN)]]$ and $[[uv(uP)]]$ denote the set of free algorithmic variables in $[[uN]]$ and
    $[[uP]]$ respectively.
  \item We write $[[Γ ⊢ Θ]]$ to denote that each
    declarative context associated with an algorithmic variable in $[[Θ]]$ is 
    a subcontext (subset) of $[[Γ]]$.
\end{itemize}

\paragraph*{Equivalent Substitutions}
In the proofs of the algorithm correctness, 
we often state or require that two substitutions are equivalent
on a given set of algorithmic variables.
We denote it as $[[Θ ⊢ uσ' ≈ uσ : Ξ]]$ meaning that
for each $[[α̂±]]$ in $[[Ξ]]$, substitutions $[[uσ]]$ and $[[uσ']]$
map $[[α̂±]]$ to types equivalent in the corresponding context:
$[[ Θ(α̂±) ⊢ [uσ']α̂± ≈ [uσ]α̂± ]]$.



\paragraph*{Constraint well-formedness and satisfaction}
Suppose that $[[Θ]]$ is an instantiation context. 
We say that constraint set $[[SC]]$ is well-formed in $[[Θ]]$ 
(denoted as $[[Θ ⊢ SC]]$) 
if for every constrain entry $[[scE ∊ SC]]$
associating a variable $[[α̂±]]$ with a type $[[iP]]$,
this type is well-formed in the corresponding declarative context $[[Θ(α̂±)]]$.

Substitution $[[uσ]]$ satisfies a constraint \emph{entry} $[[scE]]$ restricting
algorithmic variable $[[α̂±]]$ if
$[[ [σ]α̂± ]]$ can be validly substituted for $[[α̂±]]$ in $[[scE]]$ 
(so that the corresponding equivalence or subtyping holds).

Substitution $[[uσ]]$ satisfies a constraint \emph{set} $[[SC]]$
if $[[Θ ⊢ SC]]$ and \emph{each entry} of $[[SC]]$ is satisfied by $[[uσ]]$.

%%%%%%%%%%%%%%%%%
%%%%%%%%%%%%%%%%%
\subsection{Subtyping Algorithm}
\label{sec:subtyping-algorithm}
  
  For convenience and scalability, 
  we decompose the subtyping algorithm 
  into several procedures. \Cref{fig:alg-subtyping-graph}
  shows these procedures and the dependencies between them:
  arrows denote the invocation of one procedure from another.

  Some of the procedures (in particular, the unification and the
  anti-unification) assume that the input types are normalized. Therefore, we
  call the normalization procedure before invoking them, indicating this by the
  `$\ottkw{nf}$' annotation on the arrows in \cref{fig:alg-subtyping-graph}.
  Alternatively, one could normalize the input types in the very beginning
  and preserve the normalization throughout the algorithm. However, we
  delay the normalization to the places where it is required to show that
  normalization is needed only at these stages to maintain consistent
  invariants.

\begin{figure}[t]
  \centering
  \begin{tikzpicture}
    [>={Stealth[scale=2]},node distance=2.4cm,every node/.style={draw,rectangle},every text node part/.style={align=center}]


    % Define nodes
    \node[] (1) {Negative Subtyping\\$[[Γ ; Θ ⊨ uN ≤ iM ⫤ SC]]$\\(\cref{fig:alg-subtyping})};
    \node[below of=1] (3) {Positive Subtyping\\$[[Γ ; Θ ⊨ uP ≥ iQ ⫤ SC]]$\\(\cref{fig:alg-subtyping})};
    \node[left=1.5cm of 3] (2) {Constraint Merge\\$[[Θ ⊢ SC1 & SC2 = SC3]]$\\(\cref{sec:constraint-merge})};
    \node[right=1.5cm of 3] (5) {Unification\\ $[[Γ ; Θ ⊨ uN ≈u iM ⫤ UC]]$\\ $[[Γ ; Θ ⊨ uP ≈u iQ ⫤ UC]]$\\(\cref{sec:unification})};
    \node[below of=3] (4) {Upgrade\\$[[upgrade Γ ⊢ iP to Δ = iQ]]$\\(\cref{sec:lub})};
    \node[below of=2] (6) {Least Upper Bound\\$[[Γ ⊨ iP1 ∨ iP2 = iQ]]$\\(\cref{sec:lub})};
    \node[below of=6] (7) {Anti-Unification\\$[[Γ ⊨ iP1 ≈au iP2 ⫤ ( Ξ , uQ , aus1 , aus2 )]]$\\$[[Γ ⊨ iN1 ≈au iN2 ⫤ ( Ξ , uM , aus1 , aus2 )]]$\\(\cref{sec:antiunification})};
    \node[below of=5] (8) {Unification Constraint Merge\\$[[Θ ⊢ UC1 & UC2 = UC3]]$\\(\cref{sec:constraint-merge})};

    % Define edges
    \draw[->] (1) to (2);
    \draw[->] (1) to (3);
    \draw[->] (1) to node[above, draw=none]{$\ottkw{nf}$} (5);
    \draw[->] (2) to (3);
    \draw[->] (2) to (6);
    \draw[->] (3) to (4);
    \draw[->] (3) to node[above, draw=none]{$\ottkw{nf}$} (5);
    \draw[->] (4) to (6);
    \draw[->] (5) to (8);
    \draw[->] (6) to node[left, draw=none]{$\ottkw{nf}$} (7);
  \end{tikzpicture}  
  \caption{Dependency graph of the subtyping algorithm}
  \label{fig:alg-subtyping-graph}
\end{figure}

In the remainder of this section, we will discuss each of these procedures in
detail, following the top-down order of the dependency graph. First, we present
the subtyping algorithm itself.

As an input, the subtyping algorithm takes
a type context $[[Γ]]$, an instantiation context $[[Θ]]$,
and two types of the corresponding polarity:
$[[uN]]$ and  $[[iM]]$ for the negative subtyping, and
$[[uP]]$ and  $[[iQ]]$ for the positive subtyping.
We assume the second type ($[[iM]]$ and $[[iQ]]$) to be 
declarative (with no algorithmic variables) and well-formed in $[[Γ]]$,
but the first type ($[[uN]]$ and $[[uP]]$) may contain algorithmic variables,
whose instantiation contexts are specified by $[[Θ]]$.

Notice that the shape of the input types uniquely determines the
applied subtyping rule.  If the subtyping is successful, it returns
a set of constraints $[[SC]]$ restricting the algorithmic 
variables of the first type. If the subtyping does not hold, 
there will be no inference tree with such inputs. 

\begin{figure}[t]
  \hfill\\
  \begin{multicols}{2}
    \ottdefnANsubLabeled{}
    \columnbreak\\
    \ottdefnAPsupLabeled{}
  \end{multicols}
  \caption{Subtyping Algorithm}
  \label{fig:alg-subtyping}
\end{figure}

The rules of the subtyping algorithm bijectively correspond to the rules of the declarative
system. Let us discuss them in detail.

\paragraph*{Variables} Rules \ruleref{\ottdruleANVarLabel} and \ruleref{\ottdruleAPVarLabel}
say that if both of the input types are equal declarative variables,
they are subtypes of each other, with no constraints (as there are no algorithmic variables).

\paragraph*{Shifts} Rules \ruleref{\ottdruleAShiftDLabel} and
\ruleref{\ottdruleAShiftULabel} cover the downshift and the upshift cases,
respectively. If the input types are constructed by shifts, then the subtyping
can only hold if they are equivalent. This way, the algorithm must find the
instantiations of the algorithmic variables on the left-hand side
such that these instantiations make the left-hand side and the ride-hand side
equivalent. For this purpose, the algorithm invokes the
unification procedure (\cref{sec:unification}) preceded by the normalization of the input types.
It returns the resulting constraints given by the unification algorithm. 

\paragraph*{Quantifiers}  
Rules \ruleref{\ottdruleAForallLabel} and 
\ruleref{\ottdruleAExistsLabel} are symmetric. 
Declaratively, the quantified variables on the left-hand side must 
be instantiated with types, which are not known beforehand.
We address this problem by algorithmization 
of the quantified variables (see \cref{sec:algo-syntax}).
The rule introduces fresh algorithmic variables
$[[puas]]$ or $[[nuas]]$,
puts them into the instantiation context $[[Θ]]$
(specifying that they must be instantiated in the extended context
$[[Γ, pbs]]$ or $[[Γ, nbs]]$) and substitute the quantified variables
for them in the input type. 

After algorithmization of the quantified variables, 
the algorithm proceeds with the recursive call, returning
constraints $[[SC]]$. As the output, the algorithm removes the freshly
introduced algorithmic variables from the instantiation context, This operation is
sound: it is guaranteed that $[[SC]]$ always has a solution, but the specific
instantiation of the freshly introduced algorithmic variables is not important,
as they do not occur in the input types.

\paragraph*{Functions}
To infer the subtyping of the function types, the algorithm
makes two calls: 
\begin{enumerate*}
  \item[(i)] a recursive call ensuring the subtyping of the result types, and
  \item[(ii)] a call to positive subtyping (or rather super-typing) on the argument types.
\end{enumerate*}
The resulting constraints are merged
using a special procedure defined in \cref{sec:constraint-merge}
and returned as the output.

\paragraph*{Algorithmic variables}
If one of the sides of the subtyping is a unification variable, the algorithm
creates a new constraint. Because the right-hand side of the subtyping is always
declarative, it is only the left-hand side that can be a unification variable.
Moreover, another invariant we preserve prevents the negative algorithmic
variables from occurring in types during the negative subtyping algorithm. It
means that the only possible form of the subtyping here is $[[α̂⁺]] [[≥]]
[[iP]]$, which is covered by \ruleref{\ottdruleAPUVarLabel}.

The potential problem here is that the type $[[iP]]$ might be not well-formed in
the instantiation context required for $[[α̂⁺]]$ by $[[Θ]]$ because this
context might be smaller than the current context $[[Γ]]$. As we wish the
resulting constraint set to be sound \wrt $[[Θ]]$, we cannot simply put 
$[[α̂⁺ :≥ iP]]$ into the output. Prior to that, we update the type $[[iP]]$ to its
lowest supertype $[[iQ]]$ well-formed in $[[Θ(α̂⁺)]]$. It is done by the
\emph{upgrade} procedure, which we discuss in detail in \cref{sec:lub}.

\vspace{\baselineskip}
% \indent
To summarize, the subtyping algorithm uses the following additional subroutines:
\begin{enumerate*}[noitemsep]
  \item[(i)] rules \ruleref{\ottdruleAShiftDLabel} and
    \ruleref{\ottdruleAShiftULabel} invoke the \emph{unification} algorithm
    to equate the input types;
  \item[(ii)] rule \ruleref{\ottdruleAArrowLabel} \emph{merges} the constraints
    produced by the recursive calls on the result and the argument types; and
  \item[(iii)] rule \ruleref{\ottdruleAPUVarLabel} \emph{upgrades} the input type
    to its least supertype well-formed in the context required by the
    algorithmic variable.
\end{enumerate*}
The following sections discuss these additional procedures in detail.

%%%%%%%%%%%%%%%%%
%%%%%%%%%%%%%%%%%
\subsection{Unification}
\label{sec:unification}

As an input, the unification context takes a type context $[[Γ]]$, an
instantiation context $[[Θ]]$, and two types of the required polarity: $[[uN]]$
and  $[[iM]]$ for the negative unification, and $[[uP]]$ and  $[[iQ]]$ for the
positive unification. It is assumed that only the left-hand side type may
contain algorithmic variables. This way, the left-hand side is well-formed as an
algorithmic type in $[[Γ]]$ and $[[Θ]]$, whereas the right-hand side is
well-formed declaratively in $[[Γ]]$.

Since only the left-hand side may contain algorithmic variables that the unification instantiates, we could have called this procedure \emph{matching}.
However, in \cref{sec:weakening-invariant}, we will discuss several
modifications of the type system, where this invariant is not preserved, and
therefore, this procedure requires general first-order pattern unification
\cite{miller1991:unification}.

As the output, the unification algorithm returns \emph{the weakest} set of
unification constraints $[[UC]]$ such that \emph{any} instantiation satisfying
these constraints unifies the input types.

\begin{figure}[t]
  \hfill
  \begin{multicols}{2}
  \ottdefnUNUnifLabeled{}
  \columnbreak\\
  \ottdefnUPUnifLabeled{}
  \end{multicols}
  \caption{Unification Algorithm}
  \label{fig:unification}
\end{figure}

The algorithm works as one might expect:
if both sides are formed by constructors, 
it is required that the constructors are the same, and the
types unify recursively. If one of the sides
is a unification variable (in our case it can only be the left-hand side),
we create a new unification constraint restricting it to be equal to the other side.
Let us discuss the rules that implement this strategy. 

\paragraph*{Variables}
  The variable rules \ruleref{\ottdruleUNVarLabel} and \ruleref{\ottdruleUPVarLabel}
  are trivial: as the input types do not have algorithmic variables, and are already equal, 
  the unification returns no constraints.

\paragraph*{Shifts}
  The shift rules \ruleref{\ottdruleUShiftDLabel} and \ruleref{\ottdruleUShiftULabel}
  require the two input types to be formed by the same shift constructor. 
  They remove this constructor, unify the types recursively, and return the resulting
  set of constraints.

\paragraph*{Quantifiers} 
  Similarly, the quantifier rules \ruleref{\ottdruleUForallLabel} and \ruleref{\ottdruleUExistsLabel}
  require the quantifier variables on the left-hand side and the right-hand side to be the same.
  This requirement is complete because we assume the input types of the unification 
  to be normalized, and thus, the equivalence implies alpha-equivalence. 
  In the implementation of this rule, an alpha-renaming might be needed to ensure 
  that the quantified variables are the same, however, we omit it for brevity.

\paragraph*{Functions}
  Rule \ruleref{\ottdruleUArrowLabel} unifies two functional types. 
  First, it unifies their argument types and their result types recursively.
  Then it merges the resulting constraints using the procedure described in \cref{sec:constraint-merge}.

  Notice that the resulting constraints can only have \emph{unification}
  entries. It means that they can be merged in a simpler way than general
  constraints. In particular, the merging procedure does not call any of the
  subtyping subroutines but rather simply checks the matching constraint entries
  for equality.

\paragraph*{Algorithmic variable}
  Finally, if the left-hand side of the unification is an algorithmic variable,
  \ruleref{\ottdruleUNVarLabel} or \ruleref{\ottdruleUPVarLabel} is applied. 
  It simply checks that the right-hand side type is well-formed in the required
  instantiation context, and returns a newly created constraint restricting the variable
  to be equal to the right-hand side type.

\vspace{\baselineskip}
As one can see, the unification procedure is standard; the only peculiarity
(although it is common for type inference) is that it makes sure that the resulting
instantiations agree with the input instantiation context $[[Θ]]$.  As a
subroutine, the unification algorithm only uses the (unification) constraint
merge procedure and the well-formedness checking.

%%%%%%%%%%%%%%%%%
%%%%%%%%%%%%%%%%%
\subsection{Constraint Merge}
\label{sec:constraint-merge}

In this section, we discuss the constraint merging procedure.
It allows one to combine two constraint sets into one. 
A simple union of two constraint sets is not sufficient, 
since the resulting set must not contain two entries restricting the 
same algorithmic variable---we call such entries \emph{matching}.

The matching entries must be combined into \emph{one} constraint entry, that
would represent their conjunction. This way, to merge two constraint sets, we
unite the entries of two sets and then merge the matching pairs.

\paragraph*{Merging matching constraint entries}
Two \emph{matching} entries formed in the same context $[[Γ]]$ 
can be merged as shown in \cref{fig:merge-entries}.
Suppose that $[[scE1]]$ and $[[scE2]]$ are input entries. 
The result of the merge $[[scE1 & scE2]]$ is 
\emph{the weakest entry which implies both $[[scE1]]$ and $[[scE2]]$}.

\begin{figure}[t]
  \ottdefnSCMELabeled\\
  \caption{Merge of Matching Constraint Entries}
  \label{fig:merge-entries}
\end{figure}

Suppose that one of the input entries, say $[[scE1]]$, is a \emph{unification}
constraint entry. Then the resulting entry $[[scE1]]$ must coincide with it 
(up-to-equivalence), and thus, it is only required to check that $[[scE2]]$ 
is implied by $[[scE1]]$. We consider two options:
\begin{itemize}
  \item[(i)] if $[[scE2]]$ is also a \emph{unification} entry, then the types on
    the right-hand side of $[[scE1]]$ and $[[scE2]]$ must be equivalent, as
    given by rules \ruleref{\ottdruleSCMEPEqEqLabel} and
    \ruleref{\ottdruleSCMENEqEqLabel};
  \item[(ii)] if $[[scE2]]$ is a \emph{supertype} constraint entry $[[α̂⁺ :≥ iP]]$,
    the algorithm must check that the type assigned by $[[scE1]]$ is a supertype of $[[iP]]$.
    The corresponding symmetric rules are \ruleref{\ottdruleSCMESupEqLabel} and \ruleref{\ottdruleSCMEEqSupLabel}.
    In the premises of these rules, $[[uP]]$ is the same as $[[iP]]$ below, 
    and $[[uQ]]$ is the same as $[[iQ]]$ below but relaxed to an algorithmic type.
\end{itemize}

If both input entries are supertype constraints: $[[α̂⁺ :≥ iP]]$ and $[[α̂⁺ :≥ iQ]]$,
then their conjunction is $[[α̂⁺ :≥ iP ∨ iQ]]$, as given by \ruleref{\ottdruleSCMESupSupLabel}.
The least upper bound---$[[iP ∨ iQ]]$---is the least supertype of both $[[iP]]$ and $[[iQ]]$,
and this way, $[[α̂⁺ :≥ iP ∨ iQ]]$ is the weakest constraint entry that implies
$[[α̂⁺ :≥ iP]]$ and $[[α̂⁺ :≥ iQ]]$. The algorithm finding the least upper bound
is discussed in \cref{sec:lub}.

\paragraph*{Merging constraint sets}
  The algorithm for merging constraint sets is shown in \cref{fig:merge-subtyping-constraints}.
  As discussed, the result of merge $[[SC1]]$ and $[[SC2]]$ consists of three parts: 
  \begin{enumerate*}
    \item[(i)] the entries of $[[SC1]]$ that do not match any entry of $[[SC2]]$;
    \item[(ii)] the entries of $[[SC2]]$ that do not match any entry of $[[SC1]]$; and
    \item[(iii)] the merge (\cref{fig:merge-entries}) of matching entries.
  \end{enumerate*}


\begin{figure}[t]
  Suppose that $[[Θ ⊢ SC1]]$ and $[[Θ ⊢ SC2]]$.\\
  Then $[[Θ ⊢ SC1 & SC2 = SC]]$
  defines a set of constraints $[[SC]]$ such that $[[scE]] \in [[SC]]$ iff either:
  \begin{itemize}
    \item $[[scE]] \in [[SC1]]$ and there is no matching $[[scE']] \in [[SC2]]$; or
    \item $[[scE]] \in [[SC2]]$ and there is no matching $[[scE']] \in [[SC1]]$; or
    \item $[[Θ(α̂±) ⊢ scE1 & scE2 = scE]]$ for some $[[scE1]] \in [[SC1]]$ and $[[scE2]] \in [[SC2]]$
      such that $[[scE1]]$ and $[[scE2]]$ both restrict variable $[[α̂±]]$. 
  \end{itemize}

  \caption{Constraint Merge}
  \label{fig:merge-subtyping-constraints}
\end{figure}

As shown in \cref{fig:merge-entries}, the merging procedure relies 
substantially on the least upper bound algorithm.
In the next section, we discuss this algorithm in detail,
together with the upgrade procedure, selecting the least supertype 
well-formed \emph{in a given context}.

%%%%%%%%%%%%%%%%%
%%%%%%%%%%%%%%%%%
\subsection{Type Upgrade and the Least Upper Bounds}
\label{sec:lub}

Both type upgrade and the least upper bound algorithms are used
to find a minimal supertype under certain conditions. 
For a given type $[[iP]]$ well-formed in $[[Γ]]$, the \emph{upgrade} operation 
finds the least among those supertypes of $[[iP]]$ that are well-formed
in a smaller context $[[Δ ⊆ Γ]]$.
For given two types $[[iP1]]$ and $[[iP2]]$ well-formed in $[[Γ]]$,
the \emph{least upper bound} operation finds the least among
common supertypes of $[[iP1]]$ and $[[iP2]]$ well-formed in $[[Γ]]$.
These algorithms are shown in \cref{fig:type-upgrade}.

\begin{figure}[t]
  \begin{multicols}{2}
    \ottdefnLUBUpLabeled{}
    \columnbreak\\
    \ottdefnLUBNsubLabeled{}
  \end{multicols}
  \caption{Type Upgrade and Leas Upper Bound Algorithms}
  \label{fig:type-upgrade}
\end{figure}

\paragraph*{The Type Upgarde}
The type upgrade algorithm uses the least upper bound algorithm as a subroutine.
It exploits the idea that the free variables of a positive type $[[iQ]]$
cannot disappear in its subtypes (see \cref{prop:subtyping-preserves-fv}). 
It means that if 
a type $[[iP]]$ has free variables not occurring 
in $[[iP']]$, then any common supertype of $[[iP]]$
and $[[iP']]$ must not contain these variables either.
This way, any supertype of $[[iP]]$
not containing certain variables $[[pnas]]$ must also be 
a supertype of $[[iP' = [pnbs/pnas]iP ]]$, where $[[pnbs]]$ are fresh;
and vice versa: any common supertype of $[[iP]]$ and $[[iP']]$
does not contain $[[pnas]]$ nor $[[pnbs]]$.

This way, to find the least supertype of $[[iP]]$ well-formed in $[[Δ]] = [[Γ \ {pnas}]]$
(\ie not containing $[[pnas]]$), we can do the following.
First, construct a new type $[[iP']]$ by renaming $[[pnas]]$ in $[[iP]]$ to fresh $[[pnbs]]$,
and second, find \emph{the least upper bound} of $[[iP]]$ and $[[iP']]$ in the appropriate
context. However, for reasons of symmetry, in rule
\ruleref{\ottdruleLUBUpgradeLabel} we employ a different but equivalent approach:
we create \emph{two} types $[[iP1]]$ and $[[iP2]]$ constructed by renaming $[[pnas]]$ in $[[iP]]$
to fresh disjoint variables $[[pnbs]]$ and $[[pncs]]$ respectively, and then 
find the least upper bound of $[[iP1]]$ and $[[iP2]]$.

\paragraph*{The Least Upper Bound}
The Least Upper Bound algorithm we use operates on \emph{positive}
types. This way, the inference rules of the algorithm
analyze the three possible shapes of the input types:
a variable type, an existential type, and a shifted computation.

 Rule \ruleref{\ottdruleLUBExistsLabel} covers the case when 
 at least one of the input types is an existential type.
 In this case, we can simply move the existential quantifiers
 from both sides to the context, and make a tail-recursive call.
 However, it is important to make sure that 
 the quantified variables $[[nas]]$ and $[[nbs]]$ are disjoint
 (\ie alpha-renaming might be required in the implementation).
  
 Rule \ruleref{\ottdruleLUBVarLabel} applies when 
 both sides are variables. In this case,
 the common supertype only exists if these variables are 
 the same. And if they are, the common supertypes
 must be equivalent to this variable.

Rule \ruleref{\ottdruleLUBShiftLabel} is the most 
interesting. If both sides are not quantified, and one of the sides is 
a shift, so must be the other side. 
However, the set of common upper bounds is not trivial in this case.
For example, $[[↓(β⁺ → γ1⁻)]]$ and $[[↓(β⁺ → γ2⁻)]]$ have
two non-equivalent common supertypes: 
$[[∃α⁻.↓α⁻]]$ 
(by instantiating $[[α⁻]]$ with $[[β⁺ → γ1⁻]]$ and $[[β⁺ → γ2⁻]]$ respectively)
and 
$[[∃α⁻.↓(β⁺ → α⁻)]]$ 
(by instantiating $[[α⁻]]$ with $[[γ1⁻]]$ and $[[γ2⁻]]$ respectively).
As one can see, the second supertype $[[∃α⁻.↓(β⁺ → α⁻)]]$ is the least among them
because it abstracts over a `deeper' negative subexpression.

In general, we must 
\begin{itemize*}
  \item[(i)] find the most detailed pattern (a type with `holes' at negative positions) 
    that matches both sides, and 
  \item[(ii)] abstract over the `holes' by existential quantifiers.
\end{itemize*}
The algorithm that finds the most detailed common pattern is called \emph{anti-unification}.
As output, it returns $[[(Ξ, uP, aus1, aus2)]]$, where important for us is
$[[uP]]$---the pattern---and $[[Ξ]]$--the set of `holes' represented by negative algorithmic variables.
We discuss the anti-unification algorithm in detail in the following section.


%%%%%%%%%%%%%%%%%
%%%%%%%%%%%%%%%%%
\subsection{Anti-Unification}
\label{sec:antiunification}

The anti-unification algorithm
\cite{plotkin1970:generalization,reynolds1970:transform}, is a procedure dual to
unification. For two given (potentially different) expressions, it finds the most
specific generalizer---the most detailed pattern that matches both of the input
expressions. As evidence, it can also return two substitutions that instantiate
the `holes' of the pattern to the input expressions.

In our case, we have to be more demanding on the anti-unification algorithm.
Since we use it to construct an existential type, whose (negative) quantified
variables can only be instantiated with negative types, we must make sure that
the pattern has `holes' only at negative positions. Moreover, we must make sure
that the resulting substitutions for the `holes' are well-formed in the context
from the past---at the moment when the corresponding polymorphic variables were
introduced---and do not contain variables bound later. For example, the
anti-unification of $[[iN1 = ∀β⁺.α1⁺ → ↑β⁺]]$ and $[[iN2 = ∀β⁺.α2⁺ → ↑β⁺]]$ is a
singleton `hole', which we model as an algorithmic type variable $[[γ̂⁻]]$, with
a pair of substitutions $[[γ̂⁻ ↦ iN1]]$ and $[[γ̂⁻ ↦ iN2]]$. But it
\emph{cannot} be more specific such as $[[∀β⁺.γ̂⁺ → ↑β⁺]]$ (since the hole
cannot be positive) or $[[∀β⁺.γ̂⁻]]$ (since \emph{the instantiation of $~[[γ̂⁻]]$ cannot capture the
bound variable $[[β⁺]]$}).

The algorithm that finds the most specific generalizer of two types
under required conditions is given in \cref{fig:anti-unification}.
It consists of two mutually recursive procedures:
the positive and the negative anti-unification. 
As the positive and the negative anti-unification procedures
are symmetric in their interface, let us discuss how to read
the positive judgment. 

The positive anti-unification judgment has form
$[[Γ ⊨ iP1 ≈au iP2 ⫤ ( Ξ , uQ , aus1 , aus2 )]]$.
As an input, it takes a context $[[Γ]]$, in which the `holes'
instantiations must be well-formed, 
and two positive types: $[[iP1]]$ and $[[iP2]]$;
it returns a tuple of four components:
$[[Ξ]]$---a set of `holes' represented by negative algorithmic variables,
$[[uQ]]$---a pattern represented as a positive algorithmic type,
whose algorithmic variables are in $[[Ξ]]$,
and two substitutions $[[aus1]]$ and $[[aus2]]$
instantiating the variables from $[[Ξ]]$ such that
$[[ [aus1]uQ = iP1 ]]$ and $[[ [aus2]uQ = iP2 ]]$. 

\begin{figure}[t]
    \ottdefnAUAUPLabeled{}
    \hfill\\
    \hfill\\
    \ottdefnAUAUNLabeled{}
    \caption{Anti-Unification Algorithm}
    \label{fig:anti-unification}
\end{figure}

At the high level, the algorithm scheme follows
the standard approach \cite{plotkin1970:generalization}
represented as a recursive procedure. Specifically, it follows two principles:
\begin{enumerate}
  \item[(i)] if the input terms start with the same constructor,
    we anti-unify the corresponding parts recursively and 
    unite the results. This principle is followed by 
    all the rules except \ruleref{\ottdruleAUAULabel},
    which works as follows:
  \item[(ii)] if the first principle does not apply to the input terms $[[iN]]$
    and $[[iM]]$ (for instance, if they have different outer constructors), the
    anti-unification algorithm returns a `hole' such that one substitution maps
    it to $[[iN]]$ and the other maps it to $[[iM]]$. This `hole'
    should have a name uniquely defined by the pair $([[iN]], [[iM]])$, so
    that it automatically merges with other `holes' mapped to the same
    pair of types, and thus, the initiality of the generalizer is ensured. 
\end{enumerate}

Let us discuss the specific rules of the algorithm in detail.

\paragraph*{Variables} 
  Rules \ruleref{\ottdruleAUPVarLabel} and
  \ruleref{\ottdruleAUNVarLabel} generalize two equal variables.
  In this case, the resulting pattern is the variable itself,
  and no `holes' are needed.

\paragraph*{Shifts} 
  Rules \ruleref{\ottdruleAUShiftDLabel} and
  \ruleref{\ottdruleAUShiftULabel} operate by congruence:
  they anti-unify the bodies of the shifts recursively and add
  the shift constructor back to the resulting pattern.

\paragraph*{Quantifiers}
  Rules \ruleref{\ottdruleAUForallLabel} and
  \ruleref{\ottdruleAUExistsLabel} are symmetric. 
  They generalize two quantified types congruently, 
  similarly to the shift rules. 
  However, we also require that the quantified variables
  are fresh, and that the left-hand side variables are 
  equal to the corresponding variables on the right-hand side.
  To ensure it, alpha-renaming might be required in the
  implementation.

  Notice that the context $[[Γ]]$ is \emph{not} extended with 
  the quantified variables. In this algorithm, $[[Γ]]$ 
  does not play the role of a current typing context, but rather
  a snapshot of a context at the moment of calling the anti-unification,
  \ie the context in which the instantiations of the `holes' 
  must be well-formed.

\paragraph*{Functions}
  Rule \ruleref{\ottdruleAUArrowLabel} congruently generalizes two function types. 
  An arrow type is the only binary constructor, 
  and thus, it is the only rule where the union of the anti-unification results is substantial.
  The interesting is the case when the resulting generalization of the 
  input types and the resulting generalization of the output types 
  have `holes' mapped to the same pair of types. 
  In this case, the algorithm must merge the `holes' into one.
  For example, the anti-unification of 
  $[[↓α⁻ → α⁻]]$ and $[[↓β⁻ → β⁻]]$
  must result in $[[↓γ̂⁻ → γ̂⁻]]$,
  rather than $[[↓γ1̂⁻ → γ2̂⁻]]$.

  In our representation of the anti-unification algorithm, this `merge' happens
  automatically:
  following the rule \ruleref{\ottdruleAUAULabel},
  the name of the `hole' is uniquely defined by the pair of types it is mapped to.  
  Specifically, when anti-unifying $[[↓α⁻ → α⁻]]$ and $[[↓β⁻ → β⁻]]$ our algorithm returns 
  $[[↓α̂⁻_{α⁻, β⁻} → α̂⁻_{α⁻, β⁻}]]$, that is a renaming of $[[↓γ̂⁻ → γ̂⁻]]$.

  This way, as the output the rule returns the following tuple:
  \begin{itemize}
    \item $[[Ξ1 ∪ Ξ2]]$---a simple union of the sets of `holes' 
      returned from by the recursive calls, 
    \item $[[uQ → uM]]$---the resulting pattern 
      constructed from the patterns returned recursively.
    \item $[[aus1 ∪ aus1']]$ and $[[aus2 ∪ aus'2]]$
      --- a union (in a relational sense)
      of the substitutions returned by the recursive calls. 
      It is worth noting that the union is well-defined because
      the result of the substation on a `hole' is determined by the 
      name of the `hole'.
  \end{itemize}

\paragraph*{The Anti-Unification Rule}
  Rule \ruleref{\ottdruleAUAULabel} is the base case of the anti-unification
  algorithm. If the congruent rules are not applicable, 
  it means that the input types have a substantially different structure,
  and thus, the only option is to create a `hole'. 
  There are three important aspects of this rule that we would like to discuss.

  First, as mentioned earlier, the freshly created `hole' has a name that is
  uniquely defined by the pair of input types. It is ensured by the following
  invariant: all the `holes' in the algorithm have name $[[α̂⁻]]$ indexed by the
  pair of negative types it is mapped to. This way, the returning set of `holes'
  is a singleton set $\{ [[ â⁻_{iN, iM} ]]\}$; the resulting pattern is the
  `hole' $[[â⁻_{iN, iM}]]$, and the mappings simply send it to the
  corresponding types: $[[â⁻_{iN, iM} ↦ iN]]$ and $[[â⁻_{iN, iM} ↦ iM]]$.

  Second, this rule is only applicable to negative types; moreover, the input
  types are checked to be well-formed in the outer context $[[Γ]]$. This is
  required by the usage of anti-unification: we call it to build an existential
  type that would be an upper bound of two input types via abstracting some of
  their subexpressions under existential quantifiers. The existentials quantify
  over \emph{negative} variables, and they must be instantiated in the context
  available at that moment.

  Third, the rule is only applicable if all other rules fail. Notice that it
  could happen even when the input types have matching constructors. For
  example, the generalizer of $[[↑α⁺]]$ and $[[↑β⁺]]$ is $[[γ̂⁻]]$ (with
  mappings $[[γ̂⁻ ↦ ↑α⁺]]$ and $[[γ̂⁻ ↦ ↑β⁺]]$), rather than $[[↑γ̂⁺]]$. This
  way, the algorithm must try to apply the congruent rules first, and only if
  they fail, apply \ruleref{\ottdruleAUAULabel}. As it is not known a priori
  whether the congruent rules will be applicable, the implementation must use 
  backtracking.

%%%%%%%%%%%%%%%%%
%%%%%%%%%%%%%%%%%
\subsection{Type Inference}
\label{sec:typing}

Finally, we present the type inference algorithm. Similarly to the subtyping
algorithm, it structurally corresponds to the declarative inference
specification, meaning that most of the algorithmic rules have declarative
counterparts, with respect to which they are sound and complete.

This way, the inference algorithm also consists of three mutually recursive
procedures: the positive type inference, the negative type inference, and the
application type inference. As subroutines, the inference algorithm uses
subtyping, constraint merge, and minimal instantiation. The corresponding
dependency is shown in \cref{fig:alg-typing-graph}.

\begin{figure}[t]
  \centering
  \begin{tikzpicture}
    [>={Stealth[scale=2]},node distance=2.2cm,every node/.style={draw,rectangle},every text node part/.style={align=center}]
    % Define nodes
    \node[] (1) {Positive Inference\\$[[Γ; Φ ⊨ v : iP]]$};
    \node[right=1.4cm of 1] (2) {Negative Inference\\$[[Γ; Φ ⊨ c : iN]]$};
    \node[right=1.2cm of 2] (3) {Application Inference\\$[[Γ ; Φ ; Θ1 ⊨ uN ● args ⇒> uM ⫤ Θ2 ; SC]]$};
    \node[below=0.6cm of 2] (7) {Minimal Instantiation\\$[[Γ ⊢ uP SC minby uσ]]$\\(\cref{sec:constraint-singularity})};
    \node[below=0.5cm of 7] (8) {Constraint Singularity\\$[[SC singular with uσ]]$\\(\cref{sec:constraint-singularity})};
    \node[right=0.3cm of 7] (4) {Constraint Merge\\$[[Θ ⊢ SC1 & SC2 = SC3]]$\\(\cref{sec:constraint-merge})};
    \node[left=0.3cm of 7] (6) {Negative Subtyping\\$[[Γ ; Θ ⊨ uN1 ≤ iN2 ⫤ SC]]$\\(\cref{sec:subtyping-algorithm})};
    \node[below=0.9cm of 6] (5) {Positive Subtyping\\$[[Γ ; Θ ⊨ uP1 ≥ iP2 ⫤ SC]]$\\(\cref{sec:subtyping-algorithm})};
    \draw[->] ([yshift=-.2cm]2.west) to ([yshift=-.2cm]1.east);
    \draw[->] ([yshift=+.2cm]1.east) to ([yshift=+.2cm]2.west);
    \draw[->, bend right=20] (3) to (1);
    \draw[->] (2) to (3);
    \draw[->] (2) to (4);
    \draw[->] (2) to (7);
    \draw[->] (3) to (4);
    \draw[->] (7) to (8);
    \draw[->] (1.south west) |- (5);
    \draw[->] (3) |- (5.south east);
    \draw[->] (2) to (6);
  \end{tikzpicture}  
  \caption{Dependency graph of the typing algorithm}
  \label{fig:alg-typing-graph}
\end{figure}

The positive and the negative type inference judgments have symmetric forms:
$[[Γ; Φ ⊨ v : iP]]$ and $[[Γ; Φ ⊨ c : iN]]$. Both of these algorithms
take as an input typing context $[[Γ]]$, a variable context $[[Φ]]$, and 
a term (a value or a computation) taking its type variables from $[[Γ]]$, 
and term variables from $[[Φ]]$. As an output, they return a type
of the given term, which we guarantee to be normalized. 

The application type inference judgment has form 
$[[Γ ; Φ ; Θ1 ⊨ uN ● args ⇒> uM ⫤ Θ2 ; SC]]$.
As an input, it takes three contexts: typing context $[[Γ]]$, a variable context $[[Φ]]$,
and an instantiation context $[[Θ1]]$. It also takes a head type $[[uN]]$ and 
a list of arguments (terms) $[[args]]$ the head is applied to.
The head may contain algorithmic variables specified by $[[Θ1]]$, 
in other words, $[[Γ; dom(Θ1) ⊢ uN]]$.
As a result, the application inference judgment returns 
$[[uM]]$---a normalized type of the result of the application.
Type $[[uM]]$ may contain new algorithmic variables, and thus, 
the judgment also returns $[[Θ2]]$---an updated instantiation context
and $[[SC]]$---a set of subtyping constraints.
Together $[[Θ2]]$  and $[[SC]]$ specify how the algorithmic variables 
must be instantiated.


\begin{figure}[t]
  \ottdefnATPInfLabeled{}
  \hfill \\
  \ottdefnATNInfLabeled{}
  \hfill \\
  \ottdefnATSpinInfLabeled{}
  \caption{Algorithmic Type Inference}
  \label{fig:type-inference}
\end{figure}


The inference rules are shown in
\cref{fig:type-inference}.
Next, we discuss them in detail.

\paragraph*{Variables}
  Rule \ruleref{\ottdruleATVarLabel} 
  infers the type of a positive variable by looking it up in the 
  term variable context and normalizing the result.

\paragraph*{Annotations}
  Rules \ruleref{\ottdruleATPAnnotLabel} and \ruleref{\ottdruleATNAnnotLabel}
  are symmetric.
  First, they check that the annotated type is well-formed in the 
  given context $[[Γ]]$. Then they make a recursive call to infer the 
  type of annotated expression, check that the inferred type is a subtype of 
  the annotation, and return the normalized annotation.

\paragraph*{Abstractions}
  Rule \ruleref{\ottdruleATtLamLabel} infers the type of a lambda abstraction.
  It checks the well-formedness of the annotation $[[iP]]$,
  makes a recursive call to infer the type of the body in the extended context, 
  and returns the corresponding arrow type.
  Since the annotation $[[iP]]$ is allowed to be non-normalized,
  the rule also normalizes the resulting type.

  Rule \ruleref{\ottdruleATTLamLabel} infers the type of a big lambda.
  Similarly to the previous case, it makes a recursive call to infer the type
  of the body in the extended \emph{type} context. 
  After that, it returns the corresponding universal type. 
  It is also required to normalize the result.  
  For instance, if $[[α⁺]]$ does not occur in the body of the lambda,
  the corresponding $[[∀]]$ will be removed.

\paragraph*{Return and Thunk}
  Rules \ruleref{\ottdruleATThunkLabel} and \ruleref{\ottdruleATReturnLabel}
  are similar to the declarative rules: they make a recursive call
  to type the body of the thunk or the return expression and
  put the shift on top of the result.

\paragraph*{Unpack}
  Rule \ruleref{\ottdruleATUnpackLabel}
  allows one to unpack an existential type.
  First, it infers the existential type $[[∃nas.iP]]$ of the value being unpacked,
  and since the type is guaranteed to be normalized, binds 
  the quantified variables with $[[nas]]$.
  Then it infers the type of the body in the appropriately extended context
  and checks that the inferred type does not depend on $[[nas]]$
  by checking well-formedness $[[Γ ⊢ iN]]$.

\paragraph*{Let Binders}
  Rule \ruleref{\ottdruleATVarLetLabel} represents the type inference of a
  standard let binder. It infers the type of the bound value $[[v]]$, and makes
  a recursive call to infer the type of the body in the extended context.

  Rule \ruleref{\ottdruleATCVarLetLabel} infers a type of computational let
  binder. It follows the corresponding declarative rule
  \ruleref{\ottdruleDTCVarLetLabel} but uses algorithmic judgments instead of
  declarative ones. It is worth noting that when calling the subtyping 
  $[[Γ ; · ⊨ uM ≤ ↑iP ⫤ ·]]$, both $[[uM]]$ and $[[iP]]$ are free of algorithmic
  variables: $[[uM]]$ is a type inferred for $[[c]]$, and $[[iP]]$ is given as
  an annotation.

  Rule \ruleref{\ottdruleATAppLetAnnLabel}
  infers a type of \emph{annotated} applicative let binder.
  First, it infers the type of the head of the application,
  ensuring that it is a \emph{thunked computation} $[[↓iM]]$.
  After that, it makes a recursive call
  to the application inference procedure,
  returning an algorithmic type $[[uM']]$, 
  that must be a subtype of the annotation $[[↑iP]]$.

  Then premise $[[Γ; Θ ⊨ uM' ≤ ↑iP ⫤ SC2]]$
  together with $[[Θ ⊢ SC1 & SC2 = SC]]$
  check that $[[uM']]$ can be instantiated to the annotated type $[[↑iP]]$,
  and if it is, the algorithm infers the type of the body in the extended context,
  and returns it as the result. 

  Rule \ruleref{\ottdruleATAppLetLabel}
  works similarly to \ruleref{\ottdruleATAppLetAnnLabel}.
  However, since no annotation is given,
  the algorithm must ensure that the inferred $[[uQ]]$
  has the `canonical' minimal instantiation.
  To find it, it makes a call to the minimal instantiation algorithm 
  (\cref{sec:constraint-singularity})
  that finds the substitution that satisfies the inferred constraints $[[SC]]$ and
  instantiates $[[uQ]]$ to the minimal (among other such instantiations)
  type $[[ [uσ]uQ ]]$.


\paragraph*{Application to an Empty List of Arguments}
  Rule \ruleref{\ottdruleATEmptyAppLabel}
  is the base case of application inference. 
  If the list of applied arguments is empty, 
  the inferred type is the type of the head,
  and the algorithm returns it after normalizing.

\paragraph*{Application of a Polymorphic Type $[[∀]]$}
  Rule \ruleref{\ottdruleATForallAppLabel},
  analogously to the declarative case,
  is the rule ensuring the implicit elimination of the universal quantifiers. 
  This is the place where the algorithmic variables are introduced.
  The algorithm simply replaces the quantified variables 
  $[[pas]]$ with fresh algorithmic variables $[[puas]]$,
  and makes a recursive call in the extended context.

  To ensure the rule precedence, we also require
  the head type to have at least one $[[∀]]$-quantifier, 
  and the list of arguments to be non-empty.

\paragraph*{Application of an Arrow Type}
  Rule \ruleref{\ottdruleATArrowAppLabel}
  is the main rule of algorithmic application inference.
  It is applied when the head has an arrow type $[[uQ → uN]]$.
  First, it infers the type of the first argument $[[v]]$,
  and then, calling the algorithmic subtyping,
  finds $[[SC1]]$---the \emph{minimal} constraint ensuring that 
  $[[uQ]]$ is a supertype of the type of $[[v]]$.
  Then it makes a recursive call applying $[[uN]]$ to the rest of the arguments 
  and merges the resulting constraint with $[[SC1]]$.

\subsection{Minimal Instantiation and Constraint Singularity}
\label{sec:constraint-singularity}

Multiple types $[[iM]]$ can be inferred for a type application:
$[[Γ ; Φ ⊢ iN ● args ⇒> iM]]$ but only one \emph{principal} 
type should be chosen for a variable in an unannotated let binder.
Declaratively, we require the principal type $[[iP]]$ to be
minimal among other all types $[[iP']]$ that can be inferred
for the application $[[Γ ; Φ ⊢ iN ● args ⇒> ↑iP']]$.
Algorithmically, the inference returns an
\emph{algorithmic} type $[[uP]]$ 
together with a set of \emph{necessary and sufficient} 
constraints $[[SC]]$ that any instantiation of $[[uP]]$ must satisfy.
This way, the principal type is the minimal instantiation of $[[uP]]$,
obtained by a substitution satisfying the given constraints $[[SC]]$. 

To find this substitution, we use the minimal instantiation
algorithm (\cref{fig:minimal-instantiation}). The judgment `$[[Γ ⊢ uP SC minby uσ]]$'
$[[uσ]]$ instantiates $[[uP]]$ to a subtype of any other 
instantiation of $[[uP]]$ that satisfies $[[SC]]$. 
First, it removes the existential quantifiers by \ruleref{\ottdruleSINGExistsLabel}.
Then it considers two cases. 
\begin{enumerate}
  \item If the type $[[uP]]$ is an algorithmic variable $[[α̂⁺]]$
    restricted by a \emph{subtyping} constraint $[[(α̂⁺ :≥ iQ) ∊ SC]]$,
    its minimal instantiation is the type $[[nf(iQ)]]$, and 
    \ruleref{\ottdruleSINGPUvarLabel} is applied.
  \item Otherwise, the type $[[uP]]$ is either a shifted computation, a
    declarative variable, an unrestricted algorithmic variable, or an
    algorithmic variable restricted by an \emph{equivalence} constraint. In all
    of these cases, the minimal instantiation exists if and only if there is
    only one possible instantiation of $[[uP]]$ that satisfies $[[SC]]$.
    
    The algorithm finds this instantiation by two premises:
    \begin{enumerate*}
      \item[(i)] checking that all the algorithmic variables of the type are
      restricted by the constraint set; and
      \item[(ii)] building the substitution that satisfies the constraint set on
        these variables (and simultaneously checking that such substitution is
        unique up to equivalence) using the \emph{constraint singularity algorithm}.
    \end{enumerate*}
  \end{enumerate}


\begin{figure}[t]
  \hfill\\
  \ottdefnMININSTLabeled
  \caption{Minimal Instantiation}
  \label{fig:minimal-instantiation}
\end{figure}

The singularity algorithm 
performs two tasks: it checks that a
\emph{constraint set} has a single substitution satisfying it, 
and if it does, it builds this substitution.

\begin{figure}[t]
  \begin{minipage}{0.49\textwidth}
    `$[[SC singular with uσ]]$' means:
    \begin{itemize}[leftmargin=*]
      \item[$+$] for any \emph{positive} constraint entry $[[scE ∊ SC]]$ restricting 
        a variable $[[β̂⁺]]$, there exists $[[iP]]$ such that $[[scE singular with iP]]$
        (as defined in \cref{fig:constraint-entry-singularity}),
        and $[[ [uσ]β̂⁺ = iP ]]$; and
      \item[$-$] the symmetric property holds for all \emph{negative} $[[scE ∊ SC]]$; and
      \item[$\notin$] for any $[[α̂± ∉ dom(SC)]]$, $[[ [uσ]α̂± = α̂± ]]$.
    \end{itemize}
    \caption{Singular Constraint}
    \label{fig:constraint-singularity}
  \end{minipage}
  \begin{minipage}{0.48\textwidth}
    \ottdefnSINGLabeled
    \caption{Singular Constraint Entry}
    \label{fig:constraint-entry-singularity}
  \end{minipage}
\end{figure}

To implement the singularity algorithm, we define a partial function 
`$[[SC singular with uσ]]$', taking a subtyping constraint $[[SC]]$ as an argument and
returning a substitution $[[uσ]]$---the only solution of $[[SC]]$. 

The constraint $[[SC]]$ is composed of constraint entries. Therefore, we define
singularity by combining the singularity of each constraint entry
(\cref{fig:constraint-singularity}). In order for $[[SC]]$ to be singular, each
entry must have a unique instantiation, and the resulting substitution $[[uσ]]$
must be a union of these instantiations. In addition, $[[uσ]]$ must act as an
identity on the variables not restricted by $[[SC]]$.

The singularity of constraint \emph{entries} is defined in
\cref{fig:constraint-entry-singularity}. 
\begin{itemize}
  \item The \emph{equivalence} entries are
    always singular, as the only possible type satisfying them is the one given in
    the constraint itself, which is reflected in rules \ruleref{\ottdruleSINGNEqLabel} and
    \ruleref{\ottdruleSINGPEqLabel}. 
  \item The \emph{subtyping} constraints are trickier.
    As will be discussed in \cref{sec:proof-lub-upgrade}, variables (and equivalent
    them $[[∃nas.pa]]$) do not have proper supertypes, and thus, the
    constraints of a form $[[pua :≥ ∃nas.pa]]$ are singular with the only possible
    normalized solution $[[pa]]$---see rule \ruleref{\ottdruleSINGSupVarLabel}. 
  \item However, if the body of the existential type is guarded by a shift
    $[[∃nas.↓iN]]$, it is singular if and only if $[[iN]]$ is equivalent to some
    $[[αi⁻ ∊ {nas}]]$ bound by the quantifier---see rule
    \ruleref{\ottdruleSINGSupShiftLabel}. The completeness of this criterion is
    justified by the fact that if $[[iN]]$ is \emph{not} equivalent to any
    $[[αi⁻]]$, then there are two non-equivalent solutions of constraint $[[(pua :≥
    ∃nas.↓iN)]]$: the trivial $[[∃nas.↓iN]]$ and $[[∃nas.↓a1⁻]]$ (which is a
    supertype of $[[∃nas.↓iN]]$ since $[[a1⁻]]$ can be instantiated to
    $[[iN]]$).

\end{itemize}



\section{Algorithm Correctness}
\label{sec:proofs}
The central results ensuring the correctness of the inference algorithm are its
soundness and completeness with respect to the declarative specification. The
soundness means the algorithm will always produce a typing \emph{allowed} by the
declarative system; dually, the completeness says that once a term has some type
declaratively, the inference algorithm succeeds. 

The formal statements of soundness and completeness are given in the theorems
below. Notice that the theorems also include the soundness and completeness of
\emph{application inference} (labeled as $\bullet$), which is more complex. As
such, let us discuss it in more detail.

Both soundness and completeness of application inference assume that the input
head type $[[uN]]$ is free from \emph{negative} algorithmic variables---it is
achieved by polarization invariants preserved by the inference rules. The
soundness states that the output of the algorithm---$[[uM]]$ and $[[SC]]$---is
viable. Specifically, that the constraint set $[[SC]]$ provides a sufficient set
of restrictions that any substitution $[[uσ]]$ must satisfy to ensure the
\emph{declarative} inference of the output type $[[uM]]$, that is 
$[[ Γ ; Φ ⊢ [uσ]uN ● args ⇒> [uσ]uM ]]$.

The application inference completeness means that if there exists a substitution
$[[uσ]]$ and the resulting type $[[iM]]$ ensuring the declarative inference
$[[Γ; Φ ⊢ [uσ]uN ● args ⇒> iM]]$ then the algorithm succeeds and gives the most
general result $[[uM0]]$ and $[[SC0]]$. The property of `being the most general'
is specified in pt. (\ref{point:mostGeneral}). Intuitively, it means that for
any other solution---substitution $[[uσ]]$ and the resulting type $[[iM]]$, if
it ensures the declarative inference, then $[[uσ]]$ can be extended in a
$[[SC0]]$-complying way to equate $[[uM0]]$ with $[[iM]]$.

\footnotetext[1]{The presented properties hold, 
            but the actual inductive proof requires strengthening 
            of the statement and the corresponding theorem is more involved. 
            See the appendix (\cref{sec:appendix}) for details.}

\begin{theorem*}[Soundness of Typing]
    \label{thm:soundness-typing}
    Suppose that $[[Γ ⊢ Φ]]$. Then\footnotemark[1]
    \hfill
    \begin{itemize}
        \item [$+$] $[[Γ; Φ ⊨ v : iP]]$ implies $[[Γ; Φ ⊢ v : iP]]$,
        \item [$-$] $[[Γ; Φ ⊨ c : iN]]$ implies $[[Γ; Φ ⊢ c : iN]]$,
        \item [$\bullet$] $[[Γ; Φ; Θ ⊨ uN ● args ⇒> uM ⫤ Θ'; SC]]$ implies $[[ Γ ; Φ ⊢ [uσ]uN ● args ⇒> [uσ]uM ]]$, 
            for any instantiation of $[[uσ]]$ satisfying constraints $[[SC]]$.
            All of it under assumptions that $[[Γ ⊢ Θ]]$ and $[[Γ; dom(Θ) ⊢ uN]]$ and that $[[uN]]$ is free from 
            \emph{negative} algorithmic variables.
    \end{itemize}
\end{theorem*}

\begin{theorem*}[Completeness of Typing]
    \label{thm:completeness-typing}
    Suppose that $[[Γ ⊢ Φ]]$. Then\footnotemark[1]
    \begin{itemize}
        \item [$+$] $[[Γ; Φ ⊢ v : iP]]$ implies $[[Γ; Φ ⊨ v : nf(iP)]]$,
        \item [$-$] $[[Γ; Φ ⊢ c : iN]]$ implies $[[Γ; Φ ⊨ c : nf(iN)]]$,
        \item [$\bullet$] If $[[Γ; Φ ⊢ [uσ]uN ● args ⇒> iM]]$
            where 
            \begin{enumerate*}
                \item $[[Γ ⊢ Θ]]$, 
                \item $[[Γ ⊢ iM]]$,
                \item $[[Γ; dom(Θ) ⊢ uN]]$ (free from \emph{negative} algorithmic variables), and
                \item $[[Θ ⊢ uσ : uv(uN)]]$,
            \end{enumerate*}
            then there exist $[[uM0]]$, $[[Θ0]]$, and $[[SC0]]$ such that
            \begin{enumerate}
                \item $[[ Γ; Φ; Θ ⊨ uN ● args ⇒> uM0 ⫤ Θ0; SC0 ]]$ and
                \item \label{point:mostGeneral} for any other $[[uσ]]$ and $[[iM]]$ 
                (where $[[Θ ⊢ uσ : uv(uN)]]$ and $[[Γ ⊢ iM]]$)
                    such that $[[Γ; Φ ⊢ [uσ]uN ● args ⇒> iM]]$, 
                    there exists $[[uσ']]$ such that 
                    \begin{enumerate*}
                        \item $[[Θ0 ⊢ uσ' : uv uN ∪ uv uM0]]$
                            and $[[Θ0 ⊢ uσ' : SC0]]$,
                        \item $[[Θ ⊢ uσ' ≈ uσ : uv uN]]$, and 
                        \item $[[Γ ⊢ [uσ']uM0 ≈ iM]]$.
                    \end{enumerate*}
            \end{enumerate}
    \end{itemize}
\end{theorem*}



The proof of soundness and completeness result is done gradually
for all the subroutines,
following the structure of the algorithm 
(\cref{fig:alg-typing-graph,fig:alg-subtyping-graph})
bottom-up. Next, we discuss the main of these results. 

\subsection{Normalization}
    The point of type normalization is factoring out non-trivial equivalence 
    by selecting a representative from each equivalence class.
    This way, \emph{the correctness of normalization} means that
    checking for equivalence of two types is the same as checking for equality of their normal forms.
    \begin{lemma*}[Normalization correctness]
        Assuming all types are well-formed in $[[Γ]]$, we have
            $[[Γ ⊢ iN ≈ iM]] \iff [[nf(iN) = nf(iM)]]$ and 
            $[[Γ ⊢ iP ≈ iQ]] \iff [[nf(iP) = nf(iQ)]]$.
    \end{lemma*}
    To prove the correctness of normalization, 
    we introduce an \emph{intermediate} relation on types---\emph{declarative equivalence}
    (the notation is $[[iN ≈ iM]]$ and $[[iP ≈ iQ]]$).
    In contrast to $[[Γ ⊢ iN ≈ iM]]$ (which means mutual subtyping), $[[iN ≈ iM]]$ does not depend on subtyping judgments, 
    but explicitly allows quantifier reordering and redundant quantifier removal.
    Then the statement $[[Γ ⊢ iN ≈ iM]] \iff [[nf(iN) = nf(iM)]]$ (as well as its positive counterpart) 
    is split into two steps: $[[Γ ⊢ iN ≈ iM]] \iff [[iN ≈ iM]] \iff [[nf(iN) = nf(iM)]]$.

% \subsection{Unification Constraint Merge}

% \subsection{Unification}

% \begin{lemma*}[Soundness of Unification]
%     \label{lemma:unification-soundness}
%     \hfill
%     \begin{itemize}
%         \item [$+$] For normalized $[[uP]]$ and $[[iQ]]$ such that 
%         $[[Γ ; dom(Θ) ⊢ uP]]$ and $[[Γ ⊢ iQ]]$,\\ 
%         if $[[Γ ; Θ ⊨ uP ≈u iQ ⫤ UC]]$ then 
%         $[[Θ ⊢ UC : uv uP]]$ and for any normalized $[[uσ]]$ 
%         such that $[[ Θ ⊢ uσ : lift UC ]]$, $[[ [uσ]uP = iQ ]]$.

%         \item [$-$] For normalized $[[uN]]$ and $[[iM]]$ such that
%         $[[Γ ; dom(Θ) ⊢ uN]]$ and $[[Γ ⊢ iM]]$,\\
%         if $[[Γ ; Θ ⊨ uN ≈u iM ⫤ UC]]$ then 
%         $[[Θ ⊢ UC : uv uN]]$ and for any normalized $[[uσ]]$ such that
%         $[[ Θ   ⊢ uσ : lift UC ]]$, $[[ [uσ]uN = iM ]]$.
%     \end{itemize}
% \end{lemma*}

% \begin{lemma*}[Completeness of Unification]
%     \label{lemma:unification-completeness}
%     \hfill
%     \begin{itemize}
%         \item [$+$] For normalized $[[uP]]$ and $[[iQ]]$ such that
%         $[[Γ ; dom(Θ) ⊢ uP]]$ and $[[Γ ⊢ iQ]]$, 
%         suppose that there exists $[[Θ ⊢ uσ : uv(uP)]]$ such that $[[ [uσ]uP = iQ ]]$,
%         then $[[Γ ; Θ ⊨ uP ≈u iQ ⫤ UC]]$ for some $[[UC]]$.
        
%         \item [$-$] For normalized $[[uN]]$ and $[[iM]]$ such that
%         $[[Γ ; dom(Θ) ⊢  uN]]$ and $[[Γ ⊢ iM]]$,
%         suppose that there exists $[[Θ ⊢ uσ : uv(uN)]]$ such that $[[ [uσ]uN = iM ]]$,
%         then $[[Γ ; Θ ⊨ uN ≈u iM ⫤ UC]]$ for some $[[UC]]$.
%    \end{itemize}
% \end{lemma*}

\subsection{Anti-Unification}

The anti-unifier of the two types is the most specific pattern that matches
against both of them. In our setting, the anti-unifiers are restricted further:
first, the pattern might only contain `holes' at \emph{negative} positions
(because eventually, the `holes' become variables abstracted by the
existential quantifier); second, the anti-unification is parametrized with a
context $[[Γ]]$, in which the pattern instantiations must be well-formed.

This way, the correctness properties of the anti-unification algorithm are
refined accordingly. The soundness of anti-unification not only ensures that the
resulting pattern matches with the input types, but also that the pattern
instantiations are well-formed in the corresponding context, and that all the
`hole' variables are negative. In turn, completeness states that if there
exists a solution satisfying the soundness criteria, then the algorithm
succeeds.


The correctness properties are formulated by the following lemmas. For brevity,
we only provide the statements for the positive case, since the negative case is
symmetric. 


\begin{lemma*}[Soundness of (positive) anti-unification]
    \label{lemma:au-soundness}
     Assuming $[[iP1]]$ and $[[iP2]]$ are normalized,
    if $[[Γ ⊨ iP1 ≈au iP2 ⫤ (Ξ, uQ, aus1, aus2)]]$
    then 
    \begin{enumerate*}
        \item $[[Γ ; Ξ ⊢ uQ]]$,
        \item $[[Γ ; · ⊢ ausi : Ξ]]$ for $i \in \{1,2\}$
        are anti-unification substitutions (in particular, $[[Ξ]]$ contains only negative algorithmic variables), and
        \item $[[ [ausi] uQ = iPi ]]$ for $i \in \{1,2\}$.
    \end{enumerate*}
\end{lemma*}

\begin{lemma*}[Completeness of (positive) anti-unification]
    \label{lemma:au-completeness}
    Assuming that $[[iP1]]$ and $[[iP2]]$ are normalized terms well-formed in $[[Γ]]$
    and there exist  $[[(Ξ', uQ', aus'1, aus'2)]]$ such that
    \begin{enumerate*}
        \item $[[Γ ; Ξ' ⊢ uQ']]$,
        \item $[[Γ ; · ⊢ aus'i : Ξ']]$ for $i \in \{1,2\}$ 
        are anti-unification substitutions, and
        \item $[[ [aus'i] uQ' = iPi ]]$ for $i \in \{1,2\}$.
    \end{enumerate*}

    Then the anti-unification algorithm terminates, that is there exists
    $[[(Ξ, uQ, aus1, aus2)]]$ such that $[[Γ ⊨ iP1 ≈au iP2 ⫤ (Ξ, uQ, aus1, aus2)]]$.
\end{lemma*}

Notice that for the anti-unification substitution $[[aus]]$ we use a separate
signature specifying the domain and the range. $[[Γ ; Ξ2 ⊢ aus : Ξ1]]$ means
that $[[aus]]$ maps the `holes' (\ie algorithmic variables) from $[[Ξ1]]$
to \emph{algorithmic} types well-formed in $[[Γ]]$ and $[[Ξ2]]$. To put it
formally, $[[Γ ; Ξ2 ⊢ [aus]α̂⁻]]$ for any $[[α̂⁻ ∊ Ξ1]]$.
Although in the formulation of soundness and completeness, the resulting types are declarative
(\ie $[[Ξ2]]$ is always empty), we need the possibility of mapping the
`holes' to types with `holes' to formulate the \emph{initiality} of
anti-unification.

The initiality shows that the anti-unifier that the algorithm provides is the
most specific (or the most `detailed'). Specifically, it means that any other
sound anti-unification solution can be `refined' to the result of the algorithm.  
The `refinement' is represented as an instantiation of the anti-unifier---a 
substitution $[[Γ ; Ξ2 ⊢ ausr : Ξ1]]$ replacing the `holes' from $[[Ξ1]]$ with
types that themselves can contain `holes' from $[[Ξ2]]$.

\begin{lemma*}[Initiality of Anti-Unification]
        Assume that $[[iP1]]$ and $[[iP2]]$ are normalized types well-formed in $[[Γ]]$, 
        and the anti-unification algorithm succeeds: $[[Γ ⊨ iP1 ≈au iP2 ⫤ (Ξ, uQ, aus1, aus2)]]$. 
        Then $[[(Ξ, uQ, aus1, aus2)]]$ is more specific than
        any other sound anti-unifier $[[(Ξ', uQ', aus'1, aus'2)]]$, \ie if
        \begin{enumerate*}
            \item $[[Γ ; Ξ' ⊢ uQ']]$,
            \item $[[Γ ; · ⊢ aus'i : Ξ']]$ for $i \in \{1,2\}$, and
            \item $[[ [aus'i] uQ' = iPi ]]$ for $i \in \{1,2\}$
        \end{enumerate*}
        then there exists a `refining' substitution $[[ausr]]$ such that
        $[[Γ ; Ξ ⊢ ausr : (Ξ' | uv uQ')]]$ and $[[ [ausr] uQ' = uQ ]]$. 
        % Moreover, $[[ [ausr]β̂⁻]]$ is uniquely determined by $[[ [aus'1]β̂⁻ ]]$, 
        % $[[ [aus'2]β̂⁻ ]]$, and $[[Γ]]$.
\end{lemma*}

To prove the correctness properties of the anti-unification algorithm,
one extra observation is essential. The algorithm relies on the fact that
in the resulting tuple $[[(Ξ, uQ, aus1, aus2)]]$,
there are no two different `holes' $[[β̂⁻]]$
mapped to the same pair of types by $[[aus1]]$ and $[[aus2]]$.
This is used to ensure that, for example, 
the anti-unifier of $[[↓↑Int → ↑Int]]$ and 
$[[↓↑Bool → ↑Bool]]$ is $[[↓â⁻ → â⁻]]$, but not
(less specific) $[[↓â⁻ → b̂⁻]]$.
To preserve this property, we arrange the algorithm in such a way 
that the name of the `hole' is determined by the types it is mapped to.
The following lemma specifies this observation.

\begin{lemma*}[Uniqueness of Anti-unification Variable Names]
    Names of the anti-unification variables are uniquely defined by
    the types they are mapped to by the resulting substitutions. 
    Assuming $[[iP1]]$ and $[[iP2]]$ are normalized,
        if $[[Γ ⊨ iP1 ≈au iP2 ⫤ (Ξ, uQ, aus1, aus2)]]$
        then for any $[[β̂⁻]] \in [[Ξ]]$,
        $[[β̂⁻]] = [[â⁻_{[aus1]β̂⁻, [aus2]β̂⁻}]]$.
\end{lemma*}

\subsection{Least Upper Bound and Upgrade}
\label{sec:proof-lub-upgrade}

    The Least Upper Bound algorithm finds the least type that is a supertype of two given types.
    Its \emph{soundness} means that the returned type is indeed a supertype of the given ones;
    the \emph{completeness} means that the algorithm succeeds if the least upper bound exists;
    and the \emph{initiality} means that the returned type is the least among common supertypes. 

    \begin{lemma*}[Least Upper Bound soundness]
        For types $[[Γ ⊢ iP1]]$, and $[[Γ ⊢ iP2]]$,
        if $[[Γ ⊨ iP1 ∨ iP2 = iQ]]$ then
        \begin{enumerate}
            \item[(i)]  $[[Γ ⊢ iQ]]$
            \item[(ii)] $[[Γ ⊢ iQ ≥ iP1]]$ and $[[Γ ⊢ iQ ≥ iP2]]$
        \end{enumerate}
    \end{lemma*}


    \begin{lemma*}[Least Upper Bound completeness and initiality]
        For types $[[Γ ⊢ iP1]]$, $[[Γ ⊢ iP2]]$, and $[[Γ ⊢ iQ]]$
        such that $[[Γ ⊢ iQ ≥ iP1]]$ and $[[Γ ⊢ iQ ≥ iP2]]$,
        there exists $[[iQ']]$ s.t. $[[Γ ⊨ iP1 ∨ iP2 = iQ']]$ 
        and $[[Γ ⊢ iQ ≥ iQ']]$.
    \end{lemma*}

    The key observation that allows us to prove these properties is the
    characterization of positive supertypes. The following lemma justifies the
    case analysis used in the Least Upper Bound algorithm (in \cref{sec:lub}).
    In particular, it establishes the correspondence between the upper bounds of
    shifted types $[[↓iM]]$ and \emph{patterns} fitting $[[iM]]$ (represented by
    existential types), which explains the usage of anti-unification as a way to
    find a common pattern. 

    \begin{lemma*}[Characterization of normalized supertypes]
        \label{lemma:char-supertypes}
        For a normalized positive type $[[iP]]$ well-formed in $[[Γ]]$,
        the set of normalized $[[Γ]]$-formed supertypes of $[[iP]]$ is the following:
        \begin{itemize}
            \item if $[[iP]]$ is a variable $[[pb]]$, its only normalized supertype is $[[pb]]$ itself;
            \item if $[[iP]]$ is an existential type $[[ ∃nbs.iP' ]]$ then 
                its $[[Γ]]$-formed supertypes are the $([[G, nbs]])$-formed supertypes of $[[iP']]$ not using $[[nbs]]$;
            \item if $[[iP]]$ is a downshift type $[[↓iM]]$, 
                its supertypes have form $[[∃nas.↓iM']]$ such that there exists
                a $[[Γ]]$-formed instantiation of $[[nas]]$ in $[[↓iM']]$
                that makes $[[↓iM']]$ equal to $[[↓iM]]$, \ie $[[ [iNs/nas] ↓iM' = ↓iM ]]$.
        \end{itemize}
    \end{lemma*}

    Similarly to the Least Upper Bound algorithm, the Upgrade finds the least type among upper bounds
    (this time the set of considered upper bounds consists of supertypes well-formed in a \emph{smaller} context).
    This way, we also use the supertype characterization to prove the following properties of the Upgrade algorithm. 

    \begin{lemma*}[Upgrade soundness]
        Assuming $[[iP]]$ is well-formed in $[[Γ = Δ, pnas]]$,
        if $[[upgrade Γ ⊢ iP to Δ = iQ]]$
        then
        \begin{enumerate*}
            \item $[[Δ ⊢ iQ]]$ and
            \item $[[Γ ⊢ iQ ≥ iP]]$.
        \end{enumerate*}
    \end{lemma*}

    \begin{lemma*}[Upgrade Completeness]
        Assuming $[[iP]]$ is well-formed in $[[Γ = Δ, pnas]]$,
        for any $[[iQ']]$ such that $[[iQ']]$ is a $[[Δ]]$-formed upper bound of $[[iP]]$, \ie
        \begin{enumerate*}
            \item $[[Δ ⊢ iQ']]$ and
            \item $[[Γ ⊢ iQ' ≥ iP]]$,
        \end{enumerate*}
        there exists $[[iQ]]$ such that
        ($[[upgrade Γ ⊢ iP to Δ = iQ]]$) and $[[Δ ⊢ iQ' ≥ iQ]]$.
    \end{lemma*}

\subsection{Subtyping}
    \label{sec:proof-subtyping}

    As for other properties, the correctness of subtyping means that the
    algorithm produces a valid result (soundness) whenever it exists
    (completeness). In the rules defining the subtyping algorithm
    (\cref{fig:alg-subtyping}), one can see that the positive and negative
    subtyping relations are not \emph{mututally} recursive: negative subtyping
    algorithm uses the positive subtyping, but not vice versa. Because of that,
    the inductive proofs of the \emph{positive} subtyping correctness are done
    independently.

    The soundness of positive subtyping states that the output constraint
    $[[SC]]$ provides a sufficient set of restrictions. In other words, any
    substitution satisfying $[[SC]]$ ensures the desired declarative subtyping:
    $[[Γ ⊢ [uσ]uP ≥ iQ]]$. Notice that the soundness formulation assumes that
    only the left-hand side input type ($[[uP]]$) can contain algorithmic
    variables. This is one of the invariants of the algorithm that significantly
    simplifies the unification and constraint resolution.

\begin{lemma*}[Soundness of Positive Subtyping]
    If $[[Γ ⊢ Θ]]$ and $[[Γ ⊢ iQ]]$ and $[[Γ ; dom(Θ) ⊢  uP]]$ and 
    $[[Γ ; Θ ⊨ uP ≥ iQ ⫤ SC]]$,
    then $[[Θ ⊢ SC : uv uP]]$ and
    for any  $[[uσ]]$ such that $[[ Θ ⊢ uσ : SC ]]$,
    we have $[[ Γ ⊢ [uσ]uP ≥ iQ ]]$.
\end{lemma*}

    The completeness of subtyping says that if the substitution 
    ensuring the declarative subtyping exists, then the subtyping algorithm 
    succeeds (terminates).

\begin{lemma*}[Completeness of Positive Subtyping]
    Suppose that $[[Γ ⊢ Θ]]$, $[[Γ ⊢ iQ]]$ and $[[Γ ; dom(Θ) ⊢  uP]]$.
    Then if there exists $[[uσ]]$ such that $[[Θ ⊢ uσ : uv(uP)]]$ and $[[ Γ ⊢ [uσ]uP ≥ iQ ]]$,
    then the subtyping algorithm succeeds: $[[Γ; Θ ⊨ uP ≥ iQ ⫤ SC]]$.
\end{lemma*}

After the correctness properties of positive subtyping are established,
they are used to prove the correctness of the negative subtyping.
The soundness is formulated symmetrical to the positive case,
however, the completeness requires an additional invariant to be preserved.
The algorithmic input type $[[uN]]$ must be free from \emph{negative} algorithmic variables.
In particular, it ensures that the constraint restricting a \emph{negative} algorithmic
variable will never be generated, and thus, we do not need the 
Greatest Common Subtype procedure to resolve the constraints.

\begin{lemma*}[Completeness of Negative Subtyping]
    Suppose that $[[Γ ⊢ Θ]]$ and $[[Γ ⊢ iM]]$ and $[[Γ ; dom(Θ) ⊢ uN]]$
    and $[[uN]]$ does not contain negative unification variables ($[[α̂⁻]] \notin [[uv uN]]$).
    Then for any $[[Θ ⊢ uσ : uv(uN)]]$ such that $[[Γ ⊢ [uσ]uN ≤ iM]]$,
    the subtyping algorithm succeeds: $[[Γ ; Θ ⊨ uN ≤ iM ⫤ SC]]$.
\end{lemma*}

\subsection{Minimal Instantiation and Singularity}
\label{sec:singularity-proof}

Algorithmic typing relies on the \emph{minimal instantiation} procedure.
For a given positive type $[[uP]]$ and a set of constraints $[[SC]]$
it returns a substitution $[[Θ ⊢ uσ : SC]]$ such that
$[[ [uσ]uP ]]$ is a subtype of all the other instantiations of $[[uP]]$ satisfying $[[SC]]$.
This way, the correctness of minimal instantiation is formulated as the following two lemmas.

\begin{lemma*}[Soundness of Minimal Instantiation]
    \label{lemma:minimal-instantiation-soundness}
    Suppose that $[[Γ ⊢ Θ]]$, $[[Θ ⊢ SC]]$, and $[[Γ; dom(Θ) ⊢ uP]]$.
    Then `$[[Γ ⊢ uP SC minby uσ ]]$' implies that 
    $[[Θ ⊢ uσ : uv uP]]$ is a normalized substitution satisfying $[[SC]]$
    (\ie $[[Θ ⊢ uσ : SC]]$)
    and for any other substitution $[[Θ ⊢ uσ' : uv uP ]]$ satisfying $[[SC]]$,
    we have $[[Γ ⊢ [uσ']uP ≥ [uσ]uP ]]$.
\end{lemma*}

\begin{lemma*}[Completeness of Minimal Instantiation]
    Suppose that $[[Γ ⊢ Θ]]$, $[[Θ ⊢ SC]]$, $[[Γ; dom(Θ) ⊢ uP]]$, and there exists 
    $[[Θ ⊢ uσ : uv uP ]]$ satisfying $[[SC]]$ ($[[Θ ⊢ uσ : SC]]$) 
    such that for any other $[[Θ ⊢ uσ' : uv uP ]]$
    satisfying $[[SC]]$, we have
    $[[Γ ⊢ [uσ']uP ≥ [uσ]uP ]]$.  Then the minimal instantiation procedure  
    succeeds; specifically, $[[Γ ⊢ uP SC minby nf(uσ) ]]$ holds.
\end{lemma*}

The soundness is rather straightforwardly proved by induction on the 
judgment inference tree, relying on the soundness of the singularity procedure.
The proof of completeness is done by induction on the structure of the type $[[uP]]$.
It uses the fact that $[[ Γ ⊢ ∃nas.[uσ']uP ≥ ∃nas.[uσ]uP]]$ implies
$[[ Γ, nas ⊢ [uσ']uP ≥ [uσ]uP]]$.

The soundness of \emph{singularity} states that $[[SC singular with uσ]]$
implies that any substitution satisfying $[[SC]]$ is equivalent to $[[uσ]]$ 
on the domain.  The completeness states that if all the 
$[[SC]]$-compliant substitutions are equivalent, then the singularity procedure succeeds. 

\begin{lemma*}[Soundness of Singularity]
    \label{lemma:singularity-soundness}
    Suppose $[[Θ ⊢ SC : Ξ]]$, and $[[SC singular with uσ]]$. 
    Then $[[ Θ ⊢ uσ : Ξ ]]$,
     $[[ Θ ⊢ uσ : SC ]]$ and for any 
    $[[uσ']]$ such that $[[Θ ⊢ uσ : SC]]$,
    $[[Θ ⊢ uσ' ≈ uσ : Ξ]]$.
\end{lemma*}

\begin{lemma*}[Completeness of Singularity]
    \label{lemma:singularity-completeness}
    For a given set of constraints $[[Θ ⊢ SC]]$,
    suppose that all the substitutions satisfying $[[SC]]$ are equivalent
    on $[[dom(SC)]]$.
    In other words, suppose that there exists $[[Θ ⊢ uσ1 : dom(SC)]]$ such that
    for any $[[Θ ⊢ uσ : dom(SC)]]$, $[[Θ ⊢ uσ : SC]]$ implies 
    $[[Θ ⊢ uσ ≈ uσ1 : dom(SC)]]$.
    Then $[[SC singular with uσ0]]$ for some $[[uσ0]]$.
\end{lemma*}

\subsection{Typing}

Finally, we discuss the proofs of the soundness and completeness of type
inference algorithm that we stated at the beginning of this section.
There are three subtleties that we will cover that are important 
for the proof to go through: the determinacy of the algorithm,
the mutual dependence of the soundness and completeness proofs, and
the non-trivial inductive metric that we use.


\paragraph*{Determinacy}
    One of the properties that our proof relies on is the determinacy of the 
    typing algorithm: the output (the inferred type)
    is uniquely determined by the input (the term and the contexts).
    Determinacy is not hard to demonstrate by structural induction: 
    in every algorithmic inference system, only one inference rule can be applied 
    for a given input.  However, we need to prove it for \emph{every} 
    subroutine of the algorithm. Ultimately, it requires 
    the determinacy of such procedures as \emph{generation of fresh variables}, 
    which is easy to ensure, but must be taken into account in the implementation.

\begin{lemma*}[Determinacy of the Typing Algorithm]
    Suppose that $[[Γ ⊢ Φ]]$ and $[[Γ ⊢ Θ]]$. Then 
    \begin{itemize}
        \item [$+$] If $[[Γ; Φ ⊨ v : iP]]$ and $[[Γ; Φ ⊨ v : iP']]$ then $[[iP = iP']]$.
        \item [$-$] If $[[Γ; Φ ⊨ c : iN]]$ and $[[Γ; Φ ⊨ c : iN']]$ then $[[iN = iN']]$.
        \item If $[[Γ; Φ; Θ ⊨ uN ● args ⇒> uM ⫤ Θ'; SC]]$ and 
            $[[Γ; Φ; Θ ⊨ uN ● args ⇒> uM' ⫤ Θ'; SC']]$ then 
            $[[uM = uM']]$, $[[Θ]] = [[Θ']]$, and $[[SC]] =[[SC']]$.  
    \end{itemize}
\end{lemma*}

\paragraph*{Mutuality of the Soundness and Completeness Proofs}

Typically in our inductive proofs, the soundness is proven before completeness,
as the completeness requires the premise subtrees to satisfy certain properties
(which can be given by the soundness). However, in the case of the typing algorithm,
the soundness and completeness proofs cannot be separated: the inductive proof
of one requires another, and vice versa. 

The proof of soundness can be viewed as a mapping from an algorithmic tree to a
declarative one. We show that each algorithmic inference rule can be transformed
into the corresponding declarative rule, as long as the premises are transformed
accordingly, and apply the induction principle. 

The soundness requires completeness in the case of
\ruleref{\ottdruleATAppLetLabel}. To prove the soundness, in other words, to
transform this rule into its declarative counterpart
\ruleref{\ottdruleDTAppLetLabel}, one needs to prove the \emph{principality} of
the inferred application type $[[↑[uσ]uQ]]$. In other words, we need to conclude
that any other \emph{declarative} tree infers for the application a supertype of
$[[ [uσ]uQ ]]$. 

The only way to do so is by applying the soundness of minimal instantiation
(see the lemma above). However, first, the declarative
tree must be converted to the corresponding algorithmic one (by completeness!).
This way, the soundness and completeness proofs are mutually dependent.

\paragraph*{Inductive Metric}
    The soundness and completeness lemmas are proven by \emph{mutual} induction. 
    Since the declarative and the algorithmic systems do not depend on each other,
    we must introduce a uniform \emph{metric} on which the induction is conducted.
    We define the metric gradually, starting from the
    auxiliary function---the size of a judgment $\size{J}$.

    The size of \emph{declarative} and \emph{algorithmic} judgments is defined as
    a pair: the first component is the syntactic size of the terms used in the judgment. 
    The second component depends on the kind of judgment. For regular
    type inference judgments (such as $[[Γ ; Φ ⊢ v : iP]]$
    or $[[Γ ; Φ ⊨ c : iN]]$), it is always zero. For application inference judgments
    $([[Γ ; Φ ⊢ iN ● args ⇒> iM]]$ or $[[Γ ; Φ ; Θ ⊨ uN ● args ⇒> uM ⫤ Θ'; SC]]$), 
    it is equal to the number of prenex quantifiers of the head type $[[iN]]$.
    We need this adjustment to ensure the monotonicity of the metric, since the rules 
    \ruleref{\ottdruleATForallAppLabel} and \ruleref{\ottdruleDTForallAppLabel}
    only reduce the quantifiers in the head type but they do not change the list of arguments.

\begin{definition*}[Judgement Size]
    \label{def:decl-typing-size}
    For a declarative or an algorithmic typing judgment $J$, we define a metric $\size{J}$ as a pair of integers in the following way:

    \begin{multicols}{2}
    \begin{itemize}
        \item [$+$] $\size{[[Γ ; Φ ⊢ v : iP]]} = (\size{[[v]]}, 0)$;
        \item [$-$] $\size{[[Γ ; Φ ⊢ c : iN]]} = (\size{[[c]]}, 0)$;
        \item [$\bullet$] $\size{[[Γ ; Φ ⊢ iN ● args ⇒> iM]]}=$\\ 
            $(\size{[[args]]}, \npq{[[iN]]})$;
    \end{itemize}
    \columnbreak
    \begin{itemize}[leftmargin=*]
        \item [$+$] $\size{[[Γ ; Φ ⊨ v : iP]]} = (\size{[[v]]}, 0)$;
        \item [$-$] $\size{[[Γ ; Φ ⊨ c : iN]]} = (\size{[[c]]}, 0)$;
        \item [$\bullet$] $\size{[[Γ ; Φ ; Θ ⊨ uN ● args ⇒> uM ⫤ Θ'; SC]]} =$\\ 
             $(\size{[[args]]}, \npq{[[uN]]})$.
    \end{itemize}
    \end{multicols}

    Here $\size{[[v]]}$ and $\size{[[c]]}$ is the size of the 
    syntax tree of the term,
    and $\size{[[args]]}$ is the sum of sizes of the terms in $[[args]]$;
    and $\npq{[[iN]]}$ and $\npq{[[iP]]}$ represent the number of 
    prenex quantifiers, \ie
        \begin{itemize} \centering
            \item [$+$] $\npq{[[∃nas.iP]]} = |[[nas]]|$, if $[[iP ≠ ∃nbs.iP']]$,
            \item [$-$] $\npq{[[∀pas.iN]]} = |[[pas]]|$, if $[[iN ≠ ∀pbs.iN']]$.
        \end{itemize}
\end{definition*}

Notice that for \emph{algorithmic} inference system, $\size{J}$ decreases in
all the inductive steps, \ie for each inference rule, the size of the
premise judgments is strictly less than the size of the conclusion.
However, the \emph{declarative} inference system has rules
\ruleref{\ottdruleDTNEquivLabel} and \ruleref{\ottdruleDTPEquivLabel}, that
`step to' an equivalent type, and thus, technically, might keep the judgment
unchanged altogether.

To deal with this issue, we introduce the metric on the entire \emph{inference
trees} rather than on judgments, and plug into this metric the parameter that
certainly decreases in rules \ruleref{\ottdruleDTNEquivLabel} and
\ruleref{\ottdruleDTPEquivLabel}---the number of such nodes in the inference
tree. We denote this number as $\eqNodes{T}$. Then the final metric is defined
as a pair in the following way.

\begin{definition*}[Inference Tree Metric]
    For a tree $T$, inferring a declarative or an algorithmic judgement $J$, we define $\metric{T}$ as follows:
        \[
        \metric{T} = \begin{cases}
        (\size{J}, \eqNodes{T}) & \text{if } J \text{ represents a declarative judgement}, \\
        (\size{J}, 0) & \text{if } J \text{ represents an algorithmic judgement}.
        \end{cases}
        \]
    Here $\eqNodes{T}$ is the number of nodes in $T$ labeled with \ruleref{\ottdruleDTPEquivLabel} or \ruleref{\ottdruleDTNEquivLabel}.
\end{definition*}

This metric is suitable for mutual induction on the soundness and completeness
of the typing algorithm.  First, it is monotonous \wrt the inference rules,
and this way, we can always apply the induction hypothesis to premises 
of each rule. Second, the induction hypothesis is powerful enough,
so we can use the completeness of the algorithm in the soundness proof,
where required. For instance, to prove the soundness
of typing in case of $[[Γ; Φ ⊨ let x = v(args); c' : iN]]$,
we can assume, that  \emph{completeness}
holds for a term of shape $[[Γ; Φ ⊢ iM ● args ⇒> iK ]]$, since
$\size{args} < \size{let x = v(args); c'}$.
This is exactly what allows us to 
deal with the case of \ruleref{\ottdruleATAppLetLabel},
because then we can conclude that 
the inferred type (of a declarative judgment $[[Γ; Φ ⊢ iM ● args ⇒> ↑[uσ]uQ
]]$ constructed by the induction hypothesis) is unique.



\section{Extensions and Future Work}
\label{sec:extensions}

In this section, we discuss several extensions to the system and the algorithm.
Some of them can be incorporated into the system immediately, while others
are beyond the scope of this work, and thus are left for future research.
% In particular, \emph{Explicit Type Application} 

\subsection{Explicit Type Application}
\label{sec:explicit-type-application}

In our system, all type applications are inferred implicitly: the algorithm
\emph{Explicit} type application can be added to the declarative system by the following rule:

$$\ottdruleDTTypeAppLabeled{}$$

However, this rule alone would cause ambiguity. The declarative system does not
fix the order of the quantifiers, which means that $[[∀α⁺.∀β⁺.iN]]$ and
$[[∀β⁺.∀α⁺.iN]]$ can be inferred as a type of $[[c]]$ interchangeably. But that
would make explicit instantiation $[[ c[iP] ]]$ ambiguous, as it is unclear
whether $[[α⁺]]$ or $[[β⁺]]$ should be instantiated with $[[iP]]$.

\vspace{\baselineskip}
\paragraph*{Solution 1: declarative normalization}
One way to resolve this ambiguity is to fix the order of quantifiers.  
The algorithm already performs the ordering of the quantifiers in the normalization procedure.
This way, we could require the inferred type to be normalized
to specify the order of the quantifiers:
$$\ottdruleDTTypeAppOrderedLabeled{}$$

The drawback of this approach is that it adds another point where the internal
algorithmic concept---normalization---is exposed to the `surface' declarative
system.

\vspace{\baselineskip}
\paragraph*{Solution 2: elementary type inference}
An alternative approach to explicit type application
 was proposed by \cite{zhao22:elementary}.
In this work, the subtyping relation is restricted 
in such a way that 
$[[∀α⁺.∀β⁺.iN]]$ and $[[∀β⁺.∀α⁺.iN]]$ are \emph{not} mutual subtypes 
(as long as $[[α⁺ ∊ fv(iN)]]$ and $[[β⁺ ∊ fv(iN)]]$).
It implies that the order of the quantifiers of the inferred type is unique,
and thus, the explicit type application is unambiguous.

These restrictions can be incorporated into our system by
replacing the polymorphic subtyping rules
\ruleref{\ottdruleDOneForallLabel} and \ruleref{\ottdruleDOneExistsLabel}
with the following stronger versions:
% \vspace{-\baselineskip}
% \begin{multicols}{2}
% $$ \ottdruleDOneEForallR{} $$\\
% $$ \ottdruleDOneEForallL{} $$
% \end{multicols}

% \begin{multicols}{2}
% $$ \ottdruleDOneEExistsR{} $$\\
% $$ \ottdruleDOneEExistsL{} $$
% \end{multicols}
% \vspace{\baselineskip}

% \vspace{-\baselineskip}

\begin{minipage}{0.5\textwidth}
$$ \ottdruleDOneEForallLR{} $$
\end{minipage}%
\begin{minipage}{0.5\textwidth}
$$ \ottdruleDOneEExistsLR{} $$
\end{minipage}
\vskip 0.5em

According to these rules, if two quantified ($[[∀]]$ or $[[∃]]$) types are
subtypes of each other, they must have the same top-level quantifiers. Moreover,
the equivalence on types (mutual subtyping) degenerates to \emph{equality} up to
alpha-conversion. 

To accommodate these changes in the \emph{algorithm}, 
it suffices to 
    (i) replace the normalization procedure with identity: $[[nf(iN)]] \defeq [[iN]]$, $[[nf(iP)]] \defeq [[iP]]$,
        since the new equivalence classes are singletons;
    (ii) modify the least upper bound polymorphic rule \ruleref{\ottdruleLUBExistsLabel}
        so that it requires the quantifiers to be equal (and performs alpha-conversion if necessary):
        $$\ottdruleLUBEExists{}$$ 
    (iii) replace the subtyping polymorphic rule \ruleref{\ottdruleAForallLabel} 
    by the following rule:
        $$\ottdruleAForallLR{}$$
    and
    (iv) update the existential rule \ruleref{\ottdruleAExistsLabel} symmetrically.

After these changes, the rule \ruleref{\ottdruleDTTypeAppLabel} and its algorithmic counterpart
can be used to infer the the type of $[[ c[iP] ]]$.

Elementary type inference
is much more restrictive than what we present.
As mentioned, it forbids the quantifier reordering; besides, 
as soon as the right-hand side of the subtyping relation is
polymorphic, it restricts the instantiation of the left-hand side quantifiers, 
which \emph{disallows}, for example, $[[· ⊢ ∀α⁺.↑α⁺ ≤ ∀α⁺.↑↓↑α⁺]]$.
On the other hand, the elementary subtyping system
can be smoothly extended with bounded quantification, which we discuss later.

\vspace{\baselineskip}
\paragraph*{Solution 3: labeled quantifiers}

A compromise solution that resolves the ambiguity of explicit instantiation
without fixing the order of quantifiers is by using \emph{labeled} quantifiers.
Let us discuss how to introduce them to the \emph{negative} part of the system
since the positive subsystem is symmetric. 


Each type abstraction $[[Λl▷α⁺.c]]$ must be annotated with a label $[[l]]$.
This label propagates to the inferred polymorphic type. This way, each
polymorphic quantifier $[[ ∀ (li ▷* αi⁺)^I . iN ]]$ is annotated with a label
$[[li]]$ whose index is taken from the \emph{list} of labels $[[I]]$ associated
with the group of quantifiers. To instantiate a quantifying variable, one refers
to it by its label: $[[ c[lj ▷ iP] ]]$, which is expressed by the following
typing rules:

\begin{minipage}{0.4\textwidth}
$$\ottdruleDTTLamLbLabeled{}$$
\end{minipage}%
\begin{minipage}{0.6\textwidth}
$$\ottdruleDTTypeAppLbLabeled{}$$
\end{minipage}
\vskip 0.5em


The subtyping rule permits an arbitrary order of the quantifiers.
However, similarly to the elementary subtyping, 
each of the right-hand side quantifiers must have 
a matching left-hand side quantifier with the same label. These
matching quantifiers are synchronously removed
(in other words, each variable is abstractly instantiated to itself), 
and the remaining quantifiers are instantiated as usual by
a substitution $[[σ]]$.
$$ \ottdruleDOneLblForallLRLabeled{} $$

The equivalence (mutual subtyping) in this system 
allows the quantifiers within the same group to be reordered
together with their labels. For instance,
$[[∀l▷α⁺.∀m▷β⁺.iN]]$ and $[[∀m▷β⁺.∀l▷α⁺.iN]]$ are subtypes of each other.
However, the right-hand side quantifiers are removed synchronously with 
the matching left-hand side quantifiers, which means that
$[[· ⊢ ∀l▷α⁺.↑α⁺ ≤ ∀l▷α⁺.↑↓↑α⁺]]$ is still not allowed.

\subsection{Weakening of the subtyping invariants}
\label{sec:weakening-invariant}

In order to make the subtyping relation decidable,
we made the subtyping of shifts \emph{invariant}, which manifests
in the following rules:

\begin{minipage}{0.5\textwidth}
    $$ \ottdruleDOneShiftU{} $$
\end{minipage}%
\begin{minipage}{0.5\textwidth}
    $$ \ottdruleDOneShiftD{} $$
\end{minipage}
\vskip 0.5em

To make the system more expressive, we can relax these rules. 
Although making both rules covariant would make the system undecidable,
it is possible to make the up-shift subtyping covariant while keeping the down-shift
invariant: 
$$ \ottdruleDOneShiftURlxLabeled{} $$\\
However, this change requires an update to the algorithm.
As one can expect,  
the corresponding algorithmic rule \ruleref{\ottdruleAShiftULabel}
must be changed accordingly:
$$ \ottdruleAShiftURlxLabeled{} $$
Also, notice that now the algorithmic variables can occur on \emph{both}
sides of the positive subtyping relation, and thus, a more sophisticated constraint 
solver is needed. To see what kind of constraints will be generated,
let us keep the other algorithmic rules unchanged for a moment
and make several observations.

First, notice that the negative quantification variables still only occur at
\emph{invariant} positions: the negative variables are generated from the existential
quantifiers $[[∃α⁻.iP]]$, and the switch from a positive to a negative type in the syntax path
to $[[α⁻]]$ in $[[iP]]$ is `guarded' by \emph{invariant} down-shift. It means
that in the algorithm, the negative constraint entries can only be equivalence
entries ($[[nua :≈ iN]]$) but not subtyping entries. 

Second, notice that the positive subtyping is still not dependent on the negative
subtyping. It means that any leaf-to-root path in the inference tree is monotone:
in the first phase, the negative subtyping rules are applied, and in the second phase, 
only the positive ones.

The invariant that only the left-hand side of the judgment is algorithmic is not
preserved. However, the only rule violating this invariant is
\ruleref{\ottdruleDOneShiftURlxLabel}, and it is placed at the interface between
negative subtyping and positive subtyping. It means that the \emph{negative}
algorithmic variables (produced by $[[∃]]$ in the positive phase of the
inference path) still only occur on the left-hand side of the judgment, and the
positive algorithmic variables (produced by $[[∀]]$ in the first phase) can
occur either on the left-hand side or on the right-hand side (if the switch from
the negative to the positive phase is made by \ruleref{\ottdruleDOneShiftURlxLabel})
but not both.
This way, the produced entries will be of the following form:
\begin{itemize*}
    \item[(i)] $[[pua :≈ uP]]$, 
    \item[(ii)] $[[nua :≈ uN]]$, 
    \item[(iii)] $[[pua :≥ iP]]$, and
    \item[(iv)] $[[pua :≤ uP]]$ where $[[uP]]$ does not contain positive unification variables.
\end{itemize*}

We do not present the details of the constraint resolution procedure or how it
is integrated into the subtyping algorithm, leaving it for future work. However,
we believe that the system consisting of the mentioned kinds of constraints is solvable
for the following reasons.
\begin{enumerate*}
    \item[(i)] The equivalence entries $[[pua :≈ uP]]$ and $[[nua :≈ uN]]$ are resolved by standard first-order pattern unification techniques.
    \item[(ii)] The resolution of supertyping entries ($[[pua :≥ iP]]$) is discussed in the original algorithm (\cref{sec:constraint-merge}).
    \item[(iii)] The new kind of entries $[[pua :≤ uP]]$ can be
        reduced by case analysis. If $[[uP]]$ is a variable $[[β⁺]]$
        or a shifted computation $[[↓uN]]$ then 
        $[[pua :≤ uP]]$ is equivalent to $[[pua :≈ uP]]$;
        otherwise, $[[pua :≤ ∃nas.↓uN]]$ is equivalently replaced by $[[pua :≈ [nuas/nas]↓uN]]$
        for freshly created $[[nuas]]$.
\end{enumerate*}

This way, the algorithm can be extended to support the \emph{covariant} up-shift
subtyping. This, for example, enriches the relation with such subtypes as $[[· ⊢
∀α⁺.α⁺→↑α⁺ ≤ ↓↑Int → ↑∃β⁻.↓β⁻]]$, which does not hold in the original system.
Moreover, this extension also increases the expressiveness of the \emph{type
inference}. Namely, when inferring a type of an annotated let binding $[[let
x:iP = v(args); c]]$, we require $[[↑iP]]$ to be a supertype of $[[↑iQ]]$---the
resulting type of the application $[[v]]([[args]])$. In the relaxed system, it
can be replaced with the requirement that $[[iP]]$ is a supertype of $[[iQ]]$
(without the up-shift), which is more permissive.

In addition, the refinement of the unification algorithm described above
makes possible another improvement to the system---bounded quantification.

\subsection{Bounded Quantification}

After the weakening of subtyping invariants, 
it is possible to smoothly extend the \emph{elementary}
version of the system 
(the second solution discussed in \cref{sec:explicit-type-application})
 with bounded quantification.
In particular, we can add upper bounds to polymorphic $[[∀]]$-quantifiers:
$[[∀(pa ≤* iP). iN]]$. 

The declarative subtyping rules for bounded quantification are as expected:
the instantiation of quantified variables must satisfy the corresponding bounds
($[[Γ, pas ⊢ iQi ≥ [σ]βi⁺]]$ holds for each $i$);
and the right-hand side quantifiers must be more restrictive than the matching left-hand side
quantifiers ($[[Γ ⊢ iPi ≥ iP'i]]$ for each $i$):
$$\ottdruleDOneEForallLRUbLabeled{}$$

The corresponding algorithmic rule, as before,
replaces the polymorphic variables $[[β1]]\dots$ with fresh 
algorithmic variables $[[β1̂⁺]]\dots$, and merges the generated
constraints with the constraints given in the bounds
$[[βî⁺ :≤ uQi]]$:
$$\ottdruleAEForallLRUbLabeled{}$$

Notice that the algorithmic rule \ruleref{\ottdruleAEForallLRUbLabel}
requires the bounding types $[[iPs]]$ and $[[iQs]]$ to be \emph{declarative}.
Because of that, the generated constraints have shape $[[βî⁺ :≤ iQi]]$,
\ie the bounding types do not contain algorithmic variables.
This kind of constraints is covered by the resolution procedure described in
\cref{sec:weakening-invariant}, and thus, \ruleref{\ottdruleAEForallLRUbLabel} fits 
into the algorithmic system without changes to the constraint solver.

However, in order for this rule to be complete, 
the declarative system must be restricted as well.
To guarantee that no algorithmic variable is introduced in the bounding types,
we need to forbid the quantifying variables to occur in the bounding types in the declarative system.
For that purpose, we must include this requirement in the type well-formedness
(here $[[bv iN]]$ denotes the set of variables freely occurring 
in the quantifiers of $[[iN]]$ at any depth):
$$\ottdruleWFTForallUbLabeled{}$$

For brevity, we do not discuss in detail the combination of bounded and
unbounded quantification in one system. We believe this update is straightforward:
the unbounded quantifiers are treated as the weakest constraints, which are trivially
satisfied. Analogously, it is possible to extend the system with \emph{lower}
bound $[[∀]]$-quantifiers: $[[∀(pa ≥* iP). iN]]$. 
Then the generated constraints will change to $[[βî⁺ :≥ iQi]]$, which 
is also covered by the resolution procedure.
However, we do not consider bounded \emph{existential} quantification,
because in this case, the constraint resolution would require us to 
find the \emph{greatest lower bound} of two negative types, which is not
well-defined (see \cref{sec:constraint-merge}).

\subsection{Bidirectionalization}

% \subsubsection{Declarative System}
% \ottdefnDTNSynth{}
% \ottdefnDTNCheck{}
% \ottdefnDTNCheckSynth{}
% \ottdefnDTPSynth{}
% \ottdefnDTPCheck{}
% \ottdefnDTPCheckSynth{}
% \ottdefnDTSpinInfBidir{}

% \subsubsection{Algorithmic System}
% \ottdefnATNSynth{}
% \ottdefnATNCheck{}
% \ottdefnATNCheckSynth{}
% \ottdefnATPSynth{}
% \ottdefnATPCheck{}
% \ottdefnATPCheckSynth{}
% \ottdefnATSpinInfBidir{}




The algorithm we provide requires that all lambda functions are annotated. 
This restriction significantly simplifies the type inference by making the 
terms uniquely define their types. However, it leads to certain redundancies
in typing, in particular, the type of a lambda expression cannot be inferred 
from the context it is used in: the annotated $[[((λ x . return x) : (Int → ↑Int))]]$
does not infer $[[Int → ↑Int]]$.

The well-known way to incorporate this expressiveness into the system
is to make the typing bidirectional \cite{dunfield2020:bidirectional}.
The idea is to split the typing judgment into two kinds:
\begin{enumerate}
    \item[(i)] \emph{Synthesis} 
        judgments are used when the type of a term can be \emph{inferred}
        based exclusively on the term itself. 
        Syntactically, we denote synthesizing judgments
        as $[[Γ ; Φ ⊢ c ⇒ iN]]$.
    \item[(ii)] \emph{Checking} judgments 
        assume that the type of a term is \emph{given}
        from the context, and 
        it is required to \emph{check} that the given
        type can be assigned to the term.
        In checking judgments $[[Γ ; Φ ⊢ c ⇐ iN]]$, 
        the type $[[iN]]$ is considered as an \emph{input}.
\end{enumerate}

To bidirectionalize the system, each typing rule 
must be oriented either to synthesis or to checking (or both).
In particular, the \emph{annotated} lambda-abstractions are
synthesizing, and the \emph{unannotated} ones are checking.

\begin{minipage}{0.5\textwidth}
    $$ \ottdruleDTtLamSynLabeled{} $$
\end{minipage}
\hfill
\begin{minipage}{0.5\textwidth}
    $$ \ottdruleDTtLamChkLabeled{} $$
\end{minipage}
\vskip 0.5em


As common for bidirectional systems with subtyping, 
we would also need to introduce the \emph{subsumption} rules.
They allow us to `forget' the information about the type
by switching to the synthesis mode, as long as the synthesized
type is more polymorphic than the checked one:

\begin{minipage}{0.5\textwidth}
    $$ \ottdruleDTNSubLabeled{} $$
\end{minipage}
\hfill
\begin{minipage}{0.5\textwidth}
    $$ \ottdruleDTPSubLabeled{} $$
\end{minipage}
\vskip 0.5em

Most of the original rules will have two bidirectional counterparts,
one for each mode. The premises of the rules are oriented with respect to the conclusion.
For instance, we need both checking and inferring versions for $[[return v]]$: 
    $$ \ottdruleDTReturnChkSynLabeled{} $$
For some rules, however, it only makes sense to have one mode.
As mentioned above, the unannotated lambda-abstraction is always checking. 
On the other hand, an \emph{annotated} lambda, type-level lambda abstraction, 
or a variable inference is always synthesizing, since
the required type information is already given in the term:

\begin{minipage} {0.5\textwidth}
    $$ \ottdruleDTTLamSynLabeled{} $$
\end{minipage}
\hfill
\begin{minipage} {0.5\textwidth}
    $$ \ottdruleDTVarSynLabeled{} $$
\end{minipage}
\vskip 0.5em

Finally, let us discuss the application inference part of the declarative system. 
Rules \ruleref{\ottdruleDTEmptyAppLabel} and \ruleref{\ottdruleDTForallAppLabel} are 
unchanged: they simply do not have typing premises to be oriented.
However, the arrow application rule \ruleref{\ottdruleDTArrowAppLabel}
infers the result of the application of an arrow $[[iQ → iN]]$ 
to a list of arguments $[[v , args]]$, 
and it must make sure that the first argument $[[v]]$ is typeable with the expected $[[iQ]]$.
As we will discuss further, simply checking $[[Γ; Φ ⊢ v ⇐ iQ]]$ turns out to be too 
powerful and potentially leads to undecidability. 
Instead, this check is approximated by the combination of $[[Γ; Φ ⊢ v ⇒ iP]]$
and $[[Γ ⊢ iQ ≥ iP]]$.

\vspace{\baselineskip}
\paragraph*{The algorithm}
    The algorithmic system is bidirectionalized in a similar way. 
    Each typing premise and the conclusion of the rule are oriented
    to either `checking` ($[[⇐]]$) or `synthesizing` ($[[⇒]]$) mode,
    with respect to the declarative system. 
 
    For brevity, we omit the details of the bidirectional algorithm.
    Let us discuss one especially important rule---arrow application inference. 
    An intuitive way to bidirectionalize this rule is the following:
    $$ \ottdruleATArrowAppBidirLabeled{} $$
    % \paragraph*{Reaching Undecidability}
    However, this rule is too permissive. 
    The judgement $[[Γ ; Φ ⊨ v ⇐ uQ ⫤ SC1]]$
    requires us to check a term against an \emph{algorithmic} 
    type $[[uQ]]$. Further, through the
    lambda function checking \ruleref{\ottdruleDTtLamChkLabel}
    the algorithmic types infiltrate the type context
    and then through variable inference and subsumption, 
    \emph{both} sides of subtyping. 
    This way, \ruleref{\ottdruleATArrowAppBidirLabel}
    compromises the important invariant
    of the subtyping algorithm: 
    now the same algorithmic variable can occur on \emph{both}
     sides of the subtyping relation.

    We believe that the relaxation of this invariant brings the 
    constraint resolution too close to second-order pattern 
    unification, which is undecidable \cite{goldfarb81:undecidability}.
    In particular, the constraint $[[(α̂⁺ :≥ ↓↑α̂⁺)]]$ is solvable with 
    $[[α̂⁺]] = [[∃β⁻.↓β⁻]]$, 
    since by \ruleref{\ottdruleDOneExistsLabel}, 
    the existential $[[β⁻]]$ can be impredicatively instantiated to $[[↑∃β⁻.↓β⁻]]$. 

    Constraints such as $[[(α̂⁺ :≥ ↓↑α̂⁺)]]$ are not merely hypothetical. 
    It occurs, for example, when we infer the type of the following application:
    $$[[· ; · ⊢ ∀α⁺ . ↓(α⁺ → ↑Int) → ↓↑α⁺ → ↑Int ● {λx.λy.let y' = x(y); return y'} ⇒> ↑Int]]$$
    For $[[x]]$ to be applicable to $[[y]]$,
    the type of $[[y]]$---$[[↓↑α⁺]]$ must be a subtype of
    $[[α⁺]]$---the type expected by $[[x]]$. This way, 
    this inference is possible if and only if
    $[[(α̂⁺ :≥ ↓↑α̂⁺)]]$ is solvable.
    Using a similar scheme, we can construct different examples
    whose resolution significantly relies on second-order pattern unification.
    Thus, we leave the resolution of this type of constraints beyond the scope of this work.

    % \paragraph*{Restricting Inference}
    Instead, we strengthen
    \ruleref{\ottdruleATArrowAppBidirLabel}
    in such a way that it never checks a term against an algorithmic type. Type
    checking $[[v]]$ against $[[uQ]]$ is replaced by a more restrictive
    premise---checking that $[[v]]$ \emph{synthesizes a subtype} of $[[uQ]]$:
    $$ \ottdruleATArrowAppBidirSLabeled{} $$
    As one can expect, the declarative rule is also changed accordingly.
    In terms of practicality, this change disallows
    implicit checking against a \emph{polymorphic} type:
    $[[·;· ⊢ λx.return x ⇐ ∀α⁺.α⁺ → ↑α⁺]]$ is not allowed, because
    it would require adding algorithmic $[[x:α̂⁺]]$ to the context.
    However, once the instantiation is made explicit or the lambda is explicitly 
    polymorphic, the checking is allowed:
     $[[·;· ⊢ λx.return x ⇐ Int → ↑Int]]$ and 
     $[[·;· ⊢ Λα⁺.λx:α⁺.return x ⇐ ∀α⁺.α⁺ → ↑α⁺]]$ are both valid.


\section{Conclusion}
We have presented a type inference algorithm for an impredicative polymorphic
lambda calculus with existential types and subtyping. The system is designed in
the \CBPV paradigm, which allowed us to restrict it to a decidable fragment of the
language described declaratively. We presented the inference algorithm
and proved its soundness and completeness with respect to the specification. The
algorithm combines unification---a standard type inference technique---with
anti-unification---which has not been used in type inference before.


%% Bibliography
%% \balance %% necessary for two-column formats
% \printbibliography
% \bibliography{../biblio}

% \lstset{basicstyle=\ttfamily\scriptsize,keywordstyle=\color{black}\bfseries\underbar,tabsize=8,showstringspaces=false,frame=L,label= ,caption= ,captionpos=b,numbers=none}

% \clearpage
% \appendix
% \input{./tex/appendix/recovery.tex}
% \input{./tex/appendix/drf.tex}

% }

\bibliography{../biblio}

\end{document}
\endinput
