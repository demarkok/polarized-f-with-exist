\UseRawInputEncoding
% vim: ft=tex
\documentclass[a4,natbib=false]{article}
\usepackage[a4paper, total={8in, 10in}]{geometry}
\usepackage{hyperref}
\usepackage{mathpartir}

\usepackage{lscape}
\usepackage{amsmath}
\usepackage{amsthm}
\usepackage{amssymb}
\usepackage{booktabs}
\usepackage{multicol}
\usepackage{supertabular}
\usepackage[inline]{enumitem}
\usepackage{cleveref}
\usepackage{proof}

\usepackage{stackengine}

\usepackage{mathabx}
\usepackage[dvipsnames]{xcolor}
\usepackage{scalerel}


\usepackage{todonotes}

\usepackage{enumitem}
\usepackage{xparse}
\usepackage{casenum}

\usepackage{braket}

\newcommand{\niton}{\not\owns}

\newcommand{\ilyam}[1]{{\color{red} \texttt{Ilya:  #1}}}

\newtheorem{definition}{Definition}
\newtheorem{theorem}{Theorem}
\newtheorem{lemma}{Lemma}
\newtheorem{corollary}{Corollary}
\newtheorem{observation}{Observation}
\newtheorem*{assertion*}{Assertion}

% https://tex.stackexchange.com/questions/85033/colored-symbols/85035#85035
\newcommand*{\mathcolor}{}
\def\mathcolor#1#{ \mathcoloraux{#1} }
\newcommand*{\mathcoloraux}[3]{%
  \protect\leavevmode
  \begingroup
  \color#1{#2}#3%
  \endgroup
}

\newcommand{\UB}[0]{\mathsf{UB}}
\newcommand{\NFUB}[0]{\mathsf{NFUB}}


% \newcounter{casenum}

% \newenvironment{caseof}
% {%
%   \par
%   \setlength{\parskip}{6pt}%
%   % \setlength{\parindent}{0pt}%
%   \everypar{\setlength{\hangindent}{17pt}}%
%   \setcounter{casenum}{0}%
% }
% {\par\vskip.5\baselineskip}

% \NewDocumentCommand{\case}{omm}{%
%   % \vskip.5\baselineskip\par%
%   \itemindent\parindent
%   \refstepcounter{casenum}%
%   {\bfseries Case} {\bfseries \arabic{casenum}}%
%   \IfNoValueF{#1}{\label{#1}}%
%   {\bfseries:} #2\\#3 %
% }



% generated by Ott 0.32 from: grammar.ott rules.ott unification.ott
\newcommand{\ottdrule}[4][]{{\displaystyle\frac{\begin{array}{l}#2\end{array}}{#3}\quad\ottdrulename{#4}}}
\newcommand{\ottusedrule}[1]{\[#1\]}
\newcommand{\ottpremise}[1]{ #1 \\}
\newenvironment{ottdefnblock}[3][]{ \framebox{\mbox{#2}} \quad #3 \\[0pt]}{}
\newenvironment{ottfundefnblock}[3][]{ \framebox{\mbox{#2}} \quad #3 \\[0pt]\begin{displaymath}\begin{array}{l}}{\end{array}\end{displaymath}}
\newcommand{\ottfunclause}[2]{ #1 \equiv #2 \\}
\newcommand{\ottnt}[1]{\mathit{#1}}
\newcommand{\ottmv}[1]{\mathit{#1}}
\newcommand{\ottkw}[1]{\mathbf{#1}}
\newcommand{\ottsym}[1]{#1}
\newcommand{\ottcom}[1]{\text{#1}}
\newcommand{\ottdrulename}[1]{\textsc{#1}}
\newcommand{\ottcomplu}[5]{\overline{#1}^{\,#2\in #3 #4 #5}}
\newcommand{\ottcompu}[3]{\overline{#1}^{\,#2<#3}}
\newcommand{\ottcomp}[2]{\overline{#1}^{\,#2}}
\newcommand{\ottgrammartabular}[1]{\begin{supertabular}{llcllllll}#1\end{supertabular}}
\newcommand{\ottmetavartabular}[1]{\begin{supertabular}{ll}#1\end{supertabular}}
\newcommand{\ottrulehead}[3]{$#1$ & & $#2$ & & & \multicolumn{2}{l}{#3}}
\newcommand{\ottprodline}[6]{& & $#1$ & $#2$ & $#3 #4$ & $#5$ & $#6$}
\newcommand{\ottfirstprodline}[6]{\ottprodline{#1}{#2}{#3}{#4}{#5}{#6}}
\newcommand{\ottlongprodline}[2]{& & $#1$ & \multicolumn{4}{l}{$#2$}}
\newcommand{\ottfirstlongprodline}[2]{\ottlongprodline{#1}{#2}}
\newcommand{\ottbindspecprodline}[6]{\ottprodline{#1}{#2}{#3}{#4}{#5}{#6}}
\newcommand{\ottprodnewline}{\\}
\newcommand{\ottinterrule}{\\[5.0mm]}
\newcommand{\ottafterlastrule}{\\}

\newcommand{\appRightarrow}{ \mathcolor{OliveGreen}{\Rightarrow \hspace{-7pt} \Rightarrow} }

\newcommand{\tripprox}{\setbox0\hbox{$\approx$} \mbox{\makebox[0pt][l]{\raisebox{0.48\ht0}{$\approx$} }$\approx$} }

\newcommand{\approxRight}{ \mathrel{ \tripprox \hspace{-2.3pt}  \raisebox{0.24\ht0}{$>$} } }
\newcommand{\appBull}{ \mathcolor{OliveGreen}{\bullet} }
\newcommand{\rcolor}{blue}
\newcommand{\ccolor}{purple}

\usepackage{mathabx}
\usepackage{color}
\usepackage[dvipsnames,usenames]{xcolor}

% https://tex.stackexchange.com/questions/33401/a-version-of-colorbox-that-works-inside-math-environments
\setlength{\fboxsep}{1pt}
\newcommand{\ngbox}[1]{\mathchoice%
  {\colorbox{black!8}{$\displaystyle      \mathit{ #1 } $} }%
  {\colorbox{black!8}{$\textstyle         \mathit{ #1 } $} }%
  {\colorbox{black!8}{$\scriptstyle       \mathit{ #1 } $} }%
  {\colorbox{black!8}{$\scriptscriptstyle \mathit{ #1 } $} } }%

% https://tex.stackexchange.com/questions/85033/colored-symbols/85035#85035
\newcommand*{\mathcolor}{}
\def\mathcolor#1#{ \mathcoloraux{#1} }
\newcommand*{\mathcoloraux}[3]{%
  \protect\leavevmode
  \begingroup
    \color#1{#2}#3%
  \endgroup
}

\newcommand{\ottmetavars}{
\ottmetavartabular{
 $ \ottmv{x} ,\, \ottmv{y} $ & \ottcom{term variable} \\
 $ \ottmv{f} ,\, \ottmv{g} $ & \ottcom{constructors} \\
 $ \widehat{\alpha} ,\, \widehat{\beta} ,\, \widehat{\gamma} ,\, \widehat{\delta} $ & \ottcom{unification variable} \\
 $ \vec{x} ,\, \vec{y} ,\, \vec{z} ,\, \vec{t} $ & \ottcom{variable list} \\
 $ \ottmv{n} ,\, \ottmv{m} ,\, \ottmv{i} ,\, \ottmv{j} $ & \ottcom{index variables} \\
}}

\newcommand{\ottarn}{
\ottrulehead{n  ,\ k}{::=}{\ottcom{arity}}\ottprodnewline
\ottfirstprodline{|}{\ottsym{0}}{}{}{}{}\ottprodnewline
\ottprodline{|}{\ottsym{1}}{}{}{}{}\ottprodnewline
\ottprodline{|}{\ottsym{2}}{}{}{}{}\ottprodnewline
\ottprodline{|}{n_{{\mathrm{1}}}  \ottsym{+}  n_{{\mathrm{2}}}}{}{}{}{}\ottprodnewline
\ottprodline{|}{\ottsym{\mbox{$\mid$}}  \ottnt{vars}  \ottsym{\mbox{$\mid$}}}{}{}{}{}}

\newcommand{\ottvars}{
\ottrulehead{\ottnt{vars}}{::=}{\ottcom{variable list}}\ottprodnewline
\ottfirstprodline{|}{\vec{x}}{}{}{}{}\ottprodnewline
\ottprodline{|}{\ottmv{x_{{\mathrm{1}}}}  \ottsym{,} \, .. \, \ottsym{,}  \ottmv{x_{\ottmv{n}}}}{}{}{}{}\ottprodnewline
\ottprodline{|}{\ottnt{vars_{{\mathrm{1}}}}  \cap  \ottnt{vars_{{\mathrm{2}}}}}{}{}{}{}\ottprodnewline
\ottprodline{|}{\ottnt{vars_{{\mathrm{1}}}}  \sqcap  \ottnt{vars_{{\mathrm{2}}}}}{}{}{}{}\ottprodnewline
\ottprodline{|}{\ottcomp{\ottnt{vars_{\ottmv{i}}}}{\ottmv{i}}}{}{}{}{}\ottprodnewline
\ottprodline{|}{\ottkw{UVARGS} \, \ottnt{t}} {\textsf{M}}{}{\textsf{[F]}}{\ottcom{arguments of the unification variables of the term}}}

\newcommand{\ottt}{
\ottrulehead{\ottnt{t}  ,\ \ottnt{v}  ,\ \ottnt{w}  ,\ \ottnt{h}  ,\ \ottnt{d}}{::=}{\ottcom{terms}}\ottprodnewline
\ottfirstprodline{|}{\ottmv{x}}{}{}{}{}\ottprodnewline
\ottprodline{|}{\ottmv{x}  \ottsym{.}  \ottnt{t}}{}{\textsf{bind}\; \ottmv{x}\; \textsf{in}\; \ottnt{t}}{}{}\ottprodnewline
\ottprodline{|}{\ottnt{vars}  \ottsym{.}  \ottnt{t}}{}{}{}{}\ottprodnewline
\ottprodline{|}{\widehat{\alpha}  \ottsym{[}  \ottnt{vars}  \ottsym{]}}{}{}{}{}\ottprodnewline
\ottprodline{|}{\ottmv{f}  \ottsym{(}  \ottnt{t_{{\mathrm{1}}}}  \ottsym{,..,}  \ottnt{t_{\ottmv{n}}}  \ottsym{)}}{}{}{}{}\ottprodnewline
\ottprodline{|}{\ottsym{[}  \Theta  \ottsym{]}  \ottnt{v}} {\textsf{M}}{}{}{}\ottprodnewline
\ottprodline{|}{\ottsym{(}  \ottnt{v}  \ottsym{)}} {\textsf{S}}{}{}{}\ottprodnewline
\ottprodline{|}{\ottsym{\{}  \widehat{\alpha}_{{\mathrm{1}}}  \ottsym{[}  \ottnt{vars_{{\mathrm{1}}}}  \ottsym{]}  \ottsym{/}  \widehat{\alpha}_{{\mathrm{2}}}  \ottsym{[}  \ottnt{vars_{{\mathrm{2}}}}  \ottsym{]}  \ottsym{\}}  \ottnt{t}}{}{}{}{}}

\newcommand{\ottterminals}{
\ottrulehead{\ottnt{terminals}}{::=}{}\ottprodnewline
\ottfirstprodline{|}{ \in }{}{}{}{}\ottprodnewline
\ottprodline{|}{ \notin }{}{}{}{}\ottprodnewline
\ottprodline{|}{ \cdot }{}{}{}{}\ottprodnewline
\ottprodline{|}{ \vdash }{}{}{}{}\ottprodnewline
\ottprodline{|}{ \mathcolor{\rcolor}{\vDash} }{}{}{}{}\ottprodnewline
\ottprodline{|}{ \mathcolor{\rcolor}{\Dashv} }{}{}{}{}\ottprodnewline
\ottprodline{|}{ \mathcolor{\ccolor}{\VDash} }{}{}{}{}\ottprodnewline
\ottprodline{|}{ \mathcolor{\ccolor}{\DashV} }{}{}{}{}\ottprodnewline
\ottprodline{|}{ \neq }{}{}{}{}\ottprodnewline
\ottprodline{|}{ \appRightarrow }{}{}{}{}\ottprodnewline
\ottprodline{|}{ \appBull }{}{}{}{}\ottprodnewline
\ottprodline{|}{ \mathcolor{\rcolor}{\equiv} }{}{}{}{}\ottprodnewline
\ottprodline{|}{ \equiv_{n} }{}{}{}{}\ottprodnewline
\ottprodline{|}{ \searrow }{}{}{}{}\ottprodnewline
\ottprodline{|}{ \unlhd }{}{}{}{}\ottprodnewline
\ottprodline{|}{ \cap }{}{}{}{}\ottprodnewline
\ottprodline{|}{ \sqcap }{}{}{}{}\ottprodnewline
\ottprodline{|}{ \subseteq }{}{}{}{}\ottprodnewline
\ottprodline{|}{ \emptyset }{}{}{}{}\ottprodnewline
\ottprodline{|}{ \approxRight }{}{}{}{}}

\newcommand{\ottT}{
\ottrulehead{\Theta}{::=}{\ottcom{computational variable context}}\ottprodnewline
\ottfirstprodline{|}{\ottmv{x}}{}{}{}{\ottcom{a variable}}\ottprodnewline
\ottprodline{|}{\vec{x}} {\textsf{S}}{}{}{\ottcom{variables}}\ottprodnewline
\ottprodline{|}{\widehat{\alpha}  \ottsym{:}  n}{}{}{}{\ottcom{a unification variable}}\ottprodnewline
\ottprodline{|}{\widehat{\alpha}  \ottsym{:}  n  \ottsym{=}  \ottnt{t}}{}{}{}{\ottcom{instantiate a unification variable}}\ottprodnewline
\ottprodline{|}{\ottcomp{\Theta_{\ottmv{i}}}{\ottmv{i}}}{}{}{}{\ottcom{concatenate contexts}}\ottprodnewline
\ottprodline{|}{\cdot}{}{}{}{\ottcom{empty context}}\ottprodnewline
\ottprodline{|}{\Theta_{{\mathrm{1}}}  \ottsym{\{}  \Theta_{{\mathrm{2}}}  \ottsym{\}}} {\textsf{S}}{}{}{\ottcom{surgery}}\ottprodnewline
\ottprodline{|}{\ottsym{(}  \Theta  \ottsym{)}} {\textsf{S}}{}{}{}\ottprodnewline
\ottprodline{|}{\Theta_{{\mathrm{1}}}  \ottsym{\mbox{$\backslash{}$}}  \ottsym{(}  \widehat{\alpha}_{{\mathrm{1}}}  \ottsym{,..,}  \widehat{\alpha}_{\ottmv{n}}  \ottsym{)}} {\textsf{S}}{}{}{\ottcom{context subtraction}}\ottprodnewline
\ottprodline{|}{ \Theta' ^{\color{red}\star} } {\textsf{M}}{}{\textsf{[F]}}{\ottcom{context self-application}}}

\newcommand{\ottformula}{
\ottrulehead{\ottnt{formula}}{::=}{}\ottprodnewline
\ottfirstprodline{|}{\ottnt{judgement}}{}{}{}{}\ottprodnewline
\ottprodline{|}{\ottmv{x}  \in  \Theta}{}{}{}{\ottcom{lookup $\ottmv{x}$ in context $\Theta$}}\ottprodnewline
\ottprodline{|}{\widehat{\alpha}  \notin  \ottnt{t}}{}{}{}{}\ottprodnewline
\ottprodline{|}{\vec{x}  \subseteq  \Theta}{}{}{}{}\ottprodnewline
\ottprodline{|}{\ottkw{let} \, \Theta_{{\mathrm{1}}}  \ottsym{=}  \Theta_{{\mathrm{2}}}}{}{}{}{}\ottprodnewline
\ottprodline{|}{\ottkw{let} \, \vec{x}  \ottsym{=}  \ottnt{vars}}{}{}{}{}\ottprodnewline
\ottprodline{|}{\ottnt{vars}  \cap  \Theta  \ottsym{=}  \emptyset}{}{}{}{}\ottprodnewline
\ottprodline{|}{\ottnt{vars_{{\mathrm{1}}}}  \cap  \ottnt{vars_{{\mathrm{2}}}}  \ottsym{=}  \emptyset}{}{}{}{}\ottprodnewline
\ottprodline{|}{\ottkw{UV} \, \ottsym{(}  \ottnt{t}  \ottsym{)}  \ottsym{=}  \widehat{\alpha}_{{\mathrm{1}}}  \ottsym{[}  \ottnt{vars_{{\mathrm{1}}}}  \ottsym{]}  \ottsym{,..,}  \widehat{\alpha}_{\ottmv{n}}  \ottsym{[}  \ottnt{vars_{\ottmv{n}}}  \ottsym{]}}{}{}{}{}\ottprodnewline
\ottprodline{|}{\ottkw{ux} \, \ottsym{:}  n  \in  \Theta}{}{}{}{\ottcom{lookupof $\ottkw{ux}$ in context $\Theta$}}\ottprodnewline
\ottprodline{|}{\ottkw{ux} \, \ottsym{:}  n  \ottsym{=}  \ottnt{t}  \in  \Theta}{}{}{}{\ottcom{lookup type of $\ottkw{ux}$  instantiation in context $\Theta$}}\ottprodnewline
\ottprodline{|}{\ottnt{v}  \neq  \ottnt{w}}{}{}{}{}\ottprodnewline
\ottprodline{|}{\vec{x}  \ottsym{=}  \ottnt{vars}}{}{}{}{}\ottprodnewline
\ottprodline{|}{\ottkw{arity} \, \ottmv{f}  \ottsym{=}  \ottsym{[}  n_{{\mathrm{1}}}  \ottsym{,..,}  n_{\ottmv{n}}  \ottsym{]}}{}{}{}{}\ottprodnewline
\ottprodline{|}{\ottnt{formula_{{\mathrm{1}}}} \quad .. \quad \ottnt{formula_{\ottmv{n}}}}{}{}{}{}\ottprodnewline
\ottprodline{|}{ \cdots }{}{}{}{}}

\newcommand{\ottFoo}{
\ottrulehead{\ottnt{Foo}}{::=}{}\ottprodnewline
\ottfirstprodline{|}{ \Theta' ^{\color{red}\star} }{}{}{}{\ottcom{context self-application}}\ottprodnewline
\ottprodline{|}{\ottkw{UVARGS} \, \ottnt{t}  \ottsym{===}  \ottnt{vars}}{}{}{}{\ottcom{arguments of the unification variables of the term}}}

\newcommand{\ottAOne}{
\ottrulehead{\ottnt{A1}}{::=}{}\ottprodnewline
\ottfirstprodline{|}{\Theta_{{\mathrm{1}}}  \mathcolor{\rcolor}{\vDash}  \ottnt{v}  \mathcolor{\rcolor}{\equiv}  \ottnt{w}  \ottsym{:}  n  \mathcolor{\rcolor}{\Dashv}  \Theta_{{\mathrm{2}}}}{}{}{}{\ottcom{The unification}}}

\newcommand{\ottBOne}{
\ottrulehead{\ottnt{B1}}{::=}{}\ottprodnewline
\ottfirstprodline{|}{\Theta_{{\mathrm{1}}}  \mathcolor{\ccolor}{\VDash}  \ottnt{v}  \cap  \ottsym{[}  \ottnt{vars}  \ottsym{]}  \approxRight  \ottnt{w}  \mathcolor{\ccolor}{\DashV}  \Theta_{{\mathrm{2}}}}{}{}{}{\ottcom{The prunning phase}}\ottprodnewline
\ottprodline{|}{\Theta_{{\mathrm{1}}}  \mathcolor{\ccolor}{\VDash}  \ottnt{v}  \mathcolor{\rcolor}{\equiv}  \ottnt{w}  \mathcolor{\ccolor}{\DashV}  \Theta_{{\mathrm{2}}}}{}{}{}{\ottcom{The alternative unification}}\ottprodnewline
\ottprodline{|}{\ottnt{v} \, \ottkw{ext}}{}{}{}{\ottcom{The external term}}\ottprodnewline
\ottprodline{|}{\Theta \, \ottkw{ext}}{}{}{}{\ottcom{The external environment}}}

\newcommand{\ottjudgement}{
\ottrulehead{\ottnt{judgement}}{::=}{}\ottprodnewline
\ottfirstprodline{|}{\ottnt{A1}}{}{}{}{}\ottprodnewline
\ottprodline{|}{\ottnt{B1}}{}{}{}{}}

\newcommand{\ottuserXXsyntax}{
\ottrulehead{\ottnt{user\_syntax}}{::=}{}\ottprodnewline
\ottfirstprodline{|}{\ottmv{x}}{}{}{}{}\ottprodnewline
\ottprodline{|}{\ottmv{f}}{}{}{}{}\ottprodnewline
\ottprodline{|}{\widehat{\alpha}}{}{}{}{}\ottprodnewline
\ottprodline{|}{\vec{x}}{}{}{}{}\ottprodnewline
\ottprodline{|}{\ottmv{n}}{}{}{}{}\ottprodnewline
\ottprodline{|}{n}{}{}{}{}\ottprodnewline
\ottprodline{|}{\ottnt{vars}}{}{}{}{}\ottprodnewline
\ottprodline{|}{\ottnt{t}}{}{}{}{}\ottprodnewline
\ottprodline{|}{\ottnt{terminals}}{}{}{}{}\ottprodnewline
\ottprodline{|}{\Theta}{}{}{}{}\ottprodnewline
\ottprodline{|}{\ottnt{formula}}{}{}{}{}}

\newcommand{\ottgrammar}{\ottgrammartabular{
\ottarn\ottinterrule
\ottvars\ottinterrule
\ottt\ottinterrule
\ottterminals\ottinterrule
\ottT\ottinterrule
\ottformula\ottinterrule
\ottFoo\ottinterrule
\ottAOne\ottinterrule
\ottBOne\ottinterrule
\ottjudgement\ottinterrule
\ottuserXXsyntax\ottafterlastrule
}}

% defnss
% fundefns Foo
% fundefn simpl

\newcommand{\ottfundefnsimpl}[1]{\begin{ottfundefnblock}[#1]{$ \Theta' ^{\color{red}\star} $}{\ottcom{context self-application}}
\ottfunclause{ \cdot ^{\color{red}\star} }{\cdot}%
\ottfunclause{ \ottsym{(}  \Theta  \ottsym{,}  \ottmv{x}  \ottsym{)} ^{\color{red}\star} }{ \Theta ^{\color{red}\star}   \ottsym{,}  \ottmv{x}}%
\ottfunclause{ \ottsym{(}  \Theta  \ottsym{,}  \widehat{\alpha}  \ottsym{:}  n  \ottsym{)} ^{\color{red}\star} }{ \Theta ^{\color{red}\star}   \ottsym{,}  \widehat{\alpha}  \ottsym{:}  n}%
\ottfunclause{ \ottsym{(}  \Theta  \ottsym{,}  \widehat{\alpha}  \ottsym{:}  n  \ottsym{=}  \ottnt{t}  \ottsym{)} ^{\color{red}\star} }{ \Theta ^{\color{red}\star}   \ottsym{,}  \widehat{\alpha}  \ottsym{:}  n  \ottsym{=}  \ottsym{[}   \Theta ^{\color{red}\star}   \ottsym{]}  \ottnt{t}}%
\end{ottfundefnblock}}


% fundefn uvarargs

\newcommand{\ottfundefnuvarargs}[1]{\begin{ottfundefnblock}[#1]{$\ottkw{UVARGS} \, \ottnt{t}$}{\ottcom{arguments of the unification variables of the term}}
\end{ottfundefnblock}}


\newcommand{\ottfundefnsFoo}{
\ottfundefnsimpl{}
\ottfundefnuvarargs{}}

% defns A1
%% defn un
\newcommand{\ottdruleVXXV}[1]{\ottdrule[#1]{%
\ottpremise{\ottmv{x}  \in  \Theta}%
}{
\Theta  \mathcolor{\rcolor}{\vDash}  \ottmv{x}  \mathcolor{\rcolor}{\equiv}  \ottmv{x}  \ottsym{:}  \ottsym{0}  \mathcolor{\rcolor}{\Dashv}  \Theta}{%
{\ottdrulename{V\_V}}{}%
}}


\newcommand{\ottdruleBXXB}[1]{\ottdrule[#1]{%
\ottpremise{\Theta_{{\mathrm{1}}}  \ottsym{,}  \ottmv{x}  \mathcolor{\rcolor}{\vDash}  \ottnt{t}  \mathcolor{\rcolor}{\equiv}  \ottnt{t}  \ottsym{:}  n  \mathcolor{\rcolor}{\Dashv}  \Theta_{{\mathrm{2}}}  \ottsym{,}  \ottmv{x}}%
}{
\Theta_{{\mathrm{1}}}  \mathcolor{\rcolor}{\vDash}  \ottmv{x}  \ottsym{.}  \ottnt{t}  \mathcolor{\rcolor}{\equiv}  \ottmv{x}  \ottsym{.}  \ottnt{t}  \ottsym{:}  n  \ottsym{+}  \ottsym{1}  \mathcolor{\rcolor}{\Dashv}  \Theta_{{\mathrm{2}}}}{%
{\ottdrulename{B\_B}}{}%
}}


\newcommand{\ottdruleFXXF}[1]{\ottdrule[#1]{%
\ottpremise{\ottkw{arity} \, \ottmv{f}  \ottsym{=}  \ottsym{[}  k_{{\mathrm{1}}}  \ottsym{,..,}  k_{\ottmv{n}}  \ottsym{]}}%
\ottpremise{\Theta_{{\mathrm{0}}}  \mathcolor{\rcolor}{\vDash}  \ottnt{v_{{\mathrm{1}}}}  \mathcolor{\rcolor}{\equiv}  \ottnt{w_{{\mathrm{1}}}}  \ottsym{:}  k_{{\mathrm{1}}}  \mathcolor{\rcolor}{\Dashv}  \Theta_{{\mathrm{1}}}}%
\ottpremise{\Theta_{{\mathrm{1}}}  \mathcolor{\rcolor}{\vDash}  \ottsym{[}  \Theta_{{\mathrm{1}}}  \ottsym{]}  \ottnt{v_{{\mathrm{2}}}}  \mathcolor{\rcolor}{\equiv}  \ottsym{[}  \Theta_{{\mathrm{1}}}  \ottsym{]}  \ottnt{w_{{\mathrm{2}}}}  \ottsym{:}  k_{{\mathrm{2}}}  \mathcolor{\rcolor}{\Dashv}  \Theta_{{\mathrm{2}}}}%
\ottpremise{ \cdots }%
\ottpremise{\Theta_{{\ottmv{n}-1}}  \mathcolor{\rcolor}{\vDash}  \ottsym{[}  \Theta_{{\ottmv{n}-1}}  \ottsym{]}  \ottnt{v_{\ottmv{n}}}  \mathcolor{\rcolor}{\equiv}  \ottsym{[}  \Theta_{{\ottmv{n}-1}}  \ottsym{]}  \ottnt{w_{\ottmv{n}}}  \ottsym{:}  k_{\ottmv{n}}  \mathcolor{\rcolor}{\Dashv}  \Theta_{\ottmv{n}}}%
}{
\Theta_{{\mathrm{0}}}  \mathcolor{\rcolor}{\vDash}  \ottmv{f}  \ottsym{(}  \ottnt{v_{{\mathrm{1}}}}  \ottsym{,..,}  \ottnt{v_{\ottmv{n}}}  \ottsym{)}  \mathcolor{\rcolor}{\equiv}  \ottmv{f}  \ottsym{(}  \ottnt{w_{{\mathrm{1}}}}  \ottsym{,..,}  \ottnt{w_{\ottmv{n}}}  \ottsym{)}  \ottsym{:}  \ottsym{0}  \mathcolor{\rcolor}{\Dashv}  \Theta_{\ottmv{n}}}{%
{\ottdrulename{F\_F}}{}%
}}


\newcommand{\ottdruleUVXXV}[1]{\ottdrule[#1]{%
}{
\Theta  \ottsym{\{}  \widehat{\alpha}  \ottsym{:}  n  \ottsym{\}}  \mathcolor{\rcolor}{\vDash}  \widehat{\alpha}  \ottsym{[}  \vec{x}  \ottsym{]}  \mathcolor{\rcolor}{\equiv}  \ottmv{x_{\ottmv{i}}}  \ottsym{:}  \ottsym{0}  \mathcolor{\rcolor}{\Dashv}   \ottsym{(}  \Theta  \ottsym{\{}  \widehat{\alpha}  \ottsym{:}  n  \ottsym{=}  \vec{x}  \ottsym{.}  \ottmv{x_{\ottmv{i}}}  \ottsym{\}}  \ottsym{)} ^{\color{red}\star} }{%
{\ottdrulename{UV\_V}}{}%
}}


\newcommand{\ottdruleUVXXUV}[1]{\ottdrule[#1]{%
\ottpremise{\vec{z}  \ottsym{=}  \vec{x}  \sqcap  \vec{y}}%
}{
\Theta  \ottsym{\{}  \widehat{\alpha}  \ottsym{:}  n  \ottsym{\}}  \mathcolor{\rcolor}{\vDash}  \widehat{\alpha}  \ottsym{[}  \vec{x}  \ottsym{]}  \mathcolor{\rcolor}{\equiv}  \widehat{\alpha}  \ottsym{[}  \vec{y}  \ottsym{]}  \ottsym{:}  \ottsym{0}  \mathcolor{\rcolor}{\Dashv}   \ottsym{(}  \Theta  \ottsym{\{}  \widehat{\beta}  \ottsym{:}  \ottsym{\mbox{$\mid$}}  \vec{z}  \ottsym{\mbox{$\mid$}}  \ottsym{,}  \widehat{\alpha}  \ottsym{:}  n  \ottsym{=}  \vec{x}  \ottsym{.}  \widehat{\beta}  \ottsym{[}  \vec{z}  \ottsym{]}  \ottsym{\}}  \ottsym{)} ^{\color{red}\star} }{%
{\ottdrulename{UV\_UV}}{}%
}}


\newcommand{\ottdruleUVXXUVTwo}[1]{\ottdrule[#1]{%
\ottpremise{\vec{z}  \ottsym{=}  \vec{x}  \cap  \vec{y}}%
}{
\Theta_{{\mathrm{0}}}  \ottsym{,}  \widehat{\alpha}  \ottsym{:}  n  \ottsym{,}  \Theta_{{\mathrm{1}}}  \ottsym{,}  \widehat{\beta}  \ottsym{:}  k  \ottsym{,}  \Theta_{{\mathrm{2}}}  \mathcolor{\rcolor}{\vDash}  \widehat{\alpha}  \ottsym{[}  \vec{x}  \ottsym{]}  \mathcolor{\rcolor}{\equiv}  \widehat{\beta}  \ottsym{[}  \vec{y}  \ottsym{]}  \ottsym{:}  \ottsym{0}  \mathcolor{\rcolor}{\Dashv}   \ottsym{(}  \Theta_{{\mathrm{0}}}  \ottsym{,}  \ottsym{(}  \widehat{\gamma}  \ottsym{:}  \ottsym{\mbox{$\mid$}}  \vec{z}  \ottsym{\mbox{$\mid$}}  \ottsym{)}  \ottsym{,}  \ottsym{(}  \widehat{\alpha}  \ottsym{:}  n  \ottsym{=}  \vec{x}  \ottsym{.}  \widehat{\gamma}  \ottsym{[}  \vec{z}  \ottsym{]}  \ottsym{)}  \ottsym{,}  \Theta_{{\mathrm{1}}}  \ottsym{,}  \ottsym{(}  \widehat{\beta}  \ottsym{:}  k  \ottsym{=}  \vec{y}  \ottsym{.}  \widehat{\gamma}  \ottsym{[}  \vec{z}  \ottsym{]}  \ottsym{)}  \ottsym{,}  \Theta_{{\mathrm{2}}}  \ottsym{)} ^{\color{red}\star} }{%
{\ottdrulename{UV\_UV2}}{}%
}}


\newcommand{\ottdruleUVXXF}[1]{\ottdrule[#1]{%
\ottpremise{\widehat{\alpha}  \notin  \ottmv{f}  \ottsym{(}  \ottnt{t_{{\mathrm{1}}}}  \ottsym{,..,}  \ottnt{t_{\ottmv{m}}}  \ottsym{)}}%
\ottpremise{\ottkw{arity} \, \ottmv{f}  \ottsym{=}  \ottsym{[}  k_{{\mathrm{1}}}  \ottsym{,..,}  k_{\ottmv{n}}  \ottsym{]}}%
\ottpremise{\Theta_{{\mathrm{0}}}  \ottsym{\{}  \widehat{\beta}_{{\mathrm{1}}}  \ottsym{:}  n  \ottsym{+}  k_{{\mathrm{1}}}  \ottsym{,}  \widehat{\alpha}  \ottsym{:}  n  \ottsym{\}}  \mathcolor{\rcolor}{\vDash}  \vec{y}_{{\mathrm{1}}}  \ottsym{.}  \widehat{\beta}_{{\mathrm{1}}}  \ottsym{[}  \vec{x}  \ottsym{,}  \vec{y}_{{\mathrm{1}}}  \ottsym{]}  \mathcolor{\rcolor}{\equiv}  \ottnt{t_{{\mathrm{1}}}}  \ottsym{:}  k_{{\mathrm{1}}}  \mathcolor{\rcolor}{\Dashv}  \Theta_{{\mathrm{1}}}  \ottsym{\{}  \widehat{\alpha}  \ottsym{:}  n  \ottsym{\}}}%
\ottpremise{\Theta_{{\mathrm{1}}}  \ottsym{\{}  \widehat{\beta}_{{\mathrm{2}}}  \ottsym{:}  n  \ottsym{+}  k_{{\mathrm{2}}}  \ottsym{,}  \widehat{\alpha}  \ottsym{:}  n  \ottsym{\}}  \mathcolor{\rcolor}{\vDash}  \vec{y}_{{\mathrm{2}}}  \ottsym{.}  \widehat{\beta}_{{\mathrm{2}}}  \ottsym{[}  \vec{x}  \ottsym{,}  \vec{y}_{{\mathrm{2}}}  \ottsym{]}  \mathcolor{\rcolor}{\equiv}  \ottsym{[}  \Theta_{{\mathrm{1}}}  \ottsym{]}  \ottnt{t_{{\mathrm{2}}}}  \ottsym{:}  k_{{\mathrm{2}}}  \mathcolor{\rcolor}{\Dashv}  \Theta_{{\mathrm{2}}}  \ottsym{\{}  \widehat{\alpha}  \ottsym{:}  n  \ottsym{\}}}%
\ottpremise{ \cdots }%
\ottpremise{\Theta_{{\ottmv{n}-1}}  \ottsym{\{}  \widehat{\beta}_{\ottmv{m}}  \ottsym{:}  n  \ottsym{+}  k_{{\mathrm{2}}}  \ottsym{,}  \widehat{\alpha}  \ottsym{:}  n  \ottsym{\}}  \mathcolor{\rcolor}{\vDash}  \vec{y}_{\ottmv{m}}  \ottsym{.}  \widehat{\beta}_{\ottmv{m}}  \ottsym{[}  \vec{x}  \ottsym{,}  \vec{y}_{\ottmv{m}}  \ottsym{]}  \mathcolor{\rcolor}{\equiv}  \ottsym{[}  \Theta_{{\ottmv{m}-1}}  \ottsym{]}  \ottnt{t_{\ottmv{m}}}  \ottsym{:}  k_{\ottmv{m}}  \mathcolor{\rcolor}{\Dashv}  \Theta_{\ottmv{m}}  \ottsym{\{}  \widehat{\alpha}  \ottsym{:}  n  \ottsym{\}}}%
}{
\Theta_{{\mathrm{0}}}  \ottsym{\{}  \widehat{\alpha}  \ottsym{:}  n  \ottsym{\}}  \mathcolor{\rcolor}{\vDash}  \widehat{\alpha}  \ottsym{[}  \vec{x}  \ottsym{]}  \mathcolor{\rcolor}{\equiv}  \ottmv{f}  \ottsym{(}  \ottnt{t_{{\mathrm{1}}}}  \ottsym{,..,}  \ottnt{t_{\ottmv{m}}}  \ottsym{)}  \ottsym{:}  \ottsym{0}  \mathcolor{\rcolor}{\Dashv}   \ottsym{(}  \Theta_{\ottmv{m}}  \ottsym{\{}  \widehat{\alpha}  \ottsym{:}  n  \ottsym{=}  \vec{x}  \ottsym{.}  \ottmv{f}  \ottsym{(}  \vec{y}_{{\mathrm{1}}}  \ottsym{.}  \widehat{\beta}_{{\mathrm{1}}}  \ottsym{[}  \vec{x}  \ottsym{,}  \vec{y}_{{\mathrm{1}}}  \ottsym{]}  \ottsym{,..,}  \vec{y}_{\ottmv{n}}  \ottsym{.}  \widehat{\beta}_{\ottmv{n}}  \ottsym{[}  \vec{x}  \ottsym{,}  \vec{y}_{\ottmv{n}}  \ottsym{]}  \ottsym{)}  \ottsym{\}}  \ottsym{)} ^{\color{red}\star}   \ottsym{\mbox{$\backslash{}$}}  \ottsym{(}  \widehat{\beta}_{{\mathrm{1}}}  \ottsym{,..,}  \widehat{\beta}_{\ottmv{m}}  \ottsym{)}}{%
{\ottdrulename{UV\_F}}{}%
}}

\newcommand{\ottdefnun}[1]{\begin{ottdefnblock}[#1]{$\Theta_{{\mathrm{1}}}  \mathcolor{\rcolor}{\vDash}  \ottnt{v}  \mathcolor{\rcolor}{\equiv}  \ottnt{w}  \ottsym{:}  n  \mathcolor{\rcolor}{\Dashv}  \Theta_{{\mathrm{2}}}$}{\ottcom{The unification}}
\ottusedrule{\ottdruleVXXV{}}
\ottusedrule{\ottdruleBXXB{}}
\ottusedrule{\ottdruleFXXF{}}
\ottusedrule{\ottdruleUVXXV{}}
\ottusedrule{\ottdruleUVXXUV{}}
\ottusedrule{\ottdruleUVXXUVTwo{}}
\ottusedrule{\ottdruleUVXXF{}}
\end{ottdefnblock}}


\newcommand{\ottdefnsAOne}{
\ottdefnun{}}

% defns B1
%% defn prun
\newcommand{\ottdruleaUV}[1]{\ottdrule[#1]{%
\ottpremise{\vec{z}  \ottsym{=}  \vec{y}  \cap  \vec{x}}%
}{
\Theta  \ottsym{\{}  \,  \ottsym{\}}  \ottsym{\{}  \widehat{\beta}  \ottsym{:}  n  \ottsym{\}}  \mathcolor{\ccolor}{\VDash}  \widehat{\beta}  \ottsym{[}  \vec{y}  \ottsym{]}  \cap  \ottsym{[}  \vec{x}  \ottsym{]}  \approxRight  \widehat{\beta}'  \ottsym{[}  \vec{z}  \ottsym{]}  \mathcolor{\ccolor}{\DashV}  \Theta  \ottsym{\{}  \widehat{\beta}'  \ottsym{:}  \ottsym{\mbox{$\mid$}}  \vec{z}  \ottsym{\mbox{$\mid$}}  \ottsym{\}}  \ottsym{\{}  \widehat{\beta}  \ottsym{:}  n  \ottsym{=}  \vec{y}  \ottsym{.}  \widehat{\beta}'  \ottsym{[}  \vec{z}  \ottsym{]}  \ottsym{\}}}{%
{\ottdrulename{aUV}}{}%
}}


\newcommand{\ottdruleaF}[1]{\ottdrule[#1]{%
}{
\Theta_{{\mathrm{0}}}  \ottsym{\{}  \,  \ottsym{\}}  \mathcolor{\ccolor}{\VDash}  \ottmv{f}  \ottsym{(}  \vec{y}_{{\mathrm{1}}}  \ottsym{.}  \ottnt{v_{{\mathrm{1}}}}  \ottsym{,..,}  \vec{y}_{\ottmv{n}}  \ottsym{.}  \ottnt{v_{\ottmv{n}}}  \ottsym{)}  \cap  \ottsym{[}  \vec{x}  \ottsym{]}  \approxRight  \widehat{\beta}'  \ottsym{[}  \vec{z}  \ottsym{]}  \mathcolor{\ccolor}{\DashV}  \Theta  \ottsym{\{}  \widehat{\beta}'  \ottsym{:}  \ottsym{\mbox{$\mid$}}  \vec{z}  \ottsym{\mbox{$\mid$}}  \ottsym{\}}  \ottsym{\{}  \widehat{\beta}  \ottsym{:}  n  \ottsym{=}  \vec{y}  \ottsym{.}  \widehat{\beta}'  \ottsym{[}  \vec{z}  \ottsym{]}  \ottsym{\}}}{%
{\ottdrulename{aF}}{}%
}}

\newcommand{\ottdefnaprun}[1]{\begin{ottdefnblock}[#1]{$\Theta_{{\mathrm{1}}}  \mathcolor{\ccolor}{\VDash}  \ottnt{v}  \cap  \ottsym{[}  \ottnt{vars}  \ottsym{]}  \approxRight  \ottnt{w}  \mathcolor{\ccolor}{\DashV}  \Theta_{{\mathrm{2}}}$}{\ottcom{The prunning phase}}
\ottusedrule{\ottdruleaUV{}}
\ottusedrule{\ottdruleaF{}}
\end{ottdefnblock}}

%% defn un2
\newcommand{\ottdruleaVXXV}[1]{\ottdrule[#1]{%
\ottpremise{\ottmv{x}  \in  \Theta}%
}{
\Theta  \mathcolor{\ccolor}{\VDash}  \ottmv{x}  \mathcolor{\rcolor}{\equiv}  \ottmv{x}  \mathcolor{\ccolor}{\DashV}  \Theta}{%
{\ottdrulename{aV\_V}}{}%
}}


\newcommand{\ottdruleaFXXF}[1]{\ottdrule[#1]{%
\ottpremise{\ottkw{arity} \, \ottmv{f}  \ottsym{=}  \ottsym{[}  k_{{\mathrm{1}}}  \ottsym{,..,}  k_{\ottmv{n}}  \ottsym{]}}%
\ottpremise{\Theta_{{\mathrm{0}}}  \ottsym{,}  \vec{x}_{{\mathrm{1}}}  \mathcolor{\ccolor}{\VDash}  \ottnt{v_{{\mathrm{1}}}}  \mathcolor{\rcolor}{\equiv}  \ottnt{w_{{\mathrm{1}}}}  \mathcolor{\ccolor}{\DashV}  \Theta_{{\mathrm{1}}}  \ottsym{,}  \vec{x}_{{\mathrm{1}}}}%
\ottpremise{\Theta_{{\mathrm{1}}}  \ottsym{,}  \vec{x}_{{\mathrm{2}}}  \mathcolor{\ccolor}{\VDash}  \ottsym{[}  \Theta_{{\mathrm{1}}}  \ottsym{]}  \ottnt{v_{{\mathrm{2}}}}  \mathcolor{\rcolor}{\equiv}  \ottsym{[}  \Theta_{{\mathrm{1}}}  \ottsym{]}  \ottnt{w_{{\mathrm{2}}}}  \mathcolor{\ccolor}{\DashV}  \Theta_{{\mathrm{2}}}  \ottsym{,}  \vec{x}_{{\mathrm{2}}}}%
\ottpremise{ \cdots }%
\ottpremise{\Theta_{{\ottmv{n}-1}}  \ottsym{,}  \vec{x}_{\ottmv{n}}  \mathcolor{\ccolor}{\VDash}  \ottsym{[}  \Theta_{{\ottmv{n}-1}}  \ottsym{]}  \ottnt{v_{\ottmv{n}}}  \mathcolor{\rcolor}{\equiv}  \ottsym{[}  \Theta_{{\ottmv{n}-1}}  \ottsym{]}  \ottnt{w_{\ottmv{n}}}  \mathcolor{\ccolor}{\DashV}  \Theta_{\ottmv{n}}  \ottsym{,}  \vec{x}_{\ottmv{n}}}%
}{
\Theta_{{\mathrm{0}}}  \mathcolor{\ccolor}{\VDash}  \ottmv{f}  \ottsym{(}  \vec{x}_{{\mathrm{1}}}  \ottsym{.}  \ottnt{v_{{\mathrm{1}}}}  \ottsym{,..,}  \ottmv{x_{\ottmv{n}}}  \ottsym{.}  \ottnt{v_{\ottmv{n}}}  \ottsym{)}  \mathcolor{\rcolor}{\equiv}  \ottmv{f}  \ottsym{(}  \vec{x}_{{\mathrm{1}}}  \ottsym{.}  \ottnt{w_{{\mathrm{1}}}}  \ottsym{,..,}  \vec{x}_{\ottmv{n}}  \ottsym{.}  \ottnt{w_{\ottmv{n}}}  \ottsym{)}  \mathcolor{\ccolor}{\DashV}  \Theta_{\ottmv{n}}}{%
{\ottdrulename{aF\_F}}{}%
}}


\newcommand{\ottdruleaUVXXUV}[1]{\ottdrule[#1]{%
\ottpremise{\vec{z}  \ottsym{=}  \vec{x}  \sqcap  \vec{y}}%
}{
\Theta  \ottsym{\{}  \widehat{\alpha}  \ottsym{:}  n  \ottsym{\}}  \mathcolor{\ccolor}{\VDash}  \widehat{\alpha}  \ottsym{[}  \vec{x}  \ottsym{]}  \mathcolor{\rcolor}{\equiv}  \widehat{\alpha}  \ottsym{[}  \vec{y}  \ottsym{]}  \mathcolor{\ccolor}{\DashV}   \ottsym{(}  \Theta  \ottsym{\{}  \widehat{\beta}  \ottsym{:}  \ottsym{\mbox{$\mid$}}  \vec{z}  \ottsym{\mbox{$\mid$}}  \ottsym{,}  \widehat{\alpha}  \ottsym{:}  n  \ottsym{=}  \vec{x}  \ottsym{.}  \widehat{\beta}  \ottsym{[}  \vec{z}  \ottsym{]}  \ottsym{\}}  \ottsym{)} ^{\color{red}\star} }{%
{\ottdrulename{aUV\_UV}}{}%
}}


\newcommand{\ottdruleaUVXXF}[1]{\ottdrule[#1]{%
\ottpremise{\widehat{\alpha}  \notin  \ottnt{t}}%
\ottpremise{\Theta  \ottsym{\{}  \,  \ottsym{\}}  \mathcolor{\ccolor}{\VDash}  \ottnt{t}  \cap  \ottsym{[}  \vec{x}  \ottsym{]}  \approxRight  \ottnt{t'}  \mathcolor{\ccolor}{\DashV}  \Theta'  \ottsym{\{}  \widehat{\alpha}  \ottsym{:}  n  \ottsym{\}}}%
}{
\Theta  \mathcolor{\ccolor}{\VDash}  \widehat{\alpha}  \ottsym{[}  \vec{x}  \ottsym{]}  \mathcolor{\rcolor}{\equiv}  \ottnt{t}  \mathcolor{\ccolor}{\DashV}   \ottsym{(}  \Theta'  \ottsym{\{}  \widehat{\alpha}  \ottsym{:}  n  \ottsym{=}  \vec{x}  \ottsym{.}  \ottnt{t'}  \ottsym{\}}  \ottsym{)} ^{\color{red}\star} }{%
{\ottdrulename{aUV\_F}}{}%
}}

\newcommand{\ottdefnaunTwo}[1]{\begin{ottdefnblock}[#1]{$\Theta_{{\mathrm{1}}}  \mathcolor{\ccolor}{\VDash}  \ottnt{v}  \mathcolor{\rcolor}{\equiv}  \ottnt{w}  \mathcolor{\ccolor}{\DashV}  \Theta_{{\mathrm{2}}}$}{\ottcom{The alternative unification}}
\ottusedrule{\ottdruleaVXXV{}}
\ottusedrule{\ottdruleaFXXF{}}
\ottusedrule{\ottdruleaUVXXUV{}}
\ottusedrule{\ottdruleaUVXXF{}}
\end{ottdefnblock}}

%% defn ext
\newcommand{\ottdruleaV}[1]{\ottdrule[#1]{%
}{
\ottmv{x} \, \ottkw{ext}}{%
{\ottdrulename{aV}}{}%
}}


\newcommand{\ottdruleaUV}[1]{\ottdrule[#1]{%
}{
\widehat{\alpha}  \ottsym{[}  \vec{x}  \ottsym{]} \, \ottkw{ext}}{%
{\ottdrulename{aUV}}{}%
}}


\newcommand{\ottdruleaBind}[1]{\ottdrule[#1]{%
\ottpremise{\ottnt{t} \, \ottkw{ext}}%
}{
\ottmv{x}  \ottsym{.}  \ottnt{t} \, \ottkw{ext}}{%
{\ottdrulename{aBind}}{}%
}}


\newcommand{\ottdruleaConstr}[1]{\ottdrule[#1]{%
\ottpremise{\vec{x}_{{\mathrm{1}}}  \cap  \ottkw{UVARGS} \, \ottnt{t_{{\mathrm{1}}}}  \ottsym{=}  \emptyset \quad \ottnt{t_{{\mathrm{1}}}} \, \ottkw{ext}}%
\ottpremise{ \cdots }%
\ottpremise{\vec{x}_{\ottmv{n}}  \cap  \ottkw{UVARGS} \, \ottnt{t_{\ottmv{n}}}  \ottsym{=}  \emptyset \quad \ottnt{t_{\ottmv{n}}} \, \ottkw{ext}}%
}{
\ottmv{f}  \ottsym{(}  \vec{x}_{{\mathrm{1}}}  \ottsym{.}  \ottnt{t_{{\mathrm{1}}}}  \ottsym{,..,}  \vec{x}_{\ottmv{n}}  \ottsym{.}  \ottnt{t_{\ottmv{n}}}  \ottsym{)} \, \ottkw{ext}}{%
{\ottdrulename{aConstr}}{}%
}}

\newcommand{\ottdefnaext}[1]{\begin{ottdefnblock}[#1]{$\ottnt{v} \, \ottkw{ext}$}{\ottcom{The external term}}
\ottusedrule{\ottdruleaV{}}
\ottusedrule{\ottdruleaUV{}}
\ottusedrule{\ottdruleaBind{}}
\ottusedrule{\ottdruleaConstr{}}
\end{ottdefnblock}}

%% defn extC
\newcommand{\ottdruleaEmpty}[1]{\ottdrule[#1]{%
}{
\cdot \, \ottkw{ext}}{%
{\ottdrulename{aEmpty}}{}%
}}


\newcommand{\ottdruleaVar}[1]{\ottdrule[#1]{%
}{
\Theta  \ottsym{,}  \ottmv{x} \, \ottkw{ext}}{%
{\ottdrulename{aVar}}{}%
}}


\newcommand{\ottdruleaUVar}[1]{\ottdrule[#1]{%
}{
\Theta  \ottsym{,}  \widehat{\alpha}  \ottsym{:}  n \, \ottkw{ext}}{%
{\ottdrulename{aUVar}}{}%
}}


\newcommand{\ottdruleaUVarInst}[1]{\ottdrule[#1]{%
\ottpremise{\ottnt{t} \, \ottkw{ext}}%
}{
\Theta  \ottsym{,}  \widehat{\alpha}  \ottsym{:}  n  \ottsym{=}  \ottnt{t} \, \ottkw{ext}}{%
{\ottdrulename{aUVarInst}}{}%
}}

\newcommand{\ottdefnaextC}[1]{\begin{ottdefnblock}[#1]{$\Theta \, \ottkw{ext}$}{\ottcom{The external environment}}
\ottusedrule{\ottdruleaEmpty{}}
\ottusedrule{\ottdruleaVar{}}
\ottusedrule{\ottdruleaUVar{}}
\ottusedrule{\ottdruleaUVarInst{}}
\end{ottdefnblock}}


\newcommand{\ottdefnsBOne}{
\ottdefnaprun{}\ottdefnaunTwo{}\ottdefnaext{}\ottdefnaextC{}}

\newcommand{\ottdefnss}{
\ottfundefnsFoo
\ottdefnsAOne
\ottdefnsBOne
}

\newcommand{\ottall}{\ottmetavars\\[0pt]
\ottgrammar\\[5.0mm]
\ottdefnss}






% \renewcommand{\ottdruleEOneNVarName}[0]{ Hello \def\@currentlabelname{FOO} \phantomsection  }
% \newcommand{\ottdrulename}[1]{\textsc{#1}}

% \renewcommand{\ottdrule}[4][]{{\displaystyle\frac{\begin{array}{l}#2\end{array}}{#3}\quad\ottdrulename{#4}%
%   }\ruleLabel{#4}{#4}}

% \renewcommand{\ottdrule}[4][]{%
%   {\displaystyle\frac{\begin{array}{l}#2\end{array}}{#3}\quad\ottdrulename{#4}}%
%   \mpr@label{\textsc{#1}}%
% }
% \DeclareDocumentCommand \redrule {m m o}{%
%   \inferrule*[vcenter,left={#3:}]{}{#1 \tred #2}
%   \mpr@label{\textsc{#3}}
% }

% \renewcommand{\ottdrulename}[2][#1]{\textsc{#1} \label{}}


% \renewcommand{\ottdruleEOneNVarName}[0]{foo }

% \makeatletter
% \renewcommand{\ottdruleEOneNVar}[1]{\ottdrule[#1]{%
%   }{
%     % \protected@edef\@currentlabelname{foo}
%     \def\@currentlabelname{foo}%
%     \phantomsection
%     \label{boo}
%     \alpha ^{-}   \eqEOne   \alpha ^{-}
%   }{%
%     {\ottdruleEOneNVarName}{}%
%   } }
% \makeatother


\newcommand{\ruleref}[1]{Rule \nameref{#1}}

% ord varset in uN = varset'

\renewcommand{\ottdruleONVarInName}[0]{(Var$_{\in}^-$)}
\renewcommand{\ottdruleONVarNInName}[0]{(Var$_{\notin}^-$)}
\renewcommand{\ottdruleONUVarName}[0]{(UVar$^-$)}
\renewcommand{\ottdruleOShiftUName}[0]{($\uparrow$)}
\renewcommand{\ottdruleOArrowName}[0]{($\rightarrow$)}
\renewcommand{\ottdruleOForallName}[0]{($\forall$)}


% ord varset in uP = varset'

\renewcommand{\ottdruleOPVarInName}[0]{(Var$_{\in}^+$)}
\renewcommand{\ottdruleOPVarNInName}[0]{(Var$_{\notin}^+$)}
\renewcommand{\ottdruleOPUVarName}[0]{(UVar$^+$)}
\renewcommand{\ottdruleOShiftDName}[0]{($\downarrow$)}
\renewcommand{\ottdruleOExistsName}[0]{($\exists$)}



% nf(N) = M
\renewcommand{\ottdruleNrmNVarName}[0]{(Var$^-$)}
\renewcommand{\ottdruleNrmNUVarName}[0]{(UVar$^-$)}
\renewcommand{\ottdruleNrmShiftUName}[0]{($\uparrow$)}
\renewcommand{\ottdruleNrmArrowName}[0]{($\rightarrow$)}
\renewcommand{\ottdruleNrmForallName}[0]{($\forall$)}

% nf(P) = Q
\renewcommand{\ottdruleNrmPVarName}[0]{(Var$^+$)}
\renewcommand{\ottdruleNrmPUVarName}[0]{(UVar$^+$)}
\renewcommand{\ottdruleNrmShiftDName}[0]{($\downarrow$)}
\renewcommand{\ottdruleNrmExistsName}[0]{($\exists$)}


% N ≈ M

\renewcommand{\ottdruleEOneNVarName}[0]{(Var$^-$$^{\eqEOne}$)}
\renewcommand{\ottdruleEOneShiftUName}[0]{($\uparrow^{\eqEOne}$)}
\renewcommand{\ottdruleEOneArrowName}[0]{($\rightarrow^{\eqEOne}$)}
\renewcommand{\ottdruleEOneForallName}[0]{($\forall^{\eqEOne}$)}

% P ≈ Q
\renewcommand{\ottdruleEOnePVarName}[0]{(Var$^+$)}
\renewcommand{\ottdruleEOneShiftDName}[0]{($\downarrow^{\eqEOne}$)}
\renewcommand{\ottdruleEOneExistsName}[0]{($\exists^{\eqEOne}$)}


\renewcommand{\ottdruleSMESupSupName}[0]{$([[≥]]\&[[≥]])$}
\renewcommand{\ottdruleSMEEqSupName}[0]{$([[≈]]\&[[≥]])$}
\renewcommand{\ottdruleSMESupEqName}[0]{$([[≥]]\&[[≈]])$}
\renewcommand{\ottdruleSMEPEqEqName}[0]{$([[≈]]\&[[≈]]^{+})$}
\renewcommand{\ottdruleSMENEqEqName}[0]{$([[≈]]\&[[≈]]^{-})$}



\begin{document}

\section{The Vanilla System}



First, we present the top-level system, which is easy to understand.

\subsection{Grammar}
\ottgrammartabular{
  \ottP\ottinterrule
  \ottN\ottinterrule
}

\subsection{Declarative Subtyping}
\ottdefnsDZero

\section{Multi-Quantified System}
\subsection{Grammar}
\ottgrammartabular{
  \ottiP\ottinterrule
  \ottiN\ottinterrule
}
\subsection{Declarative Subtyping}
\ottdefnsDOne


\subsection{Declarative Equivalence}
\ottdefnsEOne


\section{Algorithm}

\subsection{Normalization}
\subsubsection{Ordering}
\ottdefnsOrder

\subsubsection{Quantifier Normalization}
\ottdefnsNrm

% \subsection{Algorithmic Equivalence}
% \ottdefnsEOneA

\subsection{Unification}
\ottdefnsU

\subsection{Algorithmic Subtyping}
\ottdefnsA

\subsection{Unification Solution Merge}

Unification solution is represented by a list of unification solution entries.
Each entry restrict an unification variable in two possible ways: either stating
that it must be equivalent to a certain type ($[[Δ ⊢ pua :≈ iP]]$ or $[[Δ ⊢ nua :≈
iN]]$) or that it must be a (positive) supertype of a certain type ($[[Δ ⊢ pua :≥
iP]]$).

\begin{definition} [Matching Entries]
  We call two entries matching if they are restricting the same unification variable.
\end{definition}

Two matching entries can be merged in the following way:
\begin{definition} \hfill \\
\ottdefnSME\\
\end{definition}


To merge two unification solution, we merge each pair of
matching entries, and unite the results.

\begin{definition}
  $[[us1 & us2]] = \{ [[usEntry1 & usEntry2]] \mid [[usEntry1]] \in [[us1]],
  [[usEntry2]]  \in [[us2]], \text{s.t. } [[usEntry1]] \text{ matches with } [[usEntry2]] \}$
\end{definition}


\subsection{Least Upper Bound}
\ottdefnsLUB

\subsection{Antiunification}
\ottdefnsAU

\section{Proofs}

\subsection{Variable Ordering}


\begin{observation}[Ordering is deterministic]
  \label{obs:ord-deterministic}
  If $[[ord varset in iN = ordVars1]]$ and $[[ord varset in iN = ordVars2]]$ then $[[ordVars1 = ordVars2]]$.
  If $[[ord varset in iP = ordVars1]]$ and $[[ord varset in iP = ordVars2]]$ then $[[ordVars1 = ordVars2]]$.
  This way, we can use $[[ord varset in iN]]$ and as a function on $[[iN]]$,
  and $[[ord varset in iP]]$ as a function on $[[iP]]$.
\end{observation}
\begin{proof}
  By mutual structural induction on $[[iN]]$ and $[[iP]]$.
  Notice that the shape of the term $[[iN]]$ or $[[iP]]$
  uniquely determines the last used inference rule,
  and all the premises are deterministic on the input.
\end{proof}


\begin{lemma}[Soundness of variable ordering]
  \label{lemma:ord-soundness}
  Variable ordering extracts used free variables.
  \begin{itemize}
    \item[$-$] $[[ {ord varset in iN} ]] = [[varset ∩ fv iN]]$ (as sets)
    \item[$+$] $[[ {ord varset in iP} ]] = [[varset ∩ fv iP]]$ (as sets)
  \end{itemize}
\end{lemma}
\begin{proof}
  We prove it by mutual induction on 
  $[[ ord varset in iN = ordVars ]]$ and $[[ ord varset in iP = ordVars ]]$.
  The only non-trivial cases are 
  \ruleref{\ottdruleOArrowLabel} and 
  \ruleref{\ottdruleOForallLabel}.  
  \begin{caseof}
    \item \ruleref{\ottdruleOArrowLabel}  
      Then the inferred ordering judgement has shape
      $[[ord varset in iP → iN = ordVars1, (ordVars2 \ {ordVars1})]]$
      and by inversion, 
      $[[ord varset in iP = ordVars1]]$   
      and 
      $[[ord varset in iN = ordVars2]]$.

      By definition of free variables, 
      $[[varset ∩ fv iP → iN = varset ∩ fv iP ∪ varset ∩ fv iN]]$,
      and since by the induction hypothesis 
      $[[varset ∩ fv iP = {ordVars1}]]$ and
      $[[varset ∩ fv iN = {ordVars2}]]$,
      we have
      $[[varset ∩ fv iP → iN = {ordVars1} ∪ {ordVars2}]]$.

      On the other hand, 
      As a set, $[[{ordVars1} ∪ {ordVars2}]]$
      is equal to $[[ordVars1, (ordVars2 \ {ordVars1})]]$. 
    \item  \ruleref{\ottdruleOForallLabel}.
      Then  the inferred ordering judgement has shape
      $[[ord varset in ∀pas.iN = ordVars]]$,
      and by inversion, 
      $[[varset ∩ {pas} = ∅]]$    
      $[[ord varset in iN = ordVars]]$.
      The latter implies that $[[varset ∩ fv iN = {ordVars}]]$.
      We need to show that $[[varset ∩ fv ∀pas.iN = {ordVars}]]$,
      or equivalently, that
      $[[varset ∩ (fv iN \ {pas}) = varset ∩ fv iN ]]$,
      which holds since $[[varset ∩ {pas} = ∅]]$.
\end{proof}


\begin{corollary}[Additivity of ordering]
  \label{corollary:ord-additivity}
  Variable ordering is additive (in terms of set union) with respect to its first argument.
  \begin{itemize}
    \item[$-$] $[[ {ord (varset1 ∪ varset2) in iN} 
                = 
                {ord varset1 in iN} ∪ {ord varset2 in iN}]]$ (as sets)
    \item[$+$] $[[{ord (varset1 ∪ varset2) in iP}
                =
                {ord varset1 in iP} ∪ {ord varset2 in iP}]]$ (as sets)

  \end{itemize}
\end{corollary}

\begin{lemma}[Weakening of ordering]
  \label{corollary:ord-weakening}
  Only used variables matter in the first argument of the ordering,
  \begin{itemize}
    \item[$-$] $[[ ord (varset ∩ fv iN) in iN ]] = [[ ord varset in iN ]]$
    \item[$+$] $[[ ord (varset ∩ fv iP) in iP ]] = [[ ord varset in iP ]]$
  \end{itemize}
\end{lemma}
\begin{proof}
  Mutual structural induction on $[[iN]]$ and $[[iP]]$.

  \begin{caseof}
    \item If $[[iN]]$ is a variable $[[na]]$,
      we notice that $[[na ∊ varset]]$ 
      is equivalent to $[[na ∊ varset ∩ {na}]]$.
    \item If $[[iN]]$ has shape $[[↑iP]]$, then
      the required property holds immediately by the 
      induction hypothesis, since 
      $[[fv(↑iP) = fv(iP)]]$.
    \item If the term has shape $[[iP → iN]]$ then
      \ruleref{\ottdruleOArrowLabel} was applied
      to infer $[[ ord (varset ∩ (fv iP ∪ fv iN)) in iP → iN ]]$
      and $[[ ord varset in iP → iN]]$. 
      By inversion, the result of 
      $[[ ord (varset ∩ (fv iP ∪ fv iN)) in iP → iN ]]$
      depends on 
      $A = [[ ord (varset ∩ (fv iP ∪ fv iN)) in iP]]$
      and 
      $B = [[ ord (varset ∩ (fv iP ∪ fv iN)) in iN]]$.
      The result of
       $[[ ord varset in iP → iN]]$ 
       depends on 
      $X = [[ord varset in iP]]$ and
      $Y = [[ord varset in iN]]$.

      Let us show that that $A = B$ and $X = Y$, so the results are equal. 
      By the induction hypothesis and set properties,
      $[[ ord (varset ∩ (fv iP ∪ fv iN)) in iP ]] = 
       [[ ord (varset ∩ (fv iP ∪ fv iN)) ∩ fv(iP) in iP ]] = 
       [[ ord varset ∩ fv(iP) in iP ]] = 
       [[ ord varset in iP ]]$.
      Analogously, 
      $[[ ord (varset ∩ (fv iP ∪ fv iN)) in iN ]] = 
       [[ ord varset in iN ]]$.
    \item If the term has shape $[[∀pas.iN]]$,
      we can assume that $[[pas]]$ is disjoint
      from $[[varset]]$,
      since we operate on alpha-equivalence classes.
      Then using the induction hypothesis,
      set properties and \ruleref{\ottdruleOForallLabel}: 
      $[[ord varset ∩ (fv(∀pas.iN)) in ∀pas.iN]] =
       [[ord varset ∩ (fv(iN) \ {pas}) in iN]] =
       [[ord varset ∩ (fv(iN) \ {pas}) ∩ fv(iN) in iN]] =
       [[ord varset ∩ fv(iN) in iN]] =
       [[ord varset in iN]]$.
  \end{caseof}
\end{proof}

\begin{corollary}[Idempotency of ordering]
  \label{corollary:ord-idemp}
  \hfill
  \begin{itemize}
    \item[$-$] If $[[ ord varset in iN = ordVars ]]$ then 
      $[[ ord {ordVars} in iN = ordVars ]]$,
    \item[$+$] If $[[ ord varset in iP = ordVars ]]$ then 
      $[[ ord {ordVars} in iP = ordVars ]]$;
  \end{itemize}
\end{corollary}
\begin{proof}
  By \cref{lemma:ord-soundness,corollary:ord-weakening}.
\end{proof}
  

Next we make a set-theoretical observation
that will be useful further.
In general, any injective function (its image)
distributes over set intersection.
However, for convenience we allow the bijections
on variables to be applied
\emph{outside of their domains}
(as identities), which may violate
the injectivity. To deal with these cases, 
we define a special notion of
bijections collision-free on certain sets
in such a way that
a bijection that is collision-free on $P$ and $Q$,
distributes over intersection of $P$ and $Q$.

\begin{definition} [Collision free bijection]
  We say that a bijection $\mu : A \leftrightarrow B$ between sets of
  variables is \textbf{collision free on sets} $P$ and $Q$ if and only if
  \begin{enumerate}
    \item $\mu(P \cap A) \cap Q = \emptyset$
    \item $\mu(Q \cap A) \cap P = \emptyset$
  \end{enumerate}
\end{definition}

\begin{observation}
  Suppose that $\mu : A \leftrightarrow B$ is a bijection between two sets of variables,
  and $\mu$ is collision free on $P$ and $Q$.
  Then $\mu(P \cap Q) = \mu(P) \cap \mu(Q)$.
\end{observation}
  

\begin{lemma}[Distributivity of renaming over variable ordering]
  \label{lemma:distr-mu-ord}
  Suppose that $\mu$ is a bijection between two sets of variables
  $\mu : A \leftrightarrow B$.
  
  \begin{itemize}
  \item[$-$]
    If $[[mu]]$ is collision free on $[[varset]]$ and $[[fv iN]]$ then
    $[[ [mu] (ord varset in iN) ]] = [[ord ([mu] varset) in [mu] iN ]]$
  \item[$+$]
    If $[[mu]]$ is collision free on $[[varset]]$ and $[[fv iP]]$ then
    $[[ [mu] (ord varset in iP) ]] = [[ord ([mu] varset) in [mu] iP ]]$
  \end{itemize}
\end{lemma}

\begin{proof}
  Mutual induction on $[[iN]]$ and $[[iP]]$.
  \begin{caseof}
  \item $[[iN]]$ = $[[na]]$ \label{case:distr-mu-ord:var} \\
    let us consider four cases:
    \begin{caseof}
    \item $[[na]] \in A$ and $[[na]] \in [[varset]]$\\ Then
      $
      \begin{aligned}[t] [[ [mu] (ord varset in iN) ]] &= [[ [mu] (ord varset in na)]] \\
                                                             &= [[ [mu] na ]]
                                                             && \text{by \ruleref{\ottdruleOPVarInLabel}}\\
                                                             &= [[nb]]
                                                             && \text{for some $[[nb]] \in B$ (notice that $[[nb]] \in [[ [mu]varset ]]$)} \\
                                                             &= [[ ord [mu]varset in nb ]]
                                                             && \text{by \ruleref{\ottdruleOPVarInLabel},
                                                                because $[[nb]] \in [[ [mu]varset ]]$} \\
                                                             &= [[ord [mu] varset in [mu] na ]]
       \end{aligned}
       $
     \item $[[na]] \notin A$ and $[[na]] \notin [[varset]]$\\
       Notice that
       $[[ [mu] (ord varset in iN) ]] = [[ [mu] (ord varset in na)]] = [[·]]$ by
       \ruleref{\ottdruleOPVarNInLabel}.
       On the other hand, $[[ ord [mu] varset in [mu] na = ord [mu] varset
       in na ]] = [[·]]$ The latter equality is from
       \ruleref{\ottdruleOPVarNInLabel}, because
       $[[mu]]$ is collision free on $[[varset]]$ and $[[fv iN]]$, so
       $[[fv iN]] \ni [[na]] \notin [[mu]](A \cap [[varset]]) \cup
       [[varset]] \supseteq [[ [mu] varset ]]$.
     \item $[[na]] \in A$ but $[[na]] \notin [[varset]]$\\ Then
       $[[ [mu] (ord varset in iN) ]] = [[ [mu] (ord varset in na)]] = [[·]]$
       by \ruleref{\ottdruleOPVarNInLabel}.
       To prove that $[[ ord [mu] varset in [mu] na ]] = [[·]]$, we apply
       \ruleref{\ottdruleOPVarNInLabel}. Let us show that
       $[[ [mu] na ]] \notin [[ [mu] varset ]]$.
       Since $[[ [mu] na ]] = [[mu]]([[na]])$ and
       $[[ [mu] varset ]] \subseteq [[mu]](A \cap [[varset]]) \cup [[varset]]$,
       it suffices to prove 
       $[[mu]]([[na]]) \notin [[mu]](A \cap [[varset]]) \cup [[varset]]$.

       \begin{enumerate}
       \item[(i)] If there is an element $x \in A \cap [[varset]]$ such that
         $[[mu]] x = [[mu]] [[na]]$, then $x = [[na]]$ by bijectivity of
         $[[mu]]$, which contradicts with $[[na]] \notin [[varset]]$. This way, 
         $[[mu]]([[na]]) \notin [[mu]](A \cap [[varset]])$.
       \item[(ii)]
         Since $[[mu]]$ is collision free on $[[varset]]$ and $[[fv iN]]$,
         $[[mu]] (A \cap [[fv iN]]) \ni [[mu]]([[na]]) \notin [[varset]]$.
       \end{enumerate}

     \item $[[na]] \notin A$ but $[[na]] \in [[varset]]$\\
       $[[ ord [mu] varset in [mu] na ]] = [[ ord [mu] varset in na ]] = [[na]]$.
       The latter is by \ruleref{\ottdruleOPVarNInLabel}, because
       $[[na]] = [[ [mu] na ]] \in [[ [mu] varset ]]$ since $[[na]] \in [[varset]]$.
       On the other hand, $[[ [mu] (ord varset in iN) ]] = [[ [mu] (ord varset in na)]] = [[ [mu] na ]] = [[na]]$.
    \end{caseof}
  
  \item $[[iN]] = [[↑iP]]$ \\
    $\begin{aligned}[t]
       [[ [mu] (ord varset in iN) ]] &= [[ [mu] (ord varset in ↑iP) ]] \\
                                     &= [[ [mu] (ord varset in iP) ]]
                                     && \text{by \ruleref{\ottdruleOShiftULabel}}\\
                                     &= [[ ord [mu]varset in [mu]iP ]]
                                     && \text{by the induction hypothesis}\\
                                     &= [[ ord [mu]varset in  ↑[mu]iP ]]
                                     && \text{by \ruleref{\ottdruleOShiftULabel}}\\
                                     &= [[ ord [mu]varset in  [mu]↑iP ]]
                                     && \text{by the definition of substitution}\\
                                     &= [[ ord [mu]varset in  [mu]iN ]]
            \end{aligned}$
          
   \item $[[iN]] = [[iP → iM]]$  \\
     $\begin{aligned}[t]
        [[ [mu] (ord varset in iN) ]] &= [[ [mu] (ord varset in iP → iM) ]] \\
                                      &= [[ [mu] (ordVars1, (ordVars2 \ {ordVars1})) ]]
                                      && \text{where } [[ord varset in iP = ordVars1]] \text{ and } [[ord varset in iM = ordVars2]] \\
                                      &= [[ [mu] ordVars1, [mu](ordVars2 \ {ordVars1}) ]] \\
                                      &= [[ [mu] ordVars1, ([mu]ordVars2 \ [mu]{ordVars1}) ]]
                                      && \text{by induction on $[[ordVars2]]$;
                                         the inductive step is similar to \cref{case:distr-mu-ord:var}.
                                         Notice that $[[mu]]$ is} \\
                                      & && \text{collision free on $[[{ordVars1}]]$ and $[[{ordVars2}]]$
                                           since
                                           $[[{ordVars1}]] \subseteq [[varset]]$ and
                                           $[[{ordVars2}]] \subseteq [[fv iN]]$ }\\
                                      &= [[ [mu] ordVars1, ([mu]ordVars2 \ {[mu]ordVars1}) ]]
      \end{aligned}$ \\
    $\begin{aligned}[t]
       [[  (ord [mu] varset in [mu]iN) ]] &= [[ (ord [mu] varset in [mu]iP → [mu]iM) ]] \\
                                     &= [[ (ordVarsb1, (ordVarsb2 \ {ordVarsb1})) ]]
                                     && \text{where } [[ord [mu] varset in [mu] iP = ordVarsb1]] \text{ and } [[ord [mu] varset in [mu] iM = ordVarsb2]] \\
                                          & && \text{then by the induction
                                               hypothesis,
                                               $[[ordVarsb1]] = [[ [mu] ordVars1 ]]$,
                                               $[[ordVarsb2]] = [[ [mu] ordVars2 ]]$,
                                               }\\
                                     &= [[ [mu] ordVars1, ([mu]ordVars2 \ {[mu]ordVars1}) ]]
     \end{aligned}$
   
   \item $[[iN]] = [[∀ pas.iM]]$ \\
     $\begin{aligned}[t]
          [[ [mu] (ord varset in iN) ]] &= [[ [mu] ord varset in ∀pas.iM]] \\
                                        &= [[ [mu] ord varset in iM]] \\
                                        &= [[ ord [mu] varset in [mu] iM]]
                                        && \text {by the induction hypothesis}\\
     \end{aligned}$ \\
     $ 
     \begin{aligned}[t]
       [[ (ord [mu] varset in [mu] iN) ]] &= [[ ord [mu] varset in [mu] ∀pas.iM ]] \\
                                          &= [[ ord [mu] varset in ∀pas.[mu]iM ]] \\
                                          &= [[ ord [mu] varset in [mu] iM ]] \\
     \end{aligned}
     $
     
  \end{caseof}
\end{proof}

\begin{lemma}[Ordering is not affected by independent substitutions]
  \label{lemma:ord-sigma}
  Suppose that $[[Γ2 ⊢ σ : Γ1]]$, i.e. $[[σ]]$ maps variables from $[[Γ1]]$ into types
  taking free variables from $[[Γ2]]$, and $[[varset]]$ is a set of variables
  disjoint with both $[[Γ1]]$ and $[[Γ2]]$, 
  $[[iN]]$ and $[[iP]]$ are types. Then
    \begin{itemize}
  \item[$-$] $[[ ord varset in [σ]iN ]] = [[ord varset in iN ]]$
  \item[$+$] $[[ ord varset in [σ]iP ]] = [[ord varset in iP ]]$
  \end{itemize}
\end{lemma}
\begin{proof}
  Mutual induction on $[[iN]]$ and $[[iP]]$.
  \begin{caseof}
    \item $[[iN = na]]$ \\
      If $[[na ∉ Γ1]]$ then $[[ [σ]na = na ]]$ and $[[ ord varset in [σ]na ]] = [[ ord varset in na ]]$, 
      as requried.
      If $[[na ∊ Γ1]]$ then $[[na ∉ varset]]$, so $[[ ord varset in na ]] = [[·]]$.
      Moreover, $[[Γ2 ⊢ σ : Γ1]]$ means $[[ fv([σ]na) ⊆ Γ2 ]]$, and thus, 
      as a set, $[[ ord varset in [σ]na ]] = [[varset ∩ fv([σ]na)]] \subseteq [[varset ∩ Γ2]] = [[·]]$.
    \item $[[iN = ∀pas.iM]]$\\
      We can assume $[[{pas} ∩ Γ1 = ∅]]$
      and $[[{pas} ∩ varset = ∅]]$. Then 

      $\begin{aligned}[t]
         [[ ord varset in [σ]iN ]] &= [[ ord varset in [σ]∀pas.iM ]] \\
                                   &= [[ ord varset in ∀pas.[σ]iM ]]\\
                                   &= [[ ord varset in [σ]iM ]]
                                   && \text{by the induction hypothesis}\\
                                   &= [[ ord varset in iM ]]\\
                                   &= [[ ord varset in ∀pas.iM ]]\\
                                   &= [[ ord varset in iN ]]
       \end{aligned}$
    \item $[[iN = ↑iP]]$\\
       $\begin{aligned}[t]
        [[ ord varset in [σ]iN ]] &= [[ ord varset in [σ]↑iP ]] \\
                                   &= [[ ord varset in ↑[σ]iP ]]
                                   && \text{by the definition of substitution}\\
                                   &= [[ ord varset in [σ]iP ]]
                                   && \text{by the induction hypothesis}\\
                                   &= [[ ord varset in iP ]]
                                   && \text{by the definition of substitution}\\
                                   &= [[ ord varset in ↑iP ]]
                                   && \text{by the definition of ordering}\\
                                   &= [[ ord varset in iN ]]
       \end{aligned} $

    \item $[[iN = iP → iM]]$\\
       $ \begin{aligned}
        [[ ord varset in [σ]iN ]] &= [[ ord varset in [σ](iP → iM) ]] \\
                                   &= [[ ord varset in ([σ]iP → [σ]iM) ]]
                                   && \text{by the definition of substitution}\\
                                   &= [[ ord varset in [σ]iP, (ord varset in [σ]iM \ {ord varset in [σ]iP}) ]]
                                   && \text{by the definition of ordering}\\
                                   &= [[ ord varset in iP, (ord varset in iM \ {ord varset in iP}) ]]
                                   && \text{by the induction hypothesis}\\
                                   &= [[ ord varset in iP → iM ]]
                                   && \text{by the definition of ordering}\\
                                   &= [[ ord varset in iN ]]
       \end{aligned} $
    \item The proofs of the positive cases are symmetric.
  \end{caseof}
\end{proof}

\begin{lemma}[Completeness of variable ordering]
  \label{lemma:ord-completeness}
  Variable ordering is invariant under equivalence. For arbitrary $[[varset]]$,
   \begin{itemize}
  \item[$-$] If $[[iN ≈ iM]]$ then $[[ord varset in iN]] = [[ord varset in iM]]$ (as lists)
  \item[$+$] If $[[iP ≈ iQ]]$ then $[[ord varset in iP]] = [[ord varset in iQ]]$ (as lists)
  \end{itemize}
\end{lemma}
\begin{proof}
  Mutual induction on $[[iN ≈ iM]]$ and $[[iP ≈ iQ]]$.
  Let us consider the rule inferring $[[iN ≈ iM]]$. 
  \begin{caseof}
    \item \ruleref{\ottdruleEOneNVarLabel}
    \item \ruleref{\ottdruleEOneShiftULabel}
    \item \ruleref{\ottdruleEOneArrowLabel}
      Then the equivalence has shape $[[iP → iN ≈ iQ → iM]]$,
      and by inversion, $[[iP ≈ iQ]]$ and $[[iN ≈ iM]]$.
      They by the induction hypothesis,
      $[[ord varset in iP]] = [[ord varset in iQ]]$ 
      and $[[ord varset in iN]] = [[ord varset in iM]]$.
      Since the resulting ordering for $[[iP → iN]]$ and $[[iQ → iM]]$
      depend on the ordering of the corresponding components, 
      which are equal, the results are equal.
    \item \ruleref{\ottdruleEOneForallLabel}
      Then the equivalence has shape $[[∀pas.iN ≈ ∀pbs.iM]]$.
      and by inversion there exists 
      $[[mu : ({pbs} ∩ fv iM) ↔ ({pas} ∩ fv iN)]]$ such that
      \begin{itemize}
        \item $[[{pas} ∩ fv iM = ∅]]$ and 
        \item $[[iN ≈ [mu] iM]]$
      \end{itemize}

      Let us assume that $[[varset]]$ is disjoint from 
      $[[pas]]$ and $[[pbs]]$ 
      (we can always alpha-rename the bound variables).
      Then $[[ord varset in ∀pas.iN = ord varset in iN]]$, 
      $[[ord varset in ∀pas.iM = ord varset in iM]]$
      and by the induction hypothesis,
      $[[ord varset in iN]] = [[ord varset in [mu]iM]]$.
      This way, it suffices tho show  that 
      $[[ord varset in [mu]iM = ord varset in iM]]$.
      It holds by \cref{lemma:ord-sigma} since
      $[[varset]]$ is disjoint form 
      the domain and the codomain of 
      $[[mu : ({pbs} ∩ fv iM) ↔ ({pas} ∩ fv iN)]]$ 
      by assumption.

    \item The positive cases are proved symmetrically.
  \end{caseof}
\end{proof}


\subsection{Normaliztaion}

\begin{lemma}
  \label{lemma:equiv-fv}
  Set of free variables is invariant under equivalence.
  \begin{itemize}
  \item[$-$] If $[[iN ≈ iM]]$ then $[[fv iN]] \equiv [[fv iM]]$ (as sets)
  \item[$+$] If $[[iP ≈ iQ]]$ then $[[fv iP]] \equiv [[fv iQ]]$ (as sets)
  \end{itemize}
\end{lemma}
\begin{proof}
  Straightforward mutual induction on $[[iN ≈ iM]]$ and $[[iP ≈ iQ]]$
\end{proof}


\begin{lemma}
  \label{lemma:fv-nf}
  Free variables are not changed by the normalization
  \begin{itemize}
  \item[$-$] $[[fv iN]] \equiv [[fv nf(iN)]]$
  \item[$+$] $[[fv iP]] \equiv [[fv nf(iP)]]$
  \end{itemize}
\end{lemma}
\begin{proof}
  By straightforward induction on $[[iN]]$ and mutually on $[[iP]]$.
\end{proof}

\begin{lemma}
  \label{lemma:uv-nf}
  Algorithmic variables are not changed by the normalization
  \begin{itemize}
  \item[$-$] $[[uv uN]] \equiv [[uv nf(uN)]]$
  \item[$+$] $[[uv uP]] \equiv [[uv nf(uP)]]$
  \end{itemize}
\end{lemma}
\begin{proof}
  By straightforward induction on $[[uN]]$ and mutually on $[[uP]]$.
\end{proof}


\begin{lemma}[Soundness of normalization]
  \label{lemma:normalization-soundness}
  \hfill
  \begin{itemize}
    \item[$-$] $[[iN ≈ nf(iN)]]$
    \item[$+$] $[[iP ≈ nf(iP)]]$
  \end{itemize}
\end{lemma}
\begin{proof}
  Mutual induction on $[[nf(iN) = iM]]$ and $[[nf(iP) = iQ]]$.
  Let us consider how this judgment is formed:
  \begin{caseof}
    \item{\nameref{\ottdruleNrmNVarLabel} and \nameref{\ottdruleNrmPVarLabel}}\\ By
      the corresponding equivalence rules.
    \item{\nameref{\ottdruleNrmShiftULabel}, \nameref{\ottdruleNrmShiftDLabel},
        and \nameref{\ottdruleNrmArrowLabel}}\\
      By the induction hypothesis and the corresponding congruent equivalence rules.
    \item{\nameref{\ottdruleNrmForallLabel}}, i.e. $[[nf(∀pas.uN) = ∀pas'.uN']]$ \label{case:norm-soundness:forall}\\
      From the induction hypothesis, we
      know that $[[iN ≈ iN']]$. In particular, by \cref{lemma:equiv-fv}, $[[fv
        iN]] \equiv [[fv iN']]$.
      Then by \cref{lemma:ord-soundness}, $[[{pas'}]]
      \equiv [[{pas} ∩ fv iN']] \equiv [[{pas} ∩ fv iN]]$, and thus,
      $[[{pas'} ∩ fv iN']] \equiv [[{pas} ∩ fv iN]]$.
      
      To prove $[[∀pas.iN ≈ ∀pas'.iN']]$, it suffices to provide a bijection 
      $\mu : [[{pas'} ∩ fv iN']] \leftrightarrow [[{pas} ∩ fv iN]]$ such that
      $[[iN ≈ [mu]iN']]$. Since these sets are equal, we take $\mu = id$.
    \item{\nameref{\ottdruleNrmExistsLabel}} Same as for \cref{case:norm-soundness:forall}.
  \end{caseof}
\end{proof}

\begin{lemma}[Soundness of normalization of algorithmic types]
  \label{lemma:normalization-soundness-alg}
  \hfill
  \begin{itemize}
    \item[$-$] $[[uN ≈ nf(uN)]]$
    \item[$+$] $[[uP ≈ nf(uP)]]$
  \end{itemize}
\end{lemma}
\begin{proof}
  The proof coincides with the proof of \cref{lemma:normalization-soundness}.
\end{proof}


\begin{corollary}[Normalization preserves ordering]
  \label{corollary:normalization-ord}
  For any $[[varset]]$,
  \begin{itemize}
  \item[$-$] $[[ord varset in nf(uN)]] = [[ord varset in uM]]$
  \item[$+$] $[[ord varset in nf(uP)]] = [[ord varset in uQ]]$
  \end{itemize}
\end{corollary}
\begin{proof}
  Immediately from \cref{lemma:ord-completeness,lemma:normalization-soundness}.
\end{proof}

\begin{lemma}[Distributivity of normalization over substitution]
  \label{lemma:norm-subst-distr} Normalization of a term distributes over substitution.
  Suppose that $[[σ]]$ is a substitution, $[[iN]]$ and $[[iP]]$ are types. Then
    \begin{itemize}
      \item[$-$] $[[nf([σ]iN)]] = [[ [nf(σ)] nf(iN) ]]$
      \item[$+$] $[[nf([σ]iP)]] = [[ [nf(σ)] nf(iP) ]]$
  \end{itemize}
  where $[[nf(σ)]]$ means pointwise normalization: $[[ [nf(σ)] α⁻]] = [[nf([σ]α⁻)]]$.
\end{lemma}
\begin{proof}
  Mutual induction on $[[iN]]$ and $[[iP]]$.
  \begin{caseof}
    \item $[[iN]]$ = $[[na]]$ \\
      \label{case:norm-subst-distr-var}
      $[[nf([σ]iN)]] = [[ nf([σ]na) ]] = [[ [nf(σ)]na ]] $.

      $[[ [nf(σ)] nf(iN) ]] = [[ [nf(σ)] nf(na) ]] = [[ [nf(σ)] na ]] $.
    \item $[[iP]]$ = $[[pa]]$ \\
      Similar to \cref{case:norm-subst-distr-var}.
   \item If the type is formed by $[[→]]$, $[[↑]]$, or $[[↓]]$, 
     the required equality follows from the congruence of the normalization and
     substitution, and the induction hypothesis.
     For example, if $[[iN]] = [[iP → iM]]$ then \\
     $\begin{aligned}[t]
        [[nf([σ] iN)]] &= [[ nf([σ] (iP → iM)) ]] \\
                        &= [[ nf([σ]iP → [σ]iM) ]]
                        && \text{By the congruence of substitution} \\
                        &= [[ nf([σ]iP) → nf([σ]iM) ]]
                        && \text{By the congruence of normalization, i.e. \ruleref{\ottdruleNrmArrowLabel}} \\
                        &= [[ [nf(σ)]nf(iP) → [nf(σ)]nf(iM) ]]
                        && \text{By the induction hypothesis} \\
                        &= [[ [nf(σ)](nf(iP) → nf(iM)) ]]
                        && \text{By the congruence of substitution} \\
                        &= [[ [nf(σ)]nf(iP → iM) ]]
                        && \text{By the congruence of normalization} \\
                        &= [[ [nf(σ)]nf(iN) ]]
      \end{aligned}$ \\
    \item $[[iN]] = [[∀ pas.iM]]$ \label{case:norm-subst-commute} \\
      $\begin{aligned}[t]
          [[ [nf(σ)] nf(iN) ]] &= [[ [nf(σ)] nf(∀pas.iM)]] \\
                            &= [[ [nf(σ)] ∀pas'.nf(iM) ]]
                            && \text {Where $[[pas']] = [[ ord {pas} in nf(iM)]]
                               = [[ord {pas} in iM]]$
                               (the latter is by
                               \cref{corollary:normalization-ord})}\\
        \end{aligned}$ \\

      $\begin{aligned}[t]
         [[ nf([σ]iN) ]] &= [[ nf([σ] ∀pas.iM)]] \\
                          &= [[ nf(∀pas.[σ]iM) ]]
                          && \text{Assuming $[[{pas} ∩ Γ1]] = \emptyset$
                             and $[[{pas} ∩ Γ2]] = \emptyset$}\\
                          &= [[ ∀pbs.nf([σ]iM) ]]
                          && \text {Where $[[pbs]] = [[ord {pas} in nf([σ]iM)]]
                             = [[ord {pas} in [σ]iM]]$ (the latter is by
                             \cref{corollary:normalization-ord})}\\
                          &= [[ ∀pas'.nf([σ]iM) ]]
                          && \text{By \cref{lemma:ord-sigma}, $[[pbs]] = [[pas']]$
                             since $[[{pas}]]$ is disjoint with $[[Γ1]]$ and
                             $[[Γ2]]$}\\
                          &= [[ ∀pas'.[nf(σ)]nf(iM) ]]
                          && \text {By the induction hypothesis}\\
         \end{aligned}$ \\

     To show alpha-equivalence of 
     $[[ [nf(σ)] ∀pas'.nf(iM) ]]$ and $[[ ∀pas'.[nf(σ)]nf(iM) ]]$,
     we can assume that $[[{pas'} ∩ Γ1]] = \emptyset$, and $[[{pas'} ∩ Γ2]]
     = \emptyset$.

   \item $[[iP]] = [[∃ nas.iQ]]$ \\
     Same as for \cref{case:norm-subst-commute}.
  \end{caseof}
\end{proof}


\begin{corollary}[Commutativity of normalization and renaming]
  \label{lemma:norm-subst-commute} Normalization of a term commutes with renaming.
  Suppose that $\mu$ is a bijection between two sets of variables
  $\mu : A \leftrightarrow B$. Then
  \begin{itemize}
    \item[$-$] $[[nf([mu]iN)]] = [[ [mu] nf(iN) ]]$
    \item[$+$] $[[nf([mu]iP)]] = [[ [mu] nf(iP) ]]$
  \end{itemize}
\end{corollary}
\begin{proof}
  Immediately from \cref{lemma:norm-subst-distr},
  after noticing that $[[nf(mu)]] = [[mu]]$.
\end{proof}




\begin{lemma}[Completeness of quantified normalization]
  \label{lemma:normalization-completeness}
  Normalization returns the same representative for equivalent types.

  \begin{itemize}
  \item[$-$] If $[[iN ≈ iM]]$ then $[[nf(iN)]] = [[nf(iM)]]$
  \item[$+$] If $[[iP ≈ iQ]]$ then $[[nf(iP)]] = [[nf(iQ)]]$
  \end{itemize}
  (Here equality means alpha-equivalence)
\end{lemma}

\begin{proof}
  Mutual induction on $[[iN ≈ iM]]$ and $[[iP ≈ iQ]]$.
  \begin{caseof}
  \item {\nameref{\ottdruleEOneForallLabel}} \label{case:ord-completeness:forall} \\

    From the definition of the normalization,
    \begin{itemize}
      \item $[[nf(∀pas.iN)]] = [[∀pas'.nf(iN)]]$ where $[[pas']]$ is $[[ord {pas} in nf(iN)]]$
      \item $[[nf(∀pbs.iM)]] = [[∀pbs'.nf(iM)]]$ where $[[pbs']]$ is $[[ord {pbs} in nf(iM)]]$
    \end{itemize}
    Let us take $[[mu : ({pbs} ∩ fv iM) ↔ ({pas} ∩ fv iN)]]$ from the
    inversion of the equivalence judgment. Notice that from
    \cref{lemma:fv-nf,lemma:ord-soundness}, the domain and the codomain of $\mu$ can be written
    as $[[mu : {pbs'} ↔ {pas'}]]$.
    
    To show the alpha-equivalence of $[[∀pas'.nf(iN)]]$ and $[[∀pbs'.nf(iM)]]$,
    it suffices to prove that
    \begin{enumerate*}
    \item[(i)] $[[ [mu] nf(iM) ]] = [[nf(iN)]]$ and \newline
    \item[(ii)] $[[ [mu]pbs' ]] = [[pas']]$
    \end{enumerate*}.
    
    \begin{enumerate}
    \item[(i)] $[[ [mu] nf(iM) ]] = [[nf([mu]iM)]] = [[nf(iN)]]$.
      The first equality holds by \cref{lemma:norm-subst-commute}, the second---by the induction hypothesis.

    \item[(ii)] $\begin{aligned}[t] [[ [mu]pbs' ]] &= [[ [mu] ord {pbs} in nf(iM) ]]
                                                  && \text{by the definition of $[[pbs']]$ } \\
                                                  &= [[ [mu] ord ({pbs} ∩ fv iM) in nf(iM) ]]
                                                  && \text{from \cref{lemma:fv-nf,corollary:ord-weakening} } \\
                                                  &= [[ ord [mu] ({pbs} ∩ fv iM) in [mu] nf(iM) ]]
                                                  && \text{by \cref{lemma:distr-mu-ord}, because
                                                     $[[{pas} ∩ fv iN]] \cap [[fv nf(iM)]] \subseteq [[{pas} ∩ fv iM ]]
                                                     = \emptyset$}\\
                                                  &
                                                  && \text{and $[[{pas} ∩ fv iN]] \cap [[({pbs} ∩ fv iM)]] \subseteq
                                                     [[{pas} ∩ fv iM]] = \emptyset$} \\
                                                  &= [[ ord [mu] ({pbs} ∩ fv iM) in nf(iN) ]]
                                                  && \text{since $[[ [mu] nf(iM) ]] = [[nf(iN)]]$ is proved } \\
                                                  &= [[ ord ({pas} ∩ fv iN) in nf(iN) ]]
                                                  && \text{because $\mu$ is a bijection between
                                                     $[[{pas} ∩ fv iN]]$ and $[[{pbs} ∩ fv iM]]$} \\
                                                  &= [[ ord {pas} in nf(iN) ]]
                                                  && \text{from \cref{lemma:fv-nf,corollary:ord-weakening} } \\
                                                  &= [[ pas' ]]
                                                  && \text{by the definition of $[[pas']]$} \\
      \end{aligned}$
    \end{enumerate}
  \item {\nameref{\ottdruleEOneExistsLabel}} Same as for \cref{case:ord-completeness:forall}.
  \item Other rules are congruent, and thus, proved by the corresponding congruent alpha-equivalence rule,
    which is applicable by the induction hypothesis. 
  \end{caseof}
\end{proof}


\begin{lemma}[Idempotence of normalization]
  \label{lemma:norm-idemp}
  Normalization is idempotent
  \begin{itemize}
  \item[$-$] $[[nf(nf(iN))]] = [[nf(iN)]]$
  \item[$+$] $[[nf(nf(iP))]] = [[nf(iP)]]$
  \end{itemize}
\end{lemma}
\begin{proof}
  By applying \cref{lemma:normalization-completeness} to \cref{lemma:normalization-soundness}.
\end{proof}


\begin{lemma}
  \label{lemma:normal-after-subst}
  The result of a substitution is normalized if and only if the initial type and
  the substitution are normalized.

  Suppose that $[[σ]]$ is a substitution  $[[Γ2 ⊢ σ : Γ1]]$,
  $[[iP]]$ is a positive type ($[[Γ1 ⊢ iP]]$),
  $[[iN]]$ is a negative type ($[[Γ1 ⊢ iN]]$). Then
  \begin{itemize}
  \item[$+$]
    $[[ [σ]iP  ]] \text{ is normal} \iff
    \begin{cases}
      [[σ|fv(iP)]] &\text{is normal}\\
      [[iP]]       &\text{is normal}\\
    \end{cases} $
  \item[$-$]
    $[[ [σ]iN  ]] \text{is normal} \iff
    \begin{cases}
      [[σ|fv(iN)]] &\text{is normal}\\
      [[iN]]       &\text{is normal}\\
    \end{cases} $
  \end{itemize}
\end{lemma}
\begin{proof}
  Mutual induction on $[[Γ1 ⊢ iP]]$ and $[[Γ1 ⊢ iN]]$.
  \begin{caseof}
  \item $[[iN]] = [[na]]$\\
    Then $[[iN]]$ is always normal, and
    the normality of $[[σ|{na}]]$ by the definition means $[[ [σ]na ]]$ is normal.

  \item $[[iN]] = [[iP → iM]]$\\
    \label{case:normal-after-subst-arrow}
    $
    \begin{aligned}[t]
      [[ [σ](iP → iM) ]] \text{ is normal} &\iff [[ [σ]iP → [σ]iM ]] \text{ is normal}
                                           && \text{by the substitution
                                              congruence} \\
                                           &\iff
                                             \begin{cases}
                                             [[ [σ]iP ]] &\text{is normal} \\
                                             [[ [σ]iM ]] &\text{is normal} \\
                                             \end{cases}\\
                                           &\iff
                                             \begin{cases}
                                               [[ iP ]]       &\text{is normal} \\
                                               [[ σ|fv(iP) ]] &\text{is normal} \\
                                               [[ iM ]]       &\text{is normal} \\
                                               [[ σ|fv(iM) ]] &\text{is normal} \\
                                             \end{cases}
                                           && \text{by the induction hypothesis}\\
                                           &\iff
                                             \begin{cases}
                                               [[ iP → iM ]]  &\text{is normal} \\
                                               [[ σ|fv(iP) ∪ fv(iM)]] &\text{is normal} \\
                                             \end{cases}\\
                                           &\iff
                                             \begin{cases}
                                               [[ iP → iM ]]  &\text{is normal} \\
                                               [[ σ|fv(iP→iM)]] &\text{is normal} \\
                                             \end{cases}
    \end{aligned}
    $
  \item $[[iN]] = [[↑iP]]$\\
    By congruence and the inductive hypothesis, similar to \cref{case:normal-after-subst-arrow}
  \item $[[iN]] = [[∀pas.iM]]$\\
    $
    \begin{aligned}[t]
      [[ [σ](∀pas.iM) ]] \text{ is normal} &\iff [[ (∀pas.[σ]iM) ]] \text{ is normal}
                                           && \text{assuming $[[pas]] \cap [[Γ1]] = \emptyset$ and
                                              $[[pas]] \cap [[Γ2]] = \emptyset$} \\
                                           &\iff
                                             \begin{cases}
                                             [[ [σ]iM ]] \text{ is normal} \\
                                             [[ord {pas} in [σ]iM = pas]] \\
                                             \end{cases}
                                           && \text{by the definition of normalization}\\
                                           &\iff
                                             \begin{cases}
                                               [[ [σ]iM ]] \text{ is normal} \\
                                               [[ord {pas} in iM = pas]] \\
                                             \end{cases}
                                           && \text{by \cref{lemma:ord-sigma}}\\
                                           &\iff
                                             \begin{cases}
                                               [[ σ|fv(iM) ]] \text{ is normal} \\
                                               [[ iM ]] \text{ is normal} \\
                                               [[ord {pas} in iM = pas]] \\
                                             \end{cases}
                                           && \text{by the induction hypothesis}\\
                                           &\iff
                                             \begin{cases}
                                               [[ σ|fv(∀pas.iM) ]] \text{ is normal} \\
                                               [[ ∀pas.iM ]] \text{ is normal} \\
                                             \end{cases}
                                           &&
                                              \begin{aligned}
                                              &\text{since $[[fv(∀pas.iM) = fv(iM)]]$;}\\ &\text{by the definition of normalization}
                                              \end{aligned}
    \end{aligned}
    $
  \item $[[iP]] = \dots$\\
    The positive cases are done in the same way as the negative ones.

  \end{caseof}
\end{proof}





\subsection{Upper Bounds}

\obsLubDeterministic*
\begin{proof}
  The shape of $[[iP1]]$ and $[[iP2]]$ uniquely determines the rule 
  applied to infer the upper bound.
  By looking at the inference rules,
  it is easy to see that the result of the least upper bound algorithm depends on 
  \begin{itemize}
    \item the inputs of the algorithm (that is $[[iP1]]$, $[[iP2]]$, and $[[Γ]]$),
      which are fixed;
    \item the result of the anti-unification algorithm
      applied to normalized input, which is deterministic
      by \cref{obs:au-deterministic}; 
    \item the result of the recursive call, which is deterministic by the induction hypothesis.
  \end{itemize}
\end{proof}

\lemmaShapeOfSupertypes*
\begin{proof}
  By induction on $[[G ⊢ iP]]$.
  \begin{caseof}
  \item $[[iP]] = [[pb]]$\\
    Immediately from \cref{lemma:var-subt}
  \item $[[iP = ∃nbs.iP']]$\\
    Then if $[[G ⊢ iQ ≥ ∃nbs.iP']]$, then by
    \cref{lemma:quant-rule-decomposition}, $[[G, nbs ⊢ iQ ≥ iP']]$, 
    and $[[fv iQ ∩ {nbs} = ∅]]$ by the convention. The other
    direction holds by \ruleref{\ottdruleDOneExistsLabel}. This way,
    $\{[[iQ]] \mid [[G ⊢ iQ ≥ ∃nbs.iP']] \} = \{[[iQ]] \mid  [[G, nbs ⊢ iQ
    ≥ iP']] \text{ s.t. } [[fv(iQ) ∩ {nbs} = ∅]] \}$. From the induction
    hypothesis, the latter is equal to $\UB([[G, nbs ⊢ iP']])$ not using
    $[[nbs]]$, i.e. $\UB([[G ⊢ ∃nbs.iP']])$.
  \item $[[iP = ↓iM]]$\\
    Then let us consider two subcases upper bounds without outer quantifiers (we
    denote the corresponding set restriction as $|_{\not\exists}$) and upper
    bounds with outer quantifiers ($|_{\exists}$). We prove that for both of
    these groups, the restricted sets are equal.
    % ∃a.P(f a) <=> ∃b∊Im(f).P(b)

    \begin{caseof}
      \item \label{case:sup-shape-down-zero}
      $[[iQ]] \neq [[∃nbs.iQ']]$\\
      Then the last applied rule to infer
      $[[G ⊢ iQ ≥ ↓iM]]$ must be \ruleref{\ottdruleDOneShiftDLabel},
      which means $[[iQ]] = [[↓iM']]$, and by inversion, $[[G ⊢ iM' ≈ iM]]$,
      then by \cref{lemma:equiv-completeness} and
      \ruleref{\ottdruleEOneShiftDLabel}, $[[↓iM' ≈ ↓iM]]$.
      This way, $[[iQ]] = [[↓iM']] \in \{ [[↓iM']] \mid [[↓iM' ≈ ↓iM]] \} = \UB([[Γ⊢↓iM]])|_{\not\exists}$.

      In the other direction,
      $
      \begin{aligned}[t]
        [[↓iM' ≈ ↓iM]] &\Rightarrow [[G ⊢ ↓iM' ≈ ↓iM]]
                       && \text{by \cref{lemma:equiv-soundness,lemma:wf-equiv}}\\
                       &\Rightarrow [[G ⊢ ↓iM' ≥ ↓iM]]
                       && \text{by inversion}
      \end{aligned}
      $
      \item $[[iQ]] = [[∃nbs.iQ']]$ (for non-empty $[[nbs]]$)\\
        Then the last rule applied to infer $[[G ⊢ ∃nbs.iQ' ≥ ↓iM]]$
        must be \ruleref{\ottdruleDOneExistsLabel}.
        Inversion of this rule gives us $[[G ⊢ [iNs/nbs]iQ' ≥ ↓iM]]$
        for some $[[G ⊢ iNi]]$. Notice that $[[ [iNs/nbs]iQ' ]]$ has no outer
        quantifiers. Thus from \cref{case:sup-shape-down-zero},
        $[[ [iNs/nbs]iQ' ≈ ↓iM ]]$, which is only possible if $[[iQ']] = [[↓iM']]$.
        This way, $[[iQ]] = [[∃nbs.↓iM']] \in \UB([[Γ⊢↓iM]])|_{\exists}$ (notice
        that $[[nbs]]$ is not empty).

        In the other direction,

        $
        \begin{aligned}[t]
          [[ [iNs/nbs]↓iM' ≈ ↓iM]] &\Rightarrow [[G ⊢ [iNs/nbs] ↓iM' ≈ ↓iM]]
          && \text{by \cref{lemma:equiv-soundness,lemma:wf-equiv}}\\
                                  &\Rightarrow [[G ⊢ [iNs/nbs]↓iM' ≥ ↓iM]]
         && \text{by inversion}\\
                                  &\Rightarrow [[G ⊢ ∃nbs.↓iM' ≥ ↓iM]] 
         && \text{by \ruleref{\ottdruleDOneExistsLabel}}\\
        \end{aligned}
        $
    \end{caseof}
    
  \end{caseof}
\end{proof}

\lemmaShapeOfNormalizedSupertypes*
\begin{proof}
  By induction on $[[G ⊢ iP]]$.
  \begin{caseof}
  \item $[[iP]] = [[pb]]$\\
    Then from \cref{lemma:shape-of-supertypes},
    $\{[[nf(iQ)]]\ \mid \ [[G ⊢ iQ ≥ pb]] \} = \{[[ nf(∃nas.pb) ]] \ \mid \
    \text{for some }[[nas]]\}  = \{[[pb]]\}$ 
  \item $[[iP = ∃nbs.iP']]$\\
    $
    \begin{aligned}[t]
      & \NFUB([[Γ ⊢ ∃nbs.iP']]) \\
                              &= \NFUB([[Γ, nbs ⊢ iP']]) \text{ not using $[[nbs]]$}\\
                              &= \{ [[nf(iQ)]] \mid [[Γ, nbs ⊢ iQ ≥ iP']]  \}
                                \text{ not using $[[nbs]]$}
                              && \text{by the induction hypothesis}\\
                              &= \{ [[nf(iQ)]] \mid [[Γ, nbs ⊢ iQ ≥ iP']]
                                \text{ s.t. $[[fv iQ]] \cap [[nbs]] = \emptyset$}
                                \}
                             && \text{$[[fv nf(iQ)]] = [[fv iQ]]$ by \cref{lemma:fv-nf}}\\
                              &= \{ [[nf(iQ)]] \mid [[iQ]] \in \UB([[Γ, nbs ⊢ iP']]) \text{ s.t. $[[fv iQ]] \cap [[nbs]] = \emptyset$}
                                \}
                            && \text{by \cref{lemma:shape-of-supertypes}}\\
                              &= \{ [[nf(iQ)]] \mid [[iQ]] \in \UB([[Γ ⊢ ∃nbs.iP']])
                                \}
                              && \text{by the definition of $\UB{}$}\\
                              &= \{ [[nf(iQ)]] \mid [[Γ ⊢ iQ ≥ ∃nbs.iP']]
                                \}
                              && \text{by \cref{lemma:shape-of-supertypes}}\\
    \end{aligned}
    $
  
  \item $[[iP = ↓iM]]$ Let us prove the set equality by two inclusions.
  \begin{itemize}
    \item [$\subseteq$]
      Suppose that $[[Γ ⊢ iQ ≥ ↓iM]]$ and $[[iM]]$ is normalized.

      By \cref{lemma:shape-of-supertypes},
      $[[iQ]] \in \UB([[Γ ⊢ ↓iM]])$.
      Then by definition of $\UB{}$,
      $[[iQ = ∃nas.↓iM']]$ 
      for some $[[nas]]$, $[[iM']]$, and $[[Γ ⊢ σ :{nas}]]$ s.t.  
      $[[ [σ] ↓iM' ≈ ↓iM ]]$.

      We need to show that $[[nf(iQ)]] \in \NFUB([[Γ ⊢ ↓iM]])$.
      Notice that $[[nf(iQ)]] = [[nf(∃nas.↓iM')]] = [[∃nas0.↓iM0]]$, 
      where $[[nf(iM') = iM0]]$ and $[[ord {nas} in iM0 = nas0]]$.

      The belonging of $[[∃nas0.↓iM0]]$ to $\NFUB([[Γ ⊢ ↓iM]])$ means that
      \begin{enumerate}
        \item $[[ord {nas0} in iM0 = nas0]]$ and
        \item that there exists $[[Γ ⊢ σ0 :{nas0}]]$ such that 
          $[[ [σ0] ↓iM0 = ↓iM ]]$.
      \end{enumerate}
      The first requirement holds by \cref{corollary:ord-idemp}.
      To show the second requirement, we construct
      $[[σ0]]$ as $[[nf(σ|fv iM')]]$.
      Let us show the required properties of $[[σ0]]$:
      \begin{enumerate}
        \item $[[Γ ⊢ σ0 :{nas0}]]$. 
          Notice that by \cref{lemma:subst-restr-sig},
          $[[Γ ⊢ σ|fv(iM') :  {nas} ∩ fv(iM')]]$, 
          which we rewrite as 
          $[[Γ ⊢ σ|fv(iM') :{nas0}]]$ 
          (since by \cref{lemma:ord-soundness}
          $[[{nas0} = {nas} ∩ fv iM0]]$ as sets, 
          and $[[fv(iM0) = fv(iM')]]$ by \cref{lemma:fv-nf}).
          Then by \cref{lemma:norm-subst-sig}, 
          $[[Γ ⊢ nf(σ|fv(iM')) :{nas0}]]$,
          that is $[[Γ ⊢ σ0 :{nas0}]]$.
        \item $[[ [σ0] ↓iM0 = ↓iM ]]$.
          $[[ [σ] ↓iM' ≈ ↓iM ]]$ means 
          $[[ [σ|fv(iM')]↓iM' ≈ ↓iM ]]$ by 
          \cref{lemma:subst-restr-fv}.
          Then by \cref{lemma:decl-equiv-algorithmization}, 
          $[[ nf([σ|fv(iM')]↓iM') = nf(↓iM) ]]$, 
          implying $[[ [σ0] ↓iM0 = nf(↓iM) ]]$
          by \cref{lemma:norm-subst-distr},
          and further $[[ [σ0] ↓iM0 = ↓iM ]]$
          by \cref{lemma:norm-idemp} (since $[[↓iM]]$ is normal by assumption).
      \end{enumerate}
    
    \item [$\supseteq$]
      Suppose that a type belongs to $\NFUB([[Γ ⊢ ↓iM]])$ for a normalized 
      $[[↓iM]]$.
      Then it must have shape $[[∃nas0.↓iM0]]$ for some $[[nas0]]$, $[[iM0]]$,
      and $[[Γ ⊢ σ0 :{nas0}]]$ such that 
      $[[ ord {nas0} in iM0 = nas0 ]]$ and $[[ [σ0] ↓iM0 = ↓iM ]]$.
      It suffices to show that 
      \begin{enumerate*}
        \item $[[∃nas0.↓iM0]]$ is normalized itself, and 
        \item $[[Γ ⊢ ∃nas0.↓iM0 ≥ ↓iM]]$.
      \end{enumerate*}

      \begin{enumerate}
        \item By definition, 
          $[[nf(∃nas0.↓iM0) = ∃nas1.↓iM1]]$, 
          where $[[iM1 = nf(iM0)]]$ and \\
          $[[ord {nas0} in iM1 = nas1]]$.
          First, notice that by 
          \cref{lemma:ord-completeness,lemma:normalization-soundness}, 
          $[[ord {nas0} in iM1]] = [[ord {nas0} in iM0]] = [[nas0]]$. 
          This way, $[[nf(∃nas0.↓iM0) = ∃nas0.↓nf(iM0)]]$.
          Second, $[[iM0]]$ is normalized by \cref{lemma:normal-after-subst}, 
          since $[[ [σ0] ↓iM0 = ↓iM ]]$ is normal. 
          As such,\\ $[[nf(∃nas0.↓iM0) = ∃nas0.↓iM0]]$, 
          in other words, $[[∃nas0.↓iM0]]$ is normalized.
        \item $[[Γ ⊢ ∃nas0.↓iM0 ≥ ↓iM]]$ holds immediately by 
          \ruleref{\ottdruleDOneExistsLabel} with the substitution 
          $[[σ0]]$. Notice that $[[Γ ⊢ [σ0]↓iM0 ≥ ↓iM]]$
          follows from $[[ [σ0] ↓iM0 = ↓iM ]]$
          by reflexivity of subtyping (\cref{lemma:subtyping-reflexivity}).
      \end{enumerate}



    \end{itemize}
  \end{caseof}
\end{proof}

\lemUbContextIrrelevant*
\begin{proof}
  We prove both inclusions by structural induction on  
  $[[iP]]$.
  \begin{caseof}
    \item $[[iP]] = [[pb]]$
      Then $\UB([[Γ1 ⊢ pb]]) = \UB([[Γ2 ⊢ pb]]) = 
      \{[[∃nas.pb]] \mid \text{for some }[[nas]]\}$.
      $\NFUB([[Γ1 ⊢ pb]]) = \NFUB([[Γ2 ⊢ pb]]) = \{[[pb]]\}$.
    \item $[[iP]] = [[∃nbs.iP']]$.
      Then $\UB([[Γ1 ⊢ ∃nbs.iP']]) = \UB([[Γ1, nbs ⊢ iP']])$ not using $[[nbs]]$.
      $\UB([[Γ2 ⊢ ∃nbs.iP']]) = \UB([[Γ2, nbs ⊢ iP']])$ not using $[[nbs]]$.
      By the induction hypothesis, $\UB([[Γ1, nbs ⊢ iP']]) = \UB([[Γ2, nbs ⊢ iP']])$,
      and if we restrict these sets to the same domain, they stay equal.
      Analogously, $\NFUB([[Γ1 ⊢ ∃nbs.iP']]) = \NFUB([[Γ2 ⊢ ∃nbs.iP']])$.
    \item $[[iP]] = [[↓iM]]$.
      Suppose that $[[∃nas.↓iM']] \in \UB([[Γ1 ⊢ ↓iM]])$. It means that 
      $[[Γ1, nas ⊢ iM']]$ and there exist $[[Γ1 ⊢ iNs]]$ s.t. 
      $[[ [iNs/nas] ↓iM' ≈ ↓iM ]]$, or in other terms, 
      there exists $[[Γ1 ⊢ σ :{nas}]]$ such that $[[ [σ] ↓iM' ≈ ↓iM ]]$.

      We need to show that $[[∃nas.↓iM']] \in UB([[Γ2 ⊢ ↓iM]])$,  
      in other words, $[[Γ2, nas ⊢ iM']]$ and there exists
      $[[Γ2 ⊢ σ0 :{nas}]]$ such that $[[ [σ0] ↓iM' ≈ ↓iM ]]$.

      First, let us show $[[Γ2, nas ⊢ iM']]$. 
      Notice that $[[ [σ] ↓iM' ≈ ↓iM ]]$ implies $[[ fv([σ]iM') = fv(↓iM) ]]$ 
      by \cref{lemma:equiv-fv}. By \cref{lemma:subst-fv},
      $[[ fv(iM') \ {nas} ⊆ fv([σ]iM') ]]$. This way, 
      $[[ fv(iM') \ {nas} ⊆ fv(iM) ]]$,
      implying $[[ fv(iM') ⊆ fv(iM) ∪ {nas} ]]$.
      By \cref{lemma:wf-soundness}, $[[Γ2 ⊢ ↓iM]]$ implies $[[fv iM ⊆ Γ2]]$,
      hence, $[[fv iM' ⊆ (Γ2, nas)]]$, which by \cref{corollary:wf-ctxt-strengthening}
      means $[[Γ2, nas ⊢ iM']]$.
      
      Second, let us construct the required $[[σ0]]$ in the following way:
      $$
      \begin{cases}
          [[ [σ0]αi⁻ = [σ]αi⁻  ]] & \text{for $[[αi⁻ ∊ {nas} ∩ fv(iM')]]$ }\\
          [[ [σ0]αi⁻ = ∀γ⁺.↑γ⁺ ]] & \text{for $[[αi⁻ ∊ {nas} \ fv(iM')]]$ }\\
          [[ [σ0]γ±  = γ± ]]      & \text{for any other $[[γ±]]$ }\\
      \end{cases}
      $$
      This construction of a substitution coincides with 
      the one from the proof of \cref{lemma:subt-ctxt-irrelevance}.
      This way, for $[[σ0]]$, hold the same properties:
      \begin{enumerate}
        \item $[[ [σ0]iM' = [σ]iM' ]]$,
          which in particular, implies $[[ [σ0]↓iM = [σ]↓iM ]]$,
          and thus, $[[ [σ]↓iM' ≈ ↓iM ]]$ can be rewritten to
          $[[ [σ0]↓iM' ≈ ↓iM ]]$; and
        \item $[[  fv([σ]iM') ⊢ σ0 :{nas}]]$,
          which, as noted above, can be rewritten to 
          $[[  fv(iM) ⊢ σ0 :{nas}]]$,
          and since $[[fv iM ⊆ Γ2]]$, 
          weakened to $[[ Γ2 ⊢ σ0 :{nas}]]$.
      \end{enumerate}

      The proof of $\NFUB([[Γ1 ⊢ ↓iM]]) \subseteq \NFUB([[Γ2 ⊢ ↓iM]])$
      is analogous.
      The differences are:
      \begin{enumerate}
        \item $[[ord {nas} in iM' = nas]]$ holds by assumption, 
        \item $[[ [σ] ↓iM' = ↓iM ]]$ implies $[[ fv([σ]iM') = fv(↓iM) ]]$ by rewriting,
        \item $[[ [σ] ↓iM' = ↓iM ]]$ and $[[ [σ0]↓iM = [σ]↓iM ]]$
          imply $[[ [σ0] ↓iM' = ↓iM ]]$ by rewriting.
      \end{enumerate}
  \end{caseof}
\end{proof}

\lemmaLubSoundness*
\begin{proof}
  Induction on $[[Γ ⊨ iP1 ∨ iP2 = iQ]]$.
  \begin{caseof}
  \item $[[Γ ⊨ pa ∨ pa = pa]]$\\
     Then $[[Γ ⊢ pa]]$ by assumption, and
     $[[Γ ⊢ pa ≥ pa]]$ by \ruleref{\ottdruleDOnePVarLabel}.
   \item $[[Γ ⊨ ∃nas.iP1 ∨ ∃nbs.iP2 = iQ]]$\\
     Then by inversion of $[[Γ ⊢ ∃nas.iPi]]$  and
     weakening, $[[Γ, {nas}, {nbs} ⊢ iPi]]$, hence, the induction
     hypothesis applies to $[[Γ, {nas}, {nbs} ⊨ iP1 ∨ iP2 = iQ]]$. Then
     \begin{itemize}
       \item[(i)] $[[Γ, {nas}, {nbs} ⊢ iQ]]$,
       \item[(ii)] $[[Γ, {nas}, {nbs} ⊢ iQ ≥ iP1]]$,
       \item[(iii)] $[[Γ, {nas}, {nbs} ⊢ iQ ≥ iP2]]$.
     \end{itemize}

     To prove $[[Γ ⊢ iQ]]$, it suffices to show that
     $[[fv(iQ) ∩ (Γ, {nas}, {nbs})]] = [[fv(iQ) ∩ Γ]]$ (and then apply \cref{lemma:wf-ctxt-equiv}).
     The inclusion right-to-left is self-evident. To show
     $[[fv(iQ) ∩ (Γ, {nas}, {nbs})]] \subseteq [[fv(iQ) ∩ Γ]]$, we prove that 
     $[[fv(iQ)]] \subseteq [[Γ]]$.

     $
     \begin{aligned}[t]
       [[fv(iQ)]] &\subseteq [[fv iP1 ∩ fv iP2]]
                    && \text{by \cref{lemma:fv-propagation}}\\
                  &\subseteq [[((Γ, nas) \ {nbs}) ∩ ((Γ, nbs) \ {nas})]]
                    && \text{since } [[Γ ⊢ ∃nas.iP1]],~ [[fv(iP1)]]
                        \subseteq [[(Γ, nas)]] = \\ 
                  & && [[(Γ, nas) \ {nbs}]]
                        \text{(the latter is because by the}\\ 
                  & && \text{Barendregt's convention, $[[(Γ, nas) ∩ {nbs} = ∅]]$)}\\
                  & && \text{similarly, $[[fv(iP2)]] \subseteq [[(Γ, nbs) \ {nas}]]$}\\
                  &\subseteq [[Γ]]
     \end{aligned}
     $

     To show $[[Γ ⊢ iQ ≥ ∃nas.iP1]]$, we apply
     \ruleref{\ottdruleDOneExistsLabel}.
     Then $[[Γ, nas ⊢ iQ ≥ iP1]]$ holds since
     $[[Γ, {nas}, {nbs} ⊢ iQ ≥ iP1]]$ (by the induction hypothesis),
     $[[Γ, nas ⊢ iQ]]$ (by weakening), and $[[Γ, nas ⊢ iP1]]$.

     Judgment $[[Γ ⊢ iQ ≥ ∃nbs.iP2]]$ is proved symmetrically.
  \item $[[Γ ⊨ ↓iN ∨ ↓iM = ∃nas.[nas / ToList Ξ]uP]]$.
    By the inversion, $[[G,· ⊨ nf(↓iN) ≈au nf(↓iM) ⫤ (Ξ, uP, aus1, aus2)]]$.
    Then by the soundness of anti-unification (\cref{lemma:au-soundness}),
    \begin{enumerate}
    \item[(i)] $[[Γ ; Ξ ⊢ uP]]$, then
      by \cref{lemma:var-dealgo-wf},
      \begin{equation} \label{fact:nas-uP-is-wf} [[Γ, nas ⊢ [nas / ToList Ξ]uP]] \end{equation}
    \item[(ii)] $[[Γ ; · ⊢ aus1 : Ξ]]$ and $[[Γ ; · ⊢ aus2 : Ξ]]$.
      Assuming that $[[Ξ]] = [[nub1,..,nubn]]$,
      the antiunification solutions $[[aus1]]$ and $[[aus2]]$ can be
      put explicitly as $[[aus1]] = [[(nub1 :≈ iN1,..,nubn :≈ iNn)]]$,
      and $[[aus2]] = [[(nub1 :≈ iM1,..,nubn :≈ iMn)]]$.
      Then
      \begin{equation}
        \label{fact:aus1-is-compose}
        [[ aus1 ]] = [[ (iNs / nas) ○ (nas / ToList Ξ) ]] 
      \end{equation}
      \begin{equation}
        \label{fact:aus2-is-compose}
        [[ aus2 ]] = [[ (iMs / nas) ○ (nas / ToList Ξ) ]]
      \end{equation}
    \end{enumerate}
  \item[(iii)] $[[ [aus1] uQ = iP1 ]]$ and $[[ [aus2] uQ = iP1 ]]$,
    which, by \ref{fact:aus1-is-compose} and \ref{fact:aus2-is-compose},
    means
    \begin{equation}
      \label{fact:sub-sub-uP-iN}
      [[ [iNs / nas][nas / ToList Ξ]uP = nf(↓iN) ]]
    \end{equation}
    \begin{equation}
      \label{fact:sub-sub-uP-iM}
      [[ [iMs / nas][nas / ToList Ξ]uP = nf(↓iM) ]]
    \end{equation}

    Then $[[Γ ⊢ ∃nas.[nas / ToList Ξ]uP]]$
    follows directly from \ref{fact:nas-uP-is-wf}.

    To show $[[Γ ⊢ ∃nas.[nas / ToList Ξ]uP ≥ ↓iN]]$,
    we apply \ruleref{\ottdruleDOneExistsLabel},
    instantiating $[[nas]]$ with $[[iNs]]$.
    Then $[[Γ ⊢ [iNs / nas][nas / ToList Ξ]uP ≥ ↓iN ]]$ follows
    from \ref{fact:sub-sub-uP-iN} and 
    since $[[Γ ⊢ nf(↓iN) ≥ ↓iN]]$ (by \cref{corollary:nf-sound-wrt-subt-equiv}).

    Analogously, instantiating $[[nas]]$ with $[[iMs]]$,
    gives us $[[Γ ⊢ [iMs / nas][nas / ToList Ξ]uP ≥ ↓iM ]]$
    (from \ref{fact:sub-sub-uP-iM}), and hence,
    $[[Γ ⊢ ∃nas.[nas / ToList Ξ]uP ≥ ↓iM]]$.

  \end{caseof}

\end{proof}

\lemmaLubCompleteness*
\begin{proof}
  Induction on the pair $([[iP1]], [[iP2]])$.
  From \cref{lemma:shape-supertypes-norm},
  $[[iQ]] \in \UB([[Γ ⊢ iP1]]) \cap \UB([[Γ ⊢ iP2]])$.
  Let us consider the cases of what $[[iP1]]$ and $[[iP2]]$ are (i.e. the last
  rules to infer $[[Γ ⊢ iPi]]$).
  \begin{caseof}
    \item $[[iP1]] = [[∃nbs1.iQ1]]$, $[[iP2]] = [[∃nbs2.iQ2]]$, where either
      $[[nbs1]]$ or $[[nbs2]]$ is not empty\\
      \label{case:ub-completeness-exists}

      Then\\
      $
      \begin{aligned}[t]
        [[iQ]] &\in         \UB([[Γ ⊢ ∃nbs1.iQ1]]) \cap \UB([[Γ ⊢ ∃nbs2.iQ2]]) \\
              & \subseteq  \UB([[Γ, nbs1 ⊢ iQ1]]) \cap \UB([[Γ, nbs2 ⊢ iQ2]])
              && \text{definition of $\UB{}$}\\
              & =  \UB([[Γ, {nbs1}, {nbs2} ⊢ iQ1]]) \cap \UB([[Γ, {nbs1}, {nbs2} ⊢ iQ2]])
              && \text{by \cref{observation:ub-context-irrelevant}}\\
              & = \{[[iQ']]\ \mid \ [[Γ, {nbs1}, {nbs2}  ⊢ iQ' ≥ iQ1]] \} \cap
                  \{[[iQ']]\ \mid \ [[Γ, {nbs1}, {nbs2}  ⊢ iQ' ≥ iQ2]] \}
              && \text{by \cref{lemma:shape-of-supertypes}}\\
      \end{aligned}
      $\\
      It means that $[[Γ, {nbs1}, {nbs2} ⊢ iQ ≥ iQ1]]$ and $[[Γ, {nbs1}, {nbs2} ⊢ iQ ≥ iQ2]]$. 
      Then the next step of the algorithm---the recursive call 
      $[[Γ, {nbs1}, {nbs2} ⊨ iQ1 ∨ iQ2 = iQ']]$
      terminates by the induction hypothesis, 
      and moreover, $[[ Γ, {nbs1}, {nbs2} ⊢ iQ ≥ iQ' ]]$.
      This way, the result of the algorithm is $[[iQ']]$, i.e.
      $[[Γ ⊨ iP1 ∨ iP2 = iQ']]$.

      Since both $[[iQ]]$ and $[[iQ']]$ are sound upper bounds,
      $[[Γ ⊢ iQ]]$ and $[[Γ ⊢ iQ']]$, and therefore,
      $[[ Γ, {nbs1}, {nbs2} ⊢ iQ ≥ iQ' ]]$ can be strengthened to
      $[[ Γ ⊢ iQ ≥ iQ' ]]$ by \cref{lemma:subt-ctxt-irrelevance}.

    \item $[[iP1]] = [[pa]]$ and $[[iP2]] = [[↓iN]]$\\
      \label{case:ub-completeness-unmatching}
      Then the set of common upper bounds of $[[↓iN]]$ and $[[pa]]$
      is empty, and thus, $[[iQ]] \in \UB([[Γ ⊢ iP1]]) \cap \UB([[Γ ⊢ iP2]])$
      gives a contradiction:\\
      $
      \begin{aligned}[t]
        [[iQ]] &\in         \UB([[Γ ⊢ pa]]) \cap \UB([[Γ ⊢ ↓iN]]) \\
              & = \{[[∃nas.pa]]\  \mid \cdots \} \cap
                  \{[[∃nbs.↓iM']]\ \mid \cdots \}
              && \text{by the definition of $\UB{}$}\\
              & = \emptyset
              && \text{since $[[pa]] \neq [[↓iM']]$ for any $[[iM']]$}\\
      \end{aligned}
      $
    \item $[[iP1]] = [[↓iN]]$ and $[[iP2]] = [[pa]]$\\
      Symmetric to \cref{case:ub-completeness-unmatching}

    \item $[[iP1]] = [[pa]]$ and $[[iP2]] = [[pb]]$ (where $[[pb]] \neq [[pa]]$)\\
      Similarly to \cref{case:ub-completeness-unmatching},
      the set of common upper bounds is empty, which leads to the contradiction:

      $
      \begin{aligned}[t]
      [[iQ]] &\in         \UB([[Γ ⊢ pa]]) \cap \UB([[Γ ⊢ pb]]) \\
            & = \{[[∃nas.pa]]\  \mid \cdots \} \cap
                \{[[∃nbs.pb]]\ \mid \cdots \}
            && \text{by the definition of $\UB{}$}\\
            & = \emptyset
            && \text{since $[[pa]] \neq [[pb]]$}
      \end{aligned}
      $
    \item $[[iP1]] = [[pa]]$ and $[[iP2]] = [[pa]]$\\
      Then the algorithm terminates in one step (\ruleref{\ottdruleLUBVarLabel})
      and the result is $[[pa]]$, i.e. $[[G ⊨ pa ∨ pa = pa]]$.

      Since $[[iQ]] \in \UB([[Γ ⊢ pa]])$,
      $[[iQ]] = [[∃nas.pa]]$.
      Then $[[Γ ⊢ ∃nas.pa ≥ pa]]$ by \ruleref{\ottdruleDOneExistsLabel}:
      $[[nas]]$ can be instantiated with arbitrary negative types (for example
      $[[∀β⁺.↑β⁺]]$), since the substitution for unused variables does not change the term
      $[[ [iNs/nas]pa]] = [[pa]]$,
      and then $[[Γ ⊢ pa ≥ pa]]$ by \ruleref{\ottdruleDOnePVarLabel}.

    \item $[[iP1]] = [[↓iM1]]$ and $[[iP2]] = [[↓iM2]]$ \label{case:ub-completeness-shift}\\
      Then on the next step, the algorithm tries to anti-unify $[[nf(↓iM1)]]$ and
      $[[nf(↓iM2)]]$. By \cref{lemma:au-completeness}, to show that the
      anti-unification algorithm terminates, it suffices to
      demonstrate that a sound anti-unification solution exists.

      Notice that

      $
      \begin{aligned}[t]
        [[nf(iQ)]] &\in \NFUB([[Γ ⊢ nf(↓iM1)]]) \cap \NFUB([[Γ ⊢ nf(↓iM2)]]) \\
              &= \NFUB([[Γ ⊢ ↓nf(iM1)]]) \cap \NFUB([[Γ ⊢ ↓nf(iM2)]]) \\
              &=           \begin{array}{l}
                              \Set{ [[ ∃nas.↓iM' ]] \ | \begin{array}{l}
                                                          \text{for $[[nas]]$, $[[iM']]$, and $[[iNs]]$ s.t. $[[ord {nas} in iM' = nas]]$,}\\
                                                          \text{$[[G ⊢ iNi]]$, $[[G,nas ⊢ iM']]$,  and $[[ [iNs/nas] ↓iM' = ↓nf(iM1) ]]$}
                                                        \end{array}}\\ \cap\\
                              \Set{ [[ ∃nas.↓iM' ]] \ | \begin{array}{l}
                                                          \text{for $[[nas]]$, $[[iM']]$, and $[[iNs]]$ s.t. $[[ord {nas} in iM' = nas]]$,}\\
                                                          \text{$[[G ⊢ iNs1]]$,
                                                          $[[G ⊢ iNs2]]$, $[[G,nas ⊢ iM']]$,  and $[[ [iNs/nas] ↓iM' = ↓nf(iM2) ]]$}
                                                        \end{array}}
                            \end{array}\\
                &=
                  \Set{ [[ ∃nas.↓iM' ]] \ | \begin{array}{l}
                                              \text{for $[[nas]]$, $[[iM']]$,
                                              $[[iNs1]]$ and $[[iNs2]]$ s.t. $[[ord {nas} in iM' = nas]]$,}\\
                                              \text{$[[G ⊢ iNs1]]$, $[[G ⊢ iNs2]]$, $[[G,nas ⊢ iM']]$,
                                              $[[ [iNs1/nas] ↓iM' = ↓nf(iM1)]]$}\\
                                              \text{, and $[[ [iNs2/nas] ↓iM' = ↓nf(iM2)]]$ }
                                            \end{array}}\\
      \end{aligned}
      $\\
      The fact that the latter set is non-empty means that there exist $[[nas]],
      [[iM']]$, $[[iNs1]]$ and $[[iNs2]]$ such that
      \begin{enumerate}
      \item[(i)] $[[G,nas ⊢ iM']]$ (notice that $[[iM']]$ is normal)
      \item[(ii)] $[[G ⊢ iNs1]]$ and $[[G ⊢ iNs1]]$,
      \item[(iii)] $[[ [iNs1/nas] ↓iM' = ↓nf(iM1)]]$ and $[[ [iNs2/nas] ↓iM' = ↓nf(iM2)]]$
      \end{enumerate}

      For each negative variable $[[α⁻]]$ from $[[nas]]$, let us choose a
      fresh negative anti-unification variable $[[α̂⁻]]$, and denote the
      list of these variables as $[[nuas]]$.
      Let us show that\\ $([[nuas]],~ [[ [nuas/nas]↓iM' ]],~ [[iNs1/nuas]],~ [[iNs2/nuas]])$ is a
      sound anti-unifier of $[[nf(↓iM1)]]$ and $[[nf(↓iM2)]]$ in context $[[Γ]]$:

      \begin{itemize}
        \item $[[nuas]]$ is negative by construction,
        \item $[[Γ ; {nuas} ⊢ [nuas/nas]↓iM']]$ because $[[Γ, nas ⊢ ↓iM']]$ 
        (\cref{lemma:var-algo-wf}),
        \item $[[Γ ; · ⊢ (iNs1/nuas) :{nuas}]]$ because $[[Γ ⊢ iNs1]]$ and
          $[[Γ ; · ⊢ (iNs2/nuas) :{nuas}]]$ because $[[Γ ⊢ iNs2]]$,
        \item $[[ [iNs1/nuas] [nuas/nas] ↓iM' ]] = [[ [iNs1/nas] ↓iM' ]] =
          [[↓nf(iM1)]] = [[nf(↓iM1)]]$.
        \item $[[ [iNs2/nuas] [nuas/nas] ↓iM' ]] = [[ [iNs2/nas] ↓iM' ]] = [[↓nf(iM2)]] = [[nf(↓iM2)]]$.
      \end{itemize}

      Then by the completeness of the anti-unification
      (\cref{lemma:au-completeness}), the anti-unification algorithm
      terminates, so is the Least Upper Bound algorithm invoking it, 
      i.e. $[[iQ']] = [[∃nbs.[nbs / ToList Ξ]uP]]$, where
      $[[(Ξ, uP, aus1, aus2)]]$ is the result of the anti-unification
      of $[[nf(↓iM1)]]$ and $[[nf(↓iM2)]]$ in context $[[Γ]]$.

      Moreover, \cref{lemma:au-completeness} also says that the found anti-unification 
      solution is initial, i.e. there exists $[[aus]]$ such that
      $[[Γ;Ξ ⊢ aus :{nuas}]]$ and $[[ [aus][nuas/nas]↓uM' = uP ]]$.

      Let $[[σ]]$ be a sequential Kleisli composition of the following
      substitutions:
      \begin{enumerate*}
      \item[(i)] $[[nuas/nas]]$,
      \item[(ii)] $[[aus]]$, and
      \item[(iii)] $[[nbs / ToList Ξ]]$.
      \end{enumerate*}
      Notice that $[[Γ, nbs ⊢ σ :{nas}]]$
      and $[[ [σ]↓uM' ]] = [[ [nbs / ToList Ξ][aus][nuas/nas]↓uM' ]] = [[ [nbs /
      ToList Ξ]uP ]]$. In particular, from the reflexivity of subtyping:
      $[[Γ, nbs ⊢ [σ]↓iM' ≥ [nbs / ToList Ξ]uP]]$.

      It allows us to show $[[Γ ⊢ nf(iQ) ≥ iQ']]$, i.e. $[[Γ ⊢ ∃nas.↓iM' ≥
      ∃nbs.[nbs / ToList Ξ]uP]]$, by applying \ruleref{\ottdruleDOneExistsLabel},
      instantiating $[[nas]]$ with respect to $[[σ]]$. Finally, $[[Γ ⊢ iQ ≥ iQ']]$
      by transitively combining $[[Γ ⊢ nf(iQ) ≥ iQ']]$ and $[[Γ ⊢ iQ ≥ nf(iQ)]]$ 
      (holds by \cref{corollary:nf-sound-wrt-subt-equiv} and inversion).
  \end{caseof}
\end{proof}

\end{document}
