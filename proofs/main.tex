\PassOptionsToPackage{prologue,dvipsnames}{xcolor}

% \UseRawInputEncoding
% vim: ft=tex

% \documentclass[acmsmall,natbib=false]{article}
% \usepackage[a4paper, total={8in, 10in}]{geometry}

\documentclass[acmsmall,natbib=false,review,anonymous]{acmart}
\usepackage[dvipsnames]{xcolor}
\usepackage{scalerel}


\usepackage{braket}
% \usepackage{hyperref}
\usepackage{mathpartir}

\usepackage{lscape}
\usepackage{amsmath}
\usepackage{amsthm}
\usepackage{booktabs}
\usepackage{multicol}
\usepackage{supertabular}
\usepackage[inline]{enumitem}
\usepackage{cleveref}
\usepackage{proof}
\usepackage{mathtools}

% Show the frame 
% \usepackage{showframe}

\usepackage{newtxmath}
% \usepackage{mathabx}


\usepackage{stackengine}



\usepackage{todonotes}

\usepackage{enumitem}
\usepackage{xparse}

\usepackage{../casenum}
% \usepackage{../quiver}

\usepackage{listings}
\usepackage{xspace}
\usepackage{subcaption}

% \usepackage{tikz}
\usetikzlibrary{shapes,arrows,arrows.meta,positioning}
\usetikzlibrary{shapes.multipart}

% ⫤
\usepackage{graphicx}
\makeatletter
\providecommand*{\Dashv}{%
  \mathrel{%
    \mathpalette\@Dashv\vDash
  }%
}
\newcommand*{\@Dashv}[2]{%
  \reflectbox{$\m@th#1#2$}%
}
\makeatother



\setlength{\columnsep}{1cm}

\newcommand{\niton}{\not\owns}

\newcommand{\ilyam}[1]{{\color{red} \texttt{Ilya:  #1}}}
\newcommand{\nk}[1]{{\color{purple} \texttt{Neel:  #1}}}

\newtheorem{algorithm}{Algorithm}
\newtheorem{definition}{Definition}
\newtheorem*{notation*}{Notation}
\newtheorem{theorem}{Theorem}
\newtheorem*{theorempreview}{Theorem}
\newtheorem{lemma}{Lemma}
\newtheorem{corollary}{Corollary}
\newtheorem{proposition}{Proposition}
\newtheorem{observation}{Observation}
\newtheorem{property}{Property}
\newtheorem*{assertion*}{Assertion}


\newcommand{\UB}[0]{\mathsf{UB}}
\newcommand{\NFUB}[0]{\mathsf{NFUB}}

\newcommand{\depth}[1]{\ensuremath{\mathsf{depth}(#1)}}
\newcommand{\size}[1]{\ensuremath{\mathsf{size}(#1)}}

\newcommand{\ruleref}[1]{\nameref{#1}}

\newcommand{\fexists}{F$^\pm\exists$\xspace} 
\newcommand{\etc}{F$\exists$}



\newcommand{\ie}{\text{i.e.,} }
\newcommand{\eg}{\text{e.g.,} }
\newcommand{\wrt}{w.r.t.\xspace}
\newcommand{\stt}{s.t.\xspace}
\newcommand{\aka}{a.k.a.\xspace}
\newcommand{\resp}{resp.\xspace}

\newcommand{\code}[1]{\texttt{#1}}

\newcommand{\cmark}{\ding{51}}%
\newcommand{\xmark}{\ding{55}}%

\newcommand{\CBPV}{\text{Call-By-Push-Value} }

\newcommand{\coq}{\texttt{Coq} }
\newcommand{\agda}{\texttt{Agda} }

\newcommand{\systemf}{\text{System F} }

\newcommand{\pack}{\texttt{pack} }
\newcommand{\unpack}{\texttt{unpack} }

% https://tex.stackexchange.com/questions/85033/colored-symbols/85035#85035
\providecommand*{\mathcolor}{}
\def\mathcolor#1#{\mathcoloraux{#1}}
\newcommand*{\mathcoloraux}[3]{%
  \protect\leavevmode
  \begingroup
  \color#1{#2}#3%
  \endgroup
}

\definecolor{positive}{RGB}{200, 50, 50}
\definecolor{negative}{RGB}{50, 50, 200}


\setlength\multicolsep{0pt}

\newcommand{\ottDir}{../ott/_gen}
\newcommand{\genDir}{_gen}
% generated by Ott 0.32 from: grammar.ott
\documentclass[11pt]{article}
\usepackage{amsmath,amssymb}
\usepackage{supertabular}
\usepackage{geometry}
\usepackage{ifthen}
\usepackage{alltt}%hack
\geometry{a4paper,dvips,twoside,left=22.5mm,right=22.5mm,top=20mm,bottom=30mm}
\usepackage{color}
\newcommand{\ottdrule}[4][]{{\displaystyle\frac{\begin{array}{l}#2\end{array}}{#3}\quad\ottdrulename{#4}}}
\newcommand{\ottusedrule}[1]{\[#1\]}
\newcommand{\ottpremise}[1]{ #1 \\}
\newenvironment{ottdefnblock}[3][]{ \framebox{\mbox{#2}} \quad #3 \\[0pt]}{}
\newenvironment{ottfundefnblock}[3][]{ \framebox{\mbox{#2}} \quad #3 \\[0pt]\begin{displaymath}\begin{array}{l}}{\end{array}\end{displaymath}}
\newcommand{\ottfunclause}[2]{ #1 \equiv #2 \\}
\newcommand{\ottnt}[1]{\mathit{#1}}
\newcommand{\ottmv}[1]{\mathit{#1}}
\newcommand{\ottkw}[1]{\mathbf{#1}}
\newcommand{\ottsym}[1]{#1}
\newcommand{\ottcom}[1]{\text{#1}}
\newcommand{\ottdrulename}[1]{\textsc{#1}}
\newcommand{\ottcomplu}[5]{\overline{#1}^{\,#2\in #3 #4 #5}}
\newcommand{\ottcompu}[3]{\overline{#1}^{\,#2<#3}}
\newcommand{\ottcomp}[2]{\overline{#1}^{\,#2}}
\newcommand{\ottgrammartabular}[1]{\begin{supertabular}{llcllllll}#1\end{supertabular}}
\newcommand{\ottmetavartabular}[1]{\begin{supertabular}{ll}#1\end{supertabular}}
\newcommand{\ottrulehead}[3]{$#1$ & & $#2$ & & & \multicolumn{2}{l}{#3}}
\newcommand{\ottprodline}[6]{& & $#1$ & $#2$ & $#3 #4$ & $#5$ & $#6$}
\newcommand{\ottfirstprodline}[6]{\ottprodline{#1}{#2}{#3}{#4}{#5}{#6}}
\newcommand{\ottlongprodline}[2]{& & $#1$ & \multicolumn{4}{l}{$#2$}}
\newcommand{\ottfirstlongprodline}[2]{\ottlongprodline{#1}{#2}}
\newcommand{\ottbindspecprodline}[6]{\ottprodline{#1}{#2}{#3}{#4}{#5}{#6}}
\newcommand{\ottprodnewline}{\\}
\newcommand{\ottinterrule}{\\[5.0mm]}
\newcommand{\ottafterlastrule}{\\}


\usepackage[dvipsnames,usenames]{xcolor}

% https://tex.stackexchange.com/questions/33401/a-version-of-colorbox-that-works-inside-math-environments
\setlength{\fboxsep}{1pt}
\newcommand{\ngbox}[1]{\mathchoice%
  {\colorbox{black!8}{$\displaystyle      \mathit{ #1 } $} }%
  {\colorbox{black!8}{$\textstyle         \mathit{ #1 } $} }%
  {\colorbox{black!8}{$\scriptstyle       \mathit{ #1 } $} }%
  {\colorbox{black!8}{$\scriptscriptstyle \mathit{ #1 } $} } }%


\newcommand{\ottmetavars}{
\ottmetavartabular{
 $ \ottmv{x} $ & \ottcom{term variable} \\
 $ \widehat{x} $ & \ottcom{unification term variable} \\
 $ \ottmv{a} $ & \ottcom{type variable} \\
}}

\newcommand{\ottuv}{
\ottrulehead{\ngbox{v}}{::=}{\ottcom{Potentially non-ground value terms}}\ottprodnewline
\ottfirstprodline{|}{\ottmv{x}}{}{}{}{}\ottprodnewline
\ottprodline{|}{\widehat{x}}{}{}{}{}\ottprodnewline
\ottprodline{|}{\ottkw{refl} \, \ngbox{v}}{}{}{}{}\ottprodnewline
\ottprodline{|}{\ottsym{\{}  \ngbox{t}  \ottsym{\}}}{}{}{}{}}

\newcommand{\ottut}{
\ottrulehead{\ngbox{t}}{::=}{\ottcom{Potentially non-ground computation terms}}\ottprodnewline
\ottfirstprodline{|}{\ngbox{t} \, \ngbox{v}}{}{}{}{}\ottprodnewline
\ottprodline{|}{\ottkw{force} \, \ngbox{v}}{}{}{}{}\ottprodnewline
\ottprodline{|}{ \lambda  \ottmv{x} : \ngbox{A}  . \,  \ngbox{t} }{}{\textsf{bind}\; \ottmv{x}\; \textsf{in}\; \ngbox{t}}{}{}\ottprodnewline
\ottprodline{|}{ \widehat{\lambda}  \ottmv{x} : \ngbox{A}  . \,  \ngbox{t} }{}{\textsf{bind}\; \ottmv{x}\; \textsf{in}\; \ngbox{t}}{}{}\ottprodnewline
\ottprodline{|}{ rec_{eq}^{ \ottmv{x_{{\mathrm{1}}}} . \ottmv{x_{{\mathrm{2}}}} . \ngbox{X} }( \ngbox{v} ,  \ngbox{t} ) }{}{\textsf{bind}\; \ottmv{x_{{\mathrm{1}}}}\; \textsf{in}\; \ngbox{X}}{}{}\ottprodnewline
\ottbindspecprodline{}{}{}{\textsf{bind}\; \ottmv{x_{{\mathrm{2}}}}\; \textsf{in}\; \ngbox{X}}{}{}\ottprodnewline
\ottprodline{|}{\ottkw{return} \, \ngbox{v}}{}{}{}{}\ottprodnewline
\ottprodline{|}{\ottkw{dlet} \, \ottmv{x}  \ottsym{:}  \ngbox{A}  \ottsym{:=}  \ngbox{t}_{{\mathrm{1}}} \, \ottkw{in} \, \ngbox{t}_{{\mathrm{2}}}}{}{\textsf{bind}\; \ottmv{x}\; \textsf{in}\; \ngbox{t}_{{\mathrm{2}}}}{}{}}

\newcommand{\ottuX}{
\ottrulehead{\ngbox{X}}{::=}{}\ottprodnewline
\ottfirstprodline{|}{\uparrow  \ngbox{A}}{}{}{}{}\ottprodnewline
\ottprodline{|}{ \Pi  \ottmv{x} : \ngbox{A}  . \,  \ngbox{X} }{}{\textsf{bind}\; \ottmv{x}\; \textsf{in}\; \ngbox{X}}{}{}\ottprodnewline
\ottprodline{|}{ \forall  \ottmv{x} : \ngbox{A}  . \,  \ngbox{X} }{}{\textsf{bind}\; \ottmv{x}\; \textsf{in}\; \ngbox{X}}{}{}\ottprodnewline
\ottprodline{|}{\ottkw{let} \, \ottmv{x}  \ottsym{:}  \ngbox{A}  \ottsym{:=}  \ngbox{t} \, \ottkw{in} \, \ngbox{X}}{}{\textsf{bind}\; \ottmv{x}\; \textsf{in}\; \ngbox{X}}{}{}}

\newcommand{\ottuA}{
\ottrulehead{\ngbox{A}}{::=}{}\ottprodnewline
\ottfirstprodline{|}{\downarrow  \ngbox{X}}{}{}{}{}\ottprodnewline
\ottprodline{|}{\ottkw{eq} \, \ngbox{A} \, \ngbox{v}_{{\mathrm{1}}} \, \ngbox{v}_{{\mathrm{2}}}}{}{}{}{}\ottprodnewline
\ottprodline{|}{\ottmv{a}}{}{}{}{}}

\newcommand{\ottv}{
\ottrulehead{\ottnt{v}}{::=}{\ottcom{Ground value terms}}\ottprodnewline
\ottfirstprodline{|}{\ottmv{x}}{}{}{}{}\ottprodnewline
\ottprodline{|}{\ottkw{refl} \, \ottnt{v}}{}{}{}{}\ottprodnewline
\ottprodline{|}{\ottsym{\{}  \ottnt{t}  \ottsym{\}}}{}{}{}{}}

\newcommand{\ottt}{
\ottrulehead{\ottnt{t}}{::=}{\ottcom{Ground computation terms}}\ottprodnewline
\ottfirstprodline{|}{\ottnt{t} \, \ottnt{v}}{}{}{}{}\ottprodnewline
\ottprodline{|}{\ottkw{force} \, \ottnt{v}}{}{}{}{}\ottprodnewline
\ottprodline{|}{ \lambda  \ottmv{x} : \ottnt{A}  . \,  \ottnt{t} }{}{\textsf{bind}\; \ottmv{x}\; \textsf{in}\; \ottnt{t}}{}{}\ottprodnewline
\ottprodline{|}{ \widehat{\lambda}  \ottmv{x} : \ottnt{A}  . \,  \ottnt{t} }{}{\textsf{bind}\; \ottmv{x}\; \textsf{in}\; \ottnt{t}}{}{}\ottprodnewline
\ottprodline{|}{ rec_{eq}^{ \ottmv{x_{{\mathrm{1}}}} . \ottmv{x_{{\mathrm{2}}}} . \ottnt{X} }( \ottnt{v} ,  \ottnt{t} ) }{}{}{}{}\ottprodnewline
\ottprodline{|}{\ottkw{return} \, \ottnt{v}}{}{}{}{}\ottprodnewline
\ottprodline{|}{\ottkw{dlet} \, \ottmv{x}  \ottsym{:}  \ottnt{A}  \ottsym{:=}  \ottnt{t_{{\mathrm{1}}}} \, \ottkw{in} \, \ottnt{t_{{\mathrm{2}}}}}{}{\textsf{bind}\; \ottmv{x}\; \textsf{in}\; \ottnt{t_{{\mathrm{2}}}}}{}{}}

\newcommand{\ottX}{
\ottrulehead{\ottnt{X}}{::=}{}\ottprodnewline
\ottfirstprodline{|}{\uparrow  \ottnt{A}}{}{}{}{}\ottprodnewline
\ottprodline{|}{ \Pi  \ottmv{x} : \ottnt{A}  . \,  \ottnt{X} }{}{\textsf{bind}\; \ottmv{x}\; \textsf{in}\; \ottnt{X}}{}{}\ottprodnewline
\ottprodline{|}{ \forall  \ottmv{x} : \ottnt{A}  . \,  \ottnt{X} }{}{\textsf{bind}\; \ottmv{x}\; \textsf{in}\; \ottnt{X}}{}{}\ottprodnewline
\ottprodline{|}{\ottkw{let} \, \ottmv{x}  \ottsym{:}  \ottnt{A}  \ottsym{:=}  \ottnt{t} \, \ottkw{in} \, \ottnt{X}}{}{\textsf{bind}\; \ottmv{x}\; \textsf{in}\; \ottnt{X}}{}{}}

\newcommand{\ottA}{
\ottrulehead{\ottnt{A}}{::=}{}\ottprodnewline
\ottfirstprodline{|}{\downarrow  \ottnt{X}}{}{}{}{}\ottprodnewline
\ottprodline{|}{\ottkw{eq} \, \ottnt{A} \, \ottnt{v_{{\mathrm{1}}}} \, \ottnt{v_{{\mathrm{2}}}}}{}{}{}{}\ottprodnewline
\ottprodline{|}{\ottmv{a}}{}{}{}{}}

\newcommand{\ottterminals}{
\ottrulehead{\ottnt{terminals}}{::=}{}\ottprodnewline
\ottfirstprodline{|}{ \uparrow }{}{}{}{}\ottprodnewline
\ottprodline{|}{ \downarrow }{}{}{}{}\ottprodnewline
\ottprodline{|}{ \in }{}{}{}{}\ottprodnewline
\ottprodline{|}{ \cdot }{}{}{}{}}

\newcommand{\ottformula}{
\ottrulehead{\ottnt{formula}}{::=}{}\ottprodnewline
\ottfirstprodline{|}{\ottnt{judgement}}{}{}{}{}}

\newcommand{\ottjudgement}{
\ottrulehead{\ottnt{judgement}}{::=}{}}

\newcommand{\ottuserXXsyntax}{
\ottrulehead{\ottnt{user\_syntax}}{::=}{}\ottprodnewline
\ottfirstprodline{|}{\ottmv{x}}{}{}{}{}\ottprodnewline
\ottprodline{|}{\widehat{x}}{}{}{}{}\ottprodnewline
\ottprodline{|}{\ottmv{a}}{}{}{}{}\ottprodnewline
\ottprodline{|}{\ngbox{v}}{}{}{}{}\ottprodnewline
\ottprodline{|}{\ngbox{t}}{}{}{}{}\ottprodnewline
\ottprodline{|}{\ngbox{X}}{}{}{}{}\ottprodnewline
\ottprodline{|}{\ngbox{A}}{}{}{}{}\ottprodnewline
\ottprodline{|}{\ottnt{v}}{}{}{}{}\ottprodnewline
\ottprodline{|}{\ottnt{t}}{}{}{}{}\ottprodnewline
\ottprodline{|}{\ottnt{X}}{}{}{}{}\ottprodnewline
\ottprodline{|}{\ottnt{A}}{}{}{}{}\ottprodnewline
\ottprodline{|}{\ottnt{terminals}}{}{}{}{}}

\newcommand{\ottgrammar}{\ottgrammartabular{
\ottuv\ottinterrule
\ottut\ottinterrule
\ottuX\ottinterrule
\ottuA\ottinterrule
\ottv\ottinterrule
\ottt\ottinterrule
\ottX\ottinterrule
\ottA\ottinterrule
\ottterminals\ottinterrule
\ottformula\ottinterrule
\ottjudgement\ottinterrule
\ottuserXXsyntax\ottafterlastrule
}}

% defnss
\newcommand{\ottdefnss}{
}

\newcommand{\ottall}{\ottmetavars\\[0pt]
\ottgrammar\\[5.0mm]
\ottdefnss}

\begin{document}
\ottall

\begin{verbatim}
\end{verbatim}
\end{document}

\renewcommand{\ottrulehead}[3]{\multicolumn{4}{l}{#3}\ottprodnewline$#1$ & & $#2$ & & }
% \renewenvironment{ottdefnblock}[3][]{\noindent#3 \\\framebox{\mbox{#2}}\\[0pt]}{}



% ord varset in uN = varset'

\renewcommand{\ottdruleONVarInName}[0]{(Var$_{-\in}^{\text{Ord}}$)}
\renewcommand{\ottdruleONVarNInName}[0]{(Var$_{-\notin}^{\text{Ord}}$)}
\renewcommand{\ottdruleONUVarName}[0]{(UVar$_{-} ^{\text{Ord}}$)}
\renewcommand{\ottdruleOShiftUName}[0]{($\uparrow^{\text{Ord}}$)}
\renewcommand{\ottdruleOArrowName}[0]{($\rightarrow^{\text{Ord}}$)}
\renewcommand{\ottdruleOForallName}[0]{($\forall^{\text{Ord}}$)}


% ord varset in uP = varset'

\renewcommand{\ottdruleOPVarInName}[0]{(Var$_{+\in}^{\text{Ord}}$)}
\renewcommand{\ottdruleOPVarNInName}[0]{(Var$_{+\notin}^{\text{Ord}}$)}
\renewcommand{\ottdruleOPUVarName}[0]{(UVar$_{+}^{\text{Ord}}$)}
\renewcommand{\ottdruleOShiftDName}[0]{($\downarrow^{\text{Ord}}$)}
\renewcommand{\ottdruleOExistsName}[0]{($\exists^{\text{Ord}}$)}


% nf(N) = M
\renewcommand{\ottdruleNrmNVarName}[0]{(Var$_{-}^{\text{nf}}$)}
\renewcommand{\ottdruleNrmNUVarName}[0]{(UVar$_{-}^{\text{nf}}$)}
\renewcommand{\ottdruleNrmShiftUName}[0]{($\uparrow^{\text{nf}}$)}
\renewcommand{\ottdruleNrmArrowName}[0]{($\rightarrow^{\text{nf}}$)}
\renewcommand{\ottdruleNrmForallName}[0]{($\forall^{\text{nf}}$)}

% nf(P) = Q
\renewcommand{\ottdruleNrmPVarName}[0]{(Var$_{+}^{\text{nf}}$)}
\renewcommand{\ottdruleNrmPUVarName}[0]{(UVar$_{+}^{\text{nf}}$)}
\renewcommand{\ottdruleNrmShiftDName}[0]{($\downarrow^{\text{nf}}$)}
\renewcommand{\ottdruleNrmExistsName}[0]{($\exists^{\text{nf}}$)}


% N ≈ M

\renewcommand{\ottdruleEOneNVarName}[0]{(Var$_-^{\eqEOne}$)}
\renewcommand{\ottdruleEOneShiftUName}[0]{($\uparrow^{\eqEOne}$)}
\renewcommand{\ottdruleEOneArrowName}[0]{($\rightarrow^{\eqEOne}$)}
\renewcommand{\ottdruleEOneForallName}[0]{($\forall^{\eqEOne}$)}

% P ≈ Q
\renewcommand{\ottdruleEOnePVarName}[0]{(Var$_+^{\eqEOne}$)}
\renewcommand{\ottdruleEOneShiftDName}[0]{($\downarrow^{\eqEOne}$)}
\renewcommand{\ottdruleEOneExistsName}[0]{($\exists^{\eqEOne}$)}


% G ⊢ N ≤1 M

\renewcommand{\ottdruleDOneNVarName}[0]{(Var$_-^{\subDOne}$)}
\renewcommand{\ottdruleDOneShiftUName}[0]{($\uparrow^{\subDOne}$)}
\renewcommand{\ottdruleDOneArrowName}[0]{($\rightarrow^{\subDOne}$)}
\renewcommand{\ottdruleDOneForallName}[0]{($\forall^{\subDOne}$)}

% G ⊢ P ≥1 Q
\renewcommand{\ottdruleDOnePVarName}[0]{(Var$_+^{\supDOne}$)}
\renewcommand{\ottdruleDOneShiftDName}[0]{($\downarrow^{\supDOne}$)}
\renewcommand{\ottdruleDOneExistsName}[0]{($\exists^{\supDOne}$)}


% G ⊢ N ≈1 M
\renewcommand{\ottdruleDOneNDefName}[0]{($\eqDOneNeg$)}

% G ⊢ P ≈1 Q
\renewcommand{\ottdruleDOnePDefName}[0]{($\eqDOnePos$)}



% G ⊨ iP1 ∨ iP2 = iQ
\renewcommand{\ottdruleLUBVarName}[0]{(Var$^{\vee}$)}
\renewcommand{\ottdruleLUBShiftName}[0]{($\downarrow^{\vee}$)}
\renewcommand{\ottdruleLUBExistsName}[0]{($\exists^{\vee}$)}
\renewcommand{\ottdruleLUBUpgradeName}[0]{(Upg)}


% G ; Θ ⊨ uN ≤ iM ⫤ us
\renewcommand{\ottdruleANVarName}[0]{(Var$_-^{\subA}$)}
\renewcommand{\ottdruleAShiftUName}[0]{($\uparrow^{\subA}$)}
\renewcommand{\ottdruleAArrowName}[0]{($\rightarrow^{\subA}$)}
\renewcommand{\ottdruleAForallName}[0]{($\forall^{\subA}$)}

% G ; Θ ⊨ iP ≥ uQ ⫤ us
\renewcommand{\ottdruleAPVarName}[0]{(Var$_+^{\supA}$)}
\renewcommand{\ottdruleAShiftDName}[0]{($\downarrow^{\supA}$)}
\renewcommand{\ottdruleAExistsName}[0]{($\exists^{\supA}$)}
\renewcommand{\ottdruleAPUVarName}[0]{(UVar$^{\supA}$)}


% Γ ⊢ scE1 & scE2 = scE3 :: :: E :: 'E'
\renewcommand{\ottdruleSCMESupSupName}[0]{$([[≥]]\&^{+}[[≥]])$}
\renewcommand{\ottdruleSCMEEqSupName}[0]{$([[≈]]\&^{+}[[≥]])$}
\renewcommand{\ottdruleSCMESupEqName}[0]{$([[≥]]\&^{+}[[≈]])$}
\renewcommand{\ottdruleSCMEPEqEqName}[0]{$([[≈]]\&^{+}[[≈]])$}
\renewcommand{\ottdruleSCMENEqEqName}[0]{$([[≈]]\&^{-}[[≈]])$}

% Γ ; Θ ⊨ uN ≈u iM ⫤ UC 
\renewcommand{\ottdruleUNVarName}[0]{(Var$_{-}^{[[≈u]]}$)}
\renewcommand{\ottdruleUShiftUName}[0]{($\uparrow^{[[≈u]]}$)}
\renewcommand{\ottdruleUArrowName}[0]{($\rightarrow^{[[≈u]]}$)}
\renewcommand{\ottdruleUForallName}[0]{($\forall^{[[≈u]]}$)}
\renewcommand{\ottdruleUNUVarName}[0]{(UVar$_{-}^{[[≈u]]}$)}

% Γ ; Θ ⊨ uP ≈u iQ ⫤ UC 
\renewcommand{\ottdruleUPVarName}[0]{(Var$_{+}^{[[≈u]]}$)}
\renewcommand{\ottdruleUShiftDName}[0]{($\downarrow^{[[≈u]]}$)}
\renewcommand{\ottdruleUExistsName}[0]{($\exists^{[[≈u]]}$)}
\renewcommand{\ottdruleUPUVarName}[0]{(UVar$_{+}^{[[≈u]]}$)}

% G ⊨ iP1 ≈au iP2 ⫤ ( Ξ , uQ , aus1 , aus2 )
\renewcommand{\ottdruleAUPVarName}[0]{(Var$_{+}^{[[≈au]]}$)}
\renewcommand{\ottdruleAUShiftDName}[0]{($\downarrow^{[[≈au]]}$)}
\renewcommand{\ottdruleAUExistsName}[0]{($\exists^{[[≈au]]}$)}

% G ⊨ iN1 ≈au iN2 ⫤ ( Ξ , uM , aus1 , aus2 )
\renewcommand{\ottdruleAUNVarName}[0]{(Var$_{-}^{[[≈au]]}$)}
\renewcommand{\ottdruleAUShiftUName}[0]{($\uparrow^{[[≈au]]}$)}
\renewcommand{\ottdruleAUForallName}[0]{($\forall^{[[≈au]]}$)}
\renewcommand{\ottdruleAUArrowName}[0]{($\rightarrow^{[[≈au]]}$)}
\renewcommand{\ottdruleAUAUName}[0]{(AU)}

% Γ ⊢ iN 
\renewcommand{\ottdruleWFTNVarName}[0]{(Var$_{-}^{\text{WF}}$)}
\renewcommand{\ottdruleWFTShiftUName}[0]{($\uparrow^{\text{WF}}$)}
\renewcommand{\ottdruleWFTArrowName}[0]{($\rightarrow^{\text{WF}}$)}
\renewcommand{\ottdruleWFTForallName}[0]{($\forall^{\text{WF}}$)}

% Γ ⊢ iP
\renewcommand{\ottdruleWFTPVarName}[0]{(Var$_{+}^{\text{WF}}$)}
\renewcommand{\ottdruleWFTShiftDName}[0]{($\downarrow^{\text{WF}}$)}
\renewcommand{\ottdruleWFTExistsName}[0]{($\exists^{\text{WF}}$)}

% Γ ; Ξ ⊢ uN
\renewcommand{\ottdruleWFATNVarName}[0]{(Var$_{-}^{\text{WF}}$)}
\renewcommand{\ottdruleWFATNUVarName}[0]{(UVar$_{-}^{\text{WF}}$)}
\renewcommand{\ottdruleWFATShiftUName}[0]{($\uparrow^{\text{WF}}$)}
\renewcommand{\ottdruleWFATArrowName}[0]{($\rightarrow^{\text{WF}}$)}
\renewcommand{\ottdruleWFATForallName}[0]{($\forall^{\text{WF}}$)}

% Γ ; Ξ ⊢ uP
\renewcommand{\ottdruleWFATPVarName}[0]{(Var$_{+}^{\text{WF}}$)}
\renewcommand{\ottdruleWFATPUVarName}[0]{(UVar$_{+}^{\text{WF}}$)}
\renewcommand{\ottdruleWFATShiftDName}[0]{($\downarrow^{\text{WF}}$)}
\renewcommand{\ottdruleWFATExistsName}[0]{($\exists^{\text{WF}}$)}

% Γ ⊢ iP : scE
\renewcommand{\ottdruleSATSCESupName}[0]{($[[:≥]]_{+}^\text{sat}$)}
\renewcommand{\ottdruleSATSCEPEqName}[0]{($[[:≈]]_{+}^\text{sat}$)}

% Γ ⊢ iN : scE
\renewcommand{\ottdruleSATSCENEqName}[0]{($[[:≈]]_{-}^\text{sat}$)}

% Γ ; Φ ⊢ v : iP 
\renewcommand{\ottdruleDTVarName}[0]{(Var$^{\text{inf}}$)}
\renewcommand{\ottdruleDTThunkName}[0]{($\{\}^{\text{inf}}$)}
\renewcommand{\ottdruleDTPAnnotName}[0]{(ann$_+^{\text{inf}}$)}
\renewcommand{\ottdruleDTPEquivName}[0]{($[[≈]]_+^{\text{inf}}$)}

% Γ ; Φ ⊢ c : iN 
\renewcommand{\ottdruleDTtLamName}[0]{($\lambda^{\text{inf}}$)}
\renewcommand{\ottdruleDTTLamName}[0]{($\Lambda^{\text{inf}}$)}
\renewcommand{\ottdruleDTReturnName}[0]{(ret$^{\text{inf}}$)}
\renewcommand{\ottdruleDTVarLetName}[0]{(let$^{\text{inf}}$)}
\renewcommand{\ottdruleDTAppLetName}[0]{(let$_@^{\text{inf}}$)}
\renewcommand{\ottdruleDTAppLetAnnName}[0]{(let$_{:@}^{\text{inf}}$)}
\renewcommand{\ottdruleDTUnpackName}[0]{(let$_{\exists}^{\text{inf}}$)}
\renewcommand{\ottdruleDTNAnnotName}[0]{(ann$_-^{\text{inf}}$)}
\renewcommand{\ottdruleDTNEquivName}[0]{($[[≈]]_-^{\text{inf}}$)}

% Γ ; Φ ⊢ iN ● args ⇒> iM 
\renewcommand{\ottdruleDTEmptyAppName}[0]{($\emptyset_{[[●]][[⇒>]]}^{\text{inf}}$)}
\renewcommand{\ottdruleDTArrowAppName}[0]{($\rightarrow_{[[●]][[⇒>]]}^{\text{inf}}$)}
\renewcommand{\ottdruleDTForallAppName}[0]{($\forall_{[[●]][[⇒>]]}^{\text{inf}}$)}



% Γ ; Φ ⊨ v : iP  
\renewcommand{\ottdruleATVarName}[0]{(Var$^{\text{inf}}$)}
\renewcommand{\ottdruleATThunkName}[0]{($\{\}^{\text{inf}}$)}
\renewcommand{\ottdruleATPAnnotName}[0]{(ann$_+^{\text{inf}}$)}

% Γ ; Φ ⊨ c : iN 
\renewcommand{\ottdruleATtLamName}[0]{($\lambda^{\text{inf}}$)}
\renewcommand{\ottdruleATTLamName}[0]{($\Lambda^{\text{inf}}$)}
\renewcommand{\ottdruleATReturnName}[0]{(ret$^{\text{inf}}$)}
\renewcommand{\ottdruleATVarLetName}[0]{(let$^{\text{inf}}$)}
\renewcommand{\ottdruleATAppLetName}[0]{(let$_@^{\text{inf}}$)}
\renewcommand{\ottdruleATAppLetAnnName}[0]{(let$_{:@}^{\text{inf}}$)}
\renewcommand{\ottdruleATUnpackName}[0]{(let$_{\exists}^{\text{inf}}$)}
\renewcommand{\ottdruleATNAnnotName}[0]{(ann$_-^{\text{inf}}$)}

% Γ ; Φ ; Θ1 ⊨ uN ● args ⇒> uM ⫤ Θ2 ; SC 
\renewcommand{\ottdruleATEmptyAppName}[0]{($\emptyset_{[[●]][[⇒>]]}^{\text{inf}}$)}
\renewcommand{\ottdruleATArrowAppName}[0]{($\rightarrow_{[[●]][[⇒>]]}^{\text{inf}}$)}
\renewcommand{\ottdruleATForallAppName}[0]{($\forall_{[[●]][[⇒>]]}^{\text{inf}}$)}

% scE1 singular with iP
\renewcommand{\ottdruleSINGPEqName}[0]{($[[≈]]_{+}^{\text{sing}}$)}
\renewcommand{\ottdruleSINGSupVarName}[0]{($[[:≥]]\alpha^{\text{sing}}$)}
\renewcommand{\ottdruleSINGSupShiftName}[0]{($[[:≥]][[↓]]^{\text{sing}}$)}

% scE1 singular with iN
\renewcommand{\ottdruleSINGNEqName}[0]{($[[≈]]_{-}^{\text{sing}}$)}




\usepackage{../bibliography}
% \usepackage{amssymb}

\usepackage[matha]{mathabx}

% https://tex.stackexchange.com/questions/85033/colored-symbols/85035#85035
% \newcommand*{\mathcolor}{}
% \def\mathcolor#1#{ \mathcoloraux{#1} }
% \newcommand*{\mathcoloraux}[3]{%
%   \protect\leavevmode
%   \begingroup
%   \color#1{#2}#3%
%   \endgroup
% }

\begin{document}

\title{The Proofs}

\maketitle

\tableofcontents

\newpage

\section{Declarative Type Systems}

\subsection{Grammar}
We assume that there is an infinite set of positive and 
negative \emph{type} variables. Positive type variables are denoted as 
$[[α⁺]]$, $[[β⁺]]$, $[[γ⁺]]$, etc.
Negative type variables are denoted as $[[α⁻]]$, $[[β⁻]]$, $[[γ⁻]]$, etc.
We assume there is an infinite set of \emph{term} variables,
which are denoted as $[[x]]$, $[[y]]$, $[[z]]$, etc.
A list of objects (variables, types or terms) is denoted by
an overline arrow. For instance, $[[pas]]$ is a list of positive type variables, 
$[[nbs]]$ is a list of negative type variables, 
$[[args]]$ is a list of values, which are arguments of a function.
$[[fv(iP)]]$ and $[[fv(iN)]]$ denote the set of free variables 
in a type $[[iP]]$ and $[[iN]]$, respectively.

\bigskip

% \ottgrammartabular{
%   \ottiP\ottinterrule
%   \ottiN\ottinterrule
% }

\begin{definition}[Declarative Types]
  \hfill
  \begin{multicols}{2}
    \ottgrammartabular{
      \ottiN\ottinterrule
    }

    \ottgrammartabular{
      \ottiP\ottinterrule
    }
    \columnbreak
  \end{multicols}
\end{definition}

\subsection{Equalities}
For simplicity, we assume alpha-equivalent terms equal.
This way, we assume that substitutions do not capture bound variables.
Besides, we equate
$[[∀pas.∀pbs.iN]]$ with $[[∀pas,pbs.iN]]$, 
as well as $[[∃nas.∃nbs.iP]]$ with $[[∃nas,nbs.iP]]$,
and lift these equations transitively and congruently 
to the whole system.

\subsection{Contexts and Well-formedness}

\begin{definition}[Declarative Type Context]
  \hfill \\
  Declarative type context $[[Γ]]$ is represented by a set of 
  type variables. The concatenation $[[Γ1, Γ2]]$ means the 
  union of two contexts $[[Γ1 ∪ Γ2]]$.
\end{definition}

$[[Γ ⊢ iP]]$ and $[[Γ ⊢ iN]]$ denote that the type is well-formed in the context $[[Γ]]$,
which, in fact, means that each free type variable of the type is contained in $[[Γ]]$
(it will be shown later in \cref{lemma:wf-soundness,lemma:wf-ctxt-equiv}).

Notice that checking the well-formedness of a type
is an \emph{algorithmic} procedure, in which 
both the context and the type are considered inputs.
In other words, it is syntax-directed and mode-correct 
(according to \cite{dunfieldBidirectionalTyping2020}), 
which means that 

\begin{algorithm}[Type Well-formedness]
  \label{alg:wf}
  \hfill
  
  \begin{multicols}{2}
  \ottdefnWFTNiWF{}
  \columnbreak

  \ottdefnWFTPiWF{}
  \end{multicols}

\end{algorithm}


\subsection{Substitutions}

\begin{definition}[Substitution]
  Substitutions (denoted as $[[σ]]$) 
  are represented by total functions form variables to types, preserving the polarity. 
\end{definition}

\begin{algorithm}[Substitution Application]
  Substitution application is denoted as $[[ [σ]iP ]]$ and $[[ [σ]iN ]]$.
  It is defined naturally as follows:
    \begin{multicols}{2}
      \begin{itemize}
        \item[] $[[ [σ]α⁺ ]] = [[σ]] ([[α⁺]])$
        \item[] $[[ [σ]α⁻ ]] = [[σ]] ([[α⁻]])$
        \item[] $[[ [σ]↓iN ]] = [[↓[σ]iN]]$
        \item[] $[[ [σ]↑iP ]] = [[↑[σ]iP]]$
        \item[] $[[ [σ](iP → iN) ]] = [[ [σ]iP → [σ]iN ]]$
        \item[] $[[ [σ]∃nas.iQ ]] = [[∃nas.[σ]iQ]]$ 
        \item[] $[[ [σ]∀pas.iN ]] = [[∀pas.[σ]iN]]$ (assuming the variable capture never happens)
      \end{itemize}
    \end{multicols}
\end{algorithm}

\begin{definition}[Substitution Signature]
  The signature $[[Γ' ⊢ σ : Γ]]$ means that
  \begin{enumerate}
    \item for any $[[α± ∊ Γ]], [[ Γ' ⊢ [σ]α± ]]$; and
    \item for any $[[α± ∉ Γ']], [[ [σ]α± = α± ]]$.
  \end{enumerate}
\end{definition}

A substitution can be restricted to a set of variables. 
The restricted substitution is define as expected. 
\begin{definition}[Subsitution Restriction]
  The specification $[[σ  | varset]]$ is defined as
  a function such that 
  \begin{enumerate}
    \item $[[σ|varset]]([[α± ]]) = [[σ]]([[α± ]])$, if $[[α± ]] \in [[varset]]$; and
    \item $[[σ|varset]]([[α± ]]) = [[α± ]]$, if $[[α± ]] \notin [[varset]]$.
  \end{enumerate}
\end{definition}

Two substitutions can be composed in two ways:
$[[σ2 ○ σ1]]$ corresponds to a consecutive application of $[[σ1]]$ and $[[σ2]]$,
while $[[σ2 <=< σ1]]$
depends on a signature of $[[σ1]]$ and modifies $[[σ1]]$ by applying
$[[σ2]]$ to its results on the domain.
\begin{definition}[Substitution Composition]
  $[[σ2 ○ σ1]]$ is defined as a function such that
  $[[σ2 ○ σ1]]([[α± ]]) = [[σ2]]([[σ1]]([[α± ]]))$.
\end{definition}

\begin{definition}[Monadic Substitution Composition]
  Suppose that $[[Γ' ⊢ σ1 : Γ]]$.
  Then we define $[[σ2 <=< σ1]]$ as $[[(σ2 ○ σ1)|Γ]]$.
\end{definition}
Notice that the result of $[[σ2 <=< σ1]]$ depends on the 
specification of $[[σ1]]$, which is not unique. 
However, we assume that the used specification clear from the 
context of the proof. 

\begin{definition}[Equivalent Substitutions]
  The substitution equivalence judgement $[[Γ' ⊢ σ1 ≈ σ2 : Γ]]$ 
  indicates that on the domain $[[Γ]]$, 
  the result of $[[σ1]]$ and $[[σ2]]$ are equivalent in context $[[Γ']]$.
  Formally, for any $[[α± ∊ Γ]], [[ Γ' ⊢ [σ1]α± ≈ [σ2]α± ]]$.
\end{definition}

Sometimes it is convenient to construct substitution 
explicitly mapping each variable from a list (or a set)
to a type. Such substitutions are denoted as $[[iPs / pas]]$
and $[[iNs / nas]]$, where $[[iPs]]$ and $[[iNs]]$ are lists of 
the corresponding types.
\begin{definition}[Explicit Substitution]
  \hfill
  \begin{itemize}
    \item [$-$]
      Suppose that $[[nas]]$ is a list of negative type variables,
      and $[[iNs]]$ is a list of negative types of the same length.
      Then $[[iNs / nas]]$ denotes a substitution such that 
      \begin{enumerate}
        \item for $[[αi⁺ ∊ {nas}]]$, $[[ [iNs / nas] αi⁺]] = [[iNi]]$;
        \item for $[[β⁺ ∉ {nas}]]$, $[[ [iNs / nas] β⁺]] = [[β⁺]]$.
      \end{enumerate}
    \item [$+$]
      Positive explicit substitution $[[iPs / pas]]$
      is defined symmetrically.
  \end{itemize}
\end{definition}


\subsection{Declarative Subtyping}
Subtyping is one of the key mechanism of our system. 
It realizes the polymorphism: abstract $[[∀]]$ and 
$[[∃]]$ types can be used where concrete types are expected,
exactly because the subtyping relation between them.

\begin{definition} 
  \label{def:subDOne}
  \hfill
  
  \begin{multicols}{2}
    \ottdefnDOneNsub{}

    \ottdefnDOnePsup{}
  \end{multicols}
  \hfill

  \begin{multicols}{2}
    \ottdefnDOneNeq{}

    \ottdefnDOnePeq{}
  \end{multicols}
\end{definition}

The following observations about the declarative subtyping are worth noting:
\begin{itemize}
  \item \ruleref{\ottdruleDOneNVarLabel} and \ruleref{\ottdruleDOnePVarLabel}
    make the subtyping reflexive on variables (and further, on any type).
  \item \ruleref{\ottdruleDOneArrowLabel} is standard: the arrow is covariant on the
    resulting type and contravariant on the argument type.
  \item \ruleref{\ottdruleDOneShiftDLabel}  and \ruleref{\ottdruleDOneShiftULabel} are non-standard:
    the subtyping is \emph{invariant} for shifts. 
    This way, the subtyping of shifted types in one direction implies the subtyping
    in the opposite direction.
    Although this rule restricts the
    subtyping relation, it makes the system decidable.
  \item \ruleref{\ottdruleDOneForallLabel} and \ruleref{\ottdruleDOneExistsLabel} are the only
    non-algorithmic rules: the substitution for the quantified variable is
    not specified, those, these rules `drive' the subtyping relation.
\end{itemize}

In the next section, we present the sound and complete algorithm
checking whether one type is a subtype of another according to \cref{def:subDOne}. 

\section{Algorithmic Type System}

\subsection{Grammar}

In the algorithmic system, we extend the grammar of types
by adding positive and negative \emph{algorithmic variables}
($[[α̂⁺]]$, $[[β̂⁺]]$, $[[γ̂⁺]]$, etc. and $[[α̂⁻]]$, $[[β̂⁻]]$, $[[γ̂⁻]]$, etc.).
They represent the unknown types, which will be inferred by the algorithm.
This way, we add two base cases to the grammar of 
positive and negative types and use highlight to denote that the type
can potentially contain algorithmic variables.

\begin{definition}[Algorithmic Types]
  \label{def:algo-types}
  \hfill\\
  \begin{multicols}{2}
    \ottgrammartabular{
      \ottuN\ottinterrule
    }

    \ottgrammartabular{
      \ottuP\ottinterrule
    }
    \columnbreak
  \end{multicols}
\end{definition}

\subsection{Fresh Variable Selection}
\label{sec:fresh-selection}
Both the subtyping and the type inference algorithm
rely on the ability to select fresh, unused variables.
For a set of variables $[[varset]]$, it is indicated as 
$[[varset are fresh]]$ in the inference rules.
We assume that the selection subroutine always succeeds and is 
deterministic. In other words, whenever it is called in 
an algorithmic inference rule, it returns the same result, 
uniquely determined by the input of this rule.

\subsection{Variable Algorithmization}
\label{sec:variable-algorithmization}

In several places of our algorithm, in particular, during
algorithmic subtyping,
we turn a declarative type into an algorithmic one
via replacing certain type variables with fresh algorithmic variables.
We call this procedure \emph{variable algorithmization}, and define it as follows.

\begin{definition}[Variable Algorithmization]
  Suppose that $[[nas]]$ is a list of negative type variables
  and $[[nuas]]$ is a list of negative algorithmic variables of the same length. 
  Then $[[ nuas/nas ]]$ is a substitution-like procedure replacing each $[[αi⁻ ∊ {nas}]]$
  in a type for $[[αî⁻ ∊ {nuas}]]$.
\end{definition}

Conversely, we have the opposite procedure turning algorithmic type variables
into declarative type variables via \emph{dealgorithmization}.

\begin{definition}[Variable Dealgorithmization]
  Suppose that $[[nuas]]$ is a list of negative algorithmic variables
  and $[[nas]]$ is a list of negative type variables of the same length. 
  Then $[[ nas/nuas ]]$ is a substitution-like procedure replacing each
  $[[αî⁻ ∊ {nuas}]]$ in a type for $[[αi⁻ ∊ {nas}]]$.
\end{definition}


\subsection{Contexts and Well-formedness}

\begin{definition}[Algorithmic Type Context $[[Ξ]]$]
  \hfill \\
  Algorithmic type context $[[Ξ]]$ is represented by a set of 
  \emph{algorithmic} type variables ($[[α̂⁺]]$, $[[α̂⁻]]$, $[[β̂⁺]]$, \dots).
  The concatenation $[[Ξ1, Ξ2]]$ means the union of two contexts $[[Ξ1 ∪ Ξ2]]$.
\end{definition}

$[[Γ ; Ξ ⊢ uP]]$ and $[[Γ ; Ξ ⊢ uN]]$ are used to denote
that the algorithmic type is well-formed in the contexts
$[[Γ]]$ and $[[Ξ]]$, which means that each algorithmic variable
of the type is contained in $[[Ξ]]$, and each free declarative type variable
of the type is contained in $[[Γ]]$.

\begin{algorithm}[Algorithmic Type Well-formedness]
  \hfill
  
  \begin{multicols}{2}
  \ottdefnWFATNauWF{}
  \columnbreak

  \ottdefnWFATPauWF{}
  \end{multicols}

\end{algorithm}


Algorithmic Type Context are used in the unification algorithm.
In the subtyping algorithm, 
the context needs to remember additional information.
In the subtyping context, each algorithmic variable is associated with a
context it must be instantiated in 
(i.e. the context in which the type replacing the variable must be well-formed).
This association is represented by \emph{algorithmic subtyping context} $[[Θ]]$.
\begin{definition}[Algorithmic Subtyping Context $[[Θ]]$]
  \hfill \\
  Algorithmic Subtyping Context $[[Θ]]$ is represented by a set of 
  entries of form $[[ α̂⁺[Γ] ]]$ and $[[ α̂⁻[Γ] ]]$,
  where $[[α̂⁺]]$ and $[[α̂⁻]]$ are algorithmic variables,
  and $[[Γ]]$ is a context in which they must be instantiated.
  We assume that no two entries associating the same variable
  appear in $[[Θ]]$.

  $[[dom(Θ)]]$ denotes the set of variables appearing in $[[Θ]]$:
  $[[dom(Θ)]] = \{ [[α̂±]] \mid [[α̂±[Γ] ]] \in [[Θ]] \}$.
  If $[[ α̂±[Γ] ]] \in [[Θ]]$, we denote $[[Γ]]$ as $[[Θ(α̂±)]]$.
\end{definition}


\subsection{Subsitutions}

Substitution that operates on algorithmic type variables is denoted as
$[[uσ]]$. It is defined as a total function from algorithmic 
type variables to \emph{declarative} types, preserving the polarity.

The signature $[[Θ ⊢ uσ : Ξ]]$ means that
$[[Ξ ⊆ dom(Θ)]]$ and 
$[[uσ]]$ maps each algorithmic variable 
from $[[Ξ]]$ to a type well-formed in $[[Θ(α̂±)]]$;
and for each variable not appearing in $[[dom(Θ)]]$, 
it acts as identity.

\begin{definition}[Signature of Algorithmic Substitution]
  \label{def:algo-subst-sig}
  \hfill
  \begin{itemize}
    \item $[[Θ ⊢ uσ : Ξ]]$ means that
      \begin{enumerate}
        \item for any $[[α̂± ∊ Ξ]]$,
          there exists $[[Γ]]$ such that $[[ α̂±[Γ] ∊ Θ ]]$
          and $[[ Γ ⊢ [uσ]α̂± ]]$; 
        \item for any $[[ α̂± ∉ Ξ]]$, $[[ [uσ]α̂± ]] =  [[ α̂± ]]$.
      \end{enumerate}
    \item $[[Γ ⊢ uσ : Ξ]]$ means that
      \begin{enumerate}
        \item for any $[[α̂± ∊ Ξ]]$, $[[ Γ ⊢ [uσ]α̂± ]]$; 
        \item for any $[[ α̂± ∉ Ξ]]$, $[[ [uσ]α̂± ]] =  [[ α̂± ]]$.
      \end{enumerate}
  \end{itemize}
\end{definition}

In the anti-unification algorithm, we use another kind of substitution.
In contrast to algorithmic substitution $[[uσ]]$,
it allows mapping algorithmic variables to
\emph{algorithmic} types.
Additionally, anti-unification substitution is restricted to the
\emph{negative} segment of the language.
Anti-unification substitution is denoted as $[[aus]]$ and $[[ausr]]$.a

The pair of contexts $[[Γ]]$ and $[[Ξ]]$,
in which the results of an anti-unification substitution 
are formed, is fixed for this substitution.
This way, $[[Γ; Ξ2 ⊢ aus : Ξ1]]$ means that $[[aus]]$ maps each negative algorithmic
variable appearing in $[[Ξ1]]$ to a term well-formed in $[[Γ]]$ and $[[Ξ2]]$.

\begin{definition}[Signature of Anti-unification substitution]
  $[[Γ; Ξ2 ⊢ aus : Ξ1]]$ means that
  \begin{enumerate}
    \item for any $[[ α̂⁻ ∊ Ξ1]]$, $[[ Γ; Ξ2 ⊢ [aus]α̂⁻ ]]$ and
    \item for any $[[ α̂⁻ ∉ Ξ1]]$, $[[ [aus]α̂⁻ = α̂⁻ ]]$.
  \end{enumerate}
\end{definition}

\subsection{Equivalence and Normalization}
\label{sec:equivalence-normalization}

The subtyping-induced equivalence (\cref{def:subDOne}) is non-trivial:
there are types that are subtypes of each other but not equal. 
For example, $[[∀α⁺,β⁺.α⁺ → ↑β⁺]]$ is a subtype and a supertype of $[[∀α⁺,β⁺.β⁺ → ↑α⁺]]$
and of, for example, $[[∀α⁺,β⁺.β⁺ → ↑∃γ⁻.α⁺]]$, 
although these types are not alpha-equivalent.
For the subtyping algorithm, it is crucial to be able to check whether
two types are equivalent, without checking mutual subtyping. 
For this purpose we define the normalization procedure, 
which allows us to uniformly choose the representative type of the equivalence class.
This way, the equivalence checking is reduced to normalization and equality checking. 

For clarification of the proofs and better understanding of the system, 
we introduce an intermediate relation---\emph{declarative equivalence}. 
As will be shown in \cref{lemma:equiv-soundness,lemma:equiv-completeness}, 
this relation is equivalent to the subtyping-induced equivalence, but does not 
depend on it. Although this relation is not defined algorithmically, 
it gives the intuition of what types our system considers equivalent.
Specifically, in addition to \emph{alpha-equivalence}, 
our system allows for \emph{reordering of adjacent quantifiers},
and \emph{introduction/elimination of unused quantifiers}.

The non-trivial rules of the declarative equivalence are
\ruleref{\ottdruleEOneForallLabel} and \ruleref{\ottdruleEOneExistsLabel}.
Intuitively, the variable bijection $[[μ]]$ reorders the quantifiers before
the recursive call on the body of the quantified type. 
It will be covered formally in \cref{sec:decl-equiv-lemmas}.

\begin{definition}[Declarative Type Equivalence]
  \hfill
  
  \begin{multicols}{2}
  \ottdefnEOneNeq{}
  \columnbreak\\
  \ottdefnEOnePeq{}
  \end{multicols}

\end{definition}

As the equivalence includes arbitrary reordering of quantified variables,
the normalization procedure is needed to choose the canonical order.
For this purpose, we introduce an auxiliary procedure---variable ordering. 
Intuitively, $[[ord varset in iN]]$ returns a list of variables from $[[varset]]$
in the order they appear in $[[iN]]$.

\begin{algorithm}[Variable Ordering]
  \label{alg:var-ordering}
  \hfill
  
  \ottdefnONVar{}
  \ottdefnOPVar{}

  Analogously, the variable can be ordered in 
  an \emph{algorithmic} type ($[[ord varset in uP]]$ and 
  $[[ord varset in uN]]$). In these cases, we treat the algorithmic variables
  as if they were declarative variables.

\end{algorithm}

Next, we use the variable ordering in the normalization procedure. 
Specifically, normalization recursively traverses the type, 
and for each quantified case reorders the quantified variables in a 
canonical order dictated by \cref{alg:var-ordering}, removing unused ones.

\begin{algorithm}[Type Normalization]
  \label{alg:type-nf}
  \hfill
  
  \begin{multicols}{2}
  \ottdefnNrmNNorm{}
  \columnbreak\\
  \ottdefnNrmPNorm{}
  \end{multicols}

  Analogously, we define the normalization of algorithmic types by adding base cases:

  \begin{multicols}{2}
  \ottdefnNrmuNNorm{}
  \columnbreak\\
  \ottdefnNrmuPNorm{}
  \end{multicols}

\end{algorithm}

\Cref{lemma:subt-equiv-algorithmization}
demonstrates that the equivalence of types is the same
as the equality of their normal forms.
\begin{theorempreview}[Correctness of Normalization]
  Assuming the types are well-formed in $[[Γ]]$, 
  \begin{itemize}
    \item [$-$] $[[Γ ⊢ iN ≈ iM]]$ if and only if $[[nf(iN) = nf(iM)]]$;
    \item [$+$] $[[Γ ⊢ iP ≈ iQ]]$ if and only if $[[nf(iP) = nf(iQ)]]$.
  \end{itemize}
\end{theorempreview}



\begin{algorithm}[Substitution Normalization]
  For a substitution $[[σ]]$, we define $[[nf(σ)]]$
  as a substitution that maps $[[α±]]$ into $[[nf([σ]α±)]]$.
\end{algorithm}

The rest of this chapter is devoted to the
central algorithm of the type system---the subtyping algorithm. 
\Cref{fig:subtyping-algo} shows the dependency graph of the subtyping algorithm.
The nodes represent the algorithmic procedures, and the edge $A \to B$ means that 
$A$ uses $B$ as a sub-procedure.



\begin{figure}[h]
  \centering
  \begin{tikzpicture}
    [>={Stealth[scale=2]},node distance=2.4cm,every node/.style={draw,rectangle},every text node part/.style={align=center}]


    % Define nodes
    \node[] (1) {Negative Subtyping\\$[[Γ ; Θ ⊨ uN ≤ iM ⫤ SC]]$\\(\cref{sec:subtyping})};
    \node[below of=1] (3) {Positive Subtyping\\$[[Γ ; Θ ⊨ uP ≥ iQ ⫤ SC]]$\\(\cref{sec:subtyping})};
    \node[left=1cm of 3] (2) {Subtyping Constraint Merge\\$[[Θ ⊢ SC1 & SC2 = SC3]]$\\(\cref{sec:constraint-merge})};
    \node[right=1.4cm of 3] (5) {Unification\\ $[[Γ ; Θ ⊨ uN ≈u iM ⫤ UC]]$\\ $[[Γ ; Θ ⊨ uP ≈u iQ ⫤ UC]]$\\(\cref{sec:unification})};
    \node[below of=3] (4) {Upgrade\\$[[upgrade Γ ⊢ iP to Δ = iQ]]$\\(\cref{sec:lub})};
    \node[below of=2] (6) {Least Upper Bound\\$[[Γ ⊨ iP1 ∨ iP2 = iQ]]$\\(\cref{sec:lub})};
    \node[below of=6] (7) {Anti-Unification\\$[[Γ ⊨ iP1 ≈au iP2 ⫤ ( Ξ , uQ , aus1 , aus2 )]]$\\$[[Γ ⊨ iN1 ≈au iN2 ⫤ ( Ξ , uM , aus1 , aus2 )]]$\\(\cref{sec:antiunification})};
    \node[below of=5] (8) {Unification Constraint Merge\\$[[Θ ⊢ UC1 & UC2 = UC3]]$\\(\cref{sec:constraint-merge})};
    
    % Define edges
    \draw[->] (1) to (2);
    \draw[->] (1) to (3);
    \draw[->] (1) to (5);
    \draw[->] (2) to (3);
    \draw[->] (2) to (6);
    \draw[->] (3) to (4);
    \draw[->] (3) to (5);
    \draw[->] (4) to (6);
    \draw[->] (5) to (8);
    \draw[->] (6) to (7);
  \end{tikzpicture}  
  \caption{Dependency graph of the subtyping algorithm}
  \label{fig:subtyping-algo}
\end{figure}

\subsection{Subtyping}
\label{sec:subtyping}

Now, we present the subtyping algorithm itself.
Although the algorithm is presented as a single procedure,
is important for the structure of the proof that the positive subtyping algorithm
does not invoke the negative one. This way, the correctness of the positive 
subtyping will be proved independently and used afterwards to prove the
correctness of the negative subtyping.


\begin{algorithm}[Subtyping]
  \label{alg:subtyping}
  \hfill\\
  \ottdefnANsub{}
  \ottdefnAPsup{}
\end{algorithm}

The inputs of the subtyping algorithm are the declarative context $[[Γ]]$,
the subtyping context $[[Θ]]$ (it specifies in which contexts the algorithmic variables
must be instantiated), and the types themselves: $[[uN]]$ and $[[iM]]$ for the negative case,
and $[[uP]]$ and $[[iQ]]$ for the positive case. 
As one of the invariants, we require
$[[iM]]$ and $[[iQ]]$ to be declarative (i.e. not containing algorithmic variables).
The output of the algorithm is a set of \emph{subtyping constraints} $[[SC]]$,
which will be discussed in the next section.

Let us overview the inference rules of the subtyping algorithm.
\begin{itemize}
  \item \ruleref{\ottdruleANVarLabel} and \ruleref{\ottdruleAPVarLabel} 
    are the base cases. They copy the corresponding declarative rules and
    ensure reflexivity.
  \item \ruleref{\ottdruleAPUVarLabel} is the only case generating 
    subtyping constraints. In this case, we must ensure
    that the resulting constraints guarantee that the instantiation of 
    $[[â⁺]]$ is a supertype of $[[iP]]$.
    However, the obvious constraint $[[â⁺ :≥ iP]]$ might be problematic
    if  $[[iP]]$ is not well-formed in $[[Θ(â⁺)]]$. For this reason,
    we use the \emph{upgrade} procedure (it will be covered in \cref{sec:lub})
    to find the minimal supertype of $[[iP]]$, which is well-formed in $[[Θ(â⁺)]]$. 

    Notice that this rule does not have a negative counterpart. This is
    because one of the important invariants of the algorithm: 
    in the negative subtyping, only positive algorithmic variables
    can occur in the types. 
    
  \item \ruleref{\ottdruleAShiftDLabel} and \ruleref{\ottdruleAShiftULabel} are the
    \emph{shift} rules. According to the declarative system,
    shifted subtyping requires equivalence. In the presence of the algorithmic 
    variables, it means that the left and the right-hand sides of the subtyping
    must be unified. Hence, the shift rules invoke the unification algorithm, 
    which will be discussed in \cref{sec:unification}. The unification 
    returns the minimal set of constraints $[[UC]]$, which is necessary
    and sufficient for the subtyping. 

  \item \ruleref{\ottdruleAArrowLabel}.
    In this case, the algorithm makes two calls:
    a recursive call to the negative subtyping algorithm for the argument types,
    and a call to the positive subtyping algorithm for the result types.
    After that, the resulting constraints are merged using the
    \emph{subtyping constraint merge} procedure, 
    which is discussed in \cref{sec:constraint-merge}.
  \item \ruleref{\ottdruleAForallLabel} and \ruleref{\ottdruleAExistsLabel}
    are symmetric. These are the only places where 
    the algorithmic variables are introduced.
    It is done by algorithmization (\cref{sec:variable-algorithmization}) 
    of the quantified variables: these variables are replaced by 
    fresh algorithmic variables in the body of the quantified type,
    the algorithmic variables are added to the subtyping context $[[Θ]]$,
    after that, the recursive call is made. Notice that the declarative context
    $[[Γ]]$ is extended by the quantified variables from the right-hand side,
    which matches the declarative system.
\end{itemize}


Then soundness lemma (\cref{lemma:pos-subt-soundness,lemma:neg-subt-soundness}) 
and completeness (\cref{lemma:pos-subt-completeness,lemma:neg-subt-completeness})
of the algorithm together give us the following simplified theorem:

\begin{theorempreview}[Correctness of subtyping algorithm]
  \hfill
  \begin{itemize}
    \item [$-$]  $[[ Γ ; · ⊨ uN ≤ iM ⫤ · ]]$ is equivalent to $[[ Γ ⊢ iN ≤ iM ]]$;
    \item [$+$] $[[ Γ ; · ⊨ uP ≥ iQ ⫤ · ]]$ is equivalent to $[[ Γ ⊢ iP ≥ iQ ]]$.
  \end{itemize}
\end{theorempreview}


\subsection{Constraints}

Unification and subtyping algorithms are based on constraint generation.
The constraints are represented by a set of constraint entries.

\begin{definition}[Unification Constraint]
  \hfill
  \begin{description}
    \item[unification entry] (denoted as $[[ucE]]$) is an expression of shape 
      $[[pua :≈ iP]]$ or $[[nua :≈ iN]]$;
    \item[unification constraint] (denoted as $[[UC]]$) is a set of 
      unification constraint entries.
      We denote $\{[[α̂±]] \mid [[ucE ∊ UC]] \text{ restricting $[[α̂±]]$ }\}$ 
      as $[[dom(UC)]]$.
  \end{description}
\end{definition}

However, in the subtyping, we need to consider more general
kind of constraints. Specifically,
subtyping constraint entries can restrict a variable
not only to be equivalent to a certain type, but
also to be a supertype of a positive type.

\begin{definition}[Subtyping Constraint]
  \hfill
  \begin{description}
    \item[subtyping entry] (denoted as $[[scE]]$) is an expression of shape 
      $[[pua :≥ iP]]$, $[[nua :≈ iN]]$, or $[[pua :≈ iP]]$;
    \item[subtyping constraint] (denoted as $[[SC]]$) is a set of subtyping constraint entries.
      We denote $\{[[α̂±]] \mid [[scE ∊ SC]] \text{ restricting $[[α̂±]]$ }\}$ 
      as $[[dom(SC)]]$.
  \end{description}
\end{definition}

\begin{definition}[Well-formed Constraint Entry]
  We say that a constraint entry is well-formed in a context $[[Γ]]$ if
  its associated type is well-formed in $[[Γ]]$.
  \begin{itemize}
    \item[] $[[Γ ⊢ pua :≥ iP]]$ ~iff~ $[[Γ ⊢ iP]]$;
    \item[] $[[Γ ⊢ pua :≈ iP]]$ ~iff~ $[[Γ ⊢ iP]]$;
    \item[] $[[Γ ⊢ nua :≈ iN]]$ ~iff~ $[[Γ ⊢ iN]]$.
  \end{itemize}
\end{definition}


\begin{definition}[Well-formed Constraint]
  We say that a constraint is well-formed in a
  subtyping context $[[Θ]]$ if all its entries are well-formed in
  the corresponding elements of $[[Θ]]$.
  More formally, 
  $[[Θ ⊢ SC]]$ holds iff for every $[[scE]] \in $ $[[SC]]$,
  such that $[[scE]]$ restricts $[[α̂±]]$,
  we have $[[Θ(α̂±) ⊢ scE]]$.

  We write $[[Θ ⊢ SC : Ξ]]$ to denote
  that $[[Θ ⊢ SC]]$ and $[[dom(SC) = Ξ]]$.

  $[[Θ ⊢ UC]]$ and $[[Θ ⊢ UC : Ξ]]$ are defined analogously.
\end{definition}


\subsubsection{Constraint Satisfaction}

A constraint entry restricts a type that can be assigned to a variable.
We say that a type satisfies a constraint entry if it can be assigned
to the variable restricted by the entry.

\begin{definition}[Type Satisfying a Constraint Entry]
  \hfill\\
  \begin{multicols}{2}
  \ottdefnSATSCEN{}
  \columnbreak\\
  \ottdefnSATSCEP{}
  \end{multicols}
\end{definition}

We say that a substitution satisfies a constraint---a set of constraint 
entries if each entry is satisfied by the type assigned to the variable
by the substitution. 

\begin{definition}[Substitution Satisfying a Constraint]
  We write $[[Θ ⊢ uσ : SC]]$ to denote that
  a substitution $[[uσ]]$ satisfies a constraint $[[SC]]$ in a context $[[Θ]]$.
  It presumes that $[[Θ ⊢ SC]]$ and 
   means that for any $[[ucE]] \in [[SC]]$, if $[[ucE]]$ restricts $[[α̂±]]$,
  then $[[Θ(α̂±) ⊢ [uσ]α̂± : ucE]]$.


  Unification constraint satisfaction $[[Θ ⊢ uσ : UC]]$ 
  is defined analogously as a special case of subtyping constraint satisfaction.

\end{definition}


Notice that $[[Θ ⊢ uσ : SC]]$ does not 
imply the signature $[[Θ ⊢ uσ : dom(SC)]]$, because 
the latter also specifies $[[uσ]]$ outside of the domain $[[dom(SC)]]$
(see \cref{def:algo-subst-sig}).


\subsubsection{Constraint Merge}
\label{sec:constraint-merge}

In this section, define the least upper bound 
for constraints, which we call \emph{merge}.
Intuitively, the merge of two constraints is the least
constraint such that any substitution satisfying both constraints
satisfies the merge as well.
First, we define the merge of entries,
and then extend it to the set of entries.

\begin{definition} [Matching Entries]
  We call two unification constraint entries 
  or two subtyping constraint entries matching 
  if they are restricting the same unification variable.
\end{definition}

Two matching entries formed in the same context $[[Γ]]$ 
can be merged in the following way:
\begin{algorithm}[Merge of Matching Constraint Entries]
  \label{definition:merge-matching-entries}
   \hfill 

  \ottdefnSCME\\
\end{algorithm}

\begin{itemize}
  \item \ruleref{\ottdruleSCMEPEqEqLabel} and \ruleref{\ottdruleSCMENEqEqLabel}
    are symmetric cases. To merge two matching entries restricting
    a variable to be equivalent to certain types, we check
    that these types are equivalent to each other.
    To do so, it suffices to check for \emph{equality} of their normal forms,
    as discussed in \cref{sec:equivalence-normalization}. 
    After that, we return the left-hand entry.

  \item \ruleref{\ottdruleSCMEEqSupLabel} and \ruleref{\ottdruleSCMESupEqLabel}
    are also symmetric. 
    In this case,
    since one of the entries requires the variable to be equal to 
    a type, the resulting entry must also imply that.
    However, for the soundness, it is needed to ensure that
    the equating restriction is stronger than the subtyping restriction.
    For this purpose, the premise invokes the positive subtyping.

  \item \ruleref{\ottdruleSCMESupSupLabel} 
    In this case, we find the least upper bound of the types from the input
    restrictions, 
    and as the output, restrict the variable to be a supertype of the result.
    The least upper bound procedure will be discussed in \cref{sec:lub}.
\end{itemize}


Unification constraint entries are a special case of subtyping constraint
entries. They are merged using the same algorithm 
(\cref{definition:merge-matching-entries}).
Notice that the merge of two matching unification constraint entries
is a unification constraint entry.
\begin{lemma}[Merge of Matching Unification Constraint Entries is well-defined]
  \label{lemma:merge-matching-entries-welldef}
  Suppose that $[[Γ ⊢ ucE1]]$ and $[[Γ ⊢ ucE2]]$
  are unification constraint entries. 
  Then the merge of $[[ucE1]]$ and $[[ucE2]]$ 
  $[[Γ ⊢ ucE1 & ucE2 = ucE]]$
  according to \cref{definition:merge-matching-entries},
  is a unification constraint entry.
\end{lemma}
\begin{proof}
  Since $[[ucE1]]$ and $[[ucE2]]$ are matching unification constraint entries,
  they have the shape $([[pua :≈ iP1]], [[pua :≈ iP2]])$ or
  $([[nua :≈ iN1]], [[nua :≈ iN2]])$.
  Then the merge of $[[ucE1]]$ and $[[ucE2]]$ 
  can only be defined by \ruleref{\ottdruleSCMEPEqEqLabel} or
  \ruleref{\ottdruleSCMENEqEqLabel}.
  In both cases the result, if it exists, 
  is a unification constraint entry:
  in the first case, the result has shape $[[pua :≈ iP1]]$,
  in the second case, the result has shape $[[nua :≈ iN1]]$.
\end{proof}


% Notice that in case of equivalence, the assigned types
% must be equal (i.e. alpha-equivalent) to be merged. This is because
% the unification algorithm assumes that every type is normalized,
% and hence, equivalence is alpha-equivalence 
% (\cref{corollary:nf-complete-wrt-subt-equiv,corollary:nf-sound-wrt-subt-equiv}).

\begin{algorithm}[Merge of Subtyping Constraints]
  \label{definition:merge-subtyping-constraints}
  Suppose that $[[Θ ⊢ SC1]]$ and $[[Θ ⊢ SC2]]$.\\
  Then $[[Θ ⊢ SC1 & SC2 = SC]]$
  defines a set of constraints $[[SC]]$ such that $[[scE]] \in [[SC]]$ iff either:
  \begin{itemize}
    \item $[[scE]] \in [[SC1]]$ and there is no matching $[[scE']] \in [[SC2]]$; or
    \item $[[scE]] \in [[SC2]]$ and there is no matching $[[scE']] \in [[SC1]]$; or
    \item $[[Θ(α̂±) ⊢ scE1 & scE2 = scE]]$ for some $[[scE1]] \in [[SC1]]$ and $[[scE2]] \in [[SC2]]$
      such that $[[scE1]]$ and $[[scE2]]$ both restrict variable $[[α̂±]]$. 
  \end{itemize}
\end{algorithm}

Unification constraints can be considered 
as a special case of subtyping constraints,
and the merge of unification constraints
is defined as the merge of subtyping constraints.
Then it is easy to see that the merge of two 
unification constraints is a unification constraint.

\begin{lemma}[Merge of Unification Constraints is well-defined]
  Suppose that $[[Θ ⊢ UC1]]$ and $[[Θ ⊢ UC2]]$
  are unification constraints. 
  Then the merge of $[[UC1]]$ and $[[UC2]]$ 
  $[[Θ ⊢ lift UC1 & lift UC2 = lift UC]]$
  according to \cref{definition:merge-subtyping-constraints},
  is a unification constraint.
\end{lemma}
\begin{proof}
  $[[UC]]$ consists of unmatched entries of $[[UC1]]$ and $[[UC2]]$,
  which are \emph{unification} constraint entries by assumption,
  and merge of matching entries, which also are  
  \emph{unification} constraint entries by \cref{lemma:merge-matching-entries-welldef}.
\end{proof}

\Cref{lemma:merge-soundness,lemma:merge-completeness} 
show the correctness and initiality of the merge operation,
which can be expressed in the following simplified theorem:
\begin{theorempreview}[Correctness of Constraint Merge]
  A substitution $[[uσ]]$ satisfying both constraints
  $[[SC1]]$ and $[[SC2]]$ 
  if and only if it satisfies their merge.
\end{theorempreview}

The unification constraint merge satisfies the same theorem,
however, because the merge of unification constraint entries 
$[[ucE1]]$ and $[[ucE2]]$ always results in one of them, 
a stronger soundness property holds (see \cref{lemma:unif-merge-soundness}):
\begin{theorempreview}[Soundness of Unification Constraint Merge]
  If $[[Θ ⊢ UC1 & UC2 = UC]]$ then $[[UC = UC1 ∪ UC2]]$.
\end{theorempreview}

\subsection{Unification}
\label{sec:unification}

The subtyping algorithm calls the following subtask:
given two algorithmic types, we need to find the most general substitution 
for the algorithmic variables in these types, such that the resulting 
types are equivalent. This problem is known as \emph{unification}.

In our case, the unification is restricted in the following way:
first, before unifying the types, we normalize them, which 
allows us to reduce (non-trivial) equivalence to (trivial) equality;
second, we preserve invariants which guarantee that
one side of the unification is always declarative, which in fact, 
reduces the unification to the \emph{matching} problem.

The unification procedure
returns a set of minimal constraints,
that must be satisfied by a substitution
unifying the input types.

\begin{algorithm}[Unification]
  \hfill
  \begin{multicols}{2}
  \ottdefnUNUnif{}
  \columnbreak\\
  \ottdefnUPUnif{}
  \end{multicols}
\end{algorithm}


\begin{itemize}
  \item \ruleref{\ottdruleUShiftULabel}, \ruleref{\ottdruleUShiftDLabel}, 
    \ruleref{\ottdruleUForallLabel}, and \ruleref{\ottdruleUExistsLabel}
    are defined congruently. In the shift rules, the algorithm
    removes the outermost constructor. In the
    $[[∀]]$ and $[[∃]]$ rules, it removes the quantifiers,
    adding the quantified variables to the context $[[Γ]]$.
    Notice that $[[Θ]]$, which specifies
    the contexts in which the algorithmic variables must be instantiated,
    is not changed.
  \item \ruleref{\ottdruleUNVarLabel} and \ruleref{\ottdruleUPVarLabel} 
    are the base cases. 
    Since the sides are equal and free from algorithmic variables,
    the unification returns an empty constraint. 
  \item \ruleref{\ottdruleUNVarLabel} and \ruleref{\ottdruleUPVarLabel}
    are symmetric cases constructing the constraints. 
    When an algorithmic variable is unified with a type, 
    we must check that the type is well-formed in the required context,
    and if it is, we return a constraint restricting the variable
    to be equivalent to that type.
  \item \ruleref{\ottdruleUArrowLabel}.
    In this case, the algorithm makes two recursive calls:
    it unifies the arguments and the results of the arrows.
    After that, the resulting constraints are merged using the
    \emph{unification constraint merge} procedure, 
    which is discussed in \cref{sec:constraint-merge}.
    Notice that $[[UC1]]$ and $[[UC2]]$ are guaranteed to be
    \emph{unification} constraints, not arbitrary \emph{subtyping} 
    constraints: it is important for modularizing the proofs, 
    since the properties of the \emph{unification} constraint merge
    can be proved independently from the \emph{subtyping} constraint merge.
\end{itemize}

\subsection{Least Upper Bound}
\label{sec:lub}

In this section, we present
the algorithm finding the least common supertype of two positive types. 
It is used directly by the constraint merge procedure (\cref{sec:constraint-merge}),
and indirectly, through the type upgrade by positive subtyping
(\cref{sec:subtyping}). Perhaps, the least upper bound is the least 
intuitive part of the algorithm, and its correctness will be covered
in \cref{sec:alg-upper-bounds-proofs}.

\begin{algorithm}[The Least Upper Bound Algorithm]
  \hfill\\
  \ottdefnLUBNsub{}
\end{algorithm}

\begin{itemize}
  \item \ruleref{\ottdruleLUBVarLabel}
    The base case is trivial: 
    the least upper bound of to equal variables
    is the variable itself.
  \item \ruleref{\ottdruleLUBShiftLabel}
    In case both sides of the least upper bound are shifted,
    the algorithm needs to find the anti-unifier of them. 
    Intuitively, this is because in general, the upper bounds of
    $[[↓iN]]$ are $[[∃nas.iP]]$ such that 
    $[[nas]]$ can be instantiated with some $[[iMs]]$ so that
    $[[ Γ ⊢ [iMs/nas]iP ≈ ↓iN ]]$ (see \cref{lemma:shape-of-supertypes}).
  \item \ruleref{\ottdruleLUBExistsLabel}
    In this case, we move the quantified variables to the context $[[Γ]]$, 
    and make a recursive call. 
    It is important to make sure that $[[nas]]$ and $[[nbs]]$ are disjoint.
    In this case, it is guaranteed that the resulting 
    $[[fv(iQ)]]$ will be free of $[[nas]]$ and $[[nbs]]$,
    and thus, the resulting type will be a supertype of both sides
    (it will be discussed in \cref{lemma:shape-of-supertypes}).
\end{itemize}


In the positive subtyping algorithm (\cref{sec:subtyping}),
\ruleref{\ottdruleAPUVarLabel} 
generates a restriction of a variable $[[α̂⁺]]$.
On the one hand, this restriction must imply 
$[[â⁺ :≥ iP]]$ for the subtyping to hold.
On the other hand, the type used in this restriction 
must be well-formed in a potentially stronger (smaller) 
context than $[[iP]]$.

To resolve this problem, we define the \emph{upgrade} procedure,
which for given $[[Δ]]$, $[[pnas]]$, and $[[Δ, pnas ⊢ iP]]$,
finds $[[Δ ⊢ iQ]]$---the least supertype of $[[iP]]$ 
among the types well-formed in $[[Δ]]$.

The trick is to make sure that the `forbidden' variables
$[[pnas]]$ are not used explicitly in the supertypes
of $[[iP]]$. For this purpose, we
construct new types $[[iP1]]$ and $[[iP2]]$,
in each of them replacing the forbidden variables
with fresh variables $[[pnbs]]$ and $[[pncs]]$,
and then find the least upper bound of $[[iP1]]$ and $[[iP2]]$.
It turns out that this renaming forces the common types of 
$[[iP1]]$ and $[[iP2]]$ to be agnostic to $[[pnas]]$,
and thus, the supertypes of $[[iP]]$ well-formed in $[[Δ]]$
are exactly the common supertypes of $[[iP1]]$ and $[[iP2]]$.
These properties are considered in more details in \cref{sec:upgrade-lemmas}.

\begin{algorithm}[Type Upgrade]
  \hfill\\
  \ottdefnLUBUp{}
\end{algorithm}

\begin{paragraph}{Note on the Greatest Lower Bound}
  In contrast to the least upper bound, 
  the general greatest lower bound does not exist in our system.
  For instance, consider a positive type $[[iP]]$, 
  together with its non-equivalent
  supertypes $[[iP1]]$ and $[[iP2]] \not\simeq [[iP1]]$
  (for example, $[[iP = ↓↑↓γ⁻]]$, $[[iP1 = ∃α⁻.↓↑↓α⁻]]$, 
  and $[[iP2 = ∃α⁻.↓α⁻]]$).
  Then for arbitrary $[[iQ]]$ and $[[iN]]$, 
  let us consider the common subtypes of 
  $A = [[iQ → ↓↑iQ → ↓↑iQ → iN]]$ and $B = [[iP → ↓↑iP1 → ↓↑iP2 → iN]]$.
  It is easy to see that $[[∀α⁺.∀β⁺. α⁺ → ↓↑α⁺ → ↓↑β⁺ → iN]]$ and 
  $[[∀α⁺.∀β⁺. α⁺ → ↓↑β⁺ → ↓↑α⁺ → iN]]$ are
  both \emph{maximal} common subtypes of $A$ and $B$,
  and since they are not equivalent, none of them is 
  the \emph{greatest} one.

  However, we designed the subtyping system in such a way 
  that the greatest lower bound is not needed:
  the negative variables are always `protected'
  by \emph{invariant} shifts ($[[↑]]$ and $[[↓]]$), 
  and thus, the algorithm can only require
  a substitution of a negative variable to be 
  \emph{equivalent} to some type but never 
  to be a \emph{subtype}.
\end{paragraph}

\subsection{Anti-unification}
\label{sec:antiunification}

Next, we define the anti-unification procedure,
also known as the \emph{most specific generalization}.
As an input, it takes two declarative types
(e.g., in the positive case $[[iP1]]$ and $[[iP2]]$)
and a context $[[Γ]]$.
and returns a type $[[uQ]]$---the generalizer,
containing negative placeholders 
(represented by algorithmic variables) from 
$[[Ξ]]$ and two substitutions $[[uτ1]]$ and $[[uτ2]]$.
The substitutions replace the placeholders with 
declarative types well-formed in $[[Γ]]$,
such that $[[ [uτ1]uQ = iP1 ]]$ and $[[ [uτ2]uQ = iP2 ]]$.
Moreover, the algorithm guarantees that 
$[[uQ]]$ is the most specific type with this property: 
any other generalizer can be turned into $[[uQ]]$ by some substitution
$[[uρ]]$.

It is important to note the differences between 
the standard anti-unification and our version.
First, we only allow the placeholders at \emph{negative} positions,
which means, for example, that $[[α⁺]]$ and $[[β⁺]]$ cannot be
generalized. Second, the generated pair of substitutions 
$[[uτ1]]$ and $[[uτ2]]$ must replace the placeholders with 
types well-formed in a specified context $[[Γ]]$.

The anti-unification algorithm assumes that the input types 
are normalized. This way, anti-unification up-to-equality rather than 
 anti-unification up-to-equivalence is sufficient.

\begin{algorithm}[Anti-unification]
  \hfill
  
  \ottdefnAUAUP{}
  \ottdefnAUAUN{}

\end{algorithm}

\begin{itemize}
  \item \ruleref{\ottdruleAUPVarLabel} and \ruleref{\ottdruleAUNVarLabel}
    are the base cases. 
    In this case, since the input types are equal, 
    the algorithm returns this type as a generalizer,
    without generating any placeholders.

  \item \ruleref{\ottdruleAUShiftDLabel}, \ruleref{\ottdruleAUShiftULabel},
    \ruleref{\ottdruleAUForallLabel}, and \ruleref{\ottdruleAUExistsLabel}
    are defined congruently. In the shift rules, the algorithm
    removes the outermost constructor. In the
    $[[∀]]$ and $[[∃]]$ rules, it removes the quantifiers.
    Notice that the algorithm does not add the removed variables to
    the context $[[Γ]]$. This is because $[[Γ]]$
    is used to restrict the resulting anti-unification substitutions, 
    and is fixed throughout the algorithm.

  \item \ruleref{\ottdruleAUAULabel} is the most important rule, 
    since it generates the placeholders. 
    This rule only applies if other negative rules failed.
    Because of that, the anti-unification procedure is 
    \emph{not} syntax-directed. 

    The generated placeholder is indexed with a pair of 
    types it is mapped to. It allows the algorithm to 
    automatically unite the anti-unification solutions 
    generated by the different branches of 
    \ruleref{\ottdruleAUArrowLabel}.

    Notice that this rule does not have a positive counterpart,
    since we only allow negative placeholders.

  \item \ruleref{\ottdruleAUArrowLabel}
    makes two recursive calls to the anti-unification procedure,
    and unites the results. Suppose that
    $[[uτ1]]$  and $[[uτ2]]$ are the substitutions generated by
    anti-unification of \emph{argument} types of the arrow,
    and $[[uτ1']]$ and $[[uτ2']]$ are the substitutions generated by 
    anti-unification of \emph{result} types of the arrow.
    It is important that if ($[[uτ1]]$, $[[uτ2]]$)
    and ($[[uτ1']]$, $[[uτ2']]$) send some variables to the same pair of types,
    i.e., $[[ [uτ1]α̂⁻ = [uτ1']β̂⁻]]$ and $[[ [uτ2]α̂⁻ = [uτ2']β̂⁻]]$,
    then these variables are equal, i.e., $[[α̂⁻ = β̂⁻]]$.
    This property is guaranteed by \ruleref{\ottdruleAUAULabel}:
    the name of the placeholder is determined by the pair of 
    types it is mapped to.
\end{itemize}

\newpage

\section{Declarative Typing}

In the previous section, we presented the 
type system together with subtyping specification 
and the algorithm. In this section, 
we describe the language under this type system, 
together with the type inference specification
and algorithm.  

\subsection{Grammar}

First, we define the syntax of the language.
The language combines System F with call-by-push-value 
style. 
\begin{definition}[Language Grammar]
  \hfill
  \begin{multicols}{2}
    \ottgrammartabular{
      \ottc\ottinterrule
    }

    \ottgrammartabular{
      \ottv\ottinterrule
    }
  \end{multicols}
\end{definition}
Notice that the language does not have
first-class applications: instead, 
we use applicative let bindings---
constructions that bind a result of
a fully applied function to a (positive) 
variable.
In the call-by-push-value paradigm, 
it corresponds to monadic bind or do-notation. 
Typewise, these let-binders come in two forms:
annotated and unannotated. The annotated let-binders
$[[let x:iP = v(args); c]]$ 
requires the application to infer the annotated
$[[iP]]$, whereas the unannotated 
$[[let x = v(args); c]]$ 
is used when the inferred type is unique. 

A computation of a polymorphic type is constructed 
using $[[Λα⁺ . c]]$, however, the elimination of $[[∀]]$
is implicit. Conversely, the existential types
are constructed implicitly and eliminated 
using the standard unpack mechanism:
$[[let∃ (nas, x) = v; c]]$.

Another dual pair of constructions are 
$[[return v]]$ and $[[{c}]]$. 
The former allows us to embed a value in pure 
computations. The latter, on the contrary,
encapsulates a thunk of computation in a value. 

Finally, the language has several standard constructions:
lambda-abstractions $[[λx:iP.c]]$,
standard let-bindings $[[let x = v; c]]$,
and type annotations that can be added to any value or computation:
$[[(v:iP)]]$ and $[[(c:iN)]]$.

\subsection{Declarative Type Inference}

Next, we define the specification of the type 
inference for our language. First, we introduce 
variable context specifying the types of variables 
in the scope of the current rule. 

\begin{definition}[Variable Context]
  The variable typing context $[[Φ]]$
  is represented by a set of entries of the form
  $[[x : iP]]$. 
\end{definition}

The specification is represented by an inference system of
three mutually recursive judgments:
positive inference $[[Γ ; Φ ⊢ v : iP]]$,
negative type inference $[[Γ ; Φ ⊢ c : iN]]$, and
application type inference $[[Γ ; Φ ⊢ iN ● args ⇒> iM ]]$.
In the premises, the inference rules also refer to 
the declarative subtyping (\cref{def:subDOne}),
type well-formedness (\cref{alg:wf}), 
and normalization (\cref{alg:type-nf}).
\begin{itemize}
  \item $[[Γ ; Φ ⊢ v : iP]]$ (and symmetrically, $[[Γ ; Φ ⊢ c : iN]]$)
    means that under the type context $[[Γ]]$ and
    the variable context $[[Φ]]$, for the value $[[v]]$,
    type $[[iP]]$ is inferrable. It guarantees that 
    $[[v]]$ is well-formed in $[[Γ]]$ and $[[Φ]]$ in
    the standard sense.
  \item $[[ Γ ; Φ ⊢ iN ● args ⇒> iM ]]$ is the application type inference 
    judgment. It means that if a head of type $[[iN]]$ 
    is applied to list of values $[[args]]$, 
    then the resulting computation can be typed as $[[iM]]$.
\end{itemize}

\begin{definition}[Declarative Type Inference]
  \label{def:declarative-typing}
  \hfill
  \begin{multicols}{2}
  \ottdefnDTNInf{}
  \ottdefnDTPInf{}
  \ottdefnDTSpinInf{}
  \end{multicols}
\end{definition}

Let us discuss selected rules of the declarative system:
\begin{itemize}
  \item \ruleref{\ottdruleDTVarLabel}
    says that the type of a variable is inferred from the context.
  \item \ruleref{\ottdruleDTThunkLabel} says
    that the type of a thunk is inferred by shifting up the type of the 
    contained computation. Symmetrically, \ruleref{\ottdruleDTReturnLabel}
    infers the type of a return by shifting down the type of the
    contained value.
  \item \ruleref{\ottdruleDTPAnnotLabel} and \ruleref{\ottdruleDTNAnnotLabel} are symmetric.
    They allow the inferred type to be refined by annotating it with a supertype.
  \item \ruleref{\ottdruleDTNEquivLabel} and \ruleref{\ottdruleDTPEquivLabel}
    mean that th declarative system allows to infer any type from the equivalence class.
  \item \ruleref{\ottdruleDTUnpackLabel} is standard for existential types,
    and its first premise inferring the existential type of the value being unpacked.
    It is important however that the inferred existential type is normalized. 
    This is because there might be multiple equivalent existential types 
    with different order or even number of quantified variables, 
    and to bind them, the algorithm needs to fix the canonical one.
  \item \ruleref{\ottdruleDTAppLetAnnLabel} allows us to accommodate the applications
    with annotated let-bindings. The first premise infers the type of the head of the application,
    which must be a thunked computation. Then if after applying it 
    to the arguments, the resulting type can be equated to the annotated one,
    we infer the body of the let-binding in the context extended with the bound variable.
  \item \ruleref{\ottdruleDTAppLetLabel} is similar to \ruleref{\ottdruleDTAppLetAnnLabel},
    but it is used when the type of the application is unique, and thus, the annotation
    is redundant. Here $[[Γ ; Φ ⊢ iM ● args ⇒> ↑iQ uniq]]$ means that
    if also $[[Γ ; Φ ⊢ iM ● args ⇒> ↑iQ']]$ then $[[Γ ⊢ iQ ≈ iQ']]$
\end{itemize}

Let us discuss the rules of the application inference:
\begin{itemize}
  \item \ruleref{\ottdruleDTEmptyAppLabel} 
    is the base case. If the list of arguments is empty, 
    the inferred type is the type of the head. 
    However, we relax this specification by allowing it to 
    infer any other equivalent type. 
    The relaxation of this rule is enough to guarantee 
    this property for the whole judgement:
    if $[[Γ ; Φ ⊢ iN ● args ⇒> iM]]$ then 
    $[[Γ ; Φ ⊢ iN ● args ⇒> iM']]$ for any equivalent 
    $[[iM']]$.
  \item \ruleref{\ottdruleDTArrowAppLabel}
    is where the application type is inferred: 
    if the head has an arrow type $[[iQ → iN]]$,
     we are allowed to apply it as soon as 
    as soon as the first argument has a type, which is a subtype of $[[iQ]]$.
  \item \ruleref{\ottdruleDTForallAppLabel}
    is the rule ensuring the implicit elimination of the universal quantifiers. 
    If we are applying a polymorphic computation, 
    we can instantiate its quantified variables with any types,
    which is expressed by the substitution $[[Γ ⊢ σ : {pas}]]$.
\end{itemize}



\section{Algorithmic Typing}

Next, we present the type inference algorithm, 
which is sound and complete with respect to the declarative specification
(\cref{def:declarative-typing}).

\subsection{Algorithmic Type Inference}

Mirroring the declarative typing, 
the algorithm is represented by an inference system of three mutually recursive
judgments:
\begin{itemize}
  \item $[[Γ ; Φ ⊨ v : iP]]$ and  $[[Γ ; Φ ⊨ c : iN]]$
    are the algorithmic versions of $[[Γ ; Φ ⊢ v : iP]]$ and $[[Γ ; Φ ⊢ c : iN]]$.
    In contrast with the declarative counterparts, they are deterministic,
    and guarantee that the inferred type is normalized. 
  \item $[[Γ ; Φ ; Θ1 ⊨ uN ● args ⇒> uM ⫤ Θ2 ; SC]]$
    is the algorithmization of $[[Γ ; Φ ⊢ iN ● args ⇒> iM]]$.
    Notice that $[[uN]]$ contains algorithmic variables, 
    which are specified by the context $[[Θ1]]$.
    Moreover, the inferred type $[[uM]]$ is also algorithmic,
    and can have several non-equivalent instantiations. To accommodate that, 
    the algorithm also returns $[[Θ2]]$ and $[[SC]]$ specifying 
    the variables used in $[[uM]]$: $[[Θ2]]$ defines the contexts
    in which the variables must be instantiated, and $[[SC]]$
    imposes restrictions on the variables. 
\end{itemize}
As subroutines, the algorithm calls
subtyping (\cref{alg:subtyping}),
type well-formedness (\cref{alg:wf}),
constraint merge (\cref{sec:constraint-merge}),
normalization (\cref{alg:type-nf}),
and constraint singularity which will be defined later in 
\cref{sec:constraint-singularity}.
It also relies on basic set operations and the ability to 
deterministically choose fresh variables.

\begin{algorithm}
  \hfill\\
  \ottdefnATNInf{}
  \ottdefnATPInf{}
  \ottdefnATSpinInf{}
\end{algorithm}

Let us discuss the inference rules of the algorithm:
\begin{itemize}
  \item \ruleref{\ottdruleATVarLabel} 
    infers the type of a variable by looking it up in the context
    and normalizing the result.
  \item \ruleref{\ottdruleATThunkLabel} and \ruleref{\ottdruleATReturnLabel}
    are similar to the declarative rules: they make a recursive call
    to type the body of the thunk or the return expression and
    put the shift on top of the result.
  \item \ruleref{\ottdruleATPAnnotLabel} and \ruleref{\ottdruleATNAnnotLabel}
    are symmetric. They make a recursive call to infer the type of the annotated
    expression, check that the inferred type is a subtype of the annotation,
    and return the normalized annotation.
  \item \ruleref{\ottdruleATtLamLabel} infers the type of a lambda-abstraction.
    It makes a recursive call to infer the type of the body in the extended context,
    and returns the corresponding arrow type. Notice that the algorithm also
    normalizes the result, which is because the annotation type $[[iP]]$
    is allowed to be non-normalized.
  \item \ruleref{\ottdruleATTLamLabel} infers the type of a big lambda.
    Similarly to the previous case, it makes a recursive call to infer the type
    of the body in the extended \emph{type} context. 
    After that, it returns the corresponding universal type. 
    It is also required to normalize the result, because, 
    for instance, $[[α⁺]]$ might not occur in the body of the lambda,
    in which case the $[[∀]]$ must be removed. 
  \item \ruleref{\ottdruleATVarLetLabel} is defined in a standard way:
    it makes a recursive call to infer the type of the bound value,
    and then returns the type of the body in the extended context.
  \item \ruleref{\ottdruleATAppLetAnnLabel}
    is interpreted as follows.
    First, it infers the type of the head of the application,
    ensuring that it is a thunked computation $[[↓iM]]$;
    after that, it makes a recursive call
    to the application inference procedure,
    which returns the algorithmic type, whose
    instantiation to a declarative type must be associated with the bound variable 
    $[[x]]$; then premise $[[Γ; Θ ⊨ ↑uQ ≤ ↑iP ⫤ SC2]]$
    together with $[[Θ ⊢ SC1 & SC2 = SC]]$
    check whether the instantiation to the annotated type $[[iP]]$ is possible,
    and if it is, the algorithm infers the type of the body in the extended context,
    and returns it as the result. 
  \item \ruleref{\ottdruleATAppLetAnnLabel}
    works similarly to \ruleref{\ottdruleATAppLetAnnLabel},
    However, since there is no annotation, 
    instead of checking the instantiation to it, 
    the algorithm checks that the inferred type
    $[[↑uQ]]$ is unique.
    It is the case if all the algorithmic variables of 
    $[[↑uQ]]$ are sufficiently restricted by $[[SC]]$,
    which is checked by the combination of
    $[[uv uQ = dom(SC)]]$ and $[[SC singular with uσ]]$.
    Together, these two premises guarantee that the only 
    possible instantiation of $[[↑uQ]]$ is $[[ [uσ]uQ ]]$.
  \item \ruleref{\ottdruleATUnpackLabel}
    works in the expected way. First, it infers the 
    existential type $[[∃nas.iP]]$ of the value being unpacked,
    and since the type is guaranteed to be normalized, binds 
    the quantified variables with $[[nas]]$.
    Then it infers the type of the body in the appropriately extended context,
    and checks that the inferred type does not depend on $[[nas]]$
    by checking well-formedness $[[Γ ⊢ iN]]$.
\end{itemize}

Finally, let us discuss the algorithmic rules of the application inference:
\begin{itemize}
  \item \ruleref{\ottdruleATEmptyAppLabel}
    is the base case. If the list of arguments is empty, 
    the inferred type is the type of the head,
    and the algorithm returns it after normalizing.
  \item \ruleref{\ottdruleATArrowAppLabel}
    is the main rule of algorithmic application inference.
    If the head has an arrow type $[[uQ → uN]]$,
    we find $[[SC1]]$---the minimal constraint ensuring that 
    $[[uQ]]$ is a supertype of the first argument's type.
    Then we make a recursive call applying $[[uN]]$ to the rest of the arguments,
    and merge the resulting constraint with $[[SC1]]$
  \item \ruleref{\ottdruleATForallAppLabel},
    analogously to the declarative case,
    is the rule ensuring the implicit elimination of the universal quantifiers. 
    This is the place where the algorithmic variables are generated.
    The algorithm simply replaces the quantified variables 
    $[[pas]]$ with fresh algorithmic variables $[[puas]]$,
    and makes a recursive call in the extended context. 
\end{itemize}

The correctness of the algorithm consists of its soundness and 
completeness, which is by mutual
induction in \cref{lemma:typing-soundness,lemma:typing-completeness}.
The simplified result is the following.
\begin{theorempreview}
  \hfill
  \begin{itemize}
    \item [$-$] $[[Γ; Φ ⊨ c : iN]]$ implies $[[Γ; Φ ⊢ c : iN]]$, 
      and $[[Γ; Φ ⊢ c : iN]]$ implies $[[Γ; Φ ⊨ c : nf(iN)]]$;
    \item [$+$] $[[Γ; Φ ⊨ v : iP]]$ implies $[[Γ; Φ ⊢ v : iP]]$, 
      and $[[Γ; Φ ⊢ v : iP]]$ implies $[[Γ; Φ ⊨ v : nf(iP)]]$.
  \end{itemize}
\end{theorempreview}


\subsection{Constraint Singularity}
\label{sec:constraint-singularity}

The singularity algorithm used in  \ruleref{\ottdruleATAppLetAnnLabel}
of the algorithmic typing check whether the constraint $[[SC]]$
is uniquely defines a substitution satisfying it, and if it does
returns such a substitution as the result.
To do that, we define a partial function $[[SC singular with uσ]]$,
taking a subtyping constraint $[[SC]]$ and returning a substitution 
$[[uσ]]$---the only possible solution of $[[SC]]$. 

First, we define the notion of singularity on constraint entries. 
$[[scE singular with iP]]$ and $[[scE singular with iN]]$
are considered partial functions taking a constraint entry $[[scE]]$
and returning the type satisfying $[[scE]]$ if such a type is unique. 

\begin{algorithm}[Singular Constraint Entry]
  \hfill\\
  \ottdefnSINGscEP{}
  \ottdefnSINGscEN{}
\end{algorithm}

\begin{itemize}
  \item \ruleref{\ottdruleSINGNEqLabel} and \ruleref{\ottdruleSINGPEqLabel} 
    are symmetric. If the constraint entry says that a variable must be equivalent to 
    a type $T$, then it is evidently singular, and the only (up-to-equivalence) type
    instantiating this variable could be $T$. This way, we return its normal form. 
  \item \ruleref{\ottdruleSINGSupVarLabel}
    implies that the only (normalized) solution of $[[pua :≥ ∃nas.pa]]$ is 
    $[[pa]]$ (it will be shown in \cref{lemma:var-subt}).
  \item \ruleref{\ottdruleSINGSupShiftLabel}
    is perhaps the least obvious rule.
    In type $[[∃nas.↓iN]]$, if $[[iN]]$ is anything different (non-equivalent) to 
    $[[nai ∊ {nas}]]$, there are at least 
    one proper supertype of $[[∃nas.↓iN]]$, since $[[iN]]$ can be abstracted over
    by an existential quantifier. Otherwise, any supertype of $[[∃nas.↓nai]]$
    is equivalent to it, and thus, the solution is unique. 
\end{itemize}

Next, we extrapolate the singularity function on constraints---sets of constraint entries. 
We require $[[SC]]$ to be a set of singular constraints, and the resulting substitution
sends each variable from $[[dom(SC)]]$ to the unique type satisfying the corresponding constraint.

\begin{algorithm}
  $[[SC singular with uσ]]$
  means that 
  \begin{enumerate}
    \item for any positive $[[scE ∊ SC]]$,
      there exists $[[iP]]$ such that $[[scE singular with iP]]$, 
      and for any negative $[[scE ∊ SC]]$,
      there exists $[[iN]]$ such that $[[scE singular with iN]]$;
    \item $[[uσ]]$ is defined as follows:
      $$
      [[ [uσ]β̂⁺ ]]  = 
          \begin{cases}
              [[ iP ]] & \text{if there is } [[scE]] \in [[dom(SC)]] \text{ restricting } [[β̂⁺]] 
                         \text{ and } [[scE singular with iP]] \\
              [[ β̂⁺ ]] & \text{otherwise}  \\
          \end{cases}
      $$
      $$
      [[ [uσ]β̂⁻ ]]  = 
          \begin{cases}
              [[ iN ]] & \text{if there is } [[scE]] \in [[dom(SC)]] \text{ restricting } [[β̂⁻]] 
                         \text{ and } [[scE singular with iN]]\\
              [[ β̂⁻ ]] & \text{otherwise}  \\
          \end{cases}
      $$
  \end{enumerate} 
\end{algorithm}

The correctness of the singularity algorithm is formulated as follows:
\begin{theorempreview}
  Suppose that $[[SC]]$ is a subtyping constraint.
  Then $[[SC singular with uσ]]$ holds if and only if 
  $[[uσ]]$ is the only (up-to-equivalence on $[[dom(SC)]]$) 
  normalized substitution satisfying $[[SC]]$.
\end{theorempreview}


\newpage

\section{Properties of the Declarative Type System}

\subsection{Type Well-formedness}
\begin{lemma}[Well-formedness agrees with substitution]
  \label{lemma:wf-subst}
  Suppose that $[[Γ2 ⊢ σ : Γ1]]$. Then
  \begin{itemize}
  \item[$+$] $[[Γ, Γ1 ⊢ iP]] ~\Leftrightarrow~ [[Γ, Γ2 ⊢ [σ]iP]]$
  \item[$-$] $[[Γ, Γ1 ⊢ iN]] ~\Leftrightarrow~ [[Γ, Γ2 ⊢ [σ]iN]]$
  \end{itemize}
\end{lemma}
\begin{proof}
  \ilyam{todo}
\end{proof}


\begin{corollary}
  \label{lemma:wf-subst}
  Suppose that $[[Γ2 ⊢ σ : Γ1]]$. Then
  \begin{itemize}
  \item[$+$] $[[Γ1, Γ2 ⊢ iP]] ~\Leftrightarrow~ [[Γ2 ⊢ [σ]iP]]$
  \item[$-$] $[[Γ1, Γ2 ⊢ iN]] ~\Leftrightarrow~ [[Γ2 ⊢ [σ]iN]]$
  \end{itemize}
\end{corollary}

\begin{lemma}[Equivalent Contexts]
  \label{lemma:wf-ctxt-equiv}
  In the well-formedness judgment, only used variables matter:
  \begin{itemize}
  \item[$+$] if $[[{Γ1} ∩ fv iP]] = [[{Γ2} ∩ fv iP]]$ then
    $[[Γ1 ⊢ iP]] \iff [[Γ2 ⊢ iP]]$,
  \item[$-$] if $[[{Γ1} ∩ fv iN]] = [[{Γ2} ∩ fv iN]]$ then
    $[[Γ1 ⊢ iN]] \iff [[Γ2 ⊢ iN]]$.
  \end{itemize}
\end{lemma}
\begin{proof}
  By simple mutual induction on $[[iP]]$ and $[[iQ]]$. 
\end{proof}

\begin{corollary}
  \label{lemma:mut-sub-types-wf-equiv}
  Suppose that all the types below are well-formed in $[[Γ]]$ and
  $[[{Γ'} ⊆ {Γ}]]$. Then
  \begin{itemize}
  \item[$+$] $[[Γ ⊢ iP ≈ iQ]]$ implies $[[Γ' ⊢ iP]] \iff [[Γ' ⊢ iQ]]$
  \item[$-$] $[[Γ ⊢ iN ≈ iM]]$ implies $[[Γ' ⊢ iN]] \iff [[Γ' ⊢ iM]]$
  \end{itemize}
\end{corollary}
\begin{proof}
  From \cref{lemma:wf-ctxt-equiv,corollary:fv-mut-sub}.
\end{proof}

\subsection{Substitution}
\begin{lemma}[Substitution strengthening]
  \label{lemma:subst-restr-fv}
  Restricting the substitution to the free variables of the
  substitution subject does not affect the result.
  Suppose that $[[Γ2 ⊢ σ : Γ1]]$. Then
    \begin{itemize}
  \item[$+$] if $[[Γ1 ⊢ iP]]$ then $[[ [σ]iP ]] = [[ [σ|fv iP]iP ]]$,
  \item[$-$] if $[[Γ1 ⊢ iN]]$ then $[[ [σ]iN ]] = [[ [σ|fv iN]iN ]]$
  \end{itemize}
\end{lemma}
\begin{proof}
  \ilyam{todo}
\end{proof}

\begin{lemma}[]
  Suppose that $[[{Γ'} ⊆ {Γ}]]$,
  $[[σ1]]$ and $[[σ2]]$ are substitutions of signature $[[Γ ⊢ σi : Γ']]$.
  Then 
  \begin{enumerate}
    \item [$+$] for a type $[[Γ ⊢ iP]]$, if $[[Γ ⊢ [σ1]iP ≈ [σ2]iP]]$ then 
    $[[Γ ⊢ σ1 ≈ σ2 : fv iP ∩ {Γ'}]]$;
    \item [$-$] for a type $[[Γ ⊢ iN]]$, if $[[Γ ⊢ [σ1]iN ≈ [σ2]iN]]$ then
    $[[Γ ⊢ σ1 ≈ σ2 : fv iN ∩ {Γ'}]]$.
  \end{enumerate}
\end{lemma}
\begin{proof}
  Let us make an additional assumption that $[[σ1]]$, $[[σ2]]$, 
  and the mentioned types are normalized. If they are not,
  we normalize them first.
  
  Notice that the normalization preserves
  the set of free variables (\cref{lemma:fv-nf}),
  well-formedness (\cref{corollary:wf-nf}), 
  and equivalence (\cref{lemma:subt-equiv-algorithmization}), 
  and distributes over substitution (\cref{lemma:norm-subst-distr}). 
  This way, the assumed and desired properties are equivalent to their 
  normalized versions.

  We prove it by induction on the structure of $[[iP]]$ and mutually, $[[iN]]$.
  Let us consider the shape of this type.
  \begin{caseof}
    \item $[[iP]] = [[α⁺]] \in [[Γ']]$.
      Then $[[Γ ⊢ σ1 ≈ σ2 : fv iP ∩ {Γ'}]]$ means $[[Γ ⊢ σ1 ≈ σ2 : {α⁺}]]$, 
      i.e. $[[Γ ⊢ [σ1]α⁺ ≈ [σ2]α⁺ ]]$, which holds by assumption.
    \item $[[iP]] = [[α⁺]] \in [[{Γ} \ {Γ'}]]$.
      Then $[[fv iP ∩ {Γ'}]] = [[∅]]$, 
      so $[[Γ ⊢ σ1 ≈ σ2 : fv iP ∩ {Γ'}]]$ holds vacuously.
    \item $[[iP]] = [[↓iN]]$.
      Then the induction hypothesis is applicable to type $[[iN]]$:
      \begin{enumerate}
        \item $[[iN]]$ is normalized,
        \item $[[Γ ⊢ iN]]$ by inversion of $[[Γ ⊢ ↓iN]]$,
        \item $[[Γ ⊢ [σ1]iN ≈ [σ2]iN]]$ holds by inversion of 
          $[[Γ ⊢ [σ1]↓iN ≈ [σ2]↓iN]]$, i.e. $[[Γ ⊢ ↓[σ1]iN ≈ ↓[σ2]iN]]$.
      \end{enumerate}
      This way, we obtain $[[Γ ⊢ σ1 ≈ σ2 : fv iN ∩ {Γ'}]]$, 
      which implies the required equivalence since 
      $[[fv iP ∩ {Γ'}]] = [[fv ↓iN ∩ {Γ'}]] = [[fv iN ∩ {Γ'}]]$.
    \item $[[iP]] = [[∃nas.iQ]]$
      Then the induction hypothesis is applicable to type $[[iQ]]$ 
      well-formed in context $[[Γ, nas]]$:
      \begin{enumerate}
        \item $[[{Γ'} ⊆ {Γ, nas}]]$ since $[[{Γ'} ⊆ {Γ}]]$,
        \item $[[Γ, nas ⊢ σi : Γ']]$ by weakening,
        \item $[[iQ]]$ is normalized,
        \item $[[Γ, nas ⊢ iQ]]$ by inversion of $[[Γ ⊢ ∃nas.iQ]]$,
        \item Notice that $[[ [σi]∃nas.iQ ]]$ is normalized, and thus, 
          $[[ [σ1]∃nas.iQ ≈ [σ2]∃nas.iQ]]$ implies 
          $[[ [σ1]∃nas.iQ = [σ2]∃nas.iQ ]]$
          (by \cref{lemma:subt-equiv-algorithmization}).).
          This equality means $[[ [σ1]iQ = [σ2]iQ ]]$, 
          which implies $[[Γ ⊢ [σ1]iQ ≈ [σ2]iQ]]$.
      \end{enumerate}
    \item $[[iN]] = [[iP → iM]]$
  \end{caseof}
\end{proof}

\begin{lemma}[Substitutions equivalent on the metavariables]
  \label{lemma:subst-equiv-metavar}
  Suppose that $[[Γ ⊢ Θ]]$, $[[uσ1]]$ and $[[uσ2]]$ are substitutions 
  of signature $[[Θ ⊢ uσi]]$.
  Then 
  \begin{enumerate}
    \item [$+$] for a type $[[Γ; Θ ⊢ uP]]$, if $[[Γ ⊢ [uσ1]uP ≈ [uσ2]uP]]$ then
      $[[Θ ⊢ uσ1 ≈ uσ2 : uv uP]]$;
    \item [$-$] for a type $[[Γ; Θ ⊢ uN]]$, if $[[Γ ⊢ [uσ1]uN ≈ [uσ2]uN]]$ then
      $[[Θ ⊢ uσ1 ≈ uσ2 : uv uN]]$.
  \end{enumerate}
\end{lemma}
\begin{proof}
  The proof is a trivial structural induction on 
  $[[Γ; Θ ⊢ uP]]$ and mutually, on $[[Γ; Θ ⊢ uN]]$.
\end{proof}


\begin{lemma}[Substitution composition well-formedness]
  If $[[Γ'1 ⊢ σ1 : Γ1]]$ and $[[Γ'2 ⊢ σ2 : Γ2]]$,
  then $[[Γ'1, Γ'2 ⊢ σ2 ○ σ1 : Γ1, Γ2]]$.
\end{lemma}

\begin{lemma}[Substitution monadic composition well-formedness]
  \label{lemma:subst-monad-composition-wf}
  If $[[Γ'1 ⊢ σ1 : Γ1]]$ and $[[Γ'2 ⊢ σ2 : Γ2]]$,
  then $[[Γ'2 ⊢ σ2 <=< σ1 : Γ1]]$.
\end{lemma}

\begin{lemma}[Substitution composition]
  \label{lemma:subst-composition}
    If $[[Γ'1 ⊢ σ1 : Γ1]]$, $[[Γ'2 ⊢ σ2 : Γ2]]$, 
    $[[{Γ1} ∩ {Γ'2} = ∅ ]]$ and $[[ {Γ1} ∩ {Γ2} = ∅ ]]$ then 
    $[[ σ2 ○ σ1 ]] = [[ (σ2 <=< σ1) ○ σ2 ]]$.
\end{lemma}

\begin{corollary}[Substitution composition commutativity]
  \label{corollary:subst-composition-commutativity}
  If $[[Γ'1 ⊢ σ1 : Γ1]]$, $[[Γ'2 ⊢ σ2 : Γ2]]$, and
  $[[ {Γ1} ∩ {Γ2} = ∅ ]]$, 
  $[[ {Γ1} ∩ {Γ'2} = ∅ ]]$, and
  $[[ {Γ'1} ∩ {Γ2} = ∅ ]]$ then 
  $[[ σ2 ○ σ1 ]] = [[ σ1 ○ σ2 ]]$.
\end{corollary}
\begin{proof}
  by \cref{lemma:subst-composition},
    $[[ σ2 ○ σ1 ]] = [[ (σ2 <=< σ1) ○ σ2 ]]$.
    Since the codomain of $[[σ1]]$ is $[[Γ'1]]$,
    and it is disjoint with the domain of $[[σ2]]$,
    $[[σ2 <=< σ1]] = [[σ1]]$.
\end{proof}

\begin{lemma}[Substitution domain weakening]
  \label{lemma:subst-domain-weakening}
  If $[[Γ2 ⊢ σ : Γ1]]$ and $[[ {Γ2'} ⊆ {Γ} ]]$ then $[[Γ2 ⊢ σ : Γ1, Γ2']]$
\end{lemma}
\begin{proof}
  If the variable $[[α±]]$ is in $[[Γ1]]$ then $[[Γ2 ⊢ [σ]α± ]]$ by assumption.
  If the variable $[[α±]]$ is in $[[{Γ2'} \ {Γ1}]]$ then $[[ [σ]α± = α± ]] \in [[Γ2']] ⊆ [[Γ2]]$, 
  and thus, $[[Γ2 ⊢ α± ]]$.
\end{proof}


\subsection{Declarative Subtyping}
\begin{lemma}[Free Variable Propagation] \label{lemma:fv-propagation}
  In the judgments of negative subtyping or positive supertyping,
  free variables propagate left-to-right. For a context $[[Γ]]$,
  \begin{itemize}
    \item $-$ if $[[Γ ⊢ iN ≤ iM]]$ then $[[fv(iN)]] \subseteq [[fv(iM)]]$
    \item $+$ if $[[Γ ⊢ iP ≥ iQ]]$ then $[[fv(iP)]] \subseteq [[fv(iQ)]]$
  \end{itemize}
\end{lemma}
\begin{proof}
  Mutual induction on $[[Γ ⊢ iN ≤ iM]]$ and $[[Γ ⊢ iP ≥ iQ]]$.
  \begin{caseof}
  \item $[[G ⊢ a⁻ ≤ a⁻]]$\\
    It is self-evident that $[[{a⁻} ⊆ {a⁻}]]$.
  \item $[[G ⊢ ↑iP ≤ ↑iQ]]$
    From the inversion (and unfolding $[[G ⊢ iP ≈ iQ]]$ ), we have
    $[[G ⊢ iP ≥ iQ]]$. Then by the induction hypothesis,
    $[[fv(iP)]] \subseteq [[fv(iQ)]]$. The desired 
    inclusion holds, since $[[fv(↑iP)]] = [[fv(iP)]]$ and
    $[[fv(↑iQ)]] = [[fv(iQ)]]$.
  \item $[[G ⊢ iP → iN ≤ iQ → iM]]$
    The induction hypothesis applied to the premises gives:
    $[[fv(iP)]] \subseteq [[fv(iQ)]]$ and
    $[[fv(iN)]] \subseteq [[fv(iM)]]$.
    Then $[[fv(iP → iN)]] = [[fv(iP) ∪ fv(iN)]] \subseteq
    [[fv(iQ) ∪ fv(iM)]] = [[fv(iQ → iM)]]$.

  \item $[[G ⊢ ∀pas.iN ≤ ∀pbs.iM]]$\\
    $
    \begin{aligned}[t]
      [[fv ∀pas.iN ]] &\subseteq [[fv ([iPs/pas] iN) ]] ~\setminus~ [[{pbs}]] 
                      &&   \text{here $[[{pbs}]]$ is excluded by the premise $[[fv iN ∩ {pbs} = ∅]]$}\\
                      &\subseteq [[fv iM]] ~\setminus~ [[{pbs}]]
                      &&   \text{by the induction hypothesis, } [[fv ([iPs/pas] iN) ]] \subseteq [[fv iM]] \\
                      &\subseteq [[fv ∀pbs.iM]]
    \end{aligned}
    $
  \item The positive cases are symmetric.
  \end{caseof}
\end{proof}

\begin{corollary}[Free Variables of mutual subtypes] \label{corollary:fv-mut-sub}
  \hfill
  \begin{itemize}
    \item [$-$] If $[[Γ ⊢ iN ≈ iM]]$ then $[[fv iN]] = [[fv iM]]$, 
    \item [$+$] If $[[Γ ⊢ iP ≈ iQ]]$ then $[[fv iP]] = [[fv iQ]]$
  \end{itemize}
\end{corollary}

\begin{lemma}[Decomposition of quantifier rules]
  \label{lemma:quant-rule-decomposition}
  Assuming that $[[pas]]$, $[[pbs]]$, $[[nas]]$, and $[[nas]]$ are disjoint from $[[Γ]]$,
  \begin{itemize}
    \item [$-_{R}$] $[[Γ ⊢ iN ≤ ∀pbs.iM]]$ holds if and only if $[[Γ, pbs ⊢ iN ≤ iM]]$;
    \item [$+_{R}$] $[[Γ ⊢ iP ≥ ∃nbs.iQ]]$ holds if and only if $[[Γ, nbs ⊢ iP ≥ iQ]]$;
    \item [$-_{L}$] suppose $[[iM]] \neq [[∀]]\dots$
      then $[[Γ ⊢ ∀pas.iN ≤ iM]]$ holds if and only if $[[Γ ⊢ [iPs/pas]iN ≤ iM]]$
      for some $[[Γ ⊢ iPs]]$;
    \item [$+_{L}$] suppose $[[iQ]] \neq [[∃]]\dots$
      then $[[Γ ⊢ ∃nas.iP ≥ iQ]]$ holds if and only if $[[Γ ⊢ [iNs/nas]iP ≥ iQ]]$
      for some $[[Γ ⊢ iNs]]$.
  \end{itemize}
\end{lemma}
\begin{proof}
  \hfill
  \begin{itemize}
    \item [$-_{R}$] Let us prove both directions. 
      \begin{itemize}
        \item [$\Rightarrow$] Let us assume $[[Γ ⊢ iN ≤ ∀pbs.iM]]$.
          $[[Γ ⊢ iN ≤ ∀pbs.iM]]$.
          Let us decompose $[[iM]]$ as $[[∀pbs'.iM']]$ where $[[iM']]$ does not start with $[[∀]]$, 
          and decompose $[[iN]]$ as $[[∀pas.iN']]$ where $[[iN']]$ does not start with $[[∀]]$.
          If $[[pbs]]$ is empty, then $[[Γ, pbs ⊢ iN ≤ iM]]$ holds by assumption.
          Otherwise, $[[Γ ⊢ ∀pas.iN' ≤ ∀pbs.∀pbs'.iM]]$ is inferred by
          \ruleref{\ottdruleDOneForallLabel}, and by inversion:
          $[[Γ,pbs,pbs' ⊢ [iPs/pas]iN' ≤ iM']]$ for some $[[Γ,pbs,pbs' ⊢ iPs]]$.
          Then again by \ruleref{\ottdruleDOneForallLabel} with the same $[[iPs]]$,
          $[[Γ,pbs ⊢ ∀pas.iN' ≤ ∀pbs'.iM']]$, that is $[[Γ,pbs ⊢ iN ≤ iM]]$.
        \item [$\Leftarrow$] let us assume $[[Γ, pbs ⊢ iN ≤ iM]]$, and let us decompose 
          $[[iN]]$ as $[[∀pas.iN']]$ where $[[iN']]$ does not start with $[[∀]]$, 
          and $[[iM]]$ as $[[∀pbs'.iM']]$ where $[[iM']]$ does not start with $[[∀]]$.
          if $[[pas]]$ and $[[pbs']]$ are empty then $[[Γ, pbs ⊢ iN ≤ iM]]$
          is turned into $[[Γ ⊢ iN ≤ ∀pbs.iM]]$ by \ruleref{\ottdruleDOneForallLabel}.
          Otherwise, $[[Γ, pbs ⊢ ∀pas.iN' ≤ ∀pbs'.iM']]$ is inferred by
          \ruleref{\ottdruleDOneForallLabel}, that is $[[Γ, pbs, pbs' ⊢ [iPs/pas]iN' ≤ iM']]$
          for some $[[Γ, pbs, pbs' ⊢ iPs]]$.
          Then by \ruleref{\ottdruleDOneForallLabel} again,
          $[[Γ ⊢ ∀pas.iN' ≤ ∀pbs,pbs'.iM']]$, in other words, $[[Γ ⊢ ∀pas.iN' ≤ ∀pbs.∀pbs'.iM']]$, 
          that is $[[Γ ⊢ iN ≤ ∀pbs.iM]]$.
          
      \end{itemize}
    \item [$-_{L}$] Suppose $[[iM]] \neq [[∀]]\dots$. Let us prove both directions.
      \begin{itemize}
        \item [$\Rightarrow$] Let us assume $[[Γ ⊢ ∀pas.iN ≤ iM]]$.
          then if $[[pas = ·]]$, $[[Γ ⊢ iN ≤ iM]]$ holds immediately.
          Otherwise, let us decompose  $iN$ as $[[∀pas'.iN']]$ where 
          $[[iN']]$ does not start with $[[∀]]$.
          Then $[[Γ ⊢ ∀pas.∀pas'.iN' ≤ iM']]$ is inferred by
          \ruleref{\ottdruleDOneForallLabel},
          and by inversion, 
          there exist $[[Γ ⊢ iPs,iPs']]$ 
          such that $[[Γ ⊢ [iPs/pas][iPs'/pas']iN' ≤ iM']]$ 
          (the decomposition of substitutions is possible since $[[{pas} ∩ {Γ} = ∅]]$).
          Then by \ruleref{\ottdruleDOneForallLabel} again,
          $[[Γ ⊢ ∀pas'.[iPs'/pas']iN' ≤ iM']]$ (notice that $[[ [iPs'/pas']iN' ]]$ cannot
          start with $[[∀]]$).
        \item [$\Leftarrow$] Let us assume 
          $[[Γ ⊢ [iPs/pas]iN ≤ iM]]$ for some $[[Γ ⊢ iPs]]$.
          let us decompose $iN$ as $[[∀pas'.iN']]$ where $[[iN']]$ does not start with $[[∀]]$.
          Then $[[Γ ⊢ [iPs/pas]∀pas'.iN' ≤ iM']]$ or, equivalently,
          $[[Γ ⊢ ∀pas'.[iPs/pas]iN' ≤ iM']]$ is inferred by \ruleref{\ottdruleDOneForallLabel}
          (notice that $[[ [iPs/pas]iN' ]]$ cannot start with $[[∀]]$).
          By inversion, there exist $[[Γ ⊢ iPs']]$ such that 
          $[[Γ ⊢ [iPs'/pas'][iPs/pas]iN' ≤ iM']]$. Since $[[pas']]$ is disjoint
          from the free variables of $[[iPs]]$ and from $[[pas]]$, the composition of 
          $[[iPs'/pas']]$ and $[[iPs/pas]]$ can be joined into a single substitution
          well-formed in $[[Γ]]$. Then by \ruleref{\ottdruleDOneForallLabel} again,
          $[[Γ ⊢ ∀pas.iN ≤ iM]]$.
      \end{itemize}
      \item [$+$] The positive cases are proved symmetrically.
  \end{itemize}
\end{proof}

\begin{corollary}[Redundant quantifier elimination]
  \label{corollary:red-quant-elim}
  \hfill
  \begin{itemize}
    \item [$-_{L}$] Suppose that $[[ {pas} ∩ fv(iN) = ∅]]$ then 
      $[[Γ ⊢ ∀pas.iN ≤ iM]]$ holds if and only if $[[Γ ⊢ iN ≤ iM]]$;
    \item [$-_{R}$] Suppose that $[[ {pas} ∩ fv(iM) = ∅]]$ then 
      $[[Γ ⊢ iN ≤ ∀pas.iM]]$ holds if and only if $[[Γ ⊢ iN ≤ iM]]$;
    \item [$+_{L}$] Suppose that $[[ {nas} ∩ fv(iP) = ∅]]$ then
      $[[Γ ⊢ ∃nas.iP ≥ iQ]]$ holds if and only if $[[Γ ⊢ iP ≥ iQ]]$.
    \item [$+_{R}$] Suppose that $[[ {nas} ∩ fv(iQ) = ∅]]$ then 
      $[[Γ ⊢ iP ≥ ∃nas.iQ]]$ holds if and only if $[[Γ ⊢ iP ≥ iQ]]$.
  \end{itemize}
\end{corollary}
\begin{proof}
  \begin{itemize}
    \item [$-_{R}$] Suppose that $[[ {pas} ∩ fv(iM) = ∅]]$ then 
      by \cref{lemma:quant-rule-decomposition},
      $[[Γ ⊢ iN ≤ ∀pas.iM]]$ 
      is equivalent to $[[Γ, pas ⊢ iN ≤ iM]]$,
      By \label{lemma:wf-ctxt-equiv},
      since $[[{pas} ∩ fv(iN) = ∅]]$ and $[[{pas} ∩ fv(iM) = ∅]]$,
      $[[Γ, pas ⊢ iN ≤ iM]]$ is equivalent to $[[Γ ⊢ iN ≤ iM]]$.

    \item [$-_{L}$] Suppose that $[[ {pas} ∩ fv(iN) = ∅]]$.
      Let us decompose $[[iM]]$ as $[[∀pbs.iM']]$ 
      where $[[iM']]$ does not start with $[[∀]]$.
      By \cref{lemma:quant-rule-decomposition},
      $[[Γ ⊢ ∀pas.iN ≤ ∀pbs.iM']]$ is equivalent to
      $[[Γ,pbs ⊢ ∀pas.iN ≤ iM']]$, 
      which is equivalent to 
      existence of $[[Γ,pbs ⊢ iPs]]$ such that 
      $[[Γ,pbs ⊢ [iPs/pas]iN ≤ iM']]$.
      Since $[[ [iPs/pas]iN  = iN]]$, the latter is equivalent to 
      $[[Γ,pbs ⊢ iN ≤ iM']]$,
      which is equivalent to $[[Γ ⊢ iN ≤ ∀pbs.iM']]$.
      $[[Γ,pbs ⊢ iPs]]$ can be chosen arbitrary, for example, $[[iPsi]] = [[∃α⁻.↓α⁻]]$.
    \item [$+$] The positive cases are proved symmetrically.
  \end{itemize}
\end{proof}

\begin{lemma}[Subtypes and supertypes of a variable]
  \label{lemma:var-subt}
  Assuming $[[Γ ⊢  α⁻]]$, $[[Γ ⊢ α⁺]]$, $[[Γ ⊢ iN]]$, and $[[Γ ⊢ iP]]$,
  \begin{itemize}
  \item[$+$] if $[[Γ ⊢ iP ≥ ∃nas.α⁺]]$ or $[[Γ ⊢ ∃nas.α⁺ ≥ iP ]]$ then $[[iP]] = [[∃nbs.α⁺]]$ (for some potentially empty $[[nbs]]$)
  \item[$-$] if $[[Γ ⊢ iN ≤ ∀pas.α⁻]]$ or $[[Γ ⊢ ∀pas.α⁻ ≤ iN ]]$ then $[[iN]] = [[∀pbs.α⁻]]$ (for some potentially empty $[[pbs]]$)
  \end{itemize}
\end{lemma}
\begin{proof}
  We prove by induction on the tree
  inferring $[[Γ ⊢ iP ≥ ∃nas.α⁺]]$ or $[[Γ ⊢ ∃nas.α⁺ ≥ iP ]]$ or
  or $[[Γ ⊢ iN ≤ ∀pas.α⁻]]$ or $[[Γ ⊢ ∀pas.α⁻ ≤ iN ]]$.

  Let us consider which one of these judgments is inferred.
  \begin{caseof}
  \item $[[Γ ⊢ iP ≥ ∃nas.α⁺]]$\\
    If the size of the inference tree is $1$ then the only rule that can infer
    it is \ruleref{\ottdruleDOnePVarLabel}, which
    implies that $[[nas]]$ is empty and $[[iP = α⁺]]$.

    If the size of the inference tree is $>1$ then the last rule inferring
    it must be \ruleref{\ottdruleDOneExistsLabel}. By inverting this rule,
    $[[iP = ∃nbs.iP']]$ where $[[iP']]$ does not start with $\exists$ and
    $[[Γ, nas ⊢ [iNs/nbs] iP' ≥ α⁺]]$ for some $[[G, nas ⊢ iNi]]$.

    By the induction hypothesis, $[[ [iNs/nbs] iP' = ∃ncs.α⁺]]$.
    What shape can $[[iP']]$ have?
    As mentioned, it does not start with $\exists$, and it cannot start with
    $\uparrow$ (otherwise, $[[ [iNs/nas] iP' ]]$ would also
    start with $\uparrow$ and would not be equal to $[[∃nbs.α⁺]]$).
    This way, $[[iP']]$ is a \emph{positive} variable. 
    As such, $[[ [iNs/nas] iP' = iP']]$,
    and then $[[iP' = ∃ncs.α⁺]]$ meaning that $[[ncs]]$ is empty and $[[iP' = α⁺]]$.
    This way, $[[iP]] = [[∃nbs.iP']] = [[∃nbs.α⁺]]$, as required.

  \item $[[Γ ⊢ ∃nas.α⁺ ≥ iP]]$\\
    If the size of the inference tree is $1$ then the only rule that can infer
    it is \ruleref{\ottdruleDOnePVarLabel}, which
    implies that $[[nas]]$ is empty and $[[iP = α⁺]]$.

    If the size of the inference tree is $>1$ then the last rule inferring
    it must be \ruleref{\ottdruleDOneExistsLabel}. By inverting this rule,
    $[[iP = ∃nbs.iQ]]$ where $[[G, nbs ⊢ [iNs/nas]α⁺ ≥ iQ]]$ and $[[iQ]]$ 
    does not start with $\exists$.
    Notice that since $[[α⁺]]$ is positive, $[[ [iNs/nas]α⁺ = α⁺]]$, 
    i.e. $[[G, nbs ⊢ α⁺ ≥ iQ]]$.

    By the induction hypothesis, $[[iQ = ∃nbs'.α⁺]]$,
    and since $[[iQ]]$ does not start with $\exists$, $[[nbs']]$ is empty
    This way, $[[iP]] = [[∃nbs.iQ]] = [[∃nbs.α⁺]]$, as required.

  \item The negative cases ($[[Γ ⊢ iN ≤ ∀pas.α⁻]]$ and $[[Γ ⊢ ∀pas.α⁻ ≤ iN ]]$)
    are proved analogously.
  \end{caseof}
\end{proof}

\begin{corollary}[Variables have no proper subtypes and supertypes]
  \label{corollary:vars-no-proper-subtypes}
  Assuming that all mentioned types are well-formed in $[[Γ]]$,
  \begin{align*}
    [[Γ ⊢ iP ≥ α⁺]] ~ &\iff ~ [[iP = ∃nbs.α⁺]]  ~ \iff ~ [[Γ ⊢ iP ≈ α⁺]] ~ \iff ~ [[iP ≈ α⁺]]\\
    [[Γ ⊢ α⁺≥ iP]]  ~ &\iff ~ [[iP = ∃nbs.α⁺]]  ~ \iff ~ [[Γ ⊢ iP ≈ α⁺]] ~ \iff ~ [[iP ≈ α⁺]]\\
    [[Γ ⊢ iN ≤ α⁻]] ~ &\iff ~ [[iN = ∀pbs.α⁻]]  ~ \iff ~ [[Γ ⊢ iN ≈ α⁻]] ~ \iff ~ [[iN ≈ α⁻]]\\
    [[Γ ⊢ α⁻ ≤ iN]] ~ &\iff ~ [[iN = ∀pbs.α⁻]]  ~ \iff ~ [[Γ ⊢ iN ≈ α⁻]] ~ \iff ~ [[iN ≈ α⁻]]\\
  \end{align*}
\end{corollary}
\begin{proof}
  Notice that $[[Γ ⊢ ∃nbs.α⁺ ≈ α⁺]]$ and $[[∃nbs.α⁺ ≈ α⁺]]$ and apply
  \cref{lemma:var-subt}.
\end{proof}

\begin{lemma}[Reflexivity of subtyping] \label{lemma:subtyping-reflexivity}
  Assuming all the types are well-formed in $[[Γ]]$,
  \begin{itemize}
    \item [$-$] $[[Γ ⊢ iN ≤ iN]]$
    \item [$+$] $[[Γ ⊢ iP ≥ iP]]$
  \end{itemize}
\end{lemma}
\begin{proof}
  Let us prove it by the size of $[[iN]]$ and mutually, $[[iP]]$.
  \begin{caseof}
    \item $[[iN]] = [[α⁻]]$\\
      Then $[[Γ ⊢ α⁻ ≤ α⁻]]$ is inferred immediately by \ruleref{\ottdruleDOneNVarLabel}.
    \item $[[iN]] = [[∀pas.iN']]$ where $[[pas]]$ is not empty\\
      First, we rename $[[pas]]$ to fresh $[[pbs]]$ in $[[∀pas.iN']]$ to avoid
      name clashes: $[[∀pas.iN']] = [[∀pbs.[pas/pbs]iN']]$.
      Then to infer $[[Γ ⊢ ∀pas.iN' ≤ ∀pbs.[pas/pbs]iN']]$ we can apply 
      \ruleref{\ottdruleDOneForallLabel}, instantiating $[[pas]]$ with $[[pbs]]$:
      \begin{itemize}
        \item $[[fv iN ∩ {pbs} = ∅ ]]$ by choice of $[[pbs]]$,
        \item $[[G, pbs ⊢ pbi]]$,
        \item $[[G, pbs ⊢ [pbs/pas] iN' ≤ [pbs/pas] iN']]$ by the induction hypothesis,
        since the size of $[[ [pbs/pas]iN' ]]$ is equal to the size of $[[iN']]$,
        which is smaller than the size of $[[iN]] = [[∀pas.iN']]$.
      \end{itemize}
    \item $[[iN]] = [[iP → iM]]$\\
      Then $[[Γ ⊢ iP → iM ≤ iP → iM]]$ is inferred by \ruleref{\ottdruleDOneArrowLabel},
      since $[[Γ ⊢ iP ≥ iP]]$ and $[[Γ ⊢ iM ≤ iM]]$ hold the induction hypothesis. 
    \item $[[iN]] = [[↑iP]]$\\
      Then $[[Γ ⊢ ↑iP ≤ ↑iP]]$ is inferred by \ruleref{\ottdruleDOneShiftULabel},
      since $[[Γ ⊢ iP ≥ iP]]$ holds by the induction hypothesis.
    \item The positive cases are symmetric to the negative ones.
  \end{caseof}
\end{proof}

\begin{lemma}[Substitution preserves subtyipng]
  \label{lemma:subst-pres-subt}
  Suppose that all mentioned types are well-formed in $[[Γ1]]$,
  and $[[σ]]$ is a substitution $[[Γ2 ⊢ σ : Γ1]]$.
  \begin{itemize}
    \item $-$ If $[[Γ1 ⊢ iN ≤ iM]]$ then $[[Γ2 ⊢ [σ]iN ≤ [σ]iM]]$
    \item $+$ If $[[Γ1 ⊢ iP ≥ iQ]]$ then $[[Γ2 ⊢ [σ]iP ≥ [σ]iQ]]$
  \end{itemize}
\end{lemma}
\begin{proof}
  We prove it by induction on the size of the derivation of $[[Γ1 ⊢ iN ≤ iM]]$
  and mutually, $[[Γ1 ⊢ iP ≥ iQ]]$. Let us consider the last rule 
  used in the derivation:
  \begin{caseof}
    \item \ruleref{\ottdruleDOneNVarLabel}. Then by inversion, 
      $[[iN = α⁻]]$ and $[[iM = α⁻]]$. By reflexivity of subtyping
      (\cref{lemma:subtyping-reflexivity}),
      we have $[[Γ2 ⊢ [σ]α⁻ ≤ [σ]α⁻]]$, i.e. $[[Γ2 ⊢ [σ]iN ≤ [σ]iM]]$,
      as required.
    \item  \ruleref{\ottdruleDOneForallLabel}. Then by inversion,
      $[[iN = ∀pas.iN']]$, $[[iM = ∀pbs.iM']]$, where $[[pas]]$ or $[[pbs]]$ is not empty.
      Moreover, $[[Γ1, pbs ⊢ [iPs/pas]iN' ≤ iM']]$ for some $[[Γ1, pbs ⊢ iPs]]$, and 
      $[[fv iN ∩ {pbs} = ∅ ]]$.

      Notice that since the derivation of $[[Γ1, pbs ⊢ [iPs/pas]iN' ≤ iM']]$ is
      a subderivation of the derivation of $[[Γ ⊢ iN ≤ iM]]$, its size is smaller, 
      and hence, the induction hypothesis applies
      ($[[Γ1, pbs ⊢ σ : Γ1, pbs]]$ by  \label{lemma:subst-domain-weakening})
      :
      $[[Γ2, pbs ⊢ [σ][iPs/pas]iN' ≤ [σ]iM']]$.

      Notice that by convention, $[[pas]]$ and $[[pbs]]$ are fresh, and thus,  
      $[[ [σ]∀pas.iN' ]] = [[ ∀pas.[σ]iN' ]]$ and $[[ [σ]∀pbs.iM' ]] = [[ ∀pbs.[σ]iM' ]]$, 
      which means that the required $[[Γ2, Γ ⊢ [σ]∀pas.iN' ≤ [σ]∀pbs.iM']]$ is rewritten as
      $[[Γ2 , Γ ⊢ ∀pas.[σ]iN' ≤ ∀pbs.[σ]iM']]$.

      To infer it, we apply \ruleref{\ottdruleDOneForallLabel}, 
      instantiating $[[pai]]$ with $[[ [σ]iPi ]]$:
      \begin{itemize}
        \item $[[fv [σ]iN ∩ {pbs} = ∅ ]]$;
        \item $[[Γ2, Γ,pbs⊢ [σ]iPi]]$, by \cref{lemma:wf-subst} since from the inversion,
          $[[Γ1, Γ, pbs ⊢ iPi]]$;
        \item $[[Γ, pbs ⊢ [ [σ]iPs/pas ][σ]iN' ≤ [σ]iM']]$ holds
          by \cref{lemma:subst-composition}:
          Since $[[pas]]$ is fresh, it is disjoint with the domain and the codomain of $[[σ]]$
          ($[[Γ1]]$ and $[[Γ2]]$), and thus, 
          $[[ [σ][iPs/pas]iN' ]] = [[ [ σ <=< iPs/pas ][σ]iN' ]] = [[ [ [σ]iPs/pas ][σ]iN' ]]$.
          Then $[[Γ2, Γ, pbs ⊢ [σ][iPs/pas]iN' ≤ [σ]iM']]$ holds by the induction hypothesis.
      \end{itemize}

    \item \ruleref{\ottdruleDOneArrowLabel}. Then by inversion,
      $[[iN = iP → iN1]]$, $[[iM = iQ → iM1]]$, $[[Γ ⊢ iP ≥ iQ]]$, and $[[Γ ⊢ iN1 ≤ iM1]]$.
      And by the induction hypothesis, $[[Γ' ⊢ [σ]iP ≥ [σ]iQ]]$ and $[[Γ' ⊢ [σ]iN1 ≤ [σ]iM1]]$.
      Then $[[Γ' ⊢ [σ]iN ≤ [σ]iM]]$, i.e. $[[Γ' ⊢ [σ]iP → [σ]iN1 ≤ [σ]iQ → [σ]iM1]]$,
      is inferred by \ruleref{\ottdruleDOneArrowLabel}.
    \item \ruleref{\ottdruleDOneShiftULabel}. Then by inversion,
      $[[iN = ↑iP]]$, $[[iM = ↑iQ]]$, and $[[Γ ⊢ iP ≈ iQ]]$,
      which by inversion means that $[[Γ ⊢ iP ≥ iQ]]$ and $[[Γ ⊢ iQ ≥ iP]]$.
      Then the induction hypothesis applies, and we have $[[Γ' ⊢ [σ]iP ≥ [σ]iQ]]$
      and $[[Γ' ⊢ [σ]iQ ≥ [σ]iP]]$. 
      Then by sequential application of \ruleref{\ottdruleDOneNDefLabel} 
      and \ruleref{\ottdruleDOneShiftULabel} to these judgments,
      we have $[[Γ' ⊢ ↑[σ]iP ≤ ↑[σ]iQ]]$, i.e.
      $[[Γ' ⊢ [σ]iN ≤ [σ]iM]]$, as required.
    \item The positive cases are proved symmetrically.
  \end{caseof}
\end{proof}

\begin{corollary}[Substitution preserves subtyping induced equivalence]
  \label{corollary:subst-pres-equiv}
  Suppose that $[[Γ ⊢ σ : Γ1]]$. Then
    \begin{itemize}
      \item[$+$] if $[[Γ1 ⊢ iP]]$,~ $[[Γ1 ⊢ iQ]]$,~ and $[[Γ1 ⊢ iP ≈ iQ]]$ ~ 
        then $[[Γ ⊢ [σ]iP ≈ [σ]iQ]]$
      \item[$-$] if $[[Γ1 ⊢ iN]]$,~ $[[Γ1 ⊢ iM]]$,~ and $[[Γ1 ⊢ iN ≈ iM]]$ ~ 
        then $[[Γ ⊢ [σ]iN ≈ [σ]iM]]$
    \end{itemize}
\end{corollary}


\begin{lemma}[Transitivity of subtyping] 
  \label{lemma:subtyping-transitivity}
  Assuming the types are well-formed in $[[Γ]]$,
  \begin{itemize}
    \item[$-$] if $[[Γ ⊢ iN1 ≤ iN2]]$ and $[[Γ ⊢ iN2 ≤ iN3]]$ then $[[Γ ⊢ iN1 ≤ iN3]]$,
    \item[$+$] if $[[Γ ⊢ iP1 ≥ iP2]]$ and $[[Γ ⊢ iP2 ≥ iP3]]$ then $[[Γ ⊢ iP1 ≥ iP3]]$.
  \end{itemize}
\end{lemma}
\begin{proof}
  To prove it, we formulate a stronger property, 
  which will imply the required one, taking $[[σ]] = [[Γ ⊢ id : Γ]]$.

    Assuming all the types are well-formed in $[[Γ]]$,
    \begin{itemize}
      \item[$-$] if $[[Γ ⊢ iN ≤ iM1]]$, $[[Γ ⊢ iM2 ≤ iK]]$, and for 
        $[[Γ' ⊢ σ : Γ]]$, $[[ [σ]iM1 = [σ]iM2 ]]$ then $[[Γ' ⊢ [σ]iN ≤ [σ]iK]]$
      \item[$+$] if $[[Γ ⊢ iP ≥ iQ1]]$, $[[Γ ⊢ iQ2 ≥ iR]]$, and for
        $[[Γ' ⊢ σ : Γ]]$, $[[ [σ]iQ1 = [σ]iQ2 ]]$ then $[[Γ' ⊢ [σ]iP ≥ [σ]iR]]$
    \end{itemize}

  We prove it by induction on $\size{[[Γ ⊢ iN ≤ iM1]]} + \size{[[Γ ⊢ iM2 ≤ iK]]}$ and mutually, 
  on $\size{[[Γ ⊢ iP ≥ iQ1]]} + \size{[[Γ ⊢ iQ2 ≥ iR]]}$.


  First, let us consider the 3 important cases.
  \begin{caseof}
    \item Let us consider the case when $[[iM1 = ∀pbs1.α⁻]]$. 
      Then by \cref{lemma:var-subt},
       $[[Γ ⊢ iN ≤ iM1]]$ means that $[[iN = ∀pas.α⁻]]$. 
      $[[ [σ]iM1 = [σ]iM2 ]]$ means that $[[ ∀pbs1.[σ]α⁻ = [σ]iM2 ]]$.
      Applying $[[σ]]$ to both sides of $[[Γ ⊢ iM2 ≤ iK]]$ (by \cref{lemma:subst-pres-subt}),
      we obtain $[[Γ' ⊢ [σ]iM2 ≤ [σ]iK]]$, that is $[[Γ' ⊢  ∀pbs1.[σ]α⁻ ≤ [σ]iK]]$.
      Since $[[ fv([σ]α⁻) ⊆ {Γ,α⁻} ]]$, it is disjoint from $[[pas]]$ and $[[pbs1]]$,
      This way, by \cref{corollary:red-quant-elim}, 
      $[[Γ' ⊢  ∀pbs1.[σ]α⁻ ≤ [σ]iK]]$ is equivalent to 
      $[[Γ' ⊢  [σ]α⁻ ≤ [σ]iK]]$, which is equivalent to $[[Γ' ⊢  ∀pas.[σ]α⁻ ≤ [σ]iK]]$,
      that is $[[Γ' ⊢  [σ]iN ≤ [σ]iK]]$.
    \item Let us consider the case when $[[iM2 = ∀pbs2.α⁻]]$.
      This case is symmetric to the previous one. Notice that 
      \cref{lemma:var-subt,corollary:red-quant-elim} are agnostic to the 
      side on which the the quantifiers occur, and thus, 
      the proof stays the same. 
    \item Let us decompose the types, by extracting the outer quantifiers:
      \begin{itemize}
        \item $[[iN = ∀pas.iN']]$, where $[[iN']] \neq [[∀]]\dots$,
        \item $[[iM1 = ∀pbs1.iM1']]$, where $[[iM1']] \neq [[∀]]\dots$,
        \item $[[iM2 = ∀pbs2.iM2']]$, where $[[iM2']] \neq [[∀]]\dots$,
        \item $[[iK = ∀pcs.iK']]$, where $[[iK']] \neq [[∀]]\dots$.
      \end{itemize}
      and assume that at least one of $[[pas]]$, $[[pbs1]]$, $[[pbs2]]$, and $[[pcs]]$ is not empty.
      Since $[[ [σ]iM1 = [σ]iM2 ]]$, we have $[[ ∀pbs1.[σ]iM1' = ∀pbs2.[σ]iM2' ]]$,
      and since $[[iMi']]$ are not variables 
      (which was covered by the previous cases) and do not start with $\forall$,
      $[[ [σ]iMi' ]]$ do not start with $\forall$ either,
      which means $[[pbs1]] = [[pbs2]]$ and $[[ [σ]iM1' = [σ]iM2' ]]$.
      Let us rename $[[pbs1]]$ and $[[pbs2]]$ to $[[pbs]]$.
      Then $[[iM1 = ∀pbs.iM1']]$ and $[[iM2 = ∀pbs.iM2']]$.

      By \cref{lemma:quant-rule-decomposition} applied twice
      to $[[Γ ⊢ ∀pas.iN' ≤ ∀pbs.iM1']]$ and to $[[Γ ⊢ ∀pbs.iM2' ≤ ∀pcs.iK']]$,
      we have the following:
      \begin{enumerate}
        \item $[[Γ, pbs ⊢ [iPs/pas]iN' ≤ iM1']]$ for some $[[Γ, pbs ⊢ iPs]]$;
        \item $[[Γ, pcs ⊢ [iQs/pbs]iM2' ≤ iK']]$ for some $[[Γ, pcs ⊢ iQs]]$.
      \end{enumerate}
      And since at least one of 
      $[[pas]]$, $[[pbs]]$, and $[[pcs]]$ is not empty,
      either $[[Γ ⊢ iN ≤ iM1]]$ or $[[Γ ⊢ iM2 ≤ iK]]$ is inferred 
      by \ruleref{\ottdruleDOneForallLabel}, meaning that either 
      $[[Γ, pbs ⊢ [iPs/pas]iN' ≤ iM1']]$ is a proper subderivation of $[[Γ ⊢ iN ≤ iM1]]$ or
      $[[Γ, pcs ⊢ [iQs/pbs]iM2' ≤ iK']]$ is a proper subderivation of $[[Γ ⊢ iM2 ≤ iK]]$.

      Notice that we can weaken and rearrange the contexts without changing the sizes of the 
      derivations: $[[Γ, pbs, pcs ⊢ [iPs/pas]iN' ≤ iM1']]$
      and $[[Γ, pbs, pcs ⊢ [iQs/pbs]iM2' ≤ iK']]$. This way, 
      the sum of the sizes of these derivations is smaller than the sum of the sizes of
      $[[Γ ⊢ iN ≤ iM1]]$ and $[[Γ ⊢ iM2 ≤ iK]]$.
      Let us apply the induction hypothesis to these derivations, 
      with the substitution $[[ Γ', pcs ⊢ σ ○ (iQs/pbs) : Γ, pbs, pcs  ]]$
      (\cref{lemma:subst-domain-weakening}).
      To apply the induction hypothesis, it is left to show that 
      $[[ σ ○ (iQs/pbs) ]]$ unifies $[[iM1']]$ and $[[ [iQs/pbs]iM2']]$:
      $$
      \begin{aligned}[t]
        [[ [σ ○ iQs/pbs]iM1' ]] &= [[ [σ][iQs/pbs]iM1' ]]\\
                                &= [[ [ [σ]iQs/pbs ][σ]iM2' ]]
                                && \text{by \cref{lemma:subst-composition}}\\
                                &= [[ [ [σ]iQs/pbs ][σ]iM2' ]]
                                && \text{Since $[[ [σ]iM1' = [σ]iM2' ]]$}\\
                                &= [[  [σ][iQs/pbs]iM2' ]]
                                && \text{by \cref{lemma:subst-composition}}\\
                                &= [[  [σ][iQs/pbs][iQs/pbs]iM2' ]]
                                && \text{Since $[[Γ, pcs ⊢ iQs]]$, and $[[{(Γ, pcs)} ∩ {pbs} = ∅]]$ }\\
                                &= [[  [σ ○ iQs/pbs][iQs/pbs]iM2' ]]
      \end{aligned}
      $$
      This way the induction hypothesis gives us
      $[[ Γ', pcs ⊢ [σ][iQs/pbs][iPs/pas]iN' ≤  [σ][iQs/pbs]iK' ]]$,
      and since $[[Γ, pcs ⊢ iK']]$, $[[ [iQs/pbs]iK' = iK' ]]$, that is 
      $[[ Γ', pcs ⊢ [σ][iQs/pbs][iPs/pas]iN' ≤  [σ]iK' ]]$.
      Let us rewrite the substitution that we apply to $[[iN']]$:
      $$
      \begin{aligned}[t]
        [[ [σ ○ iQs/pbs ○ iPs/pas]iN' ]] &= [[ [ (σ <=< iQs/pbs) ○ σ ○ iPs/pas]iN' ]]
                                       && \text{by \cref{lemma:subst-composition}}\\
                                       &= [[ [(σ <=< iQs/pbs) ○ (σ <=< iPs/pas) ○ σ] iN' ]]
                                       && \text{by \cref{lemma:subst-composition}}\\
                                       &= [[ [(((σ <=< iQs/pbs) ○ σ) <=< iPs/pas) ○ σ] iN' ]]
                                       && \text{Since $[[fv([σ]iN') ∩ {pbs} = ∅]]$}\\
                                       &= [[ [((σ ○ iQs/pbs) <=< iPs/pas) ○ σ] iN' ]]
                                       && \text{by \cref{lemma:subst-composition}}\\
                                       &= [[ [(σ ○ iQs/pbs) <=< iPs/pas][σ] iN' ]]
      \end{aligned}
      $$
      Notice that $[[(σ ○ iQs/pbs) <=< iPs/pas]]$
      is a substitution that turns $[[pai]]$ into $[[ [σ ○ iQs/pbs]iPi ]]$, 
      where $[[ Γ',pcs ⊢ [σ ○ iQs/pbs]iPi]]$.
      This way, 
      $[[ Γ', pcs ⊢ [(σ ○ iQs/pbs) <=< iPs/pas][σ]iN' ≤  [σ]iK' ]]$
      means $[[Γ ⊢ ∀pas.[σ]iN' ≤ ∀pcs.[σ]iK']]$
      by \cref{lemma:quant-rule-decomposition}, that is
      $[[Γ ⊢ [σ]iN ≤ [σ]iK]]$, as required.
  \end{caseof}

  Now, we can assume that neither $[[Γ ⊢ iN ≤ iM1]]$ nor $[[Γ ⊢ iM2 ≤ iK]]$ 
  is inferred by \ruleref{\ottdruleDOneForallLabel}, and that neither $[[iM1]]$ nor $[[iM2]]$
  is equivalent to a variable.  Because of that, $[[ [σ]iM1 = [σ]iM2 ]]$ means that 
  $[[iM1]]$ and $[[iM2]]$ have the same outer constructor. Let us consider the shape of $[[iM1]]$.

  \begin{caseof}
    \item $[[iM1 = α⁻]]$ this case has been considered;
    \item $[[iM1 = ∀pbs.iM1']]$ this case has been considered;
    \item $[[iM1 = ↑iQ1]]$. Then as noted above, 
      $[[ [σ]iM1 = [σ]iM2 ]]$ means that $[[iM2 = ↑iQ2]]$ and $[[ [σ]iQ1 = [σ]iQ2 ]]$.
      Moreover, $[[Γ ⊢ iN ≤ ↑iQ1]]$ can only be inferred by \ruleref{\ottdruleDOneShiftULabel},
      and thus, $[[iN = ↑iP]]$, and by inversion, $[[Γ ⊢ iP ≥ iQ1]]$ and $[[Γ ⊢ iQ1 ≥ iP]]$.
      Analogously, $[[Γ ⊢ ↑iQ2 ≤ iK]]$ means that $[[iK = ↑iR]]$, $[[Γ ⊢ iQ2 ≥ iR]]$, and $[[Γ ⊢ iR ≥ iQ2]]$.

      Notice that the derivations of $[[Γ ⊢ iP ≥ iQ1]]$ and $[[Γ ⊢ iQ1 ≥ iP]]$ are proper sub-derivations of 
      $[[Γ ⊢ iN ≤ iM1]]$, and the derivations of $[[Γ ⊢ iQ2 ≥ iR]]$ and $[[Γ ⊢ iR ≥ iQ2]]$ are proper sub-derivations of
      $[[Γ ⊢ iM2 ≤ iK]]$. This way, the induction hypothesis is applicable:
      \begin{itemize}
        \item applying the induction hypothesis to $[[Γ ⊢ iP ≥ iQ1]]$ and $[[Γ ⊢ iQ2 ≥ iR]]$ 
          with $[[Γ' ⊢ σ : Γ]]$ unifying $[[iQ1]]$ and $[[iQ2]]$, we obtain $[[Γ' ⊢ [σ]iP ≥ [σ]iR]]$;
        \item applying the induction hypothesis to $[[Γ ⊢ iR ≥ iQ2]]$ and $[[Γ ⊢ iQ1 ≥ iP]]$ 
          with $[[Γ' ⊢ σ : Γ]]$ unifying $[[iQ2]]$ and $[[iQ1]]$, we obtain $[[Γ' ⊢ [σ]iR ≥ [σ]iP]]$.
      \end{itemize}
      This way, by \ruleref{\ottdruleDOneShiftULabel}, $[[Γ' ⊢ [σ]iN ≤ [σ]iK]]$, as required. 

    \item $[[iM1 = iQ1 → iM1']]$. Then as noted above, 
      $[[ [σ]iM1 = [σ]iM2 ]]$ means that $[[iM2 = iQ2 → iM2']]$, $[[ [σ]iQ1 = [σ]iQ2 ]]$, and $[[ [σ]iM1' = [σ]iM2' ]]$.
      Moreover, $[[Γ ⊢ iN ≤ iQ1 → iM1']]$ can only be inferred by \ruleref{\ottdruleDOneArrowLabel},
      and thus, $[[iN = iP → iN']]$, and by inversion, $[[Γ ⊢ iP ≥ iQ1]]$ and $[[Γ ⊢ iN' ≤ iM1']]$.
      Analogously, $[[Γ ⊢ iQ2 → iM2' ≤ iK]]$ means that $[[iK = iR → iK']]$, $[[Γ ⊢ iQ2 ≥ iR]]$, and $[[Γ ⊢ iM2' ≤ iK']]$.

      Notice that the derivations of $[[Γ ⊢ iP ≥ iQ1]]$ and $[[Γ ⊢ iN' ≤ iM1']]$ are proper sub-derivations of
      $[[Γ ⊢ iP → iN' ≤ iQ1 → iM1']]$, and the derivations of $[[Γ ⊢ iQ2 ≥ iR]]$ and $[[Γ ⊢ iM2' ≤ iK']]$ are proper sub-derivations of
      $[[Γ ⊢ iQ2 → iM2' ≤ iR → iK']]$. This way, the induction hypothesis is applicable:
      \begin{itemize}
        \item applying the induction hypothesis to $[[Γ ⊢ iP ≥ iQ1]]$ and $[[Γ ⊢ iQ2 ≥ iR]]$ 
          with $[[Γ' ⊢ σ : Γ]]$ unifying $[[iQ1]]$ and $[[iQ2]]$, we obtain $[[Γ' ⊢ [σ]iP ≥ [σ]iR]]$;
        \item applying the induction hypothesis to $[[Γ ⊢ iN' ≤ iM1']]$ and $[[Γ ⊢ iM2' ≤ iK']]$ 
          with $[[Γ' ⊢ σ : Γ]]$ unifying $[[iM1']]$ and $[[iM2']]$, we obtain $[[Γ' ⊢ [σ]iN' ≤ [σ]iK']]$.
      \end{itemize}
      This way, by \ruleref{\ottdruleDOneArrowLabel}, $[[Γ' ⊢ [σ]iP → [σ]iN' ≤ [σ]iR → [σ]iK']]$,
      that is $[[Γ' ⊢ [σ]iN ≤ [σ]iK]]$, as required.
  \end{caseof}
  After that we consider all the 
  analogous positive cases, and prove them symmetrically.
\end{proof}




\begin{corollary}[Transitivity of equivalence] \label{corollary:equivalence-transitivity}
  Assuming the types are well-formed in $[[Γ]]$,
  \begin{itemize}
    \item[$-$] if $[[Γ ⊢ iN1 ≈ iN2]]$ and $[[Γ ⊢ iN2 ≈ iN3]]$ then $[[Γ ⊢ iN1 ≈ iN3]]$,
    \item[$+$] if $[[Γ ⊢ iP1 ≈ iP2]]$ and $[[Γ ⊢ iP2 ≈ iP3]]$ then $[[Γ ⊢ iP1 ≈ iP3]]$.
  \end{itemize}
\end{corollary}











\subsection{Equivalence}
\label{sec:decl-equiv-lemmas}
\begin{lemma}[Declarative equivalence is transitive]
  \hfill
  \label{lemma:decl-equiv-transitivity}
  \begin{itemize}
  \item[$+$] if $[[iP1 ≈ iP2]]$ and $[[iP2 ≈ iP3]]$ then $[[iP1 ≈ iP3]]$,
  \item[$-$] if $[[iN1 ≈ iN2]]$ and $[[iN2 ≈ iN3]]$ then $[[iN1 ≈ iN3]]$.
  \end{itemize}
\end{lemma}
\begin{proof}
  \ilyam{should be easy to do by induction since the types are getting smaller}
\end{proof}

\begin{lemma}[Algorithmization of declarative equivalence]
  \label{lemma:decl-equiv-algorithmization}
  Declarative equivalence is equality of normal forms. 
  \begin{itemize}
    \item[$+$] $[[iP ≈ iQ]] \iff [[nf(iP) = nf(iQ)]]$,
    \item[$-$] $[[iN ≈ iM]] \iff [[nf(iN) = nf(iM)]]$.
  \end{itemize}
\end{lemma}
\begin{proof} \hfill
  \begin{itemize}
    \item[$+$] Let us prove both directions separately.
    \begin{itemize}
      \item[$\Rightarrow$] 
        exactly by \cref{lemma:normalization-completeness},
      \item[$\Leftarrow$] 
        from \cref{lemma:normalization-soundness}, we know
        $[[iP ≈ nf(iP)]] = [[nf(iQ) ≈ iQ]]$, then by transitivity (\cref{lemma:decl-equiv-transitivity}),
        $[[iP ≈ iQ]]$.
    \end{itemize}
    \item[$-$] The proof is exactly the same.
  \end{itemize}
\end{proof}

\begin{lemma}[Type well-formedness is invariant under equivalence]
  \label{lemma:wf-equiv}
  Mutual subtyping implies declarative equivalence.
  \begin{itemize}
  \item[$+$] if $[[iP ≈ iQ]]$ then $[[Γ ⊢ iP]] \iff [[Γ ⊢ iQ]]$,
  \item[$-$] if $[[iN ≈ iM]]$ then $[[Γ ⊢ iN]] \iff [[Γ ⊢ iM]]$
  \end{itemize}
\end{lemma}
\begin{proof}
  \ilyam{todo}
\end{proof}

\begin{corollary}[Normalization preserves well-formedness]
  \label{corollary:wf-nf}
  \hfill
  \begin{itemize}
  \item[$+$] $[[Γ ⊢ iP]] \iff [[Γ ⊢ nf(iP)]]$,
  \item[$-$] $[[Γ ⊢ iN]] \iff [[Γ ⊢ nf(iN)]]$
  \end{itemize}
\end{corollary}
\begin{proof}
  Immediately from \cref{lemma:wf-equiv,lemma:normalization-soundness}.
\end{proof}

\begin{corollary}[Normalization preserves well-formedness of substitution]
  \label{corollary:wf-s-nf}
  \hfill \\
   $[[Γ2 ⊢ σ : Γ1]] \iff [[Γ2 ⊢ nf(σ) : Γ1]]$
\end{corollary}

\begin{lemma}[Soundness of equivalence]
  \label{lemma:equiv-soundness}
  Declarative equivalence implies mutual subtyping.
  \begin{itemize}
    \item[$+$] if $[[Γ ⊢ iP]]$, $[[Γ ⊢ iQ]]$, and $[[iP ≈ iQ]]$ then $[[Γ ⊢ iP ≈ iQ]]$,
    \item[$-$] if $[[Γ ⊢ iN]]$, $[[Γ ⊢ iM]]$, and $[[iN ≈ iM]]$ then $[[Γ ⊢ iN ≈ iM]]$.
  \end{itemize}
\end{lemma}
\begin{proof}
  We prove it by mutual induction on $[[iP ≈ iQ]]$ and $[[iN ≈ iM]]$.
  \begin{caseof}
    \item $[[a⁻ ≈ a⁻]]$\\
      Then $[[Γ ⊢ a⁻ ≤ a⁻]]$ by \ruleref{\ottdruleDOneNVarLabel},
      which immediately implies $[[Γ ⊢ a⁻ ≈ a⁻]]$ by \ruleref{\ottdruleDOneNDefLabel}.

    \item $[[↑iP ≈ ↑iQ]]$\\
      Then by inversion of \ruleref{\ottdruleDOneShiftULabel},
      $[[iP ≈ iQ]]$, and from the induction hypothesis, $[[Γ ⊢ iP ≈ iQ]]$,
      and (by symmetry) $[[Γ ⊢ iQ ≈ iP]]$.

      When \ruleref{\ottdruleDOneShiftULabel} is applied to $[[Γ ⊢ iP ≈ iQ]]$,
      it gives us $[[Γ ⊢ ↑iP ≤ ↑iQ]]$; when it is applied to $[[Γ ⊢ iQ ≈ iP]]$,
      we obtain $[[Γ ⊢ ↑iQ ≤ ↑iP]]$. Together, it implies $[[Γ ⊢ ↑iP ≈ ↑iQ]]$.

    \item $[[iP → iN ≈ iQ → iM]]$\\
      Then by inversion of \ruleref{\ottdruleDOneArrowLabel},
      $[[iP ≈ iQ]]$ and $[[iN ≈ iM]]$. By the induction hypothesis,
      $[[Γ ⊢ iP ≈ iQ]]$ and $[[Γ ⊢ iN ≈ iM]]$, which means by inversion:
      \begin{enumerate*}
        \item[(i)] $[[Γ ⊢ iP ≥ iQ]]$,
        \item[(ii)] $[[Γ ⊢ iQ ≥ iP]]$,
        \item[(iii)] $[[Γ ⊢ iN ≤ iM]]$,
        \item[(iv)]  $[[Γ ⊢ iM ≤ iN]]$.
      \end{enumerate*}
      Applying \ruleref{\ottdruleDOneArrowLabel} to (i) and (iii), we obtain
      $[[Γ ⊢ iP → iN ≤ iQ → iM]]$; applying it to (ii) and (iv), we have $[[Γ ⊢
      iQ → iM ≤ iP → iN]]$. Together, it implies $[[Γ ⊢ iP → iN ≈ iQ → iM]]$.
    \item $[[∀pas.iN ≈ ∀pbs.iM]]$\\
      Then by inversion, there exists bijection $[[mu : ({pbs} ∩ fv iM) ↔ ({pas}
      ∩ fv iN)]]$, such that $[[iN ≈ [mu] iM]]$. By the induction hypothesis,
      $[[Γ, pas ⊢ iN ≈ [mu] iM]]$. From \cref{corollary:subst-pres-equiv} and
      the fact that $[[mu]]$ is bijective, we also have
      $[[Γ, pbs ⊢ [mu-1]iN ≈ iM]]$.

      Let us construct a subsitution $[[pas ⊢ iPs/pbs : pbs]]$ by
      extending $[[mu]]$ with arbitrary positive types on $[[{pbs} \ fv iM]]$.

      Notice that $[[ [mu]iM ]] = [[ [iPs/pbs]iM ]]$, and therefore,
      $[[Γ, pas ⊢ iN ≈ [mu] iM]]$ implies $[[Γ, pas ⊢ [iPs/pbs]iM ≤ iN]]$. Then by
      \ruleref{\ottdruleDOneForallLabel}, $[[Γ ⊢ ∀pbs.iM ≤ ∀pas.iN]]$.

      Analogously, we construct the substitution from $[[mu-1]]$, and use it to
      instantiate $[[pas]]$ in the application of
      \ruleref{\ottdruleDOneForallLabel} to infer $[[Γ ⊢ ∀pas.iN ≤ ∀pbs.iM]]$.

      This way, $[[Γ ⊢ ∀pbs.iM ≤ ∀pas.iN]]$ and $[[Γ ⊢ ∀pas.iN ≤ ∀pbs.iM]]$
      gives us $[[Γ ⊢ ∀pbs.iM ≈ ∀pas.iN]]$.

    \item For the cases of the positive types, the proofs are symmetric.
  \end{caseof}
\end{proof}

\begin{corollary}[Normalization is sound w.r.t. subtyping-induced equivalence] \label{corollary:nf-sound-wrt-subt-equiv}
  \hfill
  \begin{itemize}
    \item [$+$] if $[[Γ ⊢ iP]]$ then $[[Γ ⊢ iP ≈ nf(iP)]]$,
    \item [$-$] if $[[Γ ⊢ iN]]$ then $[[Γ ⊢ iN ≈ nf(iN)]]$.
  \end{itemize}
\end{corollary}
\begin{proof}
  Immediately from \cref{lemma:normalization-soundness,corollary:wf-nf,lemma:equiv-soundness}.
\end{proof}

\begin{corollary}[Normalization preserves subtyping] 
  \label{corollary:nf-pres-subt}
  Assuming all the types are well-formed in context $[[Γ]]$,
  \begin{itemize}
    \item [$+$] $[[Γ ⊢ iP ≥ iQ]] \iff [[Γ ⊢ nf(iP) ≥ nf(iQ)]]$,
    \item [$-$] $[[Γ ⊢ iN ≤ iM]] \iff [[Γ ⊢ nf(iN) ≤ nf(iM)]]$.
  \end{itemize}
\end{corollary}
\begin{proof}
  \hfill
  \begin{itemize}
    \item [$+$]  
    \begin{itemize}
      \item [$\Rightarrow$] Let us assume $[[Γ ⊢ iP ≥ iQ]]$.
        By \cref{corollary:nf-sound-wrt-subt-equiv},
        $[[Γ ⊢ iP ≈ nf(iP)]]$ and $[[Γ ⊢ iQ ≈ nf(iQ)]]$, 
        in particular, by inversion, 
        $[[Γ ⊢ nf(iP) ≥ iP]]$ and $[[Γ ⊢ iQ ≥ nf(iQ)]]$.
        Then by the transitivity of subtyping 
        (\cref{corollary:subtyping-transitivity}), 
        $[[Γ ⊢ nf(iP) ≥ nf(iQ)]]$.
      \item [$\Leftarrow$] Let us assume $[[Γ ⊢ nf(iP) ≥ nf(iQ)]]$.
        Also by \cref{corollary:nf-sound-wrt-subt-equiv}
        and inversion, 
        $[[Γ ⊢ iP ≥ nf(iP)]]$ and $[[Γ ⊢ nf(iQ) ≥ iQ]]$.
        Then by the transitivity, $[[Γ ⊢ iP ≥ iQ]]$.
    \end{itemize}
    \item [$-$] The negative case is proved symmetrically.
  \end{itemize}
\end{proof}

\begin{lemma}[Subtyping induced by disjoint substitutions]
  \label{lemma:subt-ind-disj-subst}
  If two disjoint substitutions induce subtyping, they are degenerate (so is the
  subtyping).
  Suppose that $[[Γ ⊢ σ1 : Γ1]]$ and $[[Γ ⊢ σ2 : Γ1]]$,
  where $[[{Γi} ⊆ {Γ}]]$ and $[[{Γ1} ∩ {Γ2}= ∅]]$. Then
  \begin{itemize}
  \item[$-$] assuming $[[Γ ⊢ iN]]$,~
    $[[Γ ⊢ [σ1]iN ≤ [σ2]iN]]$ implies $[[Γ ⊢ σi ≈ id : Ord fv iN]]$
  \item[$+$] assuming $[[Γ ⊢ iP]]$,~
    $[[Γ ⊢ [σ1]iP ≥ [σ2]iP]]$ implies $[[Γ ⊢ σi ≈ id : Ord fv iP]]$
  \end{itemize}
\end{lemma}
\begin{proof}
  Proof by induciton on $[[Γ ⊢ iN]]$ (and mutually on $[[Γ ⊢ iP]]$).
  \begin{caseof}
    \item $[[iN]] = [[α⁻]]$\\
      Then $[[Γ ⊢ [σ1]iN ≤ [σ2]iN]]$ is rewritten as $[[Γ ⊢ [σ1]α⁻ ≤ [σ2]α⁻]]$.
      Let us consider the following cases:
      \begin{caseof}
      \item $[[α⁻ ∉ {Γ1}]]$ and $[[α⁻ ∉ {Γ2}]]$ \label{case:var-not-in-ctxts}\\
        Then $[[Γ ⊢ σi ≈ id : α⁻]]$ holds immediately,
        since $[[ [σi] α⁻]] = [[ [id] α⁻]] = [[α⁻]]$ and
        $[[Γ ⊢ α⁻ ≈ α⁻]]$.
      \item $[[α⁻ ∊ {Γ1}]]$ and $[[α⁻ ∊ {Γ2}]]$\\
        This case is not possible by assumption: $[[{Γ1} ∩ {Γ2}= ∅]]$.
      \item $[[α⁻ ∊ {Γ1}]]$ and $[[α⁻ ∉ {Γ2}]]$\\
        Then we have $[[Γ ⊢ [σ1]α⁻ ≤ α⁻]]$,
        which by \cref{corollary:vars-no-proper-subtypes} means $[[Γ ⊢ [σ1]α⁻ ≈ α⁻]]$,
        and hence, $[[Γ ⊢ σ1 ≈ id : α⁻]]$.

        $[[Γ ⊢ σ2 ≈ id : α⁻]]$ holds since $[[ [σ2]α⁻ ]] = [[α⁻]]$,
        similarly to \cref{case:var-not-in-ctxts}.

      \item $[[α⁻ ∉ {Γ1}]]$ and $[[α⁻ ∊ {Γ2}]]$\\
        Then we have $[[Γ ⊢ α⁻ ≤ [σ2]α⁻]]$,
        which by \cref{corollary:vars-no-proper-subtypes} means $[[Γ ⊢ α⁻ ≈ [σ2]α⁻]]$,
        and hence, $[[Γ ⊢ σ2 ≈ id : α⁻]]$.

        $[[Γ ⊢ σ1 ≈ id : α⁻]]$ holds since $[[ [σ1]α⁻ ]] = [[α⁻]]$,
        similarly to \cref{case:var-not-in-ctxts}.
      \end{caseof}
  \item $[[iN]] = [[∀pas.iM]]$\\
    Then by inversion, $[[Γ, pas ⊢ iM]]$.
    $[[Γ ⊢ [σ1]iN ≤ [σ2]iN]]$ is rewritten as $[[Γ ⊢ [σ1]∀pas.iM ≤ [σ2]∀pas.iM]]$.
    By the congruence of substitution and by the inversion of
    \ruleref{\ottdruleDOneForallLabel}, $[[Γ, pas ⊢ [iQs/pas][σ1]iM ≤ [σ2]iM]]$,
    where $[[Γ, pas ⊢ iQi]]$.
    Let us denote the (Kleisli) composition of $[[σ1]]$ and $[[iQs/pas]]$ as
    $[[σ1']]$, noting that $[[Γ, pas ⊢ σ1' : Γ1, pas]]$,
    and $[[{Γ1, pas} ∩ {Γ2} = ∅]]$.

    Let us apply the induction hypothesis to $[[iM]]$ and the
    substitutions $[[σ1']]$ and $[[σ2]]$ with
    $[[Γ, pas ⊢ [σ1']iM ≤ [σ2]iM]]$ to obtain:
    \begin{align}
      [[Γ, pas ⊢ σ1' ≈ id : Ord fv iM]] \label{fact:subs-proper-sub:forall-ih}\\
      [[Γ, pas ⊢ σ2 ≈ id : Ord fv iM]]  \label{fact:subs-proper-sub:forall-ih2}
    \end{align}

    Then $[[Γ ⊢ σ2 ≈ id : Ord fv ∀pas.iM]]$ holds by strengthening of
    \ref{fact:subs-proper-sub:forall-ih2}:
    for any $[[β±]] \in [[fv ∀pas.iM]] = [[fv iM \ {pas}]]$,
    $[[Γ, pas ⊢ [σ2]β± ≈ β±]]$ is strengthened to $[[Γ ⊢ [σ2]β± ≈ β±]]$, because
    $[[fv [σ2]β±]] = [[fv β±]] = \{[[β±]]\} \subseteq [[{Γ}]]$.

    To show that $[[Γ ⊢ σ1 ≈ id : Ord fv ∀pas.iM]]$, let us take an arbitrary
    $[[β±]] \in [[fv ∀pas.iM]] = [[fv iM \ {pas}]]$.

    $
    \begin{aligned}[t]
      [[β±]] &= [[ [id]β± ]]
             && \text{by definition of $[[id]]$}\\
             &\eqDOne [[ [σ1']β± ]]
             && \text{by \ref{fact:subs-proper-sub:forall-ih}}\\
             &= [[ [iQs/pas][σ1]β±]]
             && \text{by definition of $[[σ1']]$}\\
             &= [[ [σ1]β± ]]
             && \text{because $[[{pas} ∩ fv [σ1]β±]] \subseteq [[{pas} ∩ {Γ}]] = \emptyset$}
    \end{aligned}
    $\\
    This way, $[[Γ ⊢ [σ1]β± ≈ β±]]$ for any $[[β±]] \in [[fv ∀pas.iM]]$ and thus,
    $[[Γ ⊢ σ1 ≈ id : Ord fv ∀pas.iM]]$.

  \item $[[iN]] = [[iP → iM]]$\\
    Then by inversion, $[[Γ ⊢ iP]]$ and $[[Γ ⊢ iM]]$.
    $[[Γ ⊢ [σ1]iN ≤ [σ2]iN]]$ is rewritten as
    $[[Γ ⊢ [σ1](iP → iM) ≤ [σ2](iP → iM)]]$,
    then by congruence of substitution,
    $[[Γ ⊢ [σ1]iP → [σ1]iM ≤ [σ2]iP → [σ2]iM]]$,
    then by inversion
    $[[Γ ⊢ [σ1]iP ≥ [σ2]iP]]$
    and
    $[[Γ ⊢ [σ1]iM ≤ [σ2]iM]]$.

    Applying the induction hypothesis to $[[Γ ⊢ [σ1]iP ≥ [σ2]iP]]$
    and to $[[Γ ⊢ [σ1]iM ≤ [σ2]iM]]$, we obtain (respectively):
    \begin{align}
      &[[Γ ⊢ σi ≈ id : Ord fv iP]] \label{fact:subs-proper-sub:arrow-ih1}\\
      &[[Γ ⊢ σi ≈ id : Ord fv iM]] \label{fact:subs-proper-sub:arrow-ih2}
    \end{align}

    Noting that $[[fv (iP → iM)]] = [[fv iP ∪ fv iM]]$,
    we combine
    \cref{fact:subs-proper-sub:arrow-ih1,fact:subs-proper-sub:arrow-ih2}
    to conclude:
    $[[Γ ⊢ σi ≈ id : Ord fv (iP → iM)]]$.

  \item $[[iN]] = [[↑iP]]$\\
    Then by inversion, $[[Γ ⊢ iP]]$.
    $[[Γ ⊢ [σ1]iN ≤ [σ2]iN]]$ is rewritten as
    $[[Γ ⊢ [σ1]↑iP ≤ [σ2]↑iP]]$,
    then by congruence of substitution and by inversion,
    $[[Γ ⊢ [σ1]iP ≥ [σ2]iP]]$

    Applying the induction hypothesis to $[[Γ ⊢ [σ1]iP ≥ [σ2]iP]]$, we obtain
    $[[Γ ⊢ σi ≈ id : Ord fv iP]]$. Since $[[fv ↑iP]] = [[fv iP]]$, we can
    conclude: $[[Γ ⊢ σi ≈ id : Ord fv ↑iP]]$.
  \item The positive cases are proved symmetrically.
  \end{caseof}
\end{proof}

\begin{corollary}[Substitution cannot induce proper subtypes or supertypes] \label{corollary:subst-proper-subt}
  Assuming all mentioned types are well-formed in $[[Γ]]$ and $[[σ]]$ is a
  substitution $[[Γ ⊢ σ : Γ]]$,
  \begin{align*}
    [[Γ ⊢ [σ]iN ≤ iN]] ~&\Rightarrow~ [[Γ ⊢ [σ]iN ≈ iN]]
                          \text{ and } [[Γ ⊢ σ ≈ id : Ord fv iN]] \\
    [[Γ ⊢ iN ≤ [σ]iN]] ~&\Rightarrow~ [[Γ ⊢ iN ≈ [σ]iN]]
                          \text{ and } [[Γ ⊢ σ ≈ id : Ord fv iN]] \\
    [[Γ ⊢ [σ]iP ≥ iP]] ~&\Rightarrow~ [[Γ ⊢ [σ]iP ≈ iP]]
                          \text{ and } [[Γ ⊢ σ ≈ id : Ord fv iP]] \\
    [[Γ ⊢ iP ≥ [σ]iP]] ~&\Rightarrow~ [[Γ ⊢ iP ≈ [σ]iP]]
                          \text{ and } [[Γ ⊢ σ ≈ id : Ord fv iP]] \\
  \end{align*}
\end{corollary}


\begin{lemma} \label{lemma:mutual-subst-subtyping}
  Asssuming that the mentioned types ($[[iP]]$, $[[iQ]]$, $[[iN]]$, and $[[iM]]$)
  are well-formed in $[[Γ]]$ and that the substitutions ($[[σ1]]$ and $[[σ2]]$) have signature $[[Γ ⊢ σi : Γ]]$,
  \begin{itemize}
  \item[$+$] if $[[Γ ⊢ [σ1] iP ≥ iQ]]$ and $[[Γ ⊢ [σ2] iQ ≥ iP]]$\\
    then there exists a bijection $[[μ : fv iP ↔ fv iQ]]$ such that
    $[[Γ ⊢ σ1 ≈ Sub μ : Ord fv iP]]$ and $[[Γ ⊢ σ2 ≈ Sub μ-1 : Ord fv iQ]]$;
  \item[$-$] if $[[Γ ⊢ [σ1] iN ≤ iM]]$ and $[[Γ ⊢ [σ2] iN ≤ iM]]$\\
    then there exists a bijection $[[μ : fv iN ↔ fv iM]]$ such that
    $[[Γ ⊢ σ1 ≈ Sub μ : Ord fv iN]]$ and $[[Γ ⊢ σ2 ≈ Sub μ-1 : Ord fv iM]]$.
  \end{itemize}
\end{lemma}
\begin{proof}
  \hfill
  \begin{itemize}
  \item[$+$]
    Applying $[[σ2]]$ to both sides of
    $[[Γ ⊢ [σ1] iP ≥ iQ]]$ (by \cref{todo}),
    we have: $[[Γ ⊢ [σ2 ○ σ1] iP ≥ [σ2]iQ]]$.
    Composing it with $[[Γ ⊢ [σ2] iQ ≥ iP]]$ (by transitivity \cref{todo}),
    we have $[[Γ ⊢ [σ2 ○ σ1] iP ≥ iP]]$.
    Then by \cref{corollary:subst-proper-subt},
    $[[Γ ⊢ σ2 ○ σ1 ≈ id : Ord fv iP]]$.

    % Applying $[[σ1]]$ to both sides of
    % $[[Γ ⊢ [σ2]iQ ≥ iP]]$ (by \cref{todo}),
    % we have: $[[Γ ⊢ [σ1 ○ σ2] iQ ≥ [σ1]iP]]$.
    % Composing it with $[[Γ ⊢ [σ1] iP ≥ iQ]]$ (by transitivity \cref{todo}),
    % we have $[[Γ ⊢ [σ1 ○ σ2] iQ ≥ iQ]]$.
    % Then by \cref{corollary:subst-proper-subt},
    By a symmetric argument, we also have:
    $[[Γ ⊢ σ1 ○ σ2 ≈ id : Ord fv iQ]]$.

    Now, we prove that
    $[[Γ ⊢ σ2 ○ σ1 ≈ id : Ord fv iP]]$ and
    $[[Γ ⊢ σ1 ○ σ2 ≈ id : Ord fv iQ]]$
    implies that $[[σ1]]$ and $[[σ1]]$
    are (equivalent to) mutually inverse bijections.

    To do so, it suffices to prove that
    \begin{enumerate}
    \item[(i)] for any $[[α± ∊ fv iP]]$ there exists $[[β± ∊ fv iQ]]$
        such that $[[ Γ ⊢ [σ1] α± ≈ β± ]]$ and
        $[[ Γ ⊢ [σ2] β± ≈ α± ]]$; and
    \item[(ii)] for any $[[β± ∊ fv iQ]]$ there exists $[[α± ∊ fv iP]]$
        such that $[[ Γ ⊢ [σ2] β± ≈ α± ]]$ and
        $[[ Γ ⊢ [σ1] α± ≈ β± ]]$.
    \end{enumerate}
    Then the these correspondences between $[[fv iP]]$ and
    $[[fv iQ]]$ are mutually inverse functions,
    since for any $[[β±]]$ there can be at most one $[[α±]]$
    such that $[[ Γ ⊢ [σ2] β± ≈ α± ]]$ (and vice versa).

    \begin{enumerate}
    \item[(i)] Let us take $[[α± ∊ fv iP]]$.
      \begin{enumerate}
      \item if $[[α±]]$ is positive ($[[α± = α⁺]]$),
        from $[[ Γ ⊢ [σ2][σ1]α⁺ ≈ α⁺ ]]$,
        by \cref{corollary:vars-no-proper-subtypes},
        we have
        $[[ [σ2][σ1]α⁺ = ∃nbs.α⁺ ]]$.

        What shape can $[[ [σ1]α⁺ ]]$ have? It cannot be $[[∃nas.↓iN]]$ (for
        potentially empty $[[nas]]$), because the outer constructor $\downarrow$
        would remain after the substitution $[[σ2]]$, whereas $[[∃nbs.α⁺]]$ does
        not have $[[↓]]$. The only case left is $[[ [σ1]α⁺ = ∃nas.γ⁺ ]]$.

        Notice that $[[Γ ⊢ ∃nas.γ⁺ ≈ γ⁺]]$, meaning that $[[Γ ⊢ [σ1]α⁺ ≈ γ⁺]]$.
        Also notice that $[[ [σ2]∃nas.γ⁺ = ∃nbs.α⁺ ]]$ implies
        $[[Γ ⊢ [σ2]γ⁺ ≈ α⁺]]$.

      \item if $[[α±]]$ is negative ($[[α± = α⁻]]$) from $[[ Γ ⊢ [σ2][σ1]α⁻ ≈ α⁻
        ]]$, by \cref{corollary:vars-no-proper-subtypes}, we have
        $[[ [σ2][σ1]α⁻ = ∀pbs.α⁻ ]]$.

        What shape can $[[ [σ1]α⁻ ]]$ have? It cannot be $[[∀pas.↑iP]]$
        nor $[[∀pas.iP → iM]]$ (for potentially empty $[[pas]]$),
        because the outer constructor ($[[→]]$ or $[[↑]]$), remaining
        after the substitution $[[σ2]]$, is however absent in the resulting
        $[[∀pbs.α⁻]]$. Hence, the only case left is $[[ [σ1]α⁻ = ∀pas.γ⁻ ]]$
        Notice that $[[Γ ⊢ γ⁻ ≈ ∀pas.γ⁻]]$, meaning that $[[Γ ⊢ [σ1]α⁻ ≈ γ⁻]]$.
        Also notice that $[[ [σ2]∀pas.γ⁻ = ∀pbs.α⁻ ]]$ implies
        $[[Γ ⊢ [σ2]γ⁻ ≈ α⁻]]$.
      \end{enumerate}
    \item[(ii)] The proof is symmetric:
      We swap $[[iP]]$ and $[[iQ]]$,
      $[[σ1]]$ and $[[σ2]]$,
      and exploit $[[ Γ ⊢ [σ1][σ2]α± ≈ α± ]]$ instead of
      $[[ Γ ⊢ [σ2][σ1]α± ≈ α± ]]$.

    \end{enumerate}

  \item[$-$] The proof is symmetric to the positive case.
  \end{itemize}
\end{proof}

\begin{lemma}[Equivalence of polymorphic types]
  \label{lemma:poly-types-equivalence}
  \hfill
  \begin{itemize}
    \item[$-$] For $[[Γ ⊢ ∀pas.iN]]$ and $[[Γ ⊢ ∀pbs.iM]]$,\\ if $[[Γ ⊢ ∀pas.iN ≈ ∀pbs.iM ]]$
    then there exists a bijection $[[μ : {pbs} ∩ fv iM ↔ {pas} ∩ fv iN]]$
    such that $[[ Γ, pas ⊢ iN ≈ [Sub μ] iN ]]$,
    \item[$+$] For $[[Γ ⊢ ∃nas.iP]]$ and $[[Γ ⊢ ∃nbs.iQ]]$,\\  if $[[Γ ⊢ ∃nas.iP ≈ ∃nbs.iQ ]]$
    then there exists a bijection $[[μ : {nbs} ∩ fv iQ ↔ {nas} ∩ fv iP]]$
    such that $[[ Γ, nbs ⊢ iP ≈ [Sub μ] iQ ]]$.
  \end{itemize}
\end{lemma}
\begin{proof}
    \hfill
  \begin{itemize}
    \item[$-$]
    First, by $\alpha$-conversion, we ensure $[[{pas} ∩ fv iM = ∅]]$ and $[[{pbs} ∩ fv iN = ∅]]$.
    By inversion, $[[Γ ⊢ ∀pas.iN ≈ ∀pbs.iM ]]$ implies 
    \begin{enumerate} 
      \item $[[Γ,pbs ⊢ [σ1]iN ≤ iM]]$ for $[[ Γ,pbs ⊢ σ1 : pas ]]$ and 
      \item $[[Γ,pas ⊢ [σ2]iM ≤ iN]]$ for $[[ Γ,pas ⊢ σ2 : pbs ]]$.
    \end{enumerate}
    To apply \cref{lemma:mutual-subst-subtyping}, we weaken 
    and rearrange the contexts, and extend the substitutions to act as identity
    outside of their initial domain:
    \begin{enumerate} 
      \item $[[Γ,pas,pbs ⊢ [σ1]iN ≤ iM]]$ for $[[ Γ,pas,pbs ⊢ σ1 : Γ,pas,pbs ]]$ and 
      \item $[[Γ,pas,pbs ⊢ [σ2]iM ≤ iN]]$ for $[[ Γ,pas,pbs ⊢ σ2 : Γ,pas,pbs ]]$.
    \end{enumerate}
    Then from \cref{lemma:mutual-subst-subtyping}, 
    there exists a bijection $[[μ : fv iM ↔ fv iN]]$ such that 
    $[[Γ,pas,pbs ⊢ σ2 ≈ Sub μ : Ord fv iM]]$ and 
    $[[Γ,pas,pbs ⊢ σ1 ≈ Sub μ-1 : Ord fv iN]]$. 

    Let us show that if we restrict the domain of $[[μ]]$ to 
    $[[pbs]]$, its range will be contained in $[[pas]]$.
    Let us take $[[γ⁺ ∊ {pbs} ∩ fv iM]]$ and 
    assume $[[ [μ]γ⁺]] \notin [[pas]]$.
    Then since $[[ Γ,pbs ⊢ σ1 : pas ]]$, 
    $[[σ1]]$ acts as identity outside of $[[pas]]$, i.e.
    $[[ [σ1][Sub μ]γ⁺ = [Sub μ]γ⁺ ]]$.
    Since
    $[[Γ,pas,pbs ⊢ σ1 ≈ Sub μ-1 : Ord fv iN]]$, 
    application of $[[σ1]]$ is equivalent to application of $[[Sub μ-1]]$,
    then 
    $[[ Γ,pas,pbs ⊢ [Sub μ-1][Sub μ]γ⁺ ≈ [Sub μ]γ⁺ ]]$, i.e.
    $[[Γ,pas,pbs ⊢ γ⁺ ≈ [Sub μ]γ⁺]]$, 
    which means $[[γ⁺ ∊ fv [Sub μ]γ⁺]] \subseteq [[fv iN]]$.
    By assumption, $[[γ⁺ ∊ {pbs} ∩ fv iM]]$, i.e. $[[{pbs} ∩ fv iN]] \neq \emptyset$, hence contradiction.

    By \cref{todo}, 
    $[[Γ,pas,pbs ⊢ σ2 ≈ Sub μ|{pbs} : Ord fv iM]]$ implies
    $[[Γ,pas,pbs ⊢ [σ2]iM ≈ [Sub μ|{pbs}]iM]]$.
    By similar reasoning, $[[Γ,pas,pbs ⊢ [σ1]iN ≈ [Sub μ-1|{pas}]iN]]$.

    This way,
    \begin{align} 
      [[Γ,pas,pbs ⊢ [Sub μ-1|{pas}]iN ≤ iM]] \label{fact:mu-inv-n-sub-m}\\
      [[Γ,pas,pbs ⊢ [Sub μ|{pbs}]iM ≤ iN]] \label{fact:mu-m-subt-n}
    \end{align}

    By applying $[[μ|_{pbs}]]$ to both sides of \ref{fact:mu-inv-n-sub-m} (\cref{todo})
    and contracting $[[μ-1|_{pas} ○ μ|_{pbs}]] = [[μ|_{pbs}-1 ○ μ|_{pbs}]] = [[id]]$,
    we have: $[[Γ,pas,pbs ⊢ iN ≤ [Sub μ|{pbs}]iM]]$, which together with \ref{fact:mu-m-subt-n}
    means $[[Γ,pas,pbs ⊢ iN ≈ [Sub μ|{pbs}]iM]]$, and by strengthening, $[[Γ,pas⊢ iN ≈ [Sub μ|{pbs}]iM]]$.
    Symmetrically, $[[Γ,pbs ⊢ iM ≈ [Sub μ|_{pbs}-1]iN]]$.
    \item{$+$} The proof is symmetric to the proof of the negative case.
  \end{itemize}

\end{proof}


\begin{lemma}[Completeness of equivalence] \label{lemma:equiv-completeness}
  Mutual subtyping implies declarative equivalence.
  Assuming all the types below are well-formed in $[[Γ]]$: 
  \begin{itemize}
  \item[$+$] if $[[Γ ⊢ iP ≈ iQ]]$ then $[[iP ≈ iQ]]$,
  \item[$-$] if $[[Γ ⊢ iN ≈ iM]]$ then $[[iN ≈ iM]]$.
  \end{itemize}
\end{lemma}
\begin{proof}
  \begin{itemize}
    \item[$-$] 
    Induction on the sum of sizes of  $[[iN]]$ and $[[iM]]$. 
    By inversion, $[[Γ ⊢ iN ≈ iM]]$ means $[[Γ ⊢ iN ≤ iM]]$ and $[[Γ ⊢ iM ≤ iN ]]$.
    Let us consider the last rule that forms $[[Γ ⊢ iN ≤ iM]]$:
    \begin{caseof}
      \item \ruleref{\ottdruleDOneNVarLabel} i.e. $[[Γ ⊢ iN ≤ iM]]$ is of the form $[[Γ ⊢ α⁻ ≤ α⁻]]$\\
      Then $[[iN ≈ iM]]$ (i.e. $[[α⁻ ≈ α⁻]]$) holds immediately by \ruleref{\ottdruleEOneNVarLabel}.

      \item \ruleref{\ottdruleDOneShiftULabel} i.e. 
      $[[Γ ⊢ iN ≤ iM]]$ is of the form $[[Γ ⊢ ↑iP ≤ ↑iQ]]$\\
      Then by inversion, $[[Γ ⊢ iP ≈ iQ]]$, 
      and by induction hypothesis, $[[iP ≈ iQ]]$.
      Then $[[iN ≈ iM]]$ (i.e. $[[↑iP ≈ ↑iQ]]$) holds 
      by \ruleref{\ottdruleEOneShiftULabel}.

      \item \ruleref{\ottdruleDOneArrowLabel} i.e. $[[Γ ⊢ iN ≤ iM]]$ is of the form $[[Γ ⊢ iP → iN' ≤ iQ → iM']]$\\
      Then by inversion, $[[Γ ⊢ iP ≥ iQ]]$ and $[[Γ ⊢ iN' ≤ iM']]$.
      Notice that $[[Γ ⊢ iM ≤ iN]]$ is of the form $[[Γ ⊢ iQ → iM' ≤ iP → iN']]$, 
      which by inversion means $[[Γ ⊢ iQ ≥ iP]]$ and $[[Γ ⊢ iM' ≤ iN']]$.

      This way, $[[Γ ⊢ iQ ≈ iP]]$ and $[[Γ ⊢ iM' ≈ iN']]$. 
      Then by induction hypothesis, $[[iQ ≈ iP]]$ and $[[iM' ≈ iN']]$.
      Then $[[iN ≈ iM]]$ (i.e. $[[iP → iN' ≈ iQ → iM']]$) holds by \ruleref{\ottdruleEOneArrowLabel}.

      \item \ruleref{\ottdruleDOneForallLabel} i.e. $[[Γ ⊢ iN ≤ iM]]$ is of the form $[[Γ ⊢ ∀pas.iN' ≤ ∀pbs.iM']]$\\
      Then by \cref{lemma:poly-type-equivalence}, $[[Γ ⊢ ∀pas.iN' ≈ ∀pbs.iM']]$ means that 
      there exists a bijection $[[μ : {pbs} ∩ fv iM' ↔ {pas} ∩ fv iN']]$ such that  
      $[[Γ,pas ⊢ [Sub μ]iM' ≈ iN']]$. 
      
      Notice that the application of bijection $[[μ]]$ to $[[iM']]$ does
      not change its size (which is less than the size of $[[iM]]$), hence the induction hypothesis applies.
      This way, $[[ [Sub μ]iM' ≈ iN']]$ (and by symmetry, $[[iN' ≈ [Sub μ]iM']]$) holds by induction. 
      Then we apply \ruleref{\ottdruleEOneForallLabel} to get $[[∀pas.iN' ≈ ∀pbs.iM']]$, i.e. $[[iN ≈ iM]]$.
    \end{caseof}
      
\item[$+$] The proof is symmetric to the proof of the negative case.
  \end{itemize}
\end{proof}

\begin{corollary}[Normalization is complete w.r.t. subtyping-induced equivalence]
  \label{corollary:nf-complete-wrt-subt-equiv}
  Assuming all the types below are well-formed in $[[Γ]]$:
  \begin{itemize}
    \item [$+$] if $[[Γ ⊢ iP ≈ iQ]]$ then $[[nf(iP) = nf(iQ)]]$,
    \item [$-$] if $[[Γ ⊢ iN ≈ iM]]$ then $[[nf(iN) = nf(iM)]]$.
  \end{itemize}
\end{corollary}  
\begin{proof}
  Immediately from \cref{lemma:equiv-completeness,lemma:normalization-completeness}.
\end{proof}

\begin{lemma}[Algorithmization of subtyping-induced equivalence]
  \label{lemma:subt-equiv-algorithmization}
  Mutual subtyping is equality of normal forms.
 Assuming all the types below are well-formed in $[[Γ]]$:
  \begin{itemize}
    \item [$+$] $[[Γ ⊢ iP ≈ iQ]] \iff [[nf(iP) = nf(iQ)]]$,
    \item [$-$] $[[Γ ⊢ iN ≈ iM]] \iff [[nf(iN) = nf(iM)]]$.
  \end{itemize}
\end{lemma}
\begin{proof}
  Let us prove the positive case, the negative case is symmetric.
  We prove both directions of $\iff$ separately:
  \begin{itemize}
    \item [$\Rightarrow$] exactly \cref{corollary:nf-complete-wrt-subt-equiv};
    \item [$\Leftarrow$] by \cref{lemma:decl-equiv-algorithmization,lemma:equiv-soundness}.
  \end{itemize}
\end{proof}



\subsection{Variable Ordering}
\obsOrdDeterministic*
\begin{proof}
  By mutual structural induction on $[[iN]]$ and $[[iP]]$.
  Notice that the shape of the term $[[iN]]$ or $[[iP]]$
  uniquely determines the last used inference rule,
  and all the premises are deterministic on the input.
\end{proof}

\lemOrdSoundness*
\begin{proof}
  We prove it by mutual induction on 
  $[[ ord varset in iN = ordVars ]]$ and $[[ ord varset in iP = ordVars ]]$.
  The only non-trivial cases are 
  \ruleref{\ottdruleOArrowLabel} and 
  \ruleref{\ottdruleOForallLabel}.  
  \begin{caseof}
    \item \ruleref{\ottdruleOArrowLabel}  
      Then the inferred ordering judgement has shape
      $[[ord varset in iP → iN = ordVars1, (ordVars2 \ {ordVars1})]]$
      and by inversion, 
      $[[ord varset in iP = ordVars1]]$   
      and 
      $[[ord varset in iN = ordVars2]]$.

      By definition of free variables, 
      $[[varset ∩ fv iP → iN = varset ∩ fv iP ∪ varset ∩ fv iN]]$,
      and since by the induction hypothesis 
      $[[varset ∩ fv iP = {ordVars1}]]$ and
      $[[varset ∩ fv iN = {ordVars2}]]$,
      we have
      $[[varset ∩ fv iP → iN = {ordVars1} ∪ {ordVars2}]]$.

      On the other hand, 
      as a set $[[{ordVars1} ∪ {ordVars2}]]$
      is equal to $[[ordVars1, (ordVars2 \ {ordVars1})]]$. 
    \item  \ruleref{\ottdruleOForallLabel}.
      Then  the inferred ordering judgement has shape
      $[[ord varset in ∀pas.iN = ordVars]]$,
      and by inversion, 
      $[[varset ∩ {pas} = ∅]]$    
      $[[ord varset in iN = ordVars]]$.
      The latter implies that $[[varset ∩ fv iN = {ordVars}]]$.
      We need to show that $[[varset ∩ fv ∀pas.iN = {ordVars}]]$,
      or equivalently, that
      $[[varset ∩ (fv iN \ {pas}) = varset ∩ fv iN ]]$,
      which holds since $[[varset ∩ {pas} = ∅]]$.
  \end{caseof}
\end{proof}


\corOrdAdditivity*

\lemOrdWeakening*
\begin{proof}
  Mutual structural induction on $[[iN]]$ and $[[iP]]$.

  \begin{caseof}
    \item If $[[iN]]$ is a variable $[[na]]$,
      we notice that $[[na ∊ varset]]$ 
      is equivalent to $[[na ∊ varset ∩ {na}]]$.
    \item If $[[iN]]$ has shape $[[↑iP]]$, then
      the required property holds immediately by the 
      induction hypothesis, since 
      $[[fv(↑iP) = fv(iP)]]$.
    \item If the term has shape $[[iP → iN]]$ then
      \ruleref{\ottdruleOArrowLabel} was applied
      to infer $[[ ord (varset ∩ (fv iP ∪ fv iN)) in iP → iN ]]$
      and $[[ ord varset in iP → iN]]$. 
      By inversion, the result of 
      $[[ ord (varset ∩ (fv iP ∪ fv iN)) in iP → iN ]]$
      depends on 
      $A = [[ ord (varset ∩ (fv iP ∪ fv iN)) in iP]]$
      and 
      $B = [[ ord (varset ∩ (fv iP ∪ fv iN)) in iN]]$.
      The result of
       $[[ ord varset in iP → iN]]$ 
       depends on 
      $X = [[ord varset in iP]]$ and
      $Y = [[ord varset in iN]]$.

      Let us show that $A = B$ and $X = Y$, so the results are equal. 
      By the induction hypothesis and set properties,
      $[[ ord (varset ∩ (fv iP ∪ fv iN)) in iP ]] = 
       [[ ord (varset ∩ (fv iP ∪ fv iN)) ∩ fv(iP) in iP ]] = 
       [[ ord varset ∩ fv(iP) in iP ]] = 
       [[ ord varset in iP ]]$.
      Analogously, 
      $[[ ord (varset ∩ (fv iP ∪ fv iN)) in iN ]]$ $=$
      \\ $[[ ord varset in iN ]]$.
    \item If the term has shape $[[∀pas.iN]]$,
      we can assume that $[[pas]]$ is disjoint
      from $[[varset]]$,
      since we operate on alpha-equivalence classes.
      Then using the induction hypothesis,
      set properties and \ruleref{\ottdruleOForallLabel}: 
      $[[ord varset ∩ (fv(∀pas.iN)) in ∀pas.iN]] =
       [[ord varset ∩ (fv(iN) \ {pas}) in iN]] =
       [[ord varset ∩ (fv(iN) \ {pas}) ∩ fv(iN) in iN]] =
       [[ord varset ∩ fv(iN) in iN]] =
       [[ord varset in iN]]$.
  \end{caseof}
\end{proof}

\corOrdIdemp*
\begin{proof}
  By \cref{lemma:ord-soundness,corollary:ord-weakening}.
\end{proof}
  

Next, we make a set-theoretical observation
that will be useful further.
In general, any injective function (its image)
distributes over the set intersection.
However, for convenience, we allow the bijections
on variables to be applied
\emph{outside of their domains}
(as identities), which may violate
the injectivity. To deal with these cases, 
we define a special notion of
bijections collision-free on certain sets
in such a way that
a bijection that is collision-free on $P$ and $Q$,
distributes over intersection of $P$ and $Q$.

\begin{definition} [Collision-free Bijection]
  We say that a bijection $\mu : A \leftrightarrow B$ between sets of
  variables is \textbf{collision-free on sets} $P$ and $Q$ if and only if
  \begin{enumerate}
    \item $\mu(P \cap A) \cap Q = \emptyset$
    \item $\mu(Q \cap A) \cap P = \emptyset$
  \end{enumerate}
\end{definition}

\begin{observation}
  Suppose that $\mu : A \leftrightarrow B$ is a bijection between two sets of variables,
  and $\mu$ is collision-free on $P$ and $Q$.
  Then $\mu(P \cap Q) = \mu(P) \cap \mu(Q)$.
\end{observation}
  
\lemDistrMuOrd*
\begin{proof}
  Mutual induction on $[[iN]]$ and $[[iP]]$.
  \begin{caseof}
  \item $[[iN]]$ = $[[na]]$ \label{case:distr-mu-ord:var} \\
    let us consider four cases:
    \begin{caseof}
    \item $[[na]] \in A$ and $[[na]] \in [[varset]]$. Then
      \begin{align*} [[ [mu] (ord varset in iN) ]] &= [[ [mu] (ord varset in na)]] \\
                                                             &= [[ [mu] na ]]
                                                             && \text{by \ruleref{\ottdruleOPVarInLabel}}\\
                                                             &= [[nb]]
                                                             && \text{for some $[[nb]] \in B$}\\
                                                             & && \text{(notice $[[nb]] \in [[ [mu]varset ]]$)} \\
                                                             &= [[ ord [mu]varset in nb ]]
                                                             && \text{by \ruleref{\ottdruleOPVarInLabel},
                                                                as $[[nb]] \in [[ [mu]varset ]]$} \\
                                                             &= [[ord [mu] varset in [mu] na ]]
       \end{align*}
     \item $[[na]] \notin A$ and $[[na]] \notin [[varset]]$\\
       Notice that
       $[[ [mu] (ord varset in iN) ]] = [[ [mu] (ord varset in na)]] = [[·]]$ by
       \ruleref{\ottdruleOPVarNInLabel}.
       On the other hand, $[[ ord [mu] varset in [mu] na = ord [mu] varset
       in na ]] = [[·]]$ The latter equality is from
       \ruleref{\ottdruleOPVarNInLabel}, because
       $[[mu]]$ is collision-free on $[[varset]]$ and $[[fv iN]]$, so
       $[[fv iN]] \ni [[na]] \notin [[mu]](A \cap [[varset]]) \cup
       [[varset]] \supseteq [[ [mu] varset ]]$.
     \item $[[na]] \in A$ but $[[na]] \notin [[varset]]$\\ Then
       $[[ [mu] (ord varset in iN) ]] = [[ [mu] (ord varset in na)]] = [[·]]$
       by \ruleref{\ottdruleOPVarNInLabel}.
       To prove that\\ $[[ ord [mu] varset in [mu] na ]] = [[·]]$, we apply
       \ruleref{\ottdruleOPVarNInLabel}. Let us show that
       $[[ [mu] na ]] \notin [[ [mu] varset ]]$.
       Since $[[ [mu] na ]] = [[mu]]([[na]])$ and
       $[[ [mu] varset ]] \subseteq [[mu]](A \cap [[varset]]) \cup [[varset]]$,
       it suffices to prove 
       $[[mu]]([[na]]) \notin [[mu]](A \cap [[varset]]) \cup [[varset]]$.

       \begin{enumerate}
       \item[(i)] If there is an element $x \in A \cap [[varset]]$ such that
         $[[mu]] x = [[mu]] [[na]]$, then $x = [[na]]$ by bijectivity of
         $[[mu]]$, which contradicts with $[[na]] \notin [[varset]]$. This way, 
         $[[mu]]([[na]]) \notin [[mu]](A \cap [[varset]])$.
       \item[(ii)]
         Since $[[mu]]$ is collision-free on $[[varset]]$ and $[[fv iN]]$,
         $[[mu]] (A \cap [[fv iN]]) \ni [[mu]]([[na]]) \notin [[varset]]$.
       \end{enumerate}

     \item $[[na]] \notin A$ but $[[na]] \in [[varset]]$\\
       $[[ ord [mu] varset in [mu] na ]] = [[ ord [mu] varset in na ]] = [[na]]$.
       The latter is by \ruleref{\ottdruleOPVarNInLabel}, because
       $[[na]] = [[ [mu] na ]] \in [[ [mu] varset ]]$ since $[[na]] \in [[varset]]$.
       On the other hand, $[[ [mu] (ord varset in iN) ]] = [[ [mu] (ord varset in na)]] = [[ [mu] na ]] = [[na]]$.
    \end{caseof}
  
  \item $[[iN]] = [[↑iP]]$
    \begin{align*}
       [[ [mu] (ord varset in iN) ]] &= [[ [mu] (ord varset in ↑iP) ]] \\
                                     &= [[ [mu] (ord varset in iP) ]]
                                     && \text{by \ruleref{\ottdruleOShiftULabel}}\\
                                     &= [[ ord [mu]varset in [mu]iP ]]
                                     && \text{by the induction hypothesis}\\
                                     &= [[ ord [mu]varset in  ↑[mu]iP ]]
                                     && \text{by \ruleref{\ottdruleOShiftULabel}}\\
                                     &= [[ ord [mu]varset in  [mu]↑iP ]]
                                     && \text{by the definition of substitution}\\
                                     &= [[ ord [mu]varset in  [mu]iN ]]
    \end{align*}
          
   \item $[[iN]] = [[iP → iM]]$
     \begin{align*}
        & [[ [mu] (ord varset in iN) ]] \\ 
        &= [[ [mu] (ord varset in iP → iM) ]] \\
        &= [[ [mu] (ordVars1, (ordVars2 \ {ordVars1})) ]]
          && \text{where } [[ord varset in iP = ordVars1]] \text{ and } [[ord varset in iM = ordVars2]] \\
        &= [[ [mu] ordVars1, [mu](ordVars2 \ {ordVars1}) ]] \\
        &= [[ [mu] ordVars1, ([mu]ordVars2 \ [mu]{ordVars1}) ]]
          && \text{by induction on $[[ordVars2]]$;
                  the ind. step is similar to 
                  \cref{case:distr-mu-ord:var}.}\\
        & && \text{notice that $[[mu]]$ is collision free on $[[{ordVars1}]]$ and $[[{ordVars2}]]$} \\
        & && \text{since
          $[[{ordVars1}]] \subseteq [[varset]]$ and
          $[[{ordVars2}]] \subseteq [[fv iN]]$ }\\
          &= [[ [mu] ordVars1, ([mu]ordVars2 \ {[mu]ordVars1}) ]]
      \end{align*}
      \hfill\\
      On the other hand,
       $[[  ord [mu] varset in [mu]iN ]] = 
        [[ ord [mu] varset in [mu]iP → [mu]iM ]] = 
        [[ ordVarsb1, (ordVarsb2 \ {ordVarsb1}) ]] = 
        [[ [mu] ordVars1, ([mu]ordVars2 \ {[mu]ordVars1}) ]]$,
        where  $[[ord [mu] varset in [mu] iP = ordVarsb1]]$ 
        and $[[ord [mu] varset in [mu] iM = ordVarsb2]]$, 
        then by the induction hypothesis, 
        $[[ordVarsb1]] = [[ [mu] ordVars1 ]]$, 
        $[[ordVarsb2]] = [[ [mu] ordVars2 ]]$.
   
   \item $[[iN]] = [[∀ pas.iM]]$
     \begin{align*}
          [[ [mu] (ord varset in iN) ]] &= [[ [mu] ord varset in ∀pas.iM]] \\
                                        &= [[ [mu] ord varset in iM]] \\
                                        &= [[ ord [mu] varset in [mu] iM]]
                                        && \text {by the induction hypothesis}\\
     \end{align*}
     \begin{align*}
       [[ (ord [mu] varset in [mu] iN) ]] &= [[ ord [mu] varset in [mu] ∀pas.iM ]] \\
                                          &= [[ ord [mu] varset in ∀pas.[mu]iM ]] \\
                                          &= [[ ord [mu] varset in [mu] iM ]] \\
     \end{align*}
  \end{caseof}
\end{proof}

\lemOrdSigma*
\begin{proof}
  Mutual induction on $[[iN]]$ and $[[iP]]$.
  \begin{caseof}
    \item $[[iN = na]]$ \\
      If $[[na ∉ Γ1]]$ then $[[ [σ]na = na ]]$ and $[[ ord varset in [σ]na ]] = [[ ord varset in na ]]$, 
      as requried.
      If $[[na ∊ Γ1]]$ then $[[na ∉ varset]]$, so $[[ ord varset in na ]] = [[·]]$.
      Moreover, $[[Γ2 ⊢ σ : Γ1]]$ means $[[ fv([σ]na) ⊆ Γ2 ]]$, and thus, 
      as a set, $[[ ord varset in [σ]na ]] = [[varset ∩ fv([σ]na)]] \subseteq [[varset ∩ Γ2]] = [[·]]$.
    \item $[[iN = ∀pas.iM]]$
      We can assume $[[{pas} ∩ Γ1 = ∅]]$
      and $[[{pas} ∩ varset = ∅]]$. Then 
      \begin{align*}[t]
         [[ ord varset in [σ]iN ]] &= [[ ord varset in [σ]∀pas.iM ]] \\
                                   &= [[ ord varset in ∀pas.[σ]iM ]]\\
                                   &= [[ ord varset in [σ]iM ]]
                                   && \text{by the induction hypothesis}\\
                                   &= [[ ord varset in iM ]]\\
                                   &= [[ ord varset in ∀pas.iM ]]\\
                                   &= [[ ord varset in iN ]]
       \end{align*}
    \item $[[iN = ↑iP]]$
       \begin{align*}[t]
        [[ ord varset in [σ]iN ]] &= [[ ord varset in [σ]↑iP ]] \\
                                   &= [[ ord varset in ↑[σ]iP ]]
                                   && \text{by the definition of substitution}\\
                                   &= [[ ord varset in [σ]iP ]]
                                   && \text{by the induction hypothesis}\\
                                   &= [[ ord varset in iP ]]
                                   && \text{by the definition of substitution}\\
                                   &= [[ ord varset in ↑iP ]]
                                   && \text{by the definition of ordering}\\
                                   &= [[ ord varset in iN ]]
       \end{align*}

    \item $[[iN = iP → iM]]$
       \begin{align*}
        [[ ord varset in [σ]iN ]] &= [[ ord varset in [σ](iP → iM) ]] \\
                                   &= [[ ord varset in ([σ]iP → [σ]iM) ]]
                                   && \text{def. of substitution}\\
                                   &= [[ ord varset in [σ]iP]],\\
                                   &\phantom{=} ~ [[(ord varset in [σ]iM \ {ord varset in [σ]iP}) ]]
                                   && \text{def. of ordering}\\
                                   &= [[ ord varset in iP]],\\
                                   &\phantom{=} ~ [[(ord varset in iM \ {ord varset in iP}) ]]
                                   && \text{the ind. hypothesis}\\
                                   &= [[ ord varset in iP → iM ]]
                                   && \text{def. of ordering}\\
                                   &= [[ ord varset in iN ]]
       \end{align*}
    \item The proofs of the positive cases are symmetric.
  \end{caseof}
\end{proof}

\lemOrdCompleteness*
\begin{proof}
  Mutual induction on $[[iN ≈ iM]]$ and $[[iP ≈ iQ]]$.
  Let us consider the rule inferring $[[iN ≈ iM]]$. 
  \begin{caseof}
    \item \ruleref{\ottdruleEOneNVarLabel}
    \item \ruleref{\ottdruleEOneShiftULabel}
    \item \ruleref{\ottdruleEOneArrowLabel}
      Then the equivalence has shape $[[iP → iN ≈ iQ → iM]]$,
      and by inversion, $[[iP ≈ iQ]]$ and $[[iN ≈ iM]]$.
      Then by the induction hypothesis,
      $[[ord varset in iP]] = [[ord varset in iQ]]$ 
      and $[[ord varset in iN]] = [[ord varset in iM]]$.
      Since the resulting ordering for $[[iP → iN]]$ and $[[iQ → iM]]$
      depend on the ordering of the corresponding components, 
      which are equal, the results are equal.
    \item \ruleref{\ottdruleEOneForallLabel}
      Then the equivalence has shape $[[∀pas.iN ≈ ∀pbs.iM]]$.
      and by inversion there exists 
      $[[mu : ({pbs} ∩ fv iM) ↔ ({pas} ∩ fv iN)]]$ such that
      \begin{itemize}
        \item $[[{pas} ∩ fv iM = ∅]]$ and 
        \item $[[iN ≈ [mu] iM]]$
      \end{itemize}

      Let us assume that $[[varset]]$ is disjoint from 
      $[[pas]]$ and $[[pbs]]$ 
      (we can always alpha-rename the bound variables).
      Then $[[ord varset in ∀pas.iN = ord varset in iN]]$, 
      $[[ord varset in ∀pas.iM = ord varset in iM]]$
      and by the induction hypothesis,
      $[[ord varset in iN]] = [[ord varset in [mu]iM]]$.
      This way, it suffices tho show  that 
      $[[ord varset in [mu]iM = ord varset in iM]]$.
      It holds by \cref{lemma:ord-sigma} since
      $[[varset]]$ is disjoint form 
      the domain and the codomain of 
      $[[mu : ({pbs} ∩ fv iM) ↔ ({pas} ∩ fv iN)]]$ 
      by assumption.

    \item The positive cases are proved symmetrically.
  \end{caseof}
\end{proof}



\subsection{Normaliztaion}
\obsNormDeterministic*
\begin{proof}
  By straightforward induction using \cref{obs:ord-deterministic}.
\end{proof}


\lemmaFvNf*
\begin{proof}
  By mutual induction on $[[iN]]$ and $[[iP]]$.
  The base cases 
  (\ruleref{\ottdruleNrmNVarLabel} and \ruleref{\ottdruleNrmPVarLabel})
  are trivial; the congruent cases
  (\ruleref{\ottdruleNrmShiftULabel},
  \ruleref{\ottdruleNrmShiftDLabel}, and
  \ruleref{\ottdruleNrmArrowLabel}) are proved by the induction hypothesis.

  Let us consider the case when the term is formed by $[[∀]]$,
  that is the normalization judgment has a shape 
  $[[nf(∀pas.iN) = ∀pas'.iN']]$,
  where by inversion $[[nf(iN) = iN']]$
  and $[[ord {pas} in iN' = pas']]$.
  By the induction hypothesis, $[[fv iN = fv iN']]$.
  Since $[[fv(∀pas.iN) = fv iN \ {pas}]]$,
  and $[[fv(∀pas'.iN') = fv iN' \ {pas'}]]$,
  it is left to show that $[[fv iN \ {pas} = fv iN \ {pas'}]]$.
  By \cref{lemma:ord-completeness}, 
  $[[{pas'}]] = [[{pas} ∩ fv iN']] = [[{pas} ∩ fv iN]]$.
  Then
  $[[fv iN \ {pas} = fv iN \ ({pas} ∪ fv iN)]]$ by
  set-theoretic properties, and
  thus, $[[fv iN \ {pas} = fv iN \ {pas'}]]$.

  The case when the term is positive and formed by $[[∃]]$ is symmetric.
\end{proof}

\lemmaNormSoundness*
\begin{proof}
  Mutual induction on $[[nf(iN) = iM]]$ and $[[nf(iP) = iQ]]$.
  Let us consider how this judgment is formed:
  \begin{caseof}
    \item{\nameref{\ottdruleNrmNVarLabel} and \nameref{\ottdruleNrmPVarLabel}}\\ By
      the corresponding equivalence rules.
    \item{\nameref{\ottdruleNrmShiftULabel}, \nameref{\ottdruleNrmShiftDLabel},
        and \nameref{\ottdruleNrmArrowLabel}}\\
      By the induction hypothesis and the corresponding congruent equivalence rules.
    \item{\nameref{\ottdruleNrmForallLabel}}, i.e. $[[nf(∀pas.uN) = ∀pas'.uN']]$ \label{case:norm-soundness:forall}\\
      From the induction hypothesis, we
      know that $[[iN ≈ iN']]$. In particular, by \cref{lemma:equiv-fv}, $[[fv
        iN]] \equiv [[fv iN']]$.
      Then by \cref{lemma:ord-soundness}, $[[{pas'}]]
      \equiv [[{pas} ∩ fv iN']] \equiv [[{pas} ∩ fv iN]]$, and thus,
      $[[{pas'} ∩ fv iN']] \equiv [[{pas} ∩ fv iN]]$.
      
      To prove $[[∀pas.iN ≈ ∀pas'.iN']]$, it suffices to provide a bijection 
      $\mu : [[{pas'} ∩ fv iN']] \leftrightarrow [[{pas} ∩ fv iN]]$ such that
      $[[iN ≈ [mu]iN']]$. Since these sets are equal, we take $\mu = id$.
    \item{\nameref{\ottdruleNrmExistsLabel}} Same as for \cref{case:norm-soundness:forall}.
  \end{caseof}
\end{proof}

\corollaryWfNf*
\begin{proof}
  Immediately from \cref{lemma:wf-equiv,lemma:normalization-soundness}.
\end{proof}

\corollaryWfSNf*
\begin{proof}
  Let us prove the forward direction.
  Suppose that $[[α± ∊ Γ1]]$.  Let us show that $[[Γ2 ⊢ [nf(σ)]α±]]$.
  By the definition of substitution normalization,
  $[[ [nf(σ)]α± = nf([σ]α±) ]]$. Then by \cref{corollary:wf-nf},
  to show $[[Γ2 ⊢ nf([σ]α±)]]$, it suffices to show $[[Γ2 ⊢ [σ]α±]]$,
  which holds by the assumption $[[Γ2 ⊢ σ : Γ1]]$.

  The backward direction is proved analogously.
\end{proof}

\lemmaNormSubstSig*
\begin{proof}
  Suppose that $[[α± ∊ Γ1]]$. 
  Then by \cref{corollary:wf-nf}, $[[Γ2 ⊢ nf([σ]α±)]] = [[ [nf(σ)]α± ]]$ 
  is equivalent to $[[Γ2 ⊢ [σ]α±]]$.

  Suppose that $[[α± ∉ Γ1]]$. 
  $[[Γ2 ⊢ σ : Γ1]]$ means that $[[ [σ]α± = α± ]]$, 
  and then $[[ [nf(σ)]α± ]] = [[nf([σ]α±)]] = [[nf(α±)]] = [[α±]]$.
\end{proof}

\corollaryNfSoundWrtSubtEquiv*
\begin{proof}
  Immediately from \cref{lemma:normalization-soundness,corollary:wf-nf,lemma:equiv-soundness}.  
\end{proof}

\corollaryNfPresSubt*
\begin{proof}
  \hfill
  \begin{itemize}
    \item [$+$]  
    \begin{itemize}
      \item [$\Rightarrow$] Let us assume $[[Γ ⊢ iP ≥ iQ]]$.
        By \cref{corollary:nf-sound-wrt-subt-equiv},
        $[[Γ ⊢ iP ≈ nf(iP)]]$ and $[[Γ ⊢ iQ ≈ nf(iQ)]]$, 
        in particular, by inversion, 
        $[[Γ ⊢ nf(iP) ≥ iP]]$ and $[[Γ ⊢ iQ ≥ nf(iQ)]]$.
        Then by transitivity of subtyping 
        (\cref{lemma:subtyping-transitivity}), 
        $[[Γ ⊢ nf(iP) ≥ nf(iQ)]]$.
      \item [$\Leftarrow$] Let us assume $[[Γ ⊢ nf(iP) ≥ nf(iQ)]]$.
        Also by \cref{corollary:nf-sound-wrt-subt-equiv}
        and inversion, 
        $[[Γ ⊢ iP ≥ nf(iP)]]$ and $[[Γ ⊢ nf(iQ) ≥ iQ]]$.
        Then by the transitivity, $[[Γ ⊢ iP ≥ iQ]]$.
    \end{itemize}
    \item [$-$] The negative case is proved symmetrically.
  \end{itemize}
\end{proof}

\corollaryNormPreservesOrdering*
\begin{proof}
  Immediately from \cref{lemma:ord-completeness,lemma:normalization-soundness}.
\end{proof}

\lemmaNormSubstDistr*
\begin{proof}
  Mutual induction on $[[iN]]$ and $[[iP]]$.
  \begin{caseof}
    \item $[[iN]]$ = $[[na]]$ \\
      \label{case:norm-subst-distr-var}
      $[[nf([σ]iN)]] = [[ nf([σ]na) ]] = [[ [nf(σ)]na ]] $.

      $[[ [nf(σ)] nf(iN) ]] = [[ [nf(σ)] nf(na) ]] = [[ [nf(σ)] na ]] $.
    \item $[[iP]]$ = $[[pa]]$ \\
      Similar to \cref{case:norm-subst-distr-var}.
   \item If the type is formed by $[[→]]$, $[[↑]]$, or $[[↓]]$, 
     the required equality follows from the congruence of the normalization and
     substitution and the induction hypothesis.
     For example, if $[[iN]] = [[iP → iM]]$ then
     \begin{align*}
        [[nf([σ] iN)]] &= [[ nf([σ] (iP → iM)) ]] \\
                        &= [[ nf([σ]iP → [σ]iM) ]]
                        && \text{By congruence of substitution} \\
                        &= [[ nf([σ]iP) → nf([σ]iM) ]]
                        && \text{By congruence of normalization} \\
                        &= [[ [nf(σ)]nf(iP) → [nf(σ)]nf(iM) ]]
                        && \text{By induction hypothesis} \\
                        &= [[ [nf(σ)](nf(iP) → nf(iM)) ]]
                        && \text{By congruence of substitution} \\
                        &= [[ [nf(σ)]nf(iP → iM) ]]
                        && \text{By congruence of normalization} \\
                        &= [[ [nf(σ)]nf(iN) ]]
      \end{align*}
    \item $[[iN]] = [[∀ pas.iM]]$ \label{case:norm-subst-commute}
      \begin{align*}
          [[ [nf(σ)] nf(iN) ]] &= [[ [nf(σ)] nf(∀pas.iM)]] \\
                            &= [[ [nf(σ)] ∀pas'.nf(iM) ]]
                            && \text {Where $[[pas']] = [[ ord {pas} in nf(iM)]]
                               = [[ord {pas} in iM]]$}\\
                           & && 
                               \text{(the latter is by
                               \cref{corollary:normalization-ord})}\\
        \end{align*}

      \begin{align*}
         [[ nf([σ]iN) ]] &= [[ nf([σ] ∀pas.iM)]] \\
                          &= [[ nf(∀pas.[σ]iM) ]]
                          && \text{Assuming $[[{pas} ∩ Γ1]] = \emptyset$
                             and $[[{pas} ∩ Γ2]] = \emptyset$}\\
                          &= [[ ∀pbs.nf([σ]iM) ]]
                          && \text {Where $[[pbs]] = [[ord {pas} in nf([σ]iM)]]
                             = [[ord {pas} in [σ]iM]]$}\\
                          & && 
                             \text{(the latter is by \cref{corollary:normalization-ord})}\\
                          &= [[ ∀pas'.nf([σ]iM) ]]
                          && \text{By \cref{lemma:ord-sigma},}\\
                          & && \text{$[[pbs]] = [[pas']]$
                             since $[[{pas}]]$ is disjoint with $[[Γ1]]$ and
                             $[[Γ2]]$}\\
                          &= [[ ∀pas'.[nf(σ)]nf(iM) ]]
                          && \text {By the induction hypothesis}\\
         \end{align*}
     To show the alpha-equivalence of 
     $[[ [nf(σ)] ∀pas'.nf(iM) ]]$ and $[[ ∀pas'.[nf(σ)]nf(iM) ]]$,
     we can assume that $[[{pas'} ∩ Γ1]] = \emptyset$, and $[[{pas'} ∩ Γ2]]
     = \emptyset$.

   \item $[[iP]] = [[∃ nas.iQ]]$ \\
     Same as for \cref{case:norm-subst-commute}.
  \end{caseof}
\end{proof}

\lemmaNormSubstCommute*
\begin{proof}
  Immediately from \cref{lemma:norm-subst-distr}, after noticing that $[[nf(mu)]] = [[mu]]$.
\end{proof}



\lemmaNormalizationCompleteness*
\begin{proof}
  Mutual induction on $[[iN ≈ iM]]$ and $[[iP ≈ iQ]]$.
  \begin{caseof}
  \item {\nameref{\ottdruleEOneForallLabel}} \label{case:ord-completeness:forall}
   From the normalization definition,
    \begin{itemize}
      \item $[[nf(∀pas.iN)]] = [[∀pas'.nf(iN)]]$ where $[[pas']]$ is $[[ord {pas} in nf(iN)]]$
      \item $[[nf(∀pbs.iM)]] = [[∀pbs'.nf(iM)]]$ where $[[pbs']]$ is $[[ord {pbs} in nf(iM)]]$
    \end{itemize}
    Let us take $[[mu : ({pbs} ∩ fv iM) ↔ ({pas} ∩ fv iN)]]$ from the
    inversion of the equivalence judgment. Notice that from
    \cref{lemma:fv-nf,lemma:ord-soundness}, the domain and the codomain of $\mu$ can be written
    as $[[mu : {pbs'} ↔ {pas'}]]$.
    
    To show the alpha-equivalence of $[[∀pas'.nf(iN)]]$ and $[[∀pbs'.nf(iM)]]$,
    it suffices to prove that
    \begin{enumerate*}
    \item[(i)] $[[ [mu] nf(iM) ]] = [[nf(iN)]]$ and \newline
    \item[(ii)] $[[ [mu]pbs' ]] = [[pas']]$
    \end{enumerate*}.
    
    \begin{enumerate}
    \item[(i)] $[[ [mu] nf(iM) ]] = [[nf([mu]iM)]] = [[nf(iN)]]$.
      The first equality holds by \cref{lemma:norm-subst-commute}, the second---by the induction hypothesis.

    \item[(ii)] 
    \begin{align*} 
      [[ [mu]pbs' ]] &= [[ [mu] ord {pbs} in nf(iM) ]]
                      && \text{by the definition of $[[pbs']]$ } \\
                      &= [[ [mu] ord ({pbs} ∩ fv iM) in nf(iM) ]]
                      && \text{from \cref{lemma:fv-nf,corollary:ord-weakening} } \\
                      &= [[ ord [mu] ({pbs} ∩ fv iM) in [mu] nf(iM) ]]
                      && \text{by \cref{lemma:distr-mu-ord}, because}\\
                      & && \text{$[[{pas} ∩ fv iN]] \cap [[fv nf(iM)]] \subseteq [[{pas} ∩ fv iM ]]
                        = \emptyset$}\\
                      &
                      && \text{$[[{pas} ∩ fv iN]] \cap [[({pbs} ∩ fv iM)]] \subseteq
                        [[{pas} ∩ fv iM]] = \emptyset$} \\
                      &= [[ ord [mu] ({pbs} ∩ fv iM) in nf(iN) ]]
                      && \text{since $[[ [mu] nf(iM) ]] = [[nf(iN)]]$ is proved } \\
                      &= [[ ord ({pas} ∩ fv iN) in nf(iN) ]]
                      && \text{because $\mu$ is a bijection between}\\
                      & && \text{$[[{pas} ∩ fv iN]]$ and $[[{pbs} ∩ fv iM]]$} \\
                      &= [[ ord {pas} in nf(iN) ]]
                      && \text{from \cref{lemma:fv-nf,corollary:ord-weakening} } \\
                      &= [[ pas' ]]
                      && \text{by the definition of $[[pas']]$} \\
      \end{align*}
    \end{enumerate}
  \item {\nameref{\ottdruleEOneExistsLabel}} Same as for \cref{case:ord-completeness:forall}.
  \item Other rules are congruent, and thus, proved by the corresponding congruent alpha-equivalence rule,
    which is applicable by the induction hypothesis. 
  \end{caseof}
\end{proof}

\lemmaDeclEquivAlg*
\begin{proof} \hfill
  \begin{itemize}
    \item[$+$] Let us prove both directions separately.
    \begin{itemize}
      \item[$\Rightarrow$] 
        exactly by \cref{lemma:normalization-completeness},
      \item[$\Leftarrow$] 
        from \cref{lemma:normalization-soundness}, we know
        $[[iP ≈ nf(iP)]] = [[nf(iQ) ≈ iQ]]$, then by transitivity (\cref{lemma:decl-equiv-transitivity}),
        $[[iP ≈ iQ]]$.
    \end{itemize}
    \item[$-$] For the negative case, the proof is the same.
  \end{itemize}
\end{proof}

\corollaryNfCompleteWrtSubtEquiv*
\begin{proof}
  Immediately from \cref{lemma:equiv-completeness,lemma:normalization-completeness}.
\end{proof}


\lemmaNormIdemp*
\begin{proof}
  By applying \cref{lemma:normalization-completeness} to \cref{lemma:normalization-soundness}.
\end{proof}

\lemmaNormalAfterSubst*
\begin{proof}
  Mutual induction on $[[Γ1 ⊢ iP]]$ and $[[Γ1 ⊢ iN]]$.
  \begin{caseof}
  \item $[[iN]] = [[na]]$\\
    Then $[[iN]]$ is always normal, and
    the normality of $[[σ|{na}]]$ by the definition means $[[ [σ]na ]]$ is normal.

  \item $[[iN]] = [[iP → iM]]$ \label{case:normal-after-subst-arrow}
    \begin{align*}
      [[ [σ](iP → iM) ]] \text{ is normal} &\iff [[ [σ]iP → [σ]iM ]] \text{ is normal}
                                           && \text{by substitution
                                              congruence} \\
                                           &\iff
                                             \begin{cases}
                                             [[ [σ]iP ]] &\text{is normal} \\
                                             [[ [σ]iM ]] &\text{is normal} \\
                                             \end{cases}\\
                                           &\iff
                                             \begin{cases}
                                               [[ iP ]]       &\text{is normal} \\
                                               [[ σ|fv(iP) ]] &\text{is normal} \\
                                               [[ iM ]]       &\text{is normal} \\
                                               [[ σ|fv(iM) ]] &\text{is normal} \\
                                             \end{cases}
                                           && \text{by the induction hypothesis}\\
                                           &\iff
                                             \begin{cases}
                                               [[ iP → iM ]]  &\text{is normal} \\
                                               [[ σ|fv(iP) ∪ fv(iM)]] &\text{is normal} \\
                                             \end{cases}\\
                                           &\iff
                                             \begin{cases}
                                               [[ iP → iM ]]  &\text{is normal} \\
                                               [[ σ|fv(iP→iM)]] &\text{is normal} \\
                                             \end{cases}
    \end{align*}
  \item $[[iN]] = [[↑iP]]$\\
    By congruence and the inductive hypothesis, similar to \cref{case:normal-after-subst-arrow}
  \item $[[iN]] = [[∀pas.iM]]$
    \begin{align*}
      &[[ [σ](∀pas.iM) ]] \text{ is normal} \\
       &\iff [[ (∀pas.[σ]iM) ]] \text{ is normal}
                                           && \text{assuming $[[pas]] \cap [[Γ1]] = \emptyset$ and
                                              $[[pas]] \cap [[Γ2]] = \emptyset$} \\
                                           &\iff
                                             \begin{cases}
                                             [[ [σ]iM ]] \text{ is normal} \\
                                             [[ord {pas} in [σ]iM = pas]] \\
                                             \end{cases}
                                           && \text{by the definition of normalization}\\
                                           &\iff
                                             \begin{cases}
                                               [[ [σ]iM ]] \text{ is normal} \\
                                               [[ord {pas} in iM = pas]] \\
                                             \end{cases}
                                           && \text{by \cref{lemma:ord-sigma}}\\
                                           &\iff
                                             \begin{cases}
                                               [[ σ|fv(iM) ]] \text{ is normal} \\
                                               [[ iM ]] \text{ is normal} \\
                                               [[ord {pas} in iM = pas]] \\
                                             \end{cases}
                                           && \text{by the induction hypothesis}\\
                                           &\iff
                                             \begin{cases}
                                               [[ σ|fv(∀pas.iM) ]] \text{ is normal} \\
                                               [[ ∀pas.iM ]] \text{ is normal} \\
                                             \end{cases}
                                           &&
                                              \begin{aligned}
                                              &\text{since $[[fv(∀pas.iM) = fv(iM)]]$;}\\ &\text{by the definition of normalization}
                                              \end{aligned}
    \end{align*}
  \item $[[iP]] = \dots$\\
    The positive cases are done in the same way as the negative ones.

  \end{caseof}
\end{proof}

\lemmaSubtInducedEquivAlg*
\begin{proof}
  Let us prove the positive case, the negative case is symmetric.
  We prove both directions of $\iff$ separately:
  \begin{itemize}
    \item [$\Rightarrow$] exactly \cref{corollary:nf-complete-wrt-subt-equiv};
    \item [$\Leftarrow$] by \cref{lemma:decl-equiv-algorithmization,lemma:equiv-soundness}.
  \end{itemize}
\end{proof}


\corSubstPresDeclEquiv*
\begin{proof}
  \begin{align*} 
    [[ iP ≈ iQ ]] &\Rightarrow        [[ nf(iP) = nf(iQ) ]]
                  && \text{by \cref{lemma:subt-equiv-algorithmization}}\\
                  &\Rightarrow [[ [nf(σ)]nf(iP) = [nf(σ)]nf(iQ)]]\\
                  &\Rightarrow [[ nf([σ]iP) = nf([σ]iQ)]]
                  && \text{by \cref{lemma:norm-subst-distr}}\\ 
                  &\Rightarrow        [[ [σ]iP ≈ [σ]iQ ]]
                  && \text{by \cref{lemma:subt-equiv-algorithmization}}\\
  \end{align*} 
\end{proof}


\section{Properties of the Algorithmic Type System}

\subsection{Algorithmic Type Well-formedness}
\begin{lemma}[AASoundness of algorithmic type well-formedness]
    \label{lemma:wf-algo-soundness}
    \hfill
  \begin{itemize}
    \item[$+$] if $[[Γ ; Ξ ⊢ uP]]$ then $[[fv(uP) ⊆ Γ]]$ and $[[uv(uP) ⊆ Ξ]]$;
    \item[$-$] if $[[Γ ; Ξ ⊢ uN]]$ then $[[fv(uN) ⊆ Γ]]$ and $[[uv(uN) ⊆ Ξ]]$.
  \end{itemize}
\end{lemma}
\begin{proof}
  The proof is analogous to \cref{lemma:wf-soundness}.
  The additional base case is when $[[Γ ; Ξ ⊢ uP]]$ is derived by \ruleref{\ottdruleWFATPUVarLabel},
  and the symmetric negative case.
  In this case, $[[uP]] = [[α̂⁺]]$, and $[[uv(uP)]] = [[{α̂⁺} ⊆ Ξ]]$ by inversion; $[[fv(uP)]] = [[∅ ⊆ Γ]]$ vacuously.
\end{proof}

\begin{lemma}[Completeness of algorithmic type well-formedness]
  \label{lemma:wf-algo-ctxt-equiv}
  In the well-formedness judgment, only used variables matter:
  \begin{itemize}
  \item[$+$] if $[[Γ1 ∩ fv iP]] = [[Γ2 ∩ fv iP]]$
    and $[[Ξ1 ∩ uv uP]] = [[Ξ2 ∩ uv uP]]$ then
    $[[Γ1 ; Ξ1 ⊢ uP]] \iff [[Γ2 ; Ξ2 ⊢ uP]]$, and
  \item[$-$] if $[[Γ1 ∩ fv iN]] = [[Γ2 ∩ fv iN]]$
    and $[[Ξ1 ∩ uv uN]] = [[Ξ2 ∩ uv uN]]$ then
    $[[Γ1 ; Ξ1 ⊢ uN]] \iff [[Γ2 ; Ξ2 ⊢ uN]]$.
  \end{itemize}
\end{lemma}
\begin{proof}
  By mutual structural induction on $[[uP]]$ and $[[uN]]$.
\end{proof}

\begin{lemma}[Variable algorithmization agrees with well-formedness]
  \hfill
  \begin{itemize}
    \item[$+$]  $[[Γ, nas ⊢ iP]]$ implies $[[Γ; {nuas} ⊢ [nuas/nas]iP]]$;
    \item[$-$]  $[[Γ, nas ⊢ iN]]$ implies $[[Γ; {nuas} ⊢ [nuas/nas]iN]]$.
  \end{itemize}
\end{lemma}
\begin{proof}
  The proof is a simple structural induction on $[[Γ, nas ⊢ iP]]$ and mutually, on $[[Γ, nas ⊢ iN]]$.
\end{proof}


\begin{corollary}[Well-formedness Algorithmic Context Weakening]
  \label{lemma:wf-weakening-algo}
  Suppose that $[[Γ1 ⊆ Γ2]]$,
  and $[[Ξ1 ⊆ Ξ2]]$. Then
  \begin{itemize}
    \item[$+$] if $[[Γ1 ; Ξ1 ⊢ uP]]$ implies $[[Γ2 ; Ξ2 ⊢ uP]]$,
    \item [$-$] if $[[Γ1 ; Ξ1 ⊢ uN]]$ implies $[[Γ2 ; Ξ2 ⊢ uN]]$.
  \end{itemize}
\end{corollary}
\begin{proof}
  By \cref{lemma:wf-soundness},
  $[[Γ1 ⊢ iP]]$ implies $[[fv(iP) ⊆ Γ1]]$,
  which means that $[[fv(iP) ⊆ Γ2]]$,
  and thus, $[[fv(iP)]] = [[fv(iP) ∩ Γ1]] = [[fv(iP) ∩ Γ2]]$.
  Then by \cref{lemma:wf-ctxt-equiv}, $[[Γ2 ⊢ iP]]$. 
  The negative case is symmetric.
\end{proof}

\subsection{Substitution}
\lemmaSubstRestrUV*
\begin{proof}
  The proof is analogous to the proof of \cref{lemma:subst-restr-fv}.
\end{proof}

\lemmaSubstEqAlgVar*
\begin{proof}
  The proof is a simple structural induction on 
  $[[Γ; Ξ ⊢ uP]]$ and mutually, on $[[Γ; Ξ ⊢ uN]]$.
  Let us consider the shape of $[[uN]]$ (the cases of $[[uP]]$ are symmetric).
  \begin{caseof}
    \item $[[Γ; Ξ ⊢ α̂⁻]]$. Then $[[ [uσ1]α̂⁻ = [uσ2]α̂⁻ ]]$
      implies $[[uσ1|(uv α̂⁻) = uσ2|(uv α̂⁻)]]$ immediately.
    \item $[[Γ; Ξ ⊢ α⁻]]$. Then $[[uv α̂⁻ = ∅]]$, and
      $[[uσ1|(uv α̂⁻)  = uσ2|(uv α̂⁻)]]$ holds vacuously.
    \item $[[Γ; Ξ ⊢ ∀pas.uN]]$. Then we are proving that
      $[[ [uσ1]∀pas.uN = [uσ2]∀pas.uN ]]$ implies $[[uσ1|(uv ∀pas.uN) = uσ2|(uv ∀pas.uN)]]$.
      By definition of substitution and \ruleref{\ottdruleEOneForallLabel}, 
      $[[ [uσ1]uN = [uσ2]uN ]]$ implies $[[uσ1|uv uN = uσ2|uv uN]]$.
      Since $[[∀pas.uN]]$ is normalized, so is $[[Γ, pas; Ξ ⊢ uN]]$, 
      hence, the induction hypothesis is applicable and implies $[[uσ1|uv uN = uσ2|uv uN]]$,
      as required.
    \item $[[Γ; Ξ ⊢ uP → uN]]$. Then we are proving that
      $[[ [uσ1](uP → uN) = [uσ2](uP → uN) ]]$ implies $[[uσ1|(uv uP → uN) = uσ2|(uv uP → uN)]]$.
      By definition of substitution and congruence of equality, 
      $[[ [uσ1](uP → uN) = [uσ2](uP → uN) ]]$
      means $[[ [uσ1]uP = [uσ2]uP ]]$ and $[[ [uσ1]uN = [uσ2]uN ]]$.
      Notice that $[[uP]]$ and $[[uN]]$ are normalized since $[[uP → uN]]$ is normalized, 
      and well-formed in the same contexts.
      This way, by the induction hypothesis, 
      $[[uσ1|(uv uP) = uσ2|(uv uP)]]$ and $[[uσ1|(uv uN) = uσ2|(uv uN)]]$,
      which since $[[uv (uP → uN) = uv uP ∪ uv uN]]$ implies
      $[[uσ1|(uv uP → uN) = uσ2|(uv uP → uN)]]$.
    \item $[[Γ; Ξ ⊢ ↑uP]]$. The proof is similar to the previous case:
      we apply congruence of substitution, equality, and normalization,
      then the induction hypothesis, and then the fact that $[[uv (↑uP) = uv uP]]$.
  \end{caseof}
\end{proof}

\corollarySubstEquivAlgVar*
\begin{proof}
  First, let us normalize the types and the substitutions, and show that
  the given equivalences and well-formedness properties are preserved. 
  $[[Γ; Ξ ⊢ uP]]$ implies $[[Γ; Ξ ⊢ nf(uP)]]$ by \cref{corollary:wf-nf-algo}.
  $[[Θ ⊢ [uσ1]uP ≈ [uσ2]uP]]$ implies $[[ nf([uσ1]uP) = nf([uσ2]uP)]]$ by 
  \cref{lemma:subt-equiv-algorithmization}.
  Then $[[ nf([uσ1]uP) = nf([uσ2]uP)]]$ implies 
  $[[ [nf(uσ1)]nf(uP) = [nf(uσ2)]nf(uP)]]$ by \cref{lemma:norm-subst-distr}.
  Notice that by \cref{corollary:norm-subst-sig-algo}
  $[[Θ ⊢ uσi : Ξ]]$ implies $[[Θ ⊢ nf(uσi) : Ξ]]$.

  This way, by \cref{lemma:subst-eq-algovar}, 
  $[[Θ ⊢ [uσ1]uP ≈ [uσ2]uP]]$ implies 
  $[[nf(uσ1)|(uv nf(uP)) = nf(uσ2)|(uv nf(uP))]]$.
  Then by \cref{lemma:uv-nf}, 
  $[[nf(uσ1)|(uv uP) = nf(uσ2)|(uv uP)]]$,
  and by \cref{corollary:subst-subt-equiv-algorithmization},
  $[[Θ ⊢ uσ1 ≈ uσ2 : uv uP]]$.

  Symmetrically, 
  $[[Θ ⊢ [uσ1]uN ≈ [uσ2]uN]]$ implies 
  $[[Θ ⊢ uσ1 ≈ uσ2 : uv uN]]$.
\end{proof}


\subsection{Normalization}
\begin{lemma}
  \label{lemma:uv-nf}
  Algorithmic variables are not changed by the normalization
  \begin{itemize}
  \item[$-$] $[[uv uN]] \equiv [[uv nf(uN)]]$
  \item[$+$] $[[uv uP]] \equiv [[uv nf(uP)]]$
  \end{itemize}
\end{lemma}
\begin{proof}
  By straightforward induction on $[[uN]]$ and mutually on $[[uP]]$, 
  similar to the proof of \cref{lemma:fv-nf}.
\end{proof}

\begin{lemma}[Soundness of normalization of algorithmic types]
  \label{lemma:normalization-soundness-alg}
  \hfill
  \begin{itemize}
    \item[$-$] $[[uN ≈ nf(uN)]]$
    \item[$+$] $[[uP ≈ nf(uP)]]$
  \end{itemize}
\end{lemma}
\begin{proof}
  The proof coincides with the proof of \cref{lemma:normalization-soundness}.
\end{proof}


\subsection{Equivalence}
\lemmaWfEquivAlgo*
\begin{proof}
  The proof coincides with the proof of \cref{lemma:wf-equiv},
  and adds two cases for equating two positive or two negative algorithmic variables,
  which must be equal by inversion, and thus, 
  $[[Γ; Ξ ⊢ α̂±]] \iff [[Γ; Ξ  ⊢ α̂±]]$ holds trivially.
\end{proof}

\corollaryWfNfAlgo*
\begin{proof}
  Immediately from \cref{lemma:wf-equiv-algo,lemma:normalization-soundness-alg}.
\end{proof}

\corollaryNormSubstSigAlgo*
\begin{proof}
  The proof is analogous to \cref{corollary:wf-s-nf}.
\end{proof}


\corollarySubstSubtEquivAlg*
\begin{proof}
  Follows immediately from \cref{lemma:subt-equiv-algorithmization}:
  \begin{itemize}
    \item [$\Rightarrow$]
      If $[[α̂± ∉ Ξ]]$, then $[[ [nf(uσ1)|Ξ]α̂± = [nf(uσ2)|Ξ]α̂± ]] = [[α̂±]]$ by 
      definition. 
      For any $[[α̂± ∊ Ξ]]$, 
      $[[ [nf(uσ1)|Ξ]α̂± = nf([uσ1]α̂±)]]$ and 
      $[[ [nf(uσ2)|Ξ]α̂± = nf([uσ2]α̂±) ]]$; 
      $[[Θ(α̂±) ⊢ [uσ1]α̂± ≈ [uσ2]α̂±]]$ implies
      $[[ nf([uσ1]α̂±) = nf([uσ2]α̂±) ]]$ by \cref{lemma:decl-equiv-algorithmization}.
    \item [$\Leftarrow$]
      If $[[α̂± ∊ Ξ]]$, then 
      $[[nf(uσ1)|Ξ = nf(uσ2)|Ξ]]$ implies
      $[[ nf([uσ1]α̂±) = nf([uσ2]α̂±) ]]$ by definition 
      of substitution restriction and normalization.
      In turn, $[[ nf([uσ1]α̂±) = nf([uσ2]α̂±) ]]$ means
       $[[Θ(α̂±) ⊢ [uσ1]α̂± ≈ [uσ2]α̂±]]$ by \cref{lemma:decl-equiv-algorithmization}.
  \end{itemize}
\end{proof}

\subsection{Unification Constraint Merge}
\begin{lemma} [Soundness of Unification Constraint Merge]
    \label{lemma:unif-merge-soundness}
    Suppose that $[[Θ ⊢ UC1]]$ and $[[Θ ⊢ UC2]]$ 
    are normalized unification constraints.
    If $[[Θ ⊢ UC1 & UC2 = UC]]$ is defined then
    $[[UC = UC1 ∪ UC2]]$.
\end{lemma}
\begin{proof}
    \hfill
    \begin{itemize}
        \item $[[UC1 & UC2]] \subseteq [[UC1]] \cup [[UC2]]$\\
        By definition, 
        $[[UC1 & UC2]]$ consists of three parts:
        entries of $[[UC1]]$ that do not have matching entries of $[[UC2]]$,
        entries of $[[UC2]]$ that do not have matching entries of $[[UC1]]$,
        and the merge of matching entries.

        If $[[ucE]]$ is from the first or the second part, 
        then $[[ucE]] \in [[UC1]] \cup [[UC2]]$ holds immediately.
        If $[[ucE]]$ is from the third part,
        then $[[ucE]]$ is the merge of two matching entries
        $[[ucE1]] \in [[UC1]]$ and $[[ucE2]] \in [[UC2]]$.
        Since $[[UC1]]$ and $[[UC2]]$ are normalized unification , 
        $[[ucE1]]$ and $[[ucE2]]$ have one of the following forms:
        \begin{itemize}
            \item $[[α̂⁺ :≈ iP1]]$ and $[[α̂⁺ :≈ iP2]]$, 
                where $[[iP1]]$ and $[[iP2]]$ are normalized,
                and then since $[[Θ(α̂⁺) ⊢ ucE1 & ucE2 = ucE]]$ exists, 
                \ruleref{\ottdruleUCMEPEqEqLabel} was applied to infer it.
                It means that $[[ucE]] = [[ucE1]] = [[ucE2]]$;
            \item $[[α̂⁻ :≈ iN1]]$ and $[[α̂⁻ :≈ iN2]]$, 
               then symmetrically, 
               $[[Θ(α̂⁻) ⊢ ucE1 & ucE2 = ucE]] = [[ucE1]] = [[ucE2]]$
        \end{itemize}
        In both cases, $[[ucE]] \in [[UC1]] \cup [[UC2]]$.

        \item $[[UC1]] \cup [[UC2]] \subseteq [[UC1 & UC2]]$\\
        Let us take 
        an arbitrary $[[ucE1]] \in [[UC1]]$.
        Then since $[[UC1]]$ is a unification constraint,
         $[[ucE1]]$ has one of the following forms:
        \begin{itemize}
            \item $[[α̂⁺ :≈ iP]]$ where $[[iP]]$ is normalized.
            If $[[α̂⁺]] \notin [[dom(UC2)]]$, then $[[ucE1]] \in [[UC1 & UC2]]$.
            Otherwise, there is a normalized matching
            $[[ucE2]] = [[(α̂⁺ :≈ iP')]] \in [[UC2]]$ and then
            since $[[UC1 & UC2]]$ exists, 
            \ruleref{\ottdruleUCMEPEqEqLabel} was applied to construct
            $[[ucE1 & ucE2]] \in [[UC1 & UC2]]$.
            By inversion of \ruleref{\ottdruleUCMEPEqEqLabel},
            $[[ucE1 & ucE2]] = [[ucE1]]$, and
            $[[nf(iP) = nf(iP')]]$, which since $[[iP]]$
            and $[[iP']]$ are normalized, implies that $[[iP = iP']]$, 
            that is $[[ucE1]] = [[ucE2]] \in [[UC1 & UC2]]$.
            \item $[[α̂⁻ :≈ iN]]$ where $[[iN]]$ is normalized.
            Then symmetrically, $[[ucE1]] = [[ucE2]] \in [[UC1 & UC2]]$.
        \end{itemize}
        Similarly, if we take an arbitrary $[[ucE2]] \in [[UC2]]$,
        then $[[ucE1]] = [[ucE2]] \in [[UC1 & UC2]]$. 
    \end{itemize}
\end{proof}

\begin{corollary}
    \label{corollary:unif-merge-soundness}
    Suppose that $[[Θ ⊢ UC1]]$ and $[[Θ ⊢ UC2]]$ 
    are normalized unification constraints.
    If $[[Θ ⊢ UC1 & UC2 = UC]]$ is defined then
    \begin{enumerate}
        \item $[[Θ ⊢ UC]]$ is normalized unification constraint,
        \item for any substitution $[[Θ ⊢ uσ]]$, $[[(Θ  ⊢  lift UC) ⊢ uσ]]$ implies 
        $[[(Θ  ⊢  lift UC1) ⊢ uσ]]$ and $[[(Θ  ⊢  lift UC2) ⊢ uσ]]$.
    \end{enumerate}
\end{corollary}
\begin{proof}
    It is clear that since $[[UC = UC1 ∪ UC2]]$ (by \cref{lemma:unif-merge-soundness}),
    and being normalized means that all entries are normalized,
    $[[UC]]$ is a normalized unification constraint.
    Analogously, $[[Θ ⊢ UC]] = [[UC1 ∪ UC2]]$ holds immediately, 
    since $[[Θ ⊢ UC1]]$ and $[[Θ ⊢ UC2]]$.

    Let us take an arbitrary substitution $[[Θ ⊢ uσ]]$ and assume that 
    $[[(Θ  ⊢  lift UC) ⊢ uσ]]$.
    Then $[[(Θ  ⊢  lift UCi) ⊢ uσ]]$ holds by definition:
    If $[[ucE]] \in [[lift UCi]] \subseteq [[lift UC1 ∪ lift UC2]] = [[lift UC]]$.
    So $[[Θ(α̂±) ⊢ [uσ]α̂± : ucE]]$ holds.
\end{proof}


% \begin{lemma} [Completeness of Unification Constraint Merge]
%     \label{lemma:unif-merge-completeness}
%     Suppose that $[[Θ ⊢ UC|varset1]]$ and $[[Θ ⊢ UC|varset2]]$.
%     Then $[[Θ ⊢ UC|varset1 & UC|varset2 = UC']]$ exists and
%     $[[UC' = UC|varset1 ∪ varset2]]$.
% \end{lemma}
% \begin{proof}
%     $[[Θ ⊢ UC|varset1 & UC|varset2 = UC']]$ is defined as the union of three parts:
%     entries of $[[UC|varset1]]$ that do not have matching entries of $[[UC|varset2]]$,
%     entries of $[[UC|varset2]]$ that do not have matching entries of $[[UC|varset1]]$,
%     and the merge of matching entries.
%     The first two parts are defined. The merge of matching entries is defined
%     by \ruleref{\ottdruleUCMEPEqEqLabel}, since the matching entries must be equal
%     if they both belong to $[[UC]]$.

%     It remains to show that $[[UC' = UC|varset1 ∪ varset2]]$.
%     It is easy to see that the three parts comprising $[[UC']]$ 
%     correspond to the three parts comprising 
%     $[[UC|varset1 ∪ varset2]] = [[UC | (varset1 \ varset2) ∪ UC | (varset2 \ varset1) ∪ UC | varset1 ∩ varset2]]$. 
% \end{proof}

\begin{lemma} [Completeness of Unification Constraint Entry Merge]
    \label{lemma:unif-entry-merge-completeness}
    For a fixed context $[[Γ]]$,
    suppose that $[[Γ ⊢ ucE1]]$ and $[[Γ ⊢ ucE2]]$ are matching constraint entries.
    \begin{itemize}
        \item for a type $[[iP]]$ such that $[[Γ ⊢ iP : ucE1]]$ and $[[Γ ⊢ iP : ucE2]]$,
        $[[Γ ⊢ ucE1 & ucE2 = ucE]]$ is defined and $[[Γ ⊢ iP : ucE]]$.
        \item for a type $[[iN]]$ such that $[[Γ ⊢ iN : ucE1]]$ and $[[Γ ⊢ iN : ucE2]]$,
        $[[Γ ⊢ ucE1 & ucE2 = ucE]]$ is defined and $[[Γ ⊢ iN : ucE]]$.
    \end{itemize}
\end{lemma}
\begin{proof}
    The proof repeats the one of \cref{lemma:entry-merge-completeness}
    and is done by the case analysis on the shape of $[[ucE1]]$ and $[[ucE2]]$.
    However, it only needs to consider two cases.
    \begin{caseof}
        \item $[[ucE1]]$ is $[[pua :≈ iQ1]]$ and $[[ucE2]]$ is $[[pua :≈ iQ2]]$.
        \item $[[ucE1]]$ is $[[nua :≈ iN1]]$ and $[[ucE2]]$ is $[[nua :≈ iM2]]$.
    \end{caseof}
    The proof of these cases is based only on \cref{lemma:subt-equiv-algorithmization}
    and \cref{corollary:equivalence-transitivity}, and does not require the properties of 
    the least upper bound or subtyping.
\end{proof}

\begin{lemma} [Completeness of Unification Constraint Merge] 
    \label{lemma:unif-merge-completeness}
    Suppose that $[[Θ ⊢ UC1]]$ and $[[Θ ⊢ UC2]]$.
    Then for any substitution $[[Θ ⊢ uσ]]$ such that $[[(Θ  ⊢  lift UC1) ⊢ uσ]]$ and $[[(Θ  ⊢  lift UC2) ⊢ uσ]]$, 
    \begin{enumerate}
        \item $[[Θ ⊢ UC1 & UC2 = UC]]$ is defined and
        \item $[[(Θ  ⊢  UC) ⊢ uσ]]$.
    \end{enumerate}
\end{lemma}
\begin{proof}
    The proof repeats the proof of \cref{lemma:merge-completeness} but 
    uses \cref{lemma:unif-entry-merge-completeness} instead of \cref{lemma:entry-merge-completeness}.
\end{proof}


\subsection{Unification}
\begin{lemma}[Soundness of Unification] \label{lemma:unification-soundness}
    \hfill
    \begin{itemize}
        \item [$+$] For normalized $[[uP]]$ and $[[iQ]]$ such that 
        $[[Γ ; Θ ⊢ uP]]$ and $[[Γ ⊢ iQ]]$,\\ 
        if $[[Γ ; Θ ⊨ uP ≈u iQ ⫤ UC]]$ then 
        $[[Θ ⊢ UC]]$ and for any normalized $[[uσ]]$ such that $[[Θ ⊢ uσ : lift UC]]$,
        $[[ [uσ]uP = iQ ]]$.

        \item [$-$] For normalized $[[uN]]$ and $[[iM]]$ such that
        $[[Γ ; Θ ⊢ uN]]$ and $[[Γ ⊢ iM]]$,\\
        if $[[Γ ; Θ ⊨ uN ≈u iM ⫤ UC]]$ then 
        $[[Θ ⊢ UC]]$ and for any normalized $[[uσ]]$ such that $[[Θ ⊢ uσ : lift UC]]$,
        $[[ [uσ]uN = iM ]]$.
    \end{itemize}
\end{lemma}
\begin{proof}
    We prove by induction on the derivation of 
    $[[ Γ ; Θ ⊨ uN ≈u iM ⫤ UC ]]$ and mutually $[[Γ ; Θ ⊨ uP ≈u iQ ⫤ UC]]$.
    Let us consider the last rule forming this derivation. 
    \begin{caseof}
        \item \ruleref{\ottdruleUNVarLabel}, then $[[uN]] = [[α⁻]] = [[iM]]$.
        The resulting unification constraint is empty: $[[UC]] = [[·]]$.
        It satisfies $[[Θ ⊢ UC]]$ vacuously, and $[[ [us]α⁻ = α⁻ ]]$, that is $[[ [us]uN = iM ]]$.

        \item \ruleref{\ottdruleUShiftULabel}, then $[[uN]] = [[↑uP]]$ and $[[iM]] = [[↑iQ]]$.
        The algorithm makes a recursive call to $[[Γ ; Θ ⊨ uP ≈u iQ ⫤ UC]]$ returning $[[UC]]$.
        By induction hypothesis, $[[Θ ⊢ UC]]$ and for any $[[Θ ⊢ uσ : lift UC]]$,
        $[[ [uσ]uN ]] = [[ [uσ]↑uP ]] = [[ ↑[uσ]uP ]] = [[ ↑iQ ]] = [[ iM ]]$, as 
        required.

        \item \ruleref{\ottdruleUArrowLabel}, then $[[uN]] = [[uP → uN']]$ and $[[iM]] = [[iQ → iM']]$.
        The algorithm makes two recursive calls to $[[Γ ; Θ ⊨ uP ≈u iQ ⫤ UC1]]$ and
        $[[Γ ; Θ ⊨ uN' ≈u iM' ⫤ UC2]]$ returning $[[Θ ⊢ UC1 & UC2 = UC]]$ as the result.

        It is clear that $[[uP]]$, $[[uN']]$, $[[iQ]]$, and $[[iM']]$ are normalized,
        and that $[[Γ ; Θ ⊢ uP]]$, $[[Γ ; Θ ⊢ uN']]$, $[[Γ ⊢ iQ]]$, and $[[Γ ⊢ iM']]$.
        This way, the induction hypothesis is applicable to both recursive calls.

        By applying the induction hypothesis to $[[Γ ; Θ ⊨ uP ≈u iQ ⫤ UC1]]$,
        we have:
        \begin{itemize}
            \item $[[Θ ⊢ UC1]]$,
            \item for any $[[Θ ⊢ uσ' : lift UC1]]$, $[[ [uσ']uP = iQ ]]$.
        \end{itemize}
        By applying it to $[[Γ ; Θ ⊨ uN' ≈u iM' ⫤ UC2]]$, we have:
        \begin{itemize}
            \item $[[Θ ⊢ UC2]]$,
            \item for any $[[Θ ⊢ uσ' : lift UC2]]$, $[[ [uσ']uN' = iM' ]]$.
        \end{itemize}


        Let us take an arbitrary $[[Θ ⊢ uσ : lift UC]]$.
        By the soundness of the constraint merge (\cref{lemma:merge-soundness}), 
        $[[Θ ⊢ lift UC1 & lift UC2 = lift UC]]$ implies
        $[[Θ ⊢ uσ : lift UC1 ]]$ and $[[Θ ⊢ uσ : lift UC2]]$.

        Applying the induction hypothesis to $[[Θ ⊢ uσ : lift UC1]]$, we have
        $[[ [uσ]uP = iQ ]]$; applying it to $[[Θ ⊢ uσ : lift UC2]]$, we have
        $[[ [uσ]uN' = iM' ]]$.
        This way, $[[ [uσ]uN ]] = [[ [uσ]uP → [uσ]uN' ]] = [[ iQ → iM' ]] = [[ iM ]]$.

        \item \ruleref{\ottdruleUForallLabel}, then $[[uN]] = [[∀pas.uN']]$ and $[[iM]] = [[∀pas.iM']]$.
        The algorithm makes a recursive call to $[[Γ,pas ; Θ ⊨ uN' ≈u iM' ⫤ UC]]$
        returning $[[UC]]$ as the result.

        The induction hypothesis is applicable: $[[Γ,pas ; Θ ⊢ uN']]$ and $[[Γ,pas ⊢ iM']]$ hold
        by inversion, and $[[uN']]$ and $[[iM']]$ are normalized, since $[[uN]]$ and $[[iM]]$ are.
        Let us take an arbitrary $[[Θ ⊢ uσ : lift UC]]$.
        By the induction hypothesis, $[[ [uσ]uN' ]] = [[ iM' ]]$. 
        Then $[[ [uσ]uN ]] = [[ [uσ]∀pas.uN' ]] = [[ ∀pas.[uσ]uN' ]] = [[ ∀pas.iM' ]] = [[ iM ]]$.

        \item \ruleref{\ottdruleUNUVarLabel}, then $[[uN]] = [[α̂⁻]]$, $[[â⁻[Δ] ∊ Θ]]$, and $[[Δ ⊢ iM]]$.
        As the result, the algorithm returns $[[UC]] = [[ (â⁻ :≈ iM) ]]$.

        It is clear that $[[α̂⁻[Δ] ⊢ (â⁻ :≈ iM) ]]$, since $[[Δ ⊢ iM]]$, 
        meaning that $[[Θ ⊢ UC]]$.

        Let us take an arbitrary $[[uσ]]$ such that  $[[Θ ⊢ uσ : lift UC]]$.
        Since $[[UC]] = [[ (â⁻ :≈ iM) ]]$, $[[Θ ⊢ uσ : lift UC]]$ implies 
        $[[Θ(â⁻) ⊢ [uσ]â⁻ : (â⁻ :≈ iM) ]]$.
        By inversion of \ruleref{\ottdruleSATSCENEqLabel}, it  means $[[Θ(â⁻) ⊢ [uσ]â⁻ ≈ iM]]$.
        This way, $[[Θ(â⁻) ⊢ [uσ]uN ≈ iM]]$. 
        Notice that $[[uσ]]$ and $[[uN]]$ are normalized, and by \cref{lemma:norm-subst-distr}, 
        so is $[[ [uσ]uN ]]$.
        Since both sides of $[[Θ(â⁻) ⊢ [uσ]uN ≈ iM]]$ are normalized,
        by \cref{lemma:subt-equiv-algorithmization}, we have $[[ [uσ]uN = iM ]]$.

        \item The positive cases are proved symmetrically.
    \end{caseof}
\end{proof}

\begin{lemma}[Completeness of Unification] \label{lemma:unification-completeness}
    \hfill
    \begin{itemize}
        \item [$+$] For normalized $[[uP]]$ and $[[iQ]]$ such that
        $[[Γ ; Θ ⊢ uP]]$ and $[[Γ ⊢ iQ]]$, 
        for any $[[Θ ⊢ uσ]]$ such that $[[ [uσ]uP = iQ ]]$,
        there exists $[[Γ ; Θ ⊨ uP ≈u iQ ⫤ UC]]$,
        and $[[Θ ⊢ uσ : lift UC]]$.
        
        \item [$-$] For normalized $[[uN]]$ and $[[iM]]$ such that
        $[[Γ ; Θ ⊢ uN]]$ and $[[Γ ⊢ iM]]$,\\
        for any $[[Θ ⊢ uσ]]$ such that $[[ [uσ]uN = iM ]]$,
        there exists $[[Γ ; Θ ⊨ uN ≈u iM ⫤ UC]]$,
        and $[[Θ ⊢ uσ : lift UC]]$.
   \end{itemize}
\end{lemma}
\begin{proof}
    We prove it by induction on the structure of $[[uP]]$ and mutually, $[[uN]]$.
    \begin{caseof}
        \item $[[uN]] = [[α̂⁻]]$\\
            $[[Γ ; Θ ⊢ α̂⁻]]$ means that $[[ α̂⁻[Δ] ]] \in [[Θ]]$ for some $[[Δ]]$.

            Let us take an arbitrary $[[Θ ⊢ uσ]]$ such that $[[ [uσ]α̂⁻ = iM ]]$.
            $[[Θ ⊢ uσ]]$ means that $[[Δ ⊢ iM]]$.
            This way, \ruleref{\ottdruleUNUVarLabel} is applicable to infer 
            $[[Γ ; Θ ⊨ â⁻ ≈u iM ⫤ (â⁻ :≈ iM)]]$.
            $[[Θ ⊢ uσ : lift (â⁻ :≈ iM)]]$ holds by \ruleref{\ottdruleSATSCENEqLabel}. 
            
        \item $[[uN]] = [[α⁻]]$\\
            Let us take an arbitrary $[[Θ ⊢ uσ]]$ such that $[[ [uσ]α⁻ = iM ]]$.
            The latter means $[[iM = α⁻]]$.

            Then $[[ [us]α⁻ = iM ]]$ means $[[iM = α⁻]]$.
            This way, \ruleref{\ottdruleUNVarLabel} infers 
            $[[Γ; Θ ⊨ a⁻ ≈u a⁻ ⫤ ·]]$, which is rewritten as $[[Γ; Θ ⊨ uN ≈u iM ⫤ ·]]$, 
            and $[[Θ ⊢ uσ : lift ·]]$ holds trivially.

        \item $[[uN]] = [[↑uP]]$\\
            Let us take an arbitrary $[[Θ ⊢ uσ]]$ such that $[[ [uσ]↑uP = iM ]]$.
            The latter means $[[ ↑[uσ]uP = iM ]]$, i.e.
            $[[iM]] = [[↑iQ]]$ for some $[[iQ]]$ and $[[ [uσ]uP = iQ ]]$.

            Let us show that the induction hypothesis is applicable to $[[ [uσ]uP = iQ ]]$.
            Notice that $[[uP]]$ is normalized, since $[[uN]] = [[↑uP]]$ is normalized,
            $[[Γ ; Θ ⊢ uP]]$ holds by inversion of $[[Γ ; Θ ⊢ ↑uP]]$, 
            and $[[Γ ⊢ iQ]]$ holds by inversion of $[[Γ ⊢ ↑iQ]]$.

            This way, by the induction hypothesis there exists $[[UC]]$ such that
            $[[Γ ; Θ ⊨ uP ≈u iQ ⫤ UC]]$, and moreover, $[[Θ ⊢ uσ : lift UC]]$.
            
        \item $[[uN]] = [[uP → uN']]$\\
            Let us take an arbitrary $[[Θ ⊢ uσ]]$ such that $[[ [uσ](uP → uN') = iM ]]$.
            The latter means $[[ [uσ]uP → [uσ]uN' = iM ]]$, i.e.
            $[[iM]] = [[iQ → iM']]$ for some $[[iQ]]$ and $[[iM']]$, 
            such that $[[ [uσ]uP = iQ ]]$ and $[[ [uσ]uN' = iM' ]]$.

            Let us show that the induction hypothesis is applicable to 
            $[[ [uσ]uP = iQ ]]$ and to $[[ [uσ]uN' = iM' ]]$:
            \begin{itemize}
                \item $[[uP]]$ and $[[uN']]$ are normalized, since $[[uN]] = [[uP → uN']]$ is normalized
                \item $[[Γ ; Θ ⊢ uP]]$ and $[[Γ ; Θ ⊢ uN']]$ follow from the inversion of $[[Γ ; Θ ⊢ uP → uN']]$,
                \item $[[Γ ⊢ iQ]]$ and $[[Γ ⊢ iM']]$ follow from inversion of $[[Γ ⊢ iQ → iM']]$.
            \end{itemize}

            Then by the induction hypothesis, $[[Γ ; Θ ⊨ uP ≈u iQ ⫤ UC1]]$ and $[[Θ ⊢ uσ : lift UC1]]$,
            $[[Γ ; Θ ⊨ uN' ≈u iM' ⫤ UC2]]$ and $[[Θ ⊢ uσ : lift UC2]]$.
            To apply \ruleref{\ottdruleUArrowLabel} and infer the required
            $[[Γ ; Θ ⊨ uN ≈u iM ⫤ UC]]$, we need to show that
            $[[Θ ⊢ UC1 & UC2 = UC]]$ is defined and $[[Θ ⊢ uσ : lift UC]]$.
            It holds by the completeness of the unification constraint merge 
            (\cref{lemma:merge-completeness}):
            \begin{itemize}
                \item $[[Θ ⊢ UC1]]$ and $[[Θ ⊢ UC2]]$ holds by the soundness of unification (\cref{lemma:unification-soundness})
                \item $[[Θ ⊢ uσ : lift UC1]]$ and $[[Θ ⊢ uσ : lift UC2]]$ holds as noted above 
            \end{itemize}.

        \item $[[uN]] = [[∀pas.uN']]$\\
            Let us take an arbitrary $[[Θ ⊢ uσ]]$ such that $[[ [uσ]∀pas.uN' = iM ]]$.
            The latter means $[[ ∀pas.[uσ]uN' = iM ]]$, i.e.
            $[[iM]] = [[∀pas.iM']]$ for some $[[iM']]$ such that $[[ [uσ]uN' = iM' ]]$.

            Let us show that the induction hypothesis is applicable to $[[ [uσ]uN' = iM' ]]$.
            Notice that $[[uN']]$ is normalized, since $[[uN]] = [[∀pas.uN']]$ is normalized,
            $[[Γ,pas ; Θ ⊢ uN']]$ follows from inversion of $[[Γ ; Θ ⊢ ∀pas.uN']]$,
            $[[Γ,pas ⊢ iM']]$ follows from inversion of $[[Γ ⊢ ∀pas.iM']]$, and
            $[[Θ ⊢ uσ]]$ by assumption. 

            This way, by the induction hypothesis, $[[Γ,pas ; Θ ⊨ uN' ≈u iM' ⫤ UC]]$ exists and 
            moreover, $[[Θ ⊢ uσ : lift UC]]$.
            Hence, \ruleref{\ottdruleUForallLabel} is applicable to infer
            $[[Γ ; Θ ⊨ ∀pas.uN' ≈u ∀pas.iM' ⫤ UC]]$, that is $[[Γ ; Θ ⊨ uN ≈u iM ⫤ UC]]$.

        \item The positive cases are proved symmetrically.
    \end{caseof}
\end{proof}
\

\subsection{Anti-unification}
\obsAuDeterministic*
\begin{proof}
    By trivial induction on $[[Γ ⊨ iP1 ≈au iP2 ⫤ (Ξ, uQ, aus1, aus2)]]$
    and mutually on $[[Γ ⊨ iN1 ≈au iN2 ⫤ (Ξ, uM, aus1, aus2)]]$.
\end{proof}

\obsNamesDefinedByMapping*
\begin{proof}
    By simple induction on $[[Γ ⊨ iP1 ≈au iP2 ⫤ (Ξ, uQ, aus1, aus2)]]$
    and mutually on $[[Γ ⊨ iN1 ≈au iN2 ⫤ (Ξ, uM, aus1, aus2)]]$.
    Let us consider tha last rule applied to infer this judgment.
    \begin{caseof}
        \item \ruleref{\ottdruleAUPVarLabel} or \ruleref{\ottdruleAUNVarLabel},
        then $[[Ξ]] = [[·]]$, and the property holds vacuously.

        \item \ruleref{\ottdruleAUAULabel}
        Then  $[[Ξ]] = [[â⁻_{iN1, iN2}]]$,
        $[[aus1]] = [[â⁻_{iN1, iN2} ↦ iN1]]$, and $[[aus2]] = [[â⁻_{iN1, iN2} ↦ iN2]]$.
        So the property holds trivially.

        \item \ruleref{\ottdruleAUArrowLabel}
        In this case, $[[Ξ]] = [[Ξ' ∪ Ξ'']]$, $[[aus1]] = [[aus'1 ∪ aus''1]]$, and 
        $[[aus2]] = [[aus'2 ∪ aus''2]]$,
        where the property holds for ($[[Ξ']]$, $[[aus'1]]$, $[[aus'2]]$) and 
        ($[[Ξ'']]$, $[[aus''1]]$, $[[aus''2]]$) by the induction hypothesis.
        Then since the union of solutions does not change the types the variables are mapped to,
        the required property holds for $[[Ξ]]$, $[[aus1]]$, and $[[aus2]]$.

        \item For the other rules, the resulting $[[Ξ]]$ is taken from the recursive call
        and the required property holds immediately by the induction hypothesis.
    \end{caseof}
\end{proof}

\lemmaAuSoundness*
\begin{proof}
    We prove it by induction on 
    $[[Γ ⊨ iN1 ≈au iN2 ⫤ (Ξ, uM, aus1, aus2)]]$
    and mutually, $[[Γ ⊨ iP1 ≈au iP2 ⫤ (Ξ, uQ, aus1, aus2)]]$.
    Let us consider the last rule applied to infer this judgement.
    \begin{caseof}
        \item \ruleref{\ottdruleAUNVarLabel}, then $[[iN1]] = [[α⁻]] = [[iN2]]$,
              $[[Ξ]] = [[·]]$, $[[uM]] = [[α⁻]]$, and $[[aus1]] = [[aus2]] = [[·]]$.
            \begin{enumerate}
                \item $[[Γ ; · ⊢ α⁻]]$ follows from the assumption $[[Γ ⊢ α⁻]]$,
                \item $[[Γ ; · ⊢ · : ·]]$ holds trivially, and
                \item $[[ [·] α⁻ = α⁻ ]]$ holds trivially.
            \end{enumerate}
        \item \label{case:anti-unification-soundness:shiftu}
         \ruleref{\ottdruleAUShiftULabel}, then $[[iN1]] = [[↑iP1]]$,
                $[[iN2]] = [[↑iP2]]$, and the algorithm makes the recursive call:
                $[[Γ ⊨ iP1 ≈au iP2 ⫤ (Ξ, uQ, aus1, aus2)]]$, 
                returning $[[(Ξ, ↑uQ, aus1, aus2)]]$ as the result.

                Since $[[iN1]] = [[↑iP1]]$ and $[[iN2]] = [[↑iP2]]$ are normalized, 
                so are $[[iP1]]$ and $[[iP2]]$, and thus, the induction hypothesis 
                is applicable to $[[Γ ⊨ iP1 ≈au iP2 ⫤ (Ξ, uQ, aus1, aus2)]]$:
                \begin{enumerate}
                    \item $[[Γ ; Ξ ⊢ uQ]]$, and hence, $[[Γ ; Ξ ⊢ ↑uQ]]$,
                    \item $[[Γ ; · ⊢ ausi : Ξ]]$ for $i \in \{1,2\}$, and
                    \item $[[ [ausi] uQ = iPi ]]$ for $i \in \{1,2\}$, and then by 
                    the definition of the substitution, $[[ [ausi] ↑uQ = ↑iPi ]]$ for $i \in \{1,2\}$.
                \end{enumerate}

        \item \ruleref{\ottdruleAUArrowLabel}, then $[[iN1]] = [[iP1 → iN'1]]$,
                $[[iN2]] = [[iP2 → iN'2]]$, and the algorithm makes two recursive calls:
                $[[Γ ⊨ iP1 ≈au iP2 ⫤ (Ξ, uQ, aus1, aus2)]]$ and
                $[[Γ ⊨ iN'1 ≈au iN'2 ⫤ (Ξ', uM, aus'1, aus'2)]]$ and
                and returns $[[(Ξ ∪ Ξ', uQ → uM, aus1 ∪ aus'1, aus2 ∪ aus'2)]]$ as the result.
                
                Notice that the induction hypothesis is applicable to 
                $[[Γ ⊨ iP1 ≈au iP2 ⫤ (Ξ, uQ, aus1, aus2)]]$:
                $[[iP1]]$ and $[[iP2]]$ are normalized, since $[[iN1]] = [[iP1 → iN'1]]$
                and $[[iN2]] = [[iP2 → iN'2]]$ are normalized.
                Similarly, the induction hypothesis is applicable to
                $[[Γ ⊨ iN'1 ≈au iN'2 ⫤ (Ξ', uM, aus'1, aus'2)]]$.

                This way, by the induction hypothesis:
                \begin{enumerate}
                    \item $[[Γ ; Ξ ⊢ uQ]]$ and $[[Γ ; Ξ' ⊢ uM]]$. 
                    Then by weakening (\cref{lemma:wf-weakening-algo}), 
                    $[[Γ ; Ξ ∪ Ξ' ⊢ uQ]]$ and 
                    $[[Γ ; Ξ ∪ Ξ' ⊢ uM]]$, which implies $[[Γ ; Ξ ∪ Ξ' ⊢ uQ → uM]]$;

                    \item $[[Γ ; · ⊢ ausi : Ξ]]$ and $[[Γ ; · ⊢ aus'i : Ξ']]$
                        Then $[[Γ ; · ⊢ ausi ∪ aus'i : Ξ ∪ Ξ']]$
                        are well-defined anti-unification substitutions.
                        Let us take an arbitrary $[[β̂⁻]] \in [[Ξ ∪ Ξ']]$.
                        If $[[β̂⁻]] \in [[Ξ]]$.
                        then $[[Γ ; · ⊢ ausi : Ξ]]$ implies that $[[ausi]]$, and hence,
                        $[[ausi ∪ aus'i]]$ contains an entry well-formed in $[[Γ]]$.
                        If $[[β̂⁻]] \in [[Ξ']]$, the reasoning is symmetric.

                        $[[ausi ∪ aus'i]]$ is a well-defined anti-unification substitution:
                        any anti-unification variable occurs 
                        uniquely $[[ausi ∪ aus'i]]$, since by \cref{obs:names-defined-by-mapping},
                        the name of the variable is in one-to-one correspondence with 
                        the pair of types it is mapped to
                        by $[[aus1]]$ and $[[aus2]]$, 
                        an is in one-to-one correspondence with the pair of types it is mapped to
                        by $[[aus'1]]$ and $[[aus'2]]$ 
                        i.e.  if $[[β̂⁻]] \in [[Ξ ∩ Ξ']]$ then $[[ [aus1]β̂⁻ = [aus'1]β̂⁻ ]]$,
                        and $[[ [aus2]β̂⁻ = [aus'2]β̂⁻ ]]$.

                    \item $[[ [ausi] uQ = iPi ]]$ and $[[ [aus'i] uM = iN'i ]]$.
                    Since $[[ausi ∪ aus'i]]$ restricted to $[[Ξ]]$ is $[[ausi]]$,
                    and $[[ausi ∪ aus'i]]$ restricted to $[[Ξ']]$ is $[[aus'i]]$,
                    we have $[[ [ausi ∪ aus'i] uQ = iPi ]]$ and 
                    $[[ [ausi ∪ aus'i] uM = iN'i ]]$, and thus, 
                    $[[ [ausi ∪ aus'i] uQ → uM = iP1 → iN'1 ]]$
                \end{enumerate}

        \item \ruleref{\ottdruleAUForallLabel}, then $[[iN1]] = [[∀pas.iN'1]]$,
                $[[iN2]] = [[∀pas.iN'2]]$, and the algorithm makes a recursive call:
                $[[Γ ⊨ iN'1 ≈au iN'2 ⫤ (Ξ, uM, aus1, aus2)]]$ and
                returns $[[(Ξ, ∀pas.uM, aus1, aus2)]]$ as the result.

                Similarly to \cref{case:anti-unification-soundness:shiftu}, 
                we apply the induction hypothesis to
                $[[Γ ⊨ iN'1 ≈au iN'2 ⫤ (Ξ, uM, aus1, aus2)]]$ to obtain:
                \begin{enumerate}
                    \item $[[Γ; Ξ ⊢ uM]]$, and hence, $[[Γ ; Ξ ⊢ ∀pas.uM]]$;
                    \item $[[Γ; · ⊢ ausi : Ξ]]$ for $i \in \{1,2\}$, and
                    \item $[[ [ausi] uM = iN'i ]]$ for $i \in \{1,2\}$,
                        and then by the definition of the substitution,
                        $[[ [ausi] ∀pas.uM = ∀pas.iN'i ]]$ for $i \in \{1,2\}$. 
                \end{enumerate}

        \item \ruleref{\ottdruleAUAULabel}, which applies 
        when other rules do not, and $[[G ⊢ iNi]]$,
        returning as the result $[[(Ξ, uM, aus1, aus2)]] = $
        $[[({â⁻_{iN1, iN2}}, â⁻_{iN1, iN2}, (â⁻_{iN1, iN2} ↦ iN1) ,  (â⁻_{iN1, iN2} ↦ iN2))]]$.

        \begin{enumerate}
            \item $[[Γ ; Ξ ⊢ uM]]$ is rewritten as $[[Γ ; { â⁻_{iN1, iN2} } ⊢ â⁻_{iN1, iN2}]]$,
                which holds trivially;
            \item $[[Γ ; · ⊢ ausi : Ξ]]$ is rewritten as $[[Γ ; · ⊢ (â⁻_{iN1, iN2} ↦ iNi) : {â⁻_{iN1, iN2}}]]$,
                which holds since $[[Γ ⊢ iNi]]$ by the premise of the rule;
            \item $[[ [ausi] uM = iNi ]]$ is rewritten as $[[ [â⁻_{iN1, iN2} ↦ iNi] â⁻_{iN1, iN2} = iNi ]]$,
                which holds trivially by the definition of substitution.
        \end{enumerate}

        \item Positive cases are proved symmetrically.
    \end{caseof}
\end{proof}

\lemmaAuCompleteness*
\begin{proof}
    We prove it by the induction on $[[uM']]$ and mutually on $[[uQ']]$.
    \begin{caseof}
        \item $[[uM']] = [[â⁻]]$ 
            Then since $[[Γ ; · ⊢ aus'i : Ξ']]$,
            $[[Γ ⊢ [aus'i] uM']] = [[ iNi ]]$. 
            This way, \ruleref{\ottdruleAUAULabel} is always applicable
            if other rules are not.

        \item $[[uM']] = [[α⁻]]$
            Then $[[α⁻]] = [[ [aus'i] α⁻]] = [[ iNi ]]$, which means
            that \ruleref{\ottdruleAUNVarLabel} is applicable.

        \item $[[uM']] = [[↑uQ']]$
            Then $[[ ↑[aus'i]uQ']] = [[ [aus'i]↑uQ']] = [[ iNi ]]$, that is
            $[[iN1]]$ and $[[iN2]]$ have form $[[↑iP1]]$ and $[[↑iP2]]$ respectively.

            Moreover, $[[ [aus'i]uQ' = iPi ]]$, which means that 
            $[[(Ξ', uQ', aus'1, aus'2)]]$ is an anti-unifier of $[[iP1]]$ and $[[iP2]]$.
            Then by the induction hypothesis, there exists $[[(Ξ, uQ, aus1, aus2)]]$ such that
            $[[Γ ⊨ iP1 ≈au iP2 ⫤ (Ξ, uQ, aus1, aus2)]]$, and hence, 
            $[[Γ ⊨ ↑iP1 ≈au ↑iP2 ⫤ (Ξ, ↑uQ, aus1, aus2)]]$ by \ruleref{\ottdruleAUShiftULabel}.
        \item $[[uM']] = [[∀pas.uM'']]$ This case is similar to the previous one:
            we consider $\forall[[pas]]$ as a constructor. 
            Notice that $[[∀pas.[aus'i]uM'']] = [[ [aus'i]∀pas.uM'']] = [[ iNi ]]$, that is
            $[[iN1]]$ and $[[iN2]]$ have form $[[∀pas.iN''1]]$ and $[[∀pas.iN''2]]$ respectively.

            Moreover, $[[ [aus'i]uM'' = iN''i ]]$, which means that
            $[[(Ξ', uM'', aus'1, aus'2)]]$ is an anti-unifier of $[[iN''1]]$ and $[[iN''2]]$.
            Then by the induction hypothesis, there exists $[[(Ξ, uM, aus1, aus2)]]$ such that
            $[[Γ ⊨ iN''1 ≈au iN''2 ⫤ (Ξ, uM, aus1, aus2)]]$, and hence,
            $[[Γ ⊨ ∀pas.iN''1 ≈au ∀pas.iN''2 ⫤ (Ξ, ∀pas.uM, aus1, aus2)]]$ by 
            \ruleref{\ottdruleAUForallLabel}.
        \item $[[uM']] = [[uQ' → uM'']]$
            Then $[[ [aus'i]uQ' → [aus'i]uM'']] = [[ [aus'i](uQ' → uM'')]] = [[ iNi ]]$, that is
            $[[iN1]]$ and $[[iN2]]$ have form $[[uP1 → uN'1]]$ and $[[uP2 → uN'2]]$ respectively.

            Moreover, $[[ [aus'i]uQ' = iPi ]]$ and $[[ [aus'i]uM'' = iN''i ]]$, which means that
            $[[(Ξ', uQ', aus'1, aus'2)]]$ is an anti-unifier of $[[iP1]]$ and $[[iP2]]$,
            and $[[(Ξ', uM'', aus'1, aus'2)]]$ is an anti-unifier of $[[iN''1]]$ and $[[iN''2]]$.
            Then by the induction hypothesis, 
            $[[Γ ⊨ iP1 ≈au iP2 ⫤ (Ξ1, uQ, aus1, aus2)]]$ and 
            $[[Γ ⊨ iN''1 ≈au iN''2 ⫤ (Ξ2, uM, aus3, aus4)]]$ succeed.
            The result of the algorithm is $[[(Ξ1 ∪ Ξ2, uQ → uM, aus1 ∪ aus3, aus2 ∪ aus4)]]$.

        \item $[[uQ']] = [[â⁺]]$ 
            This case if not possible, since $[[Γ ; Ξ' ⊢ uQ']]$ means 
            $[[â⁺]] \in [[Ξ']]$, but $[[Ξ']]$ can only contain negative variables. 

        \item Other positive cases are proved symmetrically to the corresponding negative ones.
    \end{caseof}
\end{proof}

\lemmaAuInitial*
\begin{proof}
    First, let us assume that $[[uM']]$ is a algorithmic variable $[[α̂⁻]]$. 
    Then we can take $[[ausr]] = [[α̂⁻]] \mapsto [[uM]]$, which satisfies the required properties:
    \begin{itemize}
        \item $[[Γ ; Ξ ⊢ ausr : (Ξ' | uv uM')]]$ holds since 
        $[[Ξ' | uv uM']] = [[{α̂⁻}]]$ and $[[Γ ; Ξ ⊢ uM]]$ by the soundness of anti-unification (\cref{lemma:au-soundness});
        \item $[[ [ausr] uM' = uM ]]$ holds by construction
        \item $[[ [ausr]α̂⁻]] = [[uM]]$ is the anti-unifier of 
            $[[iN1]] = [[ [aus'1] α̂⁻]]$ and $[[iN2]] = [[ [aus'2] α̂⁻]]$
            in context $[[Γ]]$, and hence, it is uniquely determined by them (\cref{obs:au-deterministic}).
    \end{itemize}

    Now, we can assume that $[[uM']]$ is not a algorithmic variable. 
    We prove by induction on the derivation of $[[Γ ⊨ iP1 ≈au iP2 ⫤ (Ξ, uQ, aus1, aus2)]]$
    and mutually on the derivation of $[[Γ ⊨ iN1 ≈au iN2 ⫤ (Ξ, uM, aus1, aus2)]]$.

    Since $[[uM']]$ is not a algorithmic variable, 
    the substitution acting on $[[uM']]$ preserves its outer constructor. 
    In other words, 
    $[[ [aus'i] uM' = iNi ]]$ means that $[[uM']]$, 
    $[[iN1]]$ and $[[iN2]]$ have the same outer constructor. 
    Let us consider the algorithmic anti-unification rule corresponding to this constructor, 
    and show that it was successfully applied to anti-unify $[[iN1]]$ and $[[iN2]]$ 
    (or $[[iP1]]$ and $[[iP2]]$).

    \begin{caseof}
        \item \ruleref{\ottdruleAUNVarLabel}, i.e. $[[iN1]] = [[α⁻]] = [[iN2]]$.
        \label{case:anti-unification-initial:nvar}
        This rule is applicable since it has no premises. 
        
        Then $[[Ξ]] = [[·]]$, $[[uM]] = [[α⁻]]$, 
        and $[[aus1]] = [[aus2]] = [[·]]$.
        Since $[[ [aus'i] uM' = iNi ]] = [[α⁻]]$
        and $[[uM']]$ is not a algorithmic variable, $[[uM']] = [[α⁻]]$.
        Then we can take $[[ausr]] = [[·]]$, which satisfies the required properties:
        \begin{itemize}
            \item $[[Γ ; Ξ ⊢ ausr : (Ξ' | uv uM')]]$ holds vacuously since 
            $[[Ξ' | uv uM']] = [[∅]]$; 
            \item $[[ [ausr] uM' = uM ]]$, that is $[[ [·] α⁻ = α⁻ ]]$
            holds by substitution properties;
            \item the unique determination of $[[ [ausr]α̂⁻]]$ for $[[α̂⁻]] \in [[Ξ' | uv uM']] = [[∅]]$ holds vacuously.
        \end{itemize}

        \item \ruleref{\ottdruleAUShiftULabel}, i.e. 
        $[[iN1]] = [[↑iP1]]$ and $[[iN2]] = [[↑iP2]]$.
        \label{case:anti-unification-initial:shiftu}

        Then since $[[ [aus'i] uM' = iNi ]] = [[↑iPi]]$ and $[[uM']]$ is not a algorithmic variable,
        $[[uM']] = [[↑uQ']]$, where $[[ [aus'i] uQ' = iPi ]]$. 
        Let us show that $[[(Ξ', uQ', aus'1, aus'2)]]$ 
        is an anti-unifier of $[[iP1]]$ and $[[iP2]]$.
        \begin{enumerate}
            \item $[[Γ ; Ξ' ⊢ uQ']]$ holds by inversion of $[[Γ ; Ξ' ⊢ ↑uQ']]$;
            \item $[[Γ ; · ⊢ aus'i : Ξ']]$ holds by assumption;
            \item $[[ [aus'i] uQ' = iPi ]]$ holds by assumption.
        \end{enumerate}

        This way, by the completeness of anti-unification 
        (\cref{lemma:au-completeness}),
        the anti-unification algorithm succeeds on $[[iP1]]$ and $[[iP2]]$:
        $[[Γ ⊨ iP1 ≈au iP2 ⫤ (Ξ, uQ, aus1, aus2)]]$,
        which means that \ruleref{\ottdruleAUShiftULabel} is applicable to infer 
        $[[Γ ⊨ ↑iP1 ≈au ↑iP2 ⫤ (Ξ, ↑uQ, aus1, aus2)]]$.

        Moreover, by the induction hypothesis,
        $[[(Ξ, uQ, aus1, aus2)]]$ is more specific than $[[(Ξ', uQ', aus'1, aus'2)]]$,
        which immediately implies that $[[(Ξ, ↑uQ, aus1, aus2)]]$ is more specific than
        $[[(Ξ', ↑uQ', aus'1, aus'2)]]$ (we keep the same $[[ausr]]$).

        \item \ruleref{\ottdruleAUForallLabel}, i.e. 
        $[[iN1]] = [[∀pas.iN'1]]$ and $[[iN2]] = [[∀pas.iN'2]]$.
        \label{case:anti-unification-initial:forall}
        The proof is symmetric to the previous case.
        Notice that the context $[[Γ]]$ is not changed in \ruleref{\ottdruleAUForallLabel}, 
        as it represents the context in which the anti-unification variables must be instantiated,
        rather than the context forming the types that are being anti-unified.

        \item \ruleref{\ottdruleAUArrowLabel}, i.e.
        $[[iN1]] = [[iP1 → iN'1]]$ and $[[iN2]] = [[iP2 → iN'2]]$.

        Then since $[[ [aus'i] uM' = iNi ]] = [[iPi → iN'i]]$ and $[[uM']]$ is not a algorithmic variable,
        $[[uM']] = [[uQ' → uM'']]$, where $[[ [aus'i] uQ' = iPi ]]$ and $[[ [aus'i] uM'' = iN''i ]]$.

        Let us show that $[[(Ξ', uQ', aus'1, aus'2)]]$
        is an anti-unifier of $[[iP1]]$ and $[[iP2]]$.
        \begin{enumerate}
            \item $[[Γ ; Ξ' ⊢ uQ']]$ holds by inversion of $[[Γ ; Ξ' ⊢ uQ' → uM'']]$;
            \item $[[Γ ; · ⊢ aus'i : Ξ']]$ holds by assumption;
            \item $[[ [aus'i] uQ' = iPi ]]$ holds by assumption.
        \end{enumerate}

        Similarly, $[[(Ξ', uM'', aus'1, aus'2)]]$ is an anti-unifier of $[[iN''1]]$ and $[[iN''2]]$.

        Then by the completeness of anti-unification (\cref{lemma:au-completeness}),
        the anti-unification algorithm succeeds on $[[iP1]]$ and $[[iP2]]$:
        $[[Γ ⊨ iP1 ≈au iP2 ⫤ (Ξ1, uQ, aus1, aus2)]]$;
        and on $[[iN'1]]$ and $[[iN'2]]$:
        $[[Γ ⊨ iN''1 ≈au iN''2 ⫤ (Ξ2, uM''', aus3, aus4)]]$.
        Notice that $[[aus1 & aus3]]$ and $[[aus2 & aus4]]$ are defined, 
        in other words, for any $[[β̂⁻]] \in [[Ξ1]] \cap [[Ξ2]]$,
        $[[ [aus1] β̂⁻ = [aus2] β̂⁻ ]]$ and $[[ [aus3] β̂⁻ = [aus4] β̂⁻ ]]$,
        which follows immediately from \cref{obs:names-defined-by-mapping}.
        This way, the algorithm proceeds by applying \ruleref{\ottdruleAUArrowLabel} and returns
        $[[(Ξ1 ∪ Ξ2, uQ → uM''', aus1 ∪ aus3, aus2 ∪ aus4)]]$.

        It is left to construct $[[ausr]]$ such that $[[Γ ; Ξ ⊢ ausr : (Ξ' | uv uM')]]$ and $[[ [ausr] uM' = uM ]]$.
        By the induction hypothesis, there exist $[[ausr1]]$ and $[[ausr2]]$ such that
        $[[Γ ; Ξ1 ⊢ ausr1 : (Ξ' | uv uQ')]]$, 
        $[[Γ ; Ξ2 ⊢ ausr2 : (Ξ' | uv uM'')]]$,
        $[[ [ausr1] uQ' = uQ ]]$, and $[[ [ausr2] uM'' = uM''' ]]$.

        Let us show that $[[ausr]] = [[ausr1 ∪ ausr2]]$ satisfies the required properties:
        \begin{itemize}
            \item $[[Γ ; Ξ1 ∪ Ξ2 ⊢ ausr1 ∪ ausr2 : (Ξ' | uv uM')]]$ holds since 
            $[[Ξ' | uv uM']] = [[Ξ' | uv uQ' → uM'']] = [[(Ξ' | uv uQ') ∪ (Ξ' | uv uM'')]]$,
            $[[Γ ; Ξ1 ⊢ ausr1 : (Ξ' | uv uQ')]]$ and $[[Γ ; Ξ2 ⊢ ausr2 : (Ξ' | uv uM'')]]$;
            \item $[[ [ausr] uM' ]] = [[ [ausr] (uQ' → uM'') ]] = [[ [ausr | uv uQ'] uQ' → [ausr | uv uM''] uM'' ]] =
            [[ [ausr1] uQ' → [ausr2] uM'' ]] = [[ uQ → uM''']] = [[ uM ]]$;
            \item Since $[[ [ausr]β̂⁻]]$ is either equal to  $[[ [ausr1]β̂⁻]]$ or $[[ [ausr2]β̂⁻]]$,
            it inherits their property that it is uniquely determined by $[[ [aus'1]β̂⁻]]$, $[[ [aus'2]β̂⁻]]$, and $[[Γ]]$.
        \end{itemize}

        \item $[[iP1]] = [[iP2]] = [[a⁺]]$. This case is symmetric to \cref{case:anti-unification-initial:nvar}.
        \item $[[iP1]] = [[↓iN1]]$ and $[[iP2]] = [[↓iN2]]$. This case is symmetric to \cref{case:anti-unification-initial:shiftu}
        \item $[[iP1]] = [[∃nas.iP'1]]$ and $[[iP2]] = [[∃nas.iP'2]]$. This case is symmetric to \cref{case:anti-unification-initial:forall}
        \end{caseof}
\end{proof}

\subsection{Upper Bounds}
\label{sec:alg-upper-bounds-proofs}
\begin{lemma}[Decomposition of the quantifier rule]
  \ilyam{move somewhere}
  \label{lemma:qant-rule-decomposition}
  Whenever the quantifier rule (\ruleref{\ottdruleDOneExistsLabel} or
  \ruleref{\ottdruleDOneForallLabel}) is applied, one can assume that the rule
  adding quantifiers on the right-hand side was applied the last.

  \begin{itemize}
  \item[$-$]
    If $[[G ⊢ iN ≤ ∀pbs.iM]]$ then $[[G, pbs ⊢ iN ≤ iM]]$.
    \item[$+$]
      If $[[G ⊢ iP ≥ ∃nbs.iQ]]$ then $[[G, nbs ⊢ iP ≥ iQ]]$.
  \end{itemize}
\end{lemma}

\begin{lemma}[Shape of the Supertypes]
  \label{lemma:shape-of-supertypes}
  Let us define the
  set of upper bounds of a positive type $\UB([[iP]])$ in the following way:

  \hfill

  \begin{tabular}{@{}lr@{}} \toprule
    % supertypes of ... & ... are \\ 
    $[[G ⊢ iP]]$          & $\UB([[G ⊢ iP]])$ \\ \midrule
    \addlinespace[0.7em]
    $[[ G ⊢ pb ]]$        & $\{[[ ∃nas.pb ]] \ \mid \ \text{for }[[nas]]\}$ \\
    \addlinespace[0.7em]
    $[[ G ⊢ ∃nbs.iQ ]]$   & %$\Set{ [[iQ]] \ | \begin{array}{l} [[iQ]] \in \UB([[iP]]) \\  \text{ s.t. } [[fv iQ ∩ {nbs} = ∅]] \end{array}}$  \\
                            $\UB([[G, nbs ⊢ iQ]])$ not using $[[nbs]]$ \\
    \addlinespace[0.7em]
    $[[ G ⊢ ↓M ]]$        & $\Set{ [[ ∃nas.↓iM' ]] \ | \begin{array}{l}
                                                         \text{for $[[nas]]$, $[[iM']]$, and $[[iNs]]$ s.t. }\\
                                                         \text{$[[G ⊢ iNi]]$, $[[G,nas ⊢ iM']]$,  and $[[ [iNs/nas] ↓iM' ≈ ↓iM ]]$}
                                                       \end{array}}$  \\
  \end{tabular}

  Then $\UB([[G ⊢ iP]]) \equiv \{[[iQ]]\ \mid \ [[G ⊢ iQ ≥ iP]] \}$.
\end{lemma}

\begin{proof}
  By induction on $[[G ⊢ iP]]$.
  \begin{caseof}
  \item $[[iP]] = [[pb]]$\\
    Then the last rule that is applied to infer $[[G ⊢ iQ
    ≥ pb]]$ must be either \ruleref{\ottdruleDOnePVarLabel} or
    \ruleref{\ottdruleDOneExistsLabel}. The former case means that $[[iQ =
    pb]]$. In the latter case, $[[iQ = ∃nas.iQ']]$, where $[[iQ']]$ has no outer
    existential quantifiers. Then by inversion of
    \ruleref{\ottdruleDOneExistsLabel}, $[[ Γ ⊢ [iNs/nas] iQ' ≥ pb]]$ for some $[[iNs]]$.
    This time, to infer this judgment, only \ruleref{\ottdruleDOnePVarLabel} is applicable,
    which means that $[[iQ' = pb]]$, and then $[[iQ = ∃nas.pb]]$.
  \item $[[iP = ∃nbs.iP']]$\\
    Then if $[[G ⊢ iQ ≥ ∃nbs.iP']]$, then by
    \cref{lemma:qant-rule-decomposition}, $[[G, nbs ⊢ iQ ≥ iP']]$, and $[[fv
    iQ ∩ {nbs} = ∅]]$ by the the Barendregt's convention. The other
    direction holds by \ruleref{\ottdruleDOneExistsLabel}. This way,
    $\{[[iQ]] \mid [[G ⊢ iQ ≥ ∃nbs.iP']] \} = \{[[iQ]] \mid  [[G, nbs ⊢ iQ
    ≥ iP']] \text{ s.t. } [[fv(iQ) ∩ {nbs} = ∅]] \}$. From the induction
    hypothesis, the latter is equal to $\UB([[G, nbs ⊢ iP']])$ not using
    $[[nbs]]$, i.e. $\UB([[G ⊢ ∃nbs.iP']])$.
  \item $[[iP = ↓iM]]$\\
    Then let us consider two subcases upper bounds without outer quantifiers (we
    denote the corresponding set restriction as $|_{\not\exists}$) and upper
    bounds with outer quantifiers ($|_{\exists}$). We prove that for both of
    these groups, the restricted sets are equal.
    % ∃a.P(f a) <=> ∃b∊Im(f).P(b)

    \begin{caseof}
      \item \label{case:sup-shape-down-zero}
      $[[iQ]] \neq [[∃nbs.iQ']]$\\
      Then the last applied rule to infer
      $[[G ⊢ iQ ≥ ↓iM]]$ must be \ruleref{\ottdruleDOneShiftDLabel},
      which means $[[iQ]] = [[↓iM']]$, and by inversion, $[[G ⊢ iM' ≈ iM]]$,
      then by \cref{lemma:equiv-completeness} and
      \ruleref{\ottdruleEOneShiftDLabel}, $[[↓iM' ≈ ↓iM]]$.
      This way, $[[iQ]] = [[↓iM']] \in \{ [[↓iM']] \mid [[↓iM' ≈ ↓iM]] \} = \UB([[Γ⊢↓iM]])|_{\not\exists}$.

      In the other direction,
      $
      \begin{aligned}[t]
        [[↓iM' ≈ ↓iM]] &\Rightarrow [[G ⊢ ↓iM' ≈ ↓iM]]
                       && \text{by \cref{lemma:equiv-soundness}, since
                          $[[G ⊢ ↓iM']]$ by \cref{lemma:wf-equiv} }\\
                       &\Rightarrow [[G ⊢ ↓iM' ≥ ↓iM]]
                       && \text{by inversion}
      \end{aligned}
      $
      \item $[[iQ]] = [[∃nbs.iQ']]$ (for non-empty $[[nbs]]$)\\
        Then the last rule applied to infer $[[G ⊢ ∃nbs.iQ' ≥ ↓iM]]$
        must be \ruleref{\ottdruleDOneExistsLabel}.
        Inversion of this rule gives us $[[G ⊢ [iNs/nbs]iQ' ≥ ↓iM]]$
        for some $[[G ⊢ iNi]]$. Notice that $[[ [iNs/nbs]iQ' ]]$ has no outer
        quantifiers. Thus from \cref{case:sup-shape-down-zero},
        $[[ [iNs/nbs]iQ' ≈ ↓iM ]]$, which is only possible if $[[iQ']] = [[↓iM']]$.
        This way, $[[iQ]] = [[∃nbs.↓iM']] \in \UB([[Γ⊢↓iM]])|_{\exists}$ (notice
        that $[[nbs]]$ is not empty).

        In the other direction,
        $
        \begin{aligned}[t]
          [[ [iNs/nbs]↓iM' ≈ ↓iM]] &\Rightarrow [[G ⊢ [iNs/nbs] ↓iM' ≈ ↓iM]]
          && \text{by \cref{lemma:equiv-soundness}, since
             $[[G ⊢ [iNs/nbs] ↓iM']]$ by \cref{lemma:wf-equiv} }\\
                                  &\Rightarrow [[G ⊢ [iNs/nbs]↓iM' ≥ ↓iM]]
         && \text{by inversion}\\
                                  &\Rightarrow [[G ⊢ ∃nbs.↓iM' ≥ ↓iM]] 
         && \text{by \ruleref{\ottdruleDOneExistsLabel}}\\
        \end{aligned}
        $
    \end{caseof}
    
  \end{caseof}
\end{proof}


\begin{lemma}[Normalized Shape of the Supertypes]
  For a normalized positive type $[[iP = nf(iP)]]$,
  let us define the set of normalized upper bounds in the following way:
  
  \hfill

  \begin{tabular}{@{}lr@{}} \toprule
    % supertypes of ... & ... are \\ 
    $[[G ⊢ iP]]$          & $\NFUB([[G ⊢ iP]])$ \\ \midrule
    \addlinespace[0.7em]
    $[[ G ⊢ pb ]]$        & $\{ [[pb]] \}$ \\
    \addlinespace[0.7em]
    $[[ G ⊢ ∃nbs.iP ]]$   & %$\Set{ [[iQ]] \ | \begin{array}{l} [[iQ]] \in \UB([[iP]]) \\  \text{ s.t. } [[fv iQ ∩ {nbs} = ∅]] \end{array}}$  \\
    $\NFUB([[G, nbs ⊢ iP]])$ not using $[[nbs]]$ \\
    \addlinespace[0.7em]
    $[[ G ⊢ ↓M ]]$        & $\Set{ [[ ∃nas.↓iM' ]] \ | \begin{array}{l}
                                                         \text{for $[[nas]]$, $[[iM']]$, and $[[iNs]]$ s.t. $[[ord {nas} in iM' = nas]]$,}\\
                                                         \text{$[[G ⊢ iNi]]$, $[[G,nas ⊢ iM']]$,  and $[[ [iNs/nas] ↓iM' = ↓iM ]]$}
                                                       \end{array}}$  \\
  \end{tabular}

  Then $\NFUB([[G ⊢ iP]]) \equiv \{[[nf(iQ)]]\ \mid \ [[G ⊢ iQ ≥ iP]] \}$.
\end{lemma}


\begin{proof}
  By induction on $[[G ⊢ iP]]$.
  \begin{caseof}
  \item $[[iP]] = [[pb]]$\\
    Then from \cref{lemma:shape-of-supertypes},
    $\{[[nf(iQ)]]\ \mid \ [[G ⊢ iQ ≥ pb]] \} = \{[[ nf(∃nas.pb) ]] \ \mid \
    \text{for some }[[nas]]\}  = \{[[pb]]\}$ 
  \item $[[iP = ∃nbs.iP']]$\\
    $
    \begin{aligned}[t]
      \NFUB([[Γ ⊢ ∃nbs.iP']]) &= \NFUB([[Γ, nbs ⊢ iP']]) \text{ not using $[[nbs]]$}\\
                              &= \{ [[nf(iQ)]] \mid [[Γ, nbs ⊢ iQ ≥ iP']]  \}
                                \text{ not using $[[nbs]]$}
                              && \text{by the induction hypothesis}\\
                              &= \{ [[nf(iQ)]] \mid [[Γ, nbs ⊢ iQ ≥ iP']]
                                \text{ s.t. $[[fv iQ]] \cap [[nbs]] = \emptyset$}
                                \}
                             && \text{because $[[fv nf(iQ)]] = [[fv iQ]]$ by \cref{lemma:fv-nf}}\\
                              &= \{ [[nf(iQ)]] \mid [[iQ]] \in \UB([[Γ, nbs ⊢ iP']]) \text{ s.t. $[[fv iQ]] \cap [[nbs]] = \emptyset$}
                                \}
                            && \text{by \cref{lemma:shape-of-supertypes}}\\
                              &= \{ [[nf(iQ)]] \mid [[iQ]] \in \UB([[Γ ⊢ ∃nbs.iP']])
                                \}
                              && \text{by the definition of $\UB{}$}\\
                              &= \{ [[nf(iQ)]] \mid [[Γ ⊢ iQ ≥ ∃nbs.iP']]
                                \}
                              && \text{by \cref{lemma:shape-of-supertypes}}\\
    \end{aligned}
    $
  
  \item $[[iP = ↓iM]]$\\
    
In the following reasoning, we will use the following principle of variable
replacement.
\begin{observation}
  \label{observation:idemp-replacement}
  Suppose that $\nu : A \rightarrow A$ is an idempotent
  function, $P$ is a predicate on $A$, $F : A \rightarrow B$ is a
  function. Then
 \[ 
  \begin{aligned}[t]
    &\{ F(\nu x ) \mid x \in A \text{ s.t. } P(\nu x) \} =\\
    = &\{ F(x) \mid x \in A \text{ s.t. } \nu x = x \text{ and } P(x) \}.
  \end{aligned}
 \]
\end{observation}
In our case, the idempotent $\nu$ will be normalization, variable ordering, or
domain restriction.

Another observation we will use is the following.
\begin{observation}
  \label{observation:image-replacement}
  For functions $F$ and $\nu$, and
  predicates $P$ and $Q$,
 \[ 
  \begin{aligned}[t]
    &\{F(\nu x) \mid x \in A \text{ s.t. } Q(\nu x) \text{ and } P(x) \} =\\
    = &\{F(\nu x) \mid x \in A \text{ s.t. } Q(\nu x) \text{ and } (\exists x'
        \in A \text{ s.t. } P(x') \text{ and } \nu x' = \nu x) \}.
  \end{aligned}
 \]
\end{observation}
    
    $\begin{aligned}[t]
      &\mathrel{\phantom{=}} \{ [[nf(iQ)]] \mid [[Γ ⊢ iQ ≥ ↓iM]] \}=\\
      %
      %
      &=\{ [[nf(iQ)]] \mid [[iQ]] \in \UB([[ Γ ⊢ ↓iM]])\}
      && \text{by \cref{lemma:shape-of-supertypes}}\\
      %
      %
      &= \Set{[[ nf(∃nas.↓iM') ]] \ |
      \begin{array}{l}
        \text{for $[[nas]]$, $[[iM']]$, and $[[iNs]]$ s.t.  $[[G,nas ⊢ iM']]$,}\\
        \text{$[[G ⊢ iNi]]$, and $[[ [iNs/nas] ↓iM' ≈ ↓iM ]]$}\\
      \end{array}}
      && \text{by the definition of $\UB$}\\
      %
      %
      &= \Set{[[ nf(∃nas.↓iM') ]] \ |
        \begin{array}{l}
          \text{for $[[nas]]$, $[[iM']]$, and $[[σ]]$ s.t.  $[[G,nas ⊢ iM']]$,}\\
          \text{$[[G ⊢ σ : nas]]$, and $[[ [σ] ↓iM' ≈ ↓iM ]]$}
        \end{array}}
      && \text{we reassigned the substitution $[[ iNs/nas ]]$ as $[[ σ ]]$}\\
      %
      %
      &= \Set{[[ nf(∃nas.↓iM') ]] \ |
        \begin{array}{l}
          \text{for $[[nas]]$, $[[iM']]$, and $[[σ]]$ s.t. $[[G,nas ⊢ iM']]$,}\\
          \text{$[[G ⊢ σ : nas]]$, and $[[ [σ|fv iM' ] ↓iM' ≈ ↓iM ]]$}
        \end{array}}
      && \text{by \cref{lemma:subst-restr-fv}}\\
      % 
      % 
      &= \Set{[[ ∃nas'.nf(↓iM') ]] \ |
      \begin{array}{l}
        \text{for $[[nas']]$, $[[nas]]$, $[[iM']]$, $[[σ]]$ s.t. $[[G,nas ⊢ iM']]$,}\\
        \text{$[[G ⊢ σ : nas]]$,\, $[[ord {nas} in iM' = nas']]$}\\
        \text{and $[[ [σ|fv iM'] ↓iM' ≈ ↓iM ]]$}
      \end{array}}
      && \text{by the definition of normalization}\\
      %
      %
      &= \Set{[[ ∃nas'.nf(↓iM') ]] \ |
      \begin{array}{l}
        \text{for $[[nas']]$, $[[nas]]$, $[[iM']]$, $[[σ]]$ s.t. $[[G,nas ⊢ iM']]$, }\\
        \text{$[[G ⊢ σ : nas]]$,\, $[[ord {nas} in iM' = nas']]$}\\
        \text{and $[[ nf([σ|fv iM'] ↓iM') = nf(↓iM) ]]$}
      \end{array}}
      &&\begin{array}{l}
         \text{from
          \cref{lemma:normalization-soundness,lemma:normalization-completeness},
          equivalence of types can be}\\
          \text{replaced with the equality of their normal forms}
        \end{array} \\
      %
      %
      &= \Set{[[ ∃nas'.nf(↓iM') ]] \ |
        \begin{array}{l}
          \text{for $[[nas']]$, $[[nas]]$, $[[iM']]$, $[[σ]]$ s.t. $[[G,nas ⊢ iM']]$,}\\
          \text{$[[G ⊢ σ : nas]]$,\, $[[ord {nas} in iM' = nas']]$}\\
          \text{and $[[ [nf(σ|fv iM')] ↓nf(iM') = ↓nf(iM) ]]$}
        \end{array}}
      && \text{by congruence of normalization and \cref{lemma:norm-subst-distr}}\\
      %
      %
      &= \Set{[[ ∃nas'.↓iM' ]] \ |
        \begin{array}{l}
          \text{for $[[nas']]$, $[[nas]]$, $[[iM']]$, $[[σ]]$ s.t. $[[G,nas ⊢ iM']]$,}\\
          \text{$[[G ⊢ σ : nas]]$,\, $[[ord {nas} in iM' = nas']]$}\\
          \text{and $[[ [σ|fv iM'] ↓iM' = ↓iM ]]$}
        \end{array}}
      &&\begin{array}{l}
         \text{by \cref{lemma:normal-after-subst}, $[[↓iM']]$ and $[[σ|fv
          iM']]$ are already normal,}\\
          \text{since the result of the substitution is normal;}\\
          \text{$[[iM]]$ is normal by assumption}
         \end{array}\\
      % 
      % 
      &= \Set{[[ ∃nas'.↓iM' ]] \ |
        \begin{array}{l}
          \text{for $[[nas']]$, $[[nas]]$, $[[iM']]$, $[[σ]]$ s.t.  $[[G,nas ⊢ iM']]$,}\\
          \text{($\exists [[σ']]$ s.t. $[[G ⊢ σ' : nas]]$ and $[[σ|fv(↓iM')]] =
          [[σ'|fv(↓iM')]]$)}\\
          \text{$[[ord {nas} in iM' = nas']]$ and $[[ [σ|fv iM'] ↓iM' = ↓iM ]]$}\\
        \end{array}}
      &&\begin{array}{l}
      \text{We apply \cref{observation:image-replacement} (with $\nu [[σ]]$
      =  $[[σ|fv iM']]$, and}\\
      \text{$P([[σ]]) = [[G ⊢ σ : nas]]$)}
      \end{array}\\
      %
      %
      &= \Set{[[ ∃nas'.↓iM' ]] \ |
        \begin{array}{l}
          \text{for $[[nas']]$, $[[nas]]$, $[[iM']]$, $[[σ]]$ s.t.  $[[G,nas ⊢ iM']]$,}\\
          \text{$[[G ⊢ σ|fv iM' : nas']]$,\, $[[ord {nas} in iM' = nas']]$}\\
          \text{and $[[ [σ|fv iM'] ↓iM' = ↓iM ]]$}
        \end{array}}
      &&\begin{array}{l}
           \text{Notice that}\\
           \text{``$\exists [[σ']]$ s.t. ($[[G ⊢ σ' : nas]]$ and $[[σ|fv(↓iM')]] =
           [[σ'|fv(↓iM')]]$)''}\\
           \text{is equivalent to $[[G ⊢ σ|fv(↓iM') : nas']]$}\\
         \end{array} \\
         % 
         % 
      &= \Set{[[ ∃nas'.↓iM' ]] \ |
        \begin{array}{l}
          \text{for $[[nas']]$, $[[nas]]$, $[[iM']]$, $[[σ]]$ s.t.  $[[G,nas ⊢ iM']]$,}\\
          \text{$[[G ⊢ σ : nas']]$,\, $[[ord {nas} in iM' = nas']]$}\\
          \text{and $[[ [σ] ↓iM' = ↓iM ]]$}
        \end{array}}
      && \begin{array}{l}
         \text{We apply \cref{observation:idemp-replacement} to the
           restriction of $[[σ]]$, and then}\\
         \text{remove $[[σ|fv iM']] = [[σ]]$  as it follows from $[[G ⊢ σ : nas']]$}\\
         \end{array}\\
         % 
         % 
      &= \Set{[[ ∃nas'.↓iM' ]] \ |
        \begin{array}{l}
          \text{for $[[nas]]$, $[[nas']]$, $[[iM']]$, $[[σ]]$ s.t.  $[[G,nas' ⊢ iM']]$,}\\
          \text{$[[G ⊢ σ : nas']]$,\, $[[ord {nas} in iM' = nas']]$}\\
          \text{and $[[ [σ] ↓iM' = ↓iM ]]$}
        \end{array}}
      &&
         \begin{array}{l}
           \text{From \cref{lemma:wf-ctxt-equiv}, since
           $[[ {Γ, nas} ∩ fv iM' ]] = [[ {Γ, nas'} ∩ fv iM' ]]$}\\
         \end{array}\\
      % 
      % 
      &= \Set{[[ ∃nas'.↓iM' ]] \ |
        \begin{array}{l}
          \text{for $[[nas']]$,  $[[iM']]$, $[[σ]]$ s.t.  $[[G,nas' ⊢ iM']]$,}\\
          \text{$[[G ⊢ σ : nas']]$,\, $[[ord {nas'} in iM' = nas']]$}\\
          \text{and $[[ [σ] ↓iM' = ↓iM ]]$}
        \end{array}}
      && \text{We apply \cref{observation:idemp-replacement} to the ordering of $[[nas]]$}\\
      &= \Set{ [[ ∃nas.↓iM' ]] \ |
        \begin{array}{l}
          \text{for $[[nas']]$, $[[iM']]$, and $[[iNs]]$ s.t. $[[ord {nas'} in iM' = nas]]$,}\\
          \text{$[[G ⊢ iNi]]$, $[[G,nas' ⊢ iM']]$,  and $[[ [iNs/nas'] ↓iM' = ↓iM ]]$}
        \end{array}} 
      && \text{By reassigning $[[σ]]$ explicitly as $[[iNs/nas']]$}\\
      %
      %
    \end{aligned}$
  \end{caseof}
\end{proof}

\begin{lemma}[Soundness of the Least Upper Bound]
  For types $[[Γ ⊢ iP1]]$, and $[[Γ ⊢ iP2]]$,
  if $[[Γ ⊨ iP1 ∨ iP2 = iQ]]$ then
  \begin{enumerate}
    \item[(i)]  $[[Γ ⊢ iQ]]$
    \item[(ii)] $[[Γ ⊢ iQ ≥ iP1]]$ and $[[Γ ⊢ iQ ≥ iP2]]$
  \end{enumerate}
\end{lemma}

\begin{lemma}[Completeness of the Least Upper Bound]
  For types $[[Γ ⊢ iP1]]$, $[[Γ ⊢ iP2]]$, and $[[Γ ⊢ iQ']]$
  such that $[[Γ ⊢ iQ' ≥ iP1]]$ and $[[Γ ⊢ iQ' ≥ iP2]]$,
  there exists $[[iQ]]$ s.t. $[[Γ ⊨ iP1 ∨ iP2 = iQ]]$, and
  $[[Γ ⊢ iQ' ≥ iQ]]$
\end{lemma}

\begin{lemma}[Soundness of Upgrade]
  For $[[Δ]] \subseteq [[Γ]]$,
  suppose that $[[upgrade Γ ⊢ iP to Δ = iQ]]$.
  Then
  \begin{enumerate}
    \item [(i)] $[[Δ ⊢ iQ]]$
    \item [(ii)] $[[Γ ⊢ iQ ≥ iP]]$
  \end{enumerate}
\end{lemma}

\begin{lemma}[Completeness of Upgrade]
  For $[[Δ]] \subseteq [[Γ]]$,
  $[[Γ ⊢ iP]]$ and $[[Δ ⊢ iQ']]$,
  such that $[[Γ ⊢ iQ' ≥ iP]]$,
  there exists $[[iQ]]$ s.t.
  $[[upgrade Γ ⊢ iP to Δ = iQ]]$, and
  $[[Δ ⊢ iQ' ≥ iQ]]$.
\end{lemma}



\subsection{Upgrade}
\label{sec:upgrade-lemmas}
Let us consider a type $[[iP]]$ well-formed in $[[Γ]]$.
Some of its $[[Γ]]$-supertypes are also well-formed in a smaller context $[[{Δ} ⊆ Γ]]$.
The upgrade is the operation that returns the least of such supertypes.

\begin{observation}[Upgrade is deterministic]
    \label{obs:upgrade-deterministic}
    Assuming $[[iP]]$ is well-formed in $[[Γ ⊆ Δ]],$\\
    if $[[upgrade Γ ⊢ iP to Δ = iQ]]$ and $[[upgrade Γ ⊢ iP to Δ = iQ']]$ are defined 
    then $[[iQ = iQ']]$.
\end{observation}
\begin{proof}
    It follows directly from \cref{obs:lub-deterministic},
    and the convention that the fresh variables are chosen by a fixed deterministic algorithm
    (\cref{sec:fresh-selection}).
\end{proof}

\begin{lemma}[Soundness of Upgrade]\label{lemma:upgrade-soundness}
    Assuming $[[iP]]$ is well-formed in $[[Γ = Δ, pnas]]$,
    if $[[upgrade Γ ⊢ iP to Δ = iQ]]$
    then
    \begin{enumerate}
        \item $[[Δ ⊢ iQ]]$
        \item $[[Γ ⊢ iQ ≥ iP]]$
    \end{enumerate}
\end{lemma}
\begin{proof}
    By inversion, $[[upgrade Γ ⊢ iP to Δ = iQ]]$ means that 
    for fresh $[[pnbs]]$ and $[[pncs]]$,
    $[[Δ, pnbs, pncs ⊨ [pnbs/pnas]iP ∨ [pncs/pnas]iP = iQ]]$.
    Then by the soundness of the least upper bound (\cref{lemma:lub-soundness}),
    \begin{enumerate}
        \item $[[Δ, pnbs, pncs ⊢ iQ]]$, 
        \item $[[Δ, pnbs, pncs ⊢ iQ ≥ [pnbs/pnas]iP]]$, and 
        \item $[[Δ, pnbs, pncs ⊢ iQ ≥ [pncs/pnas]iP]]$.
    \end{enumerate}

    $ 
    \begin{aligned}
        [[fv iQ]] &\subseteq [[fv [pnbs/pnas]iP ∩ fv [pncs/pnas]iP]]
                  &&\text{since by \cref{lemma:fv-propagation}, 
                         $[[fv iQ ⊆ fv [pnbs/pnas]iP]]$ and
                         $[[fv iQ ⊆ fv [pncs/pnas]iP]]$}\\
                  &\subseteq [[ ((fv iP \ {pnas}) ∪ {pnbs}) ∩ ((fv iP \ {pnas}) ∪ {pncs})]]\\
                  &= [[ (fv iP \ {pnas}) ∩ (fv iP \ {pnas}) ]]
                  &&\text{since $[[pnbs]]$ and $[[pncs]]$ are fresh}\\
                  &= [[ fv iP \ {pnas} ]]\\
                  &\subseteq [[ Γ \ {pnas} ]]
                  &&\text{since $[[iP]]$ is well-formed in $[[Γ]]$}\\
                  &\subseteq [[ {Δ} ]]\\
    \end{aligned}
    $\\
    This way, by \cref{lemma:wf-ctxt-equiv}, $[[Δ ⊢ iQ]]$.
    
    Let us apply $[[pnas/pnbs]]$---the inverse of the substitution $[[ pnbs/pnas ]]$ to 
    both sides of $[[Δ, pnbs, pncs ⊢ iQ ≥ [pnbs/pnas]iP]]$ and 
    by \cref{lemma:subst-pres-subt} 
    (since $[[pnbs/pnas]]$ can be specified as 
    $[[Δ,pnbs,pncs ⊢ pnbs/pnas : Δ, pnas, pncs]]$ by \cref{lemma:subst-domain-weakening})
    obtain $[[Δ, pnas, pncs ⊢ [pnas/pnbs]iQ ≥ iP]]$.
    Notice that $[[Δ ⊢ iQ]]$ implies that $[[fv iQ ∩ {pnbs} = ∅]]$, 
    then by \cref{corollary:subst-disj}, $[[ [pnas/pnbs]iQ = iQ]]$,
     and thus $[[Δ, pnas, pncs ⊢ iQ ≥ iP]]$.
    By context strengthening, $[[Δ, pnas ⊢ iQ ≥ iP]]$.
\end{proof}

\begin{lemma}[Completeness and Initiality of Upgrade] \label{lemma:upgrade-completeness}
    The upgrade returns the least $[[Γ]]$-supertype of $[[iP]]$ well-formed in $[[Δ]]$.
    Assuming $[[iP]]$ is well-formed in $[[Γ = Δ, pnas]],$\\
    For any $[[iQ']]$ such that 
    \begin{enumerate}
        \item $[[Δ ⊢ iQ']]$ and
        \item $[[Γ ⊢ iQ' ≥ iP]]$,
    \end{enumerate}

    The result of the upgrade algorithm $[[iQ]]$ exists
    ($[[upgrade Γ ⊢ iP to Δ = iQ]]$) and satisfies $[[Δ ⊢ iQ' ≥ iQ]]$.
\end{lemma}
\begin{proof}

    Let us consider fresh (not intersecting with $[[Γ]]$) $[[pnbs]]$ and $[[pncs]]$.

    If we apply substitution $[[pnbs/pnas]]$ to both sides of $[[Δ, pnas ⊢ iQ' ≥ iP]]$,
    we have $[[Δ, pnbs ⊢ [pnbs/pnas]iQ' ≥ [pnbs/pnas]iP]]$, which by  
    \cref{corollary:subst-disj}, since $[[pnas]]$ is disjoint from $[[fv(iQ')]]$
    (because $[[Δ ⊢ iQ']]$), simplifies to $[[Δ, pnbs ⊢ iQ' ≥ [pnbs/pnas]iP]]$.

    Analogously, if we apply substitution $[[pncs/pnas]]$ to both sides of $[[Δ, pnas ⊢ iQ' ≥ iP]]$,
    we have $[[Δ, pncs ⊢ iQ' ≥ [pncs/pnas]iP]]$.

    This way, $[[iQ']]$ is a common supertype of $[[ [pnbs/pnas]iP ]]$ and $[[ [pncs/pnas]iP ]]$ in
    context $[[Δ, pnbs, pncs]]$. It means that we can apply the completeness of the least upper bound
    (\cref{lemma:lub-completeness}):
    \begin{enumerate}
        \item there exists $[[iQ]]$ s.t. $[[Γ ⊨ [pnbs/pnas]iP ∨ [pncs/pnas]iP = iQ]]$ 
        \item $[[Γ ⊢ iQ' ≥ iQ]]$.
    \end{enumerate}
    The former means that the upgrade algorithm terminates and returns $[[iQ]]$.
    The latter means that since both 
    $[[iQ']]$ and $[[iQ]]$ are well-formed in $[[Δ]]$ and $[[Γ]]$,
    by \cref{lemma:subt-ctxt-irrelevance}, $[[Δ ⊢ iQ' ≥ iQ]]$.
\end{proof}




\subsection{Constraint Satisfaction}
\begin{lemma}
    \label{lemma:constraint-sat}
    Suppose that $[[Θ ⊢ SC]]$ then there exist
    s $[[uσ]]$ such that $[[Θ ⊢ uσ : SC]]$.
\end{lemma}
\begin{proof}
    Let us define $[[uσ]]$
    on $[[dom(SC)]]$ in the following way: 
    $$
    [[ [uσ]α̂± ]] = 
    \begin{cases}
        [[iP]] & \text{if $[[(α̂± :≈ iP)]] \in [[SC]]$} \\
        [[iP]] & \text{if $[[(α̂± :≥ iP)]] \in [[SC]]$} \\
        [[iN]] & \text{if $[[(α̂± :≈ iN)]] \in [[SC]]$} \\
    \end{cases}
    $$
    Then $[[Θ ⊢ uσ : SC]]$ follows immediately
    from the reflexivity of equivalence and subtyping
    (\cref{lemma:subtyping-reflexivity}) and the corresponding
    rules 
    \ruleref{\ottdruleSATSCEPEqLabel}, 
    \ruleref{\ottdruleSATSCENEqLabel},
    and \ruleref{\ottdruleSATSCESupLabel}.
\end{proof}

\begin{lemma}
    Suppose $[[Θ ⊢ SC]]$,
    and $[[SC singular]]$. 
    Then there exist a unique (up-to equivalence)
    $[[Θ ⊢ uσ : SC]]$.
\end{lemma}
\begin{proof}

\end{proof}

\subsection{Positive Subtyping}
\begin{lemma}[Soundness of the Positive Subtyping] \label{lemma:pos-subt-soundness}
    If $[[Γ ⊢ Θ]]$, $[[Γ ⊢ iQ]]$, and $[[Γ ; Θ ⊢ uP]]$ then\\
    $[[Γ ; Θ ⊨ uP ≥ iQ ⫤ us]]$
    implies $[[us : Θ|uv uP]]$ and 
    for any $[[us']]$ such that $[[Θ ⊢ us' ⇒ us]]$,
    $[[ Γ ⊢ [us']uP ≥ iQ ]]$.
\end{lemma}
\begin{proof} 
    We prove it by induction on $[[Γ ; Θ ⊨ uP ≥ iQ ⫤ us]]$. Let us consider the last rule to infer this judgment.
    \begin{caseof}
    \item \ruleref{\ottdruleAPUVarLabel} then
    $[[Γ ; Θ ⊨ uP ≥ iQ ⫤ us]]$ has shape $[[Γ;Θ ⊨ â⁺ ≥ iP' ⫤ (â⁺ :≥ iQ')]]$ where
    $[[â⁺[Δ] ∊ Θ]]$ and $[[upgrade G ⊢ iP' to Δ = iQ']]$.

    Notice that $[[â⁺[Δ] ∊ Θ]]$ and $[[Γ ⊢ Θ]]$ 
    implies $[[Γ = Δ, pnas]]$ for some $[[pnas]]$, hence, the
    soundness of upgrade (\cref{lemma:upgrade-soundness}) is applicable:
    \begin{enumerate}
        \item $[[Δ ⊢ iQ']]$ and
        \item $[[Γ ⊢ iQ ≥ iP]]$.
    \end{enumerate}

    Since $[[â⁺[Δ] ∊ Θ|uv â⁺]]$ and $[[Δ ⊢ iQ']]$, it is clear that $[[(â⁺ :≥ iQ') : Θ | uv â⁺ ]]$.

    It is left to show that $[[Γ ⊢ [us']â⁺ ≥ iP']]$ for any $[[us']]$ s.t. $[[Θ ⊢ us' ⇒ (â⁺ :≥ iQ')]]$.
    The latter weakening statement means that either $[[us']] \ni [[â⁺ :≥ iQ'']]$ or
    $[[us']] \ni [[(â⁺ :≈ iQ'')]]$ for $[[Δ ⊢ iQ'' ≥ iQ']]$. In any case,
    $[[Δ ⊢ [us']â⁺ ≥ iQ]]$. By weakening the context to $[[Γ]]$ and combining this judgment
    transitively (\cref{todo}) with $[[Γ ⊢ iQ ≥ iP]]$, we have $[[Γ ⊢ [us']â⁺ ≥ iP]]$,
    as required. 

    \item \label{case:pos-subt-soundness:var} \ruleref{\ottdruleAPVarLabel}  
    then $[[Γ ; Θ ⊨ uP ≥ iQ ⫤ us]]$ has shape $[[Γ;Θ ⊨ a⁺ ≥ a⁺ ⫤ ·]]$ .
    Then $[[uv a⁺ = ∅]]$, and $[[us]] = [[· : ·]]$ satisfies $[[us : Θ|∅]]$.
    Since $[[uv a⁺ = ∅]]$, application of any unification solution $[[us']]$ does not change $[[a⁺]]$, i.e.
    $[[ [us'] a⁺ = a⁺]]$. Therefore, $[[Γ ⊢ [us']a⁺ ≥ a⁺]]$ holds by \ruleref{\ottdruleDOneNVarLabel}.

    \item \label{case:pos-subt-soundness:shift} 
     \ruleref{\ottdruleAShiftDLabel} then
    $[[Γ ; Θ ⊨ uP ≥ iQ ⫤ us]]$ has shape $[[Γ;Θ ⊨ ↓uN ≥ ↓iM ⫤ us]]$.\\
    Then the next step of the algorithm is the unification of $[[nf(uN)]]$ and $[[nf(iM)]]$.
    By the soundness of the unification algorithm (\cref{lemma:unification-soundness}),
    it returns an equivalence-only solution $[[us]]$ such that $[[us : Ord uv uN]]$.
    By \cref{todo}, since $[[us]]$ is equivalence-only and $[[Γ ⊢ Θ]]$, $[[Θ ⊢ us' ⇒ us]]$ means 
    $[[ Γ ⊢ Sub us' ≈ Sub us : Ord uv uN ]]$ as substitutions.
    $[[ [us]nf(uN) = nf(iM) ]]$ implies $[[Γ ⊢ [us]nf(uN) ≈ nf(iM)]]$, and then 
        $[[Γ ⊢ [us']nf(uN) ≈ nf(iM)]]$. \ilyam{add lemmas}

        Let us rewrite the left-hand side and the right-hand side of $[[Γ ⊢ [us']nf(uN) ≈ nf(iM)]]$ by 
        transitivity of equivalence (\cref{corollary:equivalence-transitivity}).
        By \cref{corollary:nf-sound-wrt-subt-equiv,corollary:subst-pres-equiv},
        $[[Γ ⊢ [us']nf(uN) ≈ [us']uN ]]$. By \cref{corollary:nf-sound-wrt-subt-equiv}, 
        $[[Γ ⊢ nf(iM) ≈ iM ]]$. 
        This way, we have $[[Γ ⊢ [us']uN ≈ iM]]$.
        Then by \ruleref{\ottdruleDOneShiftULabel}
        and congruence of substitution, $[[Γ ⊢ [us']↓uN ≥ ↓iM]]$.
    
    \item \label{case:pos-subt-soundness:exists}
       \ruleref{\ottdruleAExistsLabel} then
        $[[Γ ; Θ ⊨ uP ≥ iQ ⫤ us]]$ has shape $[[Γ;Θ ⊨ ∃nas.uP' ≥ ∃nbs.iQ' ⫤ us]]$ s.t. either $[[nas]]$ or $[[nbs]]$ is not empty.\\
        Then the algorithm creates fresh unification variables $[[â⁻*[Γ,nbs] ]]$, 
        substitutes the old $[[nas]]$ with them in $[[uP']]$, and makes the recursive call:
        $[[G, nbs; Θ, â⁻*[G, nbs] ⊨ [â⁻*/nas] uP ≥ iQ ⫤ us']]$, returning as the result
        $[[us]] = [[us' \ {α̂⁻*}]]$.

        Notice that $[[Γ, nbs ⊢ Θ, â⁻*[Γ, nbs] ]]$, $[[Γ,nbs ⊢ iQ']]$, and 
        $[[Γ,nbs; Θ, â⁻*[Γ, nbs] ⊢ [â⁻*/nas] uP' ]]$, so the induction hypothesis is applicable,
        that is $[[us' : Θ, â⁻*[Γ, nbs] | uv [â⁻*/nas]uP']]$ and $[[ Γ, nbs ⊢ [us'2][â⁻*/nas]uP' ≥ iQ' ]]$ for any
        $[[us'2]]$ s.t. $[[Θ, â⁻*[Γ, nbs] ⊢ us'2 ⇒ us']]$.

        Since the domain of $[[us']]$ is $[[uv [â⁻*/nas]uP']]$, the domain of 
        $[[us]] = [[us' \  {α̂⁻*}]]$ is $[[uv [â⁻*/nas]uP' \ {â⁻*}]] = [[uv ∃nas.uP']]$,
        this way, $[[us : Θ | uv ∃nas.uP']]$, as required.

        It is left to show that $[[Γ ⊢ [us2]∃nas.uP' ≥ ∃nbs.iQ']]$ for any $[[us2]]$ s.t. $[[Θ ⊢ us2 ⇒ us]]$.
        Let us consider an arbitrary such $[[us2]]$. Let us construct $[[us'2]]$, 
        extending $[[us2]]$ to the domain $[[uv [â⁻*/nas]uP']]$ with the values of $[[us']]$,
        i.e.  $[[us'2 : Θ | uv [â⁻*/nas]uP']]$, 
        and
        \[
            [[us'2(β̂±)]]  = 
            \begin{cases}
               [[us2(β̂±)]] & \text{if } [[β̂±]] \in [[uv uP']] \\
               [[us'(β̂±)]] & \text{if } [[β̂±]] \in [[â⁻*]]
            \end{cases}
        \] 
        , where the application of the unification solution to a variable returns the 
        corresponding \emph{unification entry}. 
        Notice that $[[Sub us'2|uv uP']] = [[us2]]$.
    It is easy to see that $[[Θ ⊢ us'2 ⇒ us']]$: 
    \begin{enumerate}
        \item if $[[β̂±]] \in [[uv uP']]$ then $[[us'2(β̂±)]] = [[us2(β̂±)]] \Rightarrow [[us(β̂±)]] = [[us'(β̂±)]]$;
        \item it $[[β̂±]] \in [[â⁻*]]$ then $[[us'2(β̂±)]] = [[us'(β̂±)]] \Rightarrow [[us'(β̂±)]]$,
    \end{enumerate}
    which means that the induction hypothesis can be applied to $[[us'2]]$, i.e.
    $[[ Γ, nbs ⊢ [us'2][â⁻*/nas]uP' ≥ iQ' ]]$.\\
    Notice that
    $
    \begin{aligned}[t]
                 [[ [us'2][â⁻*/nas]uP' ]] &= [[ [Sub us'2|{â⁻*} ○ â⁻*/nas][Sub us'2|uv uP']uP' ]]
                                          && \text{by substitution properties \ilyam{todo}}\\
                                          &= [[ [Sub us'2|{â⁻*} ○ â⁻*/nas][us2]uP' ]]
                                          && \text{since $[[Sub us'2|uv uP']] = [[us2]]$}.
    \end{aligned}
    $\\
    Also notice that the domain of $[[Sub us'2|{â⁻*} ○ â⁻*/nas]]$ is $[[nas]]$,
    and the range is $[[Γ, nbs]]$, i.e. $[[Γ, nbs ⊢ Sub us'2|{â⁻*} ○ â⁻*/nas : nas]]$, 
    which means that we can apply \ruleref{\ottdruleDOneForallLabel} to 
    $[[ Γ, nbs ⊢ [Sub us'2|{â⁻*} ○ â⁻*/nas][us2]uN' ≤ iM' ]]$
    to obtain $[[ Γ ⊢ [us2]∃nas.uP' ≥ ∃nbs.iQ' ]]$, as required.
    \end{caseof}
\end{proof}

\begin{lemma}[Completeness of the Positive Subtyping] \label{lemma:pos-subt-completeness}
    Suppose that $[[Γ ⊢ Θ]]$, $[[Γ ⊢ iQ]]$ and $[[Γ ; Θ ⊢ uP]]$ and
    there exists $[[us : Θ | uv uP]]$ such that $[[ Γ ⊢ [us]uP ≥ iQ ]]$.
    Then there exists $[[us']]$ such that $[[Γ ; Θ ⊨ uP ≥ iQ ⫤ us']]$
    and $[[Θ ⊢ us ⇒ us']]$.
\end{lemma}
\begin{proof}
    We prove it by induction on $[[ Γ ⊢ [us]uP ≥ iQ ]]$.
    Let us consider the last rule used in the derivation of  $[[ Γ ⊢ [us]uP ≥ iQ ]]$,
    but first consider the base case for the substitution $[[ [us]uP ]]$:

    \begin{caseof}
        \item \label{case:pos-subt-complete-base} $[[uP]] = [[ ∃nbs.α̂⁺ ]]$ (for potentially empty $[[pbs]]$)\\
        Then by assumption, there exists $[[us : Θ | uv uP]]$ such that $[[ Γ ⊢ ∃nbs.[us]α̂⁺ ≥ iQ ]]$.
        $[[us : Θ | uv α̂⁺]]$ means that  $[[us]]$ is either $[[ (α̂⁺ :≈ iP') ]]$ or $[[ (α̂⁺ :≥ iP') ]]$,
        where $[[Δ ⊢ iP']]$ and $[[α̂⁺[Δ] ∊ Θ]]$.
        $[[ Γ ⊢ ∃nbs.[us]α̂⁺ ≥ iQ ]]$ means that $[[Γ ⊢ iP' ≥ iQ]]$
        because multiple inversions of \ruleref{\ottdruleDOneExistsLabel} 
        gives us $[[Γ ⊢ [us]α̂⁺ ≥ iQ]]$ since $[[ {nbs} ∩ fv [us]α̂⁺ = ∅]]$.


        In the algorithm, after multiple applications of \ruleref{\ottdruleAExistsLabel}
        the type $[[∃nbs.α̂⁺]]$ is reduced to $[[α̂⁺]]$.
        Next, the algorithm tries to apply
        \ruleref{\ottdruleAPUVarLabel}
        and the resulting solution is $[[us']] = [[(α̂⁺ :≥ iQ')]]$ where
        $[[upgrade Γ ⊢ iQ to Δ = iQ']]$.

        Why does the upgrade procedure terminates?
        Because $[[iP']]$ satisfies the pre-conditions of the completeness of the upgrade
        (\cref{lemma:upgrade-completeness})
        :
        \begin{itemize}
            \item $[[Δ ⊢ iP']]$ because $[[iP' = [us]α̂⁺]]$, $[[us : Θ | uv uP]]$ and 
            $[[α̂⁺[Δ] ∊ Θ | uv uP]]$,
            \item $[[Γ ⊢ iP' ≥ iQ]]$ as noted before
        \end{itemize}
        Moreover, completeness of the upgrade also gives us $[[Γ ⊢ iP' ≥ iQ']]$
        and further, we strengthen it to $[[Δ ⊢ iP' ≥ iQ']]$
        (since by the soundness of the upgrade (\cref{lemma:upgrade-soundness}),
        $[[Δ ⊢ iQ']]$).

        It means that $[[Θ ⊢ (α̂⁺ :≈ iP') ⇒ (α̂⁺ :≥ iQ') ]]$ and 
        $[[ Θ ⊢ (α̂⁺ :≥ iP') ⇒ (α̂⁺ :≥ iQ') ]]$, which means that in any case, 
        $[[ Θ ⊢ us ⇒ us']]$.

        \item \label{case:pos-subt-complete-pvar}
        $[[ Γ ⊢ [us]uP ≥ iQ ]]$ is derived by \ruleref{\ottdruleDOnePVarLabel}, 
        i.e. $[[iP]] = [[ [us]uP ]] = [[ α⁺ ]] = [[iQ]]$.
        Here the first equality holds because $[[uP]]$ is not a unification variable:
        this case has been covered by \cref{case:pos-subt-complete-base}.
        The second equality hold because \ruleref{\ottdruleDOnePVarLabel} was applied.

        Notice that $[[us : Θ | uv α⁺]]$ = $[[us : Θ | ∅]]$ = $[[·]]$.

        The algorithm applies \ruleref{\ottdruleAPVarLabel} and 
        infers $[[us']] = [[·]]$, i.e. $[[Γ;Θ ⊨ a⁺ ≥ a⁺ ⫤ ·]]$.

        Since $[[Θ ⊢ · ⇒ ·]]$, we have $[[Θ ⊢ us ⇒ us']]$.


        \item \label{case:pos-subt-complete-upshift} $[[ Γ ⊢ [us]uP ≥ iQ ]]$ is derived by \ruleref{\ottdruleDOneShiftDLabel},
        
        Then $[[ uP ]] = [[ ↓uN ]]$, since the substitution $[[ [us]uP ]]$ must preserve the 
        top-level constructor of $[[uP]]\neq [[α̂⁺]]$ (the case $[[uP]] = [[α̂⁺]]$ has been covered
        by \cref{case:pos-subt-complete-base}), and $[[uQ]] = [[ ↓iM ]]$,
        and by inversion, $[[ Γ ⊢ [us]uN ≈ iM ]]$.

        Since both types start with $[[↓]]$, 
        the algorithm tries to apply \ruleref{\ottdruleAShiftDLabel}: 
        $[[G;Θ ⊨ ↓uN ≥ ↓iM ⫤ us']]$. The premise of this rule is the
        unification of $[[nf(uN)]]$ and $[[nf(iM)]]$:
        $[[Γ;Θ ⊨ nf(uN) ≈u nf(iM) ⫤ us']]$. Let us show that the unification successfully 
        terminates and returns $[[us']] = [[nf(us)]]$.

        To demonstrate that the unification terminates, we apply the completeness 
        of the unification algorithm (\cref{lemma:unification-completeness}). 
        In order to do that, we need to provide a unifier of 
        $[[nf(uN)]]$ and $[[nf(iM)]]$. Thankfully, $[[nf(us)]]$ does it. 

        \begin{itemize}
            \item $[[nf(uN)]]$ and $[[nf(iM)]]$ are normalized 
            \item $[[Γ ; Θ ⊢ nf(uN)]]$ because $[[Γ ; Θ ⊢ uN]]$ (\cref{todo})
            \item $[[Γ ⊢ nf(iM)]]$ because $[[Γ ⊢ iM]]$ (\cref{corollary:wf-nf})
            \item $[[nf(us) : Θ | uv nf(uN)]]$ because $[[us: Θ | uv uN]]$ (\cref{todo})
            \item $ \begin{aligned}[t]
                    [[ Γ ⊢ [us]uN ≈ iM ]] &\Rightarrow [[ [us]uN ≈ iM ]]
                                          && \text {by \cref{lemma:equiv-completeness}}\\
                                          &\Rightarrow [[ nf([us]uN) = nf(iM) ]]
                                          && \text {by \cref{lemma:normalization-completeness}}\\
                                          &\Rightarrow [[ [nf(us)]nf(uN) = nf(iM) ]]
                                          && \text {by \cref{lemma:norm-subst-distr}}\\
                    \end{aligned}
                  $
        \end{itemize}
        Then by the completeness of the unification,
        $[[Γ ; Θ ⊨ nf(uN) ≈u nf(iM) ⫤ nf(us)]]$.
        This way, the subtyping algorithm terminates and the resulting solution is
        $[[nf(us)]]$. 
        
        It is left to note that $[[Θ ⊢ us ⇒ nf(us)]]$, by \cref{todo}, since $[[Θ ⊢ us ≈ nf(us)]]$ 


      \item $[[ Γ ⊢ [us]uP ≥ iQ ]]$ is derived by \ruleref{\ottdruleDOneExistsLabel}.\\
      We should only consider the case
      when the substitution $[[ [us]uP ]]$ results in the existential type 
      $[[∃nas.iP'']]$ (for $[[iP'']] \neq [[∃]]\dots$) by congruence, 
      i.e. $[[uP = ∃nas.uP']]$ (for $[[uP']] \neq [[∃]]\dots$) and $[[ [us]uP' = iP'' ]]$.
      This is because the case when $[[uP = ∃nbs.α̂⁺]]$ has been covered
      (\cref{case:pos-subt-complete-base}), and thus, the substitution $[[us]]$ must
      preserve all the outer quantifiers of $[[uP]]$ and does not generate any new ones.

      This way, $[[uP]] = [[∃nas.uP']]$, $[[ [us]uP ]] = [[ ∃nas.[us]uP' ]]$ 
      (assuming $[[nas]]$ does not intersect with the range of $[[us]]$)
      and $[[iQ]] = [[ ∃nbs.iQ' ]]$, where either $[[nas]]$ or $[[nbs]]$ is not empty.

      By inversion, $[[ Γ ⊢ [σ][us]uP' ≥ iQ' ]]$ for some $[[Γ, nbs ⊢ σ : nas]]$.
      Since $[[σ]]$ and $[[us]]$ have disjoint domains,
      and the range of one does not intersect with the domain of the other,
      they commute, i.e. $[[ Γ, nbs ⊢ [us][σ]uP' ≥ iQ' ]]$
      (notice that the tree inferring this judgement is 
      a proper subtree of the tree inferring 
      $[[ Γ ⊢ [us]uP ≥ iQ ]]$).

      At the next step, 
      the algorithm creates fresh (disjoint with $[[uv uP']]$) 
      unification variables $[[â⁻*]]$, replaces $[[nas]]$ with them in $[[ uP' ]]$,
      and makes the recursive call:
      $[[G, nbs; Θ, â⁻*[G, nbs] ⊨ uP0 ≥ iQ ⫤ us1]]$,
      (where $[[uP0]] = [[ [â⁻*/nas]uP' ]]$),
      returning $[[us1 \ {â⁻*}]]$ as the result.

       Notice that $[[ [us][σ][nas/â⁻*]uP0 = [us][σ]uP' ]]$,
       and thus, $[[ Γ, nbs ⊢ [us][σ][nas/â⁻*]uP0 ≥ iQ' ]]$.
       Let us combine $[[us]]$ and $[[σ  ○ nas/â⁻*]]$ into one 
       $[[us0]]$---a unification solution for $[[uP0]]$:
        \[
            [[us0(β̂±)]]  = 
            \begin{cases}
               [[ β̂± :≈ [σ]αi⁻ ]] & \text{if } [[β̂±]] = [[αî⁻]] \in [[â⁻*]] \\
               [[us(β̂±)]] & \text{if } [[β̂±]] \in [[uv uP']]
            \end{cases}
       \]

       Let us show that the induction hypothesis is applicable to 
       $[[Γ, nbs ⊢ [us0]uP0 ≥ iQ' ]]$, with the meta-context $[[ Θ, â⁻*[Γ, nbs] ]]$
       \begin{itemize}
        \item $[[ us0 : (Θ, â⁻*[Γ, nbs])|uv uP0 ]]$. Notice that for every $[[αî⁻]] \in [[â⁻*]]$, 
            the type corresponding to the entry $[[us0(αî⁻)]]$ 
            is well-formed in $[[Γ, nbs]]$ since $[[Γ, nbs ⊢ σ : nas ]]$;
            and for every $[[β̂±]] \in [[uv uP']]$, 
            the type corresponding to the entry $[[us0(β̂±)]]$ is well-formed in
            context $[[Θ(β̂±)]]$ since $[[us : Θ | uv uP']]$.
        \item $[[Γ, nbs ⊢ [us0]uP0 ≥ iQ' ]]$. This is because
            $[[ [us0]uP0 ]] = [[ [us][σ][nas/â⁻*]uP0 ]] = [[ [us][σ]uP' ]]$.
        \item $[[ Γ, nbs ⊢ Θ, â⁻*[Γ, nbs] ]]$ since
            $[[Γ, nbs ⊢ Θ]]$ and $[[ {Γ, nbs} ⊆ {Γ, nbs} ]]$.
        \item $[[Γ, nbs ; Θ, â⁻*[Γ, nbs] ⊢ uP0 ]]$
       \end{itemize}

       This way, we apply the induction hypothesis to $[[Γ, nbs ⊢ [us0]uP0 ≥ iQ' ]]$ and 
       infer that there exists\\
        $[[us0' :  (Θ, â⁻*[Γ, nbs]) | uv uP0]]$ such that
       $[[G, nbs; Θ, â⁻*[G, nbs] ⊨ uP0 ≥ iQ ⫤ us0']]$,
       and $[[Θ, â⁻*[G, nbs] ⊢ us0' ⇒ us0]]$.

       This way, the algorithm terminates with the result $[[us0' \ {â⁻*}]]$
       and it is left to show $[[ Θ ⊢ us ⇒ us0' \ {â⁻*} ]]$. Notice that
       $[[us0' \ {â⁻*} : Θ | uv uP' ]]$. To show $[[ Θ ⊢ us ⇒ us0' \
       {â⁻*} ]]$, it suffices to consider a unification variable $[[β̂±]] \in 
       [[uv uP']]$ and show that $[[Γ ⊢ us(β̂±) ⇒ us0'(β̂±)]]$. It holds by the
       reflexivity of weakening (\cref{lemma:entry-weakening-preorder}) since
       $[[us0'(β̂±)]] = [[us(β̂±)]]$.

      \item The positive case when $[[Γ ⊢ [us]uP ≥ iQ]]$ is derived by 
      \ruleref{\ottdruleDOnePVarLabel} is symmetric to the corresponding negative
      \cref{case:subt-complete-nvar}.
      Notice that $[[ [us]uP = β⁺ ]]$ 
      implies $[[uP = β⁺]]$ because the case when $[[uP = α̂⁺]]$ has been covered 
      (\cref{case:subt-complete-base}).

    \end{caseof}
\end{proof}


\subsection{Subtyping Constraint Merge}
\obsEntryMergeDeterministic*
\begin{proof}
    First, notice that the shape of $[[scE1]]$ and $[[scE2]]$
    uniquely determines the rule applied to infer  
    $[[Γ ⊢ scE1 & scE2 = scE]]$,
    which is consequently, the same rule used to 
    infer $[[Γ ⊢ scE1 & scE2 = scE']]$.
    Second, notice that the premises of each rule are deterministic
    on the input:
    the positive subtyping is deterministic by \cref{obs:pos-subt-deterministic},
    and the least upper bound is deterministic by \cref{obs:lub-deterministic}.
\end{proof}

\obsSubtMergeDeterministic*
\begin{proof}
    The proof is analogous to the proof of \cref{obs:unif-merge-deterministic} 
    but uses \cref{obs:entry-merge-deterministic} to show 
    that the merge of the matching constraint entries is fixed.
\end{proof}


\lemEntryMergeSoundness*
\begin{proof}
    Let us consider the rule forming $[[Γ ⊢ scE1 & scE2 = scE]]$.
    \begin{caseof}
        \item \ruleref{\ottdruleSCMEPEqEqLabel}, i.e. 
            $[[Γ ⊢ scE1 & scE2 = scE]]$
            has form $[[Γ ⊢ (pua :≈ iQ) & (pua :≈ iQ') = (pua :≈ iQ)]]$
            and $[[nf(iQ) = nf(iQ')]]$. The latter implies $[[Γ ⊢ iQ ≈ iQ']]$ by
            \cref{lemma:subt-equiv-algorithmization}.
            Then
            \begin{enumerate}
                \item $[[Γ ⊢ scE]]$, i.e. $[[Γ ⊢ pua :≈ iQ]]$ holds by assumption;
                \item by inversion, $[[Γ ⊢ iP : (pua :≈ iQ)]]$ means $[[Γ ⊢ iP ≈ iQ]]$,
                and by transitivity of equivalence (\cref{corollary:equivalence-transitivity}), 
                $[[Γ ⊢ iP ≈ iQ']]$. Thus, $[[Γ ⊢ iP : scE1]]$ and $[[Γ ⊢ iP : scE2]]$ hold
                by \ruleref{\ottdruleSATSCEPEqLabel}.
            \end{enumerate}
        \item \ruleref{\ottdruleSCMENEqEqLabel} the negative case is proved in exactly the same way as the positive one.
        \item \ruleref{\ottdruleSCMESupSupLabel} 
            Then $[[scE1]]$ is $[[pua :≥ iQ1]]$, $[[scE2]]$ is $[[pua :≥ iQ2]]$,
            and $[[scE1 & scE2]] = [[scE]]$ is $[[pua :≥ iQ]]$ where $[[iQ]]$ is the least upper bound of $[[iQ1]]$ and $[[iQ2]]$.
            Then by \cref{lemma:lub-soundness},
            \begin{itemize}
                \item $[[Γ ⊢ iQ]]$,
                \item $[[Γ ⊢ iQ ≥ iQ1]]$,
                \item $[[Γ ⊢ iQ ≥ iQ2]]$.
            \end{itemize}

            Let us show the required properties.
            \begin{itemize}
                \item $[[Γ ⊢ scE]]$ holds from $[[Γ ⊢ iQ]]$,
                \item Assuming $[[Γ ⊢ iP : scE]]$, by inversion, we have $[[Γ ⊢ iP ≥ iQ]]$.
                    Combining it transitively with $[[Γ ⊢ iQ ≥ iQ1]]$, we have $[[Γ ⊢ iP ≥ iQ1]]$.
                    Analogously, $[[Γ ⊢ iP ≥ iQ2]]$.
                    Then $[[Γ ⊢ iP : scE1]]$ and $[[Γ ⊢ iP : scE2]]$ hold by \ruleref{\ottdruleSATSCESupLabel}.
            \end{itemize}

        \item \ruleref{\ottdruleSCMESupEqLabel}
            Then $[[scE1]]$ is $[[pua :≥ iQ1]]$, $[[scE2]]$ is $[[pua :≈ iQ2]]$, 
            where $[[Γ;· ⊨ uQ2 ≥ iQ1 ⫤ ·]]$, and the resulting   
            $[[scE1 & scE2]] = [[scE]]$ is equal to $[[scE2]]$, that is $[[pua :≈ iQ2]]$.
    
            Let us show the required properties.
            \begin{itemize}
                \item By assumption, $[[Γ ⊢ iQ]]$, and hence $[[Γ ⊢ scE]]$.
                \item Since $[[uv uQ2 = ∅]]$, 
                    $[[Γ;· ⊨ uQ2 ≥ iQ1 ⫤ ·]]$ implies $[[Γ ⊢ iQ2 ≥ iQ1]]$
                    by the soundness of positive subtyping (\cref{lemma:pos-subt-soundness}).
                    Then let us take an arbitrary $[[Γ ⊢ iP]]$ such that $[[Γ ⊢ iP : scE]]$.
                    Since $[[scE2]] = [[scE]]$, $[[Γ ⊢ iP : scE2]]$ holds immediately.
                    
                    By inversion, $[[Γ ⊢ iP : (pua :≈ iQ2)]]$ means $[[Γ ⊢ iP ≈ iQ2]]$, 
                    and then by transitivity of subtyping (\cref{lemma:subtyping-transitivity}),
                    $[[Γ⊢ iP ≥ iQ1]]$.  Then $[[Γ ⊢ iP : scE1]]$ holds by \ruleref{\ottdruleSATSCESupLabel}.
            \end{itemize}
        \item \ruleref{\ottdruleSCMEEqSupLabel} Thee proof is analogous to the previous case.
    \end{caseof}
\end{proof}

\lemMergeSoundness*
\begin{proof}
    By definition, $[[Θ ⊢ SC1 & SC2 = SC]]$ consists of three parts:
    entries of $[[SC1]]$ that do not have matching entries of $[[SC2]]$,
    entries of $[[SC2]]$ that do not have matching entries of $[[SC1]]$,
    and the merge of matching entries.

    Notice that $[[α̂± ∊ Ξ1 \ Ξ2]]$
    if and only if there is an entry $[[scE]]$ in $[[SC1]]$ 
    restricting $[[α̂±]]$, but there is no such entry in $[[SC2]]$.
    Therefore, for any $[[α̂± ∊ Ξ1 \ Ξ2]]$,
    there is an entry $[[scE]]$ in $[[SC]]$ restricting $[[α̂±]]$.
    Notice that $[[Θ(α̂±) ⊢ scE]]$ holds since $[[Θ ⊢ SC1 : Ξ1]]$.

    Analogously, for any $[[β̂± ∊ Ξ2 \ Ξ1]]$,
    there is an entry $[[scE]]$ in $[[SC]]$ restricting $[[β̂±]]$.
    Notice that  $[[Θ(β̂±) ⊢ scE]]$ holds since $[[Θ ⊢ SC2 : Ξ2]]$.

    Finally, for any $[[γ̂± ∊ Ξ1 ∩ Ξ2]]$,
    there is an entry $[[scE1]]$ in $[[SC1]]$ restricting $[[γ̂±]]$
    and an entry $[[scE2]]$ in $[[SC2]]$ restricting $[[γ̂±]]$.
    Since $[[Θ ⊢ SC1 & SC2 = SC]]$ is defined,
    $[[Θ(γ̂±) ⊢ scE1 & scE2 = scE]]$ restricting $[[γ̂±]]$ is
    defined and belongs to $[[SC]]$,
    moreover, $[[Θ(γ̂±) ⊢ scE]]$ by \cref{lemma:entry-merge-soundness}.
    This way, $[[Θ ⊢ SC : Ξ1 ∪ Ξ2]]$.

    Let us show the second property.
    We take an arbitrary $[[uσ]]$ such that $[[Θ ⊢ uσ : Ξ1 ∪ Ξ2]]$ 
    and $[[ Θ ⊢ uσ : SC ]]$.
    To prove $[[ Θ ⊢ uσ : SC1 ]]$, 
    we need to show that for any $[[scE1]] \in [[SC1]]$, 
    restricting $[[α̂±]]$, $[[Θ(α̂±) ⊢ [uσ]α̂± : scE1]]$ holds.

    Let us assume that $[[α̂±]] \notin [[dom(SC2)]]$. It means that $[[SC]] \ni [[scE1]]$, 
    and then since $[[ Θ ⊢ uσ : SC ]]$, $[[Θ(α̂±) ⊢ [uσ]α̂± : scE1]]$. 

    Otherwise, $[[SC2]]$ contains an entry $[[scE2]]$ restricting $[[α̂±]]$,
    and $[[SC]] \ni [[scE]]$ where $[[Θ(α̂±) ⊢ scE1 & scE2 = scE]]$.
    Then since $[[ Θ ⊢ uσ : SC ]]$, $[[Θ(α̂±) ⊢ [uσ]α̂± : scE]]$,
    and by \cref{lemma:entry-merge-soundness}, $[[Θ(α̂±) ⊢ [uσ]α̂± : scE1]]$.

    The proof of $[[ Θ ⊢ uσ : SC2 ]]$ is symmetric.
\end{proof}


\lemEntryMergeCompleteness*
\begin{proof}
    Let us consider the shape of $[[scE1]]$ and $[[scE2]]$.
    \begin{caseof}
        \item $[[scE1]]$ is $[[pua :≈ iQ1]]$ and $[[scE2]]$ is $[[pua :≈ iQ2]]$.
            The proof repeats the corresponding case of \cref{lemma:unif-entry-merge-completeness}
        \item $[[scE1]]$ is $[[pua :≈ iQ1]]$ and $[[scE2]]$ is $[[pua :≥ iQ2]]$.
            Then $[[Γ ⊢ iP : scE1]]$ means $[[Γ ⊢ iP ≈ iQ1]]$, 
            and $[[Γ ⊢ iP : scE2]]$ means $[[Γ ⊢ iP ≥ iQ2]]$.
            Then by transitivity of subtyping, $[[Γ ⊢ iQ1 ≥ iQ2]]$,
            which means $[[Γ ; · ⊨ uQ1 ≥ iQ2 ⫤ ·]]$ by \cref{lemma:pos-subt-completeness}.
            This way, \ruleref{\ottdruleSCMEEqSupLabel} applies to infer
            $[[Γ ⊢ scE1 & scE2 = scE1]]$, and $[[Γ ⊢ iP : scE1]]$ holds by assumption.
        \item $[[scE1]]$ is $[[pua :≥ iQ1]]$ and $[[scE2]]$ is $[[pua :≥ iQ2]]$.
            Then $[[Γ ⊢ iP : scE1]]$ means $[[Γ ⊢ iP ≥ iQ1]]$, 
            and $[[Γ ⊢ iP : scE2]]$ means $[[Γ ⊢ iP ≥ iQ2]]$.
            By the completeness of the least upper bound (\cref{lemma:lub-completeness}), 
            $[[Γ ⊨ iQ1 ∨ iQ2 = iQ]]$, and $[[Γ ⊢ iP ≥ iQ]]$. 
            This way, \ruleref{\ottdruleSCMESupSupLabel} applies to infer
            $[[Γ ⊢ scE1 & scE2 = (pua :≥ iQ)]]$, 
            and $[[Γ ⊢ iP : (pua :≥ iQ)]]$ holds by \ruleref{\ottdruleSATSCESupLabel}.
        \item The negative cases are proved symmetrically.
    \end{caseof}
\end{proof}

\lemMergeCompleteness*
\begin{proof}
    By  definition, $[[SC1 & SC2]]$ is a union of
    \begin{enumerate}
        \item entries of $[[SC1]]$, which do not have matching entries in $[[SC2]]$,
        \item entries of $[[SC2]]$, which do not have matching entries in $[[SC1]]$, and 
        \item the merge of matching entries.
    \end{enumerate}

    This way, to show that $[[Θ ⊢ SC1 & SC2 = SC]]$ is defined, we need to demonstrate that 
    each of these components is defined and satisfies 
    the required property 
    (that the result of $[[uσ]]$ satisfies the corresponding constraint entry).

    It is clear that the first two components of this union exist. 
    Moreover, if $[[scE]]$ is an entry of $[[SCi]]$
    restricting $[[α̂± ∉ dom(SC2)]]$,
    then $[[ Θ ⊢ uσ : SCi ]]$ implies $[[ Θ(α̂±) ⊢ [uσ]α̂± : scE]]$,

    Let us show that the third component exists.  
    Let us take two entries $[[scE1]] \in [[SC1]]$ and $[[scE2]] \in [[SC2]]$ restricting the same variable $[[α̂±]]$.  $[[ Θ   ⊢ uσ : SC1 ]]$ means that $[[Θ(α̂±) ⊢ [uσ]α̂± : scE1]]$ and $[[ Θ   ⊢ uσ : SC2 ]]$ means $[[Θ(α̂±) ⊢ [uσ]α̂± : scE2]]$.
    Then by \cref{lemma:entry-merge-completeness}, $[[Θ(α̂±) ⊢ scE1 & scE2 = scE]]$ is defined and $[[Θ(α̂±) ⊢ [uσ]α̂± : scE]]$.

\end{proof}



\subsection{Negative Subtyping}
\begin{lemma}[Soundness of Negative Subtyping] \label{lemma:neg-subt-soundness}
        If $[[Γ ⊢ Θ]]$, $[[Γ ⊢ iM]]$, and $[[Γ ; Θ ⊢ uN]]$ then\\ 
        $[[Γ ; Θ ⊨ uN ≤ iM ⫤ us]]$
        implies $[[us : Θ|uv uN]]$ and 
        for any $[[us']]$ such that $[[Θ ⊢ us' ⇒ us]]$,
        $[[ Γ ⊢ [us']uN ≤ iM ]]$
\end{lemma}
\begin{proof}
    We prove it by induction on $[[Γ ; Θ ⊨ uN ≤ iM ⫤ us]]$.
    Let us consider the last rule to infer this judgment. 
    \begin{caseof}
        \item \ruleref{\ottdruleAArrowLabel}, and then $[[Γ ; Θ ⊨ uN ≤ iM ⫤ us]]$ has shape
        $[[G;Θ ⊨ uP → uN' ≤ iQ → iM' ⫤ us]]$\\
        On the next step, the the algorithm makes two recursive calls:
        $[[G;Θ ⊨ uP ≥ iQ ⫤ us1]]$ and $[[G;Θ ⊨ uN' ≤ iM' ⫤ us2]]$.
        By the soundness of positive subtyping (\cref{lemma:pos-subt-soundness}) and induction hypothesis, respectively 
        \begin{enumerate}
            \item $[[us1 : Θ | uv uP]]$ and $[[ Γ ⊢ [us1']uP ≥ iQ ]]$ for any $[[us1']]$ s.t. $[[Θ ⊢ us1' ⇒ us1]]$
            \item $[[us2 : Θ | uv uN']]$ and $[[ Γ ⊢ [us2']uN' ≤ iM' ]]$ for any $[[us2']]$ s.t. $[[Θ ⊢ us2' ⇒ us2]]$
        \end{enumerate}

        Then the algorithm merges two unification solutions $[[us1]]$ and $[[us2]]$.
        By \cref{lemma:merge-soundness}, since $[[uv uP ∪ uv uN' = uv (uP → uN')]]$, 
        we have $[[us1 & us2 : Θ | uv (uP → uN')]]$, and also
        $[[Θ ⊢ us1 & us2 ⇒ us1]]$ and $[[Θ ⊢ us1 & us2 ⇒ us2]]$.
        By the transitivity of solution weakening (\cref{lemma:weakening-transitivity}),
         $[[Θ ⊢ us' ⇒ us1 & us2]]$ implies $[[Θ ⊢ us' ⇒ us1]]$ and $[[Θ ⊢ us' ⇒ us2]]$.

% \[\begin{tikzcd}[arrows=Rightarrow]
% 	& {[[us']]} \\
% 	{[[us1]]} & {[[us1 & us2]]} & {[[us2]]}
% 	\arrow[from=2-2, to=2-3]
% 	\arrow[from=2-2, to=2-1]
% 	\arrow[from=1-2, to=2-2]
% \end{tikzcd}\]

        The application of the induction hypothesis gives us 
        $[[Γ ⊢ [us']uP ≥ iQ ]]$ and $[[ Γ ⊢ [us']uN' ≤ iM' ]]$.
        Finally, by \ruleref{\ottdruleDOneArrowLabel}, $[[Γ ⊢ [us'](uP → uN') ≤ iQ → iM']]$.

        \item \ruleref{\ottdruleANVarLabel}, and then $[[Γ ; Θ ⊨ uN ≤ iM ⫤ us]]$ has shape $[[G;Θ ⊨ a⁻ ≤ a⁻ ⫤ ·]]$\\
        This case is symmetric to \cref{case:pos-subt-soundness:var} of \cref{lemma:pos-subt-soundness}.

        \item \ruleref{\ottdruleAShiftULabel}, and then $[[Γ ; Θ ⊨ uN ≤ iM ⫤ us]]$ has shape $[[G;Θ ⊨ ↑uP ≤ ↑iQ ⫤ us]]$\\
        This case is symmetric to \cref{case:pos-subt-soundness:shift} of \cref{lemma:pos-subt-soundness}.

        \item \ruleref{\ottdruleAForallLabel}, and then $[[Γ ; Θ ⊨ uN ≤ iM ⫤ us]]$ has shape $[[G;Θ ⊨ ∀pas.uN' ≤ ∀pbs.iM' ⫤ us]]$ s.t. either $[[pas]]$ or $[[pbs]]$ is not empty\\
        This case is symmetric to \cref{case:pos-subt-soundness:exists} of \cref{lemma:pos-subt-soundness}.

\end{caseof}
\end{proof}

\begin{lemma}[Completeness of the Negative Subtyping]
    Suppose that $[[Γ ⊢ Θ]]$, $[[Γ ⊢ iM]]$, $[[Γ ; Θ ⊢ uN]]$,
    $[[uN]]$ does not contain negative unification variables ($[[α̂⁻]] \notin [[uv uN]]$)
    and there exists $[[us : Θ | uv uN]]$ such that $[[ Γ ⊢ [us]uN ≤ iM ]]$.
    Then there exists $[[us']]$ such that $[[Γ ; Θ ⊨ uN ≤ iM ⫤ us']]$
    and $[[Θ ⊢ us ⇒ us']]$.
\end{lemma}
\begin{proof}
    We prove it by induction on $[[ Γ ⊢ [us]uN ≤ iM ]]$.
    Let us consider the last rule used in the derivation of $[[ Γ ⊢ [us]uN ≤ iM ]]$.
    \begin{caseof}
        \item $[[ Γ ⊢ [us]uN ≤ iM ]]$ is derived by \ruleref{\ottdruleDOneShiftULabel}\\
        Then $[[ uN ]] = [[ ↑uP ]]$, since
        the substitution $[[ [us]uN ]]$ must preserve the 
        top-level constructor of $[[uN]] \neq [[α̂⁻]]$ (since by assumption, $[[α̂⁻]] \notin [[uv uN]]$), 
        and $[[uQ]] = [[ ↓iM ]]$, and by inversion, $[[ Γ ⊢ [us]uN ≈ iM ]]$.
        The rest of the proof is symmetric to \cref{case:pos-subt-complete-upshift} of
         \cref{lemma:pos-subt-completeness}: notice that the algorithm does not make a recursive call, 
        and the difference in the induction statement for the positive and 
        the negative case here does not matter.

        \item $[[ Γ ⊢ [us]uN ≤ iM ]]$ is derived by \ruleref{\ottdruleDOneArrowLabel}, 
        i.e. $[[ [us]uN ]] = [[ [us]uP → [us]uN' ]]$ and $[[iM]] = [[iQ → iM']]$, 
        and by inversion, $[[ Γ ⊢ [us]uP ≥ iQ ]]$ and $[[ Γ ⊢ [us]uN' ≤ iM' ]]$.

        Let us consider restrictions of $[[us]]$ to 
        the set of unification variables in $[[uP]]$ and $[[uN']]$:
        $[[us | uv uP : Θ | uv uP]]$ and $[[us | uv uN' : Θ | uv uN']]$.
        Notice that 
        $[[ [us]uP = [us | uv uP]uP ]]$ and 
        $[[ [us]uN' = [us | uv uN']uN' ]]$.

       Let us apply the induction hypothesis to
       $[[ Γ ⊢ [us | uv uP]uP ≥ iQ ]]$ and
       $[[ Γ ⊢ [us | uv uN']uN' ≤ iM' ]]$ (notice that
       since $[[uv uN' ⊆ uv uN]]$, there are no negative unification variables in $[[uN']]$)
       to obtain $[[us'1]]$ and $[[us'2]]$ such that
       \begin{enumerate}
        \item $[[Γ; Θ ⊨ uP ≥ iQ ⫤ us'1]]$ and $[[Θ ⊢ us | uv uP ⇒ us'1]]$
        \item $[[Γ; Θ ⊨ uP ≥ iQ ⫤ us'2]]$ and $[[Θ ⊢ us | uv uN' ⇒ us'2]]$ 
       \end{enumerate}
       
       This way, the algorithm applies \ruleref{\ottdruleAArrowLabel}
       and terminates returning $[[us'1 & us'2]]$ as the result.

       It is left to show that $[[Θ ⊢ us ⇒ us'1 & us'2]]$.
       Since $[[us | uv uP ⊆ us]]$, 
       by \cref{lemma:weakening-monotonicity}, 
       $[[Θ ⊢ us ⇒ us | uv uP]]$,
       and then by transitivity (\cref{lemma:weakening-transitivity}), 
       $[[Θ ⊢ us ⇒ us'1]]$.
       Analogously, $[[Θ ⊢ us ⇒ us'2]]$.
       Then since by \cref{lemma:merge-completeness}, 
       $[[us'1 & us'2]]$ is the weakest restriction  
       implying both $[[us'1]]$ and $[[us'2]]$,
       we have $[[Θ ⊢ us ⇒ us'1 & us'2]]$, as required.

       \item \label{case:subt-complete-forall}
       $[[ Γ ⊢ [us]uN ≤ iM ]]$ is derived by \ruleref{\ottdruleDOneForallLabel}.
       Since $[[uN]]$ does not contain negative unification variables,
       $[[uN]]$ must be of the form $[[∀pas.uN']]$,
       such that $[[ [us]uN = ∀pas.[us]uN' ]]$ and $[[ [us]uN']] \neq [[∀]]\dots$
       (assuming $[[pas]]$ does not intersect with the range of $[[us]]$).
       Also, $[[iM]] = [[∀pbs.iM']]$ and either $[[pas]]$ or $[[pbs]]$ is non-empty.

       The rest of the proof is symmetric to \cref{case:pos-subt-complete-exists} of
       \cref{lemma:pos-subt-completeness}.
       To apply the induction hypothesis, we need to show additionally that
       there are no negative unification variables in $[[uN0]] = [[ [â⁺*/pas]uN' ]]$.
       This is because $[[ uv uN0 ⊆ uv uN ∪ {â⁺*} ]]$, and $[[uN]]$ is free of negative
       unification variables by assumption.

       \item $[[ Γ ⊢ [us]uN ≤ iM ]]$ is derived by \ruleref{\ottdruleDOneNVarLabel}.\\
       Then $[[iN]] = [[ [us]uN ]] = [[ α⁻ ]] = [[iM]]$. 
       Here the first equality holds because $[[uN]]$ is not a unification variable:
       by assumption, $[[uN]]$ is free of negative unification variables.
       The second and the third equations hold because \ruleref{\ottdruleDOneNVarLabel}
       was applied. 

       The rest of the proof is symmetric to \cref{case:pos-subt-complete-pvar} of
       \cref{lemma:pos-subt-completeness}.

    \end{caseof}
\end{proof}







\section{Properties of the Declarative Typing}
\begin{lemma} \label{lemma:app-inf-equ-stable}
    If $[[Γ; Φ ⊢ iN1 ● args ⇒> iM]]$ and $[[Γ ⊢ iN1 ≈ iN2]]$ 
    then $[[Γ; Φ ⊢ iN2 ● args ⇒> iM]]$.
\end{lemma}
\begin{proof}
    By \cref{lemma:equiv-completeness}, 
    $[[Γ ⊢ iN1 ≈ iN2]]$ implies $[[iN1 ≈ iN2]]$.
    Let us prove the required judgement by induction on $[[iN1 ≈ iN2]]$.
    Let us consider the last rule used in the derivation.
    \begin{caseof}
        \item \ruleref{\ottdruleEOneNVarLabel}.
            It means that $[[iN1]]$ is $[[α⁻]]$ and $[[iN2]]$ is $[[α⁻]]$.
            Then the required property coincides with the assumption. 
        \item \ruleref{\ottdruleEOneShiftULabel}. 
            It means that $[[iN1]]$ is $[[↑iP1]]$ and $[[iN2]]$ is $[[↑iP2]]$.
            where $[[iP1 ≈ iP2]]$.

            Then the only rule applicable to infer $[[Γ; Φ ⊢ ↑iP1 ● args ⇒> iM]]$
            is \ruleref{\ottdruleDTEmptyAppLabel},
            meaning that $[[args = ·]]$ and $[[Γ ⊢ ↑iP1 ≈ iM]]$.
            Then by transitivity of equivalence \cref{corollary:equivalence-transitivity},
            $[[Γ ⊢ ↑iP2 ≈ iM]]$, and then \ruleref{\ottdruleDTEmptyAppLabel} is applicable to infer
            $[[Γ; Φ ⊢ ↑iP2 ● · ⇒> iM]]$.
        
        \item \ruleref{\ottdruleEOneArrowLabel}.
            Then we are proving that  
            $[[Γ; Φ ⊢ (iQ1 → iN1) ● v, args ⇒> iM]]$ and $[[iQ1 → iN1 ≈ iQ2 → iN2]]$
            imply $[[Γ; Φ ⊢ (iQ2 → iN2) ● v, args ⇒> iM]]$.
            
            By inversion, $[[(iQ1 → iN1) ≈ (iQ2 → iN2)]]$
            means $[[iQ1 ≈ iQ2]]$ and $[[iN1 ≈ iN2]]$.

            By inversion of $[[Γ; Φ ⊢ (iQ1 → iN1) ● v, args ⇒> iM]]$:
            \begin{enumerate}
                \item $[[Γ ; Φ ⊢ v : iP]]$
                \item $[[Γ ⊢ iQ1 ≥ iP]]$,
                    and then by transitivity \cref{corollary:subtyping-transitivity},
                    $[[Γ ⊢ iQ2 ≥ iP]]$;
                \item $[[Γ ; Φ ⊢ iN1 ● args ⇒> iM]]$, 
                    and then by induction hypothesis, $[[Γ ; Φ ⊢ iN2 ● args ⇒> iM]]$.
            \end{enumerate}

            Since we have $[[Γ ; Φ ⊢ v : iP]]$, $[[Γ ⊢ iQ2 ≥ iP]]$ and 
            $[[Γ ; Φ ⊢ iN2 ● args ⇒> iM]]$, we can apply \ruleref{\ottdruleDTArrowAppLabel}
            to infer $[[Γ; Φ ⊢ (iQ2 → iN2) ● v, args ⇒> iM]]$.

        \item \ruleref{\ottdruleEOneForallLabel}
            Then we are proving that 
            $[[Γ ; Φ ⊢ ∀pas1.iN1' ● args ⇒> iM]]$ and $[[∀pas1.iN1' ≈ ∀pas2.iN2']]$
            imply $[[Γ ; Φ ⊢ ∀pas2.iN2' ● args ⇒> iM]]$.


            By inversion of $[[∀pas1.iN1' ≈ ∀pas2.iN2']]$:
            \begin{enumerate}
                \item $[[{pas2} ∩ fv iN1 = ∅]]$,
                \item there exists a bijection 
                    $[[mu : ({pas2} ∩ fv iN2') ↔ ({pas1} ∩ fv iN1')]]$
                    such that $[[iN1' ≈ [mu] iN2']]$.
            \end{enumerate}

            By inversion of $[[Γ ; Φ ⊢ ∀pas1.iN1' ● args ⇒> iM]]$:
            \begin{enumerate}
                \item $[[Γ ⊢ σ : pas1]]$        
                \item $[[Γ ; Φ ⊢ [σ]iN1' ● args ⇒> iM]]$
                \item $[[args ≠ ·]]$
            \end{enumerate}

            Let us construct $[[Γ ⊢ σ0 : pas2]]$ in the following way:
            $$
            \begin{cases}
                [[ [σ0]α⁺ =  [σ][mu]α⁺ ]] & \text{if } [[α⁺]] \in [[ {pas2} ∩ fv iN2' ]] \\
                [[ [σ0]α⁺ =  ∃β⁻.↓β⁻ ]] & \text{otherwise (the type does not matter here)} \\
            \end{cases}
            $$

            Then to infer $[[Γ ; Φ ⊢ iN2 ● args ⇒> iM]]$, we 
            apply \ruleref{\ottdruleDTArrowAppLabel} with $[[σ0]]$. 
            Let us show the required premises:
            \begin{enumerate}
                \item $[[Γ ⊢ σ0 : pas2]]$ by construction;
                \item $[[args ≠ ·]]$ as noted above;
                \item To show $[[Γ ; Φ ⊢ [σ0]iN2' ● args ⇒> iM]]$,
                Notice that $[[ [σ0]iN2' = [σ][mu]iN2' ]]$   
                and since $[[ [mu]iN2' ≈ iN1' ]]$, $[[ [σ][mu]iN2' ≈ [σ]iN1' ]]$.
                This way, by \cref{lemma:equiv-soundness}, $[[Γ ⊢ [σ]iN1' ≈ [σ0]iN2']]$.
                Then the required judgement holds by the induction hypothesis
                applied to $[[Γ ; Φ ⊢ [σ]iN1' ● args ⇒> iM]]$.
            \end{enumerate}
    \end{caseof}
\end{proof}

\newcommand{\pureSize}[1]{\ensuremath{\mathsf{pure\_size}(#1)}}
\newcommand{\metric}[1]{\ensuremath{\mathsf{metric}(#1)}}
\newcommand{\eqNodes}[1]{\ensuremath{\mathsf{eq\_nodes}(#1)}}
\newcommand{\size}[1]{\ensuremath{\mathsf{size}(#1)}}
\newcommand{\npq}[1]{\ensuremath{\mathsf{npq}(#1)}}

\begin{definition}[Number of prenex quantifiers]
    Let us define $\npq{[[iN]]}$ and $\npq{[[iP]]}$ as the number of prenex quantifiers in these types, i.e.
    \begin{itemize}
        \item [$+$] $\npq{[[∃nas.iP]]} = |nas|$, if $[[iP ≠ ∃nbs.iP']]$,
        \item [$-$] $\npq{[[∀pas.iN]]} = |pas|$, if $[[iN ≠ ∀pbs.iN']]$.
    \end{itemize}
\end{definition}

\begin{definition}[Size of a Judgement]
    \label{def:decl-typing-size}
    For a declarative typing judgement $J$
    let us define a metrics $\size{J}$ as a pair of numbers 
    in the following way:
    \begin{itemize}
        \item [$+$] $\size{[[Γ ; Φ ⊢ v : iP]]} = (\size{[[v]]}, 0)$;
        \item [$-$] $\size{[[Γ ; Φ ⊢ c : iN]]} = (\size{[[c]]}, 0)$;
        \item [$\bullet$] $\size{[[Γ ; Φ ⊢ iN ● args ⇒> iM]]} = 
            (\size{[[args]]}, \npq{[[iN]]})$)
    \end{itemize}
    where $\size{[[v]]}$ or $\size{[[c]]}$ is the size of the 
    syntax tree of the term $[[v]]$ or $[[c]]$
    and $\size{[[args]]}$ is the sum of sizes of the terms in $[[args]]$.
\end{definition}


\begin{definition}[Number of Equivalence Nodes]
    For a tree $T$ inferring
    a declarative typing judgement,
    let us a function $\eqNodes{T}$
    as the number of nodes in $T$ labeled with \ruleref{\ottdruleDTPEquivLabel} or 
    \ruleref{\ottdruleDTNEquivLabel}.
\end{definition}

\begin{definition}[Metric]
    For a tree $T$ inferring
    a declarative typing judgement $J$,
    let us define a metric $\metric{T}$
    as a pair $(\size{J}, \eqNodes{T})$.
\end{definition}

\begin{lemma}[Declarative typing is preserved under context equivalence]
    Assuming $[[Γ ⊢ Φ1]]$, $[[Γ ⊢ Φ2]]$, and $[[Γ ⊢ Φ1 ≈ Φ2]]$:
    \begin{itemize}
        \item [$+$] 
            for any tree $T_1$ inferring $[[Γ ; Φ1 ⊢ v : iP]]$, 
            there exists a tree $T_2$ inferring $[[Γ ; Φ2 ⊢ v : iP]]$.
        \item [$-$] 
            for any tree $T_1$ inferring $[[Γ ; Φ1 ⊢ c : iN]]$, 
            there exists a tree $T_2$ inferring $[[Γ ; Φ2 ⊢ c : iN]]$.
        \item [$\bullet$] 
            for any tree $T_1$ inferring $[[Γ ; Φ1 ⊢ iN ● args ⇒> iM]]$, 
            there exists a tree $T_2$ inferring $[[Γ ; Φ2 ⊢ iN ● args ⇒> iM]]$.
    \end{itemize}
\end{lemma}
\begin{proof}
    Let us prove it by induction on the $\metric{T_1}$.
    Let us consider the last rule applied in $T_1$ (i.e., its root node).
    \begin{caseof}
        \item \ruleref{\ottdruleDTVarLabel}\\
            Then we are proving 
            that $[[Γ ; Φ1 ⊢ x : iP]]$ implies $[[Γ ; Φ2 ⊢ x : iP]]$.
            By inversion, $[[x : iP ∊ Φ1]]$, and 
            since $[[Γ ⊢ Φ1 ≈ Φ2]]$, $[[x : iP' ∊ Φ2]]$ for some $[[iP']]$ 
            such that $[[Γ ⊢ iP ≈ iP']]$.
            Then we infer $[[Γ ; Φ2 ⊢ x : iP']]$ by \ruleref{\ottdruleDTVarLabel},
            and next, $[[Γ ; Φ2 ⊢ x : iP]]$ by \ruleref{\ottdruleDTPEquivLabel}.

        \item For \ruleref{\ottdruleDTThunkLabel},
              \ruleref{\ottdruleDTPAnnotLabel}, 
              \ruleref{\ottdruleDTTLamLabel},
              \ruleref{\ottdruleDTReturnLabel}, and
              \ruleref{\ottdruleDTNAnnotLabel}
              the proof is analogous. We
              apply the induction hypothesis to the premise of the rule
              to substitute $[[Φ1]]$ for $[[Φ2]]$ in it. 
              The induction is applicable because 
              the metric of the 
              premises is less than the metric of the conclusion:
              the term in the premise is a syntactic subterm of the
              term in the conclusion.

              And after that, we apply the same rule to infer the required judgement.
              
        \item \ruleref{\ottdruleDTPEquivLabel} and \ruleref{\ottdruleDTNEquivLabel}
            In these cases, the induction hypothesis is also applicable to the premise:
            although the first component of the metric 
            is the same for the premise and the conclusion:
            $\size{[[Γ ; Φ ⊢ c : iN']]} = \size{[[Γ ; Φ ⊢ c : iN]]} = \size{[[c]]}$,
            the second component of the metric is less for the premise by one,
            since the equivalence rule was applied to turn the premise tree into
            $T1$.
            Having made this note, we continue the proof in the same way as in the previous case.

        \item \ruleref{\ottdruleDTtLamLabel}
            Then we are proving that 
            $[[Γ ; Φ1 ⊢ λx:iP.c : iP → iN]]$ implies $[[Γ ; Φ2 ⊢ λx:iP.c : iP → iN]]$.
            Analogously to the previous cases, 
            we apply the induction hypothesis to the
            equivalent contexts $[[Γ ⊢ Φ1, x:iP ≈ Φ2, x:iP]]$
            and the premise $[[Γ ; Φ1, x:iP ⊢ c : iN]]$
            to obtain $[[Γ ; Φ2, x:iP ⊢ c : iN]]$.
            Notice that $[[c]]$ is a subterm of $[[λx:iP.c]]$,
            i.e., the metric of the premise tree is less than the metric of the conclusion, 
            and the induction hypothesis is applicable.
            Then we infer $[[Γ ; Φ2 ⊢ λx:iP.c : iP → iN]]$ by \ruleref{\ottdruleDTtLamLabel}.

        \item \ruleref{\ottdruleDTVarLetLabel}
            Then we are proving that 
            $[[Γ ; Φ1 ⊢ let x = v; c : iN]]$ implies $[[Γ ; Φ2 ⊢ let x = v; c : iN]]$.
            First, we apply the induction hypothesis to 
            $[[Γ; Φ1 ⊢ v : iP]]$ to obtain $[[Γ; Φ2 ⊢ v : iP]]$ 
            of the same pure size.
            
            Then we apply the induction hypothesis to
            the equivalent contexts $[[Γ ⊢ Φ1, x:iP ≈ Φ2, x:iP]]$
            and the premise $[[Γ ; Φ1, x:iP ⊢ c : iN]]$ to obtain
            $[[Γ ; Φ2, x:iP ⊢ c : iN]]$.
            Then we infer $[[Γ ; Φ2 ⊢ let x = v; c : iN]]$ by \ruleref{\ottdruleDTVarLetLabel}.

        \item \ruleref{\ottdruleDTAppLetLabel}
            Then we are proving that 
            $[[Γ ; Φ1 ⊢ let x = v(args); c : iN]]$ implies 
            $[[Γ ; Φ2 ⊢ let x = v(args); c : iN]]$.

            We apply the induction hypothesis to each of the premises.
            to rewrite:
            \begin{itemize}
                \item $[[Γ ; Φ1 ⊢ v : ↓iM]]$ into $[[Γ ; Φ2 ⊢ v : ↓iM]]$,
                \item $[[Γ ; Φ1 ⊢ iM ● args ⇒> ↑iQ]]$ into $[[Γ ; Φ2 ⊢ iM ● args ⇒> ↑iQ]]$.
                \item $[[Γ ; Φ1, x:iQ ⊢ c : iN]]$ into $[[Γ ; Φ2, x:iQ ⊢ c : iN]]$
                (notice that $[[Γ ⊢ Φ1, x:iQ ≈ Φ2, x:iQ]]$).
            \end{itemize}

            It is left to show the uniqueness of $[[Γ ; Φ2 ⊢ iM ● args ⇒> ↑iQ]]$.
            Let us assume that this judgement holds for other $[[iQ']]$, 
            i.e.  there exists a tree $T_0$ inferring 
            $[[Γ ; Φ2 ⊢ iM ● args ⇒> ↑iQ']]$.
            Then notice that the induction hypothesis is applicable to
            $T_0$: the first component of the first component of $\metric{T_0}$
            is $S = \sum_{[[v]] \in [[args]]} \size{[[v]]}$, and it is less than
            the corresponding component of $\metric{T_1}$, which is
            $\size{[[let x = v(args); c]]} = 1 + \size{[[v]]} + \size{[[c]]} + S$.
            This way, $[[Γ ; Φ1 ⊢ iM ● args ⇒> ↑iQ']]$ holds by the induction hypothesis,
            but since $[[Γ ; Φ1 ⊢ iM ● args ⇒> ↑iQ uniq]]$, we have $[[Γ ⊢ iQ' ≈ iQ]]$.

            Then we infer $[[Γ ; Φ2 ⊢ let x = v(args); c : iN]]$ by \ruleref{\ottdruleDTAppLetLabel}.

        \item \ruleref{\ottdruleDTAppLetAnnLabel}
            Then we are proving that
            $[[Γ ; Φ1 ⊢ let x:iP = v(args); c : iN]]$ implies
            $[[Γ ; Φ2 ⊢ let x:iP = v(args); c : iN]]$.
        
            As in the previous case, we apply the induction hypothesis to each of the premises
            and rewrite:
            \begin{itemize}
                \item $[[Γ ; Φ1 ⊢ v : ↓iM]]$ into $[[Γ ; Φ2 ⊢ v : ↓iM]]$,
                \item $[[Γ ; Φ1 ⊢ iM ● args ⇒> ↑iQ]]$ into $[[Γ ; Φ2 ⊢ iM ● args ⇒> ↑iQ]]$, 
                    and
                \item $[[Γ ; Φ1, x:iP ⊢ c : iN]]$ into $[[Γ ; Φ2, x:iP ⊢ c : iN]]$
                (notice that $[[Γ ⊢ Φ1, x:iP ≈ Φ2, x:iP]]$).
            \end{itemize}
            
            Notice that $[[Γ ⊢ iP]]$ and $[[Γ ⊢ ↑iQ ≤ ↑iP]]$ 
            do not depend on the variable context, and hold by assumption.
            Then we infer $[[Γ ; Φ2 ⊢ let x:iP = v(args); c : iN]]$ by \ruleref{\ottdruleDTAppLetAnnLabel}.

        \item \ruleref{\ottdruleDTUnpackLabel}, and \ruleref{\ottdruleDTNAnnotLabel}
            are proved in the same way.

        \item \ruleref{\ottdruleDTEmptyAppLabel}
            Then we are proving that 
            $[[Γ ; Φ1 ⊢ iN ● · ⇒> iN']]$ (inferred by \ruleref{\ottdruleDTEmptyAppLabel})
            implies $[[Γ ; Φ2 ⊢ iN ● · ⇒> iN']]$.

            To infer $[[Γ ; Φ2 ⊢ iN ● · ⇒> iN']]$, 
            we apply \ruleref{\ottdruleDTEmptyAppLabel}, noting that 
            $[[Γ ⊢ iN ≈ iN']]$ holds by assumption.

        \item \ruleref{\ottdruleDTArrowAppLabel}
            Then we are proving that 
            $[[Γ ; Φ1 ⊢ iQ → iN ● v, args ⇒> iM]]$ (inferred by \ruleref{\ottdruleDTArrowAppLabel})
            implies $[[Γ ; Φ2 ⊢ iQ → iN ● v, args ⇒> iM]]$.
            And uniqueness of the $[[iM]]$ in the first case implies uniqueness in the second case.

            By induction, we rewrite $[[Γ ; Φ1 ⊢ v : iP]]$ into $[[Γ ; Φ2 ⊢ v : iP]]$, 
            and $[[Γ ; Φ1 ⊢ iN ● args ⇒> iM]]$ into $[[Γ ; Φ2 ⊢ iN ● args ⇒> iM]]$.
            Then we infer $[[Γ ; Φ2 ⊢ iQ → iN ● v, args ⇒> iM]]$ by \ruleref{\ottdruleDTArrowAppLabel}.

            Now, let us show the uniqueness.
            The only rule that can infer $[[Γ ; Φ1 ⊢ iQ → iN ● v, args ⇒> iM]]$
            is \ruleref{\ottdruleDTArrowAppLabel}.
            Then by inversion, 
            uniqueness of $[[Γ ; Φ1 ⊢ iQ → iN ● v, args ⇒> iM]]$ implies
            uniqueness of $[[Γ ; Φ1 ⊢ iN ● args ⇒> iM]]$. By 
            the induction hypothesis, it implies the uniqueness of 
            $[[Γ ; Φ2 ⊢ iN ● args ⇒> iM]]$.


            Suppose that 
            $[[Γ ; Φ2 ⊢ iQ → iN ● v, args ⇒> iM']]$.
            By inversion, $[[Γ ; Φ2 ⊢ iN ● args ⇒> iM']]$, 
            which by uniqueness of $[[Γ ; Φ2 ⊢ iN ● args ⇒> iM]]$ implies
            $[[Γ ⊢ iM ≈ iM']]$.

        \item \ruleref{\ottdruleDTForallAppLabel}
            Then we are proving that
            $[[Γ ; Φ1 ⊢ ∀pas.iN ● args ⇒> iM]]$ (inferred by \ruleref{\ottdruleDTForallAppLabel})
            implies $[[Γ ; Φ2 ⊢ ∀pas.iN ● args ⇒> iM]]$.

            By inversion, we have $[[σ]]$ such that $[[Γ ⊢ σ : pas]]$ and
            $[[Γ ; Φ1 ⊢ [σ]iN ● args ⇒> iM]]$ is inferred.
            Let us denote the inference tree as $T_1'$.
            Notice that the induction hypothesis is applicable to $T_1'$:
            $\metric{T_1'} = ((\size{[[args]]}, 0), x)$ (for some $x$) is less than 
            $\metric{T_1} = ((\size{[[args]]}, |[[pas]]|), y)$ (for some $y$),
            since $|[[pas]]| > 0$ by inversion.

            This way, by the induction hypothesis, 
            there exists a tree $T_2'$ inferring
            $[[Γ ; Φ2 ⊢ [σ]iN ● args ⇒> iM]]$.
            Notice that the premises $[[args ≠ ·]]$, $[[Γ ⊢ σ : pas]]$,
            and $[[pas ≠ ·]]$ do not depend on the variable context,
            and hold by inversion.
            Then we infer $[[Γ ; Φ2 ⊢ ∀pas.iN ● args ⇒> iM]]$ by \ruleref{\ottdruleDTForallAppLabel}.
    \end{caseof}
    
\end{proof}





\section{Properties of the Algorithmic Typing}

\subsection{Singularity}
\lemMinInstSoundness*
\begin{proof}
    We prove it by induction on the inference of $[[Γ ⊢ uP SC minby uσ ]]$.
    Let us consider the last rule used in the inference.
    \begin{caseof}
        \item \ruleref{\ottdruleSINGPUvarLabel}, which means 
            that the inferred judgment is $[[Γ ⊢  α̂⁺ SC minby (nf(iP) / α̂⁺)]]$, 
            and by inversion, $[[(α̂⁺ :≥ iP) ∊ SC]]$. Let us show the required properties:
            \begin{itemize}
                \item $[[Θ ⊢ (nf(iP) / α̂⁺) : uv α̂⁺]]$ holds trivially;
                \item $[[Θ ⊢  (nf(iP) / α̂⁺) : SC]]$ holds since
                    $[[Γ ⊢ nf(iP) : (α̂⁺ :≥ iP)]]$, which is true 
                    since $[[Γ ⊢ nf(iP) ≥ iP]]$ by the soundness of normalization
                    (\cref{lemma:normalization-soundness});
                \item $[[(nf(iP) / α̂⁺)]]$ is normalized trivially;
                \item let us take an arbitrary $[[Θ ⊢ uσ' : {α̂⁺} ]]$
                    respecting $[[SC]]$. Since $[[uσ']]$ respects $[[SC]]$,
                    $[[Γ ⊢ [uσ']α̂⁺ ≥ iP]]$ holds, and then
                    $[[Γ ⊢ [uσ']α̂⁺ ≥ nf(iP)]]$ holds by the soundness of normalization
                    and transitivity of subtyping. Finally,
                    $[[Γ ⊢ [uσ']α̂⁺ ≥ nf(iP)]]$ can be rewritten as
                    $[[Γ ⊢ [uσ']α̂⁺ ≥ [(nf(iP) / α̂⁺)]α̂⁺]]$.
            \end{itemize}
        \item \ruleref{\ottdruleSINGExistsLabel}, which means that 
            the inferred judgment has form $[[Γ ⊢ ∃nas.uP SC minby uσ ]]$,
            and by inversion, $[[Γ, nas ⊢ uP SC minby uσ]]$.
            By applying the induction hypothesis to $[[Γ, nas ⊢ uP SC minby uσ]]$
            we have 
            \begin{itemize}
                \item $[[Θ ⊢ uσ : uv uP]]$, which also means $[[Θ ⊢ uσ : uv ∃nas.uP]]$,
                \item $[[Θ ⊢ uσ : SC]]$,
                \item $[[uσ]]$ is normalized, and
                \item for any other $[[Θ ⊢ uσ' : uv uP ]]$
                    respecting $[[SC]]$ (\ie $[[Θ ⊢ uσ' : SC]]$), we have
                    $[[Γ, nas ⊢ [uσ']uP ≥ [uσ]uP ]]$, which immediately implies 
                    $[[Γ ⊢ [uσ']∃nas.uP ≥ [uσ]∃nas.uP ]]$ (the left-hand side
                    existential variables are instantiated with the corresponding 
                    right-hand side existential variables).
            \end{itemize}
        \item \ruleref{\ottdruleSINGSingLabel}, which means that
            the inferred judgment has form $[[Γ ⊢ uP SC minby uσ]]$,
            and by inversion, $[[uv uP ⊆ dom(SC)]]$ and
            $[[SC|uv uP singular with uσ]]$. Let us apply the soundness of singularity
            (\cref{lemma:singularity-soundness}) to $[[SC|uv uP singular with uσ]]$ to
            obtain the following properties:
            \begin{itemize}
                \item $[[ Θ ⊢ uσ : uv uP ∩ dom(SC) ]]$, which also means 
                    $[[Θ ⊢ uσ : uv uP]]$,
                \item $[[Θ ⊢ uσ : SC|uv uP]]$,
                \item $[[uσ]]$ is normalized, and
                \item for any other $[[Θ ⊢ uσ' : uv uP ]]$
                    respecting $[[SC|uv uP]]$, we have
                    $[[Θ ⊢ uσ' ≈ uσ : uv uP]]$. The latter means that
                    $[[Γ ⊢ [uσ']uP ≈ [uσ]uP]]$, and in particular, 
                    $[[Γ ⊢ [uσ']uP ≥ [uσ]uP]]$.
            \end{itemize}
    \end{caseof}

\end{proof}

\lemMinInstCompleteness*
\begin{proof}
    We prove it by induction on $[[uP]]$.
    \begin{caseof}
        \item \label{case:min-inst-completeness-uvar} 
            $[[uP]] = [[α̂⁺]]$. Suppose that $[[α̂⁺ ∉ dom(SC)]]$.
            Then the instantiation of $[[α̂⁺]]$ is
            not restricted, and thus, any type can instantiate it.
            However, among unrestricted instantiations, there is no minimum:
            any type $[[iP]]$ is \emph{not} a subtype of $[[↓↑iP]]$,
            which contradicts the assumption.
            This way, $[[α̂⁺ ∊ dom(SC)]]$.

            If the entry restricting $[[α̂⁺]]$ in $[[SC]]$ is a \emph{subtyping} entry
            ($[[(α̂⁺ :≥ iQ) ∊ SC]]$), then 
            we apply \ruleref{\ottdruleSINGPUvarLabel} to infer 
            $[[Γ ⊢  α̂⁺ SC minby (nf(iQ) / α̂⁺)]]$.
            It is left to show that $[[ nf(iQ) = nf([uσ]α̂⁺) ]]$.
            Since $[[Θ ⊢ uσ : SC]]$, and $[[(α̂⁺ :≥ iQ) ∊ SC]]$, 
            we know that $[[Γ ⊢ [uσ]α̂⁺ ≥ iQ]]$. 
            On the other hand, let us consider $[[Θ ⊢ uσ' : SC]]$,
            that copies $[[uσ]]$ on $[[dom(SC)]]$ except $[[α̂⁺]]$,
            where it is instantiated with $[[iQ]]$.
            Then $[[Γ ⊢ [uσ']α̂⁺ ≥ [uσ]α̂⁺ ]]$ means $[[Γ ⊢ iQ ≥ [uσ]α̂⁺ ]]$,
            this way, $[[Γ ⊢ iQ ≈ [uσ]α̂⁺ ]]$, which by \cref{lemma:subt-equiv-algorithmization}
            means $[[nf(iQ) = nf([uσ]α̂⁺)]]$.

            If the entry restricting $[[α̂⁺]]$ in $[[SC]]$ is an \emph{equivalence} entry
            ($[[(α̂⁺ :≈ iQ) ∊ SC]]$), then we wish to apply
            \ruleref{\ottdruleSINGSingLabel}.
            The first premise $[[uv α̂⁺ ⊆ dom(SC)]]$  
            holds by assumption;
            to infer $[[SC|{α̂⁺} singular with uσ0]]$, 
            we apply the completeness of singularity (\cref{lemma:singularity-completeness}).
            It applies because all the substitutions satisfying $[[SC|{α̂⁺}]] = [[(α̂⁺ :≈ iQ)]]$
            are equivalent on $[[{α̂⁺}]]$ by transitivity of equivalence 
            (\cref{corollary:equivalence-transitivity}): 
            the satisfaction of this constraint means that the substitution 
            sends $[[α̂⁺]]$ to $[[iQ]]$ or an equivalent type.
            This way, $[[SC|{α̂⁺} singular with uσ0]]$ for some $[[uσ0]]$,
            which means $[[Γ ⊢ α̂⁺ SC minby uσ0]]$.
            To show that $[[uσ0 = nf(uσ)]]$ notice that 
            Since $[[uσ0]]$ is normalized and equivalent to $[[uσ]]$ on $[[{α̂⁺}]]$, 
            and only has $[[α̂⁺]]$ in its domain
            (by soundness of singularity, \cref{lemma:singularity-soundness}).
            This way, $[[Γ ⊢ α̂⁺ SC minby uσ]]$, as required.
        \item  \label{case:min-inst-completeness-down}
            $[[uP = ↓uN]]$. Then since
            $[[Γ ⊢ ↓[uσ']uN ≥ ↓[uσ]uN ]]$ means
            $[[Γ ⊢ ↓[uσ']uN ≈ ↓[uσ]uN ]]$ by inversion.
            Then by \cref{lemma:subst-equiv}, 
            $[[uσ]]$ is equivalent to any other substitution 
            $[[Θ ⊢ uσ' : uv uN ]]$ satisfying $[[SC|uv uN]]$,
            hence, the completeness of singularity (\cref{lemma:singularity-completeness})
            can be applied to conclude that
            \begin{itemize}
                \item $[[uv uN = dom(SC|uv uN)]]$, then $[[uv uP ⊆ dom(SC)]]$,
                \item $[[SC|uv uN singular with uσ0]]$ for some (normalized) $[[uσ0]]$.
            \end{itemize}
            It means $[[Γ ⊢ uP SC minby uσ0 ]]$, and then
            since $[[uσ0]]$ is normalized and equivalent to $[[uσ0]]$ on $[[uv uN]]$,
            and its domain is $[[uv uN]]$, $[[uσ0 = nf(uσ)]]$.
        % \item $[[uP = ∃nas.uP']]$ 
        %     Then we wish to apply the induction hypothesis to show 
        %     $[[Γ, nas ⊢ uP' SC minby nf(uσ)]]$, which would imply 
        %     $[[Γ ⊢ ∃nas.uP' SC minby nf(uσ)]]$ by \ruleref{\ottdruleMINExistsLabel}.
        %     Let us demonstrate that the premises of the induction hypothesis hold. 
        %     \begin{itemize}
        %         \item  $[[Γ, nas ⊢ Θ]]$ holds by weakening;
        %         \item  $[[Γ, nas; dom(Θ) ⊢ uP']]$ holds by inversion of $[[Γ; dom(Θ) ⊢ uP]]$;
        %         \item $[[uv uP' = uv uP]]$, and 
        %                 $[[Γ ⊢ [uσ']∃nas.uP' ≥ [uσ]∃nas.uP' ]]$ implies
        %                 $[[Γ, nas ⊢ [uσ']uP' ≥ [uσ]uP' ]]$ by \cref{lemma:subst-equiv}.
        %     \end{itemize}

        \item $[[uP = ∃nas.β⁺]]$ then 
            as there are no algorithmic variables in $[[uP]]$, 
            $[[nf([uσ]uP) = β⁺]]$,
            and thus, we wish to show that 
            $[[Γ ⊢ ∃nas.β⁺ SC minby ·]]$.
            To do so, we apply \ruleref{\ottdruleSINGExistsLabel},
            and it is left to show that $[[Γ, nas ⊢ β⁺ SC minby · ]]$,
            which holds vacuously by \ruleref{\ottdruleSINGSingLabel}.
        \item $[[uP = ∃nas.α̂⁺]]$ then 
            $[[Γ ⊢ [uσ']∃nas.α̂⁺ ≥ [uσ]∃nas.α̂⁺ ]]$ implies 
            $[[Γ ⊢ ∃nas.[uσ']α̂⁺ ≥ ∃nas.[uσ]α̂⁺ ]]$ implies 
            $[[Γ ⊢ [uσ']α̂⁺ ≥ [uσ]α̂⁺ ]]$.
            It means that $[[uσ]]$ instantiates $[[α̂⁺]]$ to the minimal 
            type among all the instantiations of $[[α̂⁺]]$ respecting $[[SC]]$.
            In other words, we can apply the reasoning from 
            \cref{case:min-inst-completeness-uvar}
            to conclude that $[[Γ ⊢ α̂⁺ SC minby nf(uσ) ]]$.
            And then \ruleref{\ottdruleSINGExistsLabel}
            gives us $[[Γ ⊢ ∃nas.α̂⁺ SC minby nf(uσ) ]]$.
        \item $[[uP = ∃nas.↓uN]]$ then 
            $[[Γ ⊢ [uσ']∃nas.↓uN ≥ ∃nas.↓[uσ]uN ]]$ implies 
            $[[Γ ⊢ ∃nas.↓[uσ']uN ≥ ∃nas.↓[uσ]uN ]]$ implies
            $[[Γ, nas ⊢ ↓[σ0][uσ']uN ≥ ↓[uσ]uN ]]$ for some $[[σ0]]$ implies
            $[[Γ, nas ⊢ [σ0][uσ']uN ≈ [uσ]uN ]]$.
            By \cref{lemma:subst-equiv}, it means in particular
            that $[[uσ']]$ and $[[uσ]]$ are equivalent on 
            $[[uv uN]]$. 
            This way, we can apply the completeness of singularity 
            (\cref{lemma:singularity-completeness}), and continue as in
            \cref{case:min-inst-completeness-down}
            to conclude that $[[Γ, nas ⊢ ↓uN SC minby nf(uσ)]]$.
            Then by \ruleref{\ottdruleSINGExistsLabel},
            we have $[[Γ ⊢ ∃nas.↓uN SC minby nf(uσ)]]$.
    \end{caseof}
\end{proof}

\obsMinInstDeterministic*
\begin{proof}
    We prove it by induction on $[[Γ ⊢ uP SC minby uσ]]$.
    It is easy to see that each inference rule is deterministic.
\end{proof}

\lemEntrySingularitySoundness*
\begin{proof}
    Let us consider how $[[scE singular with iP]]$ or $[[scE singular with iN]]$ is formed.
    \begin{caseof}
        \item \ruleref{\ottdruleSINGNEqLabel}, that is $[[scE]] = [[α̂⁻ :≈ iN0]]$.
            and $[[iN]]$ is $[[nf(iN0)]]$.
            Then $[[Γ ⊢ iN' : scE]]$ means $[[Γ ⊢ iN' ≈ iN0]]$, 
            (by inversion of \ruleref{\ottdruleSATSCENEqLabel}),
            which by transitivity, using \cref{corollary:nf-sound-wrt-subt-equiv},
            means $[[Γ ⊢ iN' ≈ nf(iN0)]]$, 
            as required.
        \item \ruleref{\ottdruleSINGPEqLabel}. This case is symmetric to the previous one.

        \item \ruleref{\ottdruleSINGSupVarLabel}, that is 
            $[[scE]] = [[α̂⁺ :≥ ∃nas.β⁺]]$, and $[[iP = β⁺]]$.

            Since $[[Γ ⊢ β⁺ ≥  ∃nas.β⁺]]$, we have $[[Γ ⊢ β⁺ : scE ]]$, 
            as required.

            Notice that $[[Γ ⊢ iP' : scE]]$ means $[[Γ ⊢ iP' ≥ ∃nas.β⁺]]$.
            Let us show that it implies $[[Γ ⊢ iP' ≈ β⁺]]$.
            By applying \cref{lemma:shape-of-supertypes} once, 
            we have $[[Γ, nas ⊢ iP' ≥ β⁺]]$.
            By applying it again, we notice that
            $[[Γ, nas ⊢ iP' ≥ β⁺]]$ implies $[[iPi = ∃nas'.β⁺]]$.
            Finally, it is easy to see that $[[Γ ⊢ ∃nas'.β⁺ ≈ β⁺]]$

        \item \ruleref{\ottdruleSINGSupShiftLabel},
            that is $[[scE]] = [[α̂⁺ :≥ ∃nbs.↓iN1]]$, 
            where $[[iN1 ≈ nbj]]$, and $[[iP = ∃α⁻.↓α⁻]]$.

            Since $[[Γ ⊢ ∃α⁻.↓α⁻ ≥ ∃nbs.↓iN1]]$ 
            (by \ruleref{\ottdruleDOneExistsLabel}, with substitution $[[iN1 / α⁻]]$),
            we have $[[Γ ⊢ ∃α⁻.↓α⁻ : scE ]]$, as required.

            Notice $[[Γ ⊢ iP' : scE]]$ means $[[Γ ⊢ iP' ≥ ∃nbs.↓iN1]]$.
            Let us show that it implies $[[Γ ⊢ iP' ≈ ∃α⁻.↓α⁻]]$.
            \begin{align*}
            [[Γ ⊢ iP' ≥ ∃nbs.↓iN1]] &\Rightarrow [[Γ ⊢ nf(iP') ≥ ∃nbs'.↓nf(iN1)]]\\
                                    &\phantom{\Rightarrow} \text{(where $[[ord {nbs} in iN' = nbs']]$)}
                                    && \text{by \cref{corollary:nf-pres-subt}} \\
                                    &\Rightarrow [[Γ ⊢ nf(iP') ≥ ∃nbs'.↓nf(nbj)]]  
                                    && \text{by \cref{lemma:normalization-completeness}}\\
                                    &\Rightarrow [[Γ ⊢ nf(iP') ≥ ∃nbs'.↓nbn]]  
                                    && \text{by definition of normalization}\\
                                    &\Rightarrow [[Γ ⊢ nf(iP') ≥ ∃nbj.↓nbj]]  
                                    && \text{since $[[ord {nbs} in nf(iN1)]] = [[nbj]]$}\\
                                    &\Rightarrow [[Γ, nbj ⊢ nf(iP') ≥ ↓nbj]] \\ 
                                    &\phantom{\Rightarrow} \text { and } [[nbj ∉ fv(nf(iP'))]]
                                    && \text{by \cref{lemma:shape-supertypes-norm}}\\
            \end{align*}
            By \cref{lemma:shape-supertypes-norm}, 
            the last subtyping means that $[[nf(iP') = ∃nas.↓iN']]$,
            such that
            \begin{enumerate}
                \item $[[Γ, nbj, nas ⊢ iN']]$
                \item $[[ord {nas} in iN' = nas]]$
                \item for some substitution $[[Γ, nbj ⊢ σ :{nas}]]$, 
                    $[[ [σ]iN' = nbj ]]$.
            \end{enumerate}
            Since $[[nbj ∉ fv(nf(iP'))]]$,
            the latter means that $[[iN' = na]]$, and then 
            $[[nf(iP') = ∃na.↓na]]$ for some $[[na]]$.
            Finally, notice that all the types of shape
            $[[∃na.↓na]]$ are equal.
   \end{caseof}

\end{proof}



\lemEntrySingularityCompleteness*
\begin{proof}
    \hfill
    \begin{itemize}
        \item [$-$] 
            By \cref{lemma:constraint-sat},
            there exists $[[Γ ⊢ iN' : scE]]$.
            Since $[[iN']]$ is negative, by inversion of
            $[[Γ ⊢ iN' : scE]]$, $[[scE]]$ has shape $[[α̂⁻ :≈ iM]]$, 
            where $[[Γ ⊢ iN' ≈ iM]]$, and transitively, $[[Γ ⊢ iN ≈ iM]]$.
            Then $[[nf(iM) = nf(iN)]]$, 
            and $[[scE singular with nf(iM)]]$ (by \ruleref{\ottdruleSINGNEqLabel})
            is rewritten as $[[scE singular with nf(iN)]]$.
        \item [$+$]
            By \cref{lemma:constraint-sat}, there exists $[[Γ ⊢ iP' : scE]]$, 
            then by assumption, $[[Γ ⊢ iP' ≈ iP]]$,
            which by \cref{lemma:entry-sat-equiv} implies $[[Γ ⊢ iP : scE]]$.

            Let us consider the shape of $[[scE]]$:
            \begin{caseof}
                \item $[[scE]] = [[(α̂⁺ :≈ iQ)]]$ then 
                    inversion of $[[Γ ⊢ iP : scE]]$
                    implies $[[Γ ⊢ iP ≈ iQ]]$, and hence, $[[nf(iP) = nf(iQ)]]$
                    (by \cref{lemma:subt-equiv-algorithmization}).
                    Then $[[scE singular with nf(iQ)]]$, 
                    which holds by \ruleref{\ottdruleSINGPEqLabel}, 
                    is rewritten as $[[scE singular with nf(iP)]]$.

                \item $[[scE]] = [[(α̂⁺ :≥ iQ)]]$.
                    Then the inversion of $[[Γ ⊢ iP : scE]]$ 
                    implies $[[Γ ⊢ iP ≥ iQ]]$.
                    Let us consider the shape of $[[iQ]]$:
                    \begin{caseof}
                        \item $[[iQ]] = [[∃nbs.β⁺]]$ (for potentially empty $[[nbs]]$).
                            Then $[[Γ ⊢ iP ≥ ∃nbs.β⁺]]$ 
                            implies $[[Γ ⊢ iP ≈ β⁺]]$ by 
                            \cref{lemma:shape-of-supertypes}, 
                            as was noted in the proof of 
                            \cref{lemma:entry-singularity-soundness},
                            and hence, $[[nf(iP) = β⁺]]$.

                            Then $[[scE singular with β⁺]]$, which holds by
                            \ruleref{\ottdruleSINGSupVarLabel},
                            can be rewritten as\\ $[[scE singular with nf(iP)]]$.

                        \item $[[iQ]] = [[∃nbs.↓iN]]$ (for potentially empty $[[nbs]]$).
                            Notice that $[[Γ ⊢ ∃γ⁻.↓γ⁻ ≥ ∃nbs.↓iN]]$ 
                            (by \ruleref{\ottdruleDOneExistsLabel}, 
                            with substitution $[[iN / γ⁻]]$), and thus, 
                            $[[Γ ⊢ ∃γ⁻.↓γ⁻ : scE]]$ by \ruleref{\ottdruleSATSCESupLabel}.
                            
                            Then by assumption, $[[Γ ⊢ ∃γ⁻.↓γ⁻ ≈ iP]]$,
                            that is $[[nf(iP) = ∃γ⁻.↓γ⁻]]$.
                            To apply \ruleref{\ottdruleSINGSupShiftLabel}
                            to infer $[[(α̂⁺ :≥ ∃nbs.↓iN) singular with ∃γ⁻.↓γ⁻]]$,
                            it is left to show that $[[iN ≈ nbi]]$ for some $i$.

                            Since $[[Γ ⊢ iQ : scE]]$, by assumption,
                            $[[Γ ⊢ iQ ≈ iP]]$, and by transitivity, 
                            $[[Γ ⊢ iQ ≈ ∃γ⁻.↓γ⁻]]$.
                            It implies
                            $[[nf(∃nbs.↓iN) = ∃γ⁻.↓γ⁻]]$ (by \cref{lemma:subt-equiv-algorithmization}), 
                            which by definition of normalization means
                            $[[∃nbs'.↓nf(iN) = ∃γ⁻.↓γ⁻]]$, where $[[ord {nbs} in iN' = nbs']]$.
                            This way, $[[nbs']]$ is a variable $[[β⁻]]$, and $[[ nf(iN) = β⁻ ]]$.
                            Notice that $[[β⁻]] \in [[nbs']] \subseteq [[nbs]]$ by \cref{lemma:ord-soundness}.
                            This way, $[[iN ≈ β⁻]]$ for $[[β⁻]] \in [[nbs]]$ (by \cref{lemma:subt-equiv-algorithmization}),
                    \end{caseof}
            \end{caseof}
    \end{itemize}
\end{proof}

\lemSingularitySoundness*
\begin{proof}
    Suppose that $[[Θ ⊢ uσ' : SC]]$.
    It means that for every $[[scE]] \in [[SC]]$ restricting $[[α̂±]]$,
    $[[Θ(α̂±) ⊢ [uσ']α̂± : scE]]$ holds.
    $[[SC singular with uσ]]$ means $[[scE singular with [uσ]α̂± ]]$,
    and hence, by \cref{lemma:entry-singularity-completeness},
    $[[ Θ(α̂±) ⊢ [uσ']α̂± ≈ [uσ]α̂±  ]]$ holds.

    Since the uniqueness holds for every variable from $[[dom(SC)]]$,
    $[[uσ]]$ is equivalent to $[[uσ']]$ on this set.
\end{proof}

\obsSingularityDeterministic*
\begin{proof}
    By \cref{lemma:singularity-soundness},
    $[[Θ ⊢ uσ : Ξ]]$ and $[[Θ ⊢ uσ' : Ξ]]$.
    It means that both $[[uσ]]$ and $[[uσ']]$
    act as identity outside of $[[Ξ]]$.

    Moreover, for any $[[α̂± ∊ Ξ]]$,
    $[[Θ ⊢ SC : Ξ]]$ means that 
    there is a unique $[[scE ∊ SC]]$ restricting $[[α̂±]]$.
    Then $[[SC singular with uσ]]$ means 
    that $[[scE singular with [uσ]α̂± ]]$.
    By looking at the inference rules, it is easy to see that
    $[[ [uσ]α̂± ]]$ is uniquely determined by $[[scE]]$, which, 
    Similarly, $[[ [uσ']α̂± ]]$ is also uniquely determined by $[[scE]]$,  
    in the same way, and hence, $[[ [uσ]α̂± = [uσ']α̂± ]]$.
\end{proof}


\lemSingularityCompleteness*
\begin{proof}

    First, let us assume $[[Ξ]] \neq [[dom(SC)]]$. 
    Then there exists $[[α̂± ∊ Ξ \ dom(SC)]]$.
    Let us take $[[Θ ⊢ uσ1 : Ξ]]$ such that 
    any other substitution $[[Θ ⊢ uσ : Ξ]]$ satisfying $[[SC]]$
    is equivalent to $[[uσ1]]$ on $[[Ξ]]$.

    Notice that $[[Θ ⊢ uσ1 : SC]]$: 
    by \cref{lemma:constraint-sat}, there exists $[[uσ']]$
    such that $[[Θ ⊢ uσ' : Ξ]]$ and $[[Θ ⊢ uσ' : SC]]$,
    and by assumption, $[[Θ ⊢ uσ' ≈ uσ1 : Ξ]]$,
    implying  $[[Θ ⊢ uσ' ≈ uσ1 : dom(SC)]]$.

    Let us construct $[[uσ2]]$ such that $[[Θ ⊢ uσ2 : Ξ]]$ as
    follows:
    $$
    \begin{cases}
        [[ [uσ2]β̂± = [uσ1]β̂±  ]] & \text{if } [[β̂± ≠ α̂±]] \\
        [[ [uσ2]α̂± ]] = T & \text{where $T$ is any closed type not equivalent to $[[ [uσ1]α̂± ]]$} \\
    \end{cases}
    $$
    It is easy to see that $[[Θ ⊢ uσ2 : SC]]$ since
    $[[uσ1|dom(SC) = uσ2|dom(SC)]]$, and $[[Θ ⊢ uσ1 : SC]]$.
    However, $[[Θ ⊢ uσ2 ≈ uσ1 : Ξ]]$ does not hold because
    by construction, $[[Θ(α̂±) ⊢ [uσ2]α̂± ≈ [uσ1]α̂±]]$ does not hold. 
    This way, we have a contradiction.

    Second, let us show $[[SC singular with uσ0]]$.
    Let us take arbitrary $[[scE ∊ SC]]$ restricting $[[β̂±]]$.
    We need to show that $[[scE]]$ is singular.
    Notice that $[[Θ ⊢ uσ1 : SC]]$ implies $[[Θ(β̂±) ⊢ [uσ1]β̂±]]$ and $[[Θ(β̂±) ⊢ [uσ1]β̂± : scE ]]$.
    We will show that any other type satisfying $[[scE]]$ is equivalent to $[[ [uσ1]β̂± ]]$,
    then by \cref{lemma:entry-singularity-completeness}, $[[scE singular with [uσ1]β̂± ]]$.
    \begin{itemize}
        \item if $[[β̂±]]$ is positive, 
            let us take any type $[[Θ(β̂±) ⊢ iP']]$
            and assume $[[Θ(β̂±) ⊢ iP' : scE]]$.
            We will show that $[[Θ(β̂±) ⊢ iP' ≈ [uσ1]β̂±]]$,
            which by \cref{lemma:subt-equiv-algorithmization} will
            imply $[[scE singular with nf([uσ1]β̂±)]]$.

            Let us construct $[[uσ2]]$ such that $[[Θ ⊢ uσ2 : Ξ]]$ as
            follows:
            $$
            \begin{cases}
                [[ [uσ2]γ̂± = [uσ1]γ̂±  ]] & \text{if } [[γ̂± ≠ β̂±]] \\
                [[ [uσ2]β̂± = iP' ]]
            \end{cases}
            $$
            It is easy to see that $[[Θ ⊢ uσ2 : SC]]$:
            for $[[scE]]$, $[[Θ(β̂±) ⊢ [uσ2]β̂± : scE ]]$ by construction,
            since $[[Θ(β̂±) ⊢ iP' : scE]]$; for any other $[[scE' ∊ SC]]$
            restricting $[[γ̂±]]$, $[[ [uσ2]γ̂± = [uσ1]γ̂± ]]$, 
            and $[[Θ(γ̂±) ⊢ [uσ1]γ̂± : scE' ]]$ since $[[Θ ⊢ uσ1 : SC]]$.

            Then by assumption, $[[Θ ⊢ uσ2 ≈ uσ1 : Ξ]]$,
            which in particular means $[[Θ(β̂±) ⊢ [uσ2]β̂± ≈ [uσ1]β̂±]]$,
            that is $[[Θ(β̂±) ⊢ iP' ≈ [uσ1]β̂±]]$.
        \item if $[[β̂±]]$ is negative, the proof is analogous.
    \end{itemize}
\end{proof}

\subsection{Correctness of the Typing Algorithm}
\begin{lemma}[Soundness of typing] \label{lemma:typing-soundness}
    \hfill
    \begin{itemize}
        \item [$+$] If $[[Γ; Φ ⊨ v : iP]]$ then $[[Γ ⊢ iP]]$ and $[[Γ; Φ ⊢ v : iP]]$
        \item [$-$] If $[[Γ; Φ ⊨ c : iN]]$ then $[[Γ ⊢ iN]]$ and $[[Γ; Φ ⊢ c : iN]]$
        \item  For $[[Γ ⊢ Θ]]$ and $[[Γ; Θ ⊢ uN]]$ free from negative metavariables,
        if $[[Γ; Φ; Θ ⊨ uN ● args ⇒> uM ⫤ Θ'; SC]]$ then
        \begin{enumerate}
            \item $[[Γ ⊢ Θ']]$
            \item $[[Θ]] \subseteq [[Θ']]$
            \item $[[Γ; Θ' ⊢ uM]]$
            \item $[[uM]]$ is normalized and free from negative metavariables
            \item $[[Θ' ⊢ SC]]$
            \item for any $[[Θ' ⊢ uσ : SC]]$, we have $[[ Γ ; Φ ⊢ [uσ]uN ● args ⇒> [uσ]uM ]]$
        \end{enumerate}
    \end{itemize}
\end{lemma}
\begin{proof}
    We prove it by induction on the typing derivation.
    Let us consider the last rule used to infer the derivation.
    \begin{caseof}
        \item \ruleref{\ottdruleATVarLabel}
        \item \ruleref{\ottdruleATThunkLabel}
        \item \ruleref{\ottdruleATPAnnotLabel}
        \item \ruleref{\ottdruleATNAnnotLabel}
        \item \ruleref{\ottdruleATtLamLabel}
        \item \ruleref{\ottdruleATTLamLabel}
        \item \ruleref{\ottdruleATReturnLabel}
        \item \ruleref{\ottdruleATVarLetLabel}
        \item \ruleref{\ottdruleATAppLetAnnLabel}
        By inversion, we have:
        \begin{enumerate}
            \item $[[c]]$ is $[[let x : iP = v(args); c']]$
            \item $[[Γ ⊢ iP]]$
            \item $[[Γ; Φ ⊨ v : ↓iM]]$
            \item $[[Γ; Φ; · ⊨ uM ● args ⇒> ↑uQ ⫤ Θ; SC1]]$
            \item $[[Γ; Θ ⊨ ↑uQ ≤ ↑iP ⫤ SC2]]$
            \item $[[Θ ⊢ SC1 & SC2 = SC]]$
            \item $[[Γ; Φ, x:iP ⊨ c' : iN]]$
        \end{enumerate}

        By the soundness of constraint merge (\cref{lemma:merge-soundness}), we have 
        $[[Θ ⊢ SC]]$. Let us take $[[uσ]]$ such that $[[Θ ⊢ uσ : SC]]$
        (it exists by \cref{lemma:substitution-existence}). Notice that by the soundness of 
        constraint merge, $[[Θ ⊢ uσ : SC1]]$ and $[[Θ ⊢ uσ : SC2]]$.

        By the induction hypothesis applied to $[[Γ; Φ ⊨ v : ↓iM]]$, we have
        $[[Γ; Φ ⊢ v : ↓iM]]$ and $[[Γ ⊢ ↓iM]]$ (and hence,$[[Γ ; Θ ⊢ uM]]$).

        By the induction hypothesis applied to I$[[Γ; Φ, x:iP ⊨ c' : iN]]$, we have
        $[[Γ; Φ, x:iP ⊢ c' : iN]]$ and $[[Γ ⊢ iN]]$. 

        By the induction hypothesis applied to $[[Γ; Φ; · ⊨ uM ● args ⇒> ↑uQ ⫤ Θ; SC1]]$, we have:
        \begin{enumerate}
            \item \label{typing-soundness:theta-wf} $[[Γ ⊢ Θ]]$,
            \item $[[Γ; Θ ⊢ ↑uQ]]$,
            \item $[[Θ' ⊢ SC1]]$,
            \item for any $[[Θ' ⊢ uσ : SC1]]$, we have $[[ Γ ; Φ ⊢ [uσ]uM ● args ⇒> [uσ]↑uQ ]]$.
            In particular, it holds for the $[[uσ]]$ chosen above. 
        \end{enumerate}

        By the soundness of negative subtyping (\cref{lemma:neg-subtyping-soundness})
        applied to $[[Γ; Θ ⊨ ↑uQ ≤ ↑iP ⫤ SC]]$, we have $[[Γ ⊢ ↑[uσ]uQ ≤ ↑iP]]$.

        To infer $[[Γ ; Φ ⊢ let x : iP = v(args); c' : iN ]]$,
        we apply the corresponding declarative rule \ruleref{\ottdruleDTAppLetAnnLabel}, where
        $[[iQ]]$ is $[[ [uσ]uQ  ]]$. Notice that all the premises were already shown to
        hold above:
        \begin{enumerate}
            \item $[[Γ ⊢ iP]]$ and $[[Γ; Φ ⊢ v : ↓iM]]$ from the inversion,
            \item $[[Γ; Φ ⊢ iM ● args ⇒> ↑[uσ]uQ]]$ holds since $[[ [uσ]↑uQ ]] = [[ ↑[uσ]uQ ]]$,
            \item $[[Γ ⊢ ↑[uσ]uQ ≤ ↑iP]]$ by the soundness of negative subtyping,
            \item $[[Γ; Φ, x:iP ⊢ c' : iN]]$ from the the induction hypothesis.
        \end{enumerate}

        \item \ruleref{\ottdruleATAppLetLabel}
        By the inversion, we have:
        \begin{enumerate}
            \item $[[c]]$ is $[[let x = v(args) ; c']]$
            \item $[[Γ; Φ ⊨ v : ↓iM]]$ 
            \item $[[Γ ; Φ ; · ⊨ uM ● args ⇒> ↑uQ ⫤ Θ; SC]]$
            \item $[[uv uQ ⊆ dom(SC)]]$
            \item $[[SC|uv(uQ) singular with uσ3]]$
            \item $[[Γ; Φ, x:[uσ3]uQ ⊨ c' : iN]]$
        \end{enumerate}

        By the induction hypothesis applied to $[[Γ; Φ ⊨ v : ↓iM]]$, we have    
        $[[Γ; Φ ⊢ v : ↓iM]]$ and $[[Γ ⊢ ↓iM]]$ (and thus, $[[Γ ; Θ  ⊢ uM]]$).
       
        By the induction hypothesis applied to $[[Γ; Φ, x:[uσ3]uQ ⊨ c' : iN]]$, we have
        $[[Γ ⊢ iN]]$ and $[[Γ; Φ, x:[uσ3]uQ ⊢ c' : iN]]$.

        By the induction hypothesis applied to 
        $[[Γ ; Φ ; · ⊨ uM ● args ⇒> ↑uQ ⫤ Θ; SC]]$, we have:
        \begin{enumerate}
            \item $[[Γ ⊢ Θ]]$
            \item $[[Γ; Θ ⊢ ↑uQ]]$
            \item $[[Θ ⊢ SC]]$
            \item for any $[[Θ ⊢ uσ : SC]]$, we have $[[ Γ ; Φ ⊢ [uσ]uM ● args ⇒> [uσ]↑uQ ]]$, 
                which, since  $[[iM]]$ is ground means $[[ Γ ; Φ ⊢ iM ● args ⇒> ↑[uσ]uQ]]$.
        \end{enumerate}

        To infer $[[Γ ; Φ ⊢ let x = v(args) ; c' : iN ]]$, 
        we apply the corresponding 
        declarative rule \ruleref{\ottdruleDTAppLetLabel}.
        Let us show that the premises hold:
        \begin{itemize}
            \item $[[Γ; Φ ⊢ v : ↓iM]]$ holds by the induction hypothesis;
            \item $[[Γ; Φ, x:[uσ3]uQ ⊢ c' : iN]]$ also holds by the induction hypothesis, as noted above;
            \item Let us take an arbitrary substitution $[[uσ]]$ 
                satisfying $[[Θ ⊢ uσ : SC]]$ (it exists by \cref{lemma:constraint-sat}).
                Then $[[Γ; Φ ⊢ iM ● args ⇒> ↑[uσ]uQ ]]$ holds, as noted above;
            \item To show the uniqueness of $[[↑[uσ]uQ]]$,
                we assume that for some other type $[[iK]]$ 
                holds $[[Γ; Φ ⊢ iM ● args ⇒> iK ]]$,
                that is $[[Γ; Φ ⊢ [·]uM ● args ⇒> iK ]]$.
                Then by the completeness of typing 
                (\cref{lemma:typing-completeness}),
                there exist $[[uN']]$, $[[Θ']]$, and $[[SC']]$ such that
                \begin{enumerate}
                    \item $[[ Γ; Φ; · ⊨ uM ● args ⇒> uN' ⫤ Θ'; SC' ]]$ and
                    \item there exists a substitution $[[Θ' ⊢ uσ' : SC']]$ such that
                    $[[Γ ⊢ [uσ']uN' ≈ iK]]$.
                \end{enumerate}
            By the determinicity of the typing algorithm (\cref{lemma:typing-determinicity}),
            $[[ Γ; Φ; · ⊨ uM ● args ⇒> uN' ⫤ Θ'; SC' ]]$,
            means that $[[SC']]$ is $[[SC]]$, $[[Θ']]$ is $[[Θ]]$, and $[[uN']]$ is
            $[[↑uQ]]$. 
            This way, $[[Γ ⊢ [uσ']↑uQ ≈ iK]]$ for a substitution 
            $[[Θ ⊢ uσ' : SC]]$. 

            It is left to show that $[[Γ ⊢ [uσ']↑uQ ≈ [uσ]↑uQ]]$, 
            then by transitivity of the equivalence, we will have $[[Γ ⊢ [uσ]↑uQ ≈ iK]]$.
            Since $[[Θ ⊢ uσ : SC|uv(uQ)]]$ and $[[Θ ⊢ uσ' : SC|uv(uQ)]]$, 
            and $[[SC|uv(uQ) singular with uσ3]]$, we have 
            $[[Θ ⊢ uσ ≈ uσ3 : dom(SC|uv(uQ))]]$
            and $[[Θ ⊢ uσ' ≈ uσ3 : dom(SC|uv(uQ))]]$.
            Then since $[[uv(uQ) ⊆ dom(SC)]]$, we have $[[dom(SC|uv(uQ)) = uv(uQ)]]$.
            This way, by transitivity and symmetry of the equivalence, 
            $[[Θ ⊢ uσ ≈ uσ' : uv(uQ)]]$, which implies
            $[[Γ ⊢ [uσ']↑uQ ≈ [uσ]↑uQ]]$. 
        \end{itemize}

        \item \ruleref{\ottdruleATUnpackLabel}
        By the inversion, we have:
        \begin{enumerate}
            \item $[[c]]$ is $[[let∃ (α⁻, x) = v; c']]$
            \item $[[Γ; Φ ⊨ v : ∃α⁻.iP]]$
            \item $[[Γ, α⁻ ; Φ, x:iP ⊨ c' : iN]]$
            \item $[[Γ ⊢ iN]]$
        \end{enumerate}

        By the induction hypothesis applied to 
        $[[Γ; Φ ⊨ v : ∃α⁻.iP]]$, we have $[[Γ; Φ ⊢ v : ∃α⁻.iP]]$.
        By the induction hypothesis applied to
        $[[Γ, α⁻ ; Φ, x:iP ⊨ c' : iN]]$, we have $[[Γ, α⁻ ; Φ, x:iP ⊢ c' : iN]]$.

        To show $[[Γ; Φ ⊢ let∃ (α⁻, x) = v; c' : iN]]$, we apply the corresponding
        declarative rule \ruleref{\ottdruleDTUnpackLabel}. Let us show that the premises hold:
        \begin{enumerate}
            \item $[[Γ ; Φ ⊢ v : ∃α⁻.iP]]$ holds by the induction hypothesis, as noted above,
            \item $[[Γ, α⁻ ; Φ, x:iP ⊢ c' : iN]]$ also holds by the induction hypothesis,
            \item $[[Γ ⊢ iN]]$ holds by the inversion, as noted above.
        \end{enumerate}

        \item \ruleref{\ottdruleATEmptyAppLabel}
        Then by assumption:
        \begin{itemize}
            \item $[[Γ ⊢ Θ]]$,
            \item $[[Γ; Θ ⊢ uN]]$ is free from negative metavariables,
            \item $[[Γ; Φ; Θ ⊨ uN ● · ⇒> nf(uN) ⫤ Θ; ·]]$, which by inversion means that $[[uN ≠ ∀pas.uM]]$.
        \end{itemize}

        Let us show the required properties: 
        \begin{enumerate}
            \item $[[Γ ⊢ Θ]]$ holds by assumption,
            \item $[[Θ]] \subseteq [[Θ]]$ holds trivially,
            \item $[[nf(uN)]]$ is evidently normalized, 
                $[[Γ; Θ ⊢ uN]]$ implies $[[Γ; Θ ⊢ nf(uN)]]$ by 
                \cref{corollary:wf-nf},
                and \cref{lemma:fv-nf} means that $[[nf(uN)]]$ is 
                inherently free from negative metavariables,
            \item $[[Θ ⊢ ·]]$ holds trivially,
            \item for any $[[Θ ⊢ uσ : ·]]$, we have $[[ Γ ; Φ ⊢ [uσ]uN ● · ⇒> [uσ]uN ]]$.
                To show $[[ Γ ; Φ ⊢ [uσ]uN ● · ⇒> [uσ]uN ]]$, we apply the corresponding 
                declarative rule \ruleref{\ottdruleDTEmptyAppLabel}. 
        \end{enumerate}

   
        \item \ruleref{\ottdruleATArrowAppLabel}\\
        By assumption:
        \begin{enumerate}
            \item $[[Γ ⊢ Θ]]$,
            \item $[[Γ; Θ ⊢ uQ → uN]]$ is free from negative metavariables,
            \item $[[Θ ⊢ SC1 & SC2 = SC]]$,
            \item $[[Γ; Φ; Θ ⊨ uQ → uN ● v , args ⇒> uM ⫤ Θ'; SC]]$, 
                and by inversion: 
                \begin{enumerate}
                    \item $[[Γ; Φ ⊨ v : iP]]$,
                        and by the induction hypothesis applied to this judgment,
                        $[[Γ; Φ ⊢ v : iP]]$,
                    \item $[[Γ; Θ ⊨ uQ ≥ iP ⫤ SC1]]$,
                        and by the soundness of subtyping:
                        $[[Θ ⊢ SC]]$ and
                        for any $[[Θ ⊢ uσ : SC1]]$, we have $[[Γ ⊢ [uσ]uQ ≥ iP]]$,
                    \item $[[Γ; Φ; Θ ⊨ uN ● args ⇒> uM ⫤ Θ'; SC2]]$,
                        and by the induction hypothesis applied to this judgment,
                        \begin{enumerate}
                            \item $[[Γ ⊢ Θ']]$,
                            \item $[[Θ ⊆ Θ']]$,
                            \item $[[Γ; Θ' ⊢ uM]]$ is free from negative metavariables,
                            \item $[[Θ' ⊢ SC2]]$,
                            \item for any $[[Θ' ⊢ uσ : SC2]]$, we have 
                                $[[ Γ ; Φ ⊢ [uσ]uN ● args ⇒> [uσ]uM ]]$.
                        \end{enumerate}
                \end{enumerate}
        \end{enumerate}

        Let us show the required properties:
        \begin{enumerate}
            \item $[[Γ ⊢ Θ']]$ is shown above,
            \item $[[Θ ⊆ Θ']]$ is shown above,
            \item $[[Γ; Θ' ⊢ uM]]$ free from negative metavariables, as shown above,
            \item $[[Θ' ⊢ SC]]$ holds:  
            $[[Θ ⊢ SC1]]$ implies $[[Θ' ⊢ SC1]]$,
            then we apply the soundness of constraint merge (\cref{lemma:merge-soundness})
            to $[[Θ' ⊢ SC1 & SC2]]$,
            \begin{enumerate}
                \item $[[Θ' ⊢ SC1]]$,
                \item for any $[[Θ' ⊢ uσ : SC]]$, $[[Θ' ⊢ uσ : SCi]]$ holds;
            \end{enumerate}
            \item suppose that $[[Θ' ⊢ uσ : SC]]$. Then to 
                show $[[ Γ ; Φ ⊢ [uσ](uQ → uN) ● v , args ⇒> [uσ]uM ]]$, 
                that is $[[ Γ ; Φ ⊢ [uσ]uQ → [uσ]uN ● v , args ⇒> [uσ]uM ]]$,
                we apply the corresponding declarative rule \ruleref{\ottdruleDTArrowAppLabel}.
                Let us show the required premises:
                \begin{enumerate}
                    \item $[[Γ; Φ ⊢ v : iP]]$ holds as shown above,
                    \item $[[Γ ⊢ [uσ]uQ ≥ iP]]$ holds by the soundness of subtyping 
                        as noted above,
                        since $[[Θ' ⊢ uσ : SC]]$ implies $[[Θ ⊢ uσ : SC1]]$.
                    \item $[[Γ; Φ ⊢ [uσ]uN ● args ⇒> [uσ]uM]]$ holds by the induction hypothesis
                        as shown above,
                        since $[[Θ' ⊢ uσ : SC]]$ implies $[[Θ' ⊢ uσ : SC2]]$.
                \end{enumerate}
        \end{enumerate}

        \item \ruleref{\ottdruleATForallAppLabel}\\
        By assumption:
        \begin{enumerate}
            \item $[[Γ ⊢ Θ]]$,
            \item $[[Γ; Θ ⊢ ∀pas.uN]]$,
            \item $[[Γ; Φ; Θ ⊨ ∀pas.uN ● args ⇒> uM ⫤ Θ'; SC]]$, which by inversion means
                $[[args ≠ ·]]$ and $[[Γ; Φ; Θ, â⁺*[Γ] ⊨ [â⁺*/pas]uN ● args ⇒> uM ⫤ Θ'; SC]]$.
                It is easy to see that the induction hypothesis is applicable to the latter judgment:
                $[[Γ ⊢ Θ, â⁺*[Γ] ]]$ is implied by $[[Γ ⊢ Θ]]$, and $[[Γ; Θ, â⁺*[Γ] ⊢ [â⁺*/pas]uN]]$
                is holds since $[[Γ; Θ ⊢ ∀pas.uN]]$.
                Let us apply the inductive hypothesis to the latter judgment to obtain:
                \begin{enumerate}
                    \item $[[Γ ⊢ Θ']]$,
                    \item $[[Θ, â⁺*[Γ] ⊆ Θ']]$,
                    \item $[[Γ; Θ' ⊢ uM]]$ is free from negative metavariables,
                    \item $[[Θ' ⊢ SC]]$,
                    \item for any $[[Θ' ⊢ uσ : SC]]$, we have $[[ Γ ; Φ ⊢ [uσ][â⁺*/pas]uN ● args ⇒> [uσ]uM ]]$.
                \end{enumerate}
        \end{enumerate}

        Let us show the required properties:
        \begin{enumerate}
            \item $[[Γ ⊢ Θ']]$ is shown above,
            \item $[[Θ ⊆ Θ']]$ since $[[Θ, â⁺*[Γ] ⊆ Θ']]$,
            \item $[[Γ; Θ' ⊢ uM]]$ is free from negative metavariables, as shown above,
            \item $[[Θ' ⊢ SC]]$ is shown above,
            \item let us assume $[[Θ' ⊢ uσ : SC]]$
            Then to show $[[ Γ ; Φ ⊢ [uσ]∀pas.uN ● args ⇒> [uσ]uM ]]$,
            we apply the corresponding declarative rule \ruleref{\ottdruleDTForallAppLabel}
            with substitution $[[ Γ ⊢ σ : pas ]]$  defined in the following way:
            $[[ [σ]αi⁺ ]] = [[ [uσ]αî⁺ ]]$.

            Let us show that its premises hold:
            \begin{enumerate}
                \item $[[Γ ⊢ σ : pas]]$, i.e.
                $[[ Γ ⊢ [σ] αi⁺ ]]$ holds since $[[ Θ' ⊢ uσ ]]$ and $[[Γ ⊢ Θ']]$;
                \item $[[Γ; Φ ⊢ [σ][uσ]uN ● args ⇒> [uσ]uM ]]$ 
                    holds by rewriting 
                    $[[ Γ ; Φ ⊢ [uσ][â⁺*/pas]uN ● args ⇒> [uσ]uM ]]$
                    using equality $[[ [uσ][â⁺*/pas]uN = [σ][uσ]uN ]]$:
                    \begin{enumerate}
                        \item for $[[ αi⁺ ]] \in [[ pas ]]$, $[[ [uσ][â⁺*/pas] αi⁺ ]] = [[ [uσ]αî⁺ ]] = [[ [σ]αi⁺ ]] = [[ [σ][uσ]αi⁺ ]]$,
                        \item for $[[ β̂± ]] \in [[ dom(uσ) ]]$, $[[ [uσ][â⁺*/pas]β̂±  ]] = [[ [uσ]β̂±  ]] = [[ [σ][uσ]β̂± ]] $, 
                            where the latter equality holds since $[[ {pas} ∩ {Γ} = ∅ ]]$.
                    \end{enumerate}

                \item $[[args ≠ ·]]$ holds by assumption
            \end{enumerate}
        \end{enumerate}
    \end{caseof}
\end{proof}


\begin{lemma}[Completeness of Typing]
    \label{lemma:typing-completeness}
    \begin{itemize}
        \item [$+$] If $[[Γ; Φ ⊢ v : iP]]$ then  $[[Γ; Φ ⊨ v : nf(iP)]]$        
        \item [$-$] If $[[Γ; Φ ⊢ c : iN]]$ then  $[[Γ; Φ ⊨ c : nf(iN)]]$
        \item [$\bullet$] Suppose that 
            $[[Γ; Φ ⊢ [uσ]uN ● args ⇒> iM]]$ holds for some
            $[[Γ ⊢ Θ]]$,
            $[[Γ; Θ ⊢ uN]]$ (free from negative metavariables, that is $[[α̂⁻]] \notin [[uv uN]]$), 
            $[[Θ ⊢ uσ]]$, and $[[Γ ⊢ iM]]$. Then
            there exist normalized $[[uM']]$, $[[Θ']]$, and $[[SC]]$ such that
            \begin{enumerate}
                \item $[[ Γ; Φ; Θ ⊨ uN ● args ⇒> uM' ⫤ Θ'; SC ]]$ and
                \item for any $[[Θ ⊢ uσ]]$ and $[[Γ ⊢ iM]]$
                    such that $[[Γ; Φ ⊢ [uσ]uN ● args ⇒> iM]]$, 
                    there exists $[[uσ']]$ such that 
                    \begin{enumerate}
                        \item $[[Θ' ⊢ uσ' : SC]]$,
                        \item $[[Θ ⊢ uσ' ≈ uσ : dom(Θ)]]$, and 
                        \item $[[Γ ⊢ [uσ']uM' ≈ iM]]$.
                    \end{enumerate}
            \end{enumerate}
    \end{itemize}
\end{lemma}
\begin{proof}
    By induction on the typing derivation.
    Let us consider the last rule applied to infer the derivation.
    % First, let us consider the case of $[[Γ; Φ ⊢ [uσ]uN ● args ⇒> iM]]$, 
    % when the substitution $[[uσ]]$ can change the outer shape of $[[uN]]$.
    % after that, we consider the last rule applied to infer the derivation
    % (assuming that $[[ uσ ]]$ preserves the outer shape of the type it applied to).
    \begin{caseof}

        \item \ruleref{\ottdruleDTAppLetLabel}\\
            By assumption, $[[c]]$ is $[[let x = v(args); c']]$. 
            Then by inversion of
            $[[Γ ; Φ ⊢ let x = v(args); c' : iN]]$: 
            \begin{itemize}
                \item $[[Γ ; Φ ⊢ v : ↓iM]]$, 
                    which by the induction hypothesis means 
                    $[[Γ; Φ ⊨ v : ↓nf(iM)]]$;
                \item $[[Γ ; Φ ⊢ iM ● args ⇒> ↑iQ uniq]]$. 
                    Then by \cref{lemma:app-inf-equ-stable}, since 
                    $[[Γ ⊢ iM ≈ nf(iM)]]$, we have
                    $[[Γ ; Φ ⊢ nf(iM) ● args ⇒> ↑iQ]]$.
                    Then the induction hypothesis applied to 
                    $[[Γ ; Φ ⊢ [·]nf(uM) ● args ⇒> ↑iQ]]$
                    means that there exist $[[uM']]$, $[[Θ]]$, and $[[SC]]$ such that
                    (considering $[[iM]]$ is ground):
                    \begin{enumerate}
                        \item $[[ Γ; Φ; · ⊨ nf(uM) ● args ⇒> uM' ⫤ Θ; SC ]]$, 
                            which, by the soundness, implies, in particular
                            that 
                            \begin{enumerate}
                                \item $[[uM']]$ is normalized and 
                                    free of negative metavariables, 
                                \item \label{point:typing-completeness:AppLet:ih-sound} 
                                    for any $[[Θ ⊢ uσ : SC]]$, 
                                    we have $[[ Γ ; Φ ⊢ nf(iM) ● args ⇒> [uσ]uM' ]]$,
                                    which, since $[[Γ ; Φ ⊢ nf(iM) ● args ⇒> ↑iQ uniq]]$,
                                    means $[[Γ ⊢ [uσ]uM' ≈ ↑iQ]]$.
                            \end{enumerate}
                            and
                        \item for any $[[Γ ⊢ iM'']]$
                            such that $[[Γ; Φ ⊢ nf(iM) ● args ⇒> iM'']]$,
                            (and in particular, for $[[Γ ⊢ ↑iQ]]$)
                            there exists $[[uσ1]]$ such that 
                            \begin{enumerate}
                                \item $[[Θ ⊢ uσ1 : SC]]$, and 
                                \item $[[Γ ⊢ [uσ1]uM' ≈ iM'']]$, and 
                                in particular, $[[Γ ⊢ [uσ1]uM' ≈ ↑iQ]]$.
                                since $[[uM']]$ is
                                normalized and free of 
                                negative metavariables means that 
                                $[[uM' = ↑uP]]$ for some $[[uP]]$, 
                                that is $[[Γ ⊢ [uσ1]uP ≈ iQ]]$.
                            \end{enumerate}
                    \end{enumerate}
                \item $[[Γ; Φ, x:iQ ⊢ c' : iN]]$
            \end{itemize}


            To infer $[[Γ ; Φ ⊢ let x = v(args); c' : nf(iN)]]$, 
            let us apply the corresponding algorithmic rule 
            (\ruleref{\ottdruleATAppLetLabel}):
            \begin{enumerate}
                \item $[[Γ ; Φ ⊨ v : ↓nf(iM)]]$ holds as noted above;

                \item $[[Γ; Φ ; · ⊨ nf(uM) ● args ⇒> ↑uP ⫤ Θ; SC]]$ holds as noted above;

                \item To show $[[uv uP ⊆ dom(SC)]]$ and 
                    $[[SC | uv uP singular with uσ0]]$ (for some $[[uσ0]]$),
                    we apply \cref{lemma:singularity-completeness}.
                    Let us show that the premise of this lemma holds.

                    As noted in \ref{point:typing-completeness:AppLet:ih-sound},
                    for any $[[uσ]]$, $[[Θ ⊢ uσ : SC]]$ implies 
                    $[[Γ ⊢ [uσ]uM' ≈ ↑iQ]]$,
                    which is rewritten as $[[Γ ⊢ [uσ]uP ≈ iQ]]$.
                    And since $[[Γ ⊢ [uσ']uP ≈ iQ]]$, 
                    we have $[[Γ ⊢ [uσ]uP ≈ [uσ']uP]]$.
                    It implies $[[Θ ⊢ uσ ≈ uσ' : uv uP]]$
                    by \cref{lemma:subst-equiv-metavar}.

                \item Let us show $[[Γ; Φ, x:[uσ0]uP ; Θ ⊨ c' : nf(iN)]]$.
                    By the soundness of singularity 
                    (\cref{lemma:singularity-soundness}),
                    $[[Θ ⊢ uσ0 : SC]]$,
                    which by \ref{point:typing-completeness:AppLet:ih-sound}
                    means $[[Γ ⊢ [uσ0]uM' ≈ ↑iQ]]$,
                    that is $[[Γ ⊢ [uσ0]uP ≈ iQ]]$.

                \item 
                    \label{point:typing-completeness:AppLet:body},
                    which by the induction hypothesis means
                    $[[Γ; Φ, x:iQ ⊨ c' : nf(iN)]]$.

            \end{enumerate}









        \item \ruleref{\ottdruleDTForallAppLabel}\\
            Since $[[uN]]$ cannot be a metavariable,  
            if $[[ [uσ]uN ]]$ starts from $[[∀]]$,
            so does $[[uN]]$. This way,
            $[[uN = ∀pas.uN1]]$.
            Then by assumption:
            \begin{enumerate}
                \item $[[Γ; Θ ⊢ ∀pas.uN1]]$ is free from negative metavariables, 
                    and then $[[Γ, pas; Θ ⊢ uN1]]$ is free from negative metavariables;
                \item $[[Θ ⊢ uσ]]$;
                \item $[[Γ ⊢ iM]]$;
                \item $[[Γ; Φ ⊢ [uσ]∀pas.uN1 ● args ⇒> iM]]$, 
                    \label{point:typing-completeness-forall-app-inversion}
                    that is $[[Γ; Φ ⊢ (∀pas.[uσ]uN1) ● args ⇒> iM]]$.
                    Then by inversion there exists $[[σ]]$ such that 
                    \begin{enumerate}
                        \item $[[Γ ⊢ σ : pas]]$;
                        \item $[[args ≠ ·]]$; and
                        \item $[[Γ ; Φ ⊢ [σ][uσ]uN1 ● args ⇒> iM]]$.
                            \label{point:typing-completeness-forall-app-inversion-2}
                            Notice that $[[σ]]$ and $[[uσ]]$ commute because 
                            the codomain of $[[σ]]$ does not contain
                            metavariables (and thus, does not intersect with 
                            the domain of $[[uσ]]$), and the codomain of $[[uσ]]$ is 
                            $[[Γ]]$ and does not intersect with $[[pas]]$---the domain of $[[σ]]$.

                            Let us construct $[[uN0]]$ = $[[ [â⁺*/pas]uN1 ]]$
                            and $[[Θ, â⁺*[Γ] ⊢ uσ0]]$ defined as
                            $$
                            \begin{cases}
                                [[ [uσ0]αî⁺ = [σ]αi⁺ ]] & \text{for $[[αî⁺]] \in  [[â⁺*]]$ }\\
                                [[ [uσ0]β̂± = [uσ]β̂± ]] & \text{for $[[β̂±]] \in [[dom(Θ)]]$}
                            \end{cases}
                            $$

                            Then it is easy to see that $[[ [uσ0][â⁺*/pas]uN1 = [σ][uσ]uN1 ]]$
                            because this substitution compositions coincide on 
                            $[[ {pas} ∪ dom(Θ) ]]$, their domain.
                            In other words, $[[ [uσ0]uN0 = [σ][uσ]uN1 ]]$.

                            Then let us apply the induction hypothesis
                            to $[[Γ; Φ ⊢ [uσ0]uN0 ● args ⇒> iM]]$ and obtain 
                            $[[uM']]$, $[[Θ']]$, and $[[SC]]$ such that
                            \begin{itemize}
                                \item $[[ Γ; Φ; Θ, â⁺*[Γ] ⊨ uN0 ● args ⇒> uM' ⫤ Θ'; SC ]]$ and
                                \item \label{point:typing-completeness-forall-app-inversion-3}
                                for any $[[Θ, â⁺*[Γ]  ⊢ uσ0]]$ and $[[Γ ⊢ iM]]$
                                    such that $[[Γ; Φ ⊢ [uσ0]uN0 ● args ⇒> iM]]$, 
                                    there exists $[[uσ0']]$ such that 
                                \begin{enumerate}
                                    \item $[[Θ' ⊢ uσ0' : SC]]$,
                                    \item $[[Θ, â⁺*[Γ] ⊢ uσ0' ≈ uσ0 : dom(Θ) ∪ {â⁺*}]]$, and 
                                    \item $[[Γ ⊢ [uσ0']uM' ≈ iM]]$.
                                \end{enumerate}
                            \end{itemize}
                    \end{enumerate}
            \end{enumerate}
            Let us take $[[uM']]$, $[[Θ']]$, and $[[SC]]$ from the induction hypothesis
            (\ref{point:typing-completeness-forall-app-inversion-2}) and show they 
            satisfy the required properties.
            \begin{enumerate}
                \item to infer $[[ Γ; Φ; Θ ⊨ ∀pas.uN1 ● args ⇒> uM' ⫤ Θ'; SC ]]$
                    we apply the corresponding algorithmic rule \ruleref{\ottdruleATForallAppLabel},
                    not that the required premises hold, as noted above:
                    \begin{enumerate}
                        \item $[[args ≠ ·]]$, and 
                        \item $[[Γ; Φ; Θ, â⁺*[Γ] ⊨ [â⁺*/pas]uN1 ● args ⇒> uM' ⫤ Θ'; SC]]$
                            can be rewritten as 
                            $[[ Γ; Φ; Θ, â⁺*[Γ] ⊨ uN0 ● args ⇒> uM' ⫤ Θ'; SC ]]$.
                    \end{enumerate}
                \item Let us take and arbitrary $[[Θ ⊢ uσ]]$ and $[[Γ ⊢ iM]]$
                    and assume $[[Γ; Φ ⊢ [uσ]∀pas.uN1  ● args ⇒> iM]]$. 
                    Then the same reasoning as in 
                    \ref{point:typing-completeness-forall-app-inversion-2}
                    applies. In particular, we construct 
                    $[[Θ, â⁺*[Γ] ⊢ uσ0]]$ as an extension of $[[uσ]]$
                    and obtain 
                    $[[Γ; Φ ⊢ [uσ0]uN0 ● args ⇒> iM]]$.

                    It means, we can apply the property inferred from the induction hypothesis 
                    (\ref{point:typing-completeness-forall-app-inversion-3})
                    to obtain $[[uσ0']]$ such that 
                    \begin{enumerate}
                        \item $[[Θ' ⊢ uσ0' : SC]]$,
                        \item $[[Θ, â⁺*[Γ] ⊢ uσ0' ≈ uσ0 : dom(Θ) ∪ {â⁺*}]]$, and 
                        \item $[[Γ ⊢ [uσ0']uM' ≈ iM]]$.
                    \end{enumerate}

                    Let us show that the obtained $[[uσ0']]$ satisfies the required properties.
                    \begin{enumerate}
                        \item $[[Θ' ⊢ uσ0' : SC]]$ holds as shown,
                        \item $[[Γ ⊢ [uσ0']uM' ≈ iM]]$ holds as shown,
                        \item $[[Θ ⊢ uσ0' ≈ uσ : dom(Θ)]]$,
                            holds. Let us take an arbitrary 
                            $[[β̂±]] \in [[dom(Θ)]] \subseteq [[dom(Θ) ∪ {â⁺*}]]$. Then 
                            since $[[Θ, â⁺*[Γ] ⊢ uσ0' ≈ uσ0 : dom(Θ) ∪ {â⁺*}]]$, 
                            we have $[[ [uσ0']β̂±  = [uσ0]β̂± ]]$ and 
                            by definition of $[[uσ0]]$, $[[ [uσ0]β̂±  = [uσ]β̂± ]]$.
                    \end{enumerate}
            \end{enumerate}
            
        \item \ruleref{\ottdruleDTArrowAppLabel}\\
            Since $[[uN]]$ cannot be a metavariable,  
            if the shape of $[[ [uσ]uN ]]$ is an arrow, 
            so is the shape of $[[uN]]$. This way, 
            $[[uN = uQ → uN1]]$.
            Then by assumption:
            \begin{enumerate}
                \item $[[Γ; Θ ⊢ uQ → uN1]]$ is free from negative metavariables;
                \item $[[Θ ⊢ uσ]]$;
                \item $[[Γ ⊢ iM]]$;
                \item $[[Γ; Φ ⊢ [uσ](uQ → uN1) ● v, args ⇒> iM]]$, 
                    \label{point:typing-completeness-arrow-app-inversion}
                    that is $[[Γ; Φ ⊢ ([uσ]uQ → [uσ]uN1) ● v, args ⇒> iM]]$,
                    and by inversion:
                    \begin{enumerate}
                        \item $[[Γ; Φ ⊢ v : iP]]$,
                            and by the induction hypothsis, 
                            $[[Γ; Φ ⊨ v : iP']]$ for some $[[iP']]$
                            such that $[[Γ ⊢ iP' ≈ iP]]$;
                        \item $[[Γ ⊢ [uσ]uQ ≥ iP]]$, 
                            which by transitivity (\cref{lemma:subtyping-transitivity}) means 
                            $[[Γ ⊢ [uσ]uQ ≥ iP']]$,
                            and then by completeness of subtyping 
                            (\cref{lemma:pos-subt-completeness}),
                            $[[ Γ; Θ ⊨ uQ ≥ iP' ⫤ SC1 ]]$, 
                            for some $[[Θ ⊢ SC1]]$, and moreover, $[[Θ ⊢ uσ : SC1]]$;
                        \item $[[Γ; Φ ⊢ [uσ]uN1 ● args ⇒> iM]]$. 
                            \label{point:completeness-arrow-app-ih}
                            Notice that the induction hypothesis is applicable to this case:
                            $[[Γ ; Θ ⊢ uN1]]$ is free from negative metavariables because
                            so is $[[uQ → uN1]]$. This way, there exist 
                            $[[uM']]$, $[[Θ']]$, and $[[SC2]]$ such that 
                            \begin{enumerate}
                                \item $[[ Γ; Φ; Θ ⊨ uN1 ● args ⇒> uM' ⫤ Θ'; SC2 ]]$
                                    and then by the soundness of typing 
                                    (i.e. the induction hypothesis), 
                                    \begin{enumerate}
                                        \item $[[Θ ⊆ Θ']]$
                                        \item $[[Γ; Θ' ⊢ uM']]$
                                    \end{enumerate}
                                \item  \label{point:new-subdst}
                                    for any $[[Θ ⊢ uσ]]$ and $[[Γ ⊢ iM]]$
                                    such that $[[Γ; Φ ⊢ [uσ]uN1 ● args ⇒> iM]]$, 
                                    there exists $[[uσ']]$ such that 
                                    \begin{enumerate}
                                        \item $[[Θ' ⊢ uσ' : SC2]]$,
                                        \item $[[Θ ⊢ uσ' ≈ uσ : dom(Θ)]]$, and 
                                        \item $[[Γ ⊢ [uσ']uM' ≈ iM]]$.
                                    \end{enumerate}
                            \end{enumerate}
                    \end{enumerate}
            \end{enumerate}

            Let us take $[[Θ ⊢ uσ]]$ and $[[iM]]$
            and construct $[[Θ' ⊢ uσ']]$ 
            by the induction hypothesis (\ref{point:new-subdst}).
            Then $[[Θ' ⊢ uσ' : SC2]]$ and $[[Θ' ⊢ uσ' : SC1]]$ 
            holds and since $[[Θ ⊢ uσ : SC1]]$ and $[[Θ ⊢ uσ' ≈ uσ : dom(Θ)]]$.
            Then by the completeness of constraint merge 
            (\cref{lemma:merge-completeness}),
            $[[Θ' ⊢ SC1 & SC2 = SC]]$ exists, $[[Θ' ⊢ SC]]$, and 
            $[[Θ' ⊢ uσ : SC]]$.

            To show the required properties, we take
            $[[uM']]$ and $[[Θ']]$ from the induction hypothesis (\ref{point:new-subdst}), 
            and $[[SC]]$ defined above. Then
            \begin{enumerate}
                \item $[[ Γ; Φ; Θ ⊨ uQ → uN1 ● v,args ⇒> uM' ⫤ Θ'; SC ]]$
                    is inferred by \ruleref{\ottdruleATArrowAppLabel}:
                    \begin{enumerate}
                        \item $[[Γ; Φ ⊨ v : iP']]$ as noted above,
                        \item $[[Γ; Θ ⊨ uQ ≥ iP' ⫤ SC1]]$ as noted above,
                        \item $[[Γ; Φ; Θ ⊨ uN1 ● args ⇒> uM' ⫤ Θ'; SC2]]$ as noted above;
                    \end{enumerate}
                \item let us take an arbitrary $[[Θ ⊢ uσ0]]$ and $[[Γ ⊢ iM0]]$
                    such that $[[Γ; Φ ⊢ [uσ0](uQ → uN1) ● v,args ⇒> iM0]]$.
                    Then by inversion
                    of $[[Γ; Φ ⊢ [uσ0]uQ → [uσ0]uN1 ● v,args ⇒> iM0]]$,
                    we have the same properties as in 
                    \ref{point:typing-completeness-arrow-app-inversion}.
                    In particular,
                    \begin{itemize}
                        \item $[[Γ; Φ ⊢ [uσ0]uN1 ● args ⇒> iM0]]$. 
                            Then by \ref{point:new-subdst}, there exists $[[uσ']]$ such that 
                            \begin{enumerate}
                                \item $[[Θ' ⊢ uσ' : SC2]]$,
                                \item $[[Θ ⊢ uσ' ≈ uσ0 : dom(Θ)]]$, and 
                                \item $[[Γ ⊢ [uσ']uM' ≈ iM0]]$.
                            \end{enumerate}
                        \item $[[Γ ⊢ [uσ0]uQ ≥ iP']]$
                            and by the completeness of subtyping 
                            (\cref{lemma:pos-subt-completeness}),
                            $[[ Θ ⊢ uσ0 : SC1 ]]$.
                    \end{itemize}
                    This way,
                    \begin{itemize}
                        \item $[[Θ ⊢ uσ' ≈ uσ0 : dom(Θ)]]$ holds as noted above;
                        \item $[[Θ' ⊢ uσ' : SC1]]$ holds because $[[Θ ⊢ uσ0 : SC1]]$ and 
                            $[[Θ ⊢ uσ' ≈ uσ0 : dom(Θ)]]$, 
                            and $[[Θ' ⊢ uσ' : SC1]]$ together with $[[Θ' ⊢ uσ' : SC2]]$  
                            implies $[[Θ' ⊢ uσ' : SC]]$ by the completeness of constraint merge 
                            (\cref{lemma:merge-completeness}); and
                        \item $[[Γ ⊢ [uσ']uM' ≈ iM0]]$ holds as noted above.
                    \end{itemize}
            \end{enumerate}

        \item \ruleref{\ottdruleDTEmptyAppLabel}\\
            By assumption: 
            \begin{enumerate}
                \item $[[Γ; Θ ⊢ uN]]$,
                \item $[[Θ ⊢ uσ]]$,
                \item $[[Γ; Φ ⊢ [uσ]uN ● · ⇒> [uσ]uN ]]$.
            \end{enumerate}
            Then we can apply the corresponding algorithmic rule
            \ruleref{\ottdruleATEmptyAppLabel} to infer
            $[[ Γ; Φ; Θ ⊨ uN ● · ⇒> uN ⫤ Θ; · ]]$.
            Let us show the required properties. 
            Let us take an arbitrary 
            $[[Θ ⊢ uσ1]]$ and $[[Γ ⊢ iM]]$
            such that $[[Γ; Φ ⊢ [uσ1]uN ● · ⇒> iM]]$. 
            Then we can take $[[uσ' = uσ1]]$:
            \begin{enumerate}
                \item $[[Θ ⊢ uσ' : ·]]$ holds vacuously,
                \item $[[Θ ⊢ uσ' ≈ uσ1 : dom(Θ)]]$ holds by reflexivity of equivalence,
                \item $[[Γ ⊢ [uσ']uN ≈ iM]]$ or equivalently, 
                    $[[Γ ⊢ [uσ]uN ≈ iM]]$ holds because 
                    $[[Γ; Φ ⊢ [uσ1]uN ● · ⇒> iM]]$ can only be inferred by 
                    \ruleref{\ottdruleDTEmptyAppLabel}, meaning 
                        $[[ [uσ1]uN =  iM ]]$.
            \end{enumerate}
        \item \ruleref{\ottdruleDTVarLabel}\\
        \item \ruleref{\ottdruleDTThunkLabel}\\
        \item \ruleref{\ottdruleDTPAnnotLabel}\\
        \item \ruleref{\ottdruleDTtLamLabel}\\
        \item \ruleref{\ottdruleDTTLamLabel}\\
        \item \ruleref{\ottdruleDTReturnLabel}\\
        \item \ruleref{\ottdruleDTVarLetLabel}\\
        \item \ruleref{\ottdruleDTAppLetAnnLabel}\\
        \item \ruleref{\ottdruleDTUnpackLabel}\\
        \item \ruleref{\ottdruleDTNAnnotLabel}\\
    \end{caseof}
\end{proof}


\printbibliography

\end{document}
