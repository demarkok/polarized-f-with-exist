\UseRawInputEncoding
% vim: ft=tex
\documentclass[a4,natbib=false]{article}
\usepackage[a4paper, total={8in, 10in}]{geometry}
\usepackage{hyperref}
\usepackage{mathpartir}


\usepackage[dvipsnames]{xcolor}
\usepackage{scalerel}


\usepackage{braket}
% \usepackage{hyperref}
\usepackage{mathpartir}

\usepackage{lscape}
\usepackage{amsmath}
\usepackage{amsthm}
\usepackage{booktabs}
\usepackage{multicol}
\usepackage{supertabular}
\usepackage[inline]{enumitem}
\usepackage{cleveref}
\usepackage{proof}
\usepackage{mathtools}

% Show the frame 
% \usepackage{showframe}

\usepackage{newtxmath}
% \usepackage{mathabx}


\usepackage{stackengine}



\usepackage{todonotes}

\usepackage{enumitem}
\usepackage{xparse}

\usepackage{../casenum}
% \usepackage{../quiver}

\usepackage{listings}
\usepackage{xspace}
\usepackage{subcaption}

% \usepackage{tikz}
\usetikzlibrary{shapes,arrows,arrows.meta,positioning}
\usetikzlibrary{shapes.multipart}

% ⫤
\usepackage{graphicx}
\makeatletter
\providecommand*{\Dashv}{%
  \mathrel{%
    \mathpalette\@Dashv\vDash
  }%
}
\newcommand*{\@Dashv}[2]{%
  \reflectbox{$\m@th#1#2$}%
}
\makeatother



\setlength{\columnsep}{1cm}

\newcommand{\niton}{\not\owns}

\newcommand{\ilyam}[1]{{\color{red} \texttt{Ilya:  #1}}}
\newcommand{\nk}[1]{{\color{purple} \texttt{Neel:  #1}}}

\newtheorem{algorithm}{Algorithm}
\newtheorem{definition}{Definition}
\newtheorem*{notation*}{Notation}
\newtheorem{theorem}{Theorem}
\newtheorem*{theorempreview}{Theorem}
\newtheorem{lemma}{Lemma}
\newtheorem{corollary}{Corollary}
\newtheorem{proposition}{Proposition}
\newtheorem{observation}{Observation}
\newtheorem{property}{Property}
\newtheorem*{assertion*}{Assertion}


\newcommand{\UB}[0]{\mathsf{UB}}
\newcommand{\NFUB}[0]{\mathsf{NFUB}}

\newcommand{\depth}[1]{\ensuremath{\mathsf{depth}(#1)}}
\newcommand{\size}[1]{\ensuremath{\mathsf{size}(#1)}}

\newcommand{\ruleref}[1]{\nameref{#1}}

\newcommand{\fexists}{F$^\pm\exists$\xspace} 
\newcommand{\etc}{F$\exists$}



\newcommand{\ie}{\text{i.e.,} }
\newcommand{\eg}{\text{e.g.,} }
\newcommand{\wrt}{w.r.t.\xspace}
\newcommand{\stt}{s.t.\xspace}
\newcommand{\aka}{a.k.a.\xspace}
\newcommand{\resp}{resp.\xspace}

\newcommand{\code}[1]{\texttt{#1}}

\newcommand{\cmark}{\ding{51}}%
\newcommand{\xmark}{\ding{55}}%

\newcommand{\CBPV}{\text{Call-By-Push-Value} }

\newcommand{\coq}{\texttt{Coq} }
\newcommand{\agda}{\texttt{Agda} }

\newcommand{\systemf}{\text{System F} }

\newcommand{\pack}{\texttt{pack} }
\newcommand{\unpack}{\texttt{unpack} }

% https://tex.stackexchange.com/questions/85033/colored-symbols/85035#85035
\providecommand*{\mathcolor}{}
\def\mathcolor#1#{\mathcoloraux{#1}}
\newcommand*{\mathcoloraux}[3]{%
  \protect\leavevmode
  \begingroup
  \color#1{#2}#3%
  \endgroup
}

\definecolor{positive}{RGB}{200, 50, 50}
\definecolor{negative}{RGB}{50, 50, 200}


\setlength\multicolsep{0pt}


% generated by Ott 0.32 from: grammar.ott rules.ott unification.ott
\newcommand{\ottdrule}[4][]{{\displaystyle\frac{\begin{array}{l}#2\end{array}}{#3}\quad\ottdrulename{#4}}}
\newcommand{\ottusedrule}[1]{\[#1\]}
\newcommand{\ottpremise}[1]{ #1 \\}
\newenvironment{ottdefnblock}[3][]{ \framebox{\mbox{#2}} \quad #3 \\[0pt]}{}
\newenvironment{ottfundefnblock}[3][]{ \framebox{\mbox{#2}} \quad #3 \\[0pt]\begin{displaymath}\begin{array}{l}}{\end{array}\end{displaymath}}
\newcommand{\ottfunclause}[2]{ #1 \equiv #2 \\}
\newcommand{\ottnt}[1]{\mathit{#1}}
\newcommand{\ottmv}[1]{\mathit{#1}}
\newcommand{\ottkw}[1]{\mathbf{#1}}
\newcommand{\ottsym}[1]{#1}
\newcommand{\ottcom}[1]{\text{#1}}
\newcommand{\ottdrulename}[1]{\textsc{#1}}
\newcommand{\ottcomplu}[5]{\overline{#1}^{\,#2\in #3 #4 #5}}
\newcommand{\ottcompu}[3]{\overline{#1}^{\,#2<#3}}
\newcommand{\ottcomp}[2]{\overline{#1}^{\,#2}}
\newcommand{\ottgrammartabular}[1]{\begin{supertabular}{llcllllll}#1\end{supertabular}}
\newcommand{\ottmetavartabular}[1]{\begin{supertabular}{ll}#1\end{supertabular}}
\newcommand{\ottrulehead}[3]{$#1$ & & $#2$ & & & \multicolumn{2}{l}{#3}}
\newcommand{\ottprodline}[6]{& & $#1$ & $#2$ & $#3 #4$ & $#5$ & $#6$}
\newcommand{\ottfirstprodline}[6]{\ottprodline{#1}{#2}{#3}{#4}{#5}{#6}}
\newcommand{\ottlongprodline}[2]{& & $#1$ & \multicolumn{4}{l}{$#2$}}
\newcommand{\ottfirstlongprodline}[2]{\ottlongprodline{#1}{#2}}
\newcommand{\ottbindspecprodline}[6]{\ottprodline{#1}{#2}{#3}{#4}{#5}{#6}}
\newcommand{\ottprodnewline}{\\}
\newcommand{\ottinterrule}{\\[5.0mm]}
\newcommand{\ottafterlastrule}{\\}

\newcommand{\appRightarrow}{ \mathcolor{OliveGreen}{\Rightarrow \hspace{-7pt} \Rightarrow} }

\newcommand{\tripprox}{\setbox0\hbox{$\approx$} \mbox{\makebox[0pt][l]{\raisebox{0.48\ht0}{$\approx$} }$\approx$} }

\newcommand{\approxRight}{ \mathrel{ \tripprox \hspace{-2.3pt}  \raisebox{0.24\ht0}{$>$} } }
\newcommand{\appBull}{ \mathcolor{OliveGreen}{\bullet} }
\newcommand{\rcolor}{blue}
\newcommand{\ccolor}{purple}

\usepackage{mathabx}
\usepackage{color}
\usepackage[dvipsnames,usenames]{xcolor}

% https://tex.stackexchange.com/questions/33401/a-version-of-colorbox-that-works-inside-math-environments
\setlength{\fboxsep}{1pt}
\newcommand{\ngbox}[1]{\mathchoice%
  {\colorbox{black!8}{$\displaystyle      \mathit{ #1 } $} }%
  {\colorbox{black!8}{$\textstyle         \mathit{ #1 } $} }%
  {\colorbox{black!8}{$\scriptstyle       \mathit{ #1 } $} }%
  {\colorbox{black!8}{$\scriptscriptstyle \mathit{ #1 } $} } }%

% https://tex.stackexchange.com/questions/85033/colored-symbols/85035#85035
\newcommand*{\mathcolor}{}
\def\mathcolor#1#{ \mathcoloraux{#1} }
\newcommand*{\mathcoloraux}[3]{%
  \protect\leavevmode
  \begingroup
    \color#1{#2}#3%
  \endgroup
}

\newcommand{\ottmetavars}{
\ottmetavartabular{
 $ \ottmv{x} ,\, \ottmv{y} $ & \ottcom{term variable} \\
 $ \ottmv{f} ,\, \ottmv{g} $ & \ottcom{constructors} \\
 $ \widehat{\alpha} ,\, \widehat{\beta} ,\, \widehat{\gamma} ,\, \widehat{\delta} $ & \ottcom{unification variable} \\
 $ \vec{x} ,\, \vec{y} ,\, \vec{z} ,\, \vec{t} $ & \ottcom{variable list} \\
 $ \ottmv{n} ,\, \ottmv{m} ,\, \ottmv{i} ,\, \ottmv{j} $ & \ottcom{index variables} \\
}}

\newcommand{\ottarn}{
\ottrulehead{n  ,\ k}{::=}{\ottcom{arity}}\ottprodnewline
\ottfirstprodline{|}{\ottsym{0}}{}{}{}{}\ottprodnewline
\ottprodline{|}{\ottsym{1}}{}{}{}{}\ottprodnewline
\ottprodline{|}{\ottsym{2}}{}{}{}{}\ottprodnewline
\ottprodline{|}{n_{{\mathrm{1}}}  \ottsym{+}  n_{{\mathrm{2}}}}{}{}{}{}\ottprodnewline
\ottprodline{|}{\ottsym{\mbox{$\mid$}}  \ottnt{vars}  \ottsym{\mbox{$\mid$}}}{}{}{}{}}

\newcommand{\ottvars}{
\ottrulehead{\ottnt{vars}}{::=}{\ottcom{variable list}}\ottprodnewline
\ottfirstprodline{|}{\vec{x}}{}{}{}{}\ottprodnewline
\ottprodline{|}{\ottmv{x_{{\mathrm{1}}}}  \ottsym{,} \, .. \, \ottsym{,}  \ottmv{x_{\ottmv{n}}}}{}{}{}{}\ottprodnewline
\ottprodline{|}{\ottnt{vars_{{\mathrm{1}}}}  \cap  \ottnt{vars_{{\mathrm{2}}}}}{}{}{}{}\ottprodnewline
\ottprodline{|}{\ottnt{vars_{{\mathrm{1}}}}  \sqcap  \ottnt{vars_{{\mathrm{2}}}}}{}{}{}{}\ottprodnewline
\ottprodline{|}{\ottcomp{\ottnt{vars_{\ottmv{i}}}}{\ottmv{i}}}{}{}{}{}\ottprodnewline
\ottprodline{|}{\ottkw{UVARGS} \, \ottnt{t}} {\textsf{M}}{}{\textsf{[F]}}{\ottcom{arguments of the unification variables of the term}}}

\newcommand{\ottt}{
\ottrulehead{\ottnt{t}  ,\ \ottnt{v}  ,\ \ottnt{w}  ,\ \ottnt{h}  ,\ \ottnt{d}}{::=}{\ottcom{terms}}\ottprodnewline
\ottfirstprodline{|}{\ottmv{x}}{}{}{}{}\ottprodnewline
\ottprodline{|}{\ottmv{x}  \ottsym{.}  \ottnt{t}}{}{\textsf{bind}\; \ottmv{x}\; \textsf{in}\; \ottnt{t}}{}{}\ottprodnewline
\ottprodline{|}{\ottnt{vars}  \ottsym{.}  \ottnt{t}}{}{}{}{}\ottprodnewline
\ottprodline{|}{\widehat{\alpha}  \ottsym{[}  \ottnt{vars}  \ottsym{]}}{}{}{}{}\ottprodnewline
\ottprodline{|}{\ottmv{f}  \ottsym{(}  \ottnt{t_{{\mathrm{1}}}}  \ottsym{,..,}  \ottnt{t_{\ottmv{n}}}  \ottsym{)}}{}{}{}{}\ottprodnewline
\ottprodline{|}{\ottsym{[}  \Theta  \ottsym{]}  \ottnt{v}} {\textsf{M}}{}{}{}\ottprodnewline
\ottprodline{|}{\ottsym{(}  \ottnt{v}  \ottsym{)}} {\textsf{S}}{}{}{}\ottprodnewline
\ottprodline{|}{\ottsym{\{}  \widehat{\alpha}_{{\mathrm{1}}}  \ottsym{[}  \ottnt{vars_{{\mathrm{1}}}}  \ottsym{]}  \ottsym{/}  \widehat{\alpha}_{{\mathrm{2}}}  \ottsym{[}  \ottnt{vars_{{\mathrm{2}}}}  \ottsym{]}  \ottsym{\}}  \ottnt{t}}{}{}{}{}}

\newcommand{\ottterminals}{
\ottrulehead{\ottnt{terminals}}{::=}{}\ottprodnewline
\ottfirstprodline{|}{ \in }{}{}{}{}\ottprodnewline
\ottprodline{|}{ \notin }{}{}{}{}\ottprodnewline
\ottprodline{|}{ \cdot }{}{}{}{}\ottprodnewline
\ottprodline{|}{ \vdash }{}{}{}{}\ottprodnewline
\ottprodline{|}{ \mathcolor{\rcolor}{\vDash} }{}{}{}{}\ottprodnewline
\ottprodline{|}{ \mathcolor{\rcolor}{\Dashv} }{}{}{}{}\ottprodnewline
\ottprodline{|}{ \mathcolor{\ccolor}{\VDash} }{}{}{}{}\ottprodnewline
\ottprodline{|}{ \mathcolor{\ccolor}{\DashV} }{}{}{}{}\ottprodnewline
\ottprodline{|}{ \neq }{}{}{}{}\ottprodnewline
\ottprodline{|}{ \appRightarrow }{}{}{}{}\ottprodnewline
\ottprodline{|}{ \appBull }{}{}{}{}\ottprodnewline
\ottprodline{|}{ \mathcolor{\rcolor}{\equiv} }{}{}{}{}\ottprodnewline
\ottprodline{|}{ \equiv_{n} }{}{}{}{}\ottprodnewline
\ottprodline{|}{ \searrow }{}{}{}{}\ottprodnewline
\ottprodline{|}{ \unlhd }{}{}{}{}\ottprodnewline
\ottprodline{|}{ \cap }{}{}{}{}\ottprodnewline
\ottprodline{|}{ \sqcap }{}{}{}{}\ottprodnewline
\ottprodline{|}{ \subseteq }{}{}{}{}\ottprodnewline
\ottprodline{|}{ \emptyset }{}{}{}{}\ottprodnewline
\ottprodline{|}{ \approxRight }{}{}{}{}}

\newcommand{\ottT}{
\ottrulehead{\Theta}{::=}{\ottcom{computational variable context}}\ottprodnewline
\ottfirstprodline{|}{\ottmv{x}}{}{}{}{\ottcom{a variable}}\ottprodnewline
\ottprodline{|}{\vec{x}} {\textsf{S}}{}{}{\ottcom{variables}}\ottprodnewline
\ottprodline{|}{\widehat{\alpha}  \ottsym{:}  n}{}{}{}{\ottcom{a unification variable}}\ottprodnewline
\ottprodline{|}{\widehat{\alpha}  \ottsym{:}  n  \ottsym{=}  \ottnt{t}}{}{}{}{\ottcom{instantiate a unification variable}}\ottprodnewline
\ottprodline{|}{\ottcomp{\Theta_{\ottmv{i}}}{\ottmv{i}}}{}{}{}{\ottcom{concatenate contexts}}\ottprodnewline
\ottprodline{|}{\cdot}{}{}{}{\ottcom{empty context}}\ottprodnewline
\ottprodline{|}{\Theta_{{\mathrm{1}}}  \ottsym{\{}  \Theta_{{\mathrm{2}}}  \ottsym{\}}} {\textsf{S}}{}{}{\ottcom{surgery}}\ottprodnewline
\ottprodline{|}{\ottsym{(}  \Theta  \ottsym{)}} {\textsf{S}}{}{}{}\ottprodnewline
\ottprodline{|}{\Theta_{{\mathrm{1}}}  \ottsym{\mbox{$\backslash{}$}}  \ottsym{(}  \widehat{\alpha}_{{\mathrm{1}}}  \ottsym{,..,}  \widehat{\alpha}_{\ottmv{n}}  \ottsym{)}} {\textsf{S}}{}{}{\ottcom{context subtraction}}\ottprodnewline
\ottprodline{|}{ \Theta' ^{\color{red}\star} } {\textsf{M}}{}{\textsf{[F]}}{\ottcom{context self-application}}}

\newcommand{\ottformula}{
\ottrulehead{\ottnt{formula}}{::=}{}\ottprodnewline
\ottfirstprodline{|}{\ottnt{judgement}}{}{}{}{}\ottprodnewline
\ottprodline{|}{\ottmv{x}  \in  \Theta}{}{}{}{\ottcom{lookup $\ottmv{x}$ in context $\Theta$}}\ottprodnewline
\ottprodline{|}{\widehat{\alpha}  \notin  \ottnt{t}}{}{}{}{}\ottprodnewline
\ottprodline{|}{\vec{x}  \subseteq  \Theta}{}{}{}{}\ottprodnewline
\ottprodline{|}{\ottkw{let} \, \Theta_{{\mathrm{1}}}  \ottsym{=}  \Theta_{{\mathrm{2}}}}{}{}{}{}\ottprodnewline
\ottprodline{|}{\ottkw{let} \, \vec{x}  \ottsym{=}  \ottnt{vars}}{}{}{}{}\ottprodnewline
\ottprodline{|}{\ottnt{vars}  \cap  \Theta  \ottsym{=}  \emptyset}{}{}{}{}\ottprodnewline
\ottprodline{|}{\ottnt{vars_{{\mathrm{1}}}}  \cap  \ottnt{vars_{{\mathrm{2}}}}  \ottsym{=}  \emptyset}{}{}{}{}\ottprodnewline
\ottprodline{|}{\ottkw{UV} \, \ottsym{(}  \ottnt{t}  \ottsym{)}  \ottsym{=}  \widehat{\alpha}_{{\mathrm{1}}}  \ottsym{[}  \ottnt{vars_{{\mathrm{1}}}}  \ottsym{]}  \ottsym{,..,}  \widehat{\alpha}_{\ottmv{n}}  \ottsym{[}  \ottnt{vars_{\ottmv{n}}}  \ottsym{]}}{}{}{}{}\ottprodnewline
\ottprodline{|}{\ottkw{ux} \, \ottsym{:}  n  \in  \Theta}{}{}{}{\ottcom{lookupof $\ottkw{ux}$ in context $\Theta$}}\ottprodnewline
\ottprodline{|}{\ottkw{ux} \, \ottsym{:}  n  \ottsym{=}  \ottnt{t}  \in  \Theta}{}{}{}{\ottcom{lookup type of $\ottkw{ux}$  instantiation in context $\Theta$}}\ottprodnewline
\ottprodline{|}{\ottnt{v}  \neq  \ottnt{w}}{}{}{}{}\ottprodnewline
\ottprodline{|}{\vec{x}  \ottsym{=}  \ottnt{vars}}{}{}{}{}\ottprodnewline
\ottprodline{|}{\ottkw{arity} \, \ottmv{f}  \ottsym{=}  \ottsym{[}  n_{{\mathrm{1}}}  \ottsym{,..,}  n_{\ottmv{n}}  \ottsym{]}}{}{}{}{}\ottprodnewline
\ottprodline{|}{\ottnt{formula_{{\mathrm{1}}}} \quad .. \quad \ottnt{formula_{\ottmv{n}}}}{}{}{}{}\ottprodnewline
\ottprodline{|}{ \cdots }{}{}{}{}}

\newcommand{\ottFoo}{
\ottrulehead{\ottnt{Foo}}{::=}{}\ottprodnewline
\ottfirstprodline{|}{ \Theta' ^{\color{red}\star} }{}{}{}{\ottcom{context self-application}}\ottprodnewline
\ottprodline{|}{\ottkw{UVARGS} \, \ottnt{t}  \ottsym{===}  \ottnt{vars}}{}{}{}{\ottcom{arguments of the unification variables of the term}}}

\newcommand{\ottAOne}{
\ottrulehead{\ottnt{A1}}{::=}{}\ottprodnewline
\ottfirstprodline{|}{\Theta_{{\mathrm{1}}}  \mathcolor{\rcolor}{\vDash}  \ottnt{v}  \mathcolor{\rcolor}{\equiv}  \ottnt{w}  \ottsym{:}  n  \mathcolor{\rcolor}{\Dashv}  \Theta_{{\mathrm{2}}}}{}{}{}{\ottcom{The unification}}}

\newcommand{\ottBOne}{
\ottrulehead{\ottnt{B1}}{::=}{}\ottprodnewline
\ottfirstprodline{|}{\Theta_{{\mathrm{1}}}  \mathcolor{\ccolor}{\VDash}  \ottnt{v}  \cap  \ottsym{[}  \ottnt{vars}  \ottsym{]}  \approxRight  \ottnt{w}  \mathcolor{\ccolor}{\DashV}  \Theta_{{\mathrm{2}}}}{}{}{}{\ottcom{The prunning phase}}\ottprodnewline
\ottprodline{|}{\Theta_{{\mathrm{1}}}  \mathcolor{\ccolor}{\VDash}  \ottnt{v}  \mathcolor{\rcolor}{\equiv}  \ottnt{w}  \mathcolor{\ccolor}{\DashV}  \Theta_{{\mathrm{2}}}}{}{}{}{\ottcom{The alternative unification}}\ottprodnewline
\ottprodline{|}{\ottnt{v} \, \ottkw{ext}}{}{}{}{\ottcom{The external term}}\ottprodnewline
\ottprodline{|}{\Theta \, \ottkw{ext}}{}{}{}{\ottcom{The external environment}}}

\newcommand{\ottjudgement}{
\ottrulehead{\ottnt{judgement}}{::=}{}\ottprodnewline
\ottfirstprodline{|}{\ottnt{A1}}{}{}{}{}\ottprodnewline
\ottprodline{|}{\ottnt{B1}}{}{}{}{}}

\newcommand{\ottuserXXsyntax}{
\ottrulehead{\ottnt{user\_syntax}}{::=}{}\ottprodnewline
\ottfirstprodline{|}{\ottmv{x}}{}{}{}{}\ottprodnewline
\ottprodline{|}{\ottmv{f}}{}{}{}{}\ottprodnewline
\ottprodline{|}{\widehat{\alpha}}{}{}{}{}\ottprodnewline
\ottprodline{|}{\vec{x}}{}{}{}{}\ottprodnewline
\ottprodline{|}{\ottmv{n}}{}{}{}{}\ottprodnewline
\ottprodline{|}{n}{}{}{}{}\ottprodnewline
\ottprodline{|}{\ottnt{vars}}{}{}{}{}\ottprodnewline
\ottprodline{|}{\ottnt{t}}{}{}{}{}\ottprodnewline
\ottprodline{|}{\ottnt{terminals}}{}{}{}{}\ottprodnewline
\ottprodline{|}{\Theta}{}{}{}{}\ottprodnewline
\ottprodline{|}{\ottnt{formula}}{}{}{}{}}

\newcommand{\ottgrammar}{\ottgrammartabular{
\ottarn\ottinterrule
\ottvars\ottinterrule
\ottt\ottinterrule
\ottterminals\ottinterrule
\ottT\ottinterrule
\ottformula\ottinterrule
\ottFoo\ottinterrule
\ottAOne\ottinterrule
\ottBOne\ottinterrule
\ottjudgement\ottinterrule
\ottuserXXsyntax\ottafterlastrule
}}

% defnss
% fundefns Foo
% fundefn simpl

\newcommand{\ottfundefnsimpl}[1]{\begin{ottfundefnblock}[#1]{$ \Theta' ^{\color{red}\star} $}{\ottcom{context self-application}}
\ottfunclause{ \cdot ^{\color{red}\star} }{\cdot}%
\ottfunclause{ \ottsym{(}  \Theta  \ottsym{,}  \ottmv{x}  \ottsym{)} ^{\color{red}\star} }{ \Theta ^{\color{red}\star}   \ottsym{,}  \ottmv{x}}%
\ottfunclause{ \ottsym{(}  \Theta  \ottsym{,}  \widehat{\alpha}  \ottsym{:}  n  \ottsym{)} ^{\color{red}\star} }{ \Theta ^{\color{red}\star}   \ottsym{,}  \widehat{\alpha}  \ottsym{:}  n}%
\ottfunclause{ \ottsym{(}  \Theta  \ottsym{,}  \widehat{\alpha}  \ottsym{:}  n  \ottsym{=}  \ottnt{t}  \ottsym{)} ^{\color{red}\star} }{ \Theta ^{\color{red}\star}   \ottsym{,}  \widehat{\alpha}  \ottsym{:}  n  \ottsym{=}  \ottsym{[}   \Theta ^{\color{red}\star}   \ottsym{]}  \ottnt{t}}%
\end{ottfundefnblock}}


% fundefn uvarargs

\newcommand{\ottfundefnuvarargs}[1]{\begin{ottfundefnblock}[#1]{$\ottkw{UVARGS} \, \ottnt{t}$}{\ottcom{arguments of the unification variables of the term}}
\end{ottfundefnblock}}


\newcommand{\ottfundefnsFoo}{
\ottfundefnsimpl{}
\ottfundefnuvarargs{}}

% defns A1
%% defn un
\newcommand{\ottdruleVXXV}[1]{\ottdrule[#1]{%
\ottpremise{\ottmv{x}  \in  \Theta}%
}{
\Theta  \mathcolor{\rcolor}{\vDash}  \ottmv{x}  \mathcolor{\rcolor}{\equiv}  \ottmv{x}  \ottsym{:}  \ottsym{0}  \mathcolor{\rcolor}{\Dashv}  \Theta}{%
{\ottdrulename{V\_V}}{}%
}}


\newcommand{\ottdruleBXXB}[1]{\ottdrule[#1]{%
\ottpremise{\Theta_{{\mathrm{1}}}  \ottsym{,}  \ottmv{x}  \mathcolor{\rcolor}{\vDash}  \ottnt{t}  \mathcolor{\rcolor}{\equiv}  \ottnt{t}  \ottsym{:}  n  \mathcolor{\rcolor}{\Dashv}  \Theta_{{\mathrm{2}}}  \ottsym{,}  \ottmv{x}}%
}{
\Theta_{{\mathrm{1}}}  \mathcolor{\rcolor}{\vDash}  \ottmv{x}  \ottsym{.}  \ottnt{t}  \mathcolor{\rcolor}{\equiv}  \ottmv{x}  \ottsym{.}  \ottnt{t}  \ottsym{:}  n  \ottsym{+}  \ottsym{1}  \mathcolor{\rcolor}{\Dashv}  \Theta_{{\mathrm{2}}}}{%
{\ottdrulename{B\_B}}{}%
}}


\newcommand{\ottdruleFXXF}[1]{\ottdrule[#1]{%
\ottpremise{\ottkw{arity} \, \ottmv{f}  \ottsym{=}  \ottsym{[}  k_{{\mathrm{1}}}  \ottsym{,..,}  k_{\ottmv{n}}  \ottsym{]}}%
\ottpremise{\Theta_{{\mathrm{0}}}  \mathcolor{\rcolor}{\vDash}  \ottnt{v_{{\mathrm{1}}}}  \mathcolor{\rcolor}{\equiv}  \ottnt{w_{{\mathrm{1}}}}  \ottsym{:}  k_{{\mathrm{1}}}  \mathcolor{\rcolor}{\Dashv}  \Theta_{{\mathrm{1}}}}%
\ottpremise{\Theta_{{\mathrm{1}}}  \mathcolor{\rcolor}{\vDash}  \ottsym{[}  \Theta_{{\mathrm{1}}}  \ottsym{]}  \ottnt{v_{{\mathrm{2}}}}  \mathcolor{\rcolor}{\equiv}  \ottsym{[}  \Theta_{{\mathrm{1}}}  \ottsym{]}  \ottnt{w_{{\mathrm{2}}}}  \ottsym{:}  k_{{\mathrm{2}}}  \mathcolor{\rcolor}{\Dashv}  \Theta_{{\mathrm{2}}}}%
\ottpremise{ \cdots }%
\ottpremise{\Theta_{{\ottmv{n}-1}}  \mathcolor{\rcolor}{\vDash}  \ottsym{[}  \Theta_{{\ottmv{n}-1}}  \ottsym{]}  \ottnt{v_{\ottmv{n}}}  \mathcolor{\rcolor}{\equiv}  \ottsym{[}  \Theta_{{\ottmv{n}-1}}  \ottsym{]}  \ottnt{w_{\ottmv{n}}}  \ottsym{:}  k_{\ottmv{n}}  \mathcolor{\rcolor}{\Dashv}  \Theta_{\ottmv{n}}}%
}{
\Theta_{{\mathrm{0}}}  \mathcolor{\rcolor}{\vDash}  \ottmv{f}  \ottsym{(}  \ottnt{v_{{\mathrm{1}}}}  \ottsym{,..,}  \ottnt{v_{\ottmv{n}}}  \ottsym{)}  \mathcolor{\rcolor}{\equiv}  \ottmv{f}  \ottsym{(}  \ottnt{w_{{\mathrm{1}}}}  \ottsym{,..,}  \ottnt{w_{\ottmv{n}}}  \ottsym{)}  \ottsym{:}  \ottsym{0}  \mathcolor{\rcolor}{\Dashv}  \Theta_{\ottmv{n}}}{%
{\ottdrulename{F\_F}}{}%
}}


\newcommand{\ottdruleUVXXV}[1]{\ottdrule[#1]{%
}{
\Theta  \ottsym{\{}  \widehat{\alpha}  \ottsym{:}  n  \ottsym{\}}  \mathcolor{\rcolor}{\vDash}  \widehat{\alpha}  \ottsym{[}  \vec{x}  \ottsym{]}  \mathcolor{\rcolor}{\equiv}  \ottmv{x_{\ottmv{i}}}  \ottsym{:}  \ottsym{0}  \mathcolor{\rcolor}{\Dashv}   \ottsym{(}  \Theta  \ottsym{\{}  \widehat{\alpha}  \ottsym{:}  n  \ottsym{=}  \vec{x}  \ottsym{.}  \ottmv{x_{\ottmv{i}}}  \ottsym{\}}  \ottsym{)} ^{\color{red}\star} }{%
{\ottdrulename{UV\_V}}{}%
}}


\newcommand{\ottdruleUVXXUV}[1]{\ottdrule[#1]{%
\ottpremise{\vec{z}  \ottsym{=}  \vec{x}  \sqcap  \vec{y}}%
}{
\Theta  \ottsym{\{}  \widehat{\alpha}  \ottsym{:}  n  \ottsym{\}}  \mathcolor{\rcolor}{\vDash}  \widehat{\alpha}  \ottsym{[}  \vec{x}  \ottsym{]}  \mathcolor{\rcolor}{\equiv}  \widehat{\alpha}  \ottsym{[}  \vec{y}  \ottsym{]}  \ottsym{:}  \ottsym{0}  \mathcolor{\rcolor}{\Dashv}   \ottsym{(}  \Theta  \ottsym{\{}  \widehat{\beta}  \ottsym{:}  \ottsym{\mbox{$\mid$}}  \vec{z}  \ottsym{\mbox{$\mid$}}  \ottsym{,}  \widehat{\alpha}  \ottsym{:}  n  \ottsym{=}  \vec{x}  \ottsym{.}  \widehat{\beta}  \ottsym{[}  \vec{z}  \ottsym{]}  \ottsym{\}}  \ottsym{)} ^{\color{red}\star} }{%
{\ottdrulename{UV\_UV}}{}%
}}


\newcommand{\ottdruleUVXXUVTwo}[1]{\ottdrule[#1]{%
\ottpremise{\vec{z}  \ottsym{=}  \vec{x}  \cap  \vec{y}}%
}{
\Theta_{{\mathrm{0}}}  \ottsym{,}  \widehat{\alpha}  \ottsym{:}  n  \ottsym{,}  \Theta_{{\mathrm{1}}}  \ottsym{,}  \widehat{\beta}  \ottsym{:}  k  \ottsym{,}  \Theta_{{\mathrm{2}}}  \mathcolor{\rcolor}{\vDash}  \widehat{\alpha}  \ottsym{[}  \vec{x}  \ottsym{]}  \mathcolor{\rcolor}{\equiv}  \widehat{\beta}  \ottsym{[}  \vec{y}  \ottsym{]}  \ottsym{:}  \ottsym{0}  \mathcolor{\rcolor}{\Dashv}   \ottsym{(}  \Theta_{{\mathrm{0}}}  \ottsym{,}  \ottsym{(}  \widehat{\gamma}  \ottsym{:}  \ottsym{\mbox{$\mid$}}  \vec{z}  \ottsym{\mbox{$\mid$}}  \ottsym{)}  \ottsym{,}  \ottsym{(}  \widehat{\alpha}  \ottsym{:}  n  \ottsym{=}  \vec{x}  \ottsym{.}  \widehat{\gamma}  \ottsym{[}  \vec{z}  \ottsym{]}  \ottsym{)}  \ottsym{,}  \Theta_{{\mathrm{1}}}  \ottsym{,}  \ottsym{(}  \widehat{\beta}  \ottsym{:}  k  \ottsym{=}  \vec{y}  \ottsym{.}  \widehat{\gamma}  \ottsym{[}  \vec{z}  \ottsym{]}  \ottsym{)}  \ottsym{,}  \Theta_{{\mathrm{2}}}  \ottsym{)} ^{\color{red}\star} }{%
{\ottdrulename{UV\_UV2}}{}%
}}


\newcommand{\ottdruleUVXXF}[1]{\ottdrule[#1]{%
\ottpremise{\widehat{\alpha}  \notin  \ottmv{f}  \ottsym{(}  \ottnt{t_{{\mathrm{1}}}}  \ottsym{,..,}  \ottnt{t_{\ottmv{m}}}  \ottsym{)}}%
\ottpremise{\ottkw{arity} \, \ottmv{f}  \ottsym{=}  \ottsym{[}  k_{{\mathrm{1}}}  \ottsym{,..,}  k_{\ottmv{n}}  \ottsym{]}}%
\ottpremise{\Theta_{{\mathrm{0}}}  \ottsym{\{}  \widehat{\beta}_{{\mathrm{1}}}  \ottsym{:}  n  \ottsym{+}  k_{{\mathrm{1}}}  \ottsym{,}  \widehat{\alpha}  \ottsym{:}  n  \ottsym{\}}  \mathcolor{\rcolor}{\vDash}  \vec{y}_{{\mathrm{1}}}  \ottsym{.}  \widehat{\beta}_{{\mathrm{1}}}  \ottsym{[}  \vec{x}  \ottsym{,}  \vec{y}_{{\mathrm{1}}}  \ottsym{]}  \mathcolor{\rcolor}{\equiv}  \ottnt{t_{{\mathrm{1}}}}  \ottsym{:}  k_{{\mathrm{1}}}  \mathcolor{\rcolor}{\Dashv}  \Theta_{{\mathrm{1}}}  \ottsym{\{}  \widehat{\alpha}  \ottsym{:}  n  \ottsym{\}}}%
\ottpremise{\Theta_{{\mathrm{1}}}  \ottsym{\{}  \widehat{\beta}_{{\mathrm{2}}}  \ottsym{:}  n  \ottsym{+}  k_{{\mathrm{2}}}  \ottsym{,}  \widehat{\alpha}  \ottsym{:}  n  \ottsym{\}}  \mathcolor{\rcolor}{\vDash}  \vec{y}_{{\mathrm{2}}}  \ottsym{.}  \widehat{\beta}_{{\mathrm{2}}}  \ottsym{[}  \vec{x}  \ottsym{,}  \vec{y}_{{\mathrm{2}}}  \ottsym{]}  \mathcolor{\rcolor}{\equiv}  \ottsym{[}  \Theta_{{\mathrm{1}}}  \ottsym{]}  \ottnt{t_{{\mathrm{2}}}}  \ottsym{:}  k_{{\mathrm{2}}}  \mathcolor{\rcolor}{\Dashv}  \Theta_{{\mathrm{2}}}  \ottsym{\{}  \widehat{\alpha}  \ottsym{:}  n  \ottsym{\}}}%
\ottpremise{ \cdots }%
\ottpremise{\Theta_{{\ottmv{n}-1}}  \ottsym{\{}  \widehat{\beta}_{\ottmv{m}}  \ottsym{:}  n  \ottsym{+}  k_{{\mathrm{2}}}  \ottsym{,}  \widehat{\alpha}  \ottsym{:}  n  \ottsym{\}}  \mathcolor{\rcolor}{\vDash}  \vec{y}_{\ottmv{m}}  \ottsym{.}  \widehat{\beta}_{\ottmv{m}}  \ottsym{[}  \vec{x}  \ottsym{,}  \vec{y}_{\ottmv{m}}  \ottsym{]}  \mathcolor{\rcolor}{\equiv}  \ottsym{[}  \Theta_{{\ottmv{m}-1}}  \ottsym{]}  \ottnt{t_{\ottmv{m}}}  \ottsym{:}  k_{\ottmv{m}}  \mathcolor{\rcolor}{\Dashv}  \Theta_{\ottmv{m}}  \ottsym{\{}  \widehat{\alpha}  \ottsym{:}  n  \ottsym{\}}}%
}{
\Theta_{{\mathrm{0}}}  \ottsym{\{}  \widehat{\alpha}  \ottsym{:}  n  \ottsym{\}}  \mathcolor{\rcolor}{\vDash}  \widehat{\alpha}  \ottsym{[}  \vec{x}  \ottsym{]}  \mathcolor{\rcolor}{\equiv}  \ottmv{f}  \ottsym{(}  \ottnt{t_{{\mathrm{1}}}}  \ottsym{,..,}  \ottnt{t_{\ottmv{m}}}  \ottsym{)}  \ottsym{:}  \ottsym{0}  \mathcolor{\rcolor}{\Dashv}   \ottsym{(}  \Theta_{\ottmv{m}}  \ottsym{\{}  \widehat{\alpha}  \ottsym{:}  n  \ottsym{=}  \vec{x}  \ottsym{.}  \ottmv{f}  \ottsym{(}  \vec{y}_{{\mathrm{1}}}  \ottsym{.}  \widehat{\beta}_{{\mathrm{1}}}  \ottsym{[}  \vec{x}  \ottsym{,}  \vec{y}_{{\mathrm{1}}}  \ottsym{]}  \ottsym{,..,}  \vec{y}_{\ottmv{n}}  \ottsym{.}  \widehat{\beta}_{\ottmv{n}}  \ottsym{[}  \vec{x}  \ottsym{,}  \vec{y}_{\ottmv{n}}  \ottsym{]}  \ottsym{)}  \ottsym{\}}  \ottsym{)} ^{\color{red}\star}   \ottsym{\mbox{$\backslash{}$}}  \ottsym{(}  \widehat{\beta}_{{\mathrm{1}}}  \ottsym{,..,}  \widehat{\beta}_{\ottmv{m}}  \ottsym{)}}{%
{\ottdrulename{UV\_F}}{}%
}}

\newcommand{\ottdefnun}[1]{\begin{ottdefnblock}[#1]{$\Theta_{{\mathrm{1}}}  \mathcolor{\rcolor}{\vDash}  \ottnt{v}  \mathcolor{\rcolor}{\equiv}  \ottnt{w}  \ottsym{:}  n  \mathcolor{\rcolor}{\Dashv}  \Theta_{{\mathrm{2}}}$}{\ottcom{The unification}}
\ottusedrule{\ottdruleVXXV{}}
\ottusedrule{\ottdruleBXXB{}}
\ottusedrule{\ottdruleFXXF{}}
\ottusedrule{\ottdruleUVXXV{}}
\ottusedrule{\ottdruleUVXXUV{}}
\ottusedrule{\ottdruleUVXXUVTwo{}}
\ottusedrule{\ottdruleUVXXF{}}
\end{ottdefnblock}}


\newcommand{\ottdefnsAOne}{
\ottdefnun{}}

% defns B1
%% defn prun
\newcommand{\ottdruleaUV}[1]{\ottdrule[#1]{%
\ottpremise{\vec{z}  \ottsym{=}  \vec{y}  \cap  \vec{x}}%
}{
\Theta  \ottsym{\{}  \,  \ottsym{\}}  \ottsym{\{}  \widehat{\beta}  \ottsym{:}  n  \ottsym{\}}  \mathcolor{\ccolor}{\VDash}  \widehat{\beta}  \ottsym{[}  \vec{y}  \ottsym{]}  \cap  \ottsym{[}  \vec{x}  \ottsym{]}  \approxRight  \widehat{\beta}'  \ottsym{[}  \vec{z}  \ottsym{]}  \mathcolor{\ccolor}{\DashV}  \Theta  \ottsym{\{}  \widehat{\beta}'  \ottsym{:}  \ottsym{\mbox{$\mid$}}  \vec{z}  \ottsym{\mbox{$\mid$}}  \ottsym{\}}  \ottsym{\{}  \widehat{\beta}  \ottsym{:}  n  \ottsym{=}  \vec{y}  \ottsym{.}  \widehat{\beta}'  \ottsym{[}  \vec{z}  \ottsym{]}  \ottsym{\}}}{%
{\ottdrulename{aUV}}{}%
}}


\newcommand{\ottdruleaF}[1]{\ottdrule[#1]{%
}{
\Theta_{{\mathrm{0}}}  \ottsym{\{}  \,  \ottsym{\}}  \mathcolor{\ccolor}{\VDash}  \ottmv{f}  \ottsym{(}  \vec{y}_{{\mathrm{1}}}  \ottsym{.}  \ottnt{v_{{\mathrm{1}}}}  \ottsym{,..,}  \vec{y}_{\ottmv{n}}  \ottsym{.}  \ottnt{v_{\ottmv{n}}}  \ottsym{)}  \cap  \ottsym{[}  \vec{x}  \ottsym{]}  \approxRight  \widehat{\beta}'  \ottsym{[}  \vec{z}  \ottsym{]}  \mathcolor{\ccolor}{\DashV}  \Theta  \ottsym{\{}  \widehat{\beta}'  \ottsym{:}  \ottsym{\mbox{$\mid$}}  \vec{z}  \ottsym{\mbox{$\mid$}}  \ottsym{\}}  \ottsym{\{}  \widehat{\beta}  \ottsym{:}  n  \ottsym{=}  \vec{y}  \ottsym{.}  \widehat{\beta}'  \ottsym{[}  \vec{z}  \ottsym{]}  \ottsym{\}}}{%
{\ottdrulename{aF}}{}%
}}

\newcommand{\ottdefnaprun}[1]{\begin{ottdefnblock}[#1]{$\Theta_{{\mathrm{1}}}  \mathcolor{\ccolor}{\VDash}  \ottnt{v}  \cap  \ottsym{[}  \ottnt{vars}  \ottsym{]}  \approxRight  \ottnt{w}  \mathcolor{\ccolor}{\DashV}  \Theta_{{\mathrm{2}}}$}{\ottcom{The prunning phase}}
\ottusedrule{\ottdruleaUV{}}
\ottusedrule{\ottdruleaF{}}
\end{ottdefnblock}}

%% defn un2
\newcommand{\ottdruleaVXXV}[1]{\ottdrule[#1]{%
\ottpremise{\ottmv{x}  \in  \Theta}%
}{
\Theta  \mathcolor{\ccolor}{\VDash}  \ottmv{x}  \mathcolor{\rcolor}{\equiv}  \ottmv{x}  \mathcolor{\ccolor}{\DashV}  \Theta}{%
{\ottdrulename{aV\_V}}{}%
}}


\newcommand{\ottdruleaFXXF}[1]{\ottdrule[#1]{%
\ottpremise{\ottkw{arity} \, \ottmv{f}  \ottsym{=}  \ottsym{[}  k_{{\mathrm{1}}}  \ottsym{,..,}  k_{\ottmv{n}}  \ottsym{]}}%
\ottpremise{\Theta_{{\mathrm{0}}}  \ottsym{,}  \vec{x}_{{\mathrm{1}}}  \mathcolor{\ccolor}{\VDash}  \ottnt{v_{{\mathrm{1}}}}  \mathcolor{\rcolor}{\equiv}  \ottnt{w_{{\mathrm{1}}}}  \mathcolor{\ccolor}{\DashV}  \Theta_{{\mathrm{1}}}  \ottsym{,}  \vec{x}_{{\mathrm{1}}}}%
\ottpremise{\Theta_{{\mathrm{1}}}  \ottsym{,}  \vec{x}_{{\mathrm{2}}}  \mathcolor{\ccolor}{\VDash}  \ottsym{[}  \Theta_{{\mathrm{1}}}  \ottsym{]}  \ottnt{v_{{\mathrm{2}}}}  \mathcolor{\rcolor}{\equiv}  \ottsym{[}  \Theta_{{\mathrm{1}}}  \ottsym{]}  \ottnt{w_{{\mathrm{2}}}}  \mathcolor{\ccolor}{\DashV}  \Theta_{{\mathrm{2}}}  \ottsym{,}  \vec{x}_{{\mathrm{2}}}}%
\ottpremise{ \cdots }%
\ottpremise{\Theta_{{\ottmv{n}-1}}  \ottsym{,}  \vec{x}_{\ottmv{n}}  \mathcolor{\ccolor}{\VDash}  \ottsym{[}  \Theta_{{\ottmv{n}-1}}  \ottsym{]}  \ottnt{v_{\ottmv{n}}}  \mathcolor{\rcolor}{\equiv}  \ottsym{[}  \Theta_{{\ottmv{n}-1}}  \ottsym{]}  \ottnt{w_{\ottmv{n}}}  \mathcolor{\ccolor}{\DashV}  \Theta_{\ottmv{n}}  \ottsym{,}  \vec{x}_{\ottmv{n}}}%
}{
\Theta_{{\mathrm{0}}}  \mathcolor{\ccolor}{\VDash}  \ottmv{f}  \ottsym{(}  \vec{x}_{{\mathrm{1}}}  \ottsym{.}  \ottnt{v_{{\mathrm{1}}}}  \ottsym{,..,}  \ottmv{x_{\ottmv{n}}}  \ottsym{.}  \ottnt{v_{\ottmv{n}}}  \ottsym{)}  \mathcolor{\rcolor}{\equiv}  \ottmv{f}  \ottsym{(}  \vec{x}_{{\mathrm{1}}}  \ottsym{.}  \ottnt{w_{{\mathrm{1}}}}  \ottsym{,..,}  \vec{x}_{\ottmv{n}}  \ottsym{.}  \ottnt{w_{\ottmv{n}}}  \ottsym{)}  \mathcolor{\ccolor}{\DashV}  \Theta_{\ottmv{n}}}{%
{\ottdrulename{aF\_F}}{}%
}}


\newcommand{\ottdruleaUVXXUV}[1]{\ottdrule[#1]{%
\ottpremise{\vec{z}  \ottsym{=}  \vec{x}  \sqcap  \vec{y}}%
}{
\Theta  \ottsym{\{}  \widehat{\alpha}  \ottsym{:}  n  \ottsym{\}}  \mathcolor{\ccolor}{\VDash}  \widehat{\alpha}  \ottsym{[}  \vec{x}  \ottsym{]}  \mathcolor{\rcolor}{\equiv}  \widehat{\alpha}  \ottsym{[}  \vec{y}  \ottsym{]}  \mathcolor{\ccolor}{\DashV}   \ottsym{(}  \Theta  \ottsym{\{}  \widehat{\beta}  \ottsym{:}  \ottsym{\mbox{$\mid$}}  \vec{z}  \ottsym{\mbox{$\mid$}}  \ottsym{,}  \widehat{\alpha}  \ottsym{:}  n  \ottsym{=}  \vec{x}  \ottsym{.}  \widehat{\beta}  \ottsym{[}  \vec{z}  \ottsym{]}  \ottsym{\}}  \ottsym{)} ^{\color{red}\star} }{%
{\ottdrulename{aUV\_UV}}{}%
}}


\newcommand{\ottdruleaUVXXF}[1]{\ottdrule[#1]{%
\ottpremise{\widehat{\alpha}  \notin  \ottnt{t}}%
\ottpremise{\Theta  \ottsym{\{}  \,  \ottsym{\}}  \mathcolor{\ccolor}{\VDash}  \ottnt{t}  \cap  \ottsym{[}  \vec{x}  \ottsym{]}  \approxRight  \ottnt{t'}  \mathcolor{\ccolor}{\DashV}  \Theta'  \ottsym{\{}  \widehat{\alpha}  \ottsym{:}  n  \ottsym{\}}}%
}{
\Theta  \mathcolor{\ccolor}{\VDash}  \widehat{\alpha}  \ottsym{[}  \vec{x}  \ottsym{]}  \mathcolor{\rcolor}{\equiv}  \ottnt{t}  \mathcolor{\ccolor}{\DashV}   \ottsym{(}  \Theta'  \ottsym{\{}  \widehat{\alpha}  \ottsym{:}  n  \ottsym{=}  \vec{x}  \ottsym{.}  \ottnt{t'}  \ottsym{\}}  \ottsym{)} ^{\color{red}\star} }{%
{\ottdrulename{aUV\_F}}{}%
}}

\newcommand{\ottdefnaunTwo}[1]{\begin{ottdefnblock}[#1]{$\Theta_{{\mathrm{1}}}  \mathcolor{\ccolor}{\VDash}  \ottnt{v}  \mathcolor{\rcolor}{\equiv}  \ottnt{w}  \mathcolor{\ccolor}{\DashV}  \Theta_{{\mathrm{2}}}$}{\ottcom{The alternative unification}}
\ottusedrule{\ottdruleaVXXV{}}
\ottusedrule{\ottdruleaFXXF{}}
\ottusedrule{\ottdruleaUVXXUV{}}
\ottusedrule{\ottdruleaUVXXF{}}
\end{ottdefnblock}}

%% defn ext
\newcommand{\ottdruleaV}[1]{\ottdrule[#1]{%
}{
\ottmv{x} \, \ottkw{ext}}{%
{\ottdrulename{aV}}{}%
}}


\newcommand{\ottdruleaUV}[1]{\ottdrule[#1]{%
}{
\widehat{\alpha}  \ottsym{[}  \vec{x}  \ottsym{]} \, \ottkw{ext}}{%
{\ottdrulename{aUV}}{}%
}}


\newcommand{\ottdruleaBind}[1]{\ottdrule[#1]{%
\ottpremise{\ottnt{t} \, \ottkw{ext}}%
}{
\ottmv{x}  \ottsym{.}  \ottnt{t} \, \ottkw{ext}}{%
{\ottdrulename{aBind}}{}%
}}


\newcommand{\ottdruleaConstr}[1]{\ottdrule[#1]{%
\ottpremise{\vec{x}_{{\mathrm{1}}}  \cap  \ottkw{UVARGS} \, \ottnt{t_{{\mathrm{1}}}}  \ottsym{=}  \emptyset \quad \ottnt{t_{{\mathrm{1}}}} \, \ottkw{ext}}%
\ottpremise{ \cdots }%
\ottpremise{\vec{x}_{\ottmv{n}}  \cap  \ottkw{UVARGS} \, \ottnt{t_{\ottmv{n}}}  \ottsym{=}  \emptyset \quad \ottnt{t_{\ottmv{n}}} \, \ottkw{ext}}%
}{
\ottmv{f}  \ottsym{(}  \vec{x}_{{\mathrm{1}}}  \ottsym{.}  \ottnt{t_{{\mathrm{1}}}}  \ottsym{,..,}  \vec{x}_{\ottmv{n}}  \ottsym{.}  \ottnt{t_{\ottmv{n}}}  \ottsym{)} \, \ottkw{ext}}{%
{\ottdrulename{aConstr}}{}%
}}

\newcommand{\ottdefnaext}[1]{\begin{ottdefnblock}[#1]{$\ottnt{v} \, \ottkw{ext}$}{\ottcom{The external term}}
\ottusedrule{\ottdruleaV{}}
\ottusedrule{\ottdruleaUV{}}
\ottusedrule{\ottdruleaBind{}}
\ottusedrule{\ottdruleaConstr{}}
\end{ottdefnblock}}

%% defn extC
\newcommand{\ottdruleaEmpty}[1]{\ottdrule[#1]{%
}{
\cdot \, \ottkw{ext}}{%
{\ottdrulename{aEmpty}}{}%
}}


\newcommand{\ottdruleaVar}[1]{\ottdrule[#1]{%
}{
\Theta  \ottsym{,}  \ottmv{x} \, \ottkw{ext}}{%
{\ottdrulename{aVar}}{}%
}}


\newcommand{\ottdruleaUVar}[1]{\ottdrule[#1]{%
}{
\Theta  \ottsym{,}  \widehat{\alpha}  \ottsym{:}  n \, \ottkw{ext}}{%
{\ottdrulename{aUVar}}{}%
}}


\newcommand{\ottdruleaUVarInst}[1]{\ottdrule[#1]{%
\ottpremise{\ottnt{t} \, \ottkw{ext}}%
}{
\Theta  \ottsym{,}  \widehat{\alpha}  \ottsym{:}  n  \ottsym{=}  \ottnt{t} \, \ottkw{ext}}{%
{\ottdrulename{aUVarInst}}{}%
}}

\newcommand{\ottdefnaextC}[1]{\begin{ottdefnblock}[#1]{$\Theta \, \ottkw{ext}$}{\ottcom{The external environment}}
\ottusedrule{\ottdruleaEmpty{}}
\ottusedrule{\ottdruleaVar{}}
\ottusedrule{\ottdruleaUVar{}}
\ottusedrule{\ottdruleaUVarInst{}}
\end{ottdefnblock}}


\newcommand{\ottdefnsBOne}{
\ottdefnaprun{}\ottdefnaunTwo{}\ottdefnaext{}\ottdefnaextC{}}

\newcommand{\ottdefnss}{
\ottfundefnsFoo
\ottdefnsAOne
\ottdefnsBOne
}

\newcommand{\ottall}{\ottmetavars\\[0pt]
\ottgrammar\\[5.0mm]
\ottdefnss}






% \renewcommand{\ottdruleEOneNVarName}[0]{ Hello \def\@currentlabelname{FOO} \phantomsection  }
% \newcommand{\ottdrulename}[1]{\textsc{#1}}

% \renewcommand{\ottdrule}[4][]{{\displaystyle\frac{\begin{array}{l}#2\end{array}}{#3}\quad\ottdrulename{#4}%
%   }\ruleLabel{#4}{#4}}

% \renewcommand{\ottdrule}[4][]{%
%   {\displaystyle\frac{\begin{array}{l}#2\end{array}}{#3}\quad\ottdrulename{#4}}%
%   \mpr@label{\textsc{#1}}%
% }
% \DeclareDocumentCommand \redrule {m m o}{%
%   \inferrule*[vcenter,left={#3:}]{}{#1 \tred #2}
%   \mpr@label{\textsc{#3}}
% }

% \renewcommand{\ottdrulename}[2][#1]{\textsc{#1} \label{}}


% \renewcommand{\ottdruleEOneNVarName}[0]{foo }

% \makeatletter
% \renewcommand{\ottdruleEOneNVar}[1]{\ottdrule[#1]{%
%   }{
%     % \protected@edef\@currentlabelname{foo}
%     \def\@currentlabelname{foo}%
%     \phantomsection
%     \label{boo}
%     \alpha ^{-}   \eqEOne   \alpha ^{-}
%   }{%
%     {\ottdruleEOneNVarName}{}%
%   } }
% \makeatother


\newcommand{\ruleref}[1]{Rule \nameref{#1}}

% ord varset in uN = varset'

\renewcommand{\ottdruleONVarInName}[0]{(Var$_{\in}^-$)}
\renewcommand{\ottdruleONVarNInName}[0]{(Var$_{\notin}^-$)}
\renewcommand{\ottdruleONUVarName}[0]{(UVar$^-$)}
\renewcommand{\ottdruleOShiftUName}[0]{($\uparrow$)}
\renewcommand{\ottdruleOArrowName}[0]{($\rightarrow$)}
\renewcommand{\ottdruleOForallName}[0]{($\forall$)}


% ord varset in uP = varset'

\renewcommand{\ottdruleOPVarInName}[0]{(Var$_{\in}^+$)}
\renewcommand{\ottdruleOPVarNInName}[0]{(Var$_{\notin}^+$)}
\renewcommand{\ottdruleOPUVarName}[0]{(UVar$^+$)}
\renewcommand{\ottdruleOShiftDName}[0]{($\downarrow$)}
\renewcommand{\ottdruleOExistsName}[0]{($\exists$)}


% nf(N) = M
\renewcommand{\ottdruleNrmNVarName}[0]{(Var$^-$)}
\renewcommand{\ottdruleNrmNUVarName}[0]{(UVar$^-$)}
\renewcommand{\ottdruleNrmShiftUName}[0]{($\uparrow$)}
\renewcommand{\ottdruleNrmArrowName}[0]{($\rightarrow$)}
\renewcommand{\ottdruleNrmForallName}[0]{($\forall$)}

% nf(P) = Q
\renewcommand{\ottdruleNrmPVarName}[0]{(Var$^+$)}
\renewcommand{\ottdruleNrmPUVarName}[0]{(UVar$^+$)}
\renewcommand{\ottdruleNrmShiftDName}[0]{($\downarrow$)}
\renewcommand{\ottdruleNrmExistsName}[0]{($\exists$)}


% N ≈ M

\renewcommand{\ottdruleEOneNVarName}[0]{(Var$^-$$^{\eqEOne}$)}
\renewcommand{\ottdruleEOneShiftUName}[0]{($\uparrow^{\eqEOne}$)}
\renewcommand{\ottdruleEOneArrowName}[0]{($\rightarrow^{\eqEOne}$)}
\renewcommand{\ottdruleEOneForallName}[0]{($\forall^{\eqEOne}$)}

% P ≈ Q
\renewcommand{\ottdruleEOnePVarName}[0]{(Var$^+$$^{\eqEOne}$)}
\renewcommand{\ottdruleEOneShiftDName}[0]{($\downarrow^{\eqEOne}$)}
\renewcommand{\ottdruleEOneExistsName}[0]{($\exists^{\eqEOne}$)}


% G ⊢ N ≤1 M

\renewcommand{\ottdruleDOneNVarName}[0]{(Var$^-$$^{\subDOne}$)}
\renewcommand{\ottdruleDOneShiftUName}[0]{($\uparrow^{\subDOne}$)}
\renewcommand{\ottdruleDOneArrowName}[0]{($\rightarrow^{\subDOne}$)}
\renewcommand{\ottdruleDOneForallName}[0]{($\forall^{\subDOne}$)}

% G ⊢ P ≥1 Q
\renewcommand{\ottdruleDOnePVarName}[0]{(Var$^+$$^{\supDOne}$)}
\renewcommand{\ottdruleDOneShiftDName}[0]{($\downarrow^{\supDOne}$)}
\renewcommand{\ottdruleDOneExistsName}[0]{($\exists^{\supDOne}$)}


% G ⊢ N ≈1 M
\renewcommand{\ottdruleDOneNDefName}[0]{($\eqDOne^{-}$)}

% G ⊢ P ≈1 Q
\renewcommand{\ottdruleDOnePDefName}[0]{($\eqDOne^{+}$)}



% G ⊨ iP1 ∨ iP2 = iQ
\renewcommand{\ottdruleLUBVarName}[0]{(Var$^{\vee}$)}
\renewcommand{\ottdruleLUBShiftName}[0]{($\downarrow^{\vee}$)}
\renewcommand{\ottdruleLUBExistsName}[0]{($\exists^{\vee}$)}
\renewcommand{\ottdruleLUBUpgradeName}[0]{(Upg)}


% G ; Θ ⊨ uN ≤ iM ⫤ us
\renewcommand{\ottdruleANVarName}[0]{(Var$^-$$^{\subA}$)}
\renewcommand{\ottdruleAShiftUName}[0]{($\uparrow^{\subA}$)}
\renewcommand{\ottdruleAArrowName}[0]{($\rightarrow^{\subA}$)}
\renewcommand{\ottdruleAForallName}[0]{($\forall^{\subA}$)}

% G ; Θ ⊨ iP ≥ uQ ⫤ us
\renewcommand{\ottdruleAPVarName}[0]{(Var$^+$$^{\supA}$)}
\renewcommand{\ottdruleAShiftDName}[0]{($\downarrow^{\supA}$)}
\renewcommand{\ottdruleAExistsName}[0]{($\exists^{\supA}$)}
\renewcommand{\ottdruleAPUVarName}[0]{(UVar$^{\supA}$)}

% Γ ⊢ usEntry1 & usEntry2 = usEntry3
\renewcommand{\ottdruleSMESupSupName}[0]{$([[≥]]\&^{+}[[≥]])$}
\renewcommand{\ottdruleSMEEqSupName}[0]{$([[≈]]\&^{+}[[≥]])$}
\renewcommand{\ottdruleSMESupEqName}[0]{$([[≥]]\&^{+}[[≈]])$}
\renewcommand{\ottdruleSMEPEqEqName}[0]{$([[≈]]\&^{+}[[≈]])$}
\renewcommand{\ottdruleSMENEqEqName}[0]{$([[≈]]\&^{-}[[≈]])$}


% Γ ⊢ usEntry1 ⇒ usEntry2
\renewcommand{\ottdruleSImpESupSupName}[0]{$([[≥]]\Rightarrow^{+}[[≥]])$}
\renewcommand{\ottdruleSImpEEqSupName}[0]{$([[≈]]\Rightarrow^{+}[[≥]])$}
\renewcommand{\ottdruleSImpEPEqEqName}[0]{$([[≈]]\Rightarrow^{+}[[≈]])$}
\renewcommand{\ottdruleSImpENEqEqName}[0]{$([[≈]]\Rightarrow^{-}[[≈]])$}

% Γ ; Θ ⊨ uN ≈u iM ⫤ us 
\renewcommand{\ottdruleUNVarName}[0]{(Var$^{-[[≈u]]}$)}
\renewcommand{\ottdruleUShiftUName}[0]{($\uparrow^{[[≈u]]}$)}
\renewcommand{\ottdruleUArrowName}[0]{($\rightarrow^{[[≈u]]}$)}
\renewcommand{\ottdruleUForallName}[0]{($\forall^{[[≈u]]}$)}
\renewcommand{\ottdruleUNUVarName}[0]{(UVar$^{-[[≈u]]}$)}

% Γ ; Θ ⊨ uP ≈u iQ ⫤ us 
\renewcommand{\ottdruleUPVarName}[0]{(Var$^{+[[≈u]]}$)}
\renewcommand{\ottdruleUShiftDName}[0]{($\downarrow^{[[≈u]]}$)}
\renewcommand{\ottdruleUExistsName}[0]{($\exists^{[[≈u]]}$)}
\renewcommand{\ottdruleUPUVarName}[0]{(UVar$^{+[[≈u]]}$)}

\begin{document}

\section{The Vanilla System}



First, we present the top-level system, which is easy to understand.

\subsection{Grammar}
\ottgrammartabular{
  \ottP\ottinterrule
  \ottN\ottinterrule
}

\subsection{Declarative Subtyping}
\ottdefnsDZero

\section{Multi-Quantified System}
\subsection{Grammar}
\ottgrammartabular{
  \ottiP\ottinterrule
  \ottiN\ottinterrule
}
\subsection{Declarative Subtyping}
\ottdefnsDOne


\subsection{Declarative Equivalence}
\ottdefnsEOne


\section{Algorithm}

\subsection{Normalization}
\subsubsection{Ordering}
\ottdefnsOrder

\subsubsection{Quantifier Normalization}
\ottdefnsNrm

% \subsection{Algorithmic Equivalence}
% \ottdefnsEOneA

\subsection{Unification}
\ottdefnsU

\subsection{Algorithmic Subtyping}
\ottdefnsA

\subsection{Unification Solution Weakening}

Unification solution is represented by a list of unification solution entries.
Each entry restricts an unification variable in two possible ways: either
stating that it must be equivalent to a certain type ($[[pua :≈ iP]]$ or $[[nua
:≈ iN]]$) or that it must be a (positive) supertype of a certain type ($[[pua :≥ iP]]$).

\begin{definition} [Matching Entries]
  We call two entries matching if they are restricting the same unification variable.
\end{definition}

Unification solutions are preordered by the weakening relation.
First, let us define the weakening on the unification solution entries.\\

\begin{definition} \hfill \\
\ottdefnSImpE{}\\
\end{definition}
Notice that  $[[Γ ⊢ usEntry1 ⇒ usEntry2]]$ means that $[[usEntry1]]$ and $[[usEntry2]]$ are matching. And 
matching is a equivalence relation on the set of unification solutions. 

Next, we lift the weakening relation to unification solutions.
Informally, $[[Θ ⊢ us1 ⇒ us2]]$ means that for any entry of $[[us2]]$,
there is a matching entry in $[[us1]]$ that is stronger than it. 

\begin{definition}
  Assuming $[[us2 : Θ]]$ and $[[us1 : Θ']] \supseteq [[Θ]]$,
  we say $[[Θ ⊢ us1 ⇒ us2]]$ iff
  $\forall [[usEntry2]] \in [[us2]]$, $\exists [[usEntry1]] \in [[us1]]$ s.t.
  \begin{enumerate}
    \item $[[usEntry1]]$ and $[[usEntry2]]$ are matching, i.e. restricting the same unification variable $[[α̂±]]$;
    \item $[[Θ(α̂±) ⊢ usEntry1 ⇒ usEntry2]]$ (where $[[Θ(α̂±)]]$ is the context corresponding to $[[α̂±]]$ in $[[Θ]]$).
  \end{enumerate}
\end{definition}


\subsection{Unification Solution Merge}

Two matching entries can be merged in the following way:
\begin{definition} \hfill \\
\ottdefnSME\\
\end{definition}
% Notice that in case of equivalence, the assigned types
% must be equal (i.e. alpha-equivalent) to be merged. This is because
% the unification algorithm assumes that every type is normalized,
% and hence, equivalence is alpha-equivalence 
% (\cref{corollary:nf-complete-wrt-subt-equiv,corollary:nf-sound-wrt-subt-equiv}).


To merge two unification solution, we merge each pair of
matching entries, and unite the results.

\begin{definition}
  $\begin{aligned}[t]
  [[us1 & us2]] &= \{ [[usEntry1 & usEntry2]] \mid [[usEntry1]] \in [[us1]],
  [[usEntry2]]  \in [[us2]], \text{s.t. } [[usEntry1]] \text{ matches with }
                                   [[usEntry2]] \}\\
                &\cup
                                     \{ [[usEntry1]] \mid [[usEntry1]] \in
                                     [[us1]], \text{ s.t. }
                                     \forall [[usEntry2]]  \in [[us2]],
                                     [[usEntry1]] \text{ does not match with }
                                                     [[usEntry2]] \}\\
        &\cup
          \{ [[usEntry2]] \mid [[usEntry2]] \in
          [[us2]], \text{ s.t. }
          \forall [[usEntry1]]  \in [[us1]],
          [[usEntry1]] \text{ does not match with }
          [[usEntry2]] \}\\
   \end{aligned}$
\end{definition}



\subsection{Least Upper Bound}
\ottdefnsLUB

\subsection{Antiunification}
\ottdefnsAU

\section{Proofs}

\subsection{Declarative Subtyping}
\begin{lemma}[Free Variable Propagation] \label{lemma:fv-propagation}
  In the judgments of negative subtyping or positive supertyping,
  free variables propagate left-to-right. For a context $[[Γ]]$,
  \begin{itemize}
    \item $-$ if $[[Γ ⊢ iN ≤ iM]]$ then $[[fv(iN)]] \subseteq [[fv(iM)]]$
    \item $+$ if $[[Γ ⊢ iP ≥ iQ]]$ then $[[fv(iP)]] \subseteq [[fv(iQ)]]$
  \end{itemize}
\end{lemma}
\begin{proof}
  Mutual induction on $[[Γ ⊢ iN ≤ iM]]$ and $[[Γ ⊢ iP ≥ iQ]]$.
  \begin{caseof}
  \item $[[G ⊢ a⁻ ≤ a⁻]]$\\
    It is self-evident that $[[{a⁻} ⊆ {a⁻}]]$.
  \item $[[G ⊢ ↑iP ≤ ↑iQ]]$
    From the inversion (and unfolding $[[G ⊢ iP ≈ iQ]]$ ), we have
    $[[G ⊢ iP ≥ iQ]]$. Then by the induction hypothesis,
    $[[fv(iP)]] \subseteq [[fv(iQ)]]$. The desired 
    inclusion holds, since $[[fv(↑iP)]] = [[fv(iP)]]$ and
    $[[fv(↑iQ)]] = [[fv(iQ)]]$.
  \item $[[G ⊢ iP → iN ≤ iQ → iM]]$
    The induction hypothesis applied to the premises gives:
    $[[fv(iP)]] \subseteq [[fv(iQ)]]$ and
    $[[fv(iN)]] \subseteq [[fv(iM)]]$.
    Then $[[fv(iP → iN)]] = [[fv(iP) ∪ fv(iN)]] \subseteq
    [[fv(iQ) ∪ fv(iM)]] = [[fv(iQ → iM)]]$.

  \item $[[G ⊢ ∀pas.iN ≤ ∀pbs.iM]]$\\
    $
    \begin{aligned}[t]
      [[fv ∀pas.iN ]] &\subseteq [[fv ([iPs/pas] iN) ]] ~\setminus~ [[{pbs}]] 
                      &&   \text{here $[[{pbs}]]$ is excluded by the premise $[[fv iN ∩ {pbs} = ∅]]$}\\
                      &\subseteq [[fv iM]] ~\setminus~ [[{pbs}]]
                      &&   \text{by the induction hypothesis, } [[fv ([iPs/pas] iN) ]] \subseteq [[fv iM]] \\
                      &\subseteq [[fv ∀pbs.iM]]
    \end{aligned}
    $
  \item The positive cases are symmetric.
  \end{caseof}
\end{proof}

\begin{corollary}[Free Variables of mutual subtypes] \label{corollary:fv-mut-sub}
  \hfill
  \begin{itemize}
    \item [$-$] If $[[Γ ⊢ iN ≈ iM]]$ then $[[fv iN]] = [[fv iM]]$, 
    \item [$+$] If $[[Γ ⊢ iP ≈ iQ]]$ then $[[fv iP]] = [[fv iQ]]$
  \end{itemize}
\end{corollary}

\begin{lemma}[Decomposition of quantifier rules]
  \label{lemma:quant-rule-decomposition}
  Assuming that $[[pas]]$, $[[pbs]]$, $[[nas]]$, and $[[nas]]$ are disjoint from $[[Γ]]$,
  \begin{itemize}
    \item [$-_{R}$] $[[Γ ⊢ iN ≤ ∀pbs.iM]]$ holds if and only if $[[Γ, pbs ⊢ iN ≤ iM]]$;
    \item [$+_{R}$] $[[Γ ⊢ iP ≥ ∃nbs.iQ]]$ holds if and only if $[[Γ, nbs ⊢ iP ≥ iQ]]$;
    \item [$-_{L}$] suppose $[[iM]] \neq [[∀]]\dots$
      then $[[Γ ⊢ ∀pas.iN ≤ iM]]$ holds if and only if $[[Γ ⊢ [iPs/pas]iN ≤ iM]]$
      for some $[[Γ ⊢ iPs]]$;
    \item [$+_{L}$] suppose $[[iQ]] \neq [[∃]]\dots$
      then $[[Γ ⊢ ∃nas.iP ≥ iQ]]$ holds if and only if $[[Γ ⊢ [iNs/nas]iP ≥ iQ]]$
      for some $[[Γ ⊢ iNs]]$.
  \end{itemize}
\end{lemma}
\begin{proof}
  \hfill
  \begin{itemize}
    \item [$-_{R}$] Let us prove both directions. 
      \begin{itemize}
        \item [$\Rightarrow$] Let us assume $[[Γ ⊢ iN ≤ ∀pbs.iM]]$.
          $[[Γ ⊢ iN ≤ ∀pbs.iM]]$.
          Let us decompose $[[iM]]$ as $[[∀pbs'.iM']]$ where $[[iM']]$ does not start with $[[∀]]$, 
          and decompose $[[iN]]$ as $[[∀pas.iN']]$ where $[[iN']]$ does not start with $[[∀]]$.
          If $[[pbs]]$ is empty, then $[[Γ, pbs ⊢ iN ≤ iM]]$ holds by assumption.
          Otherwise, $[[Γ ⊢ ∀pas.iN' ≤ ∀pbs.∀pbs'.iM]]$ is inferred by
          \ruleref{\ottdruleDOneForallLabel}, and by inversion:
          $[[Γ,pbs,pbs' ⊢ [iPs/pas]iN' ≤ iM']]$ for some $[[Γ,pbs,pbs' ⊢ iPs]]$.
          Then again by \ruleref{\ottdruleDOneForallLabel} with the same $[[iPs]]$,
          $[[Γ,pbs ⊢ ∀pas.iN' ≤ ∀pbs'.iM']]$, that is $[[Γ,pbs ⊢ iN ≤ iM]]$.
        \item [$\Leftarrow$] let us assume $[[Γ, pbs ⊢ iN ≤ iM]]$, and let us decompose 
          $[[iN]]$ as $[[∀pas.iN']]$ where $[[iN']]$ does not start with $[[∀]]$, 
          and $[[iM]]$ as $[[∀pbs'.iM']]$ where $[[iM']]$ does not start with $[[∀]]$.
          if $[[pas]]$ and $[[pbs']]$ are empty then $[[Γ, pbs ⊢ iN ≤ iM]]$
          is turned into $[[Γ ⊢ iN ≤ ∀pbs.iM]]$ by \ruleref{\ottdruleDOneForallLabel}.
          Otherwise, $[[Γ, pbs ⊢ ∀pas.iN' ≤ ∀pbs'.iM']]$ is inferred by
          \ruleref{\ottdruleDOneForallLabel}, that is $[[Γ, pbs, pbs' ⊢ [iPs/pas]iN' ≤ iM']]$
          for some $[[Γ, pbs, pbs' ⊢ iPs]]$.
          Then by \ruleref{\ottdruleDOneForallLabel} again,
          $[[Γ ⊢ ∀pas.iN' ≤ ∀pbs,pbs'.iM']]$, in other words, $[[Γ ⊢ ∀pas.iN' ≤ ∀pbs.∀pbs'.iM']]$, 
          that is $[[Γ ⊢ iN ≤ ∀pbs.iM]]$.
          
      \end{itemize}
    \item [$-_{L}$] Suppose $[[iM]] \neq [[∀]]\dots$. Let us prove both directions.
      \begin{itemize}
        \item [$\Rightarrow$] Let us assume $[[Γ ⊢ ∀pas.iN ≤ iM]]$.
          then if $[[pas = ·]]$, $[[Γ ⊢ iN ≤ iM]]$ holds immediately.
          Otherwise, let us decompose  $iN$ as $[[∀pas'.iN']]$ where 
          $[[iN']]$ does not start with $[[∀]]$.
          Then $[[Γ ⊢ ∀pas.∀pas'.iN' ≤ iM']]$ is inferred by
          \ruleref{\ottdruleDOneForallLabel},
          and by inversion, 
          there exist $[[Γ ⊢ iPs,iPs']]$ 
          such that $[[Γ ⊢ [iPs/pas][iPs'/pas']iN' ≤ iM']]$ 
          (the decomposition of substitutions is possible since $[[{pas} ∩ {Γ} = ∅]]$).
          Then by \ruleref{\ottdruleDOneForallLabel} again,
          $[[Γ ⊢ ∀pas'.[iPs'/pas']iN' ≤ iM']]$ (notice that $[[ [iPs'/pas']iN' ]]$ cannot
          start with $[[∀]]$).
        \item [$\Leftarrow$] Let us assume 
          $[[Γ ⊢ [iPs/pas]iN ≤ iM]]$ for some $[[Γ ⊢ iPs]]$.
          let us decompose $iN$ as $[[∀pas'.iN']]$ where $[[iN']]$ does not start with $[[∀]]$.
          Then $[[Γ ⊢ [iPs/pas]∀pas'.iN' ≤ iM']]$ or, equivalently,
          $[[Γ ⊢ ∀pas'.[iPs/pas]iN' ≤ iM']]$ is inferred by \ruleref{\ottdruleDOneForallLabel}
          (notice that $[[ [iPs/pas]iN' ]]$ cannot start with $[[∀]]$).
          By inversion, there exist $[[Γ ⊢ iPs']]$ such that 
          $[[Γ ⊢ [iPs'/pas'][iPs/pas]iN' ≤ iM']]$. Since $[[pas']]$ is disjoint
          from the free variables of $[[iPs]]$ and from $[[pas]]$, the composition of 
          $[[iPs'/pas']]$ and $[[iPs/pas]]$ can be joined into a single substitution
          well-formed in $[[Γ]]$. Then by \ruleref{\ottdruleDOneForallLabel} again,
          $[[Γ ⊢ ∀pas.iN ≤ iM]]$.
      \end{itemize}
      \item [$+$] The positive cases are proved symmetrically.
  \end{itemize}
\end{proof}

\begin{corollary}[Redundant quantifier elimination]
  \label{corollary:red-quant-elim}
  \hfill
  \begin{itemize}
    \item [$-_{L}$] Suppose that $[[ {pas} ∩ fv(iN) = ∅]]$ then 
      $[[Γ ⊢ ∀pas.iN ≤ iM]]$ holds if and only if $[[Γ ⊢ iN ≤ iM]]$;
    \item [$-_{R}$] Suppose that $[[ {pas} ∩ fv(iM) = ∅]]$ then 
      $[[Γ ⊢ iN ≤ ∀pas.iM]]$ holds if and only if $[[Γ ⊢ iN ≤ iM]]$;
    \item [$+_{L}$] Suppose that $[[ {nas} ∩ fv(iP) = ∅]]$ then
      $[[Γ ⊢ ∃nas.iP ≥ iQ]]$ holds if and only if $[[Γ ⊢ iP ≥ iQ]]$.
    \item [$+_{R}$] Suppose that $[[ {nas} ∩ fv(iQ) = ∅]]$ then 
      $[[Γ ⊢ iP ≥ ∃nas.iQ]]$ holds if and only if $[[Γ ⊢ iP ≥ iQ]]$.
  \end{itemize}
\end{corollary}
\begin{proof}
  \begin{itemize}
    \item [$-_{R}$] Suppose that $[[ {pas} ∩ fv(iM) = ∅]]$ then 
      by \cref{lemma:quant-rule-decomposition},
      $[[Γ ⊢ iN ≤ ∀pas.iM]]$ 
      is equivalent to $[[Γ, pas ⊢ iN ≤ iM]]$,
      By \label{lemma:wf-ctxt-equiv},
      since $[[{pas} ∩ fv(iN) = ∅]]$ and $[[{pas} ∩ fv(iM) = ∅]]$,
      $[[Γ, pas ⊢ iN ≤ iM]]$ is equivalent to $[[Γ ⊢ iN ≤ iM]]$.

    \item [$-_{L}$] Suppose that $[[ {pas} ∩ fv(iN) = ∅]]$.
      Let us decompose $[[iM]]$ as $[[∀pbs.iM']]$ 
      where $[[iM']]$ does not start with $[[∀]]$.
      By \cref{lemma:quant-rule-decomposition},
      $[[Γ ⊢ ∀pas.iN ≤ ∀pbs.iM']]$ is equivalent to
      $[[Γ,pbs ⊢ ∀pas.iN ≤ iM']]$, 
      which is equivalent to 
      existence of $[[Γ,pbs ⊢ iPs]]$ such that 
      $[[Γ,pbs ⊢ [iPs/pas]iN ≤ iM']]$.
      Since $[[ [iPs/pas]iN  = iN]]$, the latter is equivalent to 
      $[[Γ,pbs ⊢ iN ≤ iM']]$,
      which is equivalent to $[[Γ ⊢ iN ≤ ∀pbs.iM']]$.
      $[[Γ,pbs ⊢ iPs]]$ can be chosen arbitrary, for example, $[[iPsi]] = [[∃α⁻.↓α⁻]]$.
    \item [$+$] The positive cases are proved symmetrically.
  \end{itemize}
\end{proof}

\begin{lemma}[Subtypes and supertypes of a variable]
  \label{lemma:var-subt}
  Assuming $[[Γ ⊢  α⁻]]$, $[[Γ ⊢ α⁺]]$, $[[Γ ⊢ iN]]$, and $[[Γ ⊢ iP]]$,
  \begin{itemize}
  \item[$+$] if $[[Γ ⊢ iP ≥ ∃nas.α⁺]]$ or $[[Γ ⊢ ∃nas.α⁺ ≥ iP ]]$ then $[[iP]] = [[∃nbs.α⁺]]$ (for some potentially empty $[[nbs]]$)
  \item[$-$] if $[[Γ ⊢ iN ≤ ∀pas.α⁻]]$ or $[[Γ ⊢ ∀pas.α⁻ ≤ iN ]]$ then $[[iN]] = [[∀pbs.α⁻]]$ (for some potentially empty $[[pbs]]$)
  \end{itemize}
\end{lemma}
\begin{proof}
  We prove by induction on the tree
  inferring $[[Γ ⊢ iP ≥ ∃nas.α⁺]]$ or $[[Γ ⊢ ∃nas.α⁺ ≥ iP ]]$ or
  or $[[Γ ⊢ iN ≤ ∀pas.α⁻]]$ or $[[Γ ⊢ ∀pas.α⁻ ≤ iN ]]$.

  Let us consider which one of these judgments is inferred.
  \begin{caseof}
  \item $[[Γ ⊢ iP ≥ ∃nas.α⁺]]$\\
    If the size of the inference tree is $1$ then the only rule that can infer
    it is \ruleref{\ottdruleDOnePVarLabel}, which
    implies that $[[nas]]$ is empty and $[[iP = α⁺]]$.

    If the size of the inference tree is $>1$ then the last rule inferring
    it must be \ruleref{\ottdruleDOneExistsLabel}. By inverting this rule,
    $[[iP = ∃nbs.iP']]$ where $[[iP']]$ does not start with $\exists$ and
    $[[Γ, nas ⊢ [iNs/nbs] iP' ≥ α⁺]]$ for some $[[G, nas ⊢ iNi]]$.

    By the induction hypothesis, $[[ [iNs/nbs] iP' = ∃ncs.α⁺]]$.
    What shape can $[[iP']]$ have?
    As mentioned, it does not start with $\exists$, and it cannot start with
    $\uparrow$ (otherwise, $[[ [iNs/nas] iP' ]]$ would also
    start with $\uparrow$ and would not be equal to $[[∃nbs.α⁺]]$).
    This way, $[[iP']]$ is a \emph{positive} variable. 
    As such, $[[ [iNs/nas] iP' = iP']]$,
    and then $[[iP' = ∃ncs.α⁺]]$ meaning that $[[ncs]]$ is empty and $[[iP' = α⁺]]$.
    This way, $[[iP]] = [[∃nbs.iP']] = [[∃nbs.α⁺]]$, as required.

  \item $[[Γ ⊢ ∃nas.α⁺ ≥ iP]]$\\
    If the size of the inference tree is $1$ then the only rule that can infer
    it is \ruleref{\ottdruleDOnePVarLabel}, which
    implies that $[[nas]]$ is empty and $[[iP = α⁺]]$.

    If the size of the inference tree is $>1$ then the last rule inferring
    it must be \ruleref{\ottdruleDOneExistsLabel}. By inverting this rule,
    $[[iP = ∃nbs.iQ]]$ where $[[G, nbs ⊢ [iNs/nas]α⁺ ≥ iQ]]$ and $[[iQ]]$ 
    does not start with $\exists$.
    Notice that since $[[α⁺]]$ is positive, $[[ [iNs/nas]α⁺ = α⁺]]$, 
    i.e. $[[G, nbs ⊢ α⁺ ≥ iQ]]$.

    By the induction hypothesis, $[[iQ = ∃nbs'.α⁺]]$,
    and since $[[iQ]]$ does not start with $\exists$, $[[nbs']]$ is empty
    This way, $[[iP]] = [[∃nbs.iQ]] = [[∃nbs.α⁺]]$, as required.

  \item The negative cases ($[[Γ ⊢ iN ≤ ∀pas.α⁻]]$ and $[[Γ ⊢ ∀pas.α⁻ ≤ iN ]]$)
    are proved analogously.
  \end{caseof}
\end{proof}

\begin{corollary}[Variables have no proper subtypes and supertypes]
  \label{corollary:vars-no-proper-subtypes}
  Assuming that all mentioned types are well-formed in $[[Γ]]$,
  \begin{align*}
    [[Γ ⊢ iP ≥ α⁺]] ~ &\iff ~ [[iP = ∃nbs.α⁺]]  ~ \iff ~ [[Γ ⊢ iP ≈ α⁺]] ~ \iff ~ [[iP ≈ α⁺]]\\
    [[Γ ⊢ α⁺≥ iP]]  ~ &\iff ~ [[iP = ∃nbs.α⁺]]  ~ \iff ~ [[Γ ⊢ iP ≈ α⁺]] ~ \iff ~ [[iP ≈ α⁺]]\\
    [[Γ ⊢ iN ≤ α⁻]] ~ &\iff ~ [[iN = ∀pbs.α⁻]]  ~ \iff ~ [[Γ ⊢ iN ≈ α⁻]] ~ \iff ~ [[iN ≈ α⁻]]\\
    [[Γ ⊢ α⁻ ≤ iN]] ~ &\iff ~ [[iN = ∀pbs.α⁻]]  ~ \iff ~ [[Γ ⊢ iN ≈ α⁻]] ~ \iff ~ [[iN ≈ α⁻]]\\
  \end{align*}
\end{corollary}
\begin{proof}
  Notice that $[[Γ ⊢ ∃nbs.α⁺ ≈ α⁺]]$ and $[[∃nbs.α⁺ ≈ α⁺]]$ and apply
  \cref{lemma:var-subt}.
\end{proof}

\begin{lemma}[Reflexivity of subtyping] \label{lemma:subtyping-reflexivity}
  Assuming all the types are well-formed in $[[Γ]]$,
  \begin{itemize}
    \item [$-$] $[[Γ ⊢ iN ≤ iN]]$
    \item [$+$] $[[Γ ⊢ iP ≥ iP]]$
  \end{itemize}
\end{lemma}
\begin{proof}
  Let us prove it by the size of $[[iN]]$ and mutually, $[[iP]]$.
  \begin{caseof}
    \item $[[iN]] = [[α⁻]]$\\
      Then $[[Γ ⊢ α⁻ ≤ α⁻]]$ is inferred immediately by \ruleref{\ottdruleDOneNVarLabel}.
    \item $[[iN]] = [[∀pas.iN']]$ where $[[pas]]$ is not empty\\
      First, we rename $[[pas]]$ to fresh $[[pbs]]$ in $[[∀pas.iN']]$ to avoid
      name clashes: $[[∀pas.iN']] = [[∀pbs.[pas/pbs]iN']]$.
      Then to infer $[[Γ ⊢ ∀pas.iN' ≤ ∀pbs.[pas/pbs]iN']]$ we can apply 
      \ruleref{\ottdruleDOneForallLabel}, instantiating $[[pas]]$ with $[[pbs]]$:
      \begin{itemize}
        \item $[[fv iN ∩ {pbs} = ∅ ]]$ by choice of $[[pbs]]$,
        \item $[[G, pbs ⊢ pbi]]$,
        \item $[[G, pbs ⊢ [pbs/pas] iN' ≤ [pbs/pas] iN']]$ by the induction hypothesis,
        since the size of $[[ [pbs/pas]iN' ]]$ is equal to the size of $[[iN']]$,
        which is smaller than the size of $[[iN]] = [[∀pas.iN']]$.
      \end{itemize}
    \item $[[iN]] = [[iP → iM]]$\\
      Then $[[Γ ⊢ iP → iM ≤ iP → iM]]$ is inferred by \ruleref{\ottdruleDOneArrowLabel},
      since $[[Γ ⊢ iP ≥ iP]]$ and $[[Γ ⊢ iM ≤ iM]]$ hold the induction hypothesis. 
    \item $[[iN]] = [[↑iP]]$\\
      Then $[[Γ ⊢ ↑iP ≤ ↑iP]]$ is inferred by \ruleref{\ottdruleDOneShiftULabel},
      since $[[Γ ⊢ iP ≥ iP]]$ holds by the induction hypothesis.
    \item The positive cases are symmetric to the negative ones.
  \end{caseof}
\end{proof}

\begin{lemma}[Substitution preserves subtyipng]
  \label{lemma:subst-pres-subt}
  Suppose that all mentioned types are well-formed in $[[Γ1]]$,
  and $[[σ]]$ is a substitution $[[Γ2 ⊢ σ : Γ1]]$.
  \begin{itemize}
    \item $-$ If $[[Γ1 ⊢ iN ≤ iM]]$ then $[[Γ2 ⊢ [σ]iN ≤ [σ]iM]]$
    \item $+$ If $[[Γ1 ⊢ iP ≥ iQ]]$ then $[[Γ2 ⊢ [σ]iP ≥ [σ]iQ]]$
  \end{itemize}
\end{lemma}
\begin{proof}
  We prove it by induction on the size of the derivation of $[[Γ1 ⊢ iN ≤ iM]]$
  and mutually, $[[Γ1 ⊢ iP ≥ iQ]]$. Let us consider the last rule 
  used in the derivation:
  \begin{caseof}
    \item \ruleref{\ottdruleDOneNVarLabel}. Then by inversion, 
      $[[iN = α⁻]]$ and $[[iM = α⁻]]$. By reflexivity of subtyping
      (\cref{lemma:subtyping-reflexivity}),
      we have $[[Γ2 ⊢ [σ]α⁻ ≤ [σ]α⁻]]$, i.e. $[[Γ2 ⊢ [σ]iN ≤ [σ]iM]]$,
      as required.
    \item  \ruleref{\ottdruleDOneForallLabel}. Then by inversion,
      $[[iN = ∀pas.iN']]$, $[[iM = ∀pbs.iM']]$, where $[[pas]]$ or $[[pbs]]$ is not empty.
      Moreover, $[[Γ1, pbs ⊢ [iPs/pas]iN' ≤ iM']]$ for some $[[Γ1, pbs ⊢ iPs]]$, and 
      $[[fv iN ∩ {pbs} = ∅ ]]$.

      Notice that since the derivation of $[[Γ1, pbs ⊢ [iPs/pas]iN' ≤ iM']]$ is
      a subderivation of the derivation of $[[Γ ⊢ iN ≤ iM]]$, its size is smaller, 
      and hence, the induction hypothesis applies
      ($[[Γ1, pbs ⊢ σ : Γ1, pbs]]$ by  \label{lemma:subst-domain-weakening})
      :
      $[[Γ2, pbs ⊢ [σ][iPs/pas]iN' ≤ [σ]iM']]$.

      Notice that by convention, $[[pas]]$ and $[[pbs]]$ are fresh, and thus,  
      $[[ [σ]∀pas.iN' ]] = [[ ∀pas.[σ]iN' ]]$ and $[[ [σ]∀pbs.iM' ]] = [[ ∀pbs.[σ]iM' ]]$, 
      which means that the required $[[Γ2, Γ ⊢ [σ]∀pas.iN' ≤ [σ]∀pbs.iM']]$ is rewritten as
      $[[Γ2 , Γ ⊢ ∀pas.[σ]iN' ≤ ∀pbs.[σ]iM']]$.

      To infer it, we apply \ruleref{\ottdruleDOneForallLabel}, 
      instantiating $[[pai]]$ with $[[ [σ]iPi ]]$:
      \begin{itemize}
        \item $[[fv [σ]iN ∩ {pbs} = ∅ ]]$;
        \item $[[Γ2, Γ,pbs⊢ [σ]iPi]]$, by \cref{lemma:wf-subst} since from the inversion,
          $[[Γ1, Γ, pbs ⊢ iPi]]$;
        \item $[[Γ, pbs ⊢ [ [σ]iPs/pas ][σ]iN' ≤ [σ]iM']]$ holds
          by \cref{lemma:subst-composition}:
          Since $[[pas]]$ is fresh, it is disjoint with the domain and the codomain of $[[σ]]$
          ($[[Γ1]]$ and $[[Γ2]]$), and thus, 
          $[[ [σ][iPs/pas]iN' ]] = [[ [ σ <=< iPs/pas ][σ]iN' ]] = [[ [ [σ]iPs/pas ][σ]iN' ]]$.
          Then $[[Γ2, Γ, pbs ⊢ [σ][iPs/pas]iN' ≤ [σ]iM']]$ holds by the induction hypothesis.
      \end{itemize}

    \item \ruleref{\ottdruleDOneArrowLabel}. Then by inversion,
      $[[iN = iP → iN1]]$, $[[iM = iQ → iM1]]$, $[[Γ ⊢ iP ≥ iQ]]$, and $[[Γ ⊢ iN1 ≤ iM1]]$.
      And by the induction hypothesis, $[[Γ' ⊢ [σ]iP ≥ [σ]iQ]]$ and $[[Γ' ⊢ [σ]iN1 ≤ [σ]iM1]]$.
      Then $[[Γ' ⊢ [σ]iN ≤ [σ]iM]]$, i.e. $[[Γ' ⊢ [σ]iP → [σ]iN1 ≤ [σ]iQ → [σ]iM1]]$,
      is inferred by \ruleref{\ottdruleDOneArrowLabel}.
    \item \ruleref{\ottdruleDOneShiftULabel}. Then by inversion,
      $[[iN = ↑iP]]$, $[[iM = ↑iQ]]$, and $[[Γ ⊢ iP ≈ iQ]]$,
      which by inversion means that $[[Γ ⊢ iP ≥ iQ]]$ and $[[Γ ⊢ iQ ≥ iP]]$.
      Then the induction hypothesis applies, and we have $[[Γ' ⊢ [σ]iP ≥ [σ]iQ]]$
      and $[[Γ' ⊢ [σ]iQ ≥ [σ]iP]]$. 
      Then by sequential application of \ruleref{\ottdruleDOneNDefLabel} 
      and \ruleref{\ottdruleDOneShiftULabel} to these judgments,
      we have $[[Γ' ⊢ ↑[σ]iP ≤ ↑[σ]iQ]]$, i.e.
      $[[Γ' ⊢ [σ]iN ≤ [σ]iM]]$, as required.
    \item The positive cases are proved symmetrically.
  \end{caseof}
\end{proof}

\begin{corollary}[Substitution preserves subtyping induced equivalence]
  \label{corollary:subst-pres-equiv}
  Suppose that $[[Γ ⊢ σ : Γ1]]$. Then
    \begin{itemize}
      \item[$+$] if $[[Γ1 ⊢ iP]]$,~ $[[Γ1 ⊢ iQ]]$,~ and $[[Γ1 ⊢ iP ≈ iQ]]$ ~ 
        then $[[Γ ⊢ [σ]iP ≈ [σ]iQ]]$
      \item[$-$] if $[[Γ1 ⊢ iN]]$,~ $[[Γ1 ⊢ iM]]$,~ and $[[Γ1 ⊢ iN ≈ iM]]$ ~ 
        then $[[Γ ⊢ [σ]iN ≈ [σ]iM]]$
    \end{itemize}
\end{corollary}


\begin{lemma}[Transitivity of subtyping] 
  \label{lemma:subtyping-transitivity}
  Assuming the types are well-formed in $[[Γ]]$,
  \begin{itemize}
    \item[$-$] if $[[Γ ⊢ iN1 ≤ iN2]]$ and $[[Γ ⊢ iN2 ≤ iN3]]$ then $[[Γ ⊢ iN1 ≤ iN3]]$,
    \item[$+$] if $[[Γ ⊢ iP1 ≥ iP2]]$ and $[[Γ ⊢ iP2 ≥ iP3]]$ then $[[Γ ⊢ iP1 ≥ iP3]]$.
  \end{itemize}
\end{lemma}
\begin{proof}
  To prove it, we formulate a stronger property, 
  which will imply the required one, taking $[[σ]] = [[Γ ⊢ id : Γ]]$.

    Assuming all the types are well-formed in $[[Γ]]$,
    \begin{itemize}
      \item[$-$] if $[[Γ ⊢ iN ≤ iM1]]$, $[[Γ ⊢ iM2 ≤ iK]]$, and for 
        $[[Γ' ⊢ σ : Γ]]$, $[[ [σ]iM1 = [σ]iM2 ]]$ then $[[Γ' ⊢ [σ]iN ≤ [σ]iK]]$
      \item[$+$] if $[[Γ ⊢ iP ≥ iQ1]]$, $[[Γ ⊢ iQ2 ≥ iR]]$, and for
        $[[Γ' ⊢ σ : Γ]]$, $[[ [σ]iQ1 = [σ]iQ2 ]]$ then $[[Γ' ⊢ [σ]iP ≥ [σ]iR]]$
    \end{itemize}

  We prove it by induction on $\size{[[Γ ⊢ iN ≤ iM1]]} + \size{[[Γ ⊢ iM2 ≤ iK]]}$ and mutually, 
  on $\size{[[Γ ⊢ iP ≥ iQ1]]} + \size{[[Γ ⊢ iQ2 ≥ iR]]}$.


  First, let us consider the 3 important cases.
  \begin{caseof}
    \item Let us consider the case when $[[iM1 = ∀pbs1.α⁻]]$. 
      Then by \cref{lemma:var-subt},
       $[[Γ ⊢ iN ≤ iM1]]$ means that $[[iN = ∀pas.α⁻]]$. 
      $[[ [σ]iM1 = [σ]iM2 ]]$ means that $[[ ∀pbs1.[σ]α⁻ = [σ]iM2 ]]$.
      Applying $[[σ]]$ to both sides of $[[Γ ⊢ iM2 ≤ iK]]$ (by \cref{lemma:subst-pres-subt}),
      we obtain $[[Γ' ⊢ [σ]iM2 ≤ [σ]iK]]$, that is $[[Γ' ⊢  ∀pbs1.[σ]α⁻ ≤ [σ]iK]]$.
      Since $[[ fv([σ]α⁻) ⊆ {Γ,α⁻} ]]$, it is disjoint from $[[pas]]$ and $[[pbs1]]$,
      This way, by \cref{corollary:red-quant-elim}, 
      $[[Γ' ⊢  ∀pbs1.[σ]α⁻ ≤ [σ]iK]]$ is equivalent to 
      $[[Γ' ⊢  [σ]α⁻ ≤ [σ]iK]]$, which is equivalent to $[[Γ' ⊢  ∀pas.[σ]α⁻ ≤ [σ]iK]]$,
      that is $[[Γ' ⊢  [σ]iN ≤ [σ]iK]]$.
    \item Let us consider the case when $[[iM2 = ∀pbs2.α⁻]]$.
      This case is symmetric to the previous one. Notice that 
      \cref{lemma:var-subt,corollary:red-quant-elim} are agnostic to the 
      side on which the the quantifiers occur, and thus, 
      the proof stays the same. 
    \item Let us decompose the types, by extracting the outer quantifiers:
      \begin{itemize}
        \item $[[iN = ∀pas.iN']]$, where $[[iN']] \neq [[∀]]\dots$,
        \item $[[iM1 = ∀pbs1.iM1']]$, where $[[iM1']] \neq [[∀]]\dots$,
        \item $[[iM2 = ∀pbs2.iM2']]$, where $[[iM2']] \neq [[∀]]\dots$,
        \item $[[iK = ∀pcs.iK']]$, where $[[iK']] \neq [[∀]]\dots$.
      \end{itemize}
      and assume that at least one of $[[pas]]$, $[[pbs1]]$, $[[pbs2]]$, and $[[pcs]]$ is not empty.
      Since $[[ [σ]iM1 = [σ]iM2 ]]$, we have $[[ ∀pbs1.[σ]iM1' = ∀pbs2.[σ]iM2' ]]$,
      and since $[[iMi']]$ are not variables 
      (which was covered by the previous cases) and do not start with $\forall$,
      $[[ [σ]iMi' ]]$ do not start with $\forall$ either,
      which means $[[pbs1]] = [[pbs2]]$ and $[[ [σ]iM1' = [σ]iM2' ]]$.
      Let us rename $[[pbs1]]$ and $[[pbs2]]$ to $[[pbs]]$.
      Then $[[iM1 = ∀pbs.iM1']]$ and $[[iM2 = ∀pbs.iM2']]$.

      By \cref{lemma:quant-rule-decomposition} applied twice
      to $[[Γ ⊢ ∀pas.iN' ≤ ∀pbs.iM1']]$ and to $[[Γ ⊢ ∀pbs.iM2' ≤ ∀pcs.iK']]$,
      we have the following:
      \begin{enumerate}
        \item $[[Γ, pbs ⊢ [iPs/pas]iN' ≤ iM1']]$ for some $[[Γ, pbs ⊢ iPs]]$;
        \item $[[Γ, pcs ⊢ [iQs/pbs]iM2' ≤ iK']]$ for some $[[Γ, pcs ⊢ iQs]]$.
      \end{enumerate}
      And since at least one of 
      $[[pas]]$, $[[pbs]]$, and $[[pcs]]$ is not empty,
      either $[[Γ ⊢ iN ≤ iM1]]$ or $[[Γ ⊢ iM2 ≤ iK]]$ is inferred 
      by \ruleref{\ottdruleDOneForallLabel}, meaning that either 
      $[[Γ, pbs ⊢ [iPs/pas]iN' ≤ iM1']]$ is a proper subderivation of $[[Γ ⊢ iN ≤ iM1]]$ or
      $[[Γ, pcs ⊢ [iQs/pbs]iM2' ≤ iK']]$ is a proper subderivation of $[[Γ ⊢ iM2 ≤ iK]]$.

      Notice that we can weaken and rearrange the contexts without changing the sizes of the 
      derivations: $[[Γ, pbs, pcs ⊢ [iPs/pas]iN' ≤ iM1']]$
      and $[[Γ, pbs, pcs ⊢ [iQs/pbs]iM2' ≤ iK']]$. This way, 
      the sum of the sizes of these derivations is smaller than the sum of the sizes of
      $[[Γ ⊢ iN ≤ iM1]]$ and $[[Γ ⊢ iM2 ≤ iK]]$.
      Let us apply the induction hypothesis to these derivations, 
      with the substitution $[[ Γ', pcs ⊢ σ ○ (iQs/pbs) : Γ, pbs, pcs  ]]$
      (\cref{lemma:subst-domain-weakening}).
      To apply the induction hypothesis, it is left to show that 
      $[[ σ ○ (iQs/pbs) ]]$ unifies $[[iM1']]$ and $[[ [iQs/pbs]iM2']]$:
      $$
      \begin{aligned}[t]
        [[ [σ ○ iQs/pbs]iM1' ]] &= [[ [σ][iQs/pbs]iM1' ]]\\
                                &= [[ [ [σ]iQs/pbs ][σ]iM2' ]]
                                && \text{by \cref{lemma:subst-composition}}\\
                                &= [[ [ [σ]iQs/pbs ][σ]iM2' ]]
                                && \text{Since $[[ [σ]iM1' = [σ]iM2' ]]$}\\
                                &= [[  [σ][iQs/pbs]iM2' ]]
                                && \text{by \cref{lemma:subst-composition}}\\
                                &= [[  [σ][iQs/pbs][iQs/pbs]iM2' ]]
                                && \text{Since $[[Γ, pcs ⊢ iQs]]$, and $[[{(Γ, pcs)} ∩ {pbs} = ∅]]$ }\\
                                &= [[  [σ ○ iQs/pbs][iQs/pbs]iM2' ]]
      \end{aligned}
      $$
      This way the induction hypothesis gives us
      $[[ Γ', pcs ⊢ [σ][iQs/pbs][iPs/pas]iN' ≤  [σ][iQs/pbs]iK' ]]$,
      and since $[[Γ, pcs ⊢ iK']]$, $[[ [iQs/pbs]iK' = iK' ]]$, that is 
      $[[ Γ', pcs ⊢ [σ][iQs/pbs][iPs/pas]iN' ≤  [σ]iK' ]]$.
      Let us rewrite the substitution that we apply to $[[iN']]$:
      $$
      \begin{aligned}[t]
        [[ [σ ○ iQs/pbs ○ iPs/pas]iN' ]] &= [[ [ (σ <=< iQs/pbs) ○ σ ○ iPs/pas]iN' ]]
                                       && \text{by \cref{lemma:subst-composition}}\\
                                       &= [[ [(σ <=< iQs/pbs) ○ (σ <=< iPs/pas) ○ σ] iN' ]]
                                       && \text{by \cref{lemma:subst-composition}}\\
                                       &= [[ [(((σ <=< iQs/pbs) ○ σ) <=< iPs/pas) ○ σ] iN' ]]
                                       && \text{Since $[[fv([σ]iN') ∩ {pbs} = ∅]]$}\\
                                       &= [[ [((σ ○ iQs/pbs) <=< iPs/pas) ○ σ] iN' ]]
                                       && \text{by \cref{lemma:subst-composition}}\\
                                       &= [[ [(σ ○ iQs/pbs) <=< iPs/pas][σ] iN' ]]
      \end{aligned}
      $$
      Notice that $[[(σ ○ iQs/pbs) <=< iPs/pas]]$
      is a substitution that turns $[[pai]]$ into $[[ [σ ○ iQs/pbs]iPi ]]$, 
      where $[[ Γ',pcs ⊢ [σ ○ iQs/pbs]iPi]]$.
      This way, 
      $[[ Γ', pcs ⊢ [(σ ○ iQs/pbs) <=< iPs/pas][σ]iN' ≤  [σ]iK' ]]$
      means $[[Γ ⊢ ∀pas.[σ]iN' ≤ ∀pcs.[σ]iK']]$
      by \cref{lemma:quant-rule-decomposition}, that is
      $[[Γ ⊢ [σ]iN ≤ [σ]iK]]$, as required.
  \end{caseof}

  Now, we can assume that neither $[[Γ ⊢ iN ≤ iM1]]$ nor $[[Γ ⊢ iM2 ≤ iK]]$ 
  is inferred by \ruleref{\ottdruleDOneForallLabel}, and that neither $[[iM1]]$ nor $[[iM2]]$
  is equivalent to a variable.  Because of that, $[[ [σ]iM1 = [σ]iM2 ]]$ means that 
  $[[iM1]]$ and $[[iM2]]$ have the same outer constructor. Let us consider the shape of $[[iM1]]$.

  \begin{caseof}
    \item $[[iM1 = α⁻]]$ this case has been considered;
    \item $[[iM1 = ∀pbs.iM1']]$ this case has been considered;
    \item $[[iM1 = ↑iQ1]]$. Then as noted above, 
      $[[ [σ]iM1 = [σ]iM2 ]]$ means that $[[iM2 = ↑iQ2]]$ and $[[ [σ]iQ1 = [σ]iQ2 ]]$.
      Moreover, $[[Γ ⊢ iN ≤ ↑iQ1]]$ can only be inferred by \ruleref{\ottdruleDOneShiftULabel},
      and thus, $[[iN = ↑iP]]$, and by inversion, $[[Γ ⊢ iP ≥ iQ1]]$ and $[[Γ ⊢ iQ1 ≥ iP]]$.
      Analogously, $[[Γ ⊢ ↑iQ2 ≤ iK]]$ means that $[[iK = ↑iR]]$, $[[Γ ⊢ iQ2 ≥ iR]]$, and $[[Γ ⊢ iR ≥ iQ2]]$.

      Notice that the derivations of $[[Γ ⊢ iP ≥ iQ1]]$ and $[[Γ ⊢ iQ1 ≥ iP]]$ are proper sub-derivations of 
      $[[Γ ⊢ iN ≤ iM1]]$, and the derivations of $[[Γ ⊢ iQ2 ≥ iR]]$ and $[[Γ ⊢ iR ≥ iQ2]]$ are proper sub-derivations of
      $[[Γ ⊢ iM2 ≤ iK]]$. This way, the induction hypothesis is applicable:
      \begin{itemize}
        \item applying the induction hypothesis to $[[Γ ⊢ iP ≥ iQ1]]$ and $[[Γ ⊢ iQ2 ≥ iR]]$ 
          with $[[Γ' ⊢ σ : Γ]]$ unifying $[[iQ1]]$ and $[[iQ2]]$, we obtain $[[Γ' ⊢ [σ]iP ≥ [σ]iR]]$;
        \item applying the induction hypothesis to $[[Γ ⊢ iR ≥ iQ2]]$ and $[[Γ ⊢ iQ1 ≥ iP]]$ 
          with $[[Γ' ⊢ σ : Γ]]$ unifying $[[iQ2]]$ and $[[iQ1]]$, we obtain $[[Γ' ⊢ [σ]iR ≥ [σ]iP]]$.
      \end{itemize}
      This way, by \ruleref{\ottdruleDOneShiftULabel}, $[[Γ' ⊢ [σ]iN ≤ [σ]iK]]$, as required. 

    \item $[[iM1 = iQ1 → iM1']]$. Then as noted above, 
      $[[ [σ]iM1 = [σ]iM2 ]]$ means that $[[iM2 = iQ2 → iM2']]$, $[[ [σ]iQ1 = [σ]iQ2 ]]$, and $[[ [σ]iM1' = [σ]iM2' ]]$.
      Moreover, $[[Γ ⊢ iN ≤ iQ1 → iM1']]$ can only be inferred by \ruleref{\ottdruleDOneArrowLabel},
      and thus, $[[iN = iP → iN']]$, and by inversion, $[[Γ ⊢ iP ≥ iQ1]]$ and $[[Γ ⊢ iN' ≤ iM1']]$.
      Analogously, $[[Γ ⊢ iQ2 → iM2' ≤ iK]]$ means that $[[iK = iR → iK']]$, $[[Γ ⊢ iQ2 ≥ iR]]$, and $[[Γ ⊢ iM2' ≤ iK']]$.

      Notice that the derivations of $[[Γ ⊢ iP ≥ iQ1]]$ and $[[Γ ⊢ iN' ≤ iM1']]$ are proper sub-derivations of
      $[[Γ ⊢ iP → iN' ≤ iQ1 → iM1']]$, and the derivations of $[[Γ ⊢ iQ2 ≥ iR]]$ and $[[Γ ⊢ iM2' ≤ iK']]$ are proper sub-derivations of
      $[[Γ ⊢ iQ2 → iM2' ≤ iR → iK']]$. This way, the induction hypothesis is applicable:
      \begin{itemize}
        \item applying the induction hypothesis to $[[Γ ⊢ iP ≥ iQ1]]$ and $[[Γ ⊢ iQ2 ≥ iR]]$ 
          with $[[Γ' ⊢ σ : Γ]]$ unifying $[[iQ1]]$ and $[[iQ2]]$, we obtain $[[Γ' ⊢ [σ]iP ≥ [σ]iR]]$;
        \item applying the induction hypothesis to $[[Γ ⊢ iN' ≤ iM1']]$ and $[[Γ ⊢ iM2' ≤ iK']]$ 
          with $[[Γ' ⊢ σ : Γ]]$ unifying $[[iM1']]$ and $[[iM2']]$, we obtain $[[Γ' ⊢ [σ]iN' ≤ [σ]iK']]$.
      \end{itemize}
      This way, by \ruleref{\ottdruleDOneArrowLabel}, $[[Γ' ⊢ [σ]iP → [σ]iN' ≤ [σ]iR → [σ]iK']]$,
      that is $[[Γ' ⊢ [σ]iN ≤ [σ]iK]]$, as required.
  \end{caseof}
  After that we consider all the 
  analogous positive cases, and prove them symmetrically.
\end{proof}




\begin{corollary}[Transitivity of equivalence] \label{corollary:equivalence-transitivity}
  Assuming the types are well-formed in $[[Γ]]$,
  \begin{itemize}
    \item[$-$] if $[[Γ ⊢ iN1 ≈ iN2]]$ and $[[Γ ⊢ iN2 ≈ iN3]]$ then $[[Γ ⊢ iN1 ≈ iN3]]$,
    \item[$+$] if $[[Γ ⊢ iP1 ≈ iP2]]$ and $[[Γ ⊢ iP2 ≈ iP3]]$ then $[[Γ ⊢ iP1 ≈ iP3]]$.
  \end{itemize}
\end{corollary}











\subsection{Substitution}
\begin{lemma}[Substitution strengthening]
  \label{lemma:subst-restr-fv}
  Restricting the substitution to the free variables of the
  substitution subject does not affect the result.
  Suppose that $[[Γ2 ⊢ σ : Γ1]]$. Then
    \begin{itemize}
  \item[$+$] if $[[Γ1 ⊢ iP]]$ then $[[ [σ]iP ]] = [[ [σ|fv iP]iP ]]$,
  \item[$-$] if $[[Γ1 ⊢ iN]]$ then $[[ [σ]iN ]] = [[ [σ|fv iN]iN ]]$
  \end{itemize}
\end{lemma}
\begin{proof}
  \ilyam{todo}
\end{proof}

\begin{lemma}[]
  Suppose that $[[{Γ'} ⊆ {Γ}]]$,
  $[[σ1]]$ and $[[σ2]]$ are substitutions of signature $[[Γ ⊢ σi : Γ']]$.
  Then 
  \begin{enumerate}
    \item [$+$] for a type $[[Γ ⊢ iP]]$, if $[[Γ ⊢ [σ1]iP ≈ [σ2]iP]]$ then 
    $[[Γ ⊢ σ1 ≈ σ2 : fv iP ∩ {Γ'}]]$;
    \item [$-$] for a type $[[Γ ⊢ iN]]$, if $[[Γ ⊢ [σ1]iN ≈ [σ2]iN]]$ then
    $[[Γ ⊢ σ1 ≈ σ2 : fv iN ∩ {Γ'}]]$.
  \end{enumerate}
\end{lemma}
\begin{proof}
  Let us make an additional assumption that $[[σ1]]$, $[[σ2]]$, 
  and the mentioned types are normalized. If they are not,
  we normalize them first.
  
  Notice that the normalization preserves
  the set of free variables (\cref{lemma:fv-nf}),
  well-formedness (\cref{corollary:wf-nf}), 
  and equivalence (\cref{lemma:subt-equiv-algorithmization}), 
  and distributes over substitution (\cref{lemma:norm-subst-distr}). 
  This way, the assumed and desired properties are equivalent to their 
  normalized versions.

  We prove it by induction on the structure of $[[iP]]$ and mutually, $[[iN]]$.
  Let us consider the shape of this type.
  \begin{caseof}
    \item $[[iP]] = [[α⁺]] \in [[Γ']]$.
      Then $[[Γ ⊢ σ1 ≈ σ2 : fv iP ∩ {Γ'}]]$ means $[[Γ ⊢ σ1 ≈ σ2 : {α⁺}]]$, 
      i.e. $[[Γ ⊢ [σ1]α⁺ ≈ [σ2]α⁺ ]]$, which holds by assumption.
    \item $[[iP]] = [[α⁺]] \in [[{Γ} \ {Γ'}]]$.
      Then $[[fv iP ∩ {Γ'}]] = [[∅]]$, 
      so $[[Γ ⊢ σ1 ≈ σ2 : fv iP ∩ {Γ'}]]$ holds vacuously.
    \item $[[iP]] = [[↓iN]]$.
      Then the induction hypothesis is applicable to type $[[iN]]$:
      \begin{enumerate}
        \item $[[iN]]$ is normalized,
        \item $[[Γ ⊢ iN]]$ by inversion of $[[Γ ⊢ ↓iN]]$,
        \item $[[Γ ⊢ [σ1]iN ≈ [σ2]iN]]$ holds by inversion of 
          $[[Γ ⊢ [σ1]↓iN ≈ [σ2]↓iN]]$, i.e. $[[Γ ⊢ ↓[σ1]iN ≈ ↓[σ2]iN]]$.
      \end{enumerate}
      This way, we obtain $[[Γ ⊢ σ1 ≈ σ2 : fv iN ∩ {Γ'}]]$, 
      which implies the required equivalence since 
      $[[fv iP ∩ {Γ'}]] = [[fv ↓iN ∩ {Γ'}]] = [[fv iN ∩ {Γ'}]]$.
    \item $[[iP]] = [[∃nas.iQ]]$
      Then the induction hypothesis is applicable to type $[[iQ]]$ 
      well-formed in context $[[Γ, nas]]$:
      \begin{enumerate}
        \item $[[{Γ'} ⊆ {Γ, nas}]]$ since $[[{Γ'} ⊆ {Γ}]]$,
        \item $[[Γ, nas ⊢ σi : Γ']]$ by weakening,
        \item $[[iQ]]$ is normalized,
        \item $[[Γ, nas ⊢ iQ]]$ by inversion of $[[Γ ⊢ ∃nas.iQ]]$,
        \item Notice that $[[ [σi]∃nas.iQ ]]$ is normalized, and thus, 
          $[[ [σ1]∃nas.iQ ≈ [σ2]∃nas.iQ]]$ implies 
          $[[ [σ1]∃nas.iQ = [σ2]∃nas.iQ ]]$
          (by \cref{lemma:subt-equiv-algorithmization}).).
          This equality means $[[ [σ1]iQ = [σ2]iQ ]]$, 
          which implies $[[Γ ⊢ [σ1]iQ ≈ [σ2]iQ]]$.
      \end{enumerate}
    \item $[[iN]] = [[iP → iM]]$
  \end{caseof}
\end{proof}

\begin{lemma}[Substitutions equivalent on the metavariables]
  \label{lemma:subst-equiv-metavar}
  Suppose that $[[Γ ⊢ Θ]]$, $[[uσ1]]$ and $[[uσ2]]$ are substitutions 
  of signature $[[Θ ⊢ uσi]]$.
  Then 
  \begin{enumerate}
    \item [$+$] for a type $[[Γ; Θ ⊢ uP]]$, if $[[Γ ⊢ [uσ1]uP ≈ [uσ2]uP]]$ then
      $[[Θ ⊢ uσ1 ≈ uσ2 : uv uP]]$;
    \item [$-$] for a type $[[Γ; Θ ⊢ uN]]$, if $[[Γ ⊢ [uσ1]uN ≈ [uσ2]uN]]$ then
      $[[Θ ⊢ uσ1 ≈ uσ2 : uv uN]]$.
  \end{enumerate}
\end{lemma}
\begin{proof}
  The proof is a trivial structural induction on 
  $[[Γ; Θ ⊢ uP]]$ and mutually, on $[[Γ; Θ ⊢ uN]]$.
\end{proof}


\begin{lemma}[Substitution composition well-formedness]
  If $[[Γ'1 ⊢ σ1 : Γ1]]$ and $[[Γ'2 ⊢ σ2 : Γ2]]$,
  then $[[Γ'1, Γ'2 ⊢ σ2 ○ σ1 : Γ1, Γ2]]$.
\end{lemma}

\begin{lemma}[Substitution monadic composition well-formedness]
  \label{lemma:subst-monad-composition-wf}
  If $[[Γ'1 ⊢ σ1 : Γ1]]$ and $[[Γ'2 ⊢ σ2 : Γ2]]$,
  then $[[Γ'2 ⊢ σ2 <=< σ1 : Γ1]]$.
\end{lemma}

\begin{lemma}[Substitution composition]
  \label{lemma:subst-composition}
    If $[[Γ'1 ⊢ σ1 : Γ1]]$, $[[Γ'2 ⊢ σ2 : Γ2]]$, 
    $[[{Γ1} ∩ {Γ'2} = ∅ ]]$ and $[[ {Γ1} ∩ {Γ2} = ∅ ]]$ then 
    $[[ σ2 ○ σ1 ]] = [[ (σ2 <=< σ1) ○ σ2 ]]$.
\end{lemma}

\begin{corollary}[Substitution composition commutativity]
  \label{corollary:subst-composition-commutativity}
  If $[[Γ'1 ⊢ σ1 : Γ1]]$, $[[Γ'2 ⊢ σ2 : Γ2]]$, and
  $[[ {Γ1} ∩ {Γ2} = ∅ ]]$, 
  $[[ {Γ1} ∩ {Γ'2} = ∅ ]]$, and
  $[[ {Γ'1} ∩ {Γ2} = ∅ ]]$ then 
  $[[ σ2 ○ σ1 ]] = [[ σ1 ○ σ2 ]]$.
\end{corollary}
\begin{proof}
  by \cref{lemma:subst-composition},
    $[[ σ2 ○ σ1 ]] = [[ (σ2 <=< σ1) ○ σ2 ]]$.
    Since the codomain of $[[σ1]]$ is $[[Γ'1]]$,
    and it is disjoint with the domain of $[[σ2]]$,
    $[[σ2 <=< σ1]] = [[σ1]]$.
\end{proof}

\begin{lemma}[Substitution domain weakening]
  \label{lemma:subst-domain-weakening}
  If $[[Γ2 ⊢ σ : Γ1]]$ and $[[ {Γ2'} ⊆ {Γ} ]]$ then $[[Γ2 ⊢ σ : Γ1, Γ2']]$
\end{lemma}
\begin{proof}
  If the variable $[[α±]]$ is in $[[Γ1]]$ then $[[Γ2 ⊢ [σ]α± ]]$ by assumption.
  If the variable $[[α±]]$ is in $[[{Γ2'} \ {Γ1}]]$ then $[[ [σ]α± = α± ]] \in [[Γ2']] ⊆ [[Γ2]]$, 
  and thus, $[[Γ2 ⊢ α± ]]$.
\end{proof}


\subsection{Type well-formedness}
\begin{lemma}[Well-formedness agrees with substitution]
  \label{lemma:wf-subst}
  Suppose that $[[Γ2 ⊢ σ : Γ1]]$. Then
  \begin{itemize}
  \item[$+$] $[[Γ, Γ1 ⊢ iP]] ~\Leftrightarrow~ [[Γ, Γ2 ⊢ [σ]iP]]$
  \item[$-$] $[[Γ, Γ1 ⊢ iN]] ~\Leftrightarrow~ [[Γ, Γ2 ⊢ [σ]iN]]$
  \end{itemize}
\end{lemma}
\begin{proof}
  \ilyam{todo}
\end{proof}


\begin{corollary}
  \label{lemma:wf-subst}
  Suppose that $[[Γ2 ⊢ σ : Γ1]]$. Then
  \begin{itemize}
  \item[$+$] $[[Γ1, Γ2 ⊢ iP]] ~\Leftrightarrow~ [[Γ2 ⊢ [σ]iP]]$
  \item[$-$] $[[Γ1, Γ2 ⊢ iN]] ~\Leftrightarrow~ [[Γ2 ⊢ [σ]iN]]$
  \end{itemize}
\end{corollary}

\begin{lemma}[Equivalent Contexts]
  \label{lemma:wf-ctxt-equiv}
  In the well-formedness judgment, only used variables matter:
  \begin{itemize}
  \item[$+$] if $[[{Γ1} ∩ fv iP]] = [[{Γ2} ∩ fv iP]]$ then
    $[[Γ1 ⊢ iP]] \iff [[Γ2 ⊢ iP]]$,
  \item[$-$] if $[[{Γ1} ∩ fv iN]] = [[{Γ2} ∩ fv iN]]$ then
    $[[Γ1 ⊢ iN]] \iff [[Γ2 ⊢ iN]]$.
  \end{itemize}
\end{lemma}
\begin{proof}
  By simple mutual induction on $[[iP]]$ and $[[iQ]]$. 
\end{proof}

\begin{corollary}
  \label{lemma:mut-sub-types-wf-equiv}
  Suppose that all the types below are well-formed in $[[Γ]]$ and
  $[[{Γ'} ⊆ {Γ}]]$. Then
  \begin{itemize}
  \item[$+$] $[[Γ ⊢ iP ≈ iQ]]$ implies $[[Γ' ⊢ iP]] \iff [[Γ' ⊢ iQ]]$
  \item[$-$] $[[Γ ⊢ iN ≈ iM]]$ implies $[[Γ' ⊢ iN]] \iff [[Γ' ⊢ iM]]$
  \end{itemize}
\end{corollary}
\begin{proof}
  From \cref{lemma:wf-ctxt-equiv,corollary:fv-mut-sub}.
\end{proof}

\subsection{Overview}
\renewcommand\stackalignment{l}
\begin{tabular}{@{}lccc@{}} \toprule
  % supertypes of ... & ... are \\ 
  Algorithm                   & Soundness & Completeness & Initiality \\
  \midrule
  \addlinespace[0.7em]
  % $[[ ord varset in iN ]]$
  Ordering
                      &   \infer{[[ {ord varset in iN} ]] \equiv [[varset ∩ fv iN]]}{}{}
                                % \infer{\stackunder{hello}{world}}{}{}
                                % \begin{itemize}
                                % \item[$-$] $[[ {ord varset in uN} ]] \equiv [[varset ∩ fv uN]]$ (as sets)
                                % \item[$+$] $[[ {ord varset in uP} ]] \equiv [[varset ∩ fv uP]]$ (as sets)
                                % \end{itemize}
                            & \infer{[[ord varset in iN]] = [[ord varset in iM]]}{[[iN ≈ iM]]}{}
                            & --- \\

  \addlinespace[0.7em]
  % $[[ nf(iN) ]]$
  Normalization
                      &   \infer{[[iN ≈ nf(iN)]]}{}{}
                  & \infer{[[nf(iN)]] = [[nf(iM)]]}{[[iN ≈ iM]]}{}
                  & --- 
  \\

  \addlinespace[0.7em]
  % $[[ Γ ⊨ iP1 ∨ iP2 = iQ]]$
  Equivalence
                      & \infer
                        { 
                        [[Γ ⊢ iP ≈ iQ]]
                        }
                        {
                        [[Γ ⊢ iP]] & [[Γ ⊢ iQ]] & [[iP ≈ iQ]]
                        }
                        {}
                      & \infer{[[iP ≈ iQ]]}{[[Γ ⊢ iP ≈ iQ]]}{}
                      &  ---

  \\
  \addlinespace[0.7em]
  % $[[ upgrade Γ ⊢ iP to Δ = iQ ]]$
  Uppgrade
                      & \infer
                                     { [[iQ]] \text{ is sound}
                                      \begin{cases}
                                         [[Δ ⊢ iQ]]\\
                                         [[Γ ⊢ iQ ≥ iP]]
                                      \end{cases}
                                     }
                                     {[[upgrade Γ ⊢ iP to Δ = iQ]]}
                                     {}
                              & \infer{\exists [[iQ]] \text{ s.t. }
                                [[upgrade Γ ⊢ iP to Δ = iQ]]
                                }{\exists \text{ sound } [[iQ']]}{}
                             & \infer
                                 {[[Δ ⊢ iQ' ≥ iQ]]}
                                 {
                                 \stackon
                                 {$[[upgrade Γ ⊢ iP to Δ = iQ]]$}
                                 {$[[iQ']]$  is sound}
                                 }{}
  \\



  \addlinespace[0.7em]
  % $[[ Γ ⊨ iP1 ∨ iP2 = iQ]]$
  LUB
                      & \infer
                                     { [[iQ]] \text{ is sound}
                                      \begin{cases}
                                        [[Γ ⊢ iQ]]\\
                                        [[Γ ⊢ iQ ≥ iP1]]\\
                                        [[Γ ⊢ iQ ≥ iP2]]
                                      \end{cases}
                                     }
                                     {[[Γ ⊨ iP1 ∨ iP2 = iQ]]}
                                     {}
                              & \infer{\exists [[iQ]] \text{ s.t. }
                                [[Γ ⊨ iP1 ∨ iP2 = iQ]]
                                }{\exists \text{ sound } [[iQ']]}{}
                             & \infer
                                 {[[Δ ⊢ iQ' ≥ iQ]]}
                                 {
                                 \stackon
                                 {$[[Γ ⊨ iP1 ∨ iP2 = iQ]]$}
                                 {$[[iQ']]$  is sound}
                                 }{}
  \\

  \addlinespace[0.7em]
  Anti-unification
                      & \infer
                                              { \text{\stackunder
                                              {$[[(Ξ, uQ, aus1, aus2)]]$}
                                              {is sound}}
                                      \begin{cases}
                                         [[Ξ]] \text{ is negative} \\
                                         [[Γ ; Ξ ⊢ uQ]] \\
                                         [[Γ ; · ⊢ ausi : Ξ]] \\
                                         [[ [ausi] uQ = iPi ]]
                                      \end{cases}
                                     }
                                     {[[G ⊨ iP1 ≈au iP2 ⫤ (Ξ, uQ, aus1, aus2)]]}
                                     {}
                              & \infer{\stackunder
                                {$\exists [[(Ξ, uQ, aus1, aus2)]] \text{ s.t. }$}
                                {$[[G ⊨ iP1 ≈au iP2 ⫤ (Ξ, uQ, aus1, aus2)]]$}
                                }{\exists \text{ sound } [[(Ξ', uQ', aus1', aus2')]]}{}
                             & \infer
                               {
                                 \exists [[Γ;Ξ ⊢ aus : Ξ']] \text{ s.t. } [[ [aus]uQ' = uQ ]] 
                               }
                               {
                               \stackon
                               {$[[G ⊨ iP1 ≈au iP2 ⫤ (Ξ, uQ, aus1, aus2)]]$}
                               {$[[(Ξ', uQ', aus1', aus2')]] \text{ is sound}$}
                               }
                               {}
  \\

  \addlinespace[0.7em]
  % $[[ Γ ⊨ iP1 ∨ iP2 = iQ]]$
  \stackunder{Unification}
  {(matching)}
                      & \infer
                                     { [[us]] \text{ is sound}
                                      \begin{cases}
                                         [[Γ ⊢ us : Θ]]\\
                                         [[ [us]uP = iQ ]]\\
                                      \end{cases}
                                     }
                                     {[[Γ; Θ ⊨ uP ≈u iQ ⫤ us]]}
                                     {}
                              & \infer{\exists [[us]] \text{ s.t. }
                                [[Γ; Θ ⊨ uP ≈u iQ ⫤ us]]
                                }{\exists \text{ sound } [[us']]}{}
                             &  ---
  \\

  \addlinespace[0.7em]
  % $[[ Γ ⊨ iP1 ∨ iP2 = iQ]]$
  Subtyping
                      & \infer
                        { [[us]] \text{ is sound}
                        \begin{cases}
                          [[Γ ⊢ us : Θ]]\\
                          [[ Γ ⊢ [us]uN ≤ iM ]]
                        \end{cases}
                        }
                        {[[Γ ; Θ ⊨ uN ≤ iM ⫤ us]]}
                        {}
                     & \infer{\exists [[us]] \text{ s.t. }
                       [[Γ ; Θ ⊨ uN ≤ iM ⫤ us]]
                                }{\exists \text{ sound } [[us']]}{}
                     &  ---
  \\
  
\end{tabular}


\subsection{Variable Ordering}
\obsOrdDeterministic*
\begin{proof}
  By mutual structural induction on $[[iN]]$ and $[[iP]]$.
  Notice that the shape of the term $[[iN]]$ or $[[iP]]$
  uniquely determines the last used inference rule,
  and all the premises are deterministic on the input.
\end{proof}

\lemOrdSoundness*
\begin{proof}
  We prove it by mutual induction on 
  $[[ ord varset in iN = ordVars ]]$ and $[[ ord varset in iP = ordVars ]]$.
  The only non-trivial cases are 
  \ruleref{\ottdruleOArrowLabel} and 
  \ruleref{\ottdruleOForallLabel}.  
  \begin{caseof}
    \item \ruleref{\ottdruleOArrowLabel}  
      Then the inferred ordering judgement has shape
      $[[ord varset in iP → iN = ordVars1, (ordVars2 \ {ordVars1})]]$
      and by inversion, 
      $[[ord varset in iP = ordVars1]]$   
      and 
      $[[ord varset in iN = ordVars2]]$.

      By definition of free variables, 
      $[[varset ∩ fv iP → iN = varset ∩ fv iP ∪ varset ∩ fv iN]]$,
      and since by the induction hypothesis 
      $[[varset ∩ fv iP = {ordVars1}]]$ and
      $[[varset ∩ fv iN = {ordVars2}]]$,
      we have
      $[[varset ∩ fv iP → iN = {ordVars1} ∪ {ordVars2}]]$.

      On the other hand, 
      as a set $[[{ordVars1} ∪ {ordVars2}]]$
      is equal to $[[ordVars1, (ordVars2 \ {ordVars1})]]$. 
    \item  \ruleref{\ottdruleOForallLabel}.
      Then  the inferred ordering judgement has shape
      $[[ord varset in ∀pas.iN = ordVars]]$,
      and by inversion, 
      $[[varset ∩ {pas} = ∅]]$    
      $[[ord varset in iN = ordVars]]$.
      The latter implies that $[[varset ∩ fv iN = {ordVars}]]$.
      We need to show that $[[varset ∩ fv ∀pas.iN = {ordVars}]]$,
      or equivalently, that
      $[[varset ∩ (fv iN \ {pas}) = varset ∩ fv iN ]]$,
      which holds since $[[varset ∩ {pas} = ∅]]$.
  \end{caseof}
\end{proof}


\corOrdAdditivity*

\lemOrdWeakening*
\begin{proof}
  Mutual structural induction on $[[iN]]$ and $[[iP]]$.

  \begin{caseof}
    \item If $[[iN]]$ is a variable $[[na]]$,
      we notice that $[[na ∊ varset]]$ 
      is equivalent to $[[na ∊ varset ∩ {na}]]$.
    \item If $[[iN]]$ has shape $[[↑iP]]$, then
      the required property holds immediately by the 
      induction hypothesis, since 
      $[[fv(↑iP) = fv(iP)]]$.
    \item If the term has shape $[[iP → iN]]$ then
      \ruleref{\ottdruleOArrowLabel} was applied
      to infer $[[ ord (varset ∩ (fv iP ∪ fv iN)) in iP → iN ]]$
      and $[[ ord varset in iP → iN]]$. 
      By inversion, the result of 
      $[[ ord (varset ∩ (fv iP ∪ fv iN)) in iP → iN ]]$
      depends on 
      $A = [[ ord (varset ∩ (fv iP ∪ fv iN)) in iP]]$
      and 
      $B = [[ ord (varset ∩ (fv iP ∪ fv iN)) in iN]]$.
      The result of
       $[[ ord varset in iP → iN]]$ 
       depends on 
      $X = [[ord varset in iP]]$ and
      $Y = [[ord varset in iN]]$.

      Let us show that $A = B$ and $X = Y$, so the results are equal. 
      By the induction hypothesis and set properties,
      $[[ ord (varset ∩ (fv iP ∪ fv iN)) in iP ]] = 
       [[ ord (varset ∩ (fv iP ∪ fv iN)) ∩ fv(iP) in iP ]] = 
       [[ ord varset ∩ fv(iP) in iP ]] = 
       [[ ord varset in iP ]]$.
      Analogously, 
      $[[ ord (varset ∩ (fv iP ∪ fv iN)) in iN ]]$ $=$
      \\ $[[ ord varset in iN ]]$.
    \item If the term has shape $[[∀pas.iN]]$,
      we can assume that $[[pas]]$ is disjoint
      from $[[varset]]$,
      since we operate on alpha-equivalence classes.
      Then using the induction hypothesis,
      set properties and \ruleref{\ottdruleOForallLabel}: 
      $[[ord varset ∩ (fv(∀pas.iN)) in ∀pas.iN]] =
       [[ord varset ∩ (fv(iN) \ {pas}) in iN]] =
       [[ord varset ∩ (fv(iN) \ {pas}) ∩ fv(iN) in iN]] =
       [[ord varset ∩ fv(iN) in iN]] =
       [[ord varset in iN]]$.
  \end{caseof}
\end{proof}

\corOrdIdemp*
\begin{proof}
  By \cref{lemma:ord-soundness,corollary:ord-weakening}.
\end{proof}
  

Next, we make a set-theoretical observation
that will be useful further.
In general, any injective function (its image)
distributes over the set intersection.
However, for convenience, we allow the bijections
on variables to be applied
\emph{outside of their domains}
(as identities), which may violate
the injectivity. To deal with these cases, 
we define a special notion of
bijections collision-free on certain sets
in such a way that
a bijection that is collision-free on $P$ and $Q$,
distributes over intersection of $P$ and $Q$.

\begin{definition} [Collision-free Bijection]
  We say that a bijection $\mu : A \leftrightarrow B$ between sets of
  variables is \textbf{collision-free on sets} $P$ and $Q$ if and only if
  \begin{enumerate}
    \item $\mu(P \cap A) \cap Q = \emptyset$
    \item $\mu(Q \cap A) \cap P = \emptyset$
  \end{enumerate}
\end{definition}

\begin{observation}
  Suppose that $\mu : A \leftrightarrow B$ is a bijection between two sets of variables,
  and $\mu$ is collision-free on $P$ and $Q$.
  Then $\mu(P \cap Q) = \mu(P) \cap \mu(Q)$.
\end{observation}
  
\lemDistrMuOrd*
\begin{proof}
  Mutual induction on $[[iN]]$ and $[[iP]]$.
  \begin{caseof}
  \item $[[iN]]$ = $[[na]]$ \label{case:distr-mu-ord:var} \\
    let us consider four cases:
    \begin{caseof}
    \item $[[na]] \in A$ and $[[na]] \in [[varset]]$. Then
      \begin{align*} [[ [mu] (ord varset in iN) ]] &= [[ [mu] (ord varset in na)]] \\
                                                             &= [[ [mu] na ]]
                                                             && \text{by \ruleref{\ottdruleOPVarInLabel}}\\
                                                             &= [[nb]]
                                                             && \text{for some $[[nb]] \in B$}\\
                                                             & && \text{(notice $[[nb]] \in [[ [mu]varset ]]$)} \\
                                                             &= [[ ord [mu]varset in nb ]]
                                                             && \text{by \ruleref{\ottdruleOPVarInLabel},
                                                                as $[[nb]] \in [[ [mu]varset ]]$} \\
                                                             &= [[ord [mu] varset in [mu] na ]]
       \end{align*}
     \item $[[na]] \notin A$ and $[[na]] \notin [[varset]]$\\
       Notice that
       $[[ [mu] (ord varset in iN) ]] = [[ [mu] (ord varset in na)]] = [[·]]$ by
       \ruleref{\ottdruleOPVarNInLabel}.
       On the other hand, $[[ ord [mu] varset in [mu] na = ord [mu] varset
       in na ]] = [[·]]$ The latter equality is from
       \ruleref{\ottdruleOPVarNInLabel}, because
       $[[mu]]$ is collision-free on $[[varset]]$ and $[[fv iN]]$, so
       $[[fv iN]] \ni [[na]] \notin [[mu]](A \cap [[varset]]) \cup
       [[varset]] \supseteq [[ [mu] varset ]]$.
     \item $[[na]] \in A$ but $[[na]] \notin [[varset]]$\\ Then
       $[[ [mu] (ord varset in iN) ]] = [[ [mu] (ord varset in na)]] = [[·]]$
       by \ruleref{\ottdruleOPVarNInLabel}.
       To prove that\\ $[[ ord [mu] varset in [mu] na ]] = [[·]]$, we apply
       \ruleref{\ottdruleOPVarNInLabel}. Let us show that
       $[[ [mu] na ]] \notin [[ [mu] varset ]]$.
       Since $[[ [mu] na ]] = [[mu]]([[na]])$ and
       $[[ [mu] varset ]] \subseteq [[mu]](A \cap [[varset]]) \cup [[varset]]$,
       it suffices to prove 
       $[[mu]]([[na]]) \notin [[mu]](A \cap [[varset]]) \cup [[varset]]$.

       \begin{enumerate}
       \item[(i)] If there is an element $x \in A \cap [[varset]]$ such that
         $[[mu]] x = [[mu]] [[na]]$, then $x = [[na]]$ by bijectivity of
         $[[mu]]$, which contradicts with $[[na]] \notin [[varset]]$. This way, 
         $[[mu]]([[na]]) \notin [[mu]](A \cap [[varset]])$.
       \item[(ii)]
         Since $[[mu]]$ is collision-free on $[[varset]]$ and $[[fv iN]]$,
         $[[mu]] (A \cap [[fv iN]]) \ni [[mu]]([[na]]) \notin [[varset]]$.
       \end{enumerate}

     \item $[[na]] \notin A$ but $[[na]] \in [[varset]]$\\
       $[[ ord [mu] varset in [mu] na ]] = [[ ord [mu] varset in na ]] = [[na]]$.
       The latter is by \ruleref{\ottdruleOPVarNInLabel}, because
       $[[na]] = [[ [mu] na ]] \in [[ [mu] varset ]]$ since $[[na]] \in [[varset]]$.
       On the other hand, $[[ [mu] (ord varset in iN) ]] = [[ [mu] (ord varset in na)]] = [[ [mu] na ]] = [[na]]$.
    \end{caseof}
  
  \item $[[iN]] = [[↑iP]]$
    \begin{align*}
       [[ [mu] (ord varset in iN) ]] &= [[ [mu] (ord varset in ↑iP) ]] \\
                                     &= [[ [mu] (ord varset in iP) ]]
                                     && \text{by \ruleref{\ottdruleOShiftULabel}}\\
                                     &= [[ ord [mu]varset in [mu]iP ]]
                                     && \text{by the induction hypothesis}\\
                                     &= [[ ord [mu]varset in  ↑[mu]iP ]]
                                     && \text{by \ruleref{\ottdruleOShiftULabel}}\\
                                     &= [[ ord [mu]varset in  [mu]↑iP ]]
                                     && \text{by the definition of substitution}\\
                                     &= [[ ord [mu]varset in  [mu]iN ]]
    \end{align*}
          
   \item $[[iN]] = [[iP → iM]]$
     \begin{align*}
        & [[ [mu] (ord varset in iN) ]] \\ 
        &= [[ [mu] (ord varset in iP → iM) ]] \\
        &= [[ [mu] (ordVars1, (ordVars2 \ {ordVars1})) ]]
          && \text{where } [[ord varset in iP = ordVars1]] \text{ and } [[ord varset in iM = ordVars2]] \\
        &= [[ [mu] ordVars1, [mu](ordVars2 \ {ordVars1}) ]] \\
        &= [[ [mu] ordVars1, ([mu]ordVars2 \ [mu]{ordVars1}) ]]
          && \text{by induction on $[[ordVars2]]$;
                  the ind. step is similar to 
                  \cref{case:distr-mu-ord:var}.}\\
        & && \text{notice that $[[mu]]$ is collision free on $[[{ordVars1}]]$ and $[[{ordVars2}]]$} \\
        & && \text{since
          $[[{ordVars1}]] \subseteq [[varset]]$ and
          $[[{ordVars2}]] \subseteq [[fv iN]]$ }\\
          &= [[ [mu] ordVars1, ([mu]ordVars2 \ {[mu]ordVars1}) ]]
      \end{align*}
      \hfill\\
      On the other hand,
       $[[  ord [mu] varset in [mu]iN ]] = 
        [[ ord [mu] varset in [mu]iP → [mu]iM ]] = 
        [[ ordVarsb1, (ordVarsb2 \ {ordVarsb1}) ]] = 
        [[ [mu] ordVars1, ([mu]ordVars2 \ {[mu]ordVars1}) ]]$,
        where  $[[ord [mu] varset in [mu] iP = ordVarsb1]]$ 
        and $[[ord [mu] varset in [mu] iM = ordVarsb2]]$, 
        then by the induction hypothesis, 
        $[[ordVarsb1]] = [[ [mu] ordVars1 ]]$, 
        $[[ordVarsb2]] = [[ [mu] ordVars2 ]]$.
   
   \item $[[iN]] = [[∀ pas.iM]]$
     \begin{align*}
          [[ [mu] (ord varset in iN) ]] &= [[ [mu] ord varset in ∀pas.iM]] \\
                                        &= [[ [mu] ord varset in iM]] \\
                                        &= [[ ord [mu] varset in [mu] iM]]
                                        && \text {by the induction hypothesis}\\
     \end{align*}
     \begin{align*}
       [[ (ord [mu] varset in [mu] iN) ]] &= [[ ord [mu] varset in [mu] ∀pas.iM ]] \\
                                          &= [[ ord [mu] varset in ∀pas.[mu]iM ]] \\
                                          &= [[ ord [mu] varset in [mu] iM ]] \\
     \end{align*}
  \end{caseof}
\end{proof}

\lemOrdSigma*
\begin{proof}
  Mutual induction on $[[iN]]$ and $[[iP]]$.
  \begin{caseof}
    \item $[[iN = na]]$ \\
      If $[[na ∉ Γ1]]$ then $[[ [σ]na = na ]]$ and $[[ ord varset in [σ]na ]] = [[ ord varset in na ]]$, 
      as requried.
      If $[[na ∊ Γ1]]$ then $[[na ∉ varset]]$, so $[[ ord varset in na ]] = [[·]]$.
      Moreover, $[[Γ2 ⊢ σ : Γ1]]$ means $[[ fv([σ]na) ⊆ Γ2 ]]$, and thus, 
      as a set, $[[ ord varset in [σ]na ]] = [[varset ∩ fv([σ]na)]] \subseteq [[varset ∩ Γ2]] = [[·]]$.
    \item $[[iN = ∀pas.iM]]$
      We can assume $[[{pas} ∩ Γ1 = ∅]]$
      and $[[{pas} ∩ varset = ∅]]$. Then 
      \begin{align*}[t]
         [[ ord varset in [σ]iN ]] &= [[ ord varset in [σ]∀pas.iM ]] \\
                                   &= [[ ord varset in ∀pas.[σ]iM ]]\\
                                   &= [[ ord varset in [σ]iM ]]
                                   && \text{by the induction hypothesis}\\
                                   &= [[ ord varset in iM ]]\\
                                   &= [[ ord varset in ∀pas.iM ]]\\
                                   &= [[ ord varset in iN ]]
       \end{align*}
    \item $[[iN = ↑iP]]$
       \begin{align*}[t]
        [[ ord varset in [σ]iN ]] &= [[ ord varset in [σ]↑iP ]] \\
                                   &= [[ ord varset in ↑[σ]iP ]]
                                   && \text{by the definition of substitution}\\
                                   &= [[ ord varset in [σ]iP ]]
                                   && \text{by the induction hypothesis}\\
                                   &= [[ ord varset in iP ]]
                                   && \text{by the definition of substitution}\\
                                   &= [[ ord varset in ↑iP ]]
                                   && \text{by the definition of ordering}\\
                                   &= [[ ord varset in iN ]]
       \end{align*}

    \item $[[iN = iP → iM]]$
       \begin{align*}
        [[ ord varset in [σ]iN ]] &= [[ ord varset in [σ](iP → iM) ]] \\
                                   &= [[ ord varset in ([σ]iP → [σ]iM) ]]
                                   && \text{def. of substitution}\\
                                   &= [[ ord varset in [σ]iP]],\\
                                   &\phantom{=} ~ [[(ord varset in [σ]iM \ {ord varset in [σ]iP}) ]]
                                   && \text{def. of ordering}\\
                                   &= [[ ord varset in iP]],\\
                                   &\phantom{=} ~ [[(ord varset in iM \ {ord varset in iP}) ]]
                                   && \text{the ind. hypothesis}\\
                                   &= [[ ord varset in iP → iM ]]
                                   && \text{def. of ordering}\\
                                   &= [[ ord varset in iN ]]
       \end{align*}
    \item The proofs of the positive cases are symmetric.
  \end{caseof}
\end{proof}

\lemOrdCompleteness*
\begin{proof}
  Mutual induction on $[[iN ≈ iM]]$ and $[[iP ≈ iQ]]$.
  Let us consider the rule inferring $[[iN ≈ iM]]$. 
  \begin{caseof}
    \item \ruleref{\ottdruleEOneNVarLabel}
    \item \ruleref{\ottdruleEOneShiftULabel}
    \item \ruleref{\ottdruleEOneArrowLabel}
      Then the equivalence has shape $[[iP → iN ≈ iQ → iM]]$,
      and by inversion, $[[iP ≈ iQ]]$ and $[[iN ≈ iM]]$.
      Then by the induction hypothesis,
      $[[ord varset in iP]] = [[ord varset in iQ]]$ 
      and $[[ord varset in iN]] = [[ord varset in iM]]$.
      Since the resulting ordering for $[[iP → iN]]$ and $[[iQ → iM]]$
      depend on the ordering of the corresponding components, 
      which are equal, the results are equal.
    \item \ruleref{\ottdruleEOneForallLabel}
      Then the equivalence has shape $[[∀pas.iN ≈ ∀pbs.iM]]$.
      and by inversion there exists 
      $[[mu : ({pbs} ∩ fv iM) ↔ ({pas} ∩ fv iN)]]$ such that
      \begin{itemize}
        \item $[[{pas} ∩ fv iM = ∅]]$ and 
        \item $[[iN ≈ [mu] iM]]$
      \end{itemize}

      Let us assume that $[[varset]]$ is disjoint from 
      $[[pas]]$ and $[[pbs]]$ 
      (we can always alpha-rename the bound variables).
      Then $[[ord varset in ∀pas.iN = ord varset in iN]]$, 
      $[[ord varset in ∀pas.iM = ord varset in iM]]$
      and by the induction hypothesis,
      $[[ord varset in iN]] = [[ord varset in [mu]iM]]$.
      This way, it suffices tho show  that 
      $[[ord varset in [mu]iM = ord varset in iM]]$.
      It holds by \cref{lemma:ord-sigma} since
      $[[varset]]$ is disjoint form 
      the domain and the codomain of 
      $[[mu : ({pbs} ∩ fv iM) ↔ ({pas} ∩ fv iN)]]$ 
      by assumption.

    \item The positive cases are proved symmetrically.
  \end{caseof}
\end{proof}


\subsection{Normaliztaion}
\obsNormDeterministic*
\begin{proof}
  By straightforward induction using \cref{obs:ord-deterministic}.
\end{proof}


\lemmaFvNf*
\begin{proof}
  By mutual induction on $[[iN]]$ and $[[iP]]$.
  The base cases 
  (\ruleref{\ottdruleNrmNVarLabel} and \ruleref{\ottdruleNrmPVarLabel})
  are trivial; the congruent cases
  (\ruleref{\ottdruleNrmShiftULabel},
  \ruleref{\ottdruleNrmShiftDLabel}, and
  \ruleref{\ottdruleNrmArrowLabel}) are proved by the induction hypothesis.

  Let us consider the case when the term is formed by $[[∀]]$,
  that is the normalization judgment has a shape 
  $[[nf(∀pas.iN) = ∀pas'.iN']]$,
  where by inversion $[[nf(iN) = iN']]$
  and $[[ord {pas} in iN' = pas']]$.
  By the induction hypothesis, $[[fv iN = fv iN']]$.
  Since $[[fv(∀pas.iN) = fv iN \ {pas}]]$,
  and $[[fv(∀pas'.iN') = fv iN' \ {pas'}]]$,
  it is left to show that $[[fv iN \ {pas} = fv iN \ {pas'}]]$.
  By \cref{lemma:ord-completeness}, 
  $[[{pas'}]] = [[{pas} ∩ fv iN']] = [[{pas} ∩ fv iN]]$.
  Then
  $[[fv iN \ {pas} = fv iN \ ({pas} ∪ fv iN)]]$ by
  set-theoretic properties, and
  thus, $[[fv iN \ {pas} = fv iN \ {pas'}]]$.

  The case when the term is positive and formed by $[[∃]]$ is symmetric.
\end{proof}

\lemmaNormSoundness*
\begin{proof}
  Mutual induction on $[[nf(iN) = iM]]$ and $[[nf(iP) = iQ]]$.
  Let us consider how this judgment is formed:
  \begin{caseof}
    \item{\nameref{\ottdruleNrmNVarLabel} and \nameref{\ottdruleNrmPVarLabel}}\\ By
      the corresponding equivalence rules.
    \item{\nameref{\ottdruleNrmShiftULabel}, \nameref{\ottdruleNrmShiftDLabel},
        and \nameref{\ottdruleNrmArrowLabel}}\\
      By the induction hypothesis and the corresponding congruent equivalence rules.
    \item{\nameref{\ottdruleNrmForallLabel}}, i.e. $[[nf(∀pas.uN) = ∀pas'.uN']]$ \label{case:norm-soundness:forall}\\
      From the induction hypothesis, we
      know that $[[iN ≈ iN']]$. In particular, by \cref{lemma:equiv-fv}, $[[fv
        iN]] \equiv [[fv iN']]$.
      Then by \cref{lemma:ord-soundness}, $[[{pas'}]]
      \equiv [[{pas} ∩ fv iN']] \equiv [[{pas} ∩ fv iN]]$, and thus,
      $[[{pas'} ∩ fv iN']] \equiv [[{pas} ∩ fv iN]]$.
      
      To prove $[[∀pas.iN ≈ ∀pas'.iN']]$, it suffices to provide a bijection 
      $\mu : [[{pas'} ∩ fv iN']] \leftrightarrow [[{pas} ∩ fv iN]]$ such that
      $[[iN ≈ [mu]iN']]$. Since these sets are equal, we take $\mu = id$.
    \item{\nameref{\ottdruleNrmExistsLabel}} Same as for \cref{case:norm-soundness:forall}.
  \end{caseof}
\end{proof}

\corollaryWfNf*
\begin{proof}
  Immediately from \cref{lemma:wf-equiv,lemma:normalization-soundness}.
\end{proof}

\corollaryWfSNf*
\begin{proof}
  Let us prove the forward direction.
  Suppose that $[[α± ∊ Γ1]]$.  Let us show that $[[Γ2 ⊢ [nf(σ)]α±]]$.
  By the definition of substitution normalization,
  $[[ [nf(σ)]α± = nf([σ]α±) ]]$. Then by \cref{corollary:wf-nf},
  to show $[[Γ2 ⊢ nf([σ]α±)]]$, it suffices to show $[[Γ2 ⊢ [σ]α±]]$,
  which holds by the assumption $[[Γ2 ⊢ σ : Γ1]]$.

  The backward direction is proved analogously.
\end{proof}

\lemmaNormSubstSig*
\begin{proof}
  Suppose that $[[α± ∊ Γ1]]$. 
  Then by \cref{corollary:wf-nf}, $[[Γ2 ⊢ nf([σ]α±)]] = [[ [nf(σ)]α± ]]$ 
  is equivalent to $[[Γ2 ⊢ [σ]α±]]$.

  Suppose that $[[α± ∉ Γ1]]$. 
  $[[Γ2 ⊢ σ : Γ1]]$ means that $[[ [σ]α± = α± ]]$, 
  and then $[[ [nf(σ)]α± ]] = [[nf([σ]α±)]] = [[nf(α±)]] = [[α±]]$.
\end{proof}

\corollaryNfSoundWrtSubtEquiv*
\begin{proof}
  Immediately from \cref{lemma:normalization-soundness,corollary:wf-nf,lemma:equiv-soundness}.  
\end{proof}

\corollaryNfPresSubt*
\begin{proof}
  \hfill
  \begin{itemize}
    \item [$+$]  
    \begin{itemize}
      \item [$\Rightarrow$] Let us assume $[[Γ ⊢ iP ≥ iQ]]$.
        By \cref{corollary:nf-sound-wrt-subt-equiv},
        $[[Γ ⊢ iP ≈ nf(iP)]]$ and $[[Γ ⊢ iQ ≈ nf(iQ)]]$, 
        in particular, by inversion, 
        $[[Γ ⊢ nf(iP) ≥ iP]]$ and $[[Γ ⊢ iQ ≥ nf(iQ)]]$.
        Then by transitivity of subtyping 
        (\cref{lemma:subtyping-transitivity}), 
        $[[Γ ⊢ nf(iP) ≥ nf(iQ)]]$.
      \item [$\Leftarrow$] Let us assume $[[Γ ⊢ nf(iP) ≥ nf(iQ)]]$.
        Also by \cref{corollary:nf-sound-wrt-subt-equiv}
        and inversion, 
        $[[Γ ⊢ iP ≥ nf(iP)]]$ and $[[Γ ⊢ nf(iQ) ≥ iQ]]$.
        Then by the transitivity, $[[Γ ⊢ iP ≥ iQ]]$.
    \end{itemize}
    \item [$-$] The negative case is proved symmetrically.
  \end{itemize}
\end{proof}

\corollaryNormPreservesOrdering*
\begin{proof}
  Immediately from \cref{lemma:ord-completeness,lemma:normalization-soundness}.
\end{proof}

\lemmaNormSubstDistr*
\begin{proof}
  Mutual induction on $[[iN]]$ and $[[iP]]$.
  \begin{caseof}
    \item $[[iN]]$ = $[[na]]$ \\
      \label{case:norm-subst-distr-var}
      $[[nf([σ]iN)]] = [[ nf([σ]na) ]] = [[ [nf(σ)]na ]] $.

      $[[ [nf(σ)] nf(iN) ]] = [[ [nf(σ)] nf(na) ]] = [[ [nf(σ)] na ]] $.
    \item $[[iP]]$ = $[[pa]]$ \\
      Similar to \cref{case:norm-subst-distr-var}.
   \item If the type is formed by $[[→]]$, $[[↑]]$, or $[[↓]]$, 
     the required equality follows from the congruence of the normalization and
     substitution and the induction hypothesis.
     For example, if $[[iN]] = [[iP → iM]]$ then
     \begin{align*}
        [[nf([σ] iN)]] &= [[ nf([σ] (iP → iM)) ]] \\
                        &= [[ nf([σ]iP → [σ]iM) ]]
                        && \text{By congruence of substitution} \\
                        &= [[ nf([σ]iP) → nf([σ]iM) ]]
                        && \text{By congruence of normalization} \\
                        &= [[ [nf(σ)]nf(iP) → [nf(σ)]nf(iM) ]]
                        && \text{By induction hypothesis} \\
                        &= [[ [nf(σ)](nf(iP) → nf(iM)) ]]
                        && \text{By congruence of substitution} \\
                        &= [[ [nf(σ)]nf(iP → iM) ]]
                        && \text{By congruence of normalization} \\
                        &= [[ [nf(σ)]nf(iN) ]]
      \end{align*}
    \item $[[iN]] = [[∀ pas.iM]]$ \label{case:norm-subst-commute}
      \begin{align*}
          [[ [nf(σ)] nf(iN) ]] &= [[ [nf(σ)] nf(∀pas.iM)]] \\
                            &= [[ [nf(σ)] ∀pas'.nf(iM) ]]
                            && \text {Where $[[pas']] = [[ ord {pas} in nf(iM)]]
                               = [[ord {pas} in iM]]$}\\
                           & && 
                               \text{(the latter is by
                               \cref{corollary:normalization-ord})}\\
        \end{align*}

      \begin{align*}
         [[ nf([σ]iN) ]] &= [[ nf([σ] ∀pas.iM)]] \\
                          &= [[ nf(∀pas.[σ]iM) ]]
                          && \text{Assuming $[[{pas} ∩ Γ1]] = \emptyset$
                             and $[[{pas} ∩ Γ2]] = \emptyset$}\\
                          &= [[ ∀pbs.nf([σ]iM) ]]
                          && \text {Where $[[pbs]] = [[ord {pas} in nf([σ]iM)]]
                             = [[ord {pas} in [σ]iM]]$}\\
                          & && 
                             \text{(the latter is by \cref{corollary:normalization-ord})}\\
                          &= [[ ∀pas'.nf([σ]iM) ]]
                          && \text{By \cref{lemma:ord-sigma},}\\
                          & && \text{$[[pbs]] = [[pas']]$
                             since $[[{pas}]]$ is disjoint with $[[Γ1]]$ and
                             $[[Γ2]]$}\\
                          &= [[ ∀pas'.[nf(σ)]nf(iM) ]]
                          && \text {By the induction hypothesis}\\
         \end{align*}
     To show the alpha-equivalence of 
     $[[ [nf(σ)] ∀pas'.nf(iM) ]]$ and $[[ ∀pas'.[nf(σ)]nf(iM) ]]$,
     we can assume that $[[{pas'} ∩ Γ1]] = \emptyset$, and $[[{pas'} ∩ Γ2]]
     = \emptyset$.

   \item $[[iP]] = [[∃ nas.iQ]]$ \\
     Same as for \cref{case:norm-subst-commute}.
  \end{caseof}
\end{proof}

\lemmaNormSubstCommute*
\begin{proof}
  Immediately from \cref{lemma:norm-subst-distr}, after noticing that $[[nf(mu)]] = [[mu]]$.
\end{proof}



\lemmaNormalizationCompleteness*
\begin{proof}
  Mutual induction on $[[iN ≈ iM]]$ and $[[iP ≈ iQ]]$.
  \begin{caseof}
  \item {\nameref{\ottdruleEOneForallLabel}} \label{case:ord-completeness:forall}
   From the normalization definition,
    \begin{itemize}
      \item $[[nf(∀pas.iN)]] = [[∀pas'.nf(iN)]]$ where $[[pas']]$ is $[[ord {pas} in nf(iN)]]$
      \item $[[nf(∀pbs.iM)]] = [[∀pbs'.nf(iM)]]$ where $[[pbs']]$ is $[[ord {pbs} in nf(iM)]]$
    \end{itemize}
    Let us take $[[mu : ({pbs} ∩ fv iM) ↔ ({pas} ∩ fv iN)]]$ from the
    inversion of the equivalence judgment. Notice that from
    \cref{lemma:fv-nf,lemma:ord-soundness}, the domain and the codomain of $\mu$ can be written
    as $[[mu : {pbs'} ↔ {pas'}]]$.
    
    To show the alpha-equivalence of $[[∀pas'.nf(iN)]]$ and $[[∀pbs'.nf(iM)]]$,
    it suffices to prove that
    \begin{enumerate*}
    \item[(i)] $[[ [mu] nf(iM) ]] = [[nf(iN)]]$ and \newline
    \item[(ii)] $[[ [mu]pbs' ]] = [[pas']]$
    \end{enumerate*}.
    
    \begin{enumerate}
    \item[(i)] $[[ [mu] nf(iM) ]] = [[nf([mu]iM)]] = [[nf(iN)]]$.
      The first equality holds by \cref{lemma:norm-subst-commute}, the second---by the induction hypothesis.

    \item[(ii)] 
    \begin{align*} 
      [[ [mu]pbs' ]] &= [[ [mu] ord {pbs} in nf(iM) ]]
                      && \text{by the definition of $[[pbs']]$ } \\
                      &= [[ [mu] ord ({pbs} ∩ fv iM) in nf(iM) ]]
                      && \text{from \cref{lemma:fv-nf,corollary:ord-weakening} } \\
                      &= [[ ord [mu] ({pbs} ∩ fv iM) in [mu] nf(iM) ]]
                      && \text{by \cref{lemma:distr-mu-ord}, because}\\
                      & && \text{$[[{pas} ∩ fv iN]] \cap [[fv nf(iM)]] \subseteq [[{pas} ∩ fv iM ]]
                        = \emptyset$}\\
                      &
                      && \text{$[[{pas} ∩ fv iN]] \cap [[({pbs} ∩ fv iM)]] \subseteq
                        [[{pas} ∩ fv iM]] = \emptyset$} \\
                      &= [[ ord [mu] ({pbs} ∩ fv iM) in nf(iN) ]]
                      && \text{since $[[ [mu] nf(iM) ]] = [[nf(iN)]]$ is proved } \\
                      &= [[ ord ({pas} ∩ fv iN) in nf(iN) ]]
                      && \text{because $\mu$ is a bijection between}\\
                      & && \text{$[[{pas} ∩ fv iN]]$ and $[[{pbs} ∩ fv iM]]$} \\
                      &= [[ ord {pas} in nf(iN) ]]
                      && \text{from \cref{lemma:fv-nf,corollary:ord-weakening} } \\
                      &= [[ pas' ]]
                      && \text{by the definition of $[[pas']]$} \\
      \end{align*}
    \end{enumerate}
  \item {\nameref{\ottdruleEOneExistsLabel}} Same as for \cref{case:ord-completeness:forall}.
  \item Other rules are congruent, and thus, proved by the corresponding congruent alpha-equivalence rule,
    which is applicable by the induction hypothesis. 
  \end{caseof}
\end{proof}

\lemmaDeclEquivAlg*
\begin{proof} \hfill
  \begin{itemize}
    \item[$+$] Let us prove both directions separately.
    \begin{itemize}
      \item[$\Rightarrow$] 
        exactly by \cref{lemma:normalization-completeness},
      \item[$\Leftarrow$] 
        from \cref{lemma:normalization-soundness}, we know
        $[[iP ≈ nf(iP)]] = [[nf(iQ) ≈ iQ]]$, then by transitivity (\cref{lemma:decl-equiv-transitivity}),
        $[[iP ≈ iQ]]$.
    \end{itemize}
    \item[$-$] For the negative case, the proof is the same.
  \end{itemize}
\end{proof}

\corollaryNfCompleteWrtSubtEquiv*
\begin{proof}
  Immediately from \cref{lemma:equiv-completeness,lemma:normalization-completeness}.
\end{proof}


\lemmaNormIdemp*
\begin{proof}
  By applying \cref{lemma:normalization-completeness} to \cref{lemma:normalization-soundness}.
\end{proof}

\lemmaNormalAfterSubst*
\begin{proof}
  Mutual induction on $[[Γ1 ⊢ iP]]$ and $[[Γ1 ⊢ iN]]$.
  \begin{caseof}
  \item $[[iN]] = [[na]]$\\
    Then $[[iN]]$ is always normal, and
    the normality of $[[σ|{na}]]$ by the definition means $[[ [σ]na ]]$ is normal.

  \item $[[iN]] = [[iP → iM]]$ \label{case:normal-after-subst-arrow}
    \begin{align*}
      [[ [σ](iP → iM) ]] \text{ is normal} &\iff [[ [σ]iP → [σ]iM ]] \text{ is normal}
                                           && \text{by substitution
                                              congruence} \\
                                           &\iff
                                             \begin{cases}
                                             [[ [σ]iP ]] &\text{is normal} \\
                                             [[ [σ]iM ]] &\text{is normal} \\
                                             \end{cases}\\
                                           &\iff
                                             \begin{cases}
                                               [[ iP ]]       &\text{is normal} \\
                                               [[ σ|fv(iP) ]] &\text{is normal} \\
                                               [[ iM ]]       &\text{is normal} \\
                                               [[ σ|fv(iM) ]] &\text{is normal} \\
                                             \end{cases}
                                           && \text{by the induction hypothesis}\\
                                           &\iff
                                             \begin{cases}
                                               [[ iP → iM ]]  &\text{is normal} \\
                                               [[ σ|fv(iP) ∪ fv(iM)]] &\text{is normal} \\
                                             \end{cases}\\
                                           &\iff
                                             \begin{cases}
                                               [[ iP → iM ]]  &\text{is normal} \\
                                               [[ σ|fv(iP→iM)]] &\text{is normal} \\
                                             \end{cases}
    \end{align*}
  \item $[[iN]] = [[↑iP]]$\\
    By congruence and the inductive hypothesis, similar to \cref{case:normal-after-subst-arrow}
  \item $[[iN]] = [[∀pas.iM]]$
    \begin{align*}
      &[[ [σ](∀pas.iM) ]] \text{ is normal} \\
       &\iff [[ (∀pas.[σ]iM) ]] \text{ is normal}
                                           && \text{assuming $[[pas]] \cap [[Γ1]] = \emptyset$ and
                                              $[[pas]] \cap [[Γ2]] = \emptyset$} \\
                                           &\iff
                                             \begin{cases}
                                             [[ [σ]iM ]] \text{ is normal} \\
                                             [[ord {pas} in [σ]iM = pas]] \\
                                             \end{cases}
                                           && \text{by the definition of normalization}\\
                                           &\iff
                                             \begin{cases}
                                               [[ [σ]iM ]] \text{ is normal} \\
                                               [[ord {pas} in iM = pas]] \\
                                             \end{cases}
                                           && \text{by \cref{lemma:ord-sigma}}\\
                                           &\iff
                                             \begin{cases}
                                               [[ σ|fv(iM) ]] \text{ is normal} \\
                                               [[ iM ]] \text{ is normal} \\
                                               [[ord {pas} in iM = pas]] \\
                                             \end{cases}
                                           && \text{by the induction hypothesis}\\
                                           &\iff
                                             \begin{cases}
                                               [[ σ|fv(∀pas.iM) ]] \text{ is normal} \\
                                               [[ ∀pas.iM ]] \text{ is normal} \\
                                             \end{cases}
                                           &&
                                              \begin{aligned}
                                              &\text{since $[[fv(∀pas.iM) = fv(iM)]]$;}\\ &\text{by the definition of normalization}
                                              \end{aligned}
    \end{align*}
  \item $[[iP]] = \dots$\\
    The positive cases are done in the same way as the negative ones.

  \end{caseof}
\end{proof}

\lemmaSubtInducedEquivAlg*
\begin{proof}
  Let us prove the positive case, the negative case is symmetric.
  We prove both directions of $\iff$ separately:
  \begin{itemize}
    \item [$\Rightarrow$] exactly \cref{corollary:nf-complete-wrt-subt-equiv};
    \item [$\Leftarrow$] by \cref{lemma:decl-equiv-algorithmization,lemma:equiv-soundness}.
  \end{itemize}
\end{proof}


\corSubstPresDeclEquiv*
\begin{proof}
  \begin{align*} 
    [[ iP ≈ iQ ]] &\Rightarrow        [[ nf(iP) = nf(iQ) ]]
                  && \text{by \cref{lemma:subt-equiv-algorithmization}}\\
                  &\Rightarrow [[ [nf(σ)]nf(iP) = [nf(σ)]nf(iQ)]]\\
                  &\Rightarrow [[ nf([σ]iP) = nf([σ]iQ)]]
                  && \text{by \cref{lemma:norm-subst-distr}}\\ 
                  &\Rightarrow        [[ [σ]iP ≈ [σ]iQ ]]
                  && \text{by \cref{lemma:subt-equiv-algorithmization}}\\
  \end{align*} 
\end{proof}

\subsection{Equivalence}
\begin{lemma}[Declarative equivalence is transitive]
  \hfill
  \label{lemma:decl-equiv-transitivity}
  \begin{itemize}
  \item[$+$] if $[[iP1 ≈ iP2]]$ and $[[iP2 ≈ iP3]]$ then $[[iP1 ≈ iP3]]$,
  \item[$-$] if $[[iN1 ≈ iN2]]$ and $[[iN2 ≈ iN3]]$ then $[[iN1 ≈ iN3]]$.
  \end{itemize}
\end{lemma}
\begin{proof}
  \ilyam{should be easy to do by induction since the types are getting smaller}
\end{proof}

\begin{lemma}[Algorithmization of declarative equivalence]
  \label{lemma:decl-equiv-algorithmization}
  Declarative equivalence is equality of normal forms. 
  \begin{itemize}
    \item[$+$] $[[iP ≈ iQ]] \iff [[nf(iP) = nf(iQ)]]$,
    \item[$-$] $[[iN ≈ iM]] \iff [[nf(iN) = nf(iM)]]$.
  \end{itemize}
\end{lemma}
\begin{proof} \hfill
  \begin{itemize}
    \item[$+$] Let us prove both directions separately.
    \begin{itemize}
      \item[$\Rightarrow$] 
        exactly by \cref{lemma:normalization-completeness},
      \item[$\Leftarrow$] 
        from \cref{lemma:normalization-soundness}, we know
        $[[iP ≈ nf(iP)]] = [[nf(iQ) ≈ iQ]]$, then by transitivity (\cref{lemma:decl-equiv-transitivity}),
        $[[iP ≈ iQ]]$.
    \end{itemize}
    \item[$-$] The proof is exactly the same.
  \end{itemize}
\end{proof}

\begin{lemma}[Type well-formedness is invariant under equivalence]
  \label{lemma:wf-equiv}
  Mutual subtyping implies declarative equivalence.
  \begin{itemize}
  \item[$+$] if $[[iP ≈ iQ]]$ then $[[Γ ⊢ iP]] \iff [[Γ ⊢ iQ]]$,
  \item[$-$] if $[[iN ≈ iM]]$ then $[[Γ ⊢ iN]] \iff [[Γ ⊢ iM]]$
  \end{itemize}
\end{lemma}
\begin{proof}
  \ilyam{todo}
\end{proof}

\begin{corollary}[Normalization preserves well-formedness]
  \label{corollary:wf-nf}
  \hfill
  \begin{itemize}
  \item[$+$] $[[Γ ⊢ iP]] \iff [[Γ ⊢ nf(iP)]]$,
  \item[$-$] $[[Γ ⊢ iN]] \iff [[Γ ⊢ nf(iN)]]$
  \end{itemize}
\end{corollary}
\begin{proof}
  Immediately from \cref{lemma:wf-equiv,lemma:normalization-soundness}.
\end{proof}

\begin{corollary}[Normalization preserves well-formedness of substitution]
  \label{corollary:wf-s-nf}
  \hfill \\
   $[[Γ2 ⊢ σ : Γ1]] \iff [[Γ2 ⊢ nf(σ) : Γ1]]$
\end{corollary}

\begin{lemma}[Soundness of equivalence]
  \label{lemma:equiv-soundness}
  Declarative equivalence implies mutual subtyping.
  \begin{itemize}
    \item[$+$] if $[[Γ ⊢ iP]]$, $[[Γ ⊢ iQ]]$, and $[[iP ≈ iQ]]$ then $[[Γ ⊢ iP ≈ iQ]]$,
    \item[$-$] if $[[Γ ⊢ iN]]$, $[[Γ ⊢ iM]]$, and $[[iN ≈ iM]]$ then $[[Γ ⊢ iN ≈ iM]]$.
  \end{itemize}
\end{lemma}
\begin{proof}
  We prove it by mutual induction on $[[iP ≈ iQ]]$ and $[[iN ≈ iM]]$.
  \begin{caseof}
    \item $[[a⁻ ≈ a⁻]]$\\
      Then $[[Γ ⊢ a⁻ ≤ a⁻]]$ by \ruleref{\ottdruleDOneNVarLabel},
      which immediately implies $[[Γ ⊢ a⁻ ≈ a⁻]]$ by \ruleref{\ottdruleDOneNDefLabel}.

    \item $[[↑iP ≈ ↑iQ]]$\\
      Then by inversion of \ruleref{\ottdruleDOneShiftULabel},
      $[[iP ≈ iQ]]$, and from the induction hypothesis, $[[Γ ⊢ iP ≈ iQ]]$,
      and (by symmetry) $[[Γ ⊢ iQ ≈ iP]]$.

      When \ruleref{\ottdruleDOneShiftULabel} is applied to $[[Γ ⊢ iP ≈ iQ]]$,
      it gives us $[[Γ ⊢ ↑iP ≤ ↑iQ]]$; when it is applied to $[[Γ ⊢ iQ ≈ iP]]$,
      we obtain $[[Γ ⊢ ↑iQ ≤ ↑iP]]$. Together, it implies $[[Γ ⊢ ↑iP ≈ ↑iQ]]$.

    \item $[[iP → iN ≈ iQ → iM]]$\\
      Then by inversion of \ruleref{\ottdruleDOneArrowLabel},
      $[[iP ≈ iQ]]$ and $[[iN ≈ iM]]$. By the induction hypothesis,
      $[[Γ ⊢ iP ≈ iQ]]$ and $[[Γ ⊢ iN ≈ iM]]$, which means by inversion:
      \begin{enumerate*}
        \item[(i)] $[[Γ ⊢ iP ≥ iQ]]$,
        \item[(ii)] $[[Γ ⊢ iQ ≥ iP]]$,
        \item[(iii)] $[[Γ ⊢ iN ≤ iM]]$,
        \item[(iv)]  $[[Γ ⊢ iM ≤ iN]]$.
      \end{enumerate*}
      Applying \ruleref{\ottdruleDOneArrowLabel} to (i) and (iii), we obtain
      $[[Γ ⊢ iP → iN ≤ iQ → iM]]$; applying it to (ii) and (iv), we have $[[Γ ⊢
      iQ → iM ≤ iP → iN]]$. Together, it implies $[[Γ ⊢ iP → iN ≈ iQ → iM]]$.
    \item $[[∀pas.iN ≈ ∀pbs.iM]]$\\
      Then by inversion, there exists bijection $[[mu : ({pbs} ∩ fv iM) ↔ ({pas}
      ∩ fv iN)]]$, such that $[[iN ≈ [mu] iM]]$. By the induction hypothesis,
      $[[Γ, pas ⊢ iN ≈ [mu] iM]]$. From \cref{corollary:subst-pres-equiv} and
      the fact that $[[mu]]$ is bijective, we also have
      $[[Γ, pbs ⊢ [mu-1]iN ≈ iM]]$.

      Let us construct a subsitution $[[pas ⊢ iPs/pbs : pbs]]$ by
      extending $[[mu]]$ with arbitrary positive types on $[[{pbs} \ fv iM]]$.

      Notice that $[[ [mu]iM ]] = [[ [iPs/pbs]iM ]]$, and therefore,
      $[[Γ, pas ⊢ iN ≈ [mu] iM]]$ implies $[[Γ, pas ⊢ [iPs/pbs]iM ≤ iN]]$. Then by
      \ruleref{\ottdruleDOneForallLabel}, $[[Γ ⊢ ∀pbs.iM ≤ ∀pas.iN]]$.

      Analogously, we construct the substitution from $[[mu-1]]$, and use it to
      instantiate $[[pas]]$ in the application of
      \ruleref{\ottdruleDOneForallLabel} to infer $[[Γ ⊢ ∀pas.iN ≤ ∀pbs.iM]]$.

      This way, $[[Γ ⊢ ∀pbs.iM ≤ ∀pas.iN]]$ and $[[Γ ⊢ ∀pas.iN ≤ ∀pbs.iM]]$
      gives us $[[Γ ⊢ ∀pbs.iM ≈ ∀pas.iN]]$.

    \item For the cases of the positive types, the proofs are symmetric.
  \end{caseof}
\end{proof}

\begin{corollary}[Normalization is sound w.r.t. subtyping-induced equivalence] \label{corollary:nf-sound-wrt-subt-equiv}
  \hfill
  \begin{itemize}
    \item [$+$] if $[[Γ ⊢ iP]]$ then $[[Γ ⊢ iP ≈ nf(iP)]]$,
    \item [$-$] if $[[Γ ⊢ iN]]$ then $[[Γ ⊢ iN ≈ nf(iN)]]$.
  \end{itemize}
\end{corollary}
\begin{proof}
  Immediately from \cref{lemma:normalization-soundness,corollary:wf-nf,lemma:equiv-soundness}.
\end{proof}

\begin{corollary}[Normalization preserves subtyping] 
  \label{corollary:nf-pres-subt}
  Assuming all the types are well-formed in context $[[Γ]]$,
  \begin{itemize}
    \item [$+$] $[[Γ ⊢ iP ≥ iQ]] \iff [[Γ ⊢ nf(iP) ≥ nf(iQ)]]$,
    \item [$-$] $[[Γ ⊢ iN ≤ iM]] \iff [[Γ ⊢ nf(iN) ≤ nf(iM)]]$.
  \end{itemize}
\end{corollary}
\begin{proof}
  \hfill
  \begin{itemize}
    \item [$+$]  
    \begin{itemize}
      \item [$\Rightarrow$] Let us assume $[[Γ ⊢ iP ≥ iQ]]$.
        By \cref{corollary:nf-sound-wrt-subt-equiv},
        $[[Γ ⊢ iP ≈ nf(iP)]]$ and $[[Γ ⊢ iQ ≈ nf(iQ)]]$, 
        in particular, by inversion, 
        $[[Γ ⊢ nf(iP) ≥ iP]]$ and $[[Γ ⊢ iQ ≥ nf(iQ)]]$.
        Then by the transitivity of subtyping 
        (\cref{corollary:subtyping-transitivity}), 
        $[[Γ ⊢ nf(iP) ≥ nf(iQ)]]$.
      \item [$\Leftarrow$] Let us assume $[[Γ ⊢ nf(iP) ≥ nf(iQ)]]$.
        Also by \cref{corollary:nf-sound-wrt-subt-equiv}
        and inversion, 
        $[[Γ ⊢ iP ≥ nf(iP)]]$ and $[[Γ ⊢ nf(iQ) ≥ iQ]]$.
        Then by the transitivity, $[[Γ ⊢ iP ≥ iQ]]$.
    \end{itemize}
    \item [$-$] The negative case is proved symmetrically.
  \end{itemize}
\end{proof}

\begin{lemma}[Subtyping induced by disjoint substitutions]
  \label{lemma:subt-ind-disj-subst}
  If two disjoint substitutions induce subtyping, they are degenerate (so is the
  subtyping).
  Suppose that $[[Γ ⊢ σ1 : Γ1]]$ and $[[Γ ⊢ σ2 : Γ1]]$,
  where $[[{Γi} ⊆ {Γ}]]$ and $[[{Γ1} ∩ {Γ2}= ∅]]$. Then
  \begin{itemize}
  \item[$-$] assuming $[[Γ ⊢ iN]]$,~
    $[[Γ ⊢ [σ1]iN ≤ [σ2]iN]]$ implies $[[Γ ⊢ σi ≈ id : Ord fv iN]]$
  \item[$+$] assuming $[[Γ ⊢ iP]]$,~
    $[[Γ ⊢ [σ1]iP ≥ [σ2]iP]]$ implies $[[Γ ⊢ σi ≈ id : Ord fv iP]]$
  \end{itemize}
\end{lemma}
\begin{proof}
  Proof by induciton on $[[Γ ⊢ iN]]$ (and mutually on $[[Γ ⊢ iP]]$).
  \begin{caseof}
    \item $[[iN]] = [[α⁻]]$\\
      Then $[[Γ ⊢ [σ1]iN ≤ [σ2]iN]]$ is rewritten as $[[Γ ⊢ [σ1]α⁻ ≤ [σ2]α⁻]]$.
      Let us consider the following cases:
      \begin{caseof}
      \item $[[α⁻ ∉ {Γ1}]]$ and $[[α⁻ ∉ {Γ2}]]$ \label{case:var-not-in-ctxts}\\
        Then $[[Γ ⊢ σi ≈ id : α⁻]]$ holds immediately,
        since $[[ [σi] α⁻]] = [[ [id] α⁻]] = [[α⁻]]$ and
        $[[Γ ⊢ α⁻ ≈ α⁻]]$.
      \item $[[α⁻ ∊ {Γ1}]]$ and $[[α⁻ ∊ {Γ2}]]$\\
        This case is not possible by assumption: $[[{Γ1} ∩ {Γ2}= ∅]]$.
      \item $[[α⁻ ∊ {Γ1}]]$ and $[[α⁻ ∉ {Γ2}]]$\\
        Then we have $[[Γ ⊢ [σ1]α⁻ ≤ α⁻]]$,
        which by \cref{corollary:vars-no-proper-subtypes} means $[[Γ ⊢ [σ1]α⁻ ≈ α⁻]]$,
        and hence, $[[Γ ⊢ σ1 ≈ id : α⁻]]$.

        $[[Γ ⊢ σ2 ≈ id : α⁻]]$ holds since $[[ [σ2]α⁻ ]] = [[α⁻]]$,
        similarly to \cref{case:var-not-in-ctxts}.

      \item $[[α⁻ ∉ {Γ1}]]$ and $[[α⁻ ∊ {Γ2}]]$\\
        Then we have $[[Γ ⊢ α⁻ ≤ [σ2]α⁻]]$,
        which by \cref{corollary:vars-no-proper-subtypes} means $[[Γ ⊢ α⁻ ≈ [σ2]α⁻]]$,
        and hence, $[[Γ ⊢ σ2 ≈ id : α⁻]]$.

        $[[Γ ⊢ σ1 ≈ id : α⁻]]$ holds since $[[ [σ1]α⁻ ]] = [[α⁻]]$,
        similarly to \cref{case:var-not-in-ctxts}.
      \end{caseof}
  \item $[[iN]] = [[∀pas.iM]]$\\
    Then by inversion, $[[Γ, pas ⊢ iM]]$.
    $[[Γ ⊢ [σ1]iN ≤ [σ2]iN]]$ is rewritten as $[[Γ ⊢ [σ1]∀pas.iM ≤ [σ2]∀pas.iM]]$.
    By the congruence of substitution and by the inversion of
    \ruleref{\ottdruleDOneForallLabel}, $[[Γ, pas ⊢ [iQs/pas][σ1]iM ≤ [σ2]iM]]$,
    where $[[Γ, pas ⊢ iQi]]$.
    Let us denote the (Kleisli) composition of $[[σ1]]$ and $[[iQs/pas]]$ as
    $[[σ1']]$, noting that $[[Γ, pas ⊢ σ1' : Γ1, pas]]$,
    and $[[{Γ1, pas} ∩ {Γ2} = ∅]]$.

    Let us apply the induction hypothesis to $[[iM]]$ and the
    substitutions $[[σ1']]$ and $[[σ2]]$ with
    $[[Γ, pas ⊢ [σ1']iM ≤ [σ2]iM]]$ to obtain:
    \begin{align}
      [[Γ, pas ⊢ σ1' ≈ id : Ord fv iM]] \label{fact:subs-proper-sub:forall-ih}\\
      [[Γ, pas ⊢ σ2 ≈ id : Ord fv iM]]  \label{fact:subs-proper-sub:forall-ih2}
    \end{align}

    Then $[[Γ ⊢ σ2 ≈ id : Ord fv ∀pas.iM]]$ holds by strengthening of
    \ref{fact:subs-proper-sub:forall-ih2}:
    for any $[[β±]] \in [[fv ∀pas.iM]] = [[fv iM \ {pas}]]$,
    $[[Γ, pas ⊢ [σ2]β± ≈ β±]]$ is strengthened to $[[Γ ⊢ [σ2]β± ≈ β±]]$, because
    $[[fv [σ2]β±]] = [[fv β±]] = \{[[β±]]\} \subseteq [[{Γ}]]$.

    To show that $[[Γ ⊢ σ1 ≈ id : Ord fv ∀pas.iM]]$, let us take an arbitrary
    $[[β±]] \in [[fv ∀pas.iM]] = [[fv iM \ {pas}]]$.

    $
    \begin{aligned}[t]
      [[β±]] &= [[ [id]β± ]]
             && \text{by definition of $[[id]]$}\\
             &\eqDOne [[ [σ1']β± ]]
             && \text{by \ref{fact:subs-proper-sub:forall-ih}}\\
             &= [[ [iQs/pas][σ1]β±]]
             && \text{by definition of $[[σ1']]$}\\
             &= [[ [σ1]β± ]]
             && \text{because $[[{pas} ∩ fv [σ1]β±]] \subseteq [[{pas} ∩ {Γ}]] = \emptyset$}
    \end{aligned}
    $\\
    This way, $[[Γ ⊢ [σ1]β± ≈ β±]]$ for any $[[β±]] \in [[fv ∀pas.iM]]$ and thus,
    $[[Γ ⊢ σ1 ≈ id : Ord fv ∀pas.iM]]$.

  \item $[[iN]] = [[iP → iM]]$\\
    Then by inversion, $[[Γ ⊢ iP]]$ and $[[Γ ⊢ iM]]$.
    $[[Γ ⊢ [σ1]iN ≤ [σ2]iN]]$ is rewritten as
    $[[Γ ⊢ [σ1](iP → iM) ≤ [σ2](iP → iM)]]$,
    then by congruence of substitution,
    $[[Γ ⊢ [σ1]iP → [σ1]iM ≤ [σ2]iP → [σ2]iM]]$,
    then by inversion
    $[[Γ ⊢ [σ1]iP ≥ [σ2]iP]]$
    and
    $[[Γ ⊢ [σ1]iM ≤ [σ2]iM]]$.

    Applying the induction hypothesis to $[[Γ ⊢ [σ1]iP ≥ [σ2]iP]]$
    and to $[[Γ ⊢ [σ1]iM ≤ [σ2]iM]]$, we obtain (respectively):
    \begin{align}
      &[[Γ ⊢ σi ≈ id : Ord fv iP]] \label{fact:subs-proper-sub:arrow-ih1}\\
      &[[Γ ⊢ σi ≈ id : Ord fv iM]] \label{fact:subs-proper-sub:arrow-ih2}
    \end{align}

    Noting that $[[fv (iP → iM)]] = [[fv iP ∪ fv iM]]$,
    we combine
    \cref{fact:subs-proper-sub:arrow-ih1,fact:subs-proper-sub:arrow-ih2}
    to conclude:
    $[[Γ ⊢ σi ≈ id : Ord fv (iP → iM)]]$.

  \item $[[iN]] = [[↑iP]]$\\
    Then by inversion, $[[Γ ⊢ iP]]$.
    $[[Γ ⊢ [σ1]iN ≤ [σ2]iN]]$ is rewritten as
    $[[Γ ⊢ [σ1]↑iP ≤ [σ2]↑iP]]$,
    then by congruence of substitution and by inversion,
    $[[Γ ⊢ [σ1]iP ≥ [σ2]iP]]$

    Applying the induction hypothesis to $[[Γ ⊢ [σ1]iP ≥ [σ2]iP]]$, we obtain
    $[[Γ ⊢ σi ≈ id : Ord fv iP]]$. Since $[[fv ↑iP]] = [[fv iP]]$, we can
    conclude: $[[Γ ⊢ σi ≈ id : Ord fv ↑iP]]$.
  \item The positive cases are proved symmetrically.
  \end{caseof}
\end{proof}

\begin{corollary}[Substitution cannot induce proper subtypes or supertypes] \label{corollary:subst-proper-subt}
  Assuming all mentioned types are well-formed in $[[Γ]]$ and $[[σ]]$ is a
  substitution $[[Γ ⊢ σ : Γ]]$,
  \begin{align*}
    [[Γ ⊢ [σ]iN ≤ iN]] ~&\Rightarrow~ [[Γ ⊢ [σ]iN ≈ iN]]
                          \text{ and } [[Γ ⊢ σ ≈ id : Ord fv iN]] \\
    [[Γ ⊢ iN ≤ [σ]iN]] ~&\Rightarrow~ [[Γ ⊢ iN ≈ [σ]iN]]
                          \text{ and } [[Γ ⊢ σ ≈ id : Ord fv iN]] \\
    [[Γ ⊢ [σ]iP ≥ iP]] ~&\Rightarrow~ [[Γ ⊢ [σ]iP ≈ iP]]
                          \text{ and } [[Γ ⊢ σ ≈ id : Ord fv iP]] \\
    [[Γ ⊢ iP ≥ [σ]iP]] ~&\Rightarrow~ [[Γ ⊢ iP ≈ [σ]iP]]
                          \text{ and } [[Γ ⊢ σ ≈ id : Ord fv iP]] \\
  \end{align*}
\end{corollary}


\begin{lemma} \label{lemma:mutual-subst-subtyping}
  Asssuming that the mentioned types ($[[iP]]$, $[[iQ]]$, $[[iN]]$, and $[[iM]]$)
  are well-formed in $[[Γ]]$ and that the substitutions ($[[σ1]]$ and $[[σ2]]$) have signature $[[Γ ⊢ σi : Γ]]$,
  \begin{itemize}
  \item[$+$] if $[[Γ ⊢ [σ1] iP ≥ iQ]]$ and $[[Γ ⊢ [σ2] iQ ≥ iP]]$\\
    then there exists a bijection $[[μ : fv iP ↔ fv iQ]]$ such that
    $[[Γ ⊢ σ1 ≈ Sub μ : Ord fv iP]]$ and $[[Γ ⊢ σ2 ≈ Sub μ-1 : Ord fv iQ]]$;
  \item[$-$] if $[[Γ ⊢ [σ1] iN ≤ iM]]$ and $[[Γ ⊢ [σ2] iN ≤ iM]]$\\
    then there exists a bijection $[[μ : fv iN ↔ fv iM]]$ such that
    $[[Γ ⊢ σ1 ≈ Sub μ : Ord fv iN]]$ and $[[Γ ⊢ σ2 ≈ Sub μ-1 : Ord fv iM]]$.
  \end{itemize}
\end{lemma}
\begin{proof}
  \hfill
  \begin{itemize}
  \item[$+$]
    Applying $[[σ2]]$ to both sides of
    $[[Γ ⊢ [σ1] iP ≥ iQ]]$ (by \cref{todo}),
    we have: $[[Γ ⊢ [σ2 ○ σ1] iP ≥ [σ2]iQ]]$.
    Composing it with $[[Γ ⊢ [σ2] iQ ≥ iP]]$ (by transitivity \cref{todo}),
    we have $[[Γ ⊢ [σ2 ○ σ1] iP ≥ iP]]$.
    Then by \cref{corollary:subst-proper-subt},
    $[[Γ ⊢ σ2 ○ σ1 ≈ id : Ord fv iP]]$.

    % Applying $[[σ1]]$ to both sides of
    % $[[Γ ⊢ [σ2]iQ ≥ iP]]$ (by \cref{todo}),
    % we have: $[[Γ ⊢ [σ1 ○ σ2] iQ ≥ [σ1]iP]]$.
    % Composing it with $[[Γ ⊢ [σ1] iP ≥ iQ]]$ (by transitivity \cref{todo}),
    % we have $[[Γ ⊢ [σ1 ○ σ2] iQ ≥ iQ]]$.
    % Then by \cref{corollary:subst-proper-subt},
    By a symmetric argument, we also have:
    $[[Γ ⊢ σ1 ○ σ2 ≈ id : Ord fv iQ]]$.

    Now, we prove that
    $[[Γ ⊢ σ2 ○ σ1 ≈ id : Ord fv iP]]$ and
    $[[Γ ⊢ σ1 ○ σ2 ≈ id : Ord fv iQ]]$
    implies that $[[σ1]]$ and $[[σ1]]$
    are (equivalent to) mutually inverse bijections.

    To do so, it suffices to prove that
    \begin{enumerate}
    \item[(i)] for any $[[α± ∊ fv iP]]$ there exists $[[β± ∊ fv iQ]]$
        such that $[[ Γ ⊢ [σ1] α± ≈ β± ]]$ and
        $[[ Γ ⊢ [σ2] β± ≈ α± ]]$; and
    \item[(ii)] for any $[[β± ∊ fv iQ]]$ there exists $[[α± ∊ fv iP]]$
        such that $[[ Γ ⊢ [σ2] β± ≈ α± ]]$ and
        $[[ Γ ⊢ [σ1] α± ≈ β± ]]$.
    \end{enumerate}
    Then the these correspondences between $[[fv iP]]$ and
    $[[fv iQ]]$ are mutually inverse functions,
    since for any $[[β±]]$ there can be at most one $[[α±]]$
    such that $[[ Γ ⊢ [σ2] β± ≈ α± ]]$ (and vice versa).

    \begin{enumerate}
    \item[(i)] Let us take $[[α± ∊ fv iP]]$.
      \begin{enumerate}
      \item if $[[α±]]$ is positive ($[[α± = α⁺]]$),
        from $[[ Γ ⊢ [σ2][σ1]α⁺ ≈ α⁺ ]]$,
        by \cref{corollary:vars-no-proper-subtypes},
        we have
        $[[ [σ2][σ1]α⁺ = ∃nbs.α⁺ ]]$.

        What shape can $[[ [σ1]α⁺ ]]$ have? It cannot be $[[∃nas.↓iN]]$ (for
        potentially empty $[[nas]]$), because the outer constructor $\downarrow$
        would remain after the substitution $[[σ2]]$, whereas $[[∃nbs.α⁺]]$ does
        not have $[[↓]]$. The only case left is $[[ [σ1]α⁺ = ∃nas.γ⁺ ]]$.

        Notice that $[[Γ ⊢ ∃nas.γ⁺ ≈ γ⁺]]$, meaning that $[[Γ ⊢ [σ1]α⁺ ≈ γ⁺]]$.
        Also notice that $[[ [σ2]∃nas.γ⁺ = ∃nbs.α⁺ ]]$ implies
        $[[Γ ⊢ [σ2]γ⁺ ≈ α⁺]]$.

      \item if $[[α±]]$ is negative ($[[α± = α⁻]]$) from $[[ Γ ⊢ [σ2][σ1]α⁻ ≈ α⁻
        ]]$, by \cref{corollary:vars-no-proper-subtypes}, we have
        $[[ [σ2][σ1]α⁻ = ∀pbs.α⁻ ]]$.

        What shape can $[[ [σ1]α⁻ ]]$ have? It cannot be $[[∀pas.↑iP]]$
        nor $[[∀pas.iP → iM]]$ (for potentially empty $[[pas]]$),
        because the outer constructor ($[[→]]$ or $[[↑]]$), remaining
        after the substitution $[[σ2]]$, is however absent in the resulting
        $[[∀pbs.α⁻]]$. Hence, the only case left is $[[ [σ1]α⁻ = ∀pas.γ⁻ ]]$
        Notice that $[[Γ ⊢ γ⁻ ≈ ∀pas.γ⁻]]$, meaning that $[[Γ ⊢ [σ1]α⁻ ≈ γ⁻]]$.
        Also notice that $[[ [σ2]∀pas.γ⁻ = ∀pbs.α⁻ ]]$ implies
        $[[Γ ⊢ [σ2]γ⁻ ≈ α⁻]]$.
      \end{enumerate}
    \item[(ii)] The proof is symmetric:
      We swap $[[iP]]$ and $[[iQ]]$,
      $[[σ1]]$ and $[[σ2]]$,
      and exploit $[[ Γ ⊢ [σ1][σ2]α± ≈ α± ]]$ instead of
      $[[ Γ ⊢ [σ2][σ1]α± ≈ α± ]]$.

    \end{enumerate}

  \item[$-$] The proof is symmetric to the positive case.
  \end{itemize}
\end{proof}

\begin{lemma}[Equivalence of polymorphic types]
  \label{lemma:poly-types-equivalence}
  \hfill
  \begin{itemize}
    \item[$-$] For $[[Γ ⊢ ∀pas.iN]]$ and $[[Γ ⊢ ∀pbs.iM]]$,\\ if $[[Γ ⊢ ∀pas.iN ≈ ∀pbs.iM ]]$
    then there exists a bijection $[[μ : {pbs} ∩ fv iM ↔ {pas} ∩ fv iN]]$
    such that $[[ Γ, pas ⊢ iN ≈ [Sub μ] iN ]]$,
    \item[$+$] For $[[Γ ⊢ ∃nas.iP]]$ and $[[Γ ⊢ ∃nbs.iQ]]$,\\  if $[[Γ ⊢ ∃nas.iP ≈ ∃nbs.iQ ]]$
    then there exists a bijection $[[μ : {nbs} ∩ fv iQ ↔ {nas} ∩ fv iP]]$
    such that $[[ Γ, nbs ⊢ iP ≈ [Sub μ] iQ ]]$.
  \end{itemize}
\end{lemma}
\begin{proof}
    \hfill
  \begin{itemize}
    \item[$-$]
    First, by $\alpha$-conversion, we ensure $[[{pas} ∩ fv iM = ∅]]$ and $[[{pbs} ∩ fv iN = ∅]]$.
    By inversion, $[[Γ ⊢ ∀pas.iN ≈ ∀pbs.iM ]]$ implies 
    \begin{enumerate} 
      \item $[[Γ,pbs ⊢ [σ1]iN ≤ iM]]$ for $[[ Γ,pbs ⊢ σ1 : pas ]]$ and 
      \item $[[Γ,pas ⊢ [σ2]iM ≤ iN]]$ for $[[ Γ,pas ⊢ σ2 : pbs ]]$.
    \end{enumerate}
    To apply \cref{lemma:mutual-subst-subtyping}, we weaken 
    and rearrange the contexts, and extend the substitutions to act as identity
    outside of their initial domain:
    \begin{enumerate} 
      \item $[[Γ,pas,pbs ⊢ [σ1]iN ≤ iM]]$ for $[[ Γ,pas,pbs ⊢ σ1 : Γ,pas,pbs ]]$ and 
      \item $[[Γ,pas,pbs ⊢ [σ2]iM ≤ iN]]$ for $[[ Γ,pas,pbs ⊢ σ2 : Γ,pas,pbs ]]$.
    \end{enumerate}
    Then from \cref{lemma:mutual-subst-subtyping}, 
    there exists a bijection $[[μ : fv iM ↔ fv iN]]$ such that 
    $[[Γ,pas,pbs ⊢ σ2 ≈ Sub μ : Ord fv iM]]$ and 
    $[[Γ,pas,pbs ⊢ σ1 ≈ Sub μ-1 : Ord fv iN]]$. 

    Let us show that if we restrict the domain of $[[μ]]$ to 
    $[[pbs]]$, its range will be contained in $[[pas]]$.
    Let us take $[[γ⁺ ∊ {pbs} ∩ fv iM]]$ and 
    assume $[[ [μ]γ⁺]] \notin [[pas]]$.
    Then since $[[ Γ,pbs ⊢ σ1 : pas ]]$, 
    $[[σ1]]$ acts as identity outside of $[[pas]]$, i.e.
    $[[ [σ1][Sub μ]γ⁺ = [Sub μ]γ⁺ ]]$.
    Since
    $[[Γ,pas,pbs ⊢ σ1 ≈ Sub μ-1 : Ord fv iN]]$, 
    application of $[[σ1]]$ is equivalent to application of $[[Sub μ-1]]$,
    then 
    $[[ Γ,pas,pbs ⊢ [Sub μ-1][Sub μ]γ⁺ ≈ [Sub μ]γ⁺ ]]$, i.e.
    $[[Γ,pas,pbs ⊢ γ⁺ ≈ [Sub μ]γ⁺]]$, 
    which means $[[γ⁺ ∊ fv [Sub μ]γ⁺]] \subseteq [[fv iN]]$.
    By assumption, $[[γ⁺ ∊ {pbs} ∩ fv iM]]$, i.e. $[[{pbs} ∩ fv iN]] \neq \emptyset$, hence contradiction.

    By \cref{todo}, 
    $[[Γ,pas,pbs ⊢ σ2 ≈ Sub μ|{pbs} : Ord fv iM]]$ implies
    $[[Γ,pas,pbs ⊢ [σ2]iM ≈ [Sub μ|{pbs}]iM]]$.
    By similar reasoning, $[[Γ,pas,pbs ⊢ [σ1]iN ≈ [Sub μ-1|{pas}]iN]]$.

    This way,
    \begin{align} 
      [[Γ,pas,pbs ⊢ [Sub μ-1|{pas}]iN ≤ iM]] \label{fact:mu-inv-n-sub-m}\\
      [[Γ,pas,pbs ⊢ [Sub μ|{pbs}]iM ≤ iN]] \label{fact:mu-m-subt-n}
    \end{align}

    By applying $[[μ|_{pbs}]]$ to both sides of \ref{fact:mu-inv-n-sub-m} (\cref{todo})
    and contracting $[[μ-1|_{pas} ○ μ|_{pbs}]] = [[μ|_{pbs}-1 ○ μ|_{pbs}]] = [[id]]$,
    we have: $[[Γ,pas,pbs ⊢ iN ≤ [Sub μ|{pbs}]iM]]$, which together with \ref{fact:mu-m-subt-n}
    means $[[Γ,pas,pbs ⊢ iN ≈ [Sub μ|{pbs}]iM]]$, and by strengthening, $[[Γ,pas⊢ iN ≈ [Sub μ|{pbs}]iM]]$.
    Symmetrically, $[[Γ,pbs ⊢ iM ≈ [Sub μ|_{pbs}-1]iN]]$.
    \item{$+$} The proof is symmetric to the proof of the negative case.
  \end{itemize}

\end{proof}


\begin{lemma}[Completeness of equivalence] \label{lemma:equiv-completeness}
  Mutual subtyping implies declarative equivalence.
  Assuming all the types below are well-formed in $[[Γ]]$: 
  \begin{itemize}
  \item[$+$] if $[[Γ ⊢ iP ≈ iQ]]$ then $[[iP ≈ iQ]]$,
  \item[$-$] if $[[Γ ⊢ iN ≈ iM]]$ then $[[iN ≈ iM]]$.
  \end{itemize}
\end{lemma}
\begin{proof}
  \begin{itemize}
    \item[$-$] 
    Induction on the sum of sizes of  $[[iN]]$ and $[[iM]]$. 
    By inversion, $[[Γ ⊢ iN ≈ iM]]$ means $[[Γ ⊢ iN ≤ iM]]$ and $[[Γ ⊢ iM ≤ iN ]]$.
    Let us consider the last rule that forms $[[Γ ⊢ iN ≤ iM]]$:
    \begin{caseof}
      \item \ruleref{\ottdruleDOneNVarLabel} i.e. $[[Γ ⊢ iN ≤ iM]]$ is of the form $[[Γ ⊢ α⁻ ≤ α⁻]]$\\
      Then $[[iN ≈ iM]]$ (i.e. $[[α⁻ ≈ α⁻]]$) holds immediately by \ruleref{\ottdruleEOneNVarLabel}.

      \item \ruleref{\ottdruleDOneShiftULabel} i.e. 
      $[[Γ ⊢ iN ≤ iM]]$ is of the form $[[Γ ⊢ ↑iP ≤ ↑iQ]]$\\
      Then by inversion, $[[Γ ⊢ iP ≈ iQ]]$, 
      and by induction hypothesis, $[[iP ≈ iQ]]$.
      Then $[[iN ≈ iM]]$ (i.e. $[[↑iP ≈ ↑iQ]]$) holds 
      by \ruleref{\ottdruleEOneShiftULabel}.

      \item \ruleref{\ottdruleDOneArrowLabel} i.e. $[[Γ ⊢ iN ≤ iM]]$ is of the form $[[Γ ⊢ iP → iN' ≤ iQ → iM']]$\\
      Then by inversion, $[[Γ ⊢ iP ≥ iQ]]$ and $[[Γ ⊢ iN' ≤ iM']]$.
      Notice that $[[Γ ⊢ iM ≤ iN]]$ is of the form $[[Γ ⊢ iQ → iM' ≤ iP → iN']]$, 
      which by inversion means $[[Γ ⊢ iQ ≥ iP]]$ and $[[Γ ⊢ iM' ≤ iN']]$.

      This way, $[[Γ ⊢ iQ ≈ iP]]$ and $[[Γ ⊢ iM' ≈ iN']]$. 
      Then by induction hypothesis, $[[iQ ≈ iP]]$ and $[[iM' ≈ iN']]$.
      Then $[[iN ≈ iM]]$ (i.e. $[[iP → iN' ≈ iQ → iM']]$) holds by \ruleref{\ottdruleEOneArrowLabel}.

      \item \ruleref{\ottdruleDOneForallLabel} i.e. $[[Γ ⊢ iN ≤ iM]]$ is of the form $[[Γ ⊢ ∀pas.iN' ≤ ∀pbs.iM']]$\\
      Then by \cref{lemma:poly-type-equivalence}, $[[Γ ⊢ ∀pas.iN' ≈ ∀pbs.iM']]$ means that 
      there exists a bijection $[[μ : {pbs} ∩ fv iM' ↔ {pas} ∩ fv iN']]$ such that  
      $[[Γ,pas ⊢ [Sub μ]iM' ≈ iN']]$. 
      
      Notice that the application of bijection $[[μ]]$ to $[[iM']]$ does
      not change its size (which is less than the size of $[[iM]]$), hence the induction hypothesis applies.
      This way, $[[ [Sub μ]iM' ≈ iN']]$ (and by symmetry, $[[iN' ≈ [Sub μ]iM']]$) holds by induction. 
      Then we apply \ruleref{\ottdruleEOneForallLabel} to get $[[∀pas.iN' ≈ ∀pbs.iM']]$, i.e. $[[iN ≈ iM]]$.
    \end{caseof}
      
\item[$+$] The proof is symmetric to the proof of the negative case.
  \end{itemize}
\end{proof}

\begin{corollary}[Normalization is complete w.r.t. subtyping-induced equivalence]
  \label{corollary:nf-complete-wrt-subt-equiv}
  Assuming all the types below are well-formed in $[[Γ]]$:
  \begin{itemize}
    \item [$+$] if $[[Γ ⊢ iP ≈ iQ]]$ then $[[nf(iP) = nf(iQ)]]$,
    \item [$-$] if $[[Γ ⊢ iN ≈ iM]]$ then $[[nf(iN) = nf(iM)]]$.
  \end{itemize}
\end{corollary}  
\begin{proof}
  Immediately from \cref{lemma:equiv-completeness,lemma:normalization-completeness}.
\end{proof}

\begin{lemma}[Algorithmization of subtyping-induced equivalence]
  \label{lemma:subt-equiv-algorithmization}
  Mutual subtyping is equality of normal forms.
 Assuming all the types below are well-formed in $[[Γ]]$:
  \begin{itemize}
    \item [$+$] $[[Γ ⊢ iP ≈ iQ]] \iff [[nf(iP) = nf(iQ)]]$,
    \item [$-$] $[[Γ ⊢ iN ≈ iM]] \iff [[nf(iN) = nf(iM)]]$.
  \end{itemize}
\end{lemma}
\begin{proof}
  Let us prove the positive case, the negative case is symmetric.
  We prove both directions of $\iff$ separately:
  \begin{itemize}
    \item [$\Rightarrow$] exactly \cref{corollary:nf-complete-wrt-subt-equiv};
    \item [$\Leftarrow$] by \cref{lemma:decl-equiv-algorithmization,lemma:equiv-soundness}.
  \end{itemize}
\end{proof}



\subsection{Upgrade}
Let us consider a type $[[iP]]$ well-formed in $[[Γ]]$.
Some of its $[[Γ]]$-supertypes are also well-formed in a smaller context $[[{Δ} ⊆ Γ]]$.
The upgrade is the operation that returns the least of such supertypes.

\begin{observation}[Upgrade is deterministic]
    \label{obs:upgrade-deterministic}
    Assuming $[[iP]]$ is well-formed in $[[Γ ⊆ Δ]],$\\
    if $[[upgrade Γ ⊢ iP to Δ = iQ]]$ and $[[upgrade Γ ⊢ iP to Δ = iQ']]$ are defined 
    then $[[iQ = iQ']]$.
\end{observation}
\begin{proof}
    It follows directly from \cref{obs:lub-deterministic},
    and the convention that the fresh variables are chosen by a fixed deterministic algorithm
    (\cref{sec:fresh-selection}).
\end{proof}

\begin{lemma}[Soundness of Upgrade]\label{lemma:upgrade-soundness}
    Assuming $[[iP]]$ is well-formed in $[[Γ = Δ, pnas]]$,
    if $[[upgrade Γ ⊢ iP to Δ = iQ]]$
    then
    \begin{enumerate}
        \item $[[Δ ⊢ iQ]]$
        \item $[[Γ ⊢ iQ ≥ iP]]$
    \end{enumerate}
\end{lemma}
\begin{proof}
    By inversion, $[[upgrade Γ ⊢ iP to Δ = iQ]]$ means that 
    for fresh $[[pnbs]]$ and $[[pncs]]$,
    $[[Δ, pnbs, pncs ⊨ [pnbs/pnas]iP ∨ [pncs/pnas]iP = iQ]]$.
    Then by the soundness of the least upper bound (\cref{lemma:lub-soundness}),
    \begin{enumerate}
        \item $[[Δ, pnbs, pncs ⊢ iQ]]$, 
        \item $[[Δ, pnbs, pncs ⊢ iQ ≥ [pnbs/pnas]iP]]$, and 
        \item $[[Δ, pnbs, pncs ⊢ iQ ≥ [pncs/pnas]iP]]$.
    \end{enumerate}

    $ 
    \begin{aligned}
        [[fv iQ]] &\subseteq [[fv [pnbs/pnas]iP ∩ fv [pncs/pnas]iP]]
                  &&\text{since by \cref{lemma:fv-propagation}, 
                         $[[fv iQ ⊆ fv [pnbs/pnas]iP]]$ and
                         $[[fv iQ ⊆ fv [pncs/pnas]iP]]$}\\
                  &\subseteq [[ ((fv iP \ {pnas}) ∪ {pnbs}) ∩ ((fv iP \ {pnas}) ∪ {pncs})]]\\
                  &= [[ (fv iP \ {pnas}) ∩ (fv iP \ {pnas}) ]]
                  &&\text{since $[[pnbs]]$ and $[[pncs]]$ are fresh}\\
                  &= [[ fv iP \ {pnas} ]]\\
                  &\subseteq [[ Γ \ {pnas} ]]
                  &&\text{since $[[iP]]$ is well-formed in $[[Γ]]$}\\
                  &\subseteq [[ {Δ} ]]\\
    \end{aligned}
    $\\
    This way, by \cref{lemma:wf-ctxt-equiv}, $[[Δ ⊢ iQ]]$.
    
    Let us apply $[[pnas/pnbs]]$---the inverse of the substitution $[[ pnbs/pnas ]]$ to 
    both sides of $[[Δ, pnbs, pncs ⊢ iQ ≥ [pnbs/pnas]iP]]$ and 
    by \cref{lemma:subst-pres-subt} 
    (since $[[pnbs/pnas]]$ can be specified as 
    $[[Δ,pnbs,pncs ⊢ pnbs/pnas : Δ, pnas, pncs]]$ by \cref{lemma:subst-domain-weakening})
    obtain $[[Δ, pnas, pncs ⊢ [pnas/pnbs]iQ ≥ iP]]$.
    Notice that $[[Δ ⊢ iQ]]$ implies that $[[fv iQ ∩ {pnbs} = ∅]]$, 
    then by \cref{corollary:subst-disj}, $[[ [pnas/pnbs]iQ = iQ]]$,
     and thus $[[Δ, pnas, pncs ⊢ iQ ≥ iP]]$.
    By context strengthening, $[[Δ, pnas ⊢ iQ ≥ iP]]$.
\end{proof}

\begin{lemma}[Completeness and Initiality of Upgrade] \label{lemma:upgrade-completeness}
    The upgrade returns the least $[[Γ]]$-supertype of $[[iP]]$ well-formed in $[[Δ]]$.
    Assuming $[[iP]]$ is well-formed in $[[Γ = Δ, pnas]],$\\
    For any $[[iQ']]$ such that 
    \begin{enumerate}
        \item $[[Δ ⊢ iQ']]$ and
        \item $[[Γ ⊢ iQ' ≥ iP]]$,
    \end{enumerate}

    The result of the upgrade algorithm $[[iQ]]$ exists
    ($[[upgrade Γ ⊢ iP to Δ = iQ]]$) and satisfies $[[Δ ⊢ iQ' ≥ iQ]]$.
\end{lemma}
\begin{proof}

    Let us consider fresh (not intersecting with $[[Γ]]$) $[[pnbs]]$ and $[[pncs]]$.

    If we apply substitution $[[pnbs/pnas]]$ to both sides of $[[Δ, pnas ⊢ iQ' ≥ iP]]$,
    we have $[[Δ, pnbs ⊢ [pnbs/pnas]iQ' ≥ [pnbs/pnas]iP]]$, which by  
    \cref{corollary:subst-disj}, since $[[pnas]]$ is disjoint from $[[fv(iQ')]]$
    (because $[[Δ ⊢ iQ']]$), simplifies to $[[Δ, pnbs ⊢ iQ' ≥ [pnbs/pnas]iP]]$.

    Analogously, if we apply substitution $[[pncs/pnas]]$ to both sides of $[[Δ, pnas ⊢ iQ' ≥ iP]]$,
    we have $[[Δ, pncs ⊢ iQ' ≥ [pncs/pnas]iP]]$.

    This way, $[[iQ']]$ is a common supertype of $[[ [pnbs/pnas]iP ]]$ and $[[ [pncs/pnas]iP ]]$ in
    context $[[Δ, pnbs, pncs]]$. It means that we can apply the completeness of the least upper bound
    (\cref{lemma:lub-completeness}):
    \begin{enumerate}
        \item there exists $[[iQ]]$ s.t. $[[Γ ⊨ [pnbs/pnas]iP ∨ [pncs/pnas]iP = iQ]]$ 
        \item $[[Γ ⊢ iQ' ≥ iQ]]$.
    \end{enumerate}
    The former means that the upgrade algorithm terminates and returns $[[iQ]]$.
    The latter means that since both 
    $[[iQ']]$ and $[[iQ]]$ are well-formed in $[[Δ]]$ and $[[Γ]]$,
    by \cref{lemma:subt-ctxt-irrelevance}, $[[Δ ⊢ iQ' ≥ iQ]]$.
\end{proof}




\subsection{Upper Bounds}

\begin{lemma}[Decomposition of the quantifier rule]
  \ilyam{move somewhere}
  \label{lemma:qant-rule-decomposition}
  Whenever the quantifier rule (\ruleref{\ottdruleDOneExistsLabel} or
  \ruleref{\ottdruleDOneForallLabel}) is applied, one can assume that the rule
  adding quantifiers on the right-hand side was applied the last.

  \begin{itemize}
  \item[$-$]
    If $[[G ⊢ iN ≤ ∀pbs.iM]]$ then $[[G, pbs ⊢ iN ≤ iM]]$.
    \item[$+$]
      If $[[G ⊢ iP ≥ ∃nbs.iQ]]$ then $[[G, nbs ⊢ iP ≥ iQ]]$.
  \end{itemize}
\end{lemma}

\begin{lemma}[Shape of the Supertypes]
  \label{lemma:shape-of-supertypes}
  Let us define the
  set of upper bounds of a positive type $\UB([[iP]])$ in the following way:

  \hfill

  \begin{tabular}{@{}lr@{}} \toprule
    % supertypes of ... & ... are \\ 
    $[[G ⊢ iP]]$          & $\UB([[G ⊢ iP]])$ \\ \midrule
    \addlinespace[0.7em]
    $[[ G ⊢ pb ]]$        & $\{[[ ∃nas.pb ]] \ \mid \ \text{for }[[nas]]\}$ \\
    \addlinespace[0.7em]
    $[[ G ⊢ ∃nbs.iQ ]]$   & %$\Set{ [[iQ]] \ | \begin{array}{l} [[iQ]] \in \UB([[iP]]) \\  \text{ s.t. } [[fv iQ ∩ {nbs} = ∅]] \end{array}}$  \\
                            $\UB([[G, nbs ⊢ iQ]])$ not using $[[nbs]]$ \\
    \addlinespace[0.7em]
    $[[ G ⊢ ↓M ]]$        & $\Set{ [[ ∃nas.↓iM' ]] \ | \begin{array}{l}
                                                         \text{for $[[nas]]$, $[[iM']]$, and $[[iNs]]$ s.t. }\\
                                                         \text{$[[G ⊢ iNi]]$, $[[G,nas ⊢ iM']]$,  and $[[ [iNs/nas] ↓iM' ≈ ↓iM ]]$}
                                                       \end{array}}$  \\
  \end{tabular}

  Then $\UB([[G ⊢ iP]]) \equiv \{[[iQ]]\ \mid \ [[G ⊢ iQ ≥ iP]] \}$.
\end{lemma}

\begin{proof}
  By induction on $[[G ⊢ iP]]$.
  \begin{caseof}
  \item $[[iP]] = [[pb]]$\\
    Then the last rule that is applied to infer $[[G ⊢ iQ
    ≥ pb]]$ must be either \ruleref{\ottdruleDOnePVarLabel} or
    \ruleref{\ottdruleDOneExistsLabel}. The former case means that $[[iQ =
    pb]]$. In the latter case, $[[iQ = ∃nas.iQ']]$, where $[[iQ']]$ has no outer
    existential quantifiers. Then by inversion of
    \ruleref{\ottdruleDOneExistsLabel}, $[[ Γ ⊢ [iNs/nas] iQ' ≥ pb]]$ for some $[[iNs]]$.
    This time, to infer this judgment, only \ruleref{\ottdruleDOnePVarLabel} is applicable,
    which means that $[[iQ' = pb]]$, and then $[[iQ = ∃nas.pb]]$.
  \item $[[iP = ∃nbs.iP']]$\\
    Then if $[[G ⊢ iQ ≥ ∃nbs.iP']]$, then by
    \cref{lemma:qant-rule-decomposition}, $[[G, nbs ⊢ iQ ≥ iP']]$, and $[[fv
    iQ ∩ {nbs} = ∅]]$ by the the Barendregt's convention. The other
    direction holds by \ruleref{\ottdruleDOneExistsLabel}. This way,
    $\{[[iQ]] \mid [[G ⊢ iQ ≥ ∃nbs.iP']] \} = \{[[iQ]] \mid  [[G, nbs ⊢ iQ
    ≥ iP']] \text{ s.t. } [[fv(iQ) ∩ {nbs} = ∅]] \}$. From the induction
    hypothesis, the latter is equal to $\UB([[G, nbs ⊢ iP']])$ not using
    $[[nbs]]$, i.e. $\UB([[G ⊢ ∃nbs.iP']])$.
  \item $[[iP = ↓iM]]$\\
    Then let us consider two subcases upper bounds without outer quantifiers (we
    denote the corresponding set restriction as $|_{\not\exists}$) and upper
    bounds with outer quantifiers ($|_{\exists}$). We prove that for both of
    these groups, the restricted sets are equal.
    % ∃a.P(f a) <=> ∃b∊Im(f).P(b)

    \begin{caseof}
      \item \label{case:sup-shape-down-zero}
      $[[iQ]] \neq [[∃nbs.iQ']]$\\
      Then the last applied rule to infer
      $[[G ⊢ iQ ≥ ↓iM]]$ must be \ruleref{\ottdruleDOneShiftDLabel},
      which means $[[iQ]] = [[↓iM']]$, and by inversion, $[[G ⊢ iM' ≈ iM]]$,
      then by \cref{lemma:equiv-completeness} and
      \ruleref{\ottdruleEOneShiftDLabel}, $[[↓iM' ≈ ↓iM]]$.
      This way, $[[iQ]] = [[↓iM']] \in \{ [[↓iM']] \mid [[↓iM' ≈ ↓iM]] \} = \UB([[Γ⊢↓iM]])|_{\not\exists}$.

      In the other direction,
      $
      \begin{aligned}[t]
        [[↓iM' ≈ ↓iM]] &\Rightarrow [[G ⊢ ↓iM' ≈ ↓iM]]
                       && \text{by \cref{lemma:equiv-soundness}, since
                          $[[G ⊢ ↓iM']]$ by \cref{lemma:wf-equiv} }\\
                       &\Rightarrow [[G ⊢ ↓iM' ≥ ↓iM]]
                       && \text{by inversion}
      \end{aligned}
      $
      \item $[[iQ]] = [[∃nbs.iQ']]$ (for non-empty $[[nbs]]$)\\
        Then the last rule applied to infer $[[G ⊢ ∃nbs.iQ' ≥ ↓iM]]$
        must be \ruleref{\ottdruleDOneExistsLabel}.
        Inversion of this rule gives us $[[G ⊢ [iNs/nbs]iQ' ≥ ↓iM]]$
        for some $[[G ⊢ iNi]]$. Notice that $[[ [iNs/nbs]iQ' ]]$ has no outer
        quantifiers. Thus from \cref{case:sup-shape-down-zero},
        $[[ [iNs/nbs]iQ' ≈ ↓iM ]]$, which is only possible if $[[iQ']] = [[↓iM']]$.
        This way, $[[iQ]] = [[∃nbs.↓iM']] \in \UB([[Γ⊢↓iM]])|_{\exists}$ (notice
        that $[[nbs]]$ is not empty).

        In the other direction,
        $
        \begin{aligned}[t]
          [[ [iNs/nbs]↓iM' ≈ ↓iM]] &\Rightarrow [[G ⊢ [iNs/nbs] ↓iM' ≈ ↓iM]]
          && \text{by \cref{lemma:equiv-soundness}, since
             $[[G ⊢ [iNs/nbs] ↓iM']]$ by \cref{lemma:wf-equiv} }\\
                                  &\Rightarrow [[G ⊢ [iNs/nbs]↓iM' ≥ ↓iM]]
         && \text{by inversion}\\
                                  &\Rightarrow [[G ⊢ ∃nbs.↓iM' ≥ ↓iM]] 
         && \text{by \ruleref{\ottdruleDOneExistsLabel}}\\
        \end{aligned}
        $
    \end{caseof}
    
  \end{caseof}
\end{proof}


\begin{lemma}[Normalized Shape of the Supertypes]
  For a normalized positive type $[[iP = nf(iP)]]$,
  let us define the set of normalized upper bounds in the following way:
  
  \hfill

  \begin{tabular}{@{}lr@{}} \toprule
    % supertypes of ... & ... are \\ 
    $[[G ⊢ iP]]$          & $\NFUB([[G ⊢ iP]])$ \\ \midrule
    \addlinespace[0.7em]
    $[[ G ⊢ pb ]]$        & $\{ [[pb]] \}$ \\
    \addlinespace[0.7em]
    $[[ G ⊢ ∃nbs.iP ]]$   & %$\Set{ [[iQ]] \ | \begin{array}{l} [[iQ]] \in \UB([[iP]]) \\  \text{ s.t. } [[fv iQ ∩ {nbs} = ∅]] \end{array}}$  \\
    $\NFUB([[G, nbs ⊢ iP]])$ not using $[[nbs]]$ \\
    \addlinespace[0.7em]
    $[[ G ⊢ ↓M ]]$        & $\Set{ [[ ∃nas.↓iM' ]] \ | \begin{array}{l}
                                                         \text{for $[[nas]]$, $[[iM']]$, and $[[iNs]]$ s.t. $[[ord {nas} in iM' = nas]]$,}\\
                                                         \text{$[[G ⊢ iNi]]$, $[[G,nas ⊢ iM']]$,  and $[[ [iNs/nas] ↓iM' = ↓iM ]]$}
                                                       \end{array}}$  \\
  \end{tabular}

  Then $\NFUB([[G ⊢ iP]]) \equiv \{[[nf(iQ)]]\ \mid \ [[G ⊢ iQ ≥ iP]] \}$.
\end{lemma}


\begin{proof}
  By induction on $[[G ⊢ iP]]$.
  \begin{caseof}
  \item $[[iP]] = [[pb]]$\\
    Then from \cref{lemma:shape-of-supertypes},
    $\{[[nf(iQ)]]\ \mid \ [[G ⊢ iQ ≥ pb]] \} = \{[[ nf(∃nas.pb) ]] \ \mid \
    \text{for some }[[nas]]\}  = \{[[pb]]\}$ 
  \item $[[iP = ∃nbs.iP']]$\\
    $
    \begin{aligned}[t]
      \NFUB([[Γ ⊢ ∃nbs.iP']]) &= \NFUB([[Γ, nbs ⊢ iP']]) \text{ not using $[[nbs]]$}\\
                              &= \{ [[nf(iQ)]] \mid [[Γ, nbs ⊢ iQ ≥ iP']]  \}
                                \text{ not using $[[nbs]]$}
                              && \text{by the induction hypothesis}\\
                              &= \{ [[nf(iQ)]] \mid [[Γ, nbs ⊢ iQ ≥ iP']]
                                \text{ s.t. $[[fv iQ]] \cap [[nbs]] = \emptyset$}
                                \}
                             && \text{because $[[fv nf(iQ)]] = [[fv iQ]]$ by \cref{lemma:fv-nf}}\\
                              &= \{ [[nf(iQ)]] \mid [[iQ]] \in \UB([[Γ, nbs ⊢ iP']]) \text{ s.t. $[[fv iQ]] \cap [[nbs]] = \emptyset$}
                                \}
                            && \text{by \cref{lemma:shape-of-supertypes}}\\
                              &= \{ [[nf(iQ)]] \mid [[iQ]] \in \UB([[Γ ⊢ ∃nbs.iP']])
                                \}
                              && \text{by the definition of $\UB{}$}\\
                              &= \{ [[nf(iQ)]] \mid [[Γ ⊢ iQ ≥ ∃nbs.iP']]
                                \}
                              && \text{by \cref{lemma:shape-of-supertypes}}\\
    \end{aligned}
    $
  
  \item $[[iP = ↓iM]]$\\
    
In the following reasoning, we will use the following principle of variable
replacement.
\begin{observation}
  \label{observation:idemp-replacement}
  Suppose that $\nu : A \rightarrow A$ is an idempotent
  function, $P$ is a predicate on $A$, $F : A \rightarrow B$ is a
  function. Then
 \[ 
  \begin{aligned}[t]
    &\{ F(\nu x ) \mid x \in A \text{ s.t. } P(\nu x) \} =\\
    = &\{ F(x) \mid x \in A \text{ s.t. } \nu x = x \text{ and } P(x) \}.
  \end{aligned}
 \]
\end{observation}
In our case, the idempotent $\nu$ will be normalization, variable ordering, or
domain restriction.

Another observation we will use is the following.
\begin{observation}
  \label{observation:image-replacement}
  For functions $F$ and $\nu$, and
  predicates $P$ and $Q$,
 \[ 
  \begin{aligned}[t]
    &\{F(\nu x) \mid x \in A \text{ s.t. } Q(\nu x) \text{ and } P(x) \} =\\
    = &\{F(\nu x) \mid x \in A \text{ s.t. } Q(\nu x) \text{ and } (\exists x'
        \in A \text{ s.t. } P(x') \text{ and } \nu x' = \nu x) \}.
  \end{aligned}
 \]
\end{observation}
    
    $\begin{aligned}[t]
      &\mathrel{\phantom{=}} \{ [[nf(iQ)]] \mid [[Γ ⊢ iQ ≥ ↓iM]] \}=\\
      %
      %
      &=\{ [[nf(iQ)]] \mid [[iQ]] \in \UB([[ Γ ⊢ ↓iM]])\}
      && \text{by \cref{lemma:shape-of-supertypes}}\\
      %
      %
      &= \Set{[[ nf(∃nas.↓iM') ]] \ |
      \begin{array}{l}
        \text{for $[[nas]]$, $[[iM']]$, and $[[iNs]]$ s.t.  $[[G,nas ⊢ iM']]$,}\\
        \text{$[[G ⊢ iNi]]$, and $[[ [iNs/nas] ↓iM' ≈ ↓iM ]]$}\\
      \end{array}}
      && \text{by the definition of $\UB$}\\
      %
      %
      &= \Set{[[ nf(∃nas.↓iM') ]] \ |
        \begin{array}{l}
          \text{for $[[nas]]$, $[[iM']]$, and $[[σ]]$ s.t.  $[[G,nas ⊢ iM']]$,}\\
          \text{$[[G ⊢ σ : nas]]$, and $[[ [σ] ↓iM' ≈ ↓iM ]]$}
        \end{array}}
      && \text{we reassigned the substitution $[[ iNs/nas ]]$ as $[[ σ ]]$}\\
      %
      %
      &= \Set{[[ nf(∃nas.↓iM') ]] \ |
        \begin{array}{l}
          \text{for $[[nas]]$, $[[iM']]$, and $[[σ]]$ s.t. $[[G,nas ⊢ iM']]$,}\\
          \text{$[[G ⊢ σ : nas]]$, and $[[ [σ|fv iM' ] ↓iM' ≈ ↓iM ]]$}
        \end{array}}
      && \text{by \cref{lemma:subst-restr-fv}}\\
      % 
      % 
      &= \Set{[[ ∃nas'.nf(↓iM') ]] \ |
      \begin{array}{l}
        \text{for $[[nas']]$, $[[nas]]$, $[[iM']]$, $[[σ]]$ s.t. $[[G,nas ⊢ iM']]$,}\\
        \text{$[[G ⊢ σ : nas]]$,\, $[[ord {nas} in iM' = nas']]$}\\
        \text{and $[[ [σ|fv iM'] ↓iM' ≈ ↓iM ]]$}
      \end{array}}
      && \text{by the definition of normalization}\\
      %
      %
      &= \Set{[[ ∃nas'.nf(↓iM') ]] \ |
      \begin{array}{l}
        \text{for $[[nas']]$, $[[nas]]$, $[[iM']]$, $[[σ]]$ s.t. $[[G,nas ⊢ iM']]$, }\\
        \text{$[[G ⊢ σ : nas]]$,\, $[[ord {nas} in iM' = nas']]$}\\
        \text{and $[[ nf([σ|fv iM'] ↓iM') = nf(↓iM) ]]$}
      \end{array}}
      &&\begin{array}{l}
         \text{from
          \cref{lemma:normalization-soundness,lemma:normalization-completeness},
          equivalence of types can be}\\
          \text{replaced with the equality of their normal forms}
        \end{array} \\
      %
      %
      &= \Set{[[ ∃nas'.nf(↓iM') ]] \ |
        \begin{array}{l}
          \text{for $[[nas']]$, $[[nas]]$, $[[iM']]$, $[[σ]]$ s.t. $[[G,nas ⊢ iM']]$,}\\
          \text{$[[G ⊢ σ : nas]]$,\, $[[ord {nas} in iM' = nas']]$}\\
          \text{and $[[ [nf(σ|fv iM')] ↓nf(iM') = ↓nf(iM) ]]$}
        \end{array}}
      && \text{by congruence of normalization and \cref{lemma:norm-subst-distr}}\\
      %
      %
      &= \Set{[[ ∃nas'.↓iM' ]] \ |
        \begin{array}{l}
          \text{for $[[nas']]$, $[[nas]]$, $[[iM']]$, $[[σ]]$ s.t. $[[G,nas ⊢ iM']]$,}\\
          \text{$[[G ⊢ σ : nas]]$,\, $[[ord {nas} in iM' = nas']]$}\\
          \text{and $[[ [σ|fv iM'] ↓iM' = ↓iM ]]$}
        \end{array}}
      &&\begin{array}{l}
         \text{by \cref{lemma:normal-after-subst}, $[[↓iM']]$ and $[[σ|fv
          iM']]$ are already normal,}\\
          \text{since the result of the substitution is normal;}\\
          \text{$[[iM]]$ is normal by assumption}
         \end{array}\\
      % 
      % 
      &= \Set{[[ ∃nas'.↓iM' ]] \ |
        \begin{array}{l}
          \text{for $[[nas']]$, $[[nas]]$, $[[iM']]$, $[[σ]]$ s.t.  $[[G,nas ⊢ iM']]$,}\\
          \text{($\exists [[σ']]$ s.t. $[[G ⊢ σ' : nas]]$ and $[[σ|fv(↓iM')]] =
          [[σ'|fv(↓iM')]]$)}\\
          \text{$[[ord {nas} in iM' = nas']]$ and $[[ [σ|fv iM'] ↓iM' = ↓iM ]]$}\\
        \end{array}}
      &&\begin{array}{l}
      \text{We apply \cref{observation:image-replacement} (with $\nu [[σ]]$
      =  $[[σ|fv iM']]$, and}\\
      \text{$P([[σ]]) = [[G ⊢ σ : nas]]$)}
      \end{array}\\
      %
      %
      &= \Set{[[ ∃nas'.↓iM' ]] \ |
        \begin{array}{l}
          \text{for $[[nas']]$, $[[nas]]$, $[[iM']]$, $[[σ]]$ s.t.  $[[G,nas ⊢ iM']]$,}\\
          \text{$[[G ⊢ σ|fv iM' : nas']]$,\, $[[ord {nas} in iM' = nas']]$}\\
          \text{and $[[ [σ|fv iM'] ↓iM' = ↓iM ]]$}
        \end{array}}
      &&\begin{array}{l}
           \text{Notice that}\\
           \text{``$\exists [[σ']]$ s.t. ($[[G ⊢ σ' : nas]]$ and $[[σ|fv(↓iM')]] =
           [[σ'|fv(↓iM')]]$)''}\\
           \text{is equivalent to $[[G ⊢ σ|fv(↓iM') : nas']]$}\\
         \end{array} \\
         % 
         % 
      &= \Set{[[ ∃nas'.↓iM' ]] \ |
        \begin{array}{l}
          \text{for $[[nas']]$, $[[nas]]$, $[[iM']]$, $[[σ]]$ s.t.  $[[G,nas ⊢ iM']]$,}\\
          \text{$[[G ⊢ σ : nas']]$,\, $[[ord {nas} in iM' = nas']]$}\\
          \text{and $[[ [σ] ↓iM' = ↓iM ]]$}
        \end{array}}
      && \begin{array}{l}
         \text{We apply \cref{observation:idemp-replacement} to the
           restriction of $[[σ]]$, and then}\\
         \text{remove $[[σ|fv iM']] = [[σ]]$  as it follows from $[[G ⊢ σ : nas']]$}\\
         \end{array}\\
         % 
         % 
      &= \Set{[[ ∃nas'.↓iM' ]] \ |
        \begin{array}{l}
          \text{for $[[nas]]$, $[[nas']]$, $[[iM']]$, $[[σ]]$ s.t.  $[[G,nas' ⊢ iM']]$,}\\
          \text{$[[G ⊢ σ : nas']]$,\, $[[ord {nas} in iM' = nas']]$}\\
          \text{and $[[ [σ] ↓iM' = ↓iM ]]$}
        \end{array}}
      &&
         \begin{array}{l}
           \text{From \cref{lemma:wf-ctxt-equiv}, since
           $[[ {Γ, nas} ∩ fv iM' ]] = [[ {Γ, nas'} ∩ fv iM' ]]$}\\
         \end{array}\\
      % 
      % 
      &= \Set{[[ ∃nas'.↓iM' ]] \ |
        \begin{array}{l}
          \text{for $[[nas']]$,  $[[iM']]$, $[[σ]]$ s.t.  $[[G,nas' ⊢ iM']]$,}\\
          \text{$[[G ⊢ σ : nas']]$,\, $[[ord {nas'} in iM' = nas']]$}\\
          \text{and $[[ [σ] ↓iM' = ↓iM ]]$}
        \end{array}}
      && \text{We apply \cref{observation:idemp-replacement} to the ordering of $[[nas]]$}\\
      &= \Set{ [[ ∃nas.↓iM' ]] \ |
        \begin{array}{l}
          \text{for $[[nas']]$, $[[iM']]$, and $[[iNs]]$ s.t. $[[ord {nas'} in iM' = nas]]$,}\\
          \text{$[[G ⊢ iNi]]$, $[[G,nas' ⊢ iM']]$,  and $[[ [iNs/nas'] ↓iM' = ↓iM ]]$}
        \end{array}} 
      && \text{By reassigning $[[σ]]$ explicitly as $[[iNs/nas']]$}\\
      %
      %
    \end{aligned}$
  \end{caseof}
\end{proof}

\begin{lemma}[Soundness of the Least Upper Bound]
  For types $[[Γ ⊢ iP1]]$, and $[[Γ ⊢ iP2]]$,
  if $[[Γ ⊨ iP1 ∨ iP2 = iQ]]$ then
  \begin{enumerate}
    \item[(i)]  $[[Γ ⊢ iQ]]$
    \item[(ii)] $[[Γ ⊢ iQ ≥ iP1]]$ and $[[Γ ⊢ iQ ≥ iP2]]$
  \end{enumerate}
\end{lemma}

\begin{lemma}[Completeness of the Least Upper Bound]
  For types $[[Γ ⊢ iP1]]$, $[[Γ ⊢ iP2]]$, and $[[Γ ⊢ iQ']]$
  such that $[[Γ ⊢ iQ' ≥ iP1]]$ and $[[Γ ⊢ iQ' ≥ iP2]]$,
  there exists $[[iQ]]$ s.t. $[[Γ ⊨ iP1 ∨ iP2 = iQ]]$, and
  $[[Γ ⊢ iQ' ≥ iQ]]$
\end{lemma}

\begin{lemma}[Soundness of Upgrade]
  For $[[Δ]] \subseteq [[Γ]]$,
  suppose that $[[upgrade Γ ⊢ iP to Δ = iQ]]$.
  Then
  \begin{enumerate}
    \item [(i)] $[[Δ ⊢ iQ]]$
    \item [(ii)] $[[Γ ⊢ iQ ≥ iP]]$
  \end{enumerate}
\end{lemma}

\begin{lemma}[Completeness of Upgrade]
  For $[[Δ]] \subseteq [[Γ]]$,
  $[[Γ ⊢ iP]]$ and $[[Δ ⊢ iQ']]$,
  such that $[[Γ ⊢ iQ' ≥ iP]]$,
  there exists $[[iQ]]$ s.t.
  $[[upgrade Γ ⊢ iP to Δ = iQ]]$, and
  $[[Δ ⊢ iQ' ≥ iQ]]$.
\end{lemma}



\subsection{Unification}
\begin{lemma}[Soundness of Unification] \label{lemma:unification-soundness}
    \hfill
    \begin{itemize}
        \item [$+$] For normalized $[[uP]]$ and $[[iQ]]$ such that 
        $[[Γ ; Θ ⊢ uP]]$ and $[[Γ ⊢ iQ]]$,\\ 
        if $[[Γ ; Θ ⊨ uP ≈u iQ ⫤ UC]]$ then 
        $[[Θ ⊢ UC]]$ and for any normalized $[[uσ]]$ such that $[[Θ ⊢ uσ : lift UC]]$,
        $[[ [uσ]uP = iQ ]]$.

        \item [$-$] For normalized $[[uN]]$ and $[[iM]]$ such that
        $[[Γ ; Θ ⊢ uN]]$ and $[[Γ ⊢ iM]]$,\\
        if $[[Γ ; Θ ⊨ uN ≈u iM ⫤ UC]]$ then 
        $[[Θ ⊢ UC]]$ and for any normalized $[[uσ]]$ such that $[[Θ ⊢ uσ : lift UC]]$,
        $[[ [uσ]uN = iM ]]$.
    \end{itemize}
\end{lemma}
\begin{proof}
    We prove by induction on the derivation of 
    $[[ Γ ; Θ ⊨ uN ≈u iM ⫤ UC ]]$ and mutually $[[Γ ; Θ ⊨ uP ≈u iQ ⫤ UC]]$.
    Let us consider the last rule forming this derivation. 
    \begin{caseof}
        \item \ruleref{\ottdruleUNVarLabel}, then $[[uN]] = [[α⁻]] = [[iM]]$.
        The resulting unification constraint is empty: $[[UC]] = [[·]]$.
        It satisfies $[[Θ ⊢ UC]]$ vacuously, and $[[ [us]α⁻ = α⁻ ]]$, that is $[[ [us]uN = iM ]]$.

        \item \ruleref{\ottdruleUShiftULabel}, then $[[uN]] = [[↑uP]]$ and $[[iM]] = [[↑iQ]]$.
        The algorithm makes a recursive call to $[[Γ ; Θ ⊨ uP ≈u iQ ⫤ UC]]$ returning $[[UC]]$.
        By induction hypothesis, $[[Θ ⊢ UC]]$ and for any $[[Θ ⊢ uσ : lift UC]]$,
        $[[ [uσ]uN ]] = [[ [uσ]↑uP ]] = [[ ↑[uσ]uP ]] = [[ ↑iQ ]] = [[ iM ]]$, as 
        required.

        \item \ruleref{\ottdruleUArrowLabel}, then $[[uN]] = [[uP → uN']]$ and $[[iM]] = [[iQ → iM']]$.
        The algorithm makes two recursive calls to $[[Γ ; Θ ⊨ uP ≈u iQ ⫤ UC1]]$ and
        $[[Γ ; Θ ⊨ uN' ≈u iM' ⫤ UC2]]$ returning $[[Θ ⊢ UC1 & UC2 = UC]]$ as the result.

        It is clear that $[[uP]]$, $[[uN']]$, $[[iQ]]$, and $[[iM']]$ are normalized,
        and that $[[Γ ; Θ ⊢ uP]]$, $[[Γ ; Θ ⊢ uN']]$, $[[Γ ⊢ iQ]]$, and $[[Γ ⊢ iM']]$.
        This way, the induction hypothesis is applicable to both recursive calls.

        By applying the induction hypothesis to $[[Γ ; Θ ⊨ uP ≈u iQ ⫤ UC1]]$,
        we have:
        \begin{itemize}
            \item $[[Θ ⊢ UC1]]$,
            \item for any $[[Θ ⊢ uσ' : lift UC1]]$, $[[ [uσ']uP = iQ ]]$.
        \end{itemize}
        By applying it to $[[Γ ; Θ ⊨ uN' ≈u iM' ⫤ UC2]]$, we have:
        \begin{itemize}
            \item $[[Θ ⊢ UC2]]$,
            \item for any $[[Θ ⊢ uσ' : lift UC2]]$, $[[ [uσ']uN' = iM' ]]$.
        \end{itemize}


        Let us take an arbitrary $[[Θ ⊢ uσ : lift UC]]$.
        By the soundness of the constraint merge (\cref{lemma:merge-soundness}), 
        $[[Θ ⊢ lift UC1 & lift UC2 = lift UC]]$ implies
        $[[Θ ⊢ uσ : lift UC1 ]]$ and $[[Θ ⊢ uσ : lift UC2]]$.

        Applying the induction hypothesis to $[[Θ ⊢ uσ : lift UC1]]$, we have
        $[[ [uσ]uP = iQ ]]$; applying it to $[[Θ ⊢ uσ : lift UC2]]$, we have
        $[[ [uσ]uN' = iM' ]]$.
        This way, $[[ [uσ]uN ]] = [[ [uσ]uP → [uσ]uN' ]] = [[ iQ → iM' ]] = [[ iM ]]$.

        \item \ruleref{\ottdruleUForallLabel}, then $[[uN]] = [[∀pas.uN']]$ and $[[iM]] = [[∀pas.iM']]$.
        The algorithm makes a recursive call to $[[Γ,pas ; Θ ⊨ uN' ≈u iM' ⫤ UC]]$
        returning $[[UC]]$ as the result.

        The induction hypothesis is applicable: $[[Γ,pas ; Θ ⊢ uN']]$ and $[[Γ,pas ⊢ iM']]$ hold
        by inversion, and $[[uN']]$ and $[[iM']]$ are normalized, since $[[uN]]$ and $[[iM]]$ are.
        Let us take an arbitrary $[[Θ ⊢ uσ : lift UC]]$.
        By the induction hypothesis, $[[ [uσ]uN' ]] = [[ iM' ]]$. 
        Then $[[ [uσ]uN ]] = [[ [uσ]∀pas.uN' ]] = [[ ∀pas.[uσ]uN' ]] = [[ ∀pas.iM' ]] = [[ iM ]]$.

        \item \ruleref{\ottdruleUNUVarLabel}, then $[[uN]] = [[α̂⁻]]$, $[[â⁻[Δ] ∊ Θ]]$, and $[[Δ ⊢ iM]]$.
        As the result, the algorithm returns $[[UC]] = [[ (â⁻ :≈ iM) ]]$.

        It is clear that $[[α̂⁻[Δ] ⊢ (â⁻ :≈ iM) ]]$, since $[[Δ ⊢ iM]]$, 
        meaning that $[[Θ ⊢ UC]]$.

        Let us take an arbitrary $[[uσ]]$ such that  $[[Θ ⊢ uσ : lift UC]]$.
        Since $[[UC]] = [[ (â⁻ :≈ iM) ]]$, $[[Θ ⊢ uσ : lift UC]]$ implies 
        $[[Θ(â⁻) ⊢ [uσ]â⁻ : (â⁻ :≈ iM) ]]$.
        By inversion of \ruleref{\ottdruleSATSCENEqLabel}, it  means $[[Θ(â⁻) ⊢ [uσ]â⁻ ≈ iM]]$.
        This way, $[[Θ(â⁻) ⊢ [uσ]uN ≈ iM]]$. 
        Notice that $[[uσ]]$ and $[[uN]]$ are normalized, and by \cref{lemma:norm-subst-distr}, 
        so is $[[ [uσ]uN ]]$.
        Since both sides of $[[Θ(â⁻) ⊢ [uσ]uN ≈ iM]]$ are normalized,
        by \cref{lemma:subt-equiv-algorithmization}, we have $[[ [uσ]uN = iM ]]$.

        \item The positive cases are proved symmetrically.
    \end{caseof}
\end{proof}

\begin{lemma}[Completeness of Unification] \label{lemma:unification-completeness}
    \hfill
    \begin{itemize}
        \item [$+$] For normalized $[[uP]]$ and $[[iQ]]$ such that
        $[[Γ ; Θ ⊢ uP]]$ and $[[Γ ⊢ iQ]]$, 
        for any $[[Θ ⊢ uσ]]$ such that $[[ [uσ]uP = iQ ]]$,
        there exists $[[Γ ; Θ ⊨ uP ≈u iQ ⫤ UC]]$,
        and $[[Θ ⊢ uσ : lift UC]]$.
        
        \item [$-$] For normalized $[[uN]]$ and $[[iM]]$ such that
        $[[Γ ; Θ ⊢ uN]]$ and $[[Γ ⊢ iM]]$,\\
        for any $[[Θ ⊢ uσ]]$ such that $[[ [uσ]uN = iM ]]$,
        there exists $[[Γ ; Θ ⊨ uN ≈u iM ⫤ UC]]$,
        and $[[Θ ⊢ uσ : lift UC]]$.
   \end{itemize}
\end{lemma}
\begin{proof}
    We prove it by induction on the structure of $[[uP]]$ and mutually, $[[uN]]$.
    \begin{caseof}
        \item $[[uN]] = [[α̂⁻]]$\\
            $[[Γ ; Θ ⊢ α̂⁻]]$ means that $[[ α̂⁻[Δ] ]] \in [[Θ]]$ for some $[[Δ]]$.

            Let us take an arbitrary $[[Θ ⊢ uσ]]$ such that $[[ [uσ]α̂⁻ = iM ]]$.
            $[[Θ ⊢ uσ]]$ means that $[[Δ ⊢ iM]]$.
            This way, \ruleref{\ottdruleUNUVarLabel} is applicable to infer 
            $[[Γ ; Θ ⊨ â⁻ ≈u iM ⫤ (â⁻ :≈ iM)]]$.
            $[[Θ ⊢ uσ : lift (â⁻ :≈ iM)]]$ holds by \ruleref{\ottdruleSATSCENEqLabel}. 
            
        \item $[[uN]] = [[α⁻]]$\\
            Let us take an arbitrary $[[Θ ⊢ uσ]]$ such that $[[ [uσ]α⁻ = iM ]]$.
            The latter means $[[iM = α⁻]]$.

            Then $[[ [us]α⁻ = iM ]]$ means $[[iM = α⁻]]$.
            This way, \ruleref{\ottdruleUNVarLabel} infers 
            $[[Γ; Θ ⊨ a⁻ ≈u a⁻ ⫤ ·]]$, which is rewritten as $[[Γ; Θ ⊨ uN ≈u iM ⫤ ·]]$, 
            and $[[Θ ⊢ uσ : lift ·]]$ holds trivially.

        \item $[[uN]] = [[↑uP]]$\\
            Let us take an arbitrary $[[Θ ⊢ uσ]]$ such that $[[ [uσ]↑uP = iM ]]$.
            The latter means $[[ ↑[uσ]uP = iM ]]$, i.e.
            $[[iM]] = [[↑iQ]]$ for some $[[iQ]]$ and $[[ [uσ]uP = iQ ]]$.

            Let us show that the induction hypothesis is applicable to $[[ [uσ]uP = iQ ]]$.
            Notice that $[[uP]]$ is normalized, since $[[uN]] = [[↑uP]]$ is normalized,
            $[[Γ ; Θ ⊢ uP]]$ holds by inversion of $[[Γ ; Θ ⊢ ↑uP]]$, 
            and $[[Γ ⊢ iQ]]$ holds by inversion of $[[Γ ⊢ ↑iQ]]$.

            This way, by the induction hypothesis there exists $[[UC]]$ such that
            $[[Γ ; Θ ⊨ uP ≈u iQ ⫤ UC]]$, and moreover, $[[Θ ⊢ uσ : lift UC]]$.
            
        \item $[[uN]] = [[uP → uN']]$\\
            Let us take an arbitrary $[[Θ ⊢ uσ]]$ such that $[[ [uσ](uP → uN') = iM ]]$.
            The latter means $[[ [uσ]uP → [uσ]uN' = iM ]]$, i.e.
            $[[iM]] = [[iQ → iM']]$ for some $[[iQ]]$ and $[[iM']]$, 
            such that $[[ [uσ]uP = iQ ]]$ and $[[ [uσ]uN' = iM' ]]$.

            Let us show that the induction hypothesis is applicable to 
            $[[ [uσ]uP = iQ ]]$ and to $[[ [uσ]uN' = iM' ]]$:
            \begin{itemize}
                \item $[[uP]]$ and $[[uN']]$ are normalized, since $[[uN]] = [[uP → uN']]$ is normalized
                \item $[[Γ ; Θ ⊢ uP]]$ and $[[Γ ; Θ ⊢ uN']]$ follow from the inversion of $[[Γ ; Θ ⊢ uP → uN']]$,
                \item $[[Γ ⊢ iQ]]$ and $[[Γ ⊢ iM']]$ follow from inversion of $[[Γ ⊢ iQ → iM']]$.
            \end{itemize}

            Then by the induction hypothesis, $[[Γ ; Θ ⊨ uP ≈u iQ ⫤ UC1]]$ and $[[Θ ⊢ uσ : lift UC1]]$,
            $[[Γ ; Θ ⊨ uN' ≈u iM' ⫤ UC2]]$ and $[[Θ ⊢ uσ : lift UC2]]$.
            To apply \ruleref{\ottdruleUArrowLabel} and infer the required
            $[[Γ ; Θ ⊨ uN ≈u iM ⫤ UC]]$, we need to show that
            $[[Θ ⊢ UC1 & UC2 = UC]]$ is defined and $[[Θ ⊢ uσ : lift UC]]$.
            It holds by the completeness of the unification constraint merge 
            (\cref{lemma:merge-completeness}):
            \begin{itemize}
                \item $[[Θ ⊢ UC1]]$ and $[[Θ ⊢ UC2]]$ holds by the soundness of unification (\cref{lemma:unification-soundness})
                \item $[[Θ ⊢ uσ : lift UC1]]$ and $[[Θ ⊢ uσ : lift UC2]]$ holds as noted above 
            \end{itemize}.

        \item $[[uN]] = [[∀pas.uN']]$\\
            Let us take an arbitrary $[[Θ ⊢ uσ]]$ such that $[[ [uσ]∀pas.uN' = iM ]]$.
            The latter means $[[ ∀pas.[uσ]uN' = iM ]]$, i.e.
            $[[iM]] = [[∀pas.iM']]$ for some $[[iM']]$ such that $[[ [uσ]uN' = iM' ]]$.

            Let us show that the induction hypothesis is applicable to $[[ [uσ]uN' = iM' ]]$.
            Notice that $[[uN']]$ is normalized, since $[[uN]] = [[∀pas.uN']]$ is normalized,
            $[[Γ,pas ; Θ ⊢ uN']]$ follows from inversion of $[[Γ ; Θ ⊢ ∀pas.uN']]$,
            $[[Γ,pas ⊢ iM']]$ follows from inversion of $[[Γ ⊢ ∀pas.iM']]$, and
            $[[Θ ⊢ uσ]]$ by assumption. 

            This way, by the induction hypothesis, $[[Γ,pas ; Θ ⊨ uN' ≈u iM' ⫤ UC]]$ exists and 
            moreover, $[[Θ ⊢ uσ : lift UC]]$.
            Hence, \ruleref{\ottdruleUForallLabel} is applicable to infer
            $[[Γ ; Θ ⊨ ∀pas.uN' ≈u ∀pas.iM' ⫤ UC]]$, that is $[[Γ ; Θ ⊨ uN ≈u iM ⫤ UC]]$.

        \item The positive cases are proved symmetrically.
    \end{caseof}
\end{proof}
\

\subsection{Anti-unification}
\begin{lemma}[Soundness of Anti-Unification] \label{lemma:anti-unification-soundness}
    \hfill
    \begin{itemize}
        \item [$+$]  Assuming $[[iP1]]$ and $[[iP2]]$ are normalized,
        if $[[Γ ⊨ iP1 ≈au iP2 ⫤ (Ξ, uQ, aus1, aus2)]]$
        then 
        \begin{enumerate}
            \item $[[Γ ; Ξ ⊢ uQ]]$,
            \item $[[Γ ; · ⊢ ausi : Ξ]]$ for $i \in \{1,2\}$, and
            \item $[[ [ausi] uQ = iPi ]]$ for $i \in \{1,2\}$.
        \end{enumerate}

        \item [$-$] Assuming $[[iN1]]$ and $[[iN2]]$ are normalized,
        if $[[Γ ⊨ iN1 ≈au iN2 ⫤ (Ξ, uM, aus1, aus2)]]$
        then
        \begin{enumerate}
            \item $[[Γ ; Ξ ⊢ uM]]$,
            \item $[[Γ ; · ⊢ ausi : Ξ]]$ for $i \in \{1,2\}$, and
            \item $[[ [ausi] uM = iNi ]]$ for $i \in \{1,2\}$.
        \end{enumerate}
    \end{itemize}
\end{lemma}
\begin{proof}
    We prove it by induction on 
    $[[Γ ⊨ iN1 ≈au iN2 ⫤ (Ξ, uM, aus1, aus2)]]$
    and mutually, $[[Γ ⊨ iP1 ≈au iP2 ⫤ (Ξ, uQ, aus1, aus2)]]$.
    Let us consider the last rule applied to infer this judgement.
    \begin{caseof}
        \item \ruleref{\ottdruleAUNVarLabel}, then $[[iN1]] = [[α⁻]] = [[iN2]]$,
              $[[Ξ]] = [[·]]$, $[[uM]] = [[α⁻]]$, and $[[aus1]] = [[aus2]] = [[·]]$.
            \begin{enumerate}
                \item $[[Γ ; · ⊢ α⁻]]$ follows from the assumption $[[Γ ⊢ α⁻]]$,
                \item $[[Γ ; · ⊢ · : ·]]$ holds trivially, and
                \item $[[ [·] α⁻ = α⁻ ]]$ holds trivially.
            \end{enumerate}
        \item \label{case:anti-unification-soundness:shiftu}
         \ruleref{\ottdruleAUShiftULabel}, then $[[iN1]] = [[↑iP1]]$,
                $[[iN2]] = [[↑iP2]]$, and the algorithm makes the recursive call:
                $[[Γ ⊨ iP1 ≈au iP2 ⫤ (Ξ, uQ, aus1, aus2)]]$, 
                returning $[[(Ξ, ↑uQ, aus1, aus2)]]$ as the result.

                Since $[[iN1]] = [[↑iP1]]$ and $[[iN2]] = [[↑iP2]]$ are normalized, 
                so are $[[iP1]]$ and $[[iP2]]$, and thus, the induction hypothesis 
                is applicable to $[[Γ ⊨ iP1 ≈au iP2 ⫤ (Ξ, uQ, aus1, aus2)]]$:
                \begin{enumerate}
                    \item $[[Γ ; Ξ ⊢ uQ]]$, and hence, $[[Γ ; Ξ ⊢ ↑uQ]]$,
                    \item $[[Γ ; · ⊢ ausi : Ξ]]$ for $i \in \{1,2\}$, and
                    \item $[[ [ausi] uQ = iPi ]]$ for $i \in \{1,2\}$, and then by 
                    the definition of the substitution, $[[ [ausi] ↑uQ = ↑iPi ]]$ for $i \in \{1,2\}$.
                \end{enumerate}

        \item \ruleref{\ottdruleAUArrowLabel}, then $[[iN1]] = [[iP1 → iN'1]]$,
                $[[iN2]] = [[iP2 → iN'2]]$, and the algorithm makes two recursive calls:
                $[[Γ ⊨ iP1 ≈au iP2 ⫤ (Ξ, uQ, aus1, aus2)]]$ and
                $[[Γ ⊨ iN'1 ≈au iN'2 ⫤ (Ξ', uM, aus'1, aus'2)]]$ and
                and returns $[[(Ξ ∪ Ξ', uQ → uM, aus1 ∪ aus'1, aus2 ∪ aus'2)]]$ as the result.
                
                Notice, that the induction hypothesis is applicable to 
                $[[Γ ⊨ iP1 ≈au iP2 ⫤ (Ξ, uQ, aus1, aus2)]]$:
                $[[iP1]]$ and $[[iP2]]$ are normalized, since $[[iN1]] = [[iP1 → iN'1]]$
                and $[[iN2]] = [[iP2 → iN'2]]$ are normalized.
                Similarly, the induction hypothesis is applicable to
                $[[Γ ⊨ iN'1 ≈au iN'2 ⫤ (Ξ', uM, aus'1, aus'2)]]$.

                This way, by the induction hypothesis:
                \begin{enumerate}
                    \item $[[Γ ; Ξ ⊢ uQ]]$ and $[[Γ ; Ξ' ⊢ uM]]$. 
                    Then by weakening (\cref{todo}), $[[Γ ; Ξ ∪ Ξ' ⊢ uQ]]$ and 
                    $[[Γ ; Ξ ∪ Ξ' ⊢ uM]]$, which implies $[[Γ ; Ξ ∪ Ξ' ⊢ uQ → uM]]$;

                    \item $[[Γ ; · ⊢ ausi : Ξ]]$ and $[[Γ ; · ⊢ aus'i : Ξ']]$, 
                           and hence, $[[Γ ; · ⊢ ausi ∪ aus'i : Ξ ∪ Ξ']]$ (\cref{todo});

                    \item $[[ [ausi] uQ = iPi ]]$ and $[[ [aus'i] uM = iN'i ]]$.
                    Since $[[ausi ∪ aus'i]]$ restricted to $[[Ξ]]$ is $[[ausi]]$,
                    we have $[[ [ausi ∪ aus'i] uQ = iPi ]]$ and 
                    $[[ [ausi ∪ aus'i] uM = iN'i ]]$, and thus, 
                    $[[ [ausi ∪ aus'i] uQ → uM = iP1 → iN'1 ]]$
                \end{enumerate}

        \item \ruleref{\ottdruleAUForallLabel}, then $[[iN1]] = [[∀pas.iN'1]]$,
                $[[iN2]] = [[∀pas.iN'2]]$, and the algorithm makes a recursive call:
                $[[Γ ⊨ iN'1 ≈au iN'2 ⫤ (Ξ, uM, aus1, aus2)]]$ and
                returns $[[(Ξ, ∀pas.uM, aus1, aus2)]]$ as the result.

                Similarly to \cref{case:anti-unification-soundness:shiftu}, 
                we apply the induction hypothesis to
                $[[Γ ⊨ iN'1 ≈au iN'2 ⫤ (Ξ, uM, aus1, aus2)]]$ to obtain:
                \begin{enumerate}
                    \item $[[Γ; Ξ ⊢ uM]]$, and hence, $[[Γ ; Ξ ⊢ ∀pas.uM]]$;
                    \item $[[Γ; · ⊢ ausi : Ξ]]$ for $i \in \{1,2\}$, and
                    \item $[[ [ausi] uM = iN'i ]]$ for $i \in \{1,2\}$,
                        and then by the definition of the substitution,
                        $[[ [ausi] ∀pas.uM = ∀pas.iN'i ]]$ for $i \in \{1,2\}$. 
                \end{enumerate}

        \item \ruleref{\ottdruleAUAULabel}, which applies 
        when other rules do not, and $[[G ⊢ iNi]]$,
        returning as the result $[[(Ξ, uM, aus1, aus2)]] = $
        $[[(â⁻_{iN1, iN2}, â⁻_{iN1, iN2}, (â⁻_{iN1, iN2} :≈ iN1) ,  (â⁻_{iN1, iN2} :≈ iN2))]]$.

        \begin{enumerate}
            \item $[[Γ ; Ξ ⊢ uM]]$ is rewritten as $[[Γ ; â⁻_{iN1, iN2} ⊢ â⁻_{iN1, iN2}]]$,
                which holds trivially;
            \item $[[Γ ; · ⊢ ausi : Ξ]]$ is rewritten as $[[Γ ; · ⊢ (â⁻_{iN1, iN2} :≈ iNi) : â⁻_{iN1, iN2}]]$,
                which holds since $[[Γ ⊢ iNi]]$ by the premise of the rule;
            \item $[[ [ausi] uM = iNi ]]$ is rewritten as $[[ [â⁻_{iN1, iN2} :≈ iNi] â⁻_{iN1, iN2} = iNi ]]$,
                which holds trivially by the definition of substitution.
        \end{enumerate}

        \item Positive cases are proved symmetrically.
    \end{caseof}
\end{proof}

\begin{observation} \label{obs:names-defined-by-mapping}
    Names of the anti-unification variables are uniquely defined by
    the types they are mapped to by the resulting substitutions. 

    \begin{itemize}
        \item [$+$]  Assuming $[[iP1]]$ and $[[iP2]]$ are normalized,
        if $[[Γ ⊨ iP1 ≈au iP2 ⫤ (Ξ, uQ, aus1, aus2)]]$
        then for any $[[β̂⁻]] \in [[Ξ]]$,
        $[[β̂⁻]] = [[â⁻_{[aus1]β̂⁻, [aus2]β̂⁻}]]$
        \item [$-$]  Assuming $[[iN1]]$ and $[[iN2]]$ are normalized,
        if $[[Γ ⊨ iN1 ≈au iN2 ⫤ (Ξ, uM, aus1, aus2)]]$
        then for any $[[β̂⁻]] \in [[Ξ]]$,
        $[[β̂⁻]] = [[â⁻_{[aus1]β̂⁻, [aus2]β̂⁻}]]$
    \end{itemize}
\end{observation}
\begin{proof}
    By simple induction on $[[Γ ⊨ iP1 ≈au iP2 ⫤ (Ξ, uQ, aus1, aus2)]]$
    and mutually on $[[Γ ⊨ iN1 ≈au iN2 ⫤ (Ξ, uM, aus1, aus2)]]$.
    Let us consider tha last rule applied to infer this judgment.
    \begin{caseof}
        \item \ruleref{\ottdruleAUPVarLabel} or \ruleref{\ottdruleAUNVarLabel},
        then $[[Ξ]] = [[·]]$, and the property holds vacuously.

        \item \ruleref{\ottdruleAUAULabel}
        Then  $[[Ξ]] = [[â⁻_{iN1, iN2}]]$,
        $[[aus1]] = [[â⁻_{iN1, iN2} :≈ iN1]]$, and $[[aus2]] = [[â⁻_{iN1, iN2} :≈ iN2]]$.
        So the property holds trivially.

        \item \ruleref{ottdruleAUArrowLabel}
        In this case, $[[Ξ]] = [[Ξ' ∪ Ξ'']]$, $[[aus1]] = [[aus'1 & aus''1]]$, and 
        $[[aus2]] = [[aus'2 & aus''2]]$,
        where the property holds for ($[[Ξ']]$, $[[aus'1]]$, $[[aus'2]]$) and 
        ($[[Ξ'']]$, $[[aus''1]]$, $[[aus''2]]$) by the induction hypothesis.
        Then since the merge of solutions does not change the types the variables are mapped to,
        the required property holds for $[[Ξ]]$, $[[aus1]]$, and $[[aus2]]$.

        \item For the other rules, the resulting $[[Ξ]]$ is taken from the recursive call
        and the required property holds immediately by the induction hypothesis.
    \end{caseof}
\end{proof}

\begin{observation}[Anti-unification algorithm is deterministic]
    \label{obs:anti-unification-deterministic}
    \hfill\\
\begin{itemize}
    \item [$+$] If $[[Γ ⊨ iP1 ≈au iP2 ⫤ (Ξ, uQ, aus1, aus2)]]$ and 
    $[[Γ ⊨ iP1 ≈au iP2 ⫤ (Ξ', uQ', aus'1, aus'2)]]$,
    then $[[Ξ]] = [[Ξ']]$, $[[uQ]] = [[uQ']]$, $[[aus1]] = [[aus'1]]$, and $[[aus2]] = [[aus'2]]$.
    \item [$-$] If $[[Γ ⊨ iN1 ≈au iN2 ⫤ (Ξ, uM, aus1, aus2)]]$ and
    $[[Γ ⊨ iN1 ≈au iN2 ⫤ (Ξ', uM', aus'1, aus'2)]]$,
    then $[[Ξ]] = [[Ξ']]$, $[[uM]] = [[uM']]$, $[[aus1]] = [[aus'1]]$, and $[[aus2]] = [[aus'2]]$.
\end{itemize}
\end{observation}
\begin{proof}
    By trivial induction on $[[Γ ⊨ iP1 ≈au iP2 ⫤ (Ξ, uQ, aus1, aus2)]]$
    and mutually on $[[Γ ⊨ iN1 ≈au iN2 ⫤ (Ξ, uM, aus1, aus2)]]$.
\end{proof}


% \renewcommand{\ottdruleAUPVarName}[0]{(Var$^{+[[≈au]]}$)}
% \renewcommand{\ottdruleAUShiftDName}[0]{($\downarrow^{[[≈au]]}$)}
% \renewcommand{\ottdruleAUExistsName}[0]{($\exists^{[[≈au]]}$)}

% G ⊨ iN1 ≈au iN2 ⫤ ( Ξ , uM , aus1 , aus2 )
% \renewcommand{\ottdruleAUNVarName}[0]{(Var$^{-[[≈au]]}$)}
% \renewcommand{\ottdruleAUShiftUName}[0]{($\uparrow^{[[≈au]]}$)}
% \renewcommand{\ottdruleAUForallName}[0]{($\forall^{[[≈au]]}$)}
% \renewcommand{\ottdruleAUArrowName}[0]{($\rightarrow^{[[≈au]]}$)}
% \renewcommand{\ottdruleAUAUName}[0]{(AU$^{-}$)}


\begin{lemma}[Completeness of Anti-Unification] \label{lemma:anti-unification-completeness}
    \hfill
    \begin{itemize}
        \item [$+$] 
            Assume that $[[iP1]]$ and $[[iP2]]$ are normalized, and
            there exists $[[(Ξ', uQ', aus'1, aus'2)]]$ such that
            \begin{enumerate}
                \item $[[Γ ; Ξ' ⊢ uQ']]$,
                \item $[[Γ ; · ⊢ aus'i : Ξ']]$ for $i \in \{1,2\}$, and
                \item $[[ [aus'i] uQ' = iPi ]]$ for $i \in \{1,2\}$.
            \end{enumerate}

            Then the anti-unification algorithm terminates, that is there exists
            $[[(Ξ, uQ, aus1, aus2)]]$ such that $[[Γ ⊨ iP1 ≈au iP2 ⫤ (Ξ, uQ, aus1, aus2)]]$

        \item [$-$] 
            Assume that $[[iN1]]$ and $[[iN2]]$ are normalized, and
            there exists $[[(Ξ', uM', aus'1, aus'2)]]$ such that
            \begin{enumerate}
                \item $[[Γ ; Ξ' ⊢ uM']]$,
                \item $[[Γ ; · ⊢ aus'i : Ξ']]$ for $i \in \{1,2\}$, and
                \item $[[ [aus'i] uM' = iNi ]]$ for $i \in \{1,2\}$.
            \end{enumerate}

            Then the anti-unification algorithm succeeds, that is 
            there exists $[[(Ξ, uM, aus1, aus2)]]$ such that
            $[[Γ ⊨ iN1 ≈au iN2 ⫤ (Ξ, uM, aus1, aus2)]]$.
    \end{itemize}
\end{lemma}
\begin{proof}
    We prove it by the induction on $[[uM']]$ and mutually on $[[uQ']]$.
    \begin{caseof}
        \item $[[uM']] = [[â⁻]]$ 
            Then since $[[Γ ; · ⊢ aus'i : Ξ']]$,
            $[[Γ ⊢ [aus'i] uM']] = [[ iNi ]]$. 
            This way, \ruleref{\ottdruleAUAULabel} is always applicable
            if other rules are not.

        \item $[[uM']] = [[α⁻]]$
            Then $[[α⁻]] = [[ [aus'i] α⁻]] = [[ iNi ]]$, which means
            that \ruleref{\ottdruleAUNVarLabel} is applicable.

        \item $[[uM']] = [[↑uQ']]$
            Then $[[ ↑[aus'i]uQ']] = [[ [aus'i]↑uQ']] = [[ iNi ]]$, that is
            $[[iN1]]$ and $[[iN2]]$ have form $[[↑iP1]]$ and $[[↑iP2]]$ respectively.

            Moreover, $[[ [aus'i]uQ' = iPi ]]$, which means that 
            $[[(Ξ', uQ', aus'1, aus'2)]]$ is an anti-unifier of $[[iP1]]$ and $[[iP2]]$.
            Then by the induction hypothesis, there exists $[[(Ξ, uQ, aus1, aus2)]]$ such that
            $[[Γ ⊨ iP1 ≈au iP2 ⫤ (Ξ, uQ, aus1, aus2)]]$, and hence, 
            $[[Γ ⊨ ↑iP1 ≈au ↑iP2 ⫤ (Ξ, ↑uQ, aus1, aus2)]]$ by \ruleref{\ottdruleAUShiftULabel}.
        \item $[[uM']] = [[∀pas.uM'']]$ This case is similar to the previous one:
            we consider $\forall[[pas]]$ as a constructor. 
            Notice that $[[∀pas.[aus'i]uM'']] = [[ [aus'i]∀pas.uM'']] = [[ iNi ]]$, that is
            $[[iN1]]$ and $[[iN2]]$ have form $[[∀pas.iN''1]]$ and $[[∀pas.iN''2]]$ respectively.

            Moreover, $[[ [aus'i]uM'' = iN''i ]]$, which means that
            $[[(Ξ', uM'', aus'1, aus'2)]]$ is an anti-unifier of $[[iN''1]]$ and $[[iN''2]]$.
            Then by the induction hypothesis, there exists $[[(Ξ, uM, aus1, aus2)]]$ such that
            $[[Γ ⊨ iN''1 ≈au iN''2 ⫤ (Ξ, uM, aus1, aus2)]]$, and hence,
            $[[Γ ⊨ ∀pas.iN''1 ≈au ∀pas.iN''2 ⫤ (Ξ, ∀pas.uM, aus1, aus2)]]$ by 
            \ruleref{\ottdruleAUForallLabel}.
        \item $[[uM']] = [[uQ' → uM'']]$
            Then $[[ [aus'i]uQ' → [aus'i]uM'']] = [[ [aus'i](uQ' → uM'')]] = [[ iNi ]]$, that is
            $[[iN1]]$ and $[[iN2]]$ have form $[[uP1 → uN'1]]$ and $[[uP2 → uN'2]]$ respectively.

            Moreover, $[[ [aus'i]uQ' = iPi ]]$ and $[[ [aus'i]uM'' = iN''i ]]$, which means that
            $[[(Ξ', uQ', aus'1, aus'2)]]$ is an anti-unifier of $[[iP1]]$ and $[[iP2]]$,
            and $[[(Ξ', uM'', aus'1, aus'2)]]$ is an anti-unifier of $[[iN''1]]$ and $[[iN''2]]$.
            Then by the induction hypothesis, 
            $[[Γ ⊨ iP1 ≈au iP2 ⫤ (Ξ1, uQ, aus1, aus2)]]$ and 
            $[[Γ ⊨ iN''1 ≈au iN''2 ⫤ (Ξ2, uM, aus3, aus4)]]$ succeed.
            The result of the algorithm is $[[(Ξ1 ∪ Ξ2, uQ → uM, aus1 & aus3, aus2 & aus4)]]$,
            and it is left to show that $[[aus1 & aus3]]$ and $[[aus2 & aus4]]$ are defined.

            From \cref{obs:names-defined-by-mapping}, we know that 
            if $[[β̂⁻ ∊ dom(aus1) ∩ dom(aus3)]]$ then $[[ [aus1]β̂⁻ = [aus3]β̂⁻ ]]$,
            and if $[[β̂⁻ ∊ dom(aus2) ∩ dom(aus4)]]$ then $[[ [aus2]β̂⁻ = [aus4]β̂⁻ ]]$.
            This way, $[[aus1 & aus3]]$ and $[[aus2 & aus4]]$ are defined.

        \item $[[uQ']] = [[â⁺]]$ 
            This case if not possible, since $[[Γ ; Ξ' ⊢ uQ']]$ means 
            $[[â⁺]] \in [[Ξ']]$, but $[[Ξ']]$ can only contain negative variables. 

        \item Other positive cases are proved symmetrically to the corresponding negative ones.
    \end{caseof}
\end{proof}

\begin{lemma}[Initiality of Anti-Unification] \label{lemma:anti-unification-initial}
    \hfill
    \begin{itemize}
        \item [$+$] 
            Assume that $[[iP1]]$ and $[[iP2]]$ are normalized, and
            $[[Γ ⊨ iP1 ≈au iP2 ⫤ (Ξ, uQ, aus1, aus2)]]$, 
            then $[[(Ξ, uQ, aus1, aus2)]]$ is more specific than
            any other sound anti-unifier $[[(Ξ', uQ', aus'1, aus'2)]]$, i.e.
            if 
            \begin{enumerate}
                \item $[[Γ ; Ξ' ⊢ uQ']]$,
                \item $[[Γ ; · ⊢ aus'i : Ξ']]$ for $i \in \{1,2\}$, and
                \item $[[ [aus'i] uQ' = iPi ]]$ for $i \in \{1,2\}$
            \end{enumerate}
            then there exists $[[ausr]]$ such that
            $[[Γ ; Ξ ⊢ ausr : (Ξ' | uv uQ')]]$ and $[[ [ausr] uQ' = uQ ]]$. 
            Moreover, $[[ [ausr]β̂⁻]]$ can
            be uniquely determined by $[[ [aus'1]β̂⁻ ]]$, $[[ [aus'2]β̂⁻ ]]$, and
            $[[Γ]]$.
        \item [$-$] 
            Assume that $[[iN1]]$ and $[[iN2]]$ are normalized, and
            $[[Γ ⊨ iN1 ≈au iN2 ⫤ (Ξ, uM, aus1, aus2)]]$, 
            then $[[(Ξ, uM, aus1, aus2)]]$ is more specific than
            any other sound anti-unifier $[[(Ξ', uM', aus'1, aus'2)]]$, i.e.
            if
            \begin{enumerate}
                \item $[[Γ ; Ξ' ⊢ uM']]$,
                \item $[[Γ ; · ⊢ aus'i : Ξ']]$ for $i \in \{1,2\}$, and
                \item $[[ [aus'i] uM' = iNi ]]$ for $i \in \{1,2\}$
            \end{enumerate}
            then there exists $[[ausr]]$ such that
            $[[Γ ; Ξ ⊢ ausr : (Ξ' | uv uM')]]$ and $[[ [ausr] uM' = uM ]]$.
            Moreover, $[[ [ausr]β̂⁻]]$ can
            be uniquely determined by $[[ [aus'1]β̂⁻ ]]$, $[[ [aus'2]β̂⁻ ]]$, and
            $[[Γ]]$.
    \end{itemize}
\end{lemma}
\begin{proof}
    First, let us assume that $[[uM']]$ is a metavariable $[[α̂⁻]]$. 
    Then we can take $[[ausr]] = [[α̂⁻]] \mapsto [[uM]]$, which satisfies the required properties:
    \begin{itemize}
        \item $[[Γ ; Ξ ⊢ ausr : (Ξ' | uv uM')]]$ holds since 
        $[[Ξ' | uv uM']] = [[{α̂⁻}]]$ and $[[Γ ; Ξ ⊢ uM]]$ by the soundness of anti-unification (\cref{lemma:anti-unification-soundness});
        \item $[[ [ausr] uM' = uM ]]$ holds by construction
        \item $[[ [ausr]α̂⁻]] = [[uM]]$ is the anti-unifier of 
            $[[iN1]] = [[ [aus'1] α̂⁻]]$ and $[[iN2]] = [[ [aus'2] α̂⁻]]$
            in context $[[Γ]]$, and hence, it is uniquely determined by them (\cref{obs:anti-unification-deterministic}).
    \end{itemize}

    Now, we can assume that $[[uM']]$ is not a metavariable. 
    We prove by induction on the derivation of $[[Γ ⊨ iP1 ≈au iP2 ⫤ (Ξ, uQ, aus1, aus2)]]$
    and mutually on the derivation of $[[Γ ⊨ iN1 ≈au iN2 ⫤ (Ξ, uM, aus1, aus2)]]$.

    Since $[[uM']]$ is not a metavariable, 
    the substitution acting on $[[uM']]$ preserves its outer constructor. 
    In other words, 
    $[[ [aus'i] uM' = iNi ]]$ means that $[[uM']]$, 
    $[[iN1]]$ and $[[iN2]]$ have the same outer constructor. 
    Let us consider the algorithmic anti-unification rule corresponding to this constructor, 
    and show that it was successfully applied to anti-unify $[[iN1]]$ and $[[iN2]]$ 
    (or $[[iP1]]$ and $[[iP2]]$).

    \begin{caseof}
        \item \ruleref{\ottdruleAUNVarLabel}, i.e. $[[iN1]] = [[α⁻]] = [[iN2]]$.
        \label{case:anti-unification-initial:nvar}
        This rule is applicable since it has no premises. 
        
        Then $[[Ξ]] = [[·]]$, $[[uM]] = [[α⁻]]$, 
        and $[[aus1]] = [[aus2]] = [[·]]$.
        Since $[[ [aus'i] uM' = iNi ]] = [[α⁻]]$
        and $[[uM']]$ is not a metavariable, $[[uM']] = [[α⁻]]$.
        Then we can take $[[ausr]] = [[·]]$, which satisfies the required properties:
        \begin{itemize}
            \item $[[Γ ; Ξ ⊢ ausr : (Ξ' | uv uM')]]$ holds vacuously since 
            $[[Ξ' | uv uM']] = [[∅]]$; 
            \item $[[ [ausr] uM' = uM ]]$, that is $[[ [·] α⁻ = α⁻ ]]$
            holds by substitution properties;
            \item the unique determination of $[[ [ausr]α̂⁻]]$ for $[[α̂⁻]] \in [[Ξ' | uv uM']] = [[∅]]$ holds vacuously.
        \end{itemize}

        \item \ruleref{\ottdruleAUShiftULabel}, i.e. 
        $[[iN1]] = [[↑iP1]]$ and $[[iN2]] = [[↑iP2]]$.
        \label{case:anti-unification-initial:shiftu}

        Then since $[[ [aus'i] uM' = iNi ]] = [[↑iPi]]$ and $[[uM']]$ is not a metavariable,
        $[[uM']] = [[↑uQ']]$, where $[[ [aus'i] uQ' = iPi ]]$. 
        Let us show that $[[(Ξ', uQ', aus'1, aus'2)]]$ 
        is an anti-unifier of $[[iP1]]$ and $[[iP2]]$.
        \begin{enumerate}
            \item $[[Γ ; Ξ' ⊢ uQ']]$ holds by inversion of $[[Γ ; Ξ' ⊢ ↑uQ']]$;
            \item $[[Γ ; · ⊢ aus'i : Ξ']]$ holds by assumption;
            \item $[[ [aus'i] uQ' = iPi ]]$ holds by assumption.
        \end{enumerate}

        This way, by the completeness of anti-unification 
        (\cref{lemma:anti-unification-completeness}),
        the anti-unification algorithm succeeds on $[[iP1]]$ and $[[iP2]]$:
        $[[Γ ⊨ iP1 ≈au iP2 ⫤ (Ξ, uQ, aus1, aus2)]]$,
        which means that \ruleref{\ottdruleAUShiftULabel} is applicable to infer 
        $[[Γ ⊨ ↑iP1 ≈au ↑iP2 ⫤ (Ξ, ↑uQ, aus1, aus2)]]$.

        Moreover, by the induction hypothesis,
        $[[(Ξ, uQ, aus1, aus2)]]$ is more specific than $[[(Ξ', uQ', aus'1, aus'2)]]$,
        which immediately implies that $[[(Ξ, ↑uQ, aus1, aus2)]]$ is more specific than
        $[[(Ξ', ↑uQ', aus'1, aus'2)]]$ (we keep the same $[[ausr]]$).

        \item \ruleref{\ottdruleAUForallLabel}, i.e. 
        $[[iN1]] = [[∀pas.iN'1]]$ and $[[iN2]] = [[∀pas.iN'2]]$.
        \label{case:anti-unification-initial:forall}
        The proof is symmetric to the previous case.
        Notice that the context $[[Γ]]$ is not changed in \ruleref{\ottdruleAUForallLabel}, 
        as it represents the context in which the anti-unification variables must be instantiated,
        rather than the context forming the types that are being anti-unified.

        \item \ruleref{\ottdruleAUArrowLabel}, i.e.
        $[[iN1]] = [[iP1 → iN'1]]$ and $[[iN2]] = [[iP2 → iN'2]]$.

        Then since $[[ [aus'i] uM' = iNi ]] = [[iPi → iN'i]]$ and $[[uM']]$ is not a metavariable,
        $[[uM']] = [[uQ' → uM'']]$, where $[[ [aus'i] uQ' = iPi ]]$ and $[[ [aus'i] uM'' = iN''i ]]$.

        Let us show that $[[(Ξ', uQ', aus'1, aus'2)]]$
        is an anti-unifier of $[[iP1]]$ and $[[iP2]]$.
        \begin{enumerate}
            \item $[[Γ ; Ξ' ⊢ uQ']]$ holds by inversion of $[[Γ ; Ξ' ⊢ uQ' → uM'']]$;
            \item $[[Γ ; · ⊢ aus'i : Ξ']]$ holds by assumption;
            \item $[[ [aus'i] uQ' = iPi ]]$ holds by assumption.
        \end{enumerate}

        Similarly, $[[(Ξ', uM'', aus'1, aus'2)]]$ is an anti-unifier of $[[iN''1]]$ and $[[iN''2]]$.

        Then by the completeness of anti-unification (\cref{lemma:anti-unification-completeness}),
        the anti-unification algorithm succeeds on $[[iP1]]$ and $[[iP2]]$:
        $[[Γ ⊨ iP1 ≈au iP2 ⫤ (Ξ1, uQ, aus1, aus2)]]$;
        and on $[[iN'1]]$ and $[[iN'2]]$:
        $[[Γ ⊨ iN''1 ≈au iN''2 ⫤ (Ξ2, uM''', aus3, aus4)]]$.
        Notice that $[[aus1 & aus3]]$ and $[[aus2 & aus4]]$ are defined, 
        in other words, for any $[[β̂⁻]] \in [[Ξ1]] \cap [[Ξ2]]$,
        $[[ [aus1] β̂⁻ = [aus2] β̂⁻ ]]$ and $[[ [aus3] β̂⁻ = [aus4] β̂⁻ ]]$,
        which follows immediately from \cref{obs:names-defined-by-mapping}.
        This way, the algorithm proceeds by applying \ruleref{\ottdruleAUArrowLabel} and returns
        $[[(Ξ1 ∪ Ξ2, uQ → uM''', aus1 & aus3, aus2 & aus4)]]$.

        It is left to construct $[[ausr]]$ such that $[[Γ ; Ξ ⊢ ausr : (Ξ' | uv uM')]]$ and $[[ [ausr] uM' = uM ]]$.
        By the induction hypothesis, there exist $[[ausr1]]$ and $[[ausr2]]$ such that
        $[[Γ ; Ξ1 ⊢ ausr1 : (Ξ' | uv uQ')]]$, 
        $[[Γ ; Ξ2 ⊢ ausr2 : (Ξ' | uv uM'')]]$,
        $[[ [ausr1] uQ' = uQ ]]$, and $[[ [ausr2] uM'' = uM''' ]]$.

        Let us show that $[[ausr]] = [[ausr1 & ausr2]]$ is defined, 
        in other words, for any $[[β̂⁻]] \in [[(Ξ' | uv uQ')]] \cap [[(Ξ' | uv uM'')]]$,
        $[[ [ausr1] β̂⁻ = [ausr2] β̂⁻ ]]$.It holds because by the induction hypothesis,
        $[[ [ausri] β̂⁻ ]]$ is uniquely determined by $[[ [aus'1] β̂⁻ ]]$, $[[ [aus'2] β̂⁻ ]]$, and $[[Γ]]$,
        none of which depends on $i$.

        Let us show that $[[ausr]] = [[ausr1 & ausr2]]$ satisfies the required properties:
        \begin{itemize}
            \item $[[Γ ; Ξ1 ∪ Ξ2 ⊢ ausr1 & ausr2 : (Ξ' | uv uM')]]$ holds since 
            $[[Ξ' | uv uM']] = [[Ξ' | uv uQ' → uM'']] = [[(Ξ' | uv uQ') ∪ (Ξ' | uv uM'')]]$,
            $[[Γ ; Ξ1 ⊢ ausr1 : (Ξ' | uv uQ')]]$ and $[[Γ ; Ξ2 ⊢ ausr2 : (Ξ' | uv uM'')]]$
            \ilyam{add a lemma?};
            \item $[[ [ausr] uM' ]] = [[ [ausr] (uQ' → uM'') ]] = [[ [ausr | uv uQ'] uQ' → [ausr | uv uM''] uM'' ]] =
            [[ [ausr1] uQ' → [ausr2] uM'' ]] = [[ uQ → uM''']] = [[ uM ]]$;
            \item Since $[[ [ausr]β̂⁻]]$ is either equal to  $[[ [ausr1]β̂⁻]]$ or $[[ [ausr2]β̂⁻]]$,
            it inherits their property that it is uniquely determined by $[[ [aus'1]β̂⁻]]$, $[[ [aus'2]β̂⁻]]$, and $[[Γ]]$.
        \end{itemize}

        \item $[[iP1]] = [[iP2]] = [[a⁺]]$. This case is symmetric to \cref{case:anti-unification-initial:nvar}.
        \item $[[iP1]] = [[↓iN1]]$ and $[[iP2]] = [[↓iN2]]$. This case is symmetric to \cref{case:anti-unification-initial:shiftu}
        \item $[[iP1]] = [[∃nas.iP'1]]$ and $[[iP2]] = [[∃nas.iP'2]]$. This case is symmetric to \cref{case:anti-unification-initial:forall}
        \end{caseof}

\end{proof}

\subsection{Solution Merge}
\begin{lemma} \label{lemma:entry-weakening-preorder}
    Given a fixed context $[[Γ]]$, weakening forms a preorder on the set of entries well-formed in $[[Γ]]$.
\end{lemma}
\begin{proof} \hfill 
  \begin{itemize}
    \item Reflexivity: $[[Γ ⊢ usEntry ≈ usEntry]]$ then $[[Γ ⊢ usEntry ⇒ usEntry]]$.
    Let us consider the shape of $[[usEntry]]$. In all cases, there is a rule inferring
     $[[Γ ⊢ usEntry ⇒ usEntry]]$, since $[[Γ ⊢ iP ≈ iQ]]$ implies $[[Γ ⊢ iP ≥ iQ]]$ by inversion.
    \item Transitivity: $[[Γ ⊢ usEntry1 ⇒ usEntry2]]$ and $[[Γ ⊢ usEntry2 ⇒ usEntry3]]$ implies $[[Γ ⊢ usEntry1 ⇒ usEntry3]]$.
    It follows immediately by considering the rules inferring 
    $[[Γ ⊢ usEntry1 ⇒ usEntry2]]$ and $[[Γ ⊢ usEntry2 ⇒ usEntry3]]$, and applying the 
    transitivity of subtyping (\cref{corollary:subtyping-transitivity}).
  \end{itemize}
\end{proof}

\begin{lemma} [Solution Weakening forms a preorder]
    \label{lemma:solution-weakening-preorder}
    Let us consider a set of pairs $([[Θ]], [[us]])$ such that $[[us : Θ]]$.
    Then the relation defined as 
    $([[Θ1]], [[us1]]) \Rightarrow ([[Θ2]], [[us2]])$ iff $[[Θ1]] \supseteq [[Θ2]]$ 
    and $[[Θ2 ⊢ us1 ⇒ us2]]$ forms a preorder.
\end{lemma}
\begin{proof} \hfill
    \begin{itemize}
        \item Reflexivity: $([[Θ]], [[us]]) \Rightarrow ([[Θ]], [[us]])$.\\ It is clear that $[[Θ]] \supseteq [[Θ]]$.
        $[[Θ ⊢ us ⇒ us]]$ holds since for any $[[usEntry]]$ in $[[us]]$ restricting $[[α̂±]]$, $[[Θ(α̂±) ⊢ usEntry ⇒ usEntry]]$  
        by \cref{lemma:entry-weakening-preorder}
        \item Transitivity: If $([[Θ1]], [[us1]]) \Rightarrow ([[Θ2]], [[us2]])$ and $([[Θ2]], [[us2]]) \Rightarrow ([[Θ3]], [[us3]])$
        then $([[Θ1]], [[usEntry1]]) \Rightarrow ([[Θ3]], [[usEntry3]])$.\\
        It is clear that since $[[Θ1]] \subseteq [[Θ2]]$ and $[[Θ2]] \subseteq
        [[Θ3]]$ then $[[Θ1]] \subseteq [[Θ3]]$. Let us consider $[[usEntry3]]
        \in [[us3]]$. Then there exists $[[usEntry2]] \in [[us2]]$ such that
        $[[Θ3 ⊢ usEntry2 ⇒ usEntry3]]$. Also there exists $[[usEntry1]] \in [[us1]]$ 
        such that $[[Θ2 ⊢ usEntry1 ⇒ usEntry2]]$, and by weakening,
        $[[Θ3 ⊢ usEntry1 ⇒ usEntry2]]$. Then by transitivity of solution entry
        weakening (\cref{lemma:entry-weakening-preorder}), $[[Θ3 ⊢ usEntry1 ⇒
        usEntry3]]$.
    \end{itemize}
\end{proof}

\begin{corollary} [Solution Weakening is transitive]
    \label{lemma:weakening-transitivity}
    If $[[Θ ⊢ us1 ⇒ us2]]$ and $[[Θ ⊢ us2 ⇒ us3]]$ then $[[Θ ⊢ us1 ⇒ us3]]$.
\end{corollary}


\begin{lemma} [Soundness of Merge of Unification Solution Entries]
    \label{lemma:unif-sol-ent-merge-soundness}
    For a fixed context $[[Γ]]$,
    suppose that  $[[Γ ⊢ usEntry1]]$ and $[[Γ ⊢ usEntry2]]$,
    where $[[Γ ⊢ usEntryi]]$ is an equivalence-shaped restriction
    (i.e. it has shape $[[α̂⁺ :≈ iP]]$ or $[[α̂⁻ :≈ iN]]$ but not $[[α̂⁺ :≥ iP]]$).
    If $[[usEntry1 & usEntry2]]$ is defined then
    \begin{enumerate}
    \item $[[Γ ⊢ usEntry1 & usEntry2]]$
    \item $[[Γ ⊢ usEntry1 & usEntry2 ⇒ usEntry1]]$
    \item $[[Γ ⊢ usEntry1 & usEntry2 ⇒ usEntry2]]$
    \end{enumerate}
\end{lemma}
\begin{proof}
    Let us consider the rule forming $[[Γ ⊢ usEntry1 & usEntry2 = usEntry]]$.
    \begin{caseof}
        \item \ruleref{\ottdruleSMEPEqEqLabel}, i.e. 
        $[[Γ ⊢ usEntry1 & usEntry2 = usEntry]]$
        has form $[[Γ ⊢ (pua :≈ iP) & (pua :≈ iP') = (pua :≈ iP)]]$
        and $[[nf(iP) = nf(iP')]]$. Then
         \begin{enumerate}
            \item $[[Γ ⊢ usEntry]]$, i.e. $[[Γ ⊢ pua :≈ iP]]$ holds by assumption;
            \item $[[Γ ⊢ (pua :≈ iP) ⇒ (pua :≈ iP)]]$ holds by reflexivity (\cref{lemma:entry-weakening-preorder});
            \item $[[Γ ⊢ (pua :≈ iP) ⇒ (pua :≈ iP')]]$ holds
            because $[[nf(iP) = nf(iP')]]$ implies $[[Γ ⊢ iP ≈ iP']]$
            (by \cref{lemma:subt-equiv-algorithmization}), and thus, 
            \ruleref{\ottdruleSImpEPEqEqLabel} applies.
         \end{enumerate}
        \item \ruleref{\ottdruleSMENEqEqLabel}. The negative case is proved in exactly the same way.
    \end{caseof}
\end{proof}

\begin{lemma} [Soundness of Merge of Unification Solutions]
    \label{lemma:unif-sol-merge-soundness}
    Suppose that $[[us1 : Θ | varset1]]$ and $[[us2 : Θ | varset2]]$ 
    are unification solutions (i.e. $[[us1]]$ and $[[us2]]$ can only have equivalence-shaped restrictions).
    If $[[us1 & us2]]$ is defined then
    \begin{enumerate}
        \item $[[us1 & us2 : Θ|varset1 ∪ varset2]]$,
        \item $[[Θ|varset1 ⊢ us1 & us2 ⇒ us1]]$, and
        \item $[[Θ|varset2 ⊢ us1 & us2 ⇒ us2]]$.
    \end{enumerate}
\end{lemma}
\begin{proof}
    Let us prove the properties separately:
    \begin{enumerate}
        \item $[[us1 & us2 : Θ|varset1 ∪ varset2]]$.
        It suffices to prove the following two properties:
        \begin{itemize}
            \item The set of variables of the entries of $[[us1 & us2]]$ 
            coincides with $[[varset1 ∪ varset2]]$\\
            By definition, $[[us1 & us2]]$ consists of three parts:
            entries of $[[us1]]$ that do not have matching entries of $[[us2]]$,
            entries of $[[us2]]$ that do not have matching entries of $[[us1]]$,
            and the merge of matching entries.
            It means that $[[dom(us1 & us2)]] = [[dom(us1) \ dom(us2) ∪ dom(us2) \ dom(us1) ∪ 
            dom(us1) ∩ dom(us2)]] = [[dom(us1) ∪ dom(us2)]]$, which, since 
            $[[usi : Θ | varseti]]$, is equal to $[[varset1 ∪ varset2]]$. 

            \item Each entry of $[[us1 & us2]]$ restricting $[[α̂±]]$ is well-formed in
            the corresponding context $[[Θ(α̂±)]]$.\\
            Let us consider an arbitrary entry $[[usEntry]]$ of $[[us1 & us2]]$ restricting
            $[[α̂±]]$. Then there are three cases:
            \begin{enumerate}
                \item $[[usEntry]]$ the entry is from $[[us1]]$ and does not have a matching entry 
                in $[[us2]]$, i.e. $[[α̂±]] \in [[dom(us1) \ dom(us2)]]$.
                Then $[[usEntry]]$ is well-formed in $[[Θ | varset1]]$ by assumption.
                \item $[[usEntry]]$ the entry is from $[[us2]]$ and does not
                have a matching entry in $[[us1]]$. This case is symmetric.
                \item $[[usEntry]]$ is the merge of two matching entries $[[usEntry1]] \in [[us1]]$
                 and $[[usEntry2]] \in [[us2]]$ restricting $[[α̂±]]$.
                 Since $[[us1 : Θ | varset1]]$ and $[[us2 : Θ | varset2]]$,
                 $[[α̂±]] \in [[dom(Θ|varset1) ∩ dom(Θ|varset2)]]$, i.e. there 
                 is an entry $[[ α̂±[Γ] ]] \in [[Θ]]$, and 
                 $[[usEntry1]]$ and $[[usEntry2]]$ are well-formed in $[[Γ]]$.
                 Then by \cref{lemma:unif-sol-ent-merge-soundness}, 
                 $[[Γ ⊢ usEntry1 & usEntry2]]$, where $[[Γ]] = [[Θ(α̂±)]]$.
            \end{enumerate}
        \end{itemize}
        \item $[[Θ|varset1 ⊢ us1 & us2 ⇒ us1]]$
        We need to show that for every entry $[[usEntry]]$ from $[[us1]]$,
        there is an entry $[[usEntry']]$ from $[[us1 & us2]]$ such that
        $[[Θ|varset1 ⊢ usEntry' ⇒ usEntry]]$. 
        Let us consider an arbitrary $[[usEntry]]$ from $[[us1]]$ restricting $[[α̂±]]$.
        Then there are two cases:
        \begin{itemize}
            \item $[[usEntry]]$ does not have a matching entry in $[[us2]]$.
            Then $[[usEntry]]$ is also in $[[us1 & us2]]$ and $[[Θ|varset1 ⊢ usEntry ⇒ usEntry]]$ 
            by reflexivity (\cref{lemma:entry-weakening-preorder}).
            \item $[[usEntry]]$ has a matching entry $[[usEntry']]$ in $[[us2]]$.
            Then $[[usEntry & usEntry']]$ is in $[[us1 & us2]]$ and $[[Θ|varset1 ⊢ usEntry & usEntry' ⇒ usEntry]]$ by \cref{lemma:unif-sol-ent-merge-soundness}.
        \end{itemize}
        \item $[[Θ|varset2 ⊢ us1 & us2 ⇒ us2]]$ is proved analogously.
    \end{enumerate}
\end{proof}

\begin{lemma} [Soundness of Solution Entry Merge]
\label{lemma:entry-merge-soundness}
For a fixed context $[[Γ]]$,
suppose that  $[[Γ ⊢ usEntry1]]$ and $[[Γ ⊢ usEntry2]]$. 
If $[[usEntry1 & usEntry2]]$ is defined then
\begin{enumerate}
    \item $[[Γ ⊢ usEntry1 & usEntry2]]$
    \item $[[Γ ⊢ usEntry1 & usEntry2 ⇒ usEntry1]]$
    \item $[[Γ ⊢ usEntry1 & usEntry2 ⇒ usEntry2]]$
\end{enumerate}
\end{lemma}
\begin{proof}
    Let us consider the rule forming $[[Γ ⊢ usEntry1 & usEntry2 = usEntry]]$.
    \begin{caseof}
        \item \ruleref{\ottdruleSMEPEqEqLabel} or \ruleref{\ottdruleSMENEqEqLabel}
        The proof is analogous to the proof of \cref{lemma:unif-sol-ent-merge-soundness}.
        \item \ruleref{\ottdruleSMESupSupLabel} 
        Then $[[usEntry1]]$ is $[[pua :≥ iP1]]$, $[[usEntry2]]$ is $[[pua :≥ iP2]]$,
        and $[[usEntry1 & usEntry2]]$ is $[[pua :≥ iQ]]$ where $[[iQ]]$ is the least upper bound of $[[iP1]]$ and $[[iP2]]$.
        Then by \cref{lemma:lub-soundness},
        \begin{itemize}
            \item $[[Γ ⊢ iQ]]$, and hence $[[Γ ⊢ pua :≥ iQ]]$, that is $[[Γ ⊢ usEntry1 & usEntry2]]$,
            \item $[[Γ ⊢ iQ ≥ iP1]]$, and hence $[[Γ ⊢ pua :≥ iQ ⇒ pua :≥ iP1]]$, that is $[[Γ ⊢ usEntry1 & usEntry2 ⇒ usEntry1]]$,
            \item $[[Γ ⊢ iQ ≥ iQ1]]$, and hence $[[Γ ⊢ usEntry1 & usEntry2 ⇒ usEntry2]]$.
        \end{itemize}
        \item \ruleref{\ottdruleSMESupEqLabel}
        Then $[[usEntry1]]$ is $[[pua :≥ iP]]$, $[[usEntry2]]$ is $[[pua :≈ iQ]]$, 
        where $[[Γ;· ⊨ uQ ≥ iP ⫤ us']]$, and the resulting   
        $[[usEntry1 & usEntry2]]$ is equal to $[[usEntry2]]$, that is $[[pua :≈ iQ]]$.
    
        \begin{itemize}
            \item By assumption, $[[Γ ⊢ iQ]]$, and hence $[[Γ ⊢ pua :≈ iQ]]$, that is $[[Γ ⊢ usEntry1 & usEntry2]]$.
            \item Since $[[uv(uQ) = ∅]]$, 
                $[[Γ;· ⊨ uQ ≥ iP ⫤ us']]$ implies $[[Γ ⊢ iQ ≥ iP]]$
                by the soundness of positive subtyping (\cref{lemma:pos-subt-soundness}), 
                which means $[[Γ ⊢ pua :≈ iQ ⇒ pua :≥ iP]]$, that is $[[Γ ⊢ usEntry1 & usEntry2 ⇒ usEntry1]]$.
            \item  $[[Γ ⊢ pua :≈ iQ ⇒ pua :≈ iQ]]$ by reflexivity
             (\cref{lemma:entry-weakening-preorder}), and hence
              $[[Γ ⊢ usEntry1 & usEntry2 ⇒ usEntry2]]$.
        \end{itemize}
        \item \ruleref{\ottdruleSMEEqSupLabel} Thee proof is analogous to the previous case.
    \end{caseof}
\end{proof}

\begin{lemma} [Soundness of Solution Merge] \label{lemma:merge-soundness}
    Suppose that $[[us1 : Θ1]]$ and $[[us2 : Θ2]]$ 
    and $[[us1 & us2]]$ is defined.
    Then 
    \begin{enumerate}
        \item $[[us1 & us2 : Θ1 ∪ Θ2]]$,
        \item $[[Θ1 ⊢ us1 & us2 ⇒ us1]]$, and
        \item $[[Θ2 ⊢ us1 & us2 ⇒ us2]]$.
    \end{enumerate}
\end{lemma}
\begin{proof}
    The proof repeats the proof of \cref{lemma:unif-sol-merge-soundness},
    but uses \cref{lemma:entry-merge-soundness} instead of 
    \cref{lemma:unif-sol-ent-merge-soundness}.
\end{proof}


\begin{lemma} [Completeness and Initiality of Solution Merge] 
    \label{lemma:merge-completeness}
    Suppose that $[[us1 : Θ1]]$ and $[[us2 : Θ2]]$
    and there exists $[[us]]$
    such that $[[Θ1 ⊢ us1 ⇒ us]]$ and $[[ Θ2 ⊢ us2 ⇒ us]]$.
    Then $[[us1 & us2]]$ is defined and 
    $[[ Θ1 ∪ Θ2 ⊢ us1 & us2 ⇒ us]]$.
\end{lemma}

\begin{lemma} [Solution Weakening is Monotonous]
    \label{lemma:weakening-monotonicity}
    if $[[us1 : Θ]]$ and $[[us2 : Θ]]$ and 
    $[[us1 ⊆ us2]]$ then $[[Θ ⊢ us2 ⇒ us1]]$.
\end{lemma}


\subsection{Subtyping Algorithm}
\begin{theorem}[Soundness of Subtyping Algorithm] \hfill
    \begin{itemize}
        \item [$-$] 
        If $[[Γ ⊢ Θ]]$, $[[Γ ⊢ iM]]$, and $[[Γ ; Θ ⊢ uN]]$ then\\ 
        $[[Γ ; Θ ⊨ uN ≤ iM ⫤ us]]$
        implies $[[us : Θ|uv uN]]$ and 
        for any $[[us']]$ such that $[[Θ ⊢ us' ⇒ us]]$,
        $[[ Γ ⊢ [us']uN ≤ iM ]]$

        \item [$+$] 
        If $[[Γ ⊢ Θ]]$, $[[Γ ⊢ iQ]]$, and $[[Γ ; Θ ⊢ uP]]$ then\\
        $[[Γ ; Θ ⊨ uP ≥ iQ ⫤ us]]$
        implies $[[us : Θ|uv uP]]$ and 
        for any $[[us']]$ such that $[[Θ ⊢ us' ⇒ us]]$,
        $[[ Γ ⊢ [us']uP ≥ iQ ]]$.
     \end{itemize}
\end{theorem}
\begin{proof}
    We prove it by induction on $[[Γ ; Θ ⊨ uN ≤ iM ⫤ us]]$ (and mutually, on $[[Γ ; Θ ⊨ uP ≥ iQ ⫤ us]]$).
    Let us consider the last rule to infer this judgment. 
    \begin{caseof}
        \item $[[G;Θ ⊨ a⁻ ≤ a⁻ ⫤ ·]]$\\
        Then $[[uv a⁻ = ∅]]$, and $[[us]] = [[· : ·]]$ satisfies $[[us : Θ|∅]]$.
        Since $[[uv a⁻ = ∅]]$, application of any unification solution $[[us']]$ does not change $[[a⁻]]$, i.e.
        $[[ [us']a⁻ = a⁻ ]]$. Therefore, $[[Γ ⊢ [us']a⁻ ≤ a⁻]]$ holds by \ruleref{\ottdruleDOneNVarLabel}.

        \item $[[G;Θ ⊨ ↑uP ≤ ↑iQ ⫤ us]]$\\
        Then the next step of the algorithm is the unification of $[[nf(uP)]]$ and $[[nf(iQ)]]$.
        By the soundness of the unification algorithm (\cref{lemma:unification-soundness}),
        it returns an equivalence-only solution $[[us]]$ such that $[[us : Ord uv uP]]$.
        By \cref{todo}, since $[[us]]$ is equivalence-only and $[[Γ ⊢ Θ]]$, $[[Θ ⊢ us' ⇒ us]]$ means 
        $[[ Γ ⊢ Sub us' ≈ Sub us : Ord uv uP ]]$ as substitutions.

        $[[ [us]nf(uP) = nf(iQ) ]]$ implies $[[Γ ⊢ [us]nf(uP) ≈ nf(iQ)]]$, and then 
        $[[Γ ⊢ [us']nf(uP) ≈ nf(iQ)]]$. \ilyam{add lemmas}

        Let us rewrite the left-hand side and the right-hand side of $[[Γ ⊢ [us']nf(uP) ≈ nf(iQ)]]$ by 
        transitivity of equivalence (\cref{corollary:equivalence-transitivity}).
        By \cref{corollary:nf-sound-wrt-subt-equiv,corollary:subst-pres-equiv},
        $[[Γ ⊢ [us']nf(uP) ≈ [us']uP ]]$. By \cref{corollary:nf-sound-wrt-subt-equiv}, 
        $[[Γ ⊢ nf(iQ) ≈ iQ ]]$. 
        This way, we have $[[Γ ⊢ [us']uP ≈ iQ]]$.
        Then by \ruleref{\ottdruleDOneShiftULabel}
        and congruence of substitution, $[[Γ ⊢ [us']↑uP ≤ ↑iQ]]$.

        \item $[[G;Θ ⊨ uP → uN' ≤ iQ → iM' ⫤ us]]$\\
        The next step of the algorithm is two recursive calls:
        $[[G;Θ ⊨ uP ≥ iQ ⫤ us1]]$ and $[[G;Θ ⊨ uN' ≤ iM' ⫤ us2]]$.
        By the induction hypothesis, 
        \begin{enumerate}
            \item $[[us1 : Θ | uv uP]]$ and $[[ Γ ⊢ [us1']uP ≥ iQ ]]$ for any $[[us1']]$ s.t. $[[Θ ⊢ us1' ⇒ us1]]$
            \item $[[us2 : Θ | uv uN']]$ and $[[ Γ ⊢ [us2']uN' ≤ iM' ]]$ for any $[[us2']]$ s.t. $[[Θ ⊢ us2' ⇒ us2]]$
        \end{enumerate}

        Then the algorithm merges two unification solutions $[[us1]]$ and $[[us2]]$.
        By \cref{lemma:merge-soundness}, since $[[uv uP ∪ uv uN' = uv (uP → uN')]]$, 
        we have $[[us1 & us2 : Θ | uv (uP → uN')]]$, and also
        $[[Θ ⊢ us1 & us2 ⇒ us1]]$ and $[[Θ ⊢ us1 & us2 ⇒ us2]]$.
        By the transitivity of solution weakening (\cref{lemma:weakening-transitivity}),
         $[[Θ ⊢ us' ⇒ us1 & us2]]$ implies $[[Θ ⊢ us' ⇒ us1]]$ and $[[Θ ⊢ us' ⇒ us2]]$.

% \[\begin{tikzcd}[arrows=Rightarrow]
% 	& {[[us']]} \\
% 	{[[us1]]} & {[[us1 & us2]]} & {[[us2]]}
% 	\arrow[from=2-2, to=2-3]
% 	\arrow[from=2-2, to=2-1]
% 	\arrow[from=1-2, to=2-2]
% \end{tikzcd}\]

        The application of the induction hypothesis gives us 
        $[[Γ ⊢ [us']uP ≥ iQ ]]$ and $[[ Γ ⊢ [us']uN' ≤ iM' ]]$.
        Finally, by \ruleref{\ottdruleDOneArrowLabel}, $[[Γ ⊢ [us'](uP → uN') ≤ iQ → iM']]$.

        \item $[[G;Θ ⊨ ∀pas.uN' ≤ ∀pbs.iM' ⫤ us]]$ s.t. either $[[pas]]$ or $[[pbs]]$ is not empty\\
        Then the algorithm creates fresh unification variables $[[â⁺*[Γ,pbs] ]]$, 
        substitutes the old $[[pas]]$ with them in $[[uN']]$, and makes the recursive call:
        $[[G, pbs; Θ, â⁺*[G, pbs] ⊨ [â⁺*/pas] uN ≤ iM ⫤ us']]$, returning as the resulting solution 
        $[[us]] = [[us' \ {α̂⁺*}]]$.

        Notice that $[[Γ, pbs ⊢ Θ, â⁺*[G, pbs] ]]$, $[[Γ,pbs ⊢ iM']]$, and 
        $[[Γ,pbs; Θ, â⁺*[G, pbs] ⊢ [â⁺*/pas] uN ]]$, so the induction hypothesis is applicable,
        that is $[[us' : Θ | uv [â⁺*/pas]uN']]$ and $[[ Γ, pbs ⊢ [us'2][â⁺*/pas]uN' ≤ iM' ]]$ for any
        $[[us'2]]$ s.t. $[[Θ ⊢ us'2 ⇒ us']]$.

        Since the domain of $[[us']]$ is $[[uv [â⁺*/pas]uN']]$, the domain of 
        $[[us]] = [[us' \  {α̂⁺*}]]$ is $[[uv [â⁺*/pas]uN \ {â⁺*} = uv uN']] = [[uv ∀pas.uN']]$,
        this way, $[[us : Θ | uv ∀pas.uN']]$, as required.

        It is left to show that $[[Γ ⊢ [us2]∀pas.uN' ≤ ∀pbs.iM']]$ for any $[[us2]]$ s.t. $[[Θ ⊢ us2 ⇒ us]]$.
        Let us consider an arbitrary such $[[us2]]$. Let us construct $[[us'2]]$, 
        extending $[[us2]]$ to the domain $[[Ord uv [â⁺*/pas]uN']]$ with the values of $[[us']]$,
        i.e.  $[[us'2 : Θ | uv [â⁺*/pas]uN']]$, 
        and
        \[
            [[us'2(β̂±)]]  = 
            \begin{cases}
               [[us2(β̂±)]] & \text{if } [[β̂±]] \in [[uv uN']] \\
               [[us'(β̂±)]] & \text{if } [[β̂±]] \in [[â⁺*]]
            \end{cases}
        \] 
        , where the application of the unification solution to a variable returns the 
        corresponding \emph{unification entry}. 
        Notice that $[[Sub us'2|uv uN']] = [[us2]]$.
    It is easy to see that $[[Θ ⊢ us'2 ⇒ us']]$: 
    \begin{enumerate}
        \item if $[[β̂±]] \in [[uv uN']]$ then $[[us'2(β̂±)]] = [[us2(β̂±)]] \Rightarrow [[us(β̂±)]] = [[us'(β̂±)]]$;
        \item it $[[β̂±]] \in [[â⁺*]]$ then $[[us'2(β̂±)]] = [[us'(β̂±)]] \Rightarrow [[us'(β̂±)]]$,
    \end{enumerate}
    which means that the induction hypothesis can be applied to $[[us'2]]$, i.e.
    $[[ Γ, pbs ⊢ [us'2][â⁺*/pas]uN' ≤ iM' ]]$.\\
    Notice that
    $
    \begin{aligned}[t]
                 [[ [us'2][â⁺*/pas]uN' ]] &= [[ [Sub us'2|{â⁺*} ○ â⁺*/pas][Sub us'2|uv uN']uN' ]]
                                          && \text{by substitution properties \ilyam{todo}}\\
                                          &= [[ [Sub us'2|{â⁺*} ○ â⁺*/pas][us2]uN' ]]
                                          && \text{since $[[Sub us'2|uv uN']] = [[us2]]$}.
    \end{aligned}
    $\\
    Also notice that the domain of $[[Sub us'2|{â⁺*} ○ â⁺*/pas]]$ is $[[pas]]$,
    and the range is $[[Γ, pbs]]$, i.e. $[[Γ, pbs ⊢ Sub us'2|{â⁺*} ○ â⁺*/pas : pas]]$, 
    which means that we can apply \ruleref{\ottdruleDOneForallLabel} to 
    $[[ Γ, pbs ⊢ [Sub us'2|{â⁺*} ○ â⁺*/pas][us2]uN' ≤ iM' ]]$
    to get $[[ Γ ⊢ [us2]∀pas.uN' ≤ ∀pbs.iM' ]]$, as required.

    \item $[[Γ;Θ ⊨ â⁺ ≥ iP' ⫤ (â⁺ :≥ iQ')]]$ where
    $[[â⁺[Δ] ∊ Θ]]$ and $[[upgrade G ⊢ iP' to Δ = iQ']]$\\

    Notice that $[[â⁺[Δ] ∊ Θ]]$ and $[[Γ ⊢ Θ]]$ 
    implies $[[Γ = Δ, pnas]]$ for some $[[pnas]]$, hence, the
    soundness of upgrade (\cref{lemma:upgrade-soundness}) is applicable:
    \begin{enumerate}
        \item $[[Δ ⊢ iQ']]$ and
        \item $[[Γ ⊢ iQ ≥ iP]]$.
    \end{enumerate}

    Since $[[â⁺[Δ] ∊ Θ|uv â⁺]]$ and $[[Δ ⊢ iQ']]$, it is clear that $[[(â⁺ :≥ iQ') : Θ | uv â⁺ ]]$.

    It is left to show that $[[Γ ⊢ [us']â⁺ ≥ iP']]$ for any $[[us']]$ s.t. $[[Θ ⊢ us' ⇒ (â⁺ :≥ iQ')]]$.
    The latter weakening statement means that either $[[us']] \ni [[â⁺ :≥ iQ'']]$ or
    $[[us']] \ni [[(â⁺ :≈ iQ'')]]$ for $[[Δ ⊢ iQ'' ≥ iQ']]$. In any case,
    $[[Δ ⊢ [us']â⁺ ≥ iQ]]$. By weakening the context to $[[Γ]]$ and combining this judgment
    transitivity (\cref{todo}) with $[[Γ ⊢ iQ ≥ iP]]$, we have $[[Γ ⊢ [us']â⁺ ≥ iP]]$,
    as required. 

    \item For the other positive cases, the proof is symmetric to the corresponding negative cases.
    \end{caseof}
\end{proof}

Notice that the induction statement in following theorem is not symmetric for the positive and negative cases.
\begin{theorem}[Completeness and Initiality of the Subtyping Algorithm]
    \hfill
    \begin{itemize}
        \item [$-$] Suppose that $[[Γ ⊢ Θ]]$, $[[Γ ⊢ iM]]$, $[[Γ ; Θ ⊢ uN]]$,
        $[[uN]]$ does not contain negative unification variables ($[[α̂⁻]] \notin [[uv uN]]$)
        and there exists $[[us : Θ | uv uN]]$ such that $[[ Γ ⊢ [us]uN ≤ iM ]]$.
        Then there exists $[[us']]$ such that $[[Γ ; Θ ⊨ uN ≤ iM ⫤ us']]$
        and $[[Θ ⊢ us ⇒ us']]$.
        
        \item [$+$] Suppose that $[[Γ ⊢ Θ]]$, $[[Γ ⊢ iQ]]$ and $[[Γ ; Θ ⊢ uP]]$ and
        there exists $[[us : Θ | uv uP]]$ such that $[[ Γ ⊢ [us]uP ≥ iQ ]]$.
        Then there exists $[[us']]$ such that $[[Γ ; Θ ⊨ uP ≥ iQ ⫤ us']]$
        and $[[Θ ⊢ us ⇒ us']]$.
    \end{itemize}
\end{theorem}
\begin{proof}
    We prove it by induction on $[[ Γ ⊢ [us]uN ≤ iM ]]$ (and mutually, on $[[ Γ ⊢ [us]uP ≥ iQ ]]$).
    Let us consider the last rule used in the derivation of $[[ Γ ⊢ [us]uN ≤ iM ]]$ or $[[ Γ ⊢ [us]uP ≥ iQ ]]$,
    but first consider the base case for the substitution $[[ [us]uP ]]$:
    \begin{caseof}
        \item \label{case:subt-complete-base} $[[uP]] = [[ ∃nbs.α̂⁺ ]]$ (for potentially empty $[[pbs]]$)\\
        Then by assumption, there exists $[[us : Θ | uv uP]]$ such that $[[ Γ ⊢ ∃nbs.[us]α̂⁺ ≥ iQ ]]$.
        $[[us : Θ | uv α̂⁺]]$ means that  $[[us]]$ is either $[[ (α̂⁺ :≈ iP') ]]$ or $[[ (α̂⁺ :≥ iP') ]]$,
        where $[[Δ ⊢ iP']]$ and $[[α̂⁺[Δ] ∊ Θ]]$.
        $[[ Γ ⊢ ∃nbs.[us]α̂⁺ ≥ iQ ]]$ means that $[[Γ ⊢ iP' ≥ iQ]]$
        because multiple inversions of \ruleref{\ottdruleDOneExistsLabel} 
        gives us $[[Γ ⊢ [us]α̂⁺ ≥ iQ]]$ since $[[ {nbs} ∩ fv [us]α̂⁺ = ∅]]$.


        In the algorithm, after multiple applications of \ruleref{\ottdruleAExistsLabel}
        the type $[[∃nbs.α̂⁺]]$ is reduced to $[[α̂⁺]]$.
        Next, the algorithm tries to apply
        \ruleref{\ottdruleAPUVarLabel}
        and the resulting solution is $[[us']] = [[(α̂⁺ :≥ iQ')]]$ where
        $[[upgrade Γ ⊢ iQ to Δ = iQ']]$.

        Why does the upgrade procedure terminates?
        Because $[[iP']]$ satisfies the pre-conditions of the completeness of the upgrade
        (\cref{lemma:upgrade-completeness})
        :
        \begin{itemize}
            \item $[[Δ ⊢ iP']]$ because $[[iP' = [us]α̂⁺]]$, $[[us : Θ | uv uP]]$ and 
            $[[α̂⁺[Δ] ∊ Θ | uv uP]]$,
            \item $[[Γ ⊢ iP' ≥ iQ]]$ as noted before
        \end{itemize}
        Moreover, completeness of the upgrade also gives us $[[Γ ⊢ iP' ≥ iQ']]$
        and further, we strengthen it to $[[Δ ⊢ iP' ≥ iQ']]$
        (since by the soundness of the upgrade (\cref{lemma:upgrade-soundness}),
        $[[Δ ⊢ iQ']]$).

        It means that $[[Θ ⊢ (α̂⁺ :≈ iP') ⇒ (α̂⁺ :≥ iQ') ]]$ and 
        $[[ Θ ⊢ (α̂⁺ :≥ iP') ⇒ (α̂⁺ :≥ iQ') ]]$, which means that in any case, 
        $[[ Θ ⊢ us ⇒ us']]$.

        \item \label{case:subt-complete-nvar}
        $[[ Γ ⊢ [us]uN ≤ iM ]]$ is derived by \ruleref{\ottdruleDOneNVarLabel}, 
        i.e. $[[ [us]uN ]] = [[iM]] = [[ α⁻ ]] = [[iN]]$ 
        (the latter equality holds because $[[uN]] \neq [[α̂⁻]]$).

        Notice that $[[us : Θ | uv α⁻]]$ = $[[us : Θ | ∅]]$ = $[[·]]$.

        The algorithm applies \ruleref{\ottdruleANVarLabel} and 
        infers $[[us']] = [[·]]$, i.e. $[[Γ;Θ ⊨ a⁻ ≤ a⁻ ⫤ ·]]$.

        Since $[[Θ ⊢ · ⇒ ·]]$, we have $[[Θ ⊢ us ⇒ us']]$.

        \item $[[ Γ ⊢ [us]uN ≤ iM ]]$ is derived by \ruleref{\ottdruleDOneShiftULabel},
        \label{case:subt-complete-upshift}
        i.e. $[[ [us]uN ]] = [[ ↑[us]uP ]]$ (since the substitution $[[ [us]uN ]]$ must preserve the 
        top-level constructor of $[[uN]]\neq [[α̂⁻]]$), and $[[uM]] = [[ ↑iQ ]]$,
        and by inversion, $[[ Γ ⊢ [us]uP ≈ iQ ]]$.

        Since both types start with $[[↑]]$, 
        the algorithm tries to apply \ruleref{\ottdruleAShiftULabel}: 
        $[[G;Θ ⊨ ↑uP ≤ ↑iQ ⫤ us']]$. The premise of this rule is the
        unification of $[[nf(uP)]]$ and $[[nf(iQ)]]$:
        $[[Γ;Θ ⊨ nf(uP) ≈u nf(iQ) ⫤ us']]$. Let us show that the unification successfully 
        terminates and returns $[[us']] = [[nf(us)]]$.

        To demonstrate that the unification terminates, we apply the completeness 
        of the unification algorithm (\cref{lemma:unification-completeness}). 
        In order to do that, we need to provide a unifier of 
        $[[nf(uP)]]$ and $[[nf(iQ)]]$. Thankfully, $[[nf(us)]]$ does it. 

        \begin{itemize}
            \item $[[nf(uP)]]$ and $[[nf(iQ)]]$ are normalized 
            \item $[[Γ ; Θ ⊢ nf(uP)]]$ because $[[Γ ; Θ ⊢ uP]]$ (\cref{todo})
            \item $[[Γ ⊢ nf(iQ)]]$ because $[[Γ ⊢ iQ]]$ (\cref{corollary:wf-nf})
            \item $[[nf(us) : Θ | uv nf(uP)]]$ because $[[us: Θ | uv uP]]$ (\cref{todo})
            \item $ \begin{aligned}[t]
                    [[ Γ ⊢ [us]uP ≈ iQ ]] &\Rightarrow [[ [us]uP ≈ iQ ]]
                                          && \text {by \cref{lemma:equiv-completeness}}\\
                                          &\Rightarrow [[ nf([us]uP) = nf(iQ) ]]
                                          && \text {by \cref{lemma:normalization-completeness}}\\
                                          &\Rightarrow [[ [nf(us)]nf(uP) = nf(iQ) ]]
                                          && \text {by \cref{lemma:norm-subst-distr}}\\
                    \end{aligned}
                  $
        \end{itemize}
        Then by the completeness of the unification,
        $[[Γ ; Θ ⊨ nf(uP) ≈u nf(iQ) ⫤ nf(us)]]$.
        This way, the subtyping algorithm terminates and the resulting solution is
        $[[nf(us)]]$. 
        
        It is left to note that $[[Θ ⊢ us ⇒ nf(us)]]$, by \cref{todo}, since $[[Θ ⊢ us ≈ nf(us)]]$ 

        \item $[[ Γ ⊢ [us]uN ≤ iM ]]$ is derived by \ruleref{\ottdruleDOneArrowLabel}, 
        i.e. $[[ [us]uN ]] = [[ [us]uP → [us]uN' ]]$ and $[[iM]] = [[iQ → iM']]$, 
        and by inversion, $[[ Γ ⊢ [us]uP ≥ iQ ]]$ and $[[ Γ ⊢ [us]uN' ≤ iM' ]]$.

        Let us consider restrictions of $[[us]]$ to 
        the set of unification variables in $[[uP]]$ and $[[uN']]$:
        $[[us | uv uP : Θ | uv uP]]$ and $[[us | uv uN' : Θ | uv uN']]$.
        Notice that 
        $[[ [us]uP = [us | uv uP]uP ]]$ and 
        $[[ [us]uN' = [us | uv uN']uN' ]]$.

       Let us apply the induction hypothesis to
       $[[ Γ ⊢ [us | uv uP]uP ≥ iQ ]]$ and
       $[[ Γ ⊢ [us | uv uN']uN' ≤ iM' ]]$ (notice that
       since $[[uv uN' ⊆ uv uN]]$, there are no negative unification variables in $[[uN']]$)
       to obtain $[[us'1]]$ and $[[us'2]]$ such that
       \begin{enumerate}
        \item $[[Γ; Θ ⊨ uP ≥ iQ ⫤ us'1]]$ and $[[Θ ⊢ us | uv uP ⇒ us'1]]$
        \item $[[Γ; Θ ⊨ uP ≥ iQ ⫤ us'2]]$ and $[[Θ ⊢ us | uv uN' ⇒ us'2]]$ 
       \end{enumerate}
       
       This way, the algorithm applies \ruleref{\ottdruleAArrowLabel}
       and terminates returning $[[us'1 & us'2]]$ as the result.

       It is left to show that $[[Θ ⊢ us ⇒ us'1 & us'2]]$.
       Since $[[us | uv uP ⊆ us]]$, 
       by \cref{lemma:weakening-monotonicity}, 
       $[[Θ ⊢ us ⇒ us | uv uP]]$,
       and then by \cref{lemma:weakening-transitivity}, 
       $[[Θ ⊢ us | uv uP ⇒ us'1]]$.
       Analogously, $[[Θ ⊢ us | uv uN' ⇒ us'2]]$.
       Then since by \cref{lemma:merge-completeness}, 
       $[[us'1 & us'2]]$ is the weakest restriction  
       implying both $[[us'1]]$ and $[[us'2]]$,
       we have $[[Θ ⊢ us ⇒ us'1 & us'2]]$, as required.

       \item \label{case:subt-complete-forall}
       $[[ Γ ⊢ [us]uN ≤ iM ]]$ is derived by \ruleref{\ottdruleDOneForallLabel},
       i.e. $[[ [us]uN ]] = [[ ∀pas.[us]uN' ]]$
       (assuming $[[pas]]$ does not intersect with the 
       range of $[[us]]$)
       and 
       $[[iM]] = [[∀pbs.iM']]$, where either $[[pas]]$ or $[[pbs]]$
       is non-empty.
       
       By inversion, 
       $[[ Γ, pbs ⊢ [σ][us]uN' ≤ iM' ]]$ for 
       $[[Γ, pbs ⊢ σ : pas ]]$.
       Since $[[σ]]$ and $[[us]]$ have disjoint domains,
       and the range of one does not intersect with the domain of the other,
       they commute, i.e. $[[ Γ, pbs ⊢ [us][σ]uN' ≤ iM' ]]$
       (notice that the tree inferring this judgement is 
       a proper subtree of the tree inferring 
       $[[ Γ ⊢ [us]uN ≤ iM ]]$).

       On the next step, 
       the algorithm creates fresh (disjoint with $[[uv uN']]$) 
       unification variables $[[â⁺*]]$, replaces $[[pas]]$ with them in $[[ [us]uN' ]]$,
       and makes the recursive call:
       $[[G, pbs; Θ, â⁺*[G, pbs] ⊨ uN0 ≤ iM ⫤ us1]]$,
       (where $[[uN0]] = [[ [â⁺*/pas]uN' ]]$),
       returning $[[us1 \ {â⁺*}]]$ as the result.

       Notice that $[[ [us][σ][pas/â⁺*]uN0 = [us][σ]uN' ]]$,
       and thus, $[[ Γ, pbs ⊢ [us][σ][pas/â⁺*]uN0 ≤ iM' ]]$.
       Let us combine $[[us]]$ and $[[σ  ○ pas/â⁺*]]$ into one 
       $[[us0]]$ an unification solution for $[[uN0]]$:
        \[
            [[us0(β̂±)]]  = 
            \begin{cases}
               [[ β̂± :≈ [σ]αi⁺ ]] & \text{if } [[β̂±]] = [[αî⁺]] \in [[â⁺*]] \\
               [[us(β̂±)]] & \text{if } [[β̂±]] \in [[uv uN']]
            \end{cases}
       \]

       Let us show that the induction hypothesis is applicable to 
       $[[Γ, pbs ⊢ [us0]uN0 ≤ iM' ]]$.
       \begin{itemize}
        \item $[[ us0 : Θ|uv uN', â⁺*[Γ, pbs] ]]$. Notice that for every $[[αî⁺]]$, 
            the type corresponding to the entry $[[us0(αî⁺)]]$ 
            is well-formed in $[[Γ, pbs]]$ since $[[Γ, pbs ⊢ σ : pas ]]$;
            and for every $[[β̂±]] \in [[uv uN']]$, 
            the type corresponding to the entry $[[us0(β̂±)]]$ is well-formed in
            context $[[Θ(β̂±)]]$ since $[[us : Θ | uv uN']]$.
        \item $[[Γ, pbs ⊢ [us0]uN0 ≤ iM' ]]$. This is because
            $[[ [us0]uN0 ]] = [[ [us][σ][pas/â⁺*]uN0 ]] = [[ [us][σ]uN' ]]$.
        \item $[[ Γ, pbs ⊢ Θ, â⁺*[Γ, pbs] ]]$ since
            $[[Γ, pbs ⊢ Θ]]$ and $[[ {Γ, pbs} ⊆ {Γ, pbs} ]]$.
        \item $[[Γ, pbs ; Θ, â⁺*[Γ, pbs] ⊢ uN0 ]]$
        \item There are no negative unification variables in $[[uN0]]$ because
            $[[ uv uN0 ⊆ uv uN ∪ {â⁺*} ]]$.
       \end{itemize}

       This way, we apply the induction hypothesis to $[[Γ, pbs ⊢ [us0]uN0 ≤ iM' ]]$ and 
       infer that there exists\\
        $[[us0' :  (Θ, â⁺*[Γ, pbs]) | uv uN0]]$ such that
       $[[G, pbs; Θ, â⁺*[G, pbs] ⊨ uN0 ≤ iM ⫤ us0']]$,
       and $[[Θ, â⁺*[G, pbs] ⊢ us0' ⇒ us0]]$.

       This way, the algorithm terminates with the result $[[us0' \ {â⁺*}]]$
       and it is left to show $[[ Θ ⊢ us ⇒ us0' \ {â⁺*} ]]$. Notice that
       $[[us0' \ {â⁺*} : Θ | uv uN' ]]$. This way, to show $[[ Θ ⊢ us ⇒ us0' \
       {â⁺*} ]]$. it suffices to consider an unification variable $[[β̂±]] ∊
       [[uv uN']]$ and show that $[[Γ, pbs ⊢ us(β̂±) ⇒ us0'(β̂±)]]$. It holds by the
       reflexivity of weakening (\cref{lemma:weakening-reflexivity}) since
       $[[us0'(β̂±)]] = [[us(β̂±)]]$.

       \item The positive case when $[[Γ ⊢ [us]uP ≥ iQ]]$ is derived by 
       \ruleref{\ottdruleDOneShiftDLabel} is symmetric to the corresponding negative
        \cref{case:subt-complete-upshift}. Notice that the algorithm does not 
        make a recursive call in this case but rather invoke the unification.
        This way, the proof does not apply the induction hypothesis, and hence, 
        the absence of negative unification variables in a negative type
        ($[[α̂⁻]] \notin [[uN]]$ for $[[↓uN]] =[[uP]]$) is not required.

      \item The positive case when $[[Γ ⊢ [us]uP ≥ iQ]]$ is derived by 
      \ruleref{\ottdruleDOneExistsLabel} is symmetric to the corresponding negative
      \cref{case:subt-complete-forall}. Notice that we only consider the case
      when the substitution $[[ [us]uP ]]$ results in the existential type 
      $[[∃nas.iP']]$ (for $[[iP']] \neq [[∃]]\dots$) by congruence, 
      i.e. $[[uP = ∃nas.uP'']]$ (for $[[uP'']] \neq [[∃]]\dots$) and $[[ [us]uP'' = iP' ]]$.
      This is because the case when $[[uP = ∃nbs.α̂⁺]]$ has been covered
      (\cref{case:subt-complete-base}), and thus, the substitution $[[us]]$ must
      preserve all the outer quantifiers of $[[uP]]$ and does not generate any new ones.

      \item The positive case when $[[Γ ⊢ [us]uP ≥ iQ]]$ is derived by 
      \ruleref{\ottdruleDOnePVarLabel} is symmetric to the corresponding negative
      \cref{case:subt-complete-nvar}.
      Notice that $[[ [us]uP = β⁺ ]]$ 
      implies $[[uP = β⁺]]$ because the case when $[[uP = α̂⁺]]$ has been covered 
      (\cref{case:subt-complete-base}).
    \end{caseof}
\end{proof}




\end{document}
