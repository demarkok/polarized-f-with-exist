\UseRawInputEncoding
% vim: ft=tex
\documentclass[a4,natbib=false]{article}
\usepackage[a4paper, total={8in, 10in}]{geometry}
\usepackage{hyperref}
\usepackage{mathpartir}


\usepackage{lscape}
\usepackage{amsmath}
\usepackage{amsthm}
\usepackage{amssymb}
\usepackage{booktabs}
\usepackage{multicol}
\usepackage{supertabular}
\usepackage[inline]{enumitem}
\usepackage{cleveref}
\usepackage{proof}

\usepackage{stackengine}

\usepackage{mathabx}
\usepackage[dvipsnames]{xcolor}
\usepackage{scalerel}


\usepackage{todonotes}

\usepackage{enumitem}
\usepackage{xparse}
\usepackage{casenum}

\usepackage{braket}

\newcommand{\niton}{\not\owns}

\newcommand{\ilyam}[1]{{\color{red} \texttt{Ilya:  #1}}}

\newtheorem{definition}{Definition}
\newtheorem{theorem}{Theorem}
\newtheorem{lemma}{Lemma}
\newtheorem{corollary}{Corollary}
\newtheorem{observation}{Observation}
\newtheorem*{assertion*}{Assertion}

% https://tex.stackexchange.com/questions/85033/colored-symbols/85035#85035
\newcommand*{\mathcolor}{}
\def\mathcolor#1#{ \mathcoloraux{#1} }
\newcommand*{\mathcoloraux}[3]{%
  \protect\leavevmode
  \begingroup
  \color#1{#2}#3%
  \endgroup
}

\newcommand{\UB}[0]{\mathsf{UB}}
\newcommand{\NFUB}[0]{\mathsf{NFUB}}


% \newcounter{casenum}

% \newenvironment{caseof}
% {%
%   \par
%   \setlength{\parskip}{6pt}%
%   % \setlength{\parindent}{0pt}%
%   \everypar{\setlength{\hangindent}{17pt}}%
%   \setcounter{casenum}{0}%
% }
% {\par\vskip.5\baselineskip}

% \NewDocumentCommand{\case}{omm}{%
%   % \vskip.5\baselineskip\par%
%   \itemindent\parindent
%   \refstepcounter{casenum}%
%   {\bfseries Case} {\bfseries \arabic{casenum}}%
%   \IfNoValueF{#1}{\label{#1}}%
%   {\bfseries:} #2\\#3 %
% }



% generated by Ott 0.32 from: grammar.ott rules.ott unification.ott
\newcommand{\ottdrule}[4][]{{\displaystyle\frac{\begin{array}{l}#2\end{array}}{#3}\quad\ottdrulename{#4}}}
\newcommand{\ottusedrule}[1]{\[#1\]}
\newcommand{\ottpremise}[1]{ #1 \\}
\newenvironment{ottdefnblock}[3][]{ \framebox{\mbox{#2}} \quad #3 \\[0pt]}{}
\newenvironment{ottfundefnblock}[3][]{ \framebox{\mbox{#2}} \quad #3 \\[0pt]\begin{displaymath}\begin{array}{l}}{\end{array}\end{displaymath}}
\newcommand{\ottfunclause}[2]{ #1 \equiv #2 \\}
\newcommand{\ottnt}[1]{\mathit{#1}}
\newcommand{\ottmv}[1]{\mathit{#1}}
\newcommand{\ottkw}[1]{\mathbf{#1}}
\newcommand{\ottsym}[1]{#1}
\newcommand{\ottcom}[1]{\text{#1}}
\newcommand{\ottdrulename}[1]{\textsc{#1}}
\newcommand{\ottcomplu}[5]{\overline{#1}^{\,#2\in #3 #4 #5}}
\newcommand{\ottcompu}[3]{\overline{#1}^{\,#2<#3}}
\newcommand{\ottcomp}[2]{\overline{#1}^{\,#2}}
\newcommand{\ottgrammartabular}[1]{\begin{supertabular}{llcllllll}#1\end{supertabular}}
\newcommand{\ottmetavartabular}[1]{\begin{supertabular}{ll}#1\end{supertabular}}
\newcommand{\ottrulehead}[3]{$#1$ & & $#2$ & & & \multicolumn{2}{l}{#3}}
\newcommand{\ottprodline}[6]{& & $#1$ & $#2$ & $#3 #4$ & $#5$ & $#6$}
\newcommand{\ottfirstprodline}[6]{\ottprodline{#1}{#2}{#3}{#4}{#5}{#6}}
\newcommand{\ottlongprodline}[2]{& & $#1$ & \multicolumn{4}{l}{$#2$}}
\newcommand{\ottfirstlongprodline}[2]{\ottlongprodline{#1}{#2}}
\newcommand{\ottbindspecprodline}[6]{\ottprodline{#1}{#2}{#3}{#4}{#5}{#6}}
\newcommand{\ottprodnewline}{\\}
\newcommand{\ottinterrule}{\\[5.0mm]}
\newcommand{\ottafterlastrule}{\\}

\newcommand{\appRightarrow}{ \mathcolor{OliveGreen}{\Rightarrow \hspace{-7pt} \Rightarrow} }

\newcommand{\tripprox}{\setbox0\hbox{$\approx$} \mbox{\makebox[0pt][l]{\raisebox{0.48\ht0}{$\approx$} }$\approx$} }

\newcommand{\approxRight}{ \mathrel{ \tripprox \hspace{-2.3pt}  \raisebox{0.24\ht0}{$>$} } }
\newcommand{\appBull}{ \mathcolor{OliveGreen}{\bullet} }
\newcommand{\rcolor}{blue}
\newcommand{\ccolor}{purple}

\usepackage{mathabx}
\usepackage{color}
\usepackage[dvipsnames,usenames]{xcolor}

% https://tex.stackexchange.com/questions/33401/a-version-of-colorbox-that-works-inside-math-environments
\setlength{\fboxsep}{1pt}
\newcommand{\ngbox}[1]{\mathchoice%
  {\colorbox{black!8}{$\displaystyle      \mathit{ #1 } $} }%
  {\colorbox{black!8}{$\textstyle         \mathit{ #1 } $} }%
  {\colorbox{black!8}{$\scriptstyle       \mathit{ #1 } $} }%
  {\colorbox{black!8}{$\scriptscriptstyle \mathit{ #1 } $} } }%

% https://tex.stackexchange.com/questions/85033/colored-symbols/85035#85035
\newcommand*{\mathcolor}{}
\def\mathcolor#1#{ \mathcoloraux{#1} }
\newcommand*{\mathcoloraux}[3]{%
  \protect\leavevmode
  \begingroup
    \color#1{#2}#3%
  \endgroup
}

\newcommand{\ottmetavars}{
\ottmetavartabular{
 $ \ottmv{x} ,\, \ottmv{y} $ & \ottcom{term variable} \\
 $ \ottmv{f} ,\, \ottmv{g} $ & \ottcom{constructors} \\
 $ \widehat{\alpha} ,\, \widehat{\beta} ,\, \widehat{\gamma} ,\, \widehat{\delta} $ & \ottcom{unification variable} \\
 $ \vec{x} ,\, \vec{y} ,\, \vec{z} ,\, \vec{t} $ & \ottcom{variable list} \\
 $ \ottmv{n} ,\, \ottmv{m} ,\, \ottmv{i} ,\, \ottmv{j} $ & \ottcom{index variables} \\
}}

\newcommand{\ottarn}{
\ottrulehead{n  ,\ k}{::=}{\ottcom{arity}}\ottprodnewline
\ottfirstprodline{|}{\ottsym{0}}{}{}{}{}\ottprodnewline
\ottprodline{|}{\ottsym{1}}{}{}{}{}\ottprodnewline
\ottprodline{|}{\ottsym{2}}{}{}{}{}\ottprodnewline
\ottprodline{|}{n_{{\mathrm{1}}}  \ottsym{+}  n_{{\mathrm{2}}}}{}{}{}{}\ottprodnewline
\ottprodline{|}{\ottsym{\mbox{$\mid$}}  \ottnt{vars}  \ottsym{\mbox{$\mid$}}}{}{}{}{}}

\newcommand{\ottvars}{
\ottrulehead{\ottnt{vars}}{::=}{\ottcom{variable list}}\ottprodnewline
\ottfirstprodline{|}{\vec{x}}{}{}{}{}\ottprodnewline
\ottprodline{|}{\ottmv{x_{{\mathrm{1}}}}  \ottsym{,} \, .. \, \ottsym{,}  \ottmv{x_{\ottmv{n}}}}{}{}{}{}\ottprodnewline
\ottprodline{|}{\ottnt{vars_{{\mathrm{1}}}}  \cap  \ottnt{vars_{{\mathrm{2}}}}}{}{}{}{}\ottprodnewline
\ottprodline{|}{\ottnt{vars_{{\mathrm{1}}}}  \sqcap  \ottnt{vars_{{\mathrm{2}}}}}{}{}{}{}\ottprodnewline
\ottprodline{|}{\ottcomp{\ottnt{vars_{\ottmv{i}}}}{\ottmv{i}}}{}{}{}{}\ottprodnewline
\ottprodline{|}{\ottkw{UVARGS} \, \ottnt{t}} {\textsf{M}}{}{\textsf{[F]}}{\ottcom{arguments of the unification variables of the term}}}

\newcommand{\ottt}{
\ottrulehead{\ottnt{t}  ,\ \ottnt{v}  ,\ \ottnt{w}  ,\ \ottnt{h}  ,\ \ottnt{d}}{::=}{\ottcom{terms}}\ottprodnewline
\ottfirstprodline{|}{\ottmv{x}}{}{}{}{}\ottprodnewline
\ottprodline{|}{\ottmv{x}  \ottsym{.}  \ottnt{t}}{}{\textsf{bind}\; \ottmv{x}\; \textsf{in}\; \ottnt{t}}{}{}\ottprodnewline
\ottprodline{|}{\ottnt{vars}  \ottsym{.}  \ottnt{t}}{}{}{}{}\ottprodnewline
\ottprodline{|}{\widehat{\alpha}  \ottsym{[}  \ottnt{vars}  \ottsym{]}}{}{}{}{}\ottprodnewline
\ottprodline{|}{\ottmv{f}  \ottsym{(}  \ottnt{t_{{\mathrm{1}}}}  \ottsym{,..,}  \ottnt{t_{\ottmv{n}}}  \ottsym{)}}{}{}{}{}\ottprodnewline
\ottprodline{|}{\ottsym{[}  \Theta  \ottsym{]}  \ottnt{v}} {\textsf{M}}{}{}{}\ottprodnewline
\ottprodline{|}{\ottsym{(}  \ottnt{v}  \ottsym{)}} {\textsf{S}}{}{}{}\ottprodnewline
\ottprodline{|}{\ottsym{\{}  \widehat{\alpha}_{{\mathrm{1}}}  \ottsym{[}  \ottnt{vars_{{\mathrm{1}}}}  \ottsym{]}  \ottsym{/}  \widehat{\alpha}_{{\mathrm{2}}}  \ottsym{[}  \ottnt{vars_{{\mathrm{2}}}}  \ottsym{]}  \ottsym{\}}  \ottnt{t}}{}{}{}{}}

\newcommand{\ottterminals}{
\ottrulehead{\ottnt{terminals}}{::=}{}\ottprodnewline
\ottfirstprodline{|}{ \in }{}{}{}{}\ottprodnewline
\ottprodline{|}{ \notin }{}{}{}{}\ottprodnewline
\ottprodline{|}{ \cdot }{}{}{}{}\ottprodnewline
\ottprodline{|}{ \vdash }{}{}{}{}\ottprodnewline
\ottprodline{|}{ \mathcolor{\rcolor}{\vDash} }{}{}{}{}\ottprodnewline
\ottprodline{|}{ \mathcolor{\rcolor}{\Dashv} }{}{}{}{}\ottprodnewline
\ottprodline{|}{ \mathcolor{\ccolor}{\VDash} }{}{}{}{}\ottprodnewline
\ottprodline{|}{ \mathcolor{\ccolor}{\DashV} }{}{}{}{}\ottprodnewline
\ottprodline{|}{ \neq }{}{}{}{}\ottprodnewline
\ottprodline{|}{ \appRightarrow }{}{}{}{}\ottprodnewline
\ottprodline{|}{ \appBull }{}{}{}{}\ottprodnewline
\ottprodline{|}{ \mathcolor{\rcolor}{\equiv} }{}{}{}{}\ottprodnewline
\ottprodline{|}{ \equiv_{n} }{}{}{}{}\ottprodnewline
\ottprodline{|}{ \searrow }{}{}{}{}\ottprodnewline
\ottprodline{|}{ \unlhd }{}{}{}{}\ottprodnewline
\ottprodline{|}{ \cap }{}{}{}{}\ottprodnewline
\ottprodline{|}{ \sqcap }{}{}{}{}\ottprodnewline
\ottprodline{|}{ \subseteq }{}{}{}{}\ottprodnewline
\ottprodline{|}{ \emptyset }{}{}{}{}\ottprodnewline
\ottprodline{|}{ \approxRight }{}{}{}{}}

\newcommand{\ottT}{
\ottrulehead{\Theta}{::=}{\ottcom{computational variable context}}\ottprodnewline
\ottfirstprodline{|}{\ottmv{x}}{}{}{}{\ottcom{a variable}}\ottprodnewline
\ottprodline{|}{\vec{x}} {\textsf{S}}{}{}{\ottcom{variables}}\ottprodnewline
\ottprodline{|}{\widehat{\alpha}  \ottsym{:}  n}{}{}{}{\ottcom{a unification variable}}\ottprodnewline
\ottprodline{|}{\widehat{\alpha}  \ottsym{:}  n  \ottsym{=}  \ottnt{t}}{}{}{}{\ottcom{instantiate a unification variable}}\ottprodnewline
\ottprodline{|}{\ottcomp{\Theta_{\ottmv{i}}}{\ottmv{i}}}{}{}{}{\ottcom{concatenate contexts}}\ottprodnewline
\ottprodline{|}{\cdot}{}{}{}{\ottcom{empty context}}\ottprodnewline
\ottprodline{|}{\Theta_{{\mathrm{1}}}  \ottsym{\{}  \Theta_{{\mathrm{2}}}  \ottsym{\}}} {\textsf{S}}{}{}{\ottcom{surgery}}\ottprodnewline
\ottprodline{|}{\ottsym{(}  \Theta  \ottsym{)}} {\textsf{S}}{}{}{}\ottprodnewline
\ottprodline{|}{\Theta_{{\mathrm{1}}}  \ottsym{\mbox{$\backslash{}$}}  \ottsym{(}  \widehat{\alpha}_{{\mathrm{1}}}  \ottsym{,..,}  \widehat{\alpha}_{\ottmv{n}}  \ottsym{)}} {\textsf{S}}{}{}{\ottcom{context subtraction}}\ottprodnewline
\ottprodline{|}{ \Theta' ^{\color{red}\star} } {\textsf{M}}{}{\textsf{[F]}}{\ottcom{context self-application}}}

\newcommand{\ottformula}{
\ottrulehead{\ottnt{formula}}{::=}{}\ottprodnewline
\ottfirstprodline{|}{\ottnt{judgement}}{}{}{}{}\ottprodnewline
\ottprodline{|}{\ottmv{x}  \in  \Theta}{}{}{}{\ottcom{lookup $\ottmv{x}$ in context $\Theta$}}\ottprodnewline
\ottprodline{|}{\widehat{\alpha}  \notin  \ottnt{t}}{}{}{}{}\ottprodnewline
\ottprodline{|}{\vec{x}  \subseteq  \Theta}{}{}{}{}\ottprodnewline
\ottprodline{|}{\ottkw{let} \, \Theta_{{\mathrm{1}}}  \ottsym{=}  \Theta_{{\mathrm{2}}}}{}{}{}{}\ottprodnewline
\ottprodline{|}{\ottkw{let} \, \vec{x}  \ottsym{=}  \ottnt{vars}}{}{}{}{}\ottprodnewline
\ottprodline{|}{\ottnt{vars}  \cap  \Theta  \ottsym{=}  \emptyset}{}{}{}{}\ottprodnewline
\ottprodline{|}{\ottnt{vars_{{\mathrm{1}}}}  \cap  \ottnt{vars_{{\mathrm{2}}}}  \ottsym{=}  \emptyset}{}{}{}{}\ottprodnewline
\ottprodline{|}{\ottkw{UV} \, \ottsym{(}  \ottnt{t}  \ottsym{)}  \ottsym{=}  \widehat{\alpha}_{{\mathrm{1}}}  \ottsym{[}  \ottnt{vars_{{\mathrm{1}}}}  \ottsym{]}  \ottsym{,..,}  \widehat{\alpha}_{\ottmv{n}}  \ottsym{[}  \ottnt{vars_{\ottmv{n}}}  \ottsym{]}}{}{}{}{}\ottprodnewline
\ottprodline{|}{\ottkw{ux} \, \ottsym{:}  n  \in  \Theta}{}{}{}{\ottcom{lookupof $\ottkw{ux}$ in context $\Theta$}}\ottprodnewline
\ottprodline{|}{\ottkw{ux} \, \ottsym{:}  n  \ottsym{=}  \ottnt{t}  \in  \Theta}{}{}{}{\ottcom{lookup type of $\ottkw{ux}$  instantiation in context $\Theta$}}\ottprodnewline
\ottprodline{|}{\ottnt{v}  \neq  \ottnt{w}}{}{}{}{}\ottprodnewline
\ottprodline{|}{\vec{x}  \ottsym{=}  \ottnt{vars}}{}{}{}{}\ottprodnewline
\ottprodline{|}{\ottkw{arity} \, \ottmv{f}  \ottsym{=}  \ottsym{[}  n_{{\mathrm{1}}}  \ottsym{,..,}  n_{\ottmv{n}}  \ottsym{]}}{}{}{}{}\ottprodnewline
\ottprodline{|}{\ottnt{formula_{{\mathrm{1}}}} \quad .. \quad \ottnt{formula_{\ottmv{n}}}}{}{}{}{}\ottprodnewline
\ottprodline{|}{ \cdots }{}{}{}{}}

\newcommand{\ottFoo}{
\ottrulehead{\ottnt{Foo}}{::=}{}\ottprodnewline
\ottfirstprodline{|}{ \Theta' ^{\color{red}\star} }{}{}{}{\ottcom{context self-application}}\ottprodnewline
\ottprodline{|}{\ottkw{UVARGS} \, \ottnt{t}  \ottsym{===}  \ottnt{vars}}{}{}{}{\ottcom{arguments of the unification variables of the term}}}

\newcommand{\ottAOne}{
\ottrulehead{\ottnt{A1}}{::=}{}\ottprodnewline
\ottfirstprodline{|}{\Theta_{{\mathrm{1}}}  \mathcolor{\rcolor}{\vDash}  \ottnt{v}  \mathcolor{\rcolor}{\equiv}  \ottnt{w}  \ottsym{:}  n  \mathcolor{\rcolor}{\Dashv}  \Theta_{{\mathrm{2}}}}{}{}{}{\ottcom{The unification}}}

\newcommand{\ottBOne}{
\ottrulehead{\ottnt{B1}}{::=}{}\ottprodnewline
\ottfirstprodline{|}{\Theta_{{\mathrm{1}}}  \mathcolor{\ccolor}{\VDash}  \ottnt{v}  \cap  \ottsym{[}  \ottnt{vars}  \ottsym{]}  \approxRight  \ottnt{w}  \mathcolor{\ccolor}{\DashV}  \Theta_{{\mathrm{2}}}}{}{}{}{\ottcom{The prunning phase}}\ottprodnewline
\ottprodline{|}{\Theta_{{\mathrm{1}}}  \mathcolor{\ccolor}{\VDash}  \ottnt{v}  \mathcolor{\rcolor}{\equiv}  \ottnt{w}  \mathcolor{\ccolor}{\DashV}  \Theta_{{\mathrm{2}}}}{}{}{}{\ottcom{The alternative unification}}\ottprodnewline
\ottprodline{|}{\ottnt{v} \, \ottkw{ext}}{}{}{}{\ottcom{The external term}}\ottprodnewline
\ottprodline{|}{\Theta \, \ottkw{ext}}{}{}{}{\ottcom{The external environment}}}

\newcommand{\ottjudgement}{
\ottrulehead{\ottnt{judgement}}{::=}{}\ottprodnewline
\ottfirstprodline{|}{\ottnt{A1}}{}{}{}{}\ottprodnewline
\ottprodline{|}{\ottnt{B1}}{}{}{}{}}

\newcommand{\ottuserXXsyntax}{
\ottrulehead{\ottnt{user\_syntax}}{::=}{}\ottprodnewline
\ottfirstprodline{|}{\ottmv{x}}{}{}{}{}\ottprodnewline
\ottprodline{|}{\ottmv{f}}{}{}{}{}\ottprodnewline
\ottprodline{|}{\widehat{\alpha}}{}{}{}{}\ottprodnewline
\ottprodline{|}{\vec{x}}{}{}{}{}\ottprodnewline
\ottprodline{|}{\ottmv{n}}{}{}{}{}\ottprodnewline
\ottprodline{|}{n}{}{}{}{}\ottprodnewline
\ottprodline{|}{\ottnt{vars}}{}{}{}{}\ottprodnewline
\ottprodline{|}{\ottnt{t}}{}{}{}{}\ottprodnewline
\ottprodline{|}{\ottnt{terminals}}{}{}{}{}\ottprodnewline
\ottprodline{|}{\Theta}{}{}{}{}\ottprodnewline
\ottprodline{|}{\ottnt{formula}}{}{}{}{}}

\newcommand{\ottgrammar}{\ottgrammartabular{
\ottarn\ottinterrule
\ottvars\ottinterrule
\ottt\ottinterrule
\ottterminals\ottinterrule
\ottT\ottinterrule
\ottformula\ottinterrule
\ottFoo\ottinterrule
\ottAOne\ottinterrule
\ottBOne\ottinterrule
\ottjudgement\ottinterrule
\ottuserXXsyntax\ottafterlastrule
}}

% defnss
% fundefns Foo
% fundefn simpl

\newcommand{\ottfundefnsimpl}[1]{\begin{ottfundefnblock}[#1]{$ \Theta' ^{\color{red}\star} $}{\ottcom{context self-application}}
\ottfunclause{ \cdot ^{\color{red}\star} }{\cdot}%
\ottfunclause{ \ottsym{(}  \Theta  \ottsym{,}  \ottmv{x}  \ottsym{)} ^{\color{red}\star} }{ \Theta ^{\color{red}\star}   \ottsym{,}  \ottmv{x}}%
\ottfunclause{ \ottsym{(}  \Theta  \ottsym{,}  \widehat{\alpha}  \ottsym{:}  n  \ottsym{)} ^{\color{red}\star} }{ \Theta ^{\color{red}\star}   \ottsym{,}  \widehat{\alpha}  \ottsym{:}  n}%
\ottfunclause{ \ottsym{(}  \Theta  \ottsym{,}  \widehat{\alpha}  \ottsym{:}  n  \ottsym{=}  \ottnt{t}  \ottsym{)} ^{\color{red}\star} }{ \Theta ^{\color{red}\star}   \ottsym{,}  \widehat{\alpha}  \ottsym{:}  n  \ottsym{=}  \ottsym{[}   \Theta ^{\color{red}\star}   \ottsym{]}  \ottnt{t}}%
\end{ottfundefnblock}}


% fundefn uvarargs

\newcommand{\ottfundefnuvarargs}[1]{\begin{ottfundefnblock}[#1]{$\ottkw{UVARGS} \, \ottnt{t}$}{\ottcom{arguments of the unification variables of the term}}
\end{ottfundefnblock}}


\newcommand{\ottfundefnsFoo}{
\ottfundefnsimpl{}
\ottfundefnuvarargs{}}

% defns A1
%% defn un
\newcommand{\ottdruleVXXV}[1]{\ottdrule[#1]{%
\ottpremise{\ottmv{x}  \in  \Theta}%
}{
\Theta  \mathcolor{\rcolor}{\vDash}  \ottmv{x}  \mathcolor{\rcolor}{\equiv}  \ottmv{x}  \ottsym{:}  \ottsym{0}  \mathcolor{\rcolor}{\Dashv}  \Theta}{%
{\ottdrulename{V\_V}}{}%
}}


\newcommand{\ottdruleBXXB}[1]{\ottdrule[#1]{%
\ottpremise{\Theta_{{\mathrm{1}}}  \ottsym{,}  \ottmv{x}  \mathcolor{\rcolor}{\vDash}  \ottnt{t}  \mathcolor{\rcolor}{\equiv}  \ottnt{t}  \ottsym{:}  n  \mathcolor{\rcolor}{\Dashv}  \Theta_{{\mathrm{2}}}  \ottsym{,}  \ottmv{x}}%
}{
\Theta_{{\mathrm{1}}}  \mathcolor{\rcolor}{\vDash}  \ottmv{x}  \ottsym{.}  \ottnt{t}  \mathcolor{\rcolor}{\equiv}  \ottmv{x}  \ottsym{.}  \ottnt{t}  \ottsym{:}  n  \ottsym{+}  \ottsym{1}  \mathcolor{\rcolor}{\Dashv}  \Theta_{{\mathrm{2}}}}{%
{\ottdrulename{B\_B}}{}%
}}


\newcommand{\ottdruleFXXF}[1]{\ottdrule[#1]{%
\ottpremise{\ottkw{arity} \, \ottmv{f}  \ottsym{=}  \ottsym{[}  k_{{\mathrm{1}}}  \ottsym{,..,}  k_{\ottmv{n}}  \ottsym{]}}%
\ottpremise{\Theta_{{\mathrm{0}}}  \mathcolor{\rcolor}{\vDash}  \ottnt{v_{{\mathrm{1}}}}  \mathcolor{\rcolor}{\equiv}  \ottnt{w_{{\mathrm{1}}}}  \ottsym{:}  k_{{\mathrm{1}}}  \mathcolor{\rcolor}{\Dashv}  \Theta_{{\mathrm{1}}}}%
\ottpremise{\Theta_{{\mathrm{1}}}  \mathcolor{\rcolor}{\vDash}  \ottsym{[}  \Theta_{{\mathrm{1}}}  \ottsym{]}  \ottnt{v_{{\mathrm{2}}}}  \mathcolor{\rcolor}{\equiv}  \ottsym{[}  \Theta_{{\mathrm{1}}}  \ottsym{]}  \ottnt{w_{{\mathrm{2}}}}  \ottsym{:}  k_{{\mathrm{2}}}  \mathcolor{\rcolor}{\Dashv}  \Theta_{{\mathrm{2}}}}%
\ottpremise{ \cdots }%
\ottpremise{\Theta_{{\ottmv{n}-1}}  \mathcolor{\rcolor}{\vDash}  \ottsym{[}  \Theta_{{\ottmv{n}-1}}  \ottsym{]}  \ottnt{v_{\ottmv{n}}}  \mathcolor{\rcolor}{\equiv}  \ottsym{[}  \Theta_{{\ottmv{n}-1}}  \ottsym{]}  \ottnt{w_{\ottmv{n}}}  \ottsym{:}  k_{\ottmv{n}}  \mathcolor{\rcolor}{\Dashv}  \Theta_{\ottmv{n}}}%
}{
\Theta_{{\mathrm{0}}}  \mathcolor{\rcolor}{\vDash}  \ottmv{f}  \ottsym{(}  \ottnt{v_{{\mathrm{1}}}}  \ottsym{,..,}  \ottnt{v_{\ottmv{n}}}  \ottsym{)}  \mathcolor{\rcolor}{\equiv}  \ottmv{f}  \ottsym{(}  \ottnt{w_{{\mathrm{1}}}}  \ottsym{,..,}  \ottnt{w_{\ottmv{n}}}  \ottsym{)}  \ottsym{:}  \ottsym{0}  \mathcolor{\rcolor}{\Dashv}  \Theta_{\ottmv{n}}}{%
{\ottdrulename{F\_F}}{}%
}}


\newcommand{\ottdruleUVXXV}[1]{\ottdrule[#1]{%
}{
\Theta  \ottsym{\{}  \widehat{\alpha}  \ottsym{:}  n  \ottsym{\}}  \mathcolor{\rcolor}{\vDash}  \widehat{\alpha}  \ottsym{[}  \vec{x}  \ottsym{]}  \mathcolor{\rcolor}{\equiv}  \ottmv{x_{\ottmv{i}}}  \ottsym{:}  \ottsym{0}  \mathcolor{\rcolor}{\Dashv}   \ottsym{(}  \Theta  \ottsym{\{}  \widehat{\alpha}  \ottsym{:}  n  \ottsym{=}  \vec{x}  \ottsym{.}  \ottmv{x_{\ottmv{i}}}  \ottsym{\}}  \ottsym{)} ^{\color{red}\star} }{%
{\ottdrulename{UV\_V}}{}%
}}


\newcommand{\ottdruleUVXXUV}[1]{\ottdrule[#1]{%
\ottpremise{\vec{z}  \ottsym{=}  \vec{x}  \sqcap  \vec{y}}%
}{
\Theta  \ottsym{\{}  \widehat{\alpha}  \ottsym{:}  n  \ottsym{\}}  \mathcolor{\rcolor}{\vDash}  \widehat{\alpha}  \ottsym{[}  \vec{x}  \ottsym{]}  \mathcolor{\rcolor}{\equiv}  \widehat{\alpha}  \ottsym{[}  \vec{y}  \ottsym{]}  \ottsym{:}  \ottsym{0}  \mathcolor{\rcolor}{\Dashv}   \ottsym{(}  \Theta  \ottsym{\{}  \widehat{\beta}  \ottsym{:}  \ottsym{\mbox{$\mid$}}  \vec{z}  \ottsym{\mbox{$\mid$}}  \ottsym{,}  \widehat{\alpha}  \ottsym{:}  n  \ottsym{=}  \vec{x}  \ottsym{.}  \widehat{\beta}  \ottsym{[}  \vec{z}  \ottsym{]}  \ottsym{\}}  \ottsym{)} ^{\color{red}\star} }{%
{\ottdrulename{UV\_UV}}{}%
}}


\newcommand{\ottdruleUVXXUVTwo}[1]{\ottdrule[#1]{%
\ottpremise{\vec{z}  \ottsym{=}  \vec{x}  \cap  \vec{y}}%
}{
\Theta_{{\mathrm{0}}}  \ottsym{,}  \widehat{\alpha}  \ottsym{:}  n  \ottsym{,}  \Theta_{{\mathrm{1}}}  \ottsym{,}  \widehat{\beta}  \ottsym{:}  k  \ottsym{,}  \Theta_{{\mathrm{2}}}  \mathcolor{\rcolor}{\vDash}  \widehat{\alpha}  \ottsym{[}  \vec{x}  \ottsym{]}  \mathcolor{\rcolor}{\equiv}  \widehat{\beta}  \ottsym{[}  \vec{y}  \ottsym{]}  \ottsym{:}  \ottsym{0}  \mathcolor{\rcolor}{\Dashv}   \ottsym{(}  \Theta_{{\mathrm{0}}}  \ottsym{,}  \ottsym{(}  \widehat{\gamma}  \ottsym{:}  \ottsym{\mbox{$\mid$}}  \vec{z}  \ottsym{\mbox{$\mid$}}  \ottsym{)}  \ottsym{,}  \ottsym{(}  \widehat{\alpha}  \ottsym{:}  n  \ottsym{=}  \vec{x}  \ottsym{.}  \widehat{\gamma}  \ottsym{[}  \vec{z}  \ottsym{]}  \ottsym{)}  \ottsym{,}  \Theta_{{\mathrm{1}}}  \ottsym{,}  \ottsym{(}  \widehat{\beta}  \ottsym{:}  k  \ottsym{=}  \vec{y}  \ottsym{.}  \widehat{\gamma}  \ottsym{[}  \vec{z}  \ottsym{]}  \ottsym{)}  \ottsym{,}  \Theta_{{\mathrm{2}}}  \ottsym{)} ^{\color{red}\star} }{%
{\ottdrulename{UV\_UV2}}{}%
}}


\newcommand{\ottdruleUVXXF}[1]{\ottdrule[#1]{%
\ottpremise{\widehat{\alpha}  \notin  \ottmv{f}  \ottsym{(}  \ottnt{t_{{\mathrm{1}}}}  \ottsym{,..,}  \ottnt{t_{\ottmv{m}}}  \ottsym{)}}%
\ottpremise{\ottkw{arity} \, \ottmv{f}  \ottsym{=}  \ottsym{[}  k_{{\mathrm{1}}}  \ottsym{,..,}  k_{\ottmv{n}}  \ottsym{]}}%
\ottpremise{\Theta_{{\mathrm{0}}}  \ottsym{\{}  \widehat{\beta}_{{\mathrm{1}}}  \ottsym{:}  n  \ottsym{+}  k_{{\mathrm{1}}}  \ottsym{,}  \widehat{\alpha}  \ottsym{:}  n  \ottsym{\}}  \mathcolor{\rcolor}{\vDash}  \vec{y}_{{\mathrm{1}}}  \ottsym{.}  \widehat{\beta}_{{\mathrm{1}}}  \ottsym{[}  \vec{x}  \ottsym{,}  \vec{y}_{{\mathrm{1}}}  \ottsym{]}  \mathcolor{\rcolor}{\equiv}  \ottnt{t_{{\mathrm{1}}}}  \ottsym{:}  k_{{\mathrm{1}}}  \mathcolor{\rcolor}{\Dashv}  \Theta_{{\mathrm{1}}}  \ottsym{\{}  \widehat{\alpha}  \ottsym{:}  n  \ottsym{\}}}%
\ottpremise{\Theta_{{\mathrm{1}}}  \ottsym{\{}  \widehat{\beta}_{{\mathrm{2}}}  \ottsym{:}  n  \ottsym{+}  k_{{\mathrm{2}}}  \ottsym{,}  \widehat{\alpha}  \ottsym{:}  n  \ottsym{\}}  \mathcolor{\rcolor}{\vDash}  \vec{y}_{{\mathrm{2}}}  \ottsym{.}  \widehat{\beta}_{{\mathrm{2}}}  \ottsym{[}  \vec{x}  \ottsym{,}  \vec{y}_{{\mathrm{2}}}  \ottsym{]}  \mathcolor{\rcolor}{\equiv}  \ottsym{[}  \Theta_{{\mathrm{1}}}  \ottsym{]}  \ottnt{t_{{\mathrm{2}}}}  \ottsym{:}  k_{{\mathrm{2}}}  \mathcolor{\rcolor}{\Dashv}  \Theta_{{\mathrm{2}}}  \ottsym{\{}  \widehat{\alpha}  \ottsym{:}  n  \ottsym{\}}}%
\ottpremise{ \cdots }%
\ottpremise{\Theta_{{\ottmv{n}-1}}  \ottsym{\{}  \widehat{\beta}_{\ottmv{m}}  \ottsym{:}  n  \ottsym{+}  k_{{\mathrm{2}}}  \ottsym{,}  \widehat{\alpha}  \ottsym{:}  n  \ottsym{\}}  \mathcolor{\rcolor}{\vDash}  \vec{y}_{\ottmv{m}}  \ottsym{.}  \widehat{\beta}_{\ottmv{m}}  \ottsym{[}  \vec{x}  \ottsym{,}  \vec{y}_{\ottmv{m}}  \ottsym{]}  \mathcolor{\rcolor}{\equiv}  \ottsym{[}  \Theta_{{\ottmv{m}-1}}  \ottsym{]}  \ottnt{t_{\ottmv{m}}}  \ottsym{:}  k_{\ottmv{m}}  \mathcolor{\rcolor}{\Dashv}  \Theta_{\ottmv{m}}  \ottsym{\{}  \widehat{\alpha}  \ottsym{:}  n  \ottsym{\}}}%
}{
\Theta_{{\mathrm{0}}}  \ottsym{\{}  \widehat{\alpha}  \ottsym{:}  n  \ottsym{\}}  \mathcolor{\rcolor}{\vDash}  \widehat{\alpha}  \ottsym{[}  \vec{x}  \ottsym{]}  \mathcolor{\rcolor}{\equiv}  \ottmv{f}  \ottsym{(}  \ottnt{t_{{\mathrm{1}}}}  \ottsym{,..,}  \ottnt{t_{\ottmv{m}}}  \ottsym{)}  \ottsym{:}  \ottsym{0}  \mathcolor{\rcolor}{\Dashv}   \ottsym{(}  \Theta_{\ottmv{m}}  \ottsym{\{}  \widehat{\alpha}  \ottsym{:}  n  \ottsym{=}  \vec{x}  \ottsym{.}  \ottmv{f}  \ottsym{(}  \vec{y}_{{\mathrm{1}}}  \ottsym{.}  \widehat{\beta}_{{\mathrm{1}}}  \ottsym{[}  \vec{x}  \ottsym{,}  \vec{y}_{{\mathrm{1}}}  \ottsym{]}  \ottsym{,..,}  \vec{y}_{\ottmv{n}}  \ottsym{.}  \widehat{\beta}_{\ottmv{n}}  \ottsym{[}  \vec{x}  \ottsym{,}  \vec{y}_{\ottmv{n}}  \ottsym{]}  \ottsym{)}  \ottsym{\}}  \ottsym{)} ^{\color{red}\star}   \ottsym{\mbox{$\backslash{}$}}  \ottsym{(}  \widehat{\beta}_{{\mathrm{1}}}  \ottsym{,..,}  \widehat{\beta}_{\ottmv{m}}  \ottsym{)}}{%
{\ottdrulename{UV\_F}}{}%
}}

\newcommand{\ottdefnun}[1]{\begin{ottdefnblock}[#1]{$\Theta_{{\mathrm{1}}}  \mathcolor{\rcolor}{\vDash}  \ottnt{v}  \mathcolor{\rcolor}{\equiv}  \ottnt{w}  \ottsym{:}  n  \mathcolor{\rcolor}{\Dashv}  \Theta_{{\mathrm{2}}}$}{\ottcom{The unification}}
\ottusedrule{\ottdruleVXXV{}}
\ottusedrule{\ottdruleBXXB{}}
\ottusedrule{\ottdruleFXXF{}}
\ottusedrule{\ottdruleUVXXV{}}
\ottusedrule{\ottdruleUVXXUV{}}
\ottusedrule{\ottdruleUVXXUVTwo{}}
\ottusedrule{\ottdruleUVXXF{}}
\end{ottdefnblock}}


\newcommand{\ottdefnsAOne}{
\ottdefnun{}}

% defns B1
%% defn prun
\newcommand{\ottdruleaUV}[1]{\ottdrule[#1]{%
\ottpremise{\vec{z}  \ottsym{=}  \vec{y}  \cap  \vec{x}}%
}{
\Theta  \ottsym{\{}  \,  \ottsym{\}}  \ottsym{\{}  \widehat{\beta}  \ottsym{:}  n  \ottsym{\}}  \mathcolor{\ccolor}{\VDash}  \widehat{\beta}  \ottsym{[}  \vec{y}  \ottsym{]}  \cap  \ottsym{[}  \vec{x}  \ottsym{]}  \approxRight  \widehat{\beta}'  \ottsym{[}  \vec{z}  \ottsym{]}  \mathcolor{\ccolor}{\DashV}  \Theta  \ottsym{\{}  \widehat{\beta}'  \ottsym{:}  \ottsym{\mbox{$\mid$}}  \vec{z}  \ottsym{\mbox{$\mid$}}  \ottsym{\}}  \ottsym{\{}  \widehat{\beta}  \ottsym{:}  n  \ottsym{=}  \vec{y}  \ottsym{.}  \widehat{\beta}'  \ottsym{[}  \vec{z}  \ottsym{]}  \ottsym{\}}}{%
{\ottdrulename{aUV}}{}%
}}


\newcommand{\ottdruleaF}[1]{\ottdrule[#1]{%
}{
\Theta_{{\mathrm{0}}}  \ottsym{\{}  \,  \ottsym{\}}  \mathcolor{\ccolor}{\VDash}  \ottmv{f}  \ottsym{(}  \vec{y}_{{\mathrm{1}}}  \ottsym{.}  \ottnt{v_{{\mathrm{1}}}}  \ottsym{,..,}  \vec{y}_{\ottmv{n}}  \ottsym{.}  \ottnt{v_{\ottmv{n}}}  \ottsym{)}  \cap  \ottsym{[}  \vec{x}  \ottsym{]}  \approxRight  \widehat{\beta}'  \ottsym{[}  \vec{z}  \ottsym{]}  \mathcolor{\ccolor}{\DashV}  \Theta  \ottsym{\{}  \widehat{\beta}'  \ottsym{:}  \ottsym{\mbox{$\mid$}}  \vec{z}  \ottsym{\mbox{$\mid$}}  \ottsym{\}}  \ottsym{\{}  \widehat{\beta}  \ottsym{:}  n  \ottsym{=}  \vec{y}  \ottsym{.}  \widehat{\beta}'  \ottsym{[}  \vec{z}  \ottsym{]}  \ottsym{\}}}{%
{\ottdrulename{aF}}{}%
}}

\newcommand{\ottdefnaprun}[1]{\begin{ottdefnblock}[#1]{$\Theta_{{\mathrm{1}}}  \mathcolor{\ccolor}{\VDash}  \ottnt{v}  \cap  \ottsym{[}  \ottnt{vars}  \ottsym{]}  \approxRight  \ottnt{w}  \mathcolor{\ccolor}{\DashV}  \Theta_{{\mathrm{2}}}$}{\ottcom{The prunning phase}}
\ottusedrule{\ottdruleaUV{}}
\ottusedrule{\ottdruleaF{}}
\end{ottdefnblock}}

%% defn un2
\newcommand{\ottdruleaVXXV}[1]{\ottdrule[#1]{%
\ottpremise{\ottmv{x}  \in  \Theta}%
}{
\Theta  \mathcolor{\ccolor}{\VDash}  \ottmv{x}  \mathcolor{\rcolor}{\equiv}  \ottmv{x}  \mathcolor{\ccolor}{\DashV}  \Theta}{%
{\ottdrulename{aV\_V}}{}%
}}


\newcommand{\ottdruleaFXXF}[1]{\ottdrule[#1]{%
\ottpremise{\ottkw{arity} \, \ottmv{f}  \ottsym{=}  \ottsym{[}  k_{{\mathrm{1}}}  \ottsym{,..,}  k_{\ottmv{n}}  \ottsym{]}}%
\ottpremise{\Theta_{{\mathrm{0}}}  \ottsym{,}  \vec{x}_{{\mathrm{1}}}  \mathcolor{\ccolor}{\VDash}  \ottnt{v_{{\mathrm{1}}}}  \mathcolor{\rcolor}{\equiv}  \ottnt{w_{{\mathrm{1}}}}  \mathcolor{\ccolor}{\DashV}  \Theta_{{\mathrm{1}}}  \ottsym{,}  \vec{x}_{{\mathrm{1}}}}%
\ottpremise{\Theta_{{\mathrm{1}}}  \ottsym{,}  \vec{x}_{{\mathrm{2}}}  \mathcolor{\ccolor}{\VDash}  \ottsym{[}  \Theta_{{\mathrm{1}}}  \ottsym{]}  \ottnt{v_{{\mathrm{2}}}}  \mathcolor{\rcolor}{\equiv}  \ottsym{[}  \Theta_{{\mathrm{1}}}  \ottsym{]}  \ottnt{w_{{\mathrm{2}}}}  \mathcolor{\ccolor}{\DashV}  \Theta_{{\mathrm{2}}}  \ottsym{,}  \vec{x}_{{\mathrm{2}}}}%
\ottpremise{ \cdots }%
\ottpremise{\Theta_{{\ottmv{n}-1}}  \ottsym{,}  \vec{x}_{\ottmv{n}}  \mathcolor{\ccolor}{\VDash}  \ottsym{[}  \Theta_{{\ottmv{n}-1}}  \ottsym{]}  \ottnt{v_{\ottmv{n}}}  \mathcolor{\rcolor}{\equiv}  \ottsym{[}  \Theta_{{\ottmv{n}-1}}  \ottsym{]}  \ottnt{w_{\ottmv{n}}}  \mathcolor{\ccolor}{\DashV}  \Theta_{\ottmv{n}}  \ottsym{,}  \vec{x}_{\ottmv{n}}}%
}{
\Theta_{{\mathrm{0}}}  \mathcolor{\ccolor}{\VDash}  \ottmv{f}  \ottsym{(}  \vec{x}_{{\mathrm{1}}}  \ottsym{.}  \ottnt{v_{{\mathrm{1}}}}  \ottsym{,..,}  \ottmv{x_{\ottmv{n}}}  \ottsym{.}  \ottnt{v_{\ottmv{n}}}  \ottsym{)}  \mathcolor{\rcolor}{\equiv}  \ottmv{f}  \ottsym{(}  \vec{x}_{{\mathrm{1}}}  \ottsym{.}  \ottnt{w_{{\mathrm{1}}}}  \ottsym{,..,}  \vec{x}_{\ottmv{n}}  \ottsym{.}  \ottnt{w_{\ottmv{n}}}  \ottsym{)}  \mathcolor{\ccolor}{\DashV}  \Theta_{\ottmv{n}}}{%
{\ottdrulename{aF\_F}}{}%
}}


\newcommand{\ottdruleaUVXXUV}[1]{\ottdrule[#1]{%
\ottpremise{\vec{z}  \ottsym{=}  \vec{x}  \sqcap  \vec{y}}%
}{
\Theta  \ottsym{\{}  \widehat{\alpha}  \ottsym{:}  n  \ottsym{\}}  \mathcolor{\ccolor}{\VDash}  \widehat{\alpha}  \ottsym{[}  \vec{x}  \ottsym{]}  \mathcolor{\rcolor}{\equiv}  \widehat{\alpha}  \ottsym{[}  \vec{y}  \ottsym{]}  \mathcolor{\ccolor}{\DashV}   \ottsym{(}  \Theta  \ottsym{\{}  \widehat{\beta}  \ottsym{:}  \ottsym{\mbox{$\mid$}}  \vec{z}  \ottsym{\mbox{$\mid$}}  \ottsym{,}  \widehat{\alpha}  \ottsym{:}  n  \ottsym{=}  \vec{x}  \ottsym{.}  \widehat{\beta}  \ottsym{[}  \vec{z}  \ottsym{]}  \ottsym{\}}  \ottsym{)} ^{\color{red}\star} }{%
{\ottdrulename{aUV\_UV}}{}%
}}


\newcommand{\ottdruleaUVXXF}[1]{\ottdrule[#1]{%
\ottpremise{\widehat{\alpha}  \notin  \ottnt{t}}%
\ottpremise{\Theta  \ottsym{\{}  \,  \ottsym{\}}  \mathcolor{\ccolor}{\VDash}  \ottnt{t}  \cap  \ottsym{[}  \vec{x}  \ottsym{]}  \approxRight  \ottnt{t'}  \mathcolor{\ccolor}{\DashV}  \Theta'  \ottsym{\{}  \widehat{\alpha}  \ottsym{:}  n  \ottsym{\}}}%
}{
\Theta  \mathcolor{\ccolor}{\VDash}  \widehat{\alpha}  \ottsym{[}  \vec{x}  \ottsym{]}  \mathcolor{\rcolor}{\equiv}  \ottnt{t}  \mathcolor{\ccolor}{\DashV}   \ottsym{(}  \Theta'  \ottsym{\{}  \widehat{\alpha}  \ottsym{:}  n  \ottsym{=}  \vec{x}  \ottsym{.}  \ottnt{t'}  \ottsym{\}}  \ottsym{)} ^{\color{red}\star} }{%
{\ottdrulename{aUV\_F}}{}%
}}

\newcommand{\ottdefnaunTwo}[1]{\begin{ottdefnblock}[#1]{$\Theta_{{\mathrm{1}}}  \mathcolor{\ccolor}{\VDash}  \ottnt{v}  \mathcolor{\rcolor}{\equiv}  \ottnt{w}  \mathcolor{\ccolor}{\DashV}  \Theta_{{\mathrm{2}}}$}{\ottcom{The alternative unification}}
\ottusedrule{\ottdruleaVXXV{}}
\ottusedrule{\ottdruleaFXXF{}}
\ottusedrule{\ottdruleaUVXXUV{}}
\ottusedrule{\ottdruleaUVXXF{}}
\end{ottdefnblock}}

%% defn ext
\newcommand{\ottdruleaV}[1]{\ottdrule[#1]{%
}{
\ottmv{x} \, \ottkw{ext}}{%
{\ottdrulename{aV}}{}%
}}


\newcommand{\ottdruleaUV}[1]{\ottdrule[#1]{%
}{
\widehat{\alpha}  \ottsym{[}  \vec{x}  \ottsym{]} \, \ottkw{ext}}{%
{\ottdrulename{aUV}}{}%
}}


\newcommand{\ottdruleaBind}[1]{\ottdrule[#1]{%
\ottpremise{\ottnt{t} \, \ottkw{ext}}%
}{
\ottmv{x}  \ottsym{.}  \ottnt{t} \, \ottkw{ext}}{%
{\ottdrulename{aBind}}{}%
}}


\newcommand{\ottdruleaConstr}[1]{\ottdrule[#1]{%
\ottpremise{\vec{x}_{{\mathrm{1}}}  \cap  \ottkw{UVARGS} \, \ottnt{t_{{\mathrm{1}}}}  \ottsym{=}  \emptyset \quad \ottnt{t_{{\mathrm{1}}}} \, \ottkw{ext}}%
\ottpremise{ \cdots }%
\ottpremise{\vec{x}_{\ottmv{n}}  \cap  \ottkw{UVARGS} \, \ottnt{t_{\ottmv{n}}}  \ottsym{=}  \emptyset \quad \ottnt{t_{\ottmv{n}}} \, \ottkw{ext}}%
}{
\ottmv{f}  \ottsym{(}  \vec{x}_{{\mathrm{1}}}  \ottsym{.}  \ottnt{t_{{\mathrm{1}}}}  \ottsym{,..,}  \vec{x}_{\ottmv{n}}  \ottsym{.}  \ottnt{t_{\ottmv{n}}}  \ottsym{)} \, \ottkw{ext}}{%
{\ottdrulename{aConstr}}{}%
}}

\newcommand{\ottdefnaext}[1]{\begin{ottdefnblock}[#1]{$\ottnt{v} \, \ottkw{ext}$}{\ottcom{The external term}}
\ottusedrule{\ottdruleaV{}}
\ottusedrule{\ottdruleaUV{}}
\ottusedrule{\ottdruleaBind{}}
\ottusedrule{\ottdruleaConstr{}}
\end{ottdefnblock}}

%% defn extC
\newcommand{\ottdruleaEmpty}[1]{\ottdrule[#1]{%
}{
\cdot \, \ottkw{ext}}{%
{\ottdrulename{aEmpty}}{}%
}}


\newcommand{\ottdruleaVar}[1]{\ottdrule[#1]{%
}{
\Theta  \ottsym{,}  \ottmv{x} \, \ottkw{ext}}{%
{\ottdrulename{aVar}}{}%
}}


\newcommand{\ottdruleaUVar}[1]{\ottdrule[#1]{%
}{
\Theta  \ottsym{,}  \widehat{\alpha}  \ottsym{:}  n \, \ottkw{ext}}{%
{\ottdrulename{aUVar}}{}%
}}


\newcommand{\ottdruleaUVarInst}[1]{\ottdrule[#1]{%
\ottpremise{\ottnt{t} \, \ottkw{ext}}%
}{
\Theta  \ottsym{,}  \widehat{\alpha}  \ottsym{:}  n  \ottsym{=}  \ottnt{t} \, \ottkw{ext}}{%
{\ottdrulename{aUVarInst}}{}%
}}

\newcommand{\ottdefnaextC}[1]{\begin{ottdefnblock}[#1]{$\Theta \, \ottkw{ext}$}{\ottcom{The external environment}}
\ottusedrule{\ottdruleaEmpty{}}
\ottusedrule{\ottdruleaVar{}}
\ottusedrule{\ottdruleaUVar{}}
\ottusedrule{\ottdruleaUVarInst{}}
\end{ottdefnblock}}


\newcommand{\ottdefnsBOne}{
\ottdefnaprun{}\ottdefnaunTwo{}\ottdefnaext{}\ottdefnaextC{}}

\newcommand{\ottdefnss}{
\ottfundefnsFoo
\ottdefnsAOne
\ottdefnsBOne
}

\newcommand{\ottall}{\ottmetavars\\[0pt]
\ottgrammar\\[5.0mm]
\ottdefnss}






% \renewcommand{\ottdruleEOneNVarName}[0]{ Hello \def\@currentlabelname{FOO} \phantomsection  }
% \newcommand{\ottdrulename}[1]{\textsc{#1}}

% \renewcommand{\ottdrule}[4][]{{\displaystyle\frac{\begin{array}{l}#2\end{array}}{#3}\quad\ottdrulename{#4}%
%   }\ruleLabel{#4}{#4}}

% \renewcommand{\ottdrule}[4][]{%
%   {\displaystyle\frac{\begin{array}{l}#2\end{array}}{#3}\quad\ottdrulename{#4}}%
%   \mpr@label{\textsc{#1}}%
% }
% \DeclareDocumentCommand \redrule {m m o}{%
%   \inferrule*[vcenter,left={#3:}]{}{#1 \tred #2}
%   \mpr@label{\textsc{#3}}
% }

% \renewcommand{\ottdrulename}[2][#1]{\textsc{#1} \label{}}


% \renewcommand{\ottdruleEOneNVarName}[0]{foo }

% \makeatletter
% \renewcommand{\ottdruleEOneNVar}[1]{\ottdrule[#1]{%
%   }{
%     % \protected@edef\@currentlabelname{foo}
%     \def\@currentlabelname{foo}%
%     \phantomsection
%     \label{boo}
%     \alpha ^{-}   \eqEOne   \alpha ^{-}
%   }{%
%     {\ottdruleEOneNVarName}{}%
%   } }
% \makeatother


\newcommand{\depth}[1]{\mathsf{depth}(#1)}
\newcommand{\size}[1]{\mathsf{size}(#1)}

\newcommand{\ruleref}[1]{Rule \nameref{#1}}

% ord varset in uN = varset'

\renewcommand{\ottdruleONVarInName}[0]{(Var$_{\in}^-$)}
\renewcommand{\ottdruleONVarNInName}[0]{(Var$_{\notin}^-$)}
\renewcommand{\ottdruleONUVarName}[0]{(UVar$^-$)}
\renewcommand{\ottdruleOShiftUName}[0]{($\uparrow$)}
\renewcommand{\ottdruleOArrowName}[0]{($\rightarrow$)}
\renewcommand{\ottdruleOForallName}[0]{($\forall$)}


% ord varset in uP = varset'

\renewcommand{\ottdruleOPVarInName}[0]{(Var$_{\in}^+$)}
\renewcommand{\ottdruleOPVarNInName}[0]{(Var$_{\notin}^+$)}
\renewcommand{\ottdruleOPUVarName}[0]{(UVar$^+$)}
\renewcommand{\ottdruleOShiftDName}[0]{($\downarrow$)}
\renewcommand{\ottdruleOExistsName}[0]{($\exists$)}


% nf(N) = M
\renewcommand{\ottdruleNrmNVarName}[0]{(Var$^-$)}
\renewcommand{\ottdruleNrmNUVarName}[0]{(UVar$^-$)}
\renewcommand{\ottdruleNrmShiftUName}[0]{($\uparrow$)}
\renewcommand{\ottdruleNrmArrowName}[0]{($\rightarrow$)}
\renewcommand{\ottdruleNrmForallName}[0]{($\forall$)}

% nf(P) = Q
\renewcommand{\ottdruleNrmPVarName}[0]{(Var$^+$)}
\renewcommand{\ottdruleNrmPUVarName}[0]{(UVar$^+$)}
\renewcommand{\ottdruleNrmShiftDName}[0]{($\downarrow$)}
\renewcommand{\ottdruleNrmExistsName}[0]{($\exists$)}


% N ≈ M

\renewcommand{\ottdruleEOneNVarName}[0]{(Var$^-$$^{\eqEOne}$)}
\renewcommand{\ottdruleEOneShiftUName}[0]{($\uparrow^{\eqEOne}$)}
\renewcommand{\ottdruleEOneArrowName}[0]{($\rightarrow^{\eqEOne}$)}
\renewcommand{\ottdruleEOneForallName}[0]{($\forall^{\eqEOne}$)}

% P ≈ Q
\renewcommand{\ottdruleEOnePVarName}[0]{(Var$^+$$^{\eqEOne}$)}
\renewcommand{\ottdruleEOneShiftDName}[0]{($\downarrow^{\eqEOne}$)}
\renewcommand{\ottdruleEOneExistsName}[0]{($\exists^{\eqEOne}$)}


% G ⊢ N ≤1 M

\renewcommand{\ottdruleDOneNVarName}[0]{(Var$^-$$^{\subDOne}$)}
\renewcommand{\ottdruleDOneShiftUName}[0]{($\uparrow^{\subDOne}$)}
\renewcommand{\ottdruleDOneArrowName}[0]{($\rightarrow^{\subDOne}$)}
\renewcommand{\ottdruleDOneForallName}[0]{($\forall^{\subDOne}$)}

% G ⊢ P ≥1 Q
\renewcommand{\ottdruleDOnePVarName}[0]{(Var$^+$$^{\supDOne}$)}
\renewcommand{\ottdruleDOneShiftDName}[0]{($\downarrow^{\supDOne}$)}
\renewcommand{\ottdruleDOneExistsName}[0]{($\exists^{\supDOne}$)}


% G ⊢ N ≈1 M
\renewcommand{\ottdruleDOneNDefName}[0]{($\eqDOne^{-}$)}

% G ⊢ P ≈1 Q
\renewcommand{\ottdruleDOnePDefName}[0]{($\eqDOne^{+}$)}



% G ⊨ iP1 ∨ iP2 = iQ
\renewcommand{\ottdruleLUBVarName}[0]{(Var$^{\vee}$)}
\renewcommand{\ottdruleLUBShiftName}[0]{($\downarrow^{\vee}$)}
\renewcommand{\ottdruleLUBExistsName}[0]{($\exists^{\vee}$)}
\renewcommand{\ottdruleLUBUpgradeName}[0]{(Upg)}


% G ; Θ ⊨ uN ≤ iM ⫤ us
\renewcommand{\ottdruleANVarName}[0]{(Var$^-$$^{\subA}$)}
\renewcommand{\ottdruleAShiftUName}[0]{($\uparrow^{\subA}$)}
\renewcommand{\ottdruleAArrowName}[0]{($\rightarrow^{\subA}$)}
\renewcommand{\ottdruleAForallName}[0]{($\forall^{\subA}$)}

% G ; Θ ⊨ iP ≥ uQ ⫤ us
\renewcommand{\ottdruleAPVarName}[0]{(Var$^+$$^{\supA}$)}
\renewcommand{\ottdruleAShiftDName}[0]{($\downarrow^{\supA}$)}
\renewcommand{\ottdruleAExistsName}[0]{($\exists^{\supA}$)}
\renewcommand{\ottdruleAPUVarName}[0]{(UVar$^{\supA}$)}


% Γ ⊢ scE1 & scE2 = scE3 :: :: E :: 'E'
\renewcommand{\ottdruleSCMESupSupName}[0]{$([[≥]]\&^{+}[[≥]])$}
\renewcommand{\ottdruleSCMEEqSupName}[0]{$([[≈]]\&^{+}[[≥]])$}
\renewcommand{\ottdruleSCMESupEqName}[0]{$([[≥]]\&^{+}[[≈]])$}
\renewcommand{\ottdruleSCMEPEqEqName}[0]{$([[≈]]\&^{+}[[≈]])$}
\renewcommand{\ottdruleSCMENEqEqName}[0]{$([[≈]]\&^{-}[[≈]])$}

% Γ ; Θ ⊨ uN ≈u iM ⫤ UC 
\renewcommand{\ottdruleUNVarName}[0]{(Var$^{-[[≈u]]}$)}
\renewcommand{\ottdruleUShiftUName}[0]{($\uparrow^{[[≈u]]}$)}
\renewcommand{\ottdruleUArrowName}[0]{($\rightarrow^{[[≈u]]}$)}
\renewcommand{\ottdruleUForallName}[0]{($\forall^{[[≈u]]}$)}
\renewcommand{\ottdruleUNUVarName}[0]{(UVar$^{-[[≈u]]}$)}

% Γ ; Θ ⊨ uP ≈u iQ ⫤ UC 
\renewcommand{\ottdruleUPVarName}[0]{(Var$^{+[[≈u]]}$)}
\renewcommand{\ottdruleUShiftDName}[0]{($\downarrow^{[[≈u]]}$)}
\renewcommand{\ottdruleUExistsName}[0]{($\exists^{[[≈u]]}$)}
\renewcommand{\ottdruleUPUVarName}[0]{(UVar$^{+[[≈u]]}$)}

% G ⊨ iP1 ≈au iP2 ⫤ ( Ξ , uQ , aus1 , aus2 )
\renewcommand{\ottdruleAUPVarName}[0]{(Var$^{+[[≈au]]}$)}
\renewcommand{\ottdruleAUShiftDName}[0]{($\downarrow^{[[≈au]]}$)}
\renewcommand{\ottdruleAUExistsName}[0]{($\exists^{[[≈au]]}$)}

% G ⊨ iN1 ≈au iN2 ⫤ ( Ξ , uM , aus1 , aus2 )
\renewcommand{\ottdruleAUNVarName}[0]{(Var$^{-[[≈au]]}$)}
\renewcommand{\ottdruleAUShiftUName}[0]{($\uparrow^{[[≈au]]}$)}
\renewcommand{\ottdruleAUForallName}[0]{($\forall^{[[≈au]]}$)}
\renewcommand{\ottdruleAUArrowName}[0]{($\rightarrow^{[[≈au]]}$)}
\renewcommand{\ottdruleAUAUName}[0]{(AU$^{-}$)}


\begin{document}

\tableofcontents

\newpage

\section{Declarative System}
First, we present the top-level system, which is easy to understand.

\subsection{Grammar}
We assume that there is an infinite set of positive and 
negative \emph{type} variables. Positive type variables are denoted as 
$[[α⁺]]$, $[[β⁺]]$, $[[γ⁺]]$, etc.
Negative type variables are denoted as $[[α⁻]]$, $[[β⁻]]$, $[[γ⁻]]$, etc.
We assume there is an infinite set of \emph{term} variables,
which are denoted as $[[x]]$, $[[y]]$, $[[z]]$, etc.
A list of objects (variables, types or terms) is denoted by
an overline arrow. For instance, $[[pas]]$ is a list of positive type variables, 
$[[nbs]]$ is a list of negative type variables, 
$[[args]]$ is a list of values, which are arguments of a function.
$[[fv(iP)]]$ and $[[fv(iN)]]$ denote the set of free variables 
in a type $[[iP]]$ and $[[iN]]$, respectively.


\bigskip

\ottgrammartabular{
  \ottiP\ottinterrule
  \ottiN\ottinterrule
  \ottv\ottinterrule
  \ottc\ottinterrule
}


\subsection{Equalities}
For simplicity, assume alpha-equivalent terms equal. 
This way, we assume that substitutions do not capture bound variables.
Besides, we equate
$[[∀pas.∀pbs.iN]]$ with $[[∀pas,pbs.iN]]$, 
as well as $[[∃nas.∃nbs.iP]]$ with $[[∃nas,nbs.iP]]$,
and lift these equations transitively and congruently 
to the whole system.

\subsection{Contexts}

\begin{definition}[Declarative Type Context]
  \hfill \\
  Declarative type context $[[Γ]]$ is represented by a set of 
  type variables. The concatenation $[[Γ1, Γ2]]$ means the 
  union of two contexts  $[[{Γ1} ∪ {Γ2}]]$.
\end{definition}


\subsection{Substitutions}

\begin{definition}[Substitution]
  Substitutions (denoted as $[[σ]]$) 
  are represented by total functions form variables to types, preserving the polarity. 
\end{definition}

\begin{definition}[Substitution Application]
  Substitution application (denoted as $[[ [σ]iP ]]$ or $[[ [σ]iN ]]$) 
  is defined congruently as follows:
  \begin{itemize}
    \item $[[ [σ]α⁺ ]] = [[σ]] ([[α⁺]])$;
    \item $[[ [σ]α⁻ ]] = [[σ]] ([[α⁻]])$;
    \item $[[ [σ]↓iN ]] = [[↓[σ]iN]]$;
    \item $[[ [σ]↑iP ]] = [[↑[σ]iP]]$;
    \item $[[ [σ]∃nas.iQ ]] = [[∃nas.[σ]iQ]]$, 
    \item $[[ [σ]∀pas.iN ]] = [[∀pas.[σ]iN]]$(here we assume that $[[nas]]$ and $[[pas]]$ are lists of fresh variables, 
      that is the variable capture never happens);
    \item $[[ [σ](iP → iN) ]] = [[ [σ]iP → [σ]iN ]]$.
  \end{itemize}
\end{definition}

\begin{definition}[Substitution Signature]
  The signature $[[Γ' ⊢ σ : Γ]]$ means that
  \begin{enumerate}
    \item for any $[[α± ∊ {Γ}]], [[ Γ' ⊢ [σ]α± ]]$; and
    \item for any $[[α± ∉ {Γ'}]], [[ [σ]α± = α± ]]$.
  \end{enumerate}
\end{definition}

A substitution can be restricted to a set of variables. 
The restricted substitution is define as expected. 
\begin{definition}[Subsitution Restriction]
  The specification $[[σ  | varset]]$ is defined as
  a function such that 
  \begin{enumerate}
    \item $[[σ|varset]]([[α± ]]) = [[σ]]([[α± ]])$, if $[[α± ]] \in [[varset]]$; and
    \item $[[σ|varset]]([[α± ]]) = [[α± ]]$, if $[[α± ]] \notin [[varset]]$.
  \end{enumerate}
\end{definition}

Two substitutions can be composed in two ways:
$[[σ2 ○ σ1]]$ corresponds to a consecutive application of $[[σ1]]$ and $[[σ2]]$,
while $[[σ2 <=< σ1]]$
depends on a signature of $[[σ1]]$ and modifies $[[σ1]]$ by applying
$[[σ2]]$ to its results on the domain.
\begin{definition}[Substitution Composition]
  $[[σ2 ○ σ1]]$ is defined as a function such that
  $[[σ2 ○ σ1]]([[α± ]]) = [[σ2]]([[σ1]]([[α± ]]))$.
\end{definition}

\begin{definition}[Monadic Substitution Composition]
  Suppose that $[[Γ' ⊢ σ1 : Γ]]$.
  Then we define $[[σ2 <=< σ1]]$ as $[[(σ2 ○ σ1)|{Γ}]]$.
\end{definition}
Notice that the result of $[[σ2 <=< σ1]]$ depends on the 
specification of $[[σ1]]$, which is not unique. 
However, we assume that the used specification clear from the 
context of the proof. 

\begin{definition}[Equivalent Substitutions]
  The substitution equivalence judgement $[[Γ' ⊢ σ1 ≈ σ2 : Γ]]$ 
  indicates that on the domain $[[Γ]]$, 
  the result of $[[σ1]]$ and $[[σ2]]$ are equivalent in context $[[Γ']]$.
  Formally, for any $[[α± ∊ {Γ}]], [[ Γ' ⊢ [σ1]α± ≈ [σ2]α± ]]$.
\end{definition}

Sometimes it is convenient to construct substitution 
explicitly mapping each variable from a list (or a set)
to a type. Such substitutions are denoted as $[[iPs / pas]]$
and $[[iNs / nas]]$, where $[[iPs]]$ and $[[iNs]]$ are lists of 
the corresponding types.
\begin{definition}[Explicit Substitution]
  \hfill
  \begin{itemize}
    \item [$-$]
      Suppose that $[[nas]]$ is a list of negative type variables,
      and $[[iNs]]$ is a list of negative types of the same length 
      Then $[[iNs / nas]]$ denotes a substitution such that 
      \begin{enumerate}
        \item for $[[αi⁺ ∊ {nas}]]$, $[[ [iNs / nas] αi⁺]] = [[iNi]]$;
        \item for $[[β⁺ ∉ {nas}]]$, $[[ [iNs / nas] β⁺]] = [[β⁺]]$.
      \end{enumerate}
    \item [$+$]
      Positive explicit substitution $[[iPs / pas]]$
      is defined symmetrically.
  \end{itemize}
\end{definition}


\subsection{Declarative Typing}
\ottdefnsDT

\subsection{Declarative Subtyping}
\ottdefnsDOne

\subsection{Declarative Equivalence}
\ottdefnsEOne

\subsection{Well-Formedness}
\ottdefnsWFT
\ottdefnsWFAT

\section{Algorithm}

\subsection{Grammar}

In the algorithmic system, we extend the grammar of types
by adding positive and negative \emph{algorithmic variables}
($[[α̂⁺]]$, $[[β̂⁺]]$, $[[γ̂⁺]]$, etc. and $[[α̂⁻]]$, $[[β̂⁻]]$, $[[γ̂⁻]]$, etc.).
They represent the unknown types, which will be inferred by the algorithm.
This way, we add two base cases to the grammar of 
positive and negative types, and use highlight to denote that the type
can potentially contain algorithmic variables.

 
\begin{definition}[Algorithmic Types]
  \hfill \\
\ottgrammartabular{
  \ottuP\ottinterrule
  \ottuN\ottinterrule
}
\end{definition}

\subsection{Algorithmic Renaming}

In several places of our algorithm, 
we turn a declarative type into the algorithmic one
via replacing certain type variables with fresh algorithmic variables.
This is done via algorithmic renaming

\begin{definition}
  Suppose that $[[nas]]$ is a list of negative type variables, 


\subsection{Contexts}

\begin{definition}[Algorithmic Type Context]
  \hfill \\
  Algorithmic type context $[[Ξ]]$ is represented by a set of 
  \emph{algorithmic} type variables ($[[α̂⁺]]$, $[[α̂⁻]]$, $[[β̂⁺]]$, \dots).
\end{definition}

$[[Γ ; Ξ ⊢ uP]]$ and $[[Γ ; Ξ ⊢ uN]]$ are used to denote
that the algorithmic type is well-formed in the contexts
$[[Γ]]$ and $[[Ξ]]$, which means that each algorithmic variable
of the type is contained in $[[Ξ]]$, and each free declarative type variable
of the type is contained in $[[Γ]]$.

Algorithmic Type Context are used in the unification algorithm.
In the subtyping algorithm, 
the context needs to remember additional information.
In the subtyping context, each algorithmic variable is associated with a
context it must be instantiated in 
(i.e. the context in which the type replacing the variable must be well-formed).
This association is represented by \emph{algorithmic subtyping context} $[[Θ]]$.
\begin{definition}[Algorithmic Subtyping Context]
  \hfill \\
  Algorithmic Subtyping Context $[[Θ]]$ is represented by a set of 
  entries of form $[[ α̂⁺[Γ] ]]$ and $[[ α̂⁻[Γ] ]]$,
  where $[[α̂⁺]]$ and $[[α̂⁻]]$ are algorithmic variables,
  and $[[Γ]]$ is a context in which they must be instantiated.
  We assume that no two entries associating the same variable
  appear in $[[Θ]]$.

  $[[dom(Θ)]]$ denotes the set of variables appearing in $[[Θ]]$:
  $[[dom(Θ)]] = \{ [[α̂±]] \mid [[α̂±[Γ] ]] \in [[Θ]] \}$.
\end{definition}

\subsection{Subsitutions}

Substitution that operates on algorithmic type variables is denoted as
$[[uσ]]$. It is defined as a total function from algorithmic 
type variables to non-algorithmic types, preserving the polarity.

The signature $[[Θ ⊢ uσ]]$ means that $[[uσ]]$ maps each algorithmic 
variable appearing in $[[Θ]]$ to a type well-formed in the corresponding
context; and for each variable not appearing in $[[dom(Θ)]]$, acts as identity.

\begin{definition}[Signature of Algorithmic Substitution]
  $[[Θ ⊢ uσ]]$ means that
  \begin{enumerate}
    \item for any $[[ α̂±[Γ] ∊ Θ]]$, $[[ Γ ⊢ [uσ]α̂± ]]$ and
    \item for any $[[ α̂± ∉ dom(Θ)]]$, $[[ [uσ]α̂± ]] =  [[ α̂± ]]$.
  \end{enumerate}
\end{definition}

Anti-unification substitution is denoted as $[[aus]]$ and $[[ausr]]$.
In contrast to algorithmic substitution $[[uσ]]$,
it only defined on the negative algorithmic variables, 
and it allows mapping algorithmic variables to
\emph{algorithmic} terms.

The pair of contexts $[[Γ]]$ and $[[Ξ]]$,
in which the results of the anti-unification substitutions 
are formed, is fixed for the whole substitution.
This way, $[[Γ; Ξ2 ⊢ aus : Ξ1]]$ means that $[[aus]]$ maps each negative algorithmic
variable appearing in $[[Ξ1]]$ to a term well-formed in $[[Γ]]$ and $[[Ξ2]]$.

\begin{definition}[Signature of Anti-unification substitution]
  $[[Γ; Ξ2 ⊢ aus : Ξ1]]$ means that
  \begin{enumerate}
    \item for any $[[ α̂⁻ ∊ Ξ1]]$, $[[ Γ; Ξ2 ⊢ [aus]α̂⁻ ]]$ and
    \item for any $[[ α̂⁻ ∉ Ξ1]]$, $[[ [aus]α̂⁻ = α̂⁻ ]]$.
  \end{enumerate}
\end{definition}

\subsection{Normalization}

\subsubsection{Ordering}
\ottdefnsOrder

\subsubsection{Quantifier Normalization}
\ottdefnsNrm

We also define normalization of a substitution pointwise:
\begin{definition}[Substitution Normalization]
  For a substitution $[[σ]]$, we define $[[nf(σ)]]$
  as a substitution that maps $[[α±]]$ into $[[nf([σ]α±)]]$.
\end{definition}

\subsection{Singularity}
\ottdefnsSING

% \subsection{Algorithmic Equivalence}
% \ottdefnsEOneA

\subsection{Unification}
\ottdefnsU

\subsection{Algorithmic Subtyping}
\ottdefnsA


\subsection{Constraints}

Unification and subtyping algorithms are based on the constraint generation.
The constraints are represented by set of constraint entries.

\begin{definition}[Unification Constraint]
  \hfill
  \begin{itemize}
    \item Unification entry (denoted as $[[ucE]]$) is an expression of shape 
      $[[pua :≈ iP]]$ or $[[nua :≈ iN]]$;
    \item unification constraint (denoted as $[[UC]]$) is a set of 
      unification constraint entries.
  \end{itemize}

\end{definition}

\begin{definition}[Subtyping Constraint]
  \hfill
  \begin{itemize}
    \item Subtyping entry (denoted as $[[scE]]$) is an expression of shape 
      $[[pua :≥ iP]]$, $[[nua :≈ iN]]$, or $[[pua :≈ iP]]$;
    \item subtyping constraint (denoted as $[[SC]]$) is a set of subtyping constraint entries.
  \end{itemize}
\end{definition}

\begin{definition}[Well-formed Constraint Entry]
  We say that a constraint entry is well-formed in a context $[[Γ]]$ if
  the type it restricts the unification variable to is well-formed in $[[Γ]]$.
  \begin{itemize}
    \item $[[Γ ⊢ pua :≥ iP]]$ iff $[[Γ ⊢ iP]]$;
    \item $[[Γ ⊢ pua :≈ iP]]$ iff $[[Γ ⊢ iP]]$;
    \item $[[Γ ⊢ nua :≈ iN]]$ iff $[[Γ ⊢ iN]]$.
  \end{itemize}
\end{definition}

\begin{definition} [Matching Entries]
  We call two unification constraint entries 
  or two subtyping constraint entries matching 
  if they are restricting the same unification variable.
\end{definition}

Two matching entries formed in the same context $[[Γ]]$ 
can be merged in the following way:
\begin{definition}[Merge of Matching Constraint Entries]
   \hfill \\
\ottdefnSCME\\
\end{definition}
% Notice that in case of equivalence, the assigned types
% must be equal (i.e. alpha-equivalent) to be merged. This is because
% the unification algorithm assumes that every type is normalized,
% and hence, equivalence is alpha-equivalence 
% (\cref{corollary:nf-complete-wrt-subt-equiv,corollary:nf-sound-wrt-subt-equiv}).

\begin{definition}[Constraint Context]
  Constraint context (denoted as $[[Θ]]$) is a set of entries of shape $[[ pua[Γ] ]]$ 
  and $[[ nua[Γ] ]]$, specifying that the unification variable ($[[pua]]$ or $[[nua]]$)
  must be instantiated in a type well-formed in context $[[Γ]]$.
  We assume that for each $[[α̂±]]$, at most one entry of shape $[[ α̂±[Γ] ]]$ is in $[[Θ]]$.
  If $[[ α̂±[Γ] ]] \in [[Θ]]$, we denote $[[Γ]]$ as $[[Θ(α̂±)]]$.
\end{definition}

\begin{definition}[Well-formed Constraint]
  We say that a constraint is well-formed in a
  constraint context $[[Θ]]$ if all its entries are well-formed in
  the corresponding elements of $[[Θ]]$.
  More formally, 
  $[[Θ ⊢ UC]]$ iff for every $[[ucE]] \in $ $[[UC]]$,
  such that $[[ucE]]$ restricts $[[α̂±]]$,
  there exists $[[ α̂±[Γ] ]] \in [[Θ]]$
  and $[[Γ ⊢ ucE]]$.
\end{definition}

\begin{definition}[Merge of Subtyping Constraints]
  Suppose that $[[Θ ⊢ SC1]]$ and $[[Θ ⊢ SC2]]$.
  Then $[[Θ ⊢ SC1 & SC2 = SC]]$
  defines a set such that $[[ucE]] \in [[SC]]$ iff either
  \begin{itemize}
    \item $[[ucE]] \in [[SC1]]$ and there is no matching $[[ucE']] \in [[SC2]]$; or
    \item $[[ucE]] \in [[SC2]]$ and there is no matching $[[ucE']] \in [[SC1]]$; or
    \item $[[Θ(α̂±) ⊢ ucE1 & ucE2 = ucE]]$ for some $[[ucE1]] \in [[SC1]]$ and $[[ucE2]] \in [[SC2]]$
      such that $[[ucE1]]$ matches with $[[ucE2]]$ restricting variable
      $[[α̂±]]$. 
  \end{itemize}
\end{definition}

\subsection{Constraint Satisfaction}
\ottdefnsSATSCE

\subsection{Least Upper Bound}
\ottdefnsLUB

\subsection{Antiunification}
\ottdefnsAU

\subsection{Typing}
\ottdefnsAT

\section{Proofs}

\subsection{Type well-formedness}
\begin{lemma}[Equivalent Contexts]
  \label{lemma:wf-ctxt-equiv}

  In the well-formedness judgment, only used variables matter:
  \begin{itemize}
  \item[$+$] if $[[{Γ1} ∩ fv iP]] = [[{Γ2} ∩ fv iP]]$ then
    $[[Γ1 ⊢ iP]] \iff [[Γ2 ⊢ iP]]$,
  \item[$-$] if $[[{Γ1} ∩ fv iN]] = [[{Γ2} ∩ fv iN]]$ then
    $[[Γ1 ⊢ iN]] \iff [[Γ2 ⊢ iN]]$.
  \end{itemize}
\end{lemma}
\begin{proof}
  By simple mutual induction on $[[iP]]$ and $[[iQ]]$. 
\end{proof}

\begin{lemma}[Well-formedness Context Weakening]
  \label{lemma:wf-weakening}
  Suppose that $[[{Γ1} ⊆ {Γ2}]]$, then
  \begin{itemize}
    \item[$+$] if $[[Γ1 ⊢ iP]]$ then $[[Γ2 ⊢ iP]]$,
    \item[$-$] if $[[Γ1 ⊢ iN]]$ then $[[Γ2 ⊢ iN]]$.
  \end{itemize}
\end{lemma}
\begin{proof}
  By simple mutual induction on $[[iP]]$ and $[[iQ]]$. 
\end{proof}

\begin{corollary}
  \label{lemma:mut-sub-types-wf-equiv}
  Suppose that all the types below are well-formed in $[[Γ]]$ and
  $[[{Γ'} ⊆ {Γ}]]$. Then
  \begin{itemize}
  \item[$+$] $[[Γ ⊢ iP ≈ iQ]]$ implies $[[Γ' ⊢ iP]] \iff [[Γ' ⊢ iQ]]$
  \item[$-$] $[[Γ ⊢ iN ≈ iM]]$ implies $[[Γ' ⊢ iN]] \iff [[Γ' ⊢ iM]]$
  \end{itemize}
\end{corollary}
\begin{proof}
  From \cref{lemma:wf-ctxt-equiv,corollary:fv-mut-sub}.
\end{proof}


\begin{lemma}[Well-formedness agrees with substitution]
  \label{lemma:wf-subst}
  Suppose that $[[Γ2 ⊢ σ : Γ1]]$. Then
  \begin{itemize}
  \item[$+$] $[[Γ, Γ1 ⊢ iP]]$ implies $[[Γ, Γ2 ⊢ [σ]iP]]$, and
  \item[$-$] $[[Γ, Γ1 ⊢ iN]]$ implies $[[Γ, Γ2 ⊢ [σ]iN]]$.
  \end{itemize}
\end{lemma}
\begin{proof}
  We prove it by induction on $[[Γ, Γ1 ⊢ iP]]$ and mutually, on $[[Γ, Γ1 ⊢ iN]]$.
  Let us consider the last rule used in the derivation.
  \begin{caseof}
    \item \ruleref{\ottdruleWFTPVarLabel}, 
      i.e. $[[iP]]$ is $[[α⁺]]$.\\
      By inversion, $[[α⁺ ∊ {Γ, Γ1}]]$, then
      \begin{itemize}
        \item if $[[α⁺ ∊ {Γ1}]]$ then $[[ Γ2 ⊢ [σ]α⁺ ]]$, 
          and by weakening (\cref{lemma:wf-weakening}),
          $[[ Γ, Γ2 ⊢ [σ]α⁺ ]]$;
        \item if $[[α⁺ ∊ {Γ} \ {Γ1}]]$ then $[[ [σ]α⁺ = α⁺ ]]$,
          and by \ruleref{\ottdruleWFTPVarLabel}, $[[ Γ, Γ2 ⊢ α⁺ ]]$.
      \end{itemize}

    \item \ruleref{\ottdruleWFTShiftULabel},
      i.e. $[[iP]]$ is $[[↓iN]]$.\\
      Then $[[Γ, Γ1 ⊢ ↓iN]]$ means $[[Γ, Γ1 ⊢ iN]]$ by inversion,
      and by the induction hypothesis, $[[Γ, Γ2 ⊢ [σ]iN]]$.
      Then by  \ruleref{\ottdruleWFTShiftULabel}, $[[Γ, Γ2 ⊢ ↓[σ]iN]]$, 
      which by definition of substitution is rewritten as $[[Γ, Γ2 ⊢ [σ]↓iN]]$.

    \item \ruleref{\ottdruleWFTExistsLabel},
      i.e. $[[iP]]$ is $[[∃nas.iQ]]$.\\
      Then $[[Γ, Γ1 ⊢ ∃nas.iQ]]$ means $[[Γ, nas, Γ1 ⊢ iQ]]$ 
      by inversion, and by the induction hypothesis, 
      $[[Γ, nas, Γ2 ⊢ [σ]iQ]]$.
      Then by  \ruleref{\ottdruleWFTExistsLabel}, 
      $[[Γ, nas, Γ2 ⊢ ∃nas.[σ]iQ]]$, 
      which by definition of substitution is rewritten as 
      $[[Γ, Γ2 ⊢ [σ]∃nas.iQ]]$.

    \item The negative cases are proved symmetrically.
  \end{caseof}

\end{proof}

\subsection{Substitution}


\begin{lemma}[Substitution strengthening]
  \label{lemma:subst-restr-fv}
  Restricting the substitution to the free variables of the
  substitution subject does not affect the result.
  Suppose that $[[Γ2 ⊢ σ : Γ1]]$. Then
  \begin{itemize}
  \item[$+$] if $[[Γ1 ⊢ iP]]$ then $[[ [σ]iP ]] = [[ [σ|fv iP]iP ]]$,
  \item[$-$] if $[[Γ1 ⊢ iN]]$ then $[[ [σ]iN ]] = [[ [σ|fv iN]iN ]]$
  \end{itemize}
\end{lemma}
\begin{proof}
  \ilyam{todo}
\end{proof}


\begin{corollary}[Substitution preserves equivalence]
  \label{corollary:subst-pres-equiv}

  Suppose that $[[Γ ⊢ σ : Γ1]]$. Then
  \begin{itemize}
  \item[$+$] if $[[Γ1 ⊢ iP]]$,~ $[[Γ1 ⊢ iQ]]$,~ and $[[Γ1 ⊢ iP ≈ iQ]]$ ~ then $[[Γ ⊢ [σ]iP ≈ [σ]iQ]]$
  \item[$-$] if $[[Γ1 ⊢ iN]]$,~ $[[Γ1 ⊢ iM]]$,~ and $[[Γ1 ⊢ iN ≈ iM]]$ ~ then $[[Γ ⊢ [σ]iN ≈ [σ]iM]]$
  \end{itemize}
\end{corollary}

\begin{lemma}[]
  Suppose that $[[{Γ'} ⊆ {Γ}]]$,
  $[[σ1]]$ and $[[σ2]]$ are substitutions of signature $[[Γ ⊢ σi : Γ']]$.
  Then 
  \begin{enumerate}
    \item [$+$] for a type $[[Γ ⊢ iP]]$, if $[[Γ ⊢ [σ1]iP ≈ [σ2]iP]]$ then 
    $[[Γ ⊢ σ1 ≈ σ2 : fv iP ∩ {Γ'}]]$;
    \item [$-$] for a type $[[Γ ⊢ iN]]$, if $[[Γ ⊢ [σ1]iN ≈ [σ2]iN]]$ then
    $[[Γ ⊢ σ1 ≈ σ2 : fv iN ∩ {Γ'}]]$.
  \end{enumerate}
\end{lemma}
\begin{proof}
  Let us make an additional assumption that $[[σ1]]$, $[[σ2]]$, 
  and the mentioned types are normalized. If they are not,
  we normalize them first.
  
  Notice that the normalization preserves
  the set of free variables (\cref{lemma:fv-nf}),
  well-formedness (\cref{corollary:wf-nf}), 
  and equivalence (\cref{lemma:subt-equiv-algorithmization}), 
  and distributes over substitution (\cref{lemma:norm-subst-distr}). 
  This way, the assumed and desired properties are equivalent to their 
  normalized versions.

  We prove it by induction on the structure of $[[iP]]$ and mutually, $[[iN]]$.
  Let us consider the shape of this type.
  \begin{caseof}
    \item $[[iP]] = [[α⁺]] \in [[Γ']]$.
      Then $[[Γ ⊢ σ1 ≈ σ2 : fv iP ∩ {Γ'}]]$ means $[[Γ ⊢ σ1 ≈ σ2 : {α⁺}]]$, 
      i.e. $[[Γ ⊢ [σ1]α⁺ ≈ [σ2]α⁺ ]]$, which holds by assumption.
    \item $[[iP]] = [[α⁺]] \in [[{Γ} \ {Γ'}]]$.
      Then $[[fv iP ∩ {Γ'}]] = [[∅]]$, 
      so $[[Γ ⊢ σ1 ≈ σ2 : fv iP ∩ {Γ'}]]$ holds vacuously.
    \item $[[iP]] = [[↓iN]]$.
      Then the induction hypothesis is applicable to type $[[iN]]$:
      \begin{enumerate}
        \item $[[iN]]$ is normalized,
        \item $[[Γ ⊢ iN]]$ by inversion of $[[Γ ⊢ ↓iN]]$,
        \item $[[Γ ⊢ [σ1]iN ≈ [σ2]iN]]$ holds by inversion of 
          $[[Γ ⊢ [σ1]↓iN ≈ [σ2]↓iN]]$, i.e. $[[Γ ⊢ ↓[σ1]iN ≈ ↓[σ2]iN]]$.
      \end{enumerate}
      This way, we obtain $[[Γ ⊢ σ1 ≈ σ2 : fv iN ∩ {Γ'}]]$, 
      which implies the required equivalence since 
      $[[fv iP ∩ {Γ'}]] = [[fv ↓iN ∩ {Γ'}]] = [[fv iN ∩ {Γ'}]]$.
    \item $[[iP]] = [[∃nas.iQ]]$
      Then the induction hypothesis is applicable to type $[[iQ]]$ 
      well-formed in context $[[Γ, nas]]$:
      \begin{enumerate}
        \item $[[{Γ'} ⊆ {Γ, nas}]]$ since $[[{Γ'} ⊆ {Γ}]]$,
        \item $[[Γ, nas ⊢ σi : Γ']]$ by weakening,
        \item $[[iQ]]$ is normalized,
        \item $[[Γ, nas ⊢ iQ]]$ by inversion of $[[Γ ⊢ ∃nas.iQ]]$,
        \item Notice that $[[ [σi]∃nas.iQ ]]$ is normalized, and thus, 
          $[[ [σ1]∃nas.iQ ≈ [σ2]∃nas.iQ]]$ implies 
          $[[ [σ1]∃nas.iQ = [σ2]∃nas.iQ ]]$
          (by \cref{lemma:subt-equiv-algorithmization}).).
          This equality means $[[ [σ1]iQ = [σ2]iQ ]]$, 
          which implies $[[Γ ⊢ [σ1]iQ ≈ [σ2]iQ]]$.
      \end{enumerate}
    \item $[[iN]] = [[iP → iM]]$
  \end{caseof}
\end{proof}

\begin{lemma}[Substitutions equivalent on the metavariables]
  \label{lemma:subst-equiv-metavar}
  Suppose that $[[Γ ⊢ Θ]]$, $[[uσ1]]$ and $[[uσ2]]$ are substitutions 
  of signature $[[Θ ⊢ uσi]]$.
  Then 
  \begin{enumerate}
    \item [$+$] for a type $[[Γ; Θ ⊢ uP]]$, if $[[Γ ⊢ [uσ1]uP ≈ [uσ2]uP]]$ then
      $[[Θ ⊢ uσ1 ≈ uσ2 : uv uP]]$;
    \item [$-$] for a type $[[Γ; Θ ⊢ uN]]$, if $[[Γ ⊢ [uσ1]uN ≈ [uσ2]uN]]$ then
      $[[Θ ⊢ uσ1 ≈ uσ2 : uv uN]]$.
  \end{enumerate}
\end{lemma}
\begin{proof}
  The proof is a trivial structural induction on 
  $[[Γ; Θ ⊢ uP]]$ and mutually, on $[[Γ; Θ ⊢ uN]]$.
\end{proof}


\subsection{Declarative Subtyping}
\lemmaFvPropagation*
\begin{proof}
  Mutual induction on $[[Γ ⊢ iN ≤ iM]]$ and $[[Γ ⊢ iP ≥ iQ]]$.
  \begin{caseof}
  \item $[[G ⊢ a⁻ ≤ a⁻]]$\\
    It is self-evident that $[[{a⁻} ⊆ {a⁻}]]$.
  \item $[[G ⊢ ↑iP ≤ ↑iQ]]$
    From the inversion (and unfolding $[[G ⊢ iP ≈ iQ]]$ ), we have
    $[[G ⊢ iP ≥ iQ]]$. Then by the induction hypothesis,
    $[[fv(iP)]] \subseteq [[fv(iQ)]]$. The desired 
    inclusion holds, since $[[fv(↑iP)]] = [[fv(iP)]]$ and
    $[[fv(↑iQ)]] = [[fv(iQ)]]$.
  \item $[[G ⊢ iP → iN ≤ iQ → iM]]$
    The induction hypothesis applied to the premises gives:
    $[[fv(iP)]] \subseteq [[fv(iQ)]]$ and
    $[[fv(iN)]] \subseteq [[fv(iM)]]$.
    Then $[[fv(iP → iN)]] = [[fv(iP) ∪ fv(iN)]] \subseteq
    [[fv(iQ) ∪ fv(iM)]] = [[fv(iQ → iM)]]$.
  \item $[[G ⊢ ∀pas.iN ≤ ∀pbs.iM]]$
    \begin{align*}
      [[fv ∀pas.iN ]] &\subseteq [[fv ([iPs/pas] iN) ]] ~\setminus~ [[{pbs}]] 
                      &&   \text{$[[{pbs}]]$ is excluded by the premise $[[fv iN ∩ {pbs} = ∅]]$}\\
                      &\subseteq [[fv iM]] ~\setminus~ [[{pbs}]]
                      &&   \text{by the i.h., } [[fv ([iPs/pas] iN) ]] \subseteq [[fv iM]] \\
                      &\subseteq [[fv ∀pbs.iM]]
    \end{align*}
  \item The positive cases are symmetric.
  \end{caseof}
\end{proof}

\corollaryFvMutSub*

\lemmaMutSubTypesWfEquiv*
\begin{proof}
  From \cref{lemma:wf-ctxt-equiv,corollary:fv-mut-sub}.
\end{proof}


\lemmaQuantRuleDecomposition*
\begin{proof}
  \hfill
  \begin{itemize}
    \item [$-_{R}$] Let us prove both directions. 
      \begin{itemize}
        \item [$\Rightarrow$] Let us assume $[[Γ ⊢ iN ≤ ∀pbs.iM]]$.
          $[[Γ ⊢ iN ≤ ∀pbs.iM]]$.
          Let us decompose $[[iM]]$ as $[[∀pbs'.iM']]$ where $[[iM']]$ does not start with $[[∀]]$, 
          and decompose $[[iN]]$ as $[[∀pas.iN']]$ where $[[iN']]$ does not start with $[[∀]]$.
          If $[[pbs]]$ is empty, then $[[Γ, pbs ⊢ iN ≤ iM]]$ holds by assumption.
          Otherwise, $[[Γ ⊢ ∀pas.iN' ≤ ∀pbs.∀pbs'.iM]]$ is inferred by
          \ruleref{\ottdruleDOneForallLabel}, and by inversion:
          $[[Γ, {pbs}, {pbs'} ⊢ [iPs/pas]iN' ≤ iM']]$ for some $[[Γ, {pbs}, {pbs'} ⊢ iPs]]$.
          Then again by \ruleref{\ottdruleDOneForallLabel} with the same $[[iPs]]$,
          $[[Γ,pbs ⊢ ∀pas.iN' ≤ ∀pbs'.iM']]$, that is $[[Γ,pbs ⊢ iN ≤ iM]]$.
        \item [$\Leftarrow$] let us assume $[[Γ, pbs ⊢ iN ≤ iM]]$, and let us decompose 
          $[[iN]]$ as $[[∀pas.iN']]$ where $[[iN']]$ does not start with $[[∀]]$, 
          and $[[iM]]$ as $[[∀pbs'.iM']]$ where $[[iM']]$ does not start with $[[∀]]$.
          if $[[pas]]$ and $[[pbs']]$ are empty then $[[Γ, pbs ⊢ iN ≤ iM]]$
          is turned into $[[Γ ⊢ iN ≤ ∀pbs.iM]]$ by \ruleref{\ottdruleDOneForallLabel}.
          Otherwise, $[[Γ, pbs ⊢ ∀pas.iN' ≤ ∀pbs'.iM']]$ is inferred by
          \ruleref{\ottdruleDOneForallLabel}, that is $[[Γ, {pbs}, {pbs'} ⊢ [iPs/pas]iN' ≤ iM']]$
          for some $[[Γ, {pbs}, {pbs'} ⊢ iPs]]$.
          Then by \ruleref{\ottdruleDOneForallLabel} again,
          $[[Γ ⊢ ∀pas.iN' ≤ ∀pbs,pbs'.iM']]$, in other words, $[[Γ ⊢ ∀pas.iN' ≤ ∀pbs.∀pbs'.iM']]$, 
          that is $[[Γ ⊢ iN ≤ ∀pbs.iM]]$.
          
      \end{itemize}
    \item [$-_{L}$] Suppose $[[iM]] \neq [[∀]]\dots$. Let us prove both directions.
      \begin{itemize}
        \item [$\Rightarrow$] Let us assume $[[Γ ⊢ ∀pas.iN ≤ iM]]$.
          then if $[[pas = ·]]$, $[[Γ ⊢ iN ≤ iM]]$ holds immediately.
          Otherwise, let us decompose  $[[iN]]$ as $[[∀pas'.iN']]$ where 
          $[[iN']]$ does not start with $[[∀]]$.
          Then $[[Γ ⊢ ∀pas.∀pas'.iN' ≤ iM']]$ is inferred by
          \ruleref{\ottdruleDOneForallLabel},
          and by inversion, 
          there exist $[[Γ ⊢ iPs]]$ and $[[Γ ⊢ iPs']]$ 
          such that $[[Γ ⊢ [iPs/pas][iPs'/pas']iN' ≤ iM']]$ 
          (the decomposition of substitutions is possible since $[[{pas} ∩ Γ = ∅]]$).
          Then by \ruleref{\ottdruleDOneForallLabel} again,
          $[[Γ ⊢ ∀pas'.[iPs'/pas']iN' ≤ iM']]$ (notice that $[[ [iPs'/pas']iN' ]]$ cannot
          start with $[[∀]]$).
        \item [$\Leftarrow$] Let us assume 
          $[[Γ ⊢ [iPs/pas]iN ≤ iM]]$ for some $[[Γ ⊢ iPs]]$.
          let us decompose $[[iN]]$ as $[[∀pas'.iN']]$ where $[[iN']]$ does not start with $[[∀]]$.
          Then $[[Γ ⊢ [iPs/pas]∀pas'.iN' ≤ iM']]$ or, equivalently,
          $[[Γ ⊢ ∀pas'.[iPs/pas]iN' ≤ iM']]$ is inferred by \ruleref{\ottdruleDOneForallLabel}
          (notice that $[[ [iPs/pas]iN' ]]$ cannot start with $[[∀]]$).
          By inversion, there exist $[[Γ ⊢ iPs']]$ such that 
          $[[Γ ⊢ [iPs'/pas'][iPs/pas]iN' ≤ iM']]$. Since $[[pas']]$ is disjoint
          from the free variables of $[[iPs]]$ and from $[[pas]]$, the composition of 
          $[[iPs'/pas']]$ and $[[iPs/pas]]$ can be joined into a single substitution
          well-formed in $[[Γ]]$. Then by \ruleref{\ottdruleDOneForallLabel} again,
          $[[Γ ⊢ ∀pas.iN ≤ iM]]$.
      \end{itemize}
      \item [$+$] The positive cases are proved symmetrically.
  \end{itemize}
\end{proof}

\corollaryRedQuantElim*
\begin{proof}
  \begin{itemize}
    \item [$-_{R}$] Suppose that $[[ {pas} ∩ fv(iM) = ∅]]$ then 
      by \cref{lemma:quant-rule-decomposition},
      $[[Γ ⊢ iN ≤ ∀pas.iM]]$ 
      is equivalent to $[[Γ, pas ⊢ iN ≤ iM]]$,
      By \cref{lemma:wf-ctxt-equiv},
      since $[[{pas} ∩ fv(iN) = ∅]]$ and $[[{pas} ∩ fv(iM) = ∅]]$,
      $[[Γ, pas ⊢ iN ≤ iM]]$ is equivalent to $[[Γ ⊢ iN ≤ iM]]$.

    \item [$-_{L}$] Suppose that $[[ {pas} ∩ fv(iN) = ∅]]$.
      Let us decompose $[[iM]]$ as $[[∀pbs.iM']]$ 
      where $[[iM']]$ does not start with $[[∀]]$.
      By \cref{lemma:quant-rule-decomposition},
      $[[Γ ⊢ ∀pas.iN ≤ ∀pbs.iM']]$ is equivalent to
      $[[Γ,pbs ⊢ ∀pas.iN ≤ iM']]$, 
      which is equivalent to 
      existence of $[[Γ,pbs ⊢ iPs]]$ such that 
      $[[Γ,pbs ⊢ [iPs/pas]iN ≤ iM']]$.
      Since $[[ [iPs/pas]iN  = iN]]$, the latter is equivalent to 
      $[[Γ,pbs ⊢ iN ≤ iM']]$,
      which is equivalent to $[[Γ ⊢ iN ≤ ∀pbs.iM']]$.
      $[[Γ,pbs ⊢ iPs]]$ can be chosen arbitrary, for example, $[[iPsi]] = [[∃α⁻.↓α⁻]]$.
    \item [$+$] The positive cases are proved symmetrically.
  \end{itemize}
\end{proof}

\lemmaVarSubt*
\begin{proof}
  We prove by induction on the tree
  inferring $[[Γ ⊢ iP ≥ ∃nas.α⁺]]$ or $[[Γ ⊢ ∃nas.α⁺ ≥ iP ]]$ or
  or $[[Γ ⊢ iN ≤ ∀pas.α⁻]]$ or $[[Γ ⊢ ∀pas.α⁻ ≤ iN ]]$.

  Let us consider which one of these judgments is inferred.
  \begin{caseof}
  \item $[[Γ ⊢ iP ≥ ∃nas.α⁺]]$\\
    If the size of the inference tree is $1$ then the only rule that can infer
    it is \ruleref{\ottdruleDOnePVarLabel}, which
    implies that $[[nas]]$ is empty and $[[iP = α⁺]]$.

    If the size of the inference tree is $>1$ then the last rule inferring
    it must be \ruleref{\ottdruleDOneExistsLabel}. By inverting this rule,
    $[[iP = ∃nbs.iP']]$ where $[[iP']]$ does not start with $\exists$ and
    $[[Γ, nas ⊢ [iNs/nbs] iP' ≥ α⁺]]$ for some $[[G, nas ⊢ iNi]]$.

    By the induction hypothesis, $[[ [iNs/nbs] iP' = ∃ncs.α⁺]]$.
    What shape can $[[iP']]$ have?
    As mentioned, it does not start with $\exists$, and it cannot start with
    $\uparrow$ (otherwise, $[[ [iNs/nas] iP' ]]$ would also
    start with $\uparrow$ and would not be equal to $[[∃nbs.α⁺]]$).
    This way, $[[iP']]$ is a \emph{positive} variable. 
    As such, $[[ [iNs/nas] iP' = iP']]$,
    and then $[[iP' = ∃ncs.α⁺]]$ meaning that $[[ncs]]$ is empty and $[[iP' = α⁺]]$.
    This way, $[[iP]] = [[∃nbs.iP']] = [[∃nbs.α⁺]]$, as required.

  \item $[[Γ ⊢ ∃nas.α⁺ ≥ iP]]$\\
    If the size of the inference tree is $1$ then the only rule that can infer
    it is \ruleref{\ottdruleDOnePVarLabel}, which
    implies that $[[nas]]$ is empty and $[[iP = α⁺]]$.

    If the size of the inference tree is $>1$ then the last rule inferring
    it must be \ruleref{\ottdruleDOneExistsLabel}. By inverting this rule,
    $[[iP = ∃nbs.iQ]]$ where $[[G, nbs ⊢ [iNs/nas]α⁺ ≥ iQ]]$ and $[[iQ]]$ 
    does not start with $\exists$.
    Notice that since $[[α⁺]]$ is positive, $[[ [iNs/nas]α⁺ = α⁺]]$, 
    i.e. $[[G, nbs ⊢ α⁺ ≥ iQ]]$.

    By the induction hypothesis, $[[iQ = ∃nbs'.α⁺]]$,
    and since $[[iQ]]$ does not start with $\exists$, $[[nbs']]$ is empty
    This way, $[[iP]] = [[∃nbs.iQ]] = [[∃nbs.α⁺]]$, as required.

  \item The negative cases ($[[Γ ⊢ iN ≤ ∀pas.α⁻]]$ and $[[Γ ⊢ ∀pas.α⁻ ≤ iN ]]$)
    are proved analogously.
  \end{caseof}
\end{proof}

\corollaryVarsNoProperSubtypes*
\begin{proof}
  Notice that $[[Γ ⊢ ∃nbs.α⁺ ≈ α⁺]]$ and $[[∃nbs.α⁺ ≈ α⁺]]$ and apply
  \cref{lemma:var-subt}.
\end{proof}


\lemmaSubtCtxtIrrelevance*
\begin{proof}
  We prove it by induction on 
  the size of $[[Γ1 ⊢ iP ≥ iQ]]$ and mutually,
  the size of $[[Γ1 ⊢ iN ≤ iM]]$.

  All the cases except \ruleref{\ottdruleDOneExistsLabel} and \ruleref{\ottdruleDOneForallLabel}
  are proven congruently:
  first, we apply the inversion to $[[Γ1 ⊢ iP ≥ iQ]]$ to obtain the premises of the
  corresponding rule $X$, then we apply the induction hypothesis to each premise,
  and build the inference tree (with $[[Γ2]]$) by the same rule $X$. 

  Suppose that the judgment is inferred by \ruleref{\ottdruleDOneExistsLabel}.
  Then we are proving that $[[Γ1 ⊢ ∃nas.iP ≥ ∃nbs.iQ]]$ implies 
  $[[Γ2 ⊢ ∃nas.iP ≥ ∃nbs.iQ]]$ (the other implication is proven symmetrically).

  By inversion  of $[[Γ1 ⊢ ∃nas.iP ≥ ∃nbs.iQ]]$, 
  we obtain $[[σ]]$ such that $[[ Γ1, nbs ⊢ σ :{nas}]]$
  and $[[Γ1, nbs ⊢ [σ]iP ≥ iQ]]$.
  By \cref{lemma:fv-propagation}, $[[fv([σ]iP) ⊆ fv(iQ)]]$.

  From the well-formedness statements 
  $[[Γi ⊢ ∃nas.iP]]$ and $[[Γi ⊢ ∃nbs.iQ]]$ we have:
  \begin{itemize}
    \item $[[Γ1, nas ⊢ iP]]$, which also means $[[Γ1, nbs ⊢ [σ]iP]]$
      by \cref{lemma:wf-subst};
    \item $[[Γ2, nas ⊢ iP]]$;
    \item $[[Γ1, nbs ⊢ iQ]]$; and 
    \item $[[Γ2, nbs ⊢ iQ]]$, which means $[[ fv(iQ) ⊆ Γ2, nbs ]]$
      by \cref{lemma:wf-soundness},
      and  combining it with $[[fv([σ]iP) ⊆ fv(iQ)]]$, 
      we have $[[fv([σ]iP) ⊆ Γ2, nbs]]$.
  \end{itemize}

  Let us construct a substitution $[[σ0]]$ in the following way:
  $$
  \begin{cases}
      [[ [σ0]αi⁻ = [σ]αi⁻  ]] & \text{for $[[αi⁻ ∊ {nas} ∩ fv(iP)]]$ }\\
      [[ [σ0]αi⁻ = ∀γ⁺.↑γ⁺ ]] & \text{for $[[αi⁻ ∊ {nas} \ fv(iP)]]$ }\\
      [[ [σ0]γ±  = γ± ]]      & \text{for any other $[[γ±]]$ }\\
  \end{cases}
  $$
  Notice that 
  \begin{enumerate}
    \item $[[ [σ0]iP = [σ]iP ]]$.
      Since $[[σ0|fv(iP)]] = [[σ|fv(iP)]]$ as functions
      (which follows from the construction of 
      $[[σ0]]$ and the signature of $[[σ]]$), 
      $[[ [σ0]iP = [σ0|fv(iP)]iP = [σ|fv(iP)]iP = [σ]iP]]$
      (where the first and the last equalities are by
       \cref{lemma:subst-restr-fv}).

    \item $[[  fv([σ]iP) ⊢ σ0 :{nas}]]$.
      To show that, let us consider $[[αi⁻]]$ 
      \begin{itemize}
        \item if $[[αi⁻ ∊ {nas} \ fv(iP)]]$ then
          $[[· ⊢ [σ0]αi⁻]]$, which can be weakened to 
          $[[ fv([σ]iP) ⊢ [σ0]αi⁻]]$;
        \item if $[[αi⁻ ∊ {nas} ∩ fv(iP)]]$, 
          we have $[[ [σ0]αi⁻ = [σ]αi⁻ ]]$, and thus, by
          specification of $[[σ]]$, $[[Γ1, pbs ⊢ [σ0]αi⁻ ]]$. 
          By \cref{corollary:wf-ctxt-strengthening}, 
          it means $[[ fv([σ0]αi⁻) ⊢ [σ0]αi⁻]]$,
          which we weaken (\cref{lemma:wf-weakening}) 
          to $[[ fv([σ]iP) ⊢ [σ0]αi⁻]]$ 
          (since $[[fv([σ0]αi⁻) ⊆ fv([σ0]iP)]]$ by \cref{lemma:subst-fv-image},
          and $[[ [σ0]iP = [σ]iP]]$, as noted above).
      \end{itemize} 
  \end{enumerate}

  By \cref{corollary:wf-ctxt-strengthening}, 
  $[[Γ1, nbs ⊢ [σ]iP]]$ implies 
  $[[  fv([σ]iP) ⊢ [σ]iP ]]$, which,
  since $[[fv([σ]iP) ⊆ Γ2, nbs]]$,
  is weakened to $[[Γ2, nbs ⊢ [σ]iP]]$.
  and rewritten as $[[Γ2, nbs ⊢ [σ0]iP]]$.
  
  Notice that the premises of the induction hold:
  \begin{enumerate}
    \item $[[Γi, nbs ⊢ [σ0]iP]]$, 
    \item $[[Γi, nbs ⊢ iQ]]$, and
    \item $[[Γ1, nbs ⊢ [σ0]iP ≥ iQ ]]$, 
      notice that the tree inferring this judgment 
      is the same tree inferring $[[Γ1, nbs ⊢ [σ]iP ≥ iQ ]]$ 
      (since $[[ [σ0]iP = [σ]iP]]$), i.e., 
      it is a subtree of $[[Γ1 ⊢ ∃nas.iP ≥ ∃nbs.iQ]]$.
  \end{enumerate}
  This way, by the induction hypothesis,
  $[[Γ2, nbs ⊢ [σ0]iP ≥ iQ ]]$.
  Combining it with $[[ Γ2, nbs ⊢ σ0 :{nas}]]$
  by \ruleref{\ottdruleDOneExistsLabel},
  we obtain $[[Γ2 ⊢ ∃nas.iP ≥ ∃nbs.iQ]]$.

  The case of $[[Γ1 ⊢ ∀pas.iN ≤ ∀pbs.iM]]$ is symmetric.
\end{proof}


\lemmaSubtWeakening*
\begin{proof}
  By straightforward induction on the subtyping derivation.
  The polymorphic cases follow from \cref{lemma:subst-range-weakening}.
\end{proof}

\lemmaSubtypingReflexivity*
\begin{proof}
  Let us prove it by the size of $[[iN]]$ and mutually, $[[iP]]$.
  \begin{caseof}
    \item $[[iN]] = [[α⁻]]$\\
      Then $[[Γ ⊢ α⁻ ≤ α⁻]]$ is inferred immediately by \ruleref{\ottdruleDOneNVarLabel}.
    \item $[[iN]] = [[∀pas.iN']]$ where $[[pas]]$ is not empty\\
      First, we rename $[[pas]]$ to fresh $[[pbs]]$ in $[[∀pas.iN']]$ to avoid
      name clashes: $[[∀pas.iN']] = [[∀pbs.[pas/pbs]iN']]$.
      Then to infer $[[Γ ⊢ ∀pas.iN' ≤ ∀pbs.[pas/pbs]iN']]$ we can apply 
      \ruleref{\ottdruleDOneForallLabel}, instantiating $[[pas]]$ with $[[pbs]]$:
      \begin{itemize}
        \item $[[fv iN ∩ {pbs} = ∅ ]]$ by choice of $[[pbs]]$,
        \item $[[G, pbs ⊢ pbi]]$,
        \item $[[G, pbs ⊢ [pbs/pas] iN' ≤ [pbs/pas] iN']]$ by the induction hypothesis,
        since the size of $[[ [pbs/pas]iN' ]]$ is equal to the size of $[[iN']]$,
        which is smaller than the size of $[[iN]] = [[∀pas.iN']]$.
      \end{itemize}
    \item $[[iN]] = [[iP → iM]]$\\
      Then $[[Γ ⊢ iP → iM ≤ iP → iM]]$ is inferred by \ruleref{\ottdruleDOneArrowLabel},
      since $[[Γ ⊢ iP ≥ iP]]$ and $[[Γ ⊢ iM ≤ iM]]$ hold the induction hypothesis. 
    \item $[[iN]] = [[↑iP]]$\\
      Then $[[Γ ⊢ ↑iP ≤ ↑iP]]$ is inferred by \ruleref{\ottdruleDOneShiftULabel},
      since $[[Γ ⊢ iP ≥ iP]]$ holds by the induction hypothesis.
    \item The positive cases are symmetric to the negative ones.
  \end{caseof}
\end{proof}


\lemmaSubstPresSubt*
\begin{proof}
  We prove it by induction on the size of the derivation of $[[Γ1 ⊢ iN ≤ iM]]$
  and mutually, $[[Γ1 ⊢ iP ≥ iQ]]$. Let us consider the last rule 
  used in the derivation:
  \begin{caseof}
    \item \ruleref{\ottdruleDOneNVarLabel}. Then by inversion, 
      $[[iN = α⁻]]$ and $[[iM = α⁻]]$. By reflexivity of subtyping
      (\cref{lemma:subtyping-reflexivity}),
      we have $[[Γ2 ⊢ [σ]α⁻ ≤ [σ]α⁻]]$, i.e. $[[Γ2 ⊢ [σ]iN ≤ [σ]iM]]$,
      as required.
    \item  \ruleref{\ottdruleDOneForallLabel}. Then by inversion,
      $[[iN = ∀pas.iN']]$, $[[iM = ∀pbs.iM']]$, where $[[pas]]$ or $[[pbs]]$ is not empty.
      Moreover, $[[Γ1, pbs ⊢ [iPs/pas]iN' ≤ iM']]$ for some $[[Γ1, pbs ⊢ iPs]]$, and 
      $[[fv iN ∩ {pbs} = ∅ ]]$.

      Notice that since the derivation of $[[Γ1, pbs ⊢ [iPs/pas]iN' ≤ iM']]$ is
      a subderivation of the derivation of $[[Γ ⊢ iN ≤ iM]]$, its size is smaller, 
      and hence, the induction hypothesis applies
      ($[[Γ1, pbs ⊢ σ : Γ1, pbs]]$ by \cref{lemma:subst-domain-weakening})
      :
      $[[Γ2, pbs ⊢ [σ][iPs/pas]iN' ≤ [σ]iM']]$.

      Notice that by convention, $[[pas]]$ and $[[pbs]]$ are fresh, and thus,  
      $[[ [σ]∀pas.iN' ]] = [[ ∀pas.[σ]iN' ]]$ and $[[ [σ]∀pbs.iM' ]] = [[ ∀pbs.[σ]iM' ]]$, 
      which means that the required $[[Γ2, Γ ⊢ [σ]∀pas.iN' ≤ [σ]∀pbs.iM']]$ is rewritten as
      $[[Γ2 , Γ ⊢ ∀pas.[σ]iN' ≤ ∀pbs.[σ]iM']]$.

      To infer it, we apply \ruleref{\ottdruleDOneForallLabel}, 
      instantiating $[[pai]]$ with $[[ [σ]iPi ]]$:
      \begin{itemize}
        \item $[[fv [σ]iN ∩ {pbs} = ∅ ]]$;
        \item $[[Γ2, Γ,pbs⊢ [σ]iPi]]$, by \cref{lemma:wf-subst} since from the inversion,
          $[[Γ1, Γ, pbs ⊢ iPi]]$;
        \item $[[Γ, pbs ⊢ [ [σ]iPs/pas ][σ]iN' ≤ [σ]iM']]$ holds
          by \cref{lemma:subst-composition}:
          Since $[[pas]]$ is fresh, it is disjoint with the domain and the codomain of $[[σ]]$
          ($[[Γ1]]$ and $[[Γ2]]$), and thus, 
          $[[ [σ][iPs/pas]iN' ]] = [[ [ σ <=< iPs/pas ][σ]iN' ]] = [[ [ [σ]iPs/pas ][σ]iN' ]]$.
          Then $[[Γ2, Γ, pbs ⊢ [σ][iPs/pas]iN' ≤ [σ]iM']]$ holds by the induction hypothesis.
      \end{itemize}

    \item \ruleref{\ottdruleDOneArrowLabel}. Then by inversion,
      $[[iN = iP → iN1]]$, $[[iM = iQ → iM1]]$, $[[Γ ⊢ iP ≥ iQ]]$, and $[[Γ ⊢ iN1 ≤ iM1]]$.
      And by the induction hypothesis, $[[Γ' ⊢ [σ]iP ≥ [σ]iQ]]$ and $[[Γ' ⊢ [σ]iN1 ≤ [σ]iM1]]$.
      Then $[[Γ' ⊢ [σ]iN ≤ [σ]iM]]$, i.e. $[[Γ' ⊢ [σ]iP → [σ]iN1 ≤ [σ]iQ → [σ]iM1]]$,
      is inferred by \ruleref{\ottdruleDOneArrowLabel}.
    \item \ruleref{\ottdruleDOneShiftULabel}. Then by inversion,
      $[[iN = ↑iP]]$, $[[iM = ↑iQ]]$, and $[[Γ ⊢ iP ≈ iQ]]$,
      which by inversion means that $[[Γ ⊢ iP ≥ iQ]]$ and $[[Γ ⊢ iQ ≥ iP]]$.
      Then the induction hypothesis applies, and we have $[[Γ' ⊢ [σ]iP ≥ [σ]iQ]]$
      and $[[Γ' ⊢ [σ]iQ ≥ [σ]iP]]$. 
      Then by sequential application of \ruleref{\ottdruleDOneNDefLabel} 
      and \ruleref{\ottdruleDOneShiftULabel} to these judgments,
      we have $[[Γ' ⊢ ↑[σ]iP ≤ ↑[σ]iQ]]$, i.e.
      $[[Γ' ⊢ [σ]iN ≤ [σ]iM]]$, as required.
    \item The positive cases are proved symmetrically.
  \end{caseof}
\end{proof}

\corollarySubstPresEquiv*

\lemmaSubtypingTransitivity*
\begin{proof}
  To prove it, we formulate a stronger property, 
  which will imply the required one, taking $[[σ]] = [[Γ ⊢ id : Γ]]$.

    Assuming all the types are well-formed in $[[Γ]]$,
    \begin{itemize}
      \item[$-$] if $[[Γ ⊢ iN ≤ iM1]]$, $[[Γ ⊢ iM2 ≤ iK]]$, and for 
        $[[Γ' ⊢ σ : Γ]]$, $[[ [σ]iM1 = [σ]iM2 ]]$ then $[[Γ' ⊢ [σ]iN ≤ [σ]iK]]$
      \item[$+$] if $[[Γ ⊢ iP ≥ iQ1]]$, $[[Γ ⊢ iQ2 ≥ iR]]$, and for
        $[[Γ' ⊢ σ : Γ]]$, $[[ [σ]iQ1 = [σ]iQ2 ]]$ then $[[Γ' ⊢ [σ]iP ≥ [σ]iR]]$
    \end{itemize}

  We prove it by induction on $\size{[[Γ ⊢ iN ≤ iM1]]} + \size{[[Γ ⊢ iM2 ≤ iK]]}$ and mutually, 
  on $\size{[[Γ ⊢ iP ≥ iQ1]]} + \size{[[Γ ⊢ iQ2 ≥ iR]]}$.


  First, let us consider the 3 important cases.
  \begin{caseof}
    \item Let us consider the case when $[[iM1 = ∀pbs1.α⁻]]$. 
      Then by \cref{lemma:var-subt},
       $[[Γ ⊢ iN ≤ iM1]]$ means that $[[iN = ∀pas.α⁻]]$. 
      $[[ [σ]iM1 = [σ]iM2 ]]$ means that $[[ ∀pbs1.[σ]α⁻ = [σ]iM2 ]]$.
      Applying $[[σ]]$ to both sides of $[[Γ ⊢ iM2 ≤ iK]]$ (by \cref{lemma:subst-pres-subt}),
      we obtain $[[Γ' ⊢ [σ]iM2 ≤ [σ]iK]]$, that is $[[Γ' ⊢  ∀pbs1.[σ]α⁻ ≤ [σ]iK]]$.
      Since $[[ fv([σ]α⁻) ⊆ Γ,α⁻ ]]$, it is disjoint from $[[pas]]$ and $[[pbs1]]$,
      This way, by \cref{corollary:red-quant-elim}, 
      $[[Γ' ⊢  ∀pbs1.[σ]α⁻ ≤ [σ]iK]]$ is equivalent to 
      $[[Γ' ⊢  [σ]α⁻ ≤ [σ]iK]]$, which is equivalent to $[[Γ' ⊢  ∀pas.[σ]α⁻ ≤ [σ]iK]]$,
      that is $[[Γ' ⊢  [σ]iN ≤ [σ]iK]]$.
    \item Let us consider the case when $[[iM2 = ∀pbs2.α⁻]]$.
      This case is symmetric to the previous one. Notice that 
      \cref{lemma:var-subt,corollary:red-quant-elim} are agnostic to the 
      side on which the quantifiers occur, and thus, 
      the proof stays the same. 
    \item Let us decompose the types, by extracting the outer quantifiers:
      \begin{itemize}
        \item $[[iN = ∀pas.iN']]$, where $[[iN']] \neq [[∀]]\dots$,
        \item $[[iM1 = ∀pbs1.iM1']]$, where $[[iM1']] \neq [[∀]]\dots$,
        \item $[[iM2 = ∀pbs2.iM2']]$, where $[[iM2']] \neq [[∀]]\dots$,
        \item $[[iK = ∀pcs.iK']]$, where $[[iK']] \neq [[∀]]\dots$.
      \end{itemize}
      and assume that at least one of $[[pas]]$, $[[pbs1]]$, $[[pbs2]]$, and $[[pcs]]$ is not empty.
      Since $[[ [σ]iM1 = [σ]iM2 ]]$, we have $[[ ∀pbs1.[σ]iM1' = ∀pbs2.[σ]iM2' ]]$,
      and since $[[iMi']]$ are not variables 
      (which was covered by the previous cases) and do not start with $\forall$,
      $[[ [σ]iMi' ]]$ do not start with $\forall$ either,
      which means $[[pbs1]] = [[pbs2]]$ and $[[ [σ]iM1' = [σ]iM2' ]]$.
      Let us rename $[[pbs1]]$ and $[[pbs2]]$ to $[[pbs]]$.
      Then $[[iM1 = ∀pbs.iM1']]$ and $[[iM2 = ∀pbs.iM2']]$.

      By \cref{lemma:quant-rule-decomposition} applied twice
      to $[[Γ ⊢ ∀pas.iN' ≤ ∀pbs.iM1']]$ and to $[[Γ ⊢ ∀pbs.iM2' ≤ ∀pcs.iK']]$,
      we have the following:
      \begin{enumerate}
        \item $[[Γ, pbs ⊢ [iPs/pas]iN' ≤ iM1']]$ for some $[[Γ, pbs ⊢ iPs]]$;
        \item $[[Γ, pcs ⊢ [iQs/pbs]iM2' ≤ iK']]$ for some $[[Γ, pcs ⊢ iQs]]$.
      \end{enumerate}
      And since at least one of 
      $[[pas]]$, $[[pbs]]$, and $[[pcs]]$ is not empty,
      either $[[Γ ⊢ iN ≤ iM1]]$ or $[[Γ ⊢ iM2 ≤ iK]]$ is inferred 
      by \ruleref{\ottdruleDOneForallLabel}, meaning that either 
      $[[Γ, pbs ⊢ [iPs/pas]iN' ≤ iM1']]$ is a proper subderivation of $[[Γ ⊢ iN ≤ iM1]]$ or
      $[[Γ, pcs ⊢ [iQs/pbs]iM2' ≤ iK']]$ is a proper subderivation of $[[Γ ⊢ iM2 ≤ iK]]$.

      Notice that we can weaken and rearrange the contexts without changing the sizes of the 
      derivations: $[[Γ, {pbs}, {pcs} ⊢ [iPs/pas]iN' ≤ iM1']]$
      and $[[Γ, {pbs}, {pcs} ⊢ [iQs/pbs]iM2' ≤ iK']]$. This way, 
      the sum of the sizes of these derivations is smaller than the sum of the sizes of
      $[[Γ ⊢ iN ≤ iM1]]$ and $[[Γ ⊢ iM2 ≤ iK]]$.
      Let us apply the induction hypothesis to these derivations, 
      with the substitution $[[ Γ', pcs ⊢ σ ○ (iQs/pbs) : Γ, {pbs}, {pcs}  ]]$
      (\cref{lemma:subst-domain-weakening}).
      To apply the induction hypothesis, it is left to show that 
      $[[ σ ○ (iQs/pbs) ]]$ unifies $[[iM1']]$ and $[[ [iQs/pbs]iM2']]$:

      \begin{align*}
        [[ [σ ○ iQs/pbs]iM1' ]] &= [[ [σ][iQs/pbs]iM1' ]]\\
                                &= [[ [ [σ]iQs/pbs ][σ]iM2' ]]
                                && \text{by \cref{lemma:subst-composition}}\\
                                &= [[ [ [σ]iQs/pbs ][σ]iM2' ]]
                                && \text{Since $[[ [σ]iM1' = [σ]iM2' ]]$}\\
                                &= [[  [σ][iQs/pbs]iM2' ]]
                                && \text{by \cref{lemma:subst-composition}}\\
                                &= [[  [σ][iQs/pbs][iQs/pbs]iM2' ]]
                                && \text{Since $[[Γ, pcs ⊢ iQs]]$, and $[[{(Γ, pcs)} ∩ {pbs} = ∅]]$ }\\
                                &= [[  [σ ○ iQs/pbs][iQs/pbs]iM2' ]]
      \end{align*}

      This way the induction hypothesis gives us
      $[[ Γ', pcs ⊢ [σ][iQs/pbs][iPs/pas]iN' ≤  [σ][iQs/pbs]iK' ]]$,
      and since $[[Γ, pcs ⊢ iK']]$, $[[ [iQs/pbs]iK' = iK' ]]$, that is 
      $[[ Γ', pcs ⊢ [σ][iQs/pbs][iPs/pas]iN' ≤  [σ]iK' ]]$.
      Let us rewrite the substitution that we apply to $[[iN']]$:
      \begin{align*}
        [[ [σ ○ iQs/pbs ○ iPs/pas]iN' ]] &= [[ [ (σ <=< iQs/pbs) ○ σ ○ iPs/pas]iN' ]]
                                       && \text{by \cref{lemma:subst-composition}}\\
                                       &= [[ [(σ <=< iQs/pbs) ○ (σ <=< iPs/pas) ○ σ] iN' ]]
                                       && \text{by \cref{lemma:subst-composition}}\\
                                       &= [[ [(((σ <=< iQs/pbs) ○ σ) <=< iPs/pas) ○ σ] iN' ]]
                                       && [[fv([σ]iN') ∩ {pbs} = ∅]]\\
                                       &= [[ [((σ ○ iQs/pbs) <=< iPs/pas) ○ σ] iN' ]]
                                       && \text{by \cref{lemma:subst-composition}}\\
                                       &= [[ [(σ ○ iQs/pbs) <=< iPs/pas][σ] iN' ]]
      \end{align*}
      Notice that $[[(σ ○ iQs/pbs) <=< iPs/pas]]$
      is a substitution that turns $[[pai]]$ into $[[ [σ ○ iQs/pbs]iPi ]]$, 
      where $[[ Γ',pcs ⊢ [σ ○ iQs/pbs]iPi]]$.
      This way, 
      $[[ Γ', pcs ⊢ [(σ ○ iQs/pbs) <=< iPs/pas][σ]iN' ≤  [σ]iK' ]]$
      means $[[Γ ⊢ ∀pas.[σ]iN' ≤ ∀pcs.[σ]iK']]$
      by \cref{lemma:quant-rule-decomposition}, that is
      $[[Γ ⊢ [σ]iN ≤ [σ]iK]]$, as required.
  \end{caseof}

  Now, we can assume that neither $[[Γ ⊢ iN ≤ iM1]]$ nor $[[Γ ⊢ iM2 ≤ iK]]$ 
  is inferred by \ruleref{\ottdruleDOneForallLabel}, and that neither $[[iM1]]$ nor $[[iM2]]$
  is equivalent to a variable.  Because of that, $[[ [σ]iM1 = [σ]iM2 ]]$ means that 
  $[[iM1]]$ and $[[iM2]]$ have the same outer constructor. Let us consider the shape of $[[iM1]]$.

  \begin{caseof}
    \item $[[iM1 = α⁻]]$ this case has been considered;
    \item $[[iM1 = ∀pbs.iM1']]$ this case has been considered;
    \item $[[iM1 = ↑iQ1]]$. Then as noted above, 
      $[[ [σ]iM1 = [σ]iM2 ]]$ means that $[[iM2 = ↑iQ2]]$ and $[[ [σ]iQ1 = [σ]iQ2 ]]$.
      Moreover, $[[Γ ⊢ iN ≤ ↑iQ1]]$ can only be inferred by \ruleref{\ottdruleDOneShiftULabel},
      and thus, $[[iN = ↑iP]]$, and by inversion, $[[Γ ⊢ iP ≥ iQ1]]$ and $[[Γ ⊢ iQ1 ≥ iP]]$.
      Analogously, $[[Γ ⊢ ↑iQ2 ≤ iK]]$ means that $[[iK = ↑iR]]$, $[[Γ ⊢ iQ2 ≥ iR]]$, and $[[Γ ⊢ iR ≥ iQ2]]$.

      Notice that the derivations of $[[Γ ⊢ iP ≥ iQ1]]$ and $[[Γ ⊢ iQ1 ≥ iP]]$ are proper sub-derivations of 
      $[[Γ ⊢ iN ≤ iM1]]$, and the derivations of $[[Γ ⊢ iQ2 ≥ iR]]$ and $[[Γ ⊢ iR ≥ iQ2]]$ are proper sub-derivations of
      $[[Γ ⊢ iM2 ≤ iK]]$. This way, the induction hypothesis is applicable:
      \begin{itemize}
        \item applying the induction hypothesis to $[[Γ ⊢ iP ≥ iQ1]]$ and $[[Γ ⊢ iQ2 ≥ iR]]$ 
          with $[[Γ' ⊢ σ : Γ]]$ unifying $[[iQ1]]$ and $[[iQ2]]$, we obtain $[[Γ' ⊢ [σ]iP ≥ [σ]iR]]$;
        \item applying the induction hypothesis to $[[Γ ⊢ iR ≥ iQ2]]$ and $[[Γ ⊢ iQ1 ≥ iP]]$ 
          with $[[Γ' ⊢ σ : Γ]]$ unifying $[[iQ2]]$ and $[[iQ1]]$, we obtain $[[Γ' ⊢ [σ]iR ≥ [σ]iP]]$.
      \end{itemize}
      This way, by \ruleref{\ottdruleDOneShiftULabel}, $[[Γ' ⊢ [σ]iN ≤ [σ]iK]]$, as required. 

    \item $[[iM1 = iQ1 → iM1']]$. Then as noted above, 
      $[[ [σ]iM1 = [σ]iM2 ]]$ means that $[[iM2 = iQ2 → iM2']]$, $[[ [σ]iQ1 = [σ]iQ2 ]]$, and $[[ [σ]iM1' = [σ]iM2' ]]$.
      Moreover, $[[Γ ⊢ iN ≤ iQ1 → iM1']]$ can only be inferred by \ruleref{\ottdruleDOneArrowLabel},
      and thus, $[[iN = iP → iN']]$, and by inversion, $[[Γ ⊢ iP ≥ iQ1]]$ and $[[Γ ⊢ iN' ≤ iM1']]$.
      Analogously, $[[Γ ⊢ iQ2 → iM2' ≤ iK]]$ means that $[[iK = iR → iK']]$, $[[Γ ⊢ iQ2 ≥ iR]]$, and $[[Γ ⊢ iM2' ≤ iK']]$.

      Notice that the derivations of $[[Γ ⊢ iP ≥ iQ1]]$ and $[[Γ ⊢ iN' ≤ iM1']]$ are proper sub-derivations of
      $[[Γ ⊢ iP → iN' ≤ iQ1 → iM1']]$, and the derivations of $[[Γ ⊢ iQ2 ≥ iR]]$ and $[[Γ ⊢ iM2' ≤ iK']]$ are proper sub-derivations of
      $[[Γ ⊢ iQ2 → iM2' ≤ iR → iK']]$. This way, the induction hypothesis is applicable:
      \begin{itemize}
        \item applying the induction hypothesis to $[[Γ ⊢ iP ≥ iQ1]]$ and $[[Γ ⊢ iQ2 ≥ iR]]$ 
          with $[[Γ' ⊢ σ : Γ]]$ unifying $[[iQ1]]$ and $[[iQ2]]$, we obtain $[[Γ' ⊢ [σ]iP ≥ [σ]iR]]$;
        \item applying the induction hypothesis to $[[Γ ⊢ iN' ≤ iM1']]$ and $[[Γ ⊢ iM2' ≤ iK']]$ 
          with $[[Γ' ⊢ σ : Γ]]$ unifying $[[iM1']]$ and $[[iM2']]$, we obtain $[[Γ' ⊢ [σ]iN' ≤ [σ]iK']]$.
      \end{itemize}
      This way, by \ruleref{\ottdruleDOneArrowLabel}, $[[Γ' ⊢ [σ]iP → [σ]iN' ≤ [σ]iR → [σ]iK']]$,
      that is $[[Γ' ⊢ [σ]iN ≤ [σ]iK]]$, as required.
  \end{caseof}
  After that, we consider all the 
  analogous positive cases and prove them symmetrically.
\end{proof}

\corollaryEquivalenceTransitivity*


\subsection{Overview}
\renewcommand\stackalignment{l}
\begin{tabular}{@{}lccc@{}} \toprule
  % supertypes of ... & ... are \\ 
  Algorithm                   & Soundness & Completeness & Initiality \\
  \midrule
  \addlinespace[0.7em]
  % $[[ ord varset in iN ]]$
  Ordering
                      &   \infer{[[ {ord varset in iN} ]] \equiv [[varset ∩ fv iN]]}{}{}
                                % \infer{\stackunder{hello}{world}}{}{}
                                % \begin{itemize}
                                % \item[$-$] $[[ {ord varset in uN} ]] \equiv [[varset ∩ fv uN]]$ (as sets)
                                % \item[$+$] $[[ {ord varset in uP} ]] \equiv [[varset ∩ fv uP]]$ (as sets)
                                % \end{itemize}
                            & \infer{[[ord varset in iN]] = [[ord varset in iM]]}{[[iN ≈ iM]]}{}
                            & --- \\

  \addlinespace[0.7em]
  % $[[ nf(iN) ]]$
  Normalization
                      &   \infer{[[iN ≈ nf(iN)]]}{}{}
                  & \infer{[[nf(iN)]] = [[nf(iM)]]}{[[iN ≈ iM]]}{}
                  & --- 
  \\

  \addlinespace[0.7em]
  % $[[ Γ ⊨ iP1 ∨ iP2 = iQ]]$
  Equivalence
                      & \infer
                        { 
                        [[Γ ⊢ iP ≈ iQ]]
                        }
                        {
                        [[Γ ⊢ iP]] & [[Γ ⊢ iQ]] & [[iP ≈ iQ]]
                        }
                        {}
                      & \infer{[[iP ≈ iQ]]}{[[Γ ⊢ iP ≈ iQ]]}{}
                      &  ---

  \\
  \addlinespace[0.7em]
  % $[[ upgrade Γ ⊢ iP to Δ = iQ ]]$
  Uppgrade
                      & \infer
                                     { [[iQ]] \text{ is sound}
                                      \begin{cases}
                                         [[Δ ⊢ iQ]]\\
                                         [[Γ ⊢ iQ ≥ iP]]
                                      \end{cases}
                                     }
                                     {[[upgrade Γ ⊢ iP to Δ = iQ]]}
                                     {}
                              & \infer{\exists [[iQ]] \text{ s.t. }
                                [[upgrade Γ ⊢ iP to Δ = iQ]]
                                }{\exists \text{ sound } [[iQ']]}{}
                             & \infer
                                 {[[Δ ⊢ iQ' ≥ iQ]]}
                                 {
                                 \stackon
                                 {$[[upgrade Γ ⊢ iP to Δ = iQ]]$}
                                 {$[[iQ']]$  is sound}
                                 }{}
  \\



  \addlinespace[0.7em]
  % $[[ Γ ⊨ iP1 ∨ iP2 = iQ]]$
  LUB
                      & \infer
                                     { [[iQ]] \text{ is sound}
                                      \begin{cases}
                                        [[Γ ⊢ iQ]]\\
                                        [[Γ ⊢ iQ ≥ iP1]]\\
                                        [[Γ ⊢ iQ ≥ iP2]]
                                      \end{cases}
                                     }
                                     {[[Γ ⊨ iP1 ∨ iP2 = iQ]]}
                                     {}
                              & \infer{\exists [[iQ]] \text{ s.t. }
                                [[Γ ⊨ iP1 ∨ iP2 = iQ]]
                                }{\exists \text{ sound } [[iQ']]}{}
                             & \infer
                                 {[[Δ ⊢ iQ' ≥ iQ]]}
                                 {
                                 \stackon
                                 {$[[Γ ⊨ iP1 ∨ iP2 = iQ]]$}
                                 {$[[iQ']]$  is sound}
                                 }{}
  \\

  \addlinespace[0.7em]
  Anti-unification
                      & \infer
                                              { \text{\stackunder
                                              {$[[(Ξ, uQ, aus1, aus2)]]$}
                                              {is sound}}
                                      \begin{cases}
                                         [[Ξ]] \text{ is negative} \\
                                         [[Γ ; Ξ ⊢ uQ]] \\
                                         [[Γ ; · ⊢ ausi : Ξ]] \\
                                         [[ [ausi] uQ = iPi ]]
                                      \end{cases}
                                     }
                                     {[[G ⊨ iP1 ≈au iP2 ⫤ (Ξ, uQ, aus1, aus2)]]}
                                     {}
                              & \infer{\stackunder
                                {$\exists [[(Ξ, uQ, aus1, aus2)]] \text{ s.t. }$}
                                {$[[G ⊨ iP1 ≈au iP2 ⫤ (Ξ, uQ, aus1, aus2)]]$}
                                }{\exists \text{ sound } [[(Ξ', uQ', aus1', aus2')]]}{}
                             & \infer
                               {
                                 \exists [[Γ;Ξ ⊢ aus : Ξ']] \text{ s.t. } [[ [aus]uQ' = uQ ]] 
                               }
                               {
                               \stackon
                               {$[[G ⊨ iP1 ≈au iP2 ⫤ (Ξ, uQ, aus1, aus2)]]$}
                               {$[[(Ξ', uQ', aus1', aus2')]] \text{ is sound}$}
                               }
                               {}
  \\

  \addlinespace[0.7em]
  % $[[ Γ ⊨ iP1 ∨ iP2 = iQ]]$
  \stackunder{Unification}
  {(matching)}
                      &  
                      &
                      &  ---
  \\

  \addlinespace[0.7em]
  % $[[ Γ ⊨ iP1 ∨ iP2 = iQ]]$
  Subtyping
                     & 
                     & 
                     &  ---
  \\
  
\end{tabular}


\subsection{Variable Ordering}
\begin{observation}[Ordering is deterministic]
  \label{obs:ord-deterministic}
  If $[[ord varset in iN = ordVars1]]$ and $[[ord varset in iN = ordVars2]]$ then $[[ordVars1 = ordVars2]]$.
  If $[[ord varset in iP = ordVars1]]$ and $[[ord varset in iP = ordVars2]]$ then $[[ordVars1 = ordVars2]]$.
  This way, we can use $[[ord varset in iN]]$ and as a function on $[[iN]]$,
  and $[[ord varset in iP]]$ as a function on $[[iP]]$.
\end{observation}
\begin{proof}
  By mutual structural induction on $[[iN]]$ and $[[iP]]$.
  Notice that the shape of the term $[[iN]]$ or $[[iP]]$
  uniquely determines the last used inference rule,
  and all the premises are deterministic on the input.
\end{proof}


\begin{lemma}[Soundness of variable ordering]
  \label{lemma:ord-soundness}
  Variable ordering extracts used free variables.
  \begin{itemize}
    \item[$-$] $[[ {ord varset in iN} ]] = [[varset ∩ fv iN]]$ (as sets)
    \item[$+$] $[[ {ord varset in iP} ]] = [[varset ∩ fv iP]]$ (as sets)
  \end{itemize}
\end{lemma}
\begin{proof}
  We prove it by mutual induction on 
  $[[ ord varset in iN = ordVars ]]$ and $[[ ord varset in iP = ordVars ]]$.
  The only non-trivial cases are 
  \ruleref{\ottdruleOArrowLabel} and 
  \ruleref{\ottdruleOForallLabel}.  
  \begin{caseof}
    \item \ruleref{\ottdruleOArrowLabel}  
      Then the inferred ordering judgement has shape
      $[[ord varset in iP → iN = ordVars1, (ordVars2 \ {ordVars1})]]$
      and by inversion, 
      $[[ord varset in iP = ordVars1]]$   
      and 
      $[[ord varset in iN = ordVars2]]$.

      By definition of free variables, 
      $[[varset ∩ fv iP → iN = varset ∩ fv iP ∪ varset ∩ fv iN]]$,
      and since by the induction hypothesis 
      $[[varset ∩ fv iP = {ordVars1}]]$ and
      $[[varset ∩ fv iN = {ordVars2}]]$,
      we have
      $[[varset ∩ fv iP → iN = {ordVars1} ∪ {ordVars2}]]$.

      On the other hand, 
      As a set, $[[{ordVars1} ∪ {ordVars2}]]$
      is equal to $[[ordVars1, (ordVars2 \ {ordVars1})]]$. 
    \item  \ruleref{\ottdruleOForallLabel}.
      Then  the inferred ordering judgement has shape
      $[[ord varset in ∀pas.iN = ordVars]]$,
      and by inversion, 
      $[[varset ∩ {pas} = ∅]]$    
      $[[ord varset in iN = ordVars]]$.
      The latter implies that $[[varset ∩ fv iN = {ordVars}]]$.
      We need to show that $[[varset ∩ fv ∀pas.iN = {ordVars}]]$,
      or equivalently, that
      $[[varset ∩ (fv iN \ {pas}) = varset ∩ fv iN ]]$,
      which holds since $[[varset ∩ {pas} = ∅]]$.
\end{proof}


\begin{corollary}[Additivity of ordering]
  \label{corollary:ord-additivity}
  Variable ordering is additive (in terms of set union) with respect to its first argument.
  \begin{itemize}
    \item[$-$] $[[ {ord (varset1 ∪ varset2) in iN} 
                = 
                {ord varset1 in iN} ∪ {ord varset2 in iN}]]$ (as sets)
    \item[$+$] $[[{ord (varset1 ∪ varset2) in iP}
                =
                {ord varset1 in iP} ∪ {ord varset2 in iP}]]$ (as sets)

  \end{itemize}
\end{corollary}

\begin{lemma}[Weakening of ordering]
  \label{corollary:ord-weakening}
  Only used variables matter in the first argument of the ordering,
  \begin{itemize}
    \item[$-$] $[[ ord (varset ∩ fv iN) in iN ]] = [[ ord varset in iN ]]$
    \item[$+$] $[[ ord (varset ∩ fv iP) in iP ]] = [[ ord varset in iP ]]$
  \end{itemize}
\end{lemma}
\begin{proof}
  Mutual structural induction on $[[iN]]$ and $[[iP]]$.

  \begin{caseof}
    \item If $[[iN]]$ is a variable $[[na]]$,
      we notice that $[[na ∊ varset]]$ 
      is equivalent to $[[na ∊ varset ∩ {na}]]$.
    \item If $[[iN]]$ has shape $[[↑iP]]$, then
      the required property holds immediately by the 
      induction hypothesis, since 
      $[[fv(↑iP) = fv(iP)]]$.
    \item If the term has shape $[[iP → iN]]$ then
      \ruleref{\ottdruleOArrowLabel} was applied
      to infer $[[ ord (varset ∩ (fv iP ∪ fv iN)) in iP → iN ]]$
      and $[[ ord varset in iP → iN]]$. 
      By inversion, the result of 
      $[[ ord (varset ∩ (fv iP ∪ fv iN)) in iP → iN ]]$
      depends on 
      $A = [[ ord (varset ∩ (fv iP ∪ fv iN)) in iP]]$
      and 
      $B = [[ ord (varset ∩ (fv iP ∪ fv iN)) in iN]]$.
      The result of
       $[[ ord varset in iP → iN]]$ 
       depends on 
      $X = [[ord varset in iP]]$ and
      $Y = [[ord varset in iN]]$.

      Let us show that that $A = B$ and $X = Y$, so the results are equal. 
      By the induction hypothesis and set properties,
      $[[ ord (varset ∩ (fv iP ∪ fv iN)) in iP ]] = 
       [[ ord (varset ∩ (fv iP ∪ fv iN)) ∩ fv(iP) in iP ]] = 
       [[ ord varset ∩ fv(iP) in iP ]] = 
       [[ ord varset in iP ]]$.
      Analogously, 
      $[[ ord (varset ∩ (fv iP ∪ fv iN)) in iN ]] = 
       [[ ord varset in iN ]]$.
    \item If the term has shape $[[∀pas.iN]]$,
      we can assume that $[[pas]]$ is disjoint
      from $[[varset]]$,
      since we operate on alpha-equivalence classes.
      Then using the induction hypothesis,
      set properties and \ruleref{\ottdruleOForallLabel}: 
      $[[ord varset ∩ (fv(∀pas.iN)) in ∀pas.iN]] =
       [[ord varset ∩ (fv(iN) \ {pas}) in iN]] =
       [[ord varset ∩ (fv(iN) \ {pas}) ∩ fv(iN) in iN]] =
       [[ord varset ∩ fv(iN) in iN]] =
       [[ord varset in iN]]$.
  \end{caseof}
\end{proof}

\begin{corollary}[Idempotency of ordering]
  \label{corollary:ord-idemp}
  \hfill
  \begin{itemize}
    \item[$-$] If $[[ ord varset in iN = ordVars ]]$ then 
      $[[ ord {ordVars} in iN = ordVars ]]$,
    \item[$+$] If $[[ ord varset in iP = ordVars ]]$ then 
      $[[ ord {ordVars} in iP = ordVars ]]$;
  \end{itemize}
\end{corollary}
\begin{proof}
  By \cref{lemma:ord-soundness,corollary:ord-weakening}.
\end{proof}
  

Next we make a set-theoretical observation
that will be useful further.
In general, any injective function (its image)
distributes over set intersection.
However, for convenience we allow the bijections
on variables to be applied
\emph{outside of their domains}
(as identities), which may violate
the injectivity. To deal with these cases, 
we define a special notion of
bijections collision-free on certain sets
in such a way that
a bijection that is collision-free on $P$ and $Q$,
distributes over intersection of $P$ and $Q$.

\begin{definition} [Collision free bijection]
  We say that a bijection $\mu : A \leftrightarrow B$ between sets of
  variables is \textbf{collision free on sets} $P$ and $Q$ if and only if
  \begin{enumerate}
    \item $\mu(P \cap A) \cap Q = \emptyset$
    \item $\mu(Q \cap A) \cap P = \emptyset$
  \end{enumerate}
\end{definition}

\begin{observation}
  Suppose that $\mu : A \leftrightarrow B$ is a bijection between two sets of variables,
  and $\mu$ is collision free on $P$ and $Q$.
  Then $\mu(P \cap Q) = \mu(P) \cap \mu(Q)$.
\end{observation}
  

\begin{lemma}[Distributivity of renaming over variable ordering]
  \label{lemma:distr-mu-ord}
  Suppose that $\mu$ is a bijection between two sets of variables
  $\mu : A \leftrightarrow B$.
  
  \begin{itemize}
  \item[$-$]
    If $[[mu]]$ is collision free on $[[varset]]$ and $[[fv iN]]$ then
    $[[ [mu] (ord varset in iN) ]] = [[ord ([mu] varset) in [mu] iN ]]$
  \item[$+$]
    If $[[mu]]$ is collision free on $[[varset]]$ and $[[fv iP]]$ then
    $[[ [mu] (ord varset in iP) ]] = [[ord ([mu] varset) in [mu] iP ]]$
  \end{itemize}
\end{lemma}

\begin{proof}
  Mutual induction on $[[iN]]$ and $[[iP]]$.
  \begin{caseof}
  \item $[[iN]]$ = $[[na]]$ \label{case:distr-mu-ord:var} \\
    let us consider four cases:
    \begin{caseof}
    \item $[[na]] \in A$ and $[[na]] \in [[varset]]$\\ Then
      $
      \begin{aligned}[t] [[ [mu] (ord varset in iN) ]] &= [[ [mu] (ord varset in na)]] \\
                                                             &= [[ [mu] na ]]
                                                             && \text{by \ruleref{\ottdruleOPVarInLabel}}\\
                                                             &= [[nb]]
                                                             && \text{for some $[[nb]] \in B$ (notice that $[[nb]] \in [[ [mu]varset ]]$)} \\
                                                             &= [[ ord [mu]varset in nb ]]
                                                             && \text{by \ruleref{\ottdruleOPVarInLabel},
                                                                because $[[nb]] \in [[ [mu]varset ]]$} \\
                                                             &= [[ord [mu] varset in [mu] na ]]
       \end{aligned}
       $
     \item $[[na]] \notin A$ and $[[na]] \notin [[varset]]$\\
       Notice that
       $[[ [mu] (ord varset in iN) ]] = [[ [mu] (ord varset in na)]] = [[·]]$ by
       \ruleref{\ottdruleOPVarNInLabel}.
       On the other hand, $[[ ord [mu] varset in [mu] na = ord [mu] varset
       in na ]] = [[·]]$ The latter equality is from
       \ruleref{\ottdruleOPVarNInLabel}, because
       $[[mu]]$ is collision free on $[[varset]]$ and $[[fv iN]]$, so
       $[[fv iN]] \ni [[na]] \notin [[mu]](A \cap [[varset]]) \cup
       [[varset]] \supseteq [[ [mu] varset ]]$.
     \item $[[na]] \in A$ but $[[na]] \notin [[varset]]$\\ Then
       $[[ [mu] (ord varset in iN) ]] = [[ [mu] (ord varset in na)]] = [[·]]$
       by \ruleref{\ottdruleOPVarNInLabel}.
       To prove that $[[ ord [mu] varset in [mu] na ]] = [[·]]$, we apply
       \ruleref{\ottdruleOPVarNInLabel}. Let us show that
       $[[ [mu] na ]] \notin [[ [mu] varset ]]$.
       Since $[[ [mu] na ]] = [[mu]]([[na]])$ and
       $[[ [mu] varset ]] \subseteq [[mu]](A \cap [[varset]]) \cup [[varset]]$,
       it suffices to prove 
       $[[mu]]([[na]]) \notin [[mu]](A \cap [[varset]]) \cup [[varset]]$.

       \begin{enumerate}
       \item[(i)] If there is an element $x \in A \cap [[varset]]$ such that
         $[[mu]] x = [[mu]] [[na]]$, then $x = [[na]]$ by bijectivity of
         $[[mu]]$, which contradicts with $[[na]] \notin [[varset]]$. This way, 
         $[[mu]]([[na]]) \notin [[mu]](A \cap [[varset]])$.
       \item[(ii)]
         Since $[[mu]]$ is collision free on $[[varset]]$ and $[[fv iN]]$,
         $[[mu]] (A \cap [[fv iN]]) \ni [[mu]]([[na]]) \notin [[varset]]$.
       \end{enumerate}

     \item $[[na]] \notin A$ but $[[na]] \in [[varset]]$\\
       $[[ ord [mu] varset in [mu] na ]] = [[ ord [mu] varset in na ]] = [[na]]$.
       The latter is by \ruleref{\ottdruleOPVarNInLabel}, because
       $[[na]] = [[ [mu] na ]] \in [[ [mu] varset ]]$ since $[[na]] \in [[varset]]$.
       On the other hand, $[[ [mu] (ord varset in iN) ]] = [[ [mu] (ord varset in na)]] = [[ [mu] na ]] = [[na]]$.
    \end{caseof}
  
  \item $[[iN]] = [[↑iP]]$ \\
    $\begin{aligned}[t]
       [[ [mu] (ord varset in iN) ]] &= [[ [mu] (ord varset in ↑iP) ]] \\
                                     &= [[ [mu] (ord varset in iP) ]]
                                     && \text{by \ruleref{\ottdruleOShiftULabel}}\\
                                     &= [[ ord [mu]varset in [mu]iP ]]
                                     && \text{by the induction hypothesis}\\
                                     &= [[ ord [mu]varset in  ↑[mu]iP ]]
                                     && \text{by \ruleref{\ottdruleOShiftULabel}}\\
                                     &= [[ ord [mu]varset in  [mu]↑iP ]]
                                     && \text{by the definition of substitution}\\
                                     &= [[ ord [mu]varset in  [mu]iN ]]
            \end{aligned}$
          
   \item $[[iN]] = [[iP → iM]]$  \\
     $\begin{aligned}[t]
        [[ [mu] (ord varset in iN) ]] &= [[ [mu] (ord varset in iP → iM) ]] \\
                                      &= [[ [mu] (ordVars1, (ordVars2 \ {ordVars1})) ]]
                                      && \text{where } [[ord varset in iP = ordVars1]] \text{ and } [[ord varset in iM = ordVars2]] \\
                                      &= [[ [mu] ordVars1, [mu](ordVars2 \ {ordVars1}) ]] \\
                                      &= [[ [mu] ordVars1, ([mu]ordVars2 \ [mu]{ordVars1}) ]]
                                      && \text{by induction on $[[ordVars2]]$;
                                         the inductive step is similar to \cref{case:distr-mu-ord:var}.
                                         Notice that $[[mu]]$ is} \\
                                      & && \text{collision free on $[[{ordVars1}]]$ and $[[{ordVars2}]]$
                                           since
                                           $[[{ordVars1}]] \subseteq [[varset]]$ and
                                           $[[{ordVars2}]] \subseteq [[fv iN]]$ }\\
                                      &= [[ [mu] ordVars1, ([mu]ordVars2 \ {[mu]ordVars1}) ]]
      \end{aligned}$ \\
    $\begin{aligned}[t]
       [[  (ord [mu] varset in [mu]iN) ]] &= [[ (ord [mu] varset in [mu]iP → [mu]iM) ]] \\
                                     &= [[ (ordVarsb1, (ordVarsb2 \ {ordVarsb1})) ]]
                                     && \text{where } [[ord [mu] varset in [mu] iP = ordVarsb1]] \text{ and } [[ord [mu] varset in [mu] iM = ordVarsb2]] \\
                                          & && \text{then by the induction
                                               hypothesis,
                                               $[[ordVarsb1]] = [[ [mu] ordVars1 ]]$,
                                               $[[ordVarsb2]] = [[ [mu] ordVars2 ]]$,
                                               }\\
                                     &= [[ [mu] ordVars1, ([mu]ordVars2 \ {[mu]ordVars1}) ]]
     \end{aligned}$
   
   \item $[[iN]] = [[∀ pas.iM]]$ \\
     $\begin{aligned}[t]
          [[ [mu] (ord varset in iN) ]] &= [[ [mu] ord varset in ∀pas.iM]] \\
                                        &= [[ [mu] ord varset in iM]] \\
                                        &= [[ ord [mu] varset in [mu] iM]]
                                        && \text {by the induction hypothesis}\\
     \end{aligned}$ \\
     $ 
     \begin{aligned}[t]
       [[ (ord [mu] varset in [mu] iN) ]] &= [[ ord [mu] varset in [mu] ∀pas.iM ]] \\
                                          &= [[ ord [mu] varset in ∀pas.[mu]iM ]] \\
                                          &= [[ ord [mu] varset in [mu] iM ]] \\
     \end{aligned}
     $
     
  \end{caseof}
\end{proof}

\begin{lemma}[Ordering is not affected by independent substitutions]
  \label{lemma:ord-sigma}
  Suppose that $[[Γ2 ⊢ σ : Γ1]]$, i.e. $[[σ]]$ maps variables from $[[Γ1]]$ into types
  taking free variables from $[[Γ2]]$, and $[[varset]]$ is a set of variables
  disjoint with both $[[Γ1]]$ and $[[Γ2]]$, 
  $[[iN]]$ and $[[iP]]$ are types. Then
    \begin{itemize}
  \item[$-$] $[[ ord varset in [σ]iN ]] = [[ord varset in iN ]]$
  \item[$+$] $[[ ord varset in [σ]iP ]] = [[ord varset in iP ]]$
  \end{itemize}
\end{lemma}
\begin{proof}
  Mutual induction on $[[iN]]$ and $[[iP]]$.
  \begin{caseof}
    \item $[[iN = na]]$ \\
      If $[[na ∉ Γ1]]$ then $[[ [σ]na = na ]]$ and $[[ ord varset in [σ]na ]] = [[ ord varset in na ]]$, 
      as requried.
      If $[[na ∊ Γ1]]$ then $[[na ∉ varset]]$, so $[[ ord varset in na ]] = [[·]]$.
      Moreover, $[[Γ2 ⊢ σ : Γ1]]$ means $[[ fv([σ]na) ⊆ Γ2 ]]$, and thus, 
      as a set, $[[ ord varset in [σ]na ]] = [[varset ∩ fv([σ]na)]] \subseteq [[varset ∩ Γ2]] = [[·]]$.
    \item $[[iN = ∀pas.iM]]$\\
      We can assume $[[{pas} ∩ Γ1 = ∅]]$
      and $[[{pas} ∩ varset = ∅]]$. Then 

      $\begin{aligned}[t]
         [[ ord varset in [σ]iN ]] &= [[ ord varset in [σ]∀pas.iM ]] \\
                                   &= [[ ord varset in ∀pas.[σ]iM ]]\\
                                   &= [[ ord varset in [σ]iM ]]
                                   && \text{by the induction hypothesis}\\
                                   &= [[ ord varset in iM ]]\\
                                   &= [[ ord varset in ∀pas.iM ]]\\
                                   &= [[ ord varset in iN ]]
       \end{aligned}$
    \item $[[iN = ↑iP]]$\\
       $\begin{aligned}[t]
        [[ ord varset in [σ]iN ]] &= [[ ord varset in [σ]↑iP ]] \\
                                   &= [[ ord varset in ↑[σ]iP ]]
                                   && \text{by the definition of substitution}\\
                                   &= [[ ord varset in [σ]iP ]]
                                   && \text{by the induction hypothesis}\\
                                   &= [[ ord varset in iP ]]
                                   && \text{by the definition of substitution}\\
                                   &= [[ ord varset in ↑iP ]]
                                   && \text{by the definition of ordering}\\
                                   &= [[ ord varset in iN ]]
       \end{aligned} $

    \item $[[iN = iP → iM]]$\\
       $ \begin{aligned}
        [[ ord varset in [σ]iN ]] &= [[ ord varset in [σ](iP → iM) ]] \\
                                   &= [[ ord varset in ([σ]iP → [σ]iM) ]]
                                   && \text{by the definition of substitution}\\
                                   &= [[ ord varset in [σ]iP, (ord varset in [σ]iM \ {ord varset in [σ]iP}) ]]
                                   && \text{by the definition of ordering}\\
                                   &= [[ ord varset in iP, (ord varset in iM \ {ord varset in iP}) ]]
                                   && \text{by the induction hypothesis}\\
                                   &= [[ ord varset in iP → iM ]]
                                   && \text{by the definition of ordering}\\
                                   &= [[ ord varset in iN ]]
       \end{aligned} $
    \item The proofs of the positive cases are symmetric.
  \end{caseof}
\end{proof}

\begin{lemma}[Completeness of variable ordering]
  \label{lemma:ord-completeness}
  Variable ordering is invariant under equivalence. For arbitrary $[[varset]]$,
   \begin{itemize}
  \item[$-$] If $[[iN ≈ iM]]$ then $[[ord varset in iN]] = [[ord varset in iM]]$ (as lists)
  \item[$+$] If $[[iP ≈ iQ]]$ then $[[ord varset in iP]] = [[ord varset in iQ]]$ (as lists)
  \end{itemize}
\end{lemma}
\begin{proof}
  Mutual induction on $[[iN ≈ iM]]$ and $[[iP ≈ iQ]]$.
  Let us consider the rule inferring $[[iN ≈ iM]]$. 
  \begin{caseof}
    \item \ruleref{\ottdruleEOneNVarLabel}
    \item \ruleref{\ottdruleEOneShiftULabel}
    \item \ruleref{\ottdruleEOneArrowLabel}
      Then the equivalence has shape $[[iP → iN ≈ iQ → iM]]$,
      and by inversion, $[[iP ≈ iQ]]$ and $[[iN ≈ iM]]$.
      They by the induction hypothesis,
      $[[ord varset in iP]] = [[ord varset in iQ]]$ 
      and $[[ord varset in iN]] = [[ord varset in iM]]$.
      Since the resulting ordering for $[[iP → iN]]$ and $[[iQ → iM]]$
      depend on the ordering of the corresponding components, 
      which are equal, the results are equal.
    \item \ruleref{\ottdruleEOneForallLabel}
      Then the equivalence has shape $[[∀pas.iN ≈ ∀pbs.iM]]$.
      and by inversion there exists 
      $[[mu : ({pbs} ∩ fv iM) ↔ ({pas} ∩ fv iN)]]$ such that
      \begin{itemize}
        \item $[[{pas} ∩ fv iM = ∅]]$ and 
        \item $[[iN ≈ [mu] iM]]$
      \end{itemize}

      Let us assume that $[[varset]]$ is disjoint from 
      $[[pas]]$ and $[[pbs]]$ 
      (we can always alpha-rename the bound variables).
      Then $[[ord varset in ∀pas.iN = ord varset in iN]]$, 
      $[[ord varset in ∀pas.iM = ord varset in iM]]$
      and by the induction hypothesis,
      $[[ord varset in iN]] = [[ord varset in [mu]iM]]$.
      This way, it suffices tho show  that 
      $[[ord varset in [mu]iM = ord varset in iM]]$.
      It holds by \cref{lemma:ord-sigma} since
      $[[varset]]$ is disjoint form 
      the domain and the codomain of 
      $[[mu : ({pbs} ∩ fv iM) ↔ ({pas} ∩ fv iN)]]$ 
      by assumption.

    \item The positive cases are proved symmetrically.
  \end{caseof}
\end{proof}


\subsection{Normaliztaion}
\begin{lemma}
  \label{lemma:equiv-fv}
  Set of free variables is invariant under equivalence.
  \begin{itemize}
  \item[$-$] If $[[iN ≈ iM]]$ then $[[fv iN]] \equiv [[fv iM]]$ (as sets)
  \item[$+$] If $[[iP ≈ iQ]]$ then $[[fv iP]] \equiv [[fv iQ]]$ (as sets)
  \end{itemize}
\end{lemma}
\begin{proof}
  Straightforward mutual induction on $[[iN ≈ iM]]$ and $[[iP ≈ iQ]]$
\end{proof}


\begin{lemma}
  \label{lemma:fv-nf}
  Free variables are not changed by the normalization
  \begin{itemize}
  \item[$-$] $[[fv iN]] \equiv [[fv nf(iN)]]$
  \item[$+$] $[[fv iP]] \equiv [[fv nf(iP)]]$
  \end{itemize}
\end{lemma}
\begin{proof}
  By straightforward induction on $[[iN]]$ and mutually on $[[iP]]$.
\end{proof}

\begin{lemma}
  \label{lemma:uv-nf}
  Algorithmic variables are not changed by the normalization
  \begin{itemize}
  \item[$-$] $[[uv uN]] \equiv [[uv nf(uN)]]$
  \item[$+$] $[[uv uP]] \equiv [[uv nf(uP)]]$
  \end{itemize}
\end{lemma}
\begin{proof}
  By straightforward induction on $[[uN]]$ and mutually on $[[uP]]$.
\end{proof}


\begin{lemma}[Soundness of normalization]
  \label{lemma:normalization-soundness}
  \hfill
  \begin{itemize}
    \item[$-$] $[[iN ≈ nf(iN)]]$
    \item[$+$] $[[iP ≈ nf(iP)]]$
  \end{itemize}
\end{lemma}
\begin{proof}
  Mutual induction on $[[nf(iN) = iM]]$ and $[[nf(iP) = iQ]]$.
  Let us consider how this judgment is formed:
  \begin{caseof}
    \item{\nameref{\ottdruleNrmNVarLabel} and \nameref{\ottdruleNrmPVarLabel}}\\ By
      the corresponding equivalence rules.
    \item{\nameref{\ottdruleNrmShiftULabel}, \nameref{\ottdruleNrmShiftDLabel},
        and \nameref{\ottdruleNrmArrowLabel}}\\
      By the induction hypothesis and the corresponding congruent equivalence rules.
    \item{\nameref{\ottdruleNrmForallLabel}}, i.e. $[[nf(∀pas.uN) = ∀pas'.uN']]$ \label{case:norm-soundness:forall}\\
      From the induction hypothesis, we
      know that $[[iN ≈ iN']]$. In particular, by \cref{lemma:equiv-fv}, $[[fv
        iN]] \equiv [[fv iN']]$.
      Then by \cref{lemma:ord-soundness}, $[[{pas'}]]
      \equiv [[{pas} ∩ fv iN']] \equiv [[{pas} ∩ fv iN]]$, and thus,
      $[[{pas'} ∩ fv iN']] \equiv [[{pas} ∩ fv iN]]$.
      
      To prove $[[∀pas.iN ≈ ∀pas'.iN']]$, it suffices to provide a bijection 
      $\mu : [[{pas'} ∩ fv iN']] \leftrightarrow [[{pas} ∩ fv iN]]$ such that
      $[[iN ≈ [mu]iN']]$. Since these sets are equal, we take $\mu = id$.
    \item{\nameref{\ottdruleNrmExistsLabel}} Same as for \cref{case:norm-soundness:forall}.
  \end{caseof}
\end{proof}

\begin{lemma}[Soundness of normalization of algorithmic types]
  \label{lemma:normalization-soundness-alg}
  \hfill
  \begin{itemize}
    \item[$-$] $[[uN ≈ nf(uN)]]$
    \item[$+$] $[[uP ≈ nf(uP)]]$
  \end{itemize}
\end{lemma}
\begin{proof}
  The proof coincides with the proof of \cref{lemma:normalization-soundness}.
\end{proof}


\begin{corollary}[Normalization preserves ordering]
  \label{corollary:normalization-ord}
  For any $[[varset]]$,
  \begin{itemize}
  \item[$-$] $[[ord varset in nf(uN)]] = [[ord varset in uM]]$
  \item[$+$] $[[ord varset in nf(uP)]] = [[ord varset in uQ]]$
  \end{itemize}
\end{corollary}
\begin{proof}
  Immediately from \cref{lemma:ord-completeness,lemma:normalization-soundness}.
\end{proof}

\begin{lemma}[Distributivity of normalization over substitution]
  \label{lemma:norm-subst-distr} Normalization of a term distributes over substitution.
  Suppose that $[[σ]]$ is a substitution, $[[iN]]$ and $[[iP]]$ are types. Then
    \begin{itemize}
      \item[$-$] $[[nf([σ]iN)]] = [[ [nf(σ)] nf(iN) ]]$
      \item[$+$] $[[nf([σ]iP)]] = [[ [nf(σ)] nf(iP) ]]$
  \end{itemize}
  where $[[nf(σ)]]$ means pointwise normalization: $[[ [nf(σ)] α⁻]] = [[nf([σ]α⁻)]]$.
\end{lemma}
\begin{proof}
  Mutual induction on $[[iN]]$ and $[[iP]]$.
  \begin{caseof}
    \item $[[iN]]$ = $[[na]]$ \\
      \label{case:norm-subst-distr-var}
      $[[nf([σ]iN)]] = [[ nf([σ]na) ]] = [[ [nf(σ)]na ]] $.

      $[[ [nf(σ)] nf(iN) ]] = [[ [nf(σ)] nf(na) ]] = [[ [nf(σ)] na ]] $.
    \item $[[iP]]$ = $[[pa]]$ \\
      Similar to \cref{case:norm-subst-distr-var}.
   \item If the type is formed by $[[→]]$, $[[↑]]$, or $[[↓]]$, 
     the required equality follows from the congruence of the normalization and
     substitution, and the induction hypothesis.
     For example, if $[[iN]] = [[iP → iM]]$ then \\
     $\begin{aligned}[t]
        [[nf([σ] iN)]] &= [[ nf([σ] (iP → iM)) ]] \\
                        &= [[ nf([σ]iP → [σ]iM) ]]
                        && \text{By the congruence of substitution} \\
                        &= [[ nf([σ]iP) → nf([σ]iM) ]]
                        && \text{By the congruence of normalization, i.e. \ruleref{\ottdruleNrmArrowLabel}} \\
                        &= [[ [nf(σ)]nf(iP) → [nf(σ)]nf(iM) ]]
                        && \text{By the induction hypothesis} \\
                        &= [[ [nf(σ)](nf(iP) → nf(iM)) ]]
                        && \text{By the congruence of substitution} \\
                        &= [[ [nf(σ)]nf(iP → iM) ]]
                        && \text{By the congruence of normalization} \\
                        &= [[ [nf(σ)]nf(iN) ]]
      \end{aligned}$ \\
    \item $[[iN]] = [[∀ pas.iM]]$ \label{case:norm-subst-commute} \\
      $\begin{aligned}[t]
          [[ [nf(σ)] nf(iN) ]] &= [[ [nf(σ)] nf(∀pas.iM)]] \\
                            &= [[ [nf(σ)] ∀pas'.nf(iM) ]]
                            && \text {Where $[[pas']] = [[ ord {pas} in nf(iM)]]
                               = [[ord {pas} in iM]]$
                               (the latter is by
                               \cref{corollary:normalization-ord})}\\
        \end{aligned}$ \\

      $\begin{aligned}[t]
         [[ nf([σ]iN) ]] &= [[ nf([σ] ∀pas.iM)]] \\
                          &= [[ nf(∀pas.[σ]iM) ]]
                          && \text{Assuming $[[{pas} ∩ Γ1]] = \emptyset$
                             and $[[{pas} ∩ Γ2]] = \emptyset$}\\
                          &= [[ ∀pbs.nf([σ]iM) ]]
                          && \text {Where $[[pbs]] = [[ord {pas} in nf([σ]iM)]]
                             = [[ord {pas} in [σ]iM]]$ (the latter is by
                             \cref{corollary:normalization-ord})}\\
                          &= [[ ∀pas'.nf([σ]iM) ]]
                          && \text{By \cref{lemma:ord-sigma}, $[[pbs]] = [[pas']]$
                             since $[[{pas}]]$ is disjoint with $[[Γ1]]$ and
                             $[[Γ2]]$}\\
                          &= [[ ∀pas'.[nf(σ)]nf(iM) ]]
                          && \text {By the induction hypothesis}\\
         \end{aligned}$ \\

     To show alpha-equivalence of 
     $[[ [nf(σ)] ∀pas'.nf(iM) ]]$ and $[[ ∀pas'.[nf(σ)]nf(iM) ]]$,
     we can assume that $[[{pas'} ∩ Γ1]] = \emptyset$, and $[[{pas'} ∩ Γ2]]
     = \emptyset$.

   \item $[[iP]] = [[∃ nas.iQ]]$ \\
     Same as for \cref{case:norm-subst-commute}.
  \end{caseof}
\end{proof}


\begin{corollary}[Commutativity of normalization and renaming]
  \label{lemma:norm-subst-commute} Normalization of a term commutes with renaming.
  Suppose that $\mu$ is a bijection between two sets of variables
  $\mu : A \leftrightarrow B$. Then
  \begin{itemize}
    \item[$-$] $[[nf([mu]iN)]] = [[ [mu] nf(iN) ]]$
    \item[$+$] $[[nf([mu]iP)]] = [[ [mu] nf(iP) ]]$
  \end{itemize}
\end{corollary}
\begin{proof}
  Immediately from \cref{lemma:norm-subst-distr},
  after noticing that $[[nf(mu)]] = [[mu]]$.
\end{proof}




\begin{lemma}[Completeness of quantified normalization]
  \label{lemma:normalization-completeness}
  Normalization returns the same representative for equivalent types.

  \begin{itemize}
  \item[$-$] If $[[iN ≈ iM]]$ then $[[nf(iN)]] = [[nf(iM)]]$
  \item[$+$] If $[[iP ≈ iQ]]$ then $[[nf(iP)]] = [[nf(iQ)]]$
  \end{itemize}
  (Here equality means alpha-equivalence)
\end{lemma}

\begin{proof}
  Mutual induction on $[[iN ≈ iM]]$ and $[[iP ≈ iQ]]$.
  \begin{caseof}
  \item {\nameref{\ottdruleEOneForallLabel}} \label{case:ord-completeness:forall} \\

    From the definition of the normalization,
    \begin{itemize}
      \item $[[nf(∀pas.iN)]] = [[∀pas'.nf(iN)]]$ where $[[pas']]$ is $[[ord {pas} in nf(iN)]]$
      \item $[[nf(∀pbs.iM)]] = [[∀pbs'.nf(iM)]]$ where $[[pbs']]$ is $[[ord {pbs} in nf(iM)]]$
    \end{itemize}
    Let us take $[[mu : ({pbs} ∩ fv iM) ↔ ({pas} ∩ fv iN)]]$ from the
    inversion of the equivalence judgment. Notice that from
    \cref{lemma:fv-nf,lemma:ord-soundness}, the domain and the codomain of $\mu$ can be written
    as $[[mu : {pbs'} ↔ {pas'}]]$.
    
    To show the alpha-equivalence of $[[∀pas'.nf(iN)]]$ and $[[∀pbs'.nf(iM)]]$,
    it suffices to prove that
    \begin{enumerate*}
    \item[(i)] $[[ [mu] nf(iM) ]] = [[nf(iN)]]$ and \newline
    \item[(ii)] $[[ [mu]pbs' ]] = [[pas']]$
    \end{enumerate*}.
    
    \begin{enumerate}
    \item[(i)] $[[ [mu] nf(iM) ]] = [[nf([mu]iM)]] = [[nf(iN)]]$.
      The first equality holds by \cref{lemma:norm-subst-commute}, the second---by the induction hypothesis.

    \item[(ii)] $\begin{aligned}[t] [[ [mu]pbs' ]] &= [[ [mu] ord {pbs} in nf(iM) ]]
                                                  && \text{by the definition of $[[pbs']]$ } \\
                                                  &= [[ [mu] ord ({pbs} ∩ fv iM) in nf(iM) ]]
                                                  && \text{from \cref{lemma:fv-nf,corollary:ord-weakening} } \\
                                                  &= [[ ord [mu] ({pbs} ∩ fv iM) in [mu] nf(iM) ]]
                                                  && \text{by \cref{lemma:distr-mu-ord}, because
                                                     $[[{pas} ∩ fv iN]] \cap [[fv nf(iM)]] \subseteq [[{pas} ∩ fv iM ]]
                                                     = \emptyset$}\\
                                                  &
                                                  && \text{and $[[{pas} ∩ fv iN]] \cap [[({pbs} ∩ fv iM)]] \subseteq
                                                     [[{pas} ∩ fv iM]] = \emptyset$} \\
                                                  &= [[ ord [mu] ({pbs} ∩ fv iM) in nf(iN) ]]
                                                  && \text{since $[[ [mu] nf(iM) ]] = [[nf(iN)]]$ is proved } \\
                                                  &= [[ ord ({pas} ∩ fv iN) in nf(iN) ]]
                                                  && \text{because $\mu$ is a bijection between
                                                     $[[{pas} ∩ fv iN]]$ and $[[{pbs} ∩ fv iM]]$} \\
                                                  &= [[ ord {pas} in nf(iN) ]]
                                                  && \text{from \cref{lemma:fv-nf,corollary:ord-weakening} } \\
                                                  &= [[ pas' ]]
                                                  && \text{by the definition of $[[pas']]$} \\
      \end{aligned}$
    \end{enumerate}
  \item {\nameref{\ottdruleEOneExistsLabel}} Same as for \cref{case:ord-completeness:forall}.
  \item Other rules are congruent, and thus, proved by the corresponding congruent alpha-equivalence rule,
    which is applicable by the induction hypothesis. 
  \end{caseof}
\end{proof}


\begin{lemma}[Idempotence of normalization]
  \label{lemma:norm-idemp}
  Normalization is idempotent
  \begin{itemize}
  \item[$-$] $[[nf(nf(iN))]] = [[nf(iN)]]$
  \item[$+$] $[[nf(nf(iP))]] = [[nf(iP)]]$
  \end{itemize}
\end{lemma}
\begin{proof}
  By applying \cref{lemma:normalization-completeness} to \cref{lemma:normalization-soundness}.
\end{proof}


\begin{lemma}
  \label{lemma:normal-after-subst}
  The result of a substitution is normalized if and only if the initial type and
  the substitution are normalized.

  Suppose that $[[σ]]$ is a substitution  $[[Γ2 ⊢ σ : Γ1]]$,
  $[[iP]]$ is a positive type ($[[Γ1 ⊢ iP]]$),
  $[[iN]]$ is a negative type ($[[Γ1 ⊢ iN]]$). Then
  \begin{itemize}
  \item[$+$]
    $[[ [σ]iP  ]] \text{ is normal} \iff
    \begin{cases}
      [[σ|fv(iP)]] &\text{is normal}\\
      [[iP]]       &\text{is normal}\\
    \end{cases} $
  \item[$-$]
    $[[ [σ]iN  ]] \text{is normal} \iff
    \begin{cases}
      [[σ|fv(iN)]] &\text{is normal}\\
      [[iN]]       &\text{is normal}\\
    \end{cases} $
  \end{itemize}
\end{lemma}
\begin{proof}
  Mutual induction on $[[Γ1 ⊢ iP]]$ and $[[Γ1 ⊢ iN]]$.
  \begin{caseof}
  \item $[[iN]] = [[na]]$\\
    Then $[[iN]]$ is always normal, and
    the normality of $[[σ|{na}]]$ by the definition means $[[ [σ]na ]]$ is normal.

  \item $[[iN]] = [[iP → iM]]$\\
    \label{case:normal-after-subst-arrow}
    $
    \begin{aligned}[t]
      [[ [σ](iP → iM) ]] \text{ is normal} &\iff [[ [σ]iP → [σ]iM ]] \text{ is normal}
                                           && \text{by the substitution
                                              congruence} \\
                                           &\iff
                                             \begin{cases}
                                             [[ [σ]iP ]] &\text{is normal} \\
                                             [[ [σ]iM ]] &\text{is normal} \\
                                             \end{cases}\\
                                           &\iff
                                             \begin{cases}
                                               [[ iP ]]       &\text{is normal} \\
                                               [[ σ|fv(iP) ]] &\text{is normal} \\
                                               [[ iM ]]       &\text{is normal} \\
                                               [[ σ|fv(iM) ]] &\text{is normal} \\
                                             \end{cases}
                                           && \text{by the induction hypothesis}\\
                                           &\iff
                                             \begin{cases}
                                               [[ iP → iM ]]  &\text{is normal} \\
                                               [[ σ|fv(iP) ∪ fv(iM)]] &\text{is normal} \\
                                             \end{cases}\\
                                           &\iff
                                             \begin{cases}
                                               [[ iP → iM ]]  &\text{is normal} \\
                                               [[ σ|fv(iP→iM)]] &\text{is normal} \\
                                             \end{cases}
    \end{aligned}
    $
  \item $[[iN]] = [[↑iP]]$\\
    By congruence and the inductive hypothesis, similar to \cref{case:normal-after-subst-arrow}
  \item $[[iN]] = [[∀pas.iM]]$\\
    $
    \begin{aligned}[t]
      [[ [σ](∀pas.iM) ]] \text{ is normal} &\iff [[ (∀pas.[σ]iM) ]] \text{ is normal}
                                           && \text{assuming $[[pas]] \cap [[Γ1]] = \emptyset$ and
                                              $[[pas]] \cap [[Γ2]] = \emptyset$} \\
                                           &\iff
                                             \begin{cases}
                                             [[ [σ]iM ]] \text{ is normal} \\
                                             [[ord {pas} in [σ]iM = pas]] \\
                                             \end{cases}
                                           && \text{by the definition of normalization}\\
                                           &\iff
                                             \begin{cases}
                                               [[ [σ]iM ]] \text{ is normal} \\
                                               [[ord {pas} in iM = pas]] \\
                                             \end{cases}
                                           && \text{by \cref{lemma:ord-sigma}}\\
                                           &\iff
                                             \begin{cases}
                                               [[ σ|fv(iM) ]] \text{ is normal} \\
                                               [[ iM ]] \text{ is normal} \\
                                               [[ord {pas} in iM = pas]] \\
                                             \end{cases}
                                           && \text{by the induction hypothesis}\\
                                           &\iff
                                             \begin{cases}
                                               [[ σ|fv(∀pas.iM) ]] \text{ is normal} \\
                                               [[ ∀pas.iM ]] \text{ is normal} \\
                                             \end{cases}
                                           &&
                                              \begin{aligned}
                                              &\text{since $[[fv(∀pas.iM) = fv(iM)]]$;}\\ &\text{by the definition of normalization}
                                              \end{aligned}
    \end{aligned}
    $
  \item $[[iP]] = \dots$\\
    The positive cases are done in the same way as the negative ones.

  \end{caseof}
\end{proof}





\subsection{Equivalence}
\begin{lemma}[Declarative equivalence is transitive]
  \hfill
  \label{lemma:decl-equiv-transitivity}
  \begin{itemize}
  \item[$+$] if $[[iP1 ≈ iP2]]$ and $[[iP2 ≈ iP3]]$ then $[[iP1 ≈ iP3]]$,
  \item[$-$] if $[[iN1 ≈ iN2]]$ and $[[iN2 ≈ iN3]]$ then $[[iN1 ≈ iN3]]$.
  \end{itemize}
\end{lemma}
\begin{proof}
  \ilyam{should be easy to do by induction since the types are getting smaller}
\end{proof}
\begin{lemma}[Algorithmization of declarative equivalence]
  \label{lemma:decl-equiv-algorithmization}
  Declarative equivalence is equality of normal forms. 
  \begin{itemize}
    \item[$+$] $[[iP ≈ iQ]] \iff [[nf(iP) = nf(iQ)]]$,
    \item[$-$] $[[iN ≈ iM]] \iff [[nf(iN) = nf(iM)]]$.
  \end{itemize}
\end{lemma}
\begin{proof} \hfill
  \begin{itemize}
    \item[$+$] Let us prove both directions separately.
    \begin{itemize}
      \item[$\Rightarrow$] 
        exactly by \cref{lemma:normalization-completeness},
      \item[$\Leftarrow$] 
        from \cref{lemma:normalization-soundness}, we know
        $[[iP ≈ nf(iP)]] = [[nf(iQ) ≈ iQ]]$, then by transitivity (\cref{lemma:decl-equiv-transitivity}),
        $[[iP ≈ iQ]]$.
    \end{itemize}
    \item[$-$] The proof is exactly the same.
  \end{itemize}
\end{proof}

\begin{lemma}[Type well-formedness is invariant under equivalence]
  \label{lemma:wf-equiv}
  Mutual subtyping implies declarative equivalence.
  \begin{itemize}
  \item[$+$] if $[[iP ≈ iQ]]$ then $[[Γ ⊢ iP]] \iff [[Γ ⊢ iQ]]$,
  \item[$-$] if $[[iN ≈ iM]]$ then $[[Γ ⊢ iN]] \iff [[Γ ⊢ iM]]$
  \end{itemize}
\end{lemma}
\begin{proof}
  \ilyam{todo}
\end{proof}

\begin{corollary}[Normalization preserves well-formedness]
  \label{corollary:wf-nf}
  \hfill
  \begin{itemize}
  \item[$+$] $[[Γ ⊢ iP]] \iff [[Γ ⊢ nf(iP)]]$,
  \item[$-$] $[[Γ ⊢ iN]] \iff [[Γ ⊢ nf(iN)]]$
  \end{itemize}
\end{corollary}
\begin{proof}
  Immediately from \cref{lemma:wf-equiv,lemma:normalization-soundness}.
\end{proof}

\begin{corollary}[Normalization preserves well-formedness of substitution]
  \label{corollary:wf-s-nf}
  \hfill \\
   $[[Γ2 ⊢ σ : Γ1]] \iff [[Γ2 ⊢ nf(σ) : Γ1]]$
\end{corollary}

\begin{lemma}[Soundness of equivalence]
  \label{lemma:equiv-soundness}
  Declarative equivalence implies mutual subtyping.
  \begin{itemize}
    \item[$+$] if $[[Γ ⊢ iP]]$, $[[Γ ⊢ iQ]]$, and $[[iP ≈ iQ]]$ then $[[Γ ⊢ iP ≈ iQ]]$,
    \item[$-$] if $[[Γ ⊢ iN]]$, $[[Γ ⊢ iM]]$, and $[[iN ≈ iM]]$ then $[[Γ ⊢ iN ≈ iM]]$.
  \end{itemize}
\end{lemma}
\begin{proof}
  We prove it by mutual induction on $[[iP ≈ iQ]]$ and $[[iN ≈ iM]]$.
  \begin{caseof}
    \item $[[a⁻ ≈ a⁻]]$\\
      Then $[[Γ ⊢ a⁻ ≤ a⁻]]$ by \ruleref{\ottdruleDOneNVarLabel},
      which immediately implies $[[Γ ⊢ a⁻ ≈ a⁻]]$ by \ruleref{\ottdruleDOneNDefLabel}.

    \item $[[↑iP ≈ ↑iQ]]$\\
      Then by inversion of \ruleref{\ottdruleDOneShiftULabel},
      $[[iP ≈ iQ]]$, and from the induction hypothesis, $[[Γ ⊢ iP ≈ iQ]]$,
      and (by symmetry) $[[Γ ⊢ iQ ≈ iP]]$.

      When \ruleref{\ottdruleDOneShiftULabel} is applied to $[[Γ ⊢ iP ≈ iQ]]$,
      it gives us $[[Γ ⊢ ↑iP ≤ ↑iQ]]$; when it is applied to $[[Γ ⊢ iQ ≈ iP]]$,
      we obtain $[[Γ ⊢ ↑iQ ≤ ↑iP]]$. Together, it implies $[[Γ ⊢ ↑iP ≈ ↑iQ]]$.

    \item $[[iP → iN ≈ iQ → iM]]$\\
      Then by inversion of \ruleref{\ottdruleDOneArrowLabel},
      $[[iP ≈ iQ]]$ and $[[iN ≈ iM]]$. By the induction hypothesis,
      $[[Γ ⊢ iP ≈ iQ]]$ and $[[Γ ⊢ iN ≈ iM]]$, which means by inversion:
      \begin{enumerate*}
        \item[(i)] $[[Γ ⊢ iP ≥ iQ]]$,
        \item[(ii)] $[[Γ ⊢ iQ ≥ iP]]$,
        \item[(iii)] $[[Γ ⊢ iN ≤ iM]]$,
        \item[(iv)]  $[[Γ ⊢ iM ≤ iN]]$.
      \end{enumerate*}
      Applying \ruleref{\ottdruleDOneArrowLabel} to (i) and (iii), we obtain
      $[[Γ ⊢ iP → iN ≤ iQ → iM]]$; applying it to (ii) and (iv), we have $[[Γ ⊢
      iQ → iM ≤ iP → iN]]$. Together, it implies $[[Γ ⊢ iP → iN ≈ iQ → iM]]$.
    \item $[[∀pas.iN ≈ ∀pbs.iM]]$\\
      Then by inversion, there exists bijection $[[mu : ({pbs} ∩ fv iM) ↔ ({pas}
      ∩ fv iN)]]$, such that $[[iN ≈ [mu] iM]]$. By the induction hypothesis,
      $[[Γ, pas ⊢ iN ≈ [mu] iM]]$. From \cref{corollary:subst-pres-equiv} and
      the fact that $[[mu]]$ is bijective, we also have
      $[[Γ, pbs ⊢ [mu-1]iN ≈ iM]]$.

      Let us construct a subsitution $[[pas ⊢ iPs/pbs : pbs]]$ by
      extending $[[mu]]$ with arbitrary positive types on $[[{pbs} \ fv iM]]$.

      Notice that $[[ [mu]iM ]] = [[ [iPs/pbs]iM ]]$, and therefore,
      $[[Γ, pas ⊢ iN ≈ [mu] iM]]$ implies $[[Γ, pas ⊢ [iPs/pbs]iM ≤ iN]]$. Then by
      \ruleref{\ottdruleDOneForallLabel}, $[[Γ ⊢ ∀pbs.iM ≤ ∀pas.iN]]$.

      Analogously, we construct the substitution from $[[mu-1]]$, and use it to
      instantiate $[[pas]]$ in the application of
      \ruleref{\ottdruleDOneForallLabel} to infer $[[Γ ⊢ ∀pas.iN ≤ ∀pbs.iM]]$.

      This way, $[[Γ ⊢ ∀pbs.iM ≤ ∀pas.iN]]$ and $[[Γ ⊢ ∀pas.iN ≤ ∀pbs.iM]]$
      gives us $[[Γ ⊢ ∀pbs.iM ≈ ∀pas.iN]]$.

    \item For the cases of the positive types, the proofs are symmetric.
  \end{caseof}
\end{proof}

\begin{corollary}[Normalization is sound w.r.t. subtyping-induced equivalence] \label{corollary:nf-sound-wrt-subt-equiv}
  \hfill
  \begin{itemize}
    \item [$+$] if $[[Γ ⊢ iP]]$ then $[[Γ ⊢ iP ≈ nf(iP)]]$,
    \item [$-$] if $[[Γ ⊢ iN]]$ then $[[Γ ⊢ iN ≈ nf(iN)]]$.
  \end{itemize}
\end{corollary}
\begin{proof}
  Immediately from \cref{lemma:normalization-soundness,corollary:wf-nf,lemma:equiv-soundness}.
\end{proof}


\begin{lemma}[Subtyping induced by disjoint substitutions]
  \label{lemma:subt-ind-disj-subst}
  If two disjoint substitutions induce subtyping, they are degenerate (so is the
  subtyping).
  Suppose that $[[Γ ⊢ σ1 : Γ1]]$ and $[[Γ ⊢ σ2 : Γ1]]$,
  where $[[{Γi} ⊆ {Γ}]]$ and $[[{Γ1} ∩ {Γ2}= ∅]]$. Then
  \begin{itemize}
  \item[$-$] assuming $[[Γ ⊢ iN]]$,~
    $[[Γ ⊢ [σ1]iN ≤ [σ2]iN]]$ implies $[[Γ ⊢ σi ≈ id : Ord fv iN]]$
  \item[$+$] assuming $[[Γ ⊢ iP]]$,~
    $[[Γ ⊢ [σ1]iP ≥ [σ2]iP]]$ implies $[[Γ ⊢ σi ≈ id : Ord fv iP]]$
  \end{itemize}
\end{lemma}
\begin{proof}
  Proof by induciton on $[[Γ ⊢ iN]]$ (and mutually on $[[Γ ⊢ iP]]$).
  \begin{caseof}
    \item $[[iN]] = [[α⁻]]$\\
      Then $[[Γ ⊢ [σ1]iN ≤ [σ2]iN]]$ is rewritten as $[[Γ ⊢ [σ1]α⁻ ≤ [σ2]α⁻]]$.
      Let us consider the following cases:
      \begin{caseof}
      \item $[[α⁻ ∉ {Γ1}]]$ and $[[α⁻ ∉ {Γ2}]]$ \label{case:var-not-in-ctxts}\\
        Then $[[Γ ⊢ σi ≈ id : α⁻]]$ holds immediately,
        since $[[ [σi] α⁻]] = [[ [id] α⁻]] = [[α⁻]]$ and
        $[[Γ ⊢ α⁻ ≈ α⁻]]$.
      \item $[[α⁻ ∊ {Γ1}]]$ and $[[α⁻ ∊ {Γ2}]]$\\
        This case is not possible by assumption: $[[{Γ1} ∩ {Γ2}= ∅]]$.
      \item $[[α⁻ ∊ {Γ1}]]$ and $[[α⁻ ∉ {Γ2}]]$\\
        Then we have $[[Γ ⊢ [σ1]α⁻ ≤ α⁻]]$,
        which by \cref{corollary:vars-no-proper-subtypes} means $[[Γ ⊢ [σ1]α⁻ ≈ α⁻]]$,
        and hence, $[[Γ ⊢ σ1 ≈ id : α⁻]]$.

        $[[Γ ⊢ σ2 ≈ id : α⁻]]$ holds since $[[ [σ2]α⁻ ]] = [[α⁻]]$,
        similarly to \cref{case:var-not-in-ctxts}.

      \item $[[α⁻ ∉ {Γ1}]]$ and $[[α⁻ ∊ {Γ2}]]$\\
        Then we have $[[Γ ⊢ α⁻ ≤ [σ2]α⁻]]$,
        which by \cref{corollary:vars-no-proper-subtypes} means $[[Γ ⊢ α⁻ ≈ [σ2]α⁻]]$,
        and hence, $[[Γ ⊢ σ2 ≈ id : α⁻]]$.

        $[[Γ ⊢ σ1 ≈ id : α⁻]]$ holds since $[[ [σ1]α⁻ ]] = [[α⁻]]$,
        similarly to \cref{case:var-not-in-ctxts}.
      \end{caseof}
  \item $[[iN]] = [[∀pas.iM]]$\\
    Then by inversion, $[[Γ, pas ⊢ iM]]$.
    $[[Γ ⊢ [σ1]iN ≤ [σ2]iN]]$ is rewritten as $[[Γ ⊢ [σ1]∀pas.iM ≤ [σ2]∀pas.iM]]$.
    By the congruence of substitution and by the inversion of
    \ruleref{\ottdruleDOneForallLabel}, $[[Γ, pas ⊢ [iQs/pas][σ1]iM ≤ [σ2]iM]]$,
    where $[[Γ, pas ⊢ iQi]]$.
    Let us denote the (Kleisli) composition of $[[σ1]]$ and $[[iQs/pas]]$ as
    $[[σ1']]$, noting that $[[Γ, pas ⊢ σ1' : Γ1, pas]]$,
    and $[[{Γ1, pas} ∩ {Γ2} = ∅]]$.

    Let us apply the induction hypothesis to $[[iM]]$ and the
    substitutions $[[σ1']]$ and $[[σ2]]$ with
    $[[Γ, pas ⊢ [σ1']iM ≤ [σ2]iM]]$ to obtain:
    \begin{align}
      [[Γ, pas ⊢ σ1' ≈ id : Ord fv iM]] \label{fact:subs-proper-sub:forall-ih}\\
      [[Γ, pas ⊢ σ2 ≈ id : Ord fv iM]]  \label{fact:subs-proper-sub:forall-ih2}
    \end{align}

    Then $[[Γ ⊢ σ2 ≈ id : Ord fv ∀pas.iM]]$ holds by strengthening of
    \ref{fact:subs-proper-sub:forall-ih2}:
    for any $[[β±]] \in [[fv ∀pas.iM]] = [[fv iM \ {pas}]]$,
    $[[Γ, pas ⊢ [σ2]β± ≈ β±]]$ is strengthened to $[[Γ ⊢ [σ2]β± ≈ β±]]$, because
    $[[fv [σ2]β±]] = [[fv β±]] = \{[[β±]]\} \subseteq [[{Γ}]]$.

    To show that $[[Γ ⊢ σ1 ≈ id : Ord fv ∀pas.iM]]$, let us take an arbitrary
    $[[β±]] \in [[fv ∀pas.iM]] = [[fv iM \ {pas}]]$.

    $
    \begin{aligned}[t]
      [[β±]] &= [[ [id]β± ]]
             && \text{by definition of $[[id]]$}\\
             &\eqDOne [[ [σ1']β± ]]
             && \text{by \ref{fact:subs-proper-sub:forall-ih}}\\
             &= [[ [iQs/pas][σ1]β±]]
             && \text{by definition of $[[σ1']]$}\\
             &= [[ [σ1]β± ]]
             && \text{because $[[{pas} ∩ fv [σ1]β±]] \subseteq [[{pas} ∩ {Γ}]] = \emptyset$}
    \end{aligned}
    $\\
    This way, $[[Γ ⊢ [σ1]β± ≈ β±]]$ for any $[[β±]] \in [[fv ∀pas.iM]]$ and thus,
    $[[Γ ⊢ σ1 ≈ id : Ord fv ∀pas.iM]]$.

  \item $[[iN]] = [[iP → iM]]$\\
    Then by inversion, $[[Γ ⊢ iP]]$ and $[[Γ ⊢ iM]]$.
    $[[Γ ⊢ [σ1]iN ≤ [σ2]iN]]$ is rewritten as
    $[[Γ ⊢ [σ1](iP → iM) ≤ [σ2](iP → iM)]]$,
    then by congruence of substitution,
    $[[Γ ⊢ [σ1]iP → [σ1]iM ≤ [σ2]iP → [σ2]iM]]$,
    then by inversion
    $[[Γ ⊢ [σ1]iP ≥ [σ2]iP]]$
    and
    $[[Γ ⊢ [σ1]iM ≤ [σ2]iM]]$.

    Applying the induction hypothesis to $[[Γ ⊢ [σ1]iP ≥ [σ2]iP]]$
    and to $[[Γ ⊢ [σ1]iM ≤ [σ2]iM]]$, we obtain (respectively):
    \begin{align}
      &[[Γ ⊢ σi ≈ id : Ord fv iP]] \label{fact:subs-proper-sub:arrow-ih1}\\
      &[[Γ ⊢ σi ≈ id : Ord fv iM]] \label{fact:subs-proper-sub:arrow-ih2}
    \end{align}

    Noting that $[[fv (iP → iM)]] = [[fv iP ∪ fv iM]]$,
    we combine
    \cref{fact:subs-proper-sub:arrow-ih1,fact:subs-proper-sub:arrow-ih2}
    to conclude:
    $[[Γ ⊢ σi ≈ id : Ord fv (iP → iM)]]$.

  \item $[[iN]] = [[↑iP]]$\\
    Then by inversion, $[[Γ ⊢ iP]]$.
    $[[Γ ⊢ [σ1]iN ≤ [σ2]iN]]$ is rewritten as
    $[[Γ ⊢ [σ1]↑iP ≤ [σ2]↑iP]]$,
    then by congruence of substitution and by inversion,
    $[[Γ ⊢ [σ1]iP ≥ [σ2]iP]]$

    Applying the induction hypothesis to $[[Γ ⊢ [σ1]iP ≥ [σ2]iP]]$, we obtain
    $[[Γ ⊢ σi ≈ id : Ord fv iP]]$. Since $[[fv ↑iP]] = [[fv iP]]$, we can
    conclude: $[[Γ ⊢ σi ≈ id : Ord fv ↑iP]]$.
  \item The positive cases are proved symmetrically.
  \end{caseof}
\end{proof}

\begin{corollary}[Substitution cannot induce proper subtypes or supertypes] \label{corollary:subst-proper-subt}
  Assuming all mentioned types are well-formed in $[[Γ]]$ and $[[σ]]$ is a
  substitution $[[Γ ⊢ σ : Γ]]$,
  \begin{align*}
    [[Γ ⊢ [σ]iN ≤ iN]] ~&\Rightarrow~ [[Γ ⊢ [σ]iN ≈ iN]]
                          \text{ and } [[Γ ⊢ σ ≈ id : Ord fv iN]] \\
    [[Γ ⊢ iN ≤ [σ]iN]] ~&\Rightarrow~ [[Γ ⊢ iN ≈ [σ]iN]]
                          \text{ and } [[Γ ⊢ σ ≈ id : Ord fv iN]] \\
    [[Γ ⊢ [σ]iP ≥ iP]] ~&\Rightarrow~ [[Γ ⊢ [σ]iP ≈ iP]]
                          \text{ and } [[Γ ⊢ σ ≈ id : Ord fv iP]] \\
    [[Γ ⊢ iP ≥ [σ]iP]] ~&\Rightarrow~ [[Γ ⊢ iP ≈ [σ]iP]]
                          \text{ and } [[Γ ⊢ σ ≈ id : Ord fv iP]] \\
  \end{align*}
\end{corollary}


\begin{lemma} \label{lemma:mutual-subst-subtyping}
  Asssuming that the mentioned types ($[[iP]]$, $[[iQ]]$, $[[iN]]$, and $[[iM]]$)
  are well-formed in $[[Γ]]$ and that the substitutions ($[[σ1]]$ and $[[σ2]]$) have signature $[[Γ ⊢ σi : Γ]]$,
  \begin{itemize}
  \item[$+$] if $[[Γ ⊢ [σ1] iP ≥ iQ]]$ and $[[Γ ⊢ [σ2] iQ ≥ iP]]$\\
    then there exists a bijection $[[μ : fv iP ↔ fv iQ]]$ such that
    $[[Γ ⊢ σ1 ≈ Sub μ : Ord fv iP]]$ and $[[Γ ⊢ σ2 ≈ Sub μ-1 : Ord fv iQ]]$;
  \item[$-$] if $[[Γ ⊢ [σ1] iN ≤ iM]]$ and $[[Γ ⊢ [σ2] iN ≤ iM]]$\\
    then there exists a bijection $[[μ : fv iN ↔ fv iM]]$ such that
    $[[Γ ⊢ σ1 ≈ Sub μ : Ord fv iN]]$ and $[[Γ ⊢ σ2 ≈ Sub μ-1 : Ord fv iM]]$.
  \end{itemize}
\end{lemma}
\begin{proof}
  \hfill
  \begin{itemize}
  \item[$+$]
    Applying $[[σ2]]$ to both sides of
    $[[Γ ⊢ [σ1] iP ≥ iQ]]$ (by \cref{todo}),
    we have: $[[Γ ⊢ [σ2 ○ σ1] iP ≥ [σ2]iQ]]$.
    Composing it with $[[Γ ⊢ [σ2] iQ ≥ iP]]$ (by transitivity \cref{todo}),
    we have $[[Γ ⊢ [σ2 ○ σ1] iP ≥ iP]]$.
    Then by \cref{corollary:subst-proper-subt},
    $[[Γ ⊢ σ2 ○ σ1 ≈ id : Ord fv iP]]$.

    % Applying $[[σ1]]$ to both sides of
    % $[[Γ ⊢ [σ2]iQ ≥ iP]]$ (by \cref{todo}),
    % we have: $[[Γ ⊢ [σ1 ○ σ2] iQ ≥ [σ1]iP]]$.
    % Composing it with $[[Γ ⊢ [σ1] iP ≥ iQ]]$ (by transitivity \cref{todo}),
    % we have $[[Γ ⊢ [σ1 ○ σ2] iQ ≥ iQ]]$.
    % Then by \cref{corollary:subst-proper-subt},
    By a symmetric argument, we also have:
    $[[Γ ⊢ σ1 ○ σ2 ≈ id : Ord fv iQ]]$.

    Now, we prove that
    $[[Γ ⊢ σ2 ○ σ1 ≈ id : Ord fv iP]]$ and
    $[[Γ ⊢ σ1 ○ σ2 ≈ id : Ord fv iQ]]$
    implies that $[[σ1]]$ and $[[σ1]]$
    are (equivalent to) mutually inverse bijections.

    To do so, it suffices to prove that
    \begin{enumerate}
    \item[(i)] for any $[[α± ∊ fv iP]]$ there exists $[[β± ∊ fv iQ]]$
        such that $[[ Γ ⊢ [σ1] α± ≈ β± ]]$ and
        $[[ Γ ⊢ [σ2] β± ≈ α± ]]$; and
    \item[(ii)] for any $[[β± ∊ fv iQ]]$ there exists $[[α± ∊ fv iP]]$
        such that $[[ Γ ⊢ [σ2] β± ≈ α± ]]$ and
        $[[ Γ ⊢ [σ1] α± ≈ β± ]]$.
    \end{enumerate}
    Then the these correspondences between $[[fv iP]]$ and
    $[[fv iQ]]$ are mutually inverse functions,
    since for any $[[β±]]$ there can be at most one $[[α±]]$
    such that $[[ Γ ⊢ [σ2] β± ≈ α± ]]$ (and vice versa).

    \begin{enumerate}
    \item[(i)] Let us take $[[α± ∊ fv iP]]$.
      \begin{enumerate}
      \item if $[[α±]]$ is positive ($[[α± = α⁺]]$),
        from $[[ Γ ⊢ [σ2][σ1]α⁺ ≈ α⁺ ]]$,
        by \cref{corollary:vars-no-proper-subtypes},
        we have
        $[[ [σ2][σ1]α⁺ = ∃nbs.α⁺ ]]$.

        What shape can $[[ [σ1]α⁺ ]]$ have? It cannot be $[[∃nas.↓iN]]$ (for
        potentially empty $[[nas]]$), because the outer constructor $\downarrow$
        would remain after the substitution $[[σ2]]$, whereas $[[∃nbs.α⁺]]$ does
        not have $[[↓]]$. The only case left is $[[ [σ1]α⁺ = ∃nas.γ⁺ ]]$.

        Notice that $[[Γ ⊢ ∃nas.γ⁺ ≈ γ⁺]]$, meaning that $[[Γ ⊢ [σ1]α⁺ ≈ γ⁺]]$.
        Also notice that $[[ [σ2]∃nas.γ⁺ = ∃nbs.α⁺ ]]$ implies
        $[[Γ ⊢ [σ2]γ⁺ ≈ α⁺]]$.

      \item if $[[α±]]$ is negative ($[[α± = α⁻]]$) from $[[ Γ ⊢ [σ2][σ1]α⁻ ≈ α⁻
        ]]$, by \cref{corollary:vars-no-proper-subtypes}, we have
        $[[ [σ2][σ1]α⁻ = ∀pbs.α⁻ ]]$.

        What shape can $[[ [σ1]α⁻ ]]$ have? It cannot be $[[∀pas.↑iP]]$
        nor $[[∀pas.iP → iM]]$ (for potentially empty $[[pas]]$),
        because the outer constructor ($[[→]]$ or $[[↑]]$), remaining
        after the substitution $[[σ2]]$, is however absent in the resulting
        $[[∀pbs.α⁻]]$. Hence, the only case left is $[[ [σ1]α⁻ = ∀pas.γ⁻ ]]$
        Notice that $[[Γ ⊢ γ⁻ ≈ ∀pas.γ⁻]]$, meaning that $[[Γ ⊢ [σ1]α⁻ ≈ γ⁻]]$.
        Also notice that $[[ [σ2]∀pas.γ⁻ = ∀pbs.α⁻ ]]$ implies
        $[[Γ ⊢ [σ2]γ⁻ ≈ α⁻]]$.
      \end{enumerate}
    \item[(ii)] The proof is symmetric:
      We swap $[[iP]]$ and $[[iQ]]$,
      $[[σ1]]$ and $[[σ2]]$,
      and exploit $[[ Γ ⊢ [σ1][σ2]α± ≈ α± ]]$ instead of
      $[[ Γ ⊢ [σ2][σ1]α± ≈ α± ]]$.

    \end{enumerate}

  \item[$-$] The proof is symmetric to the positive case.
  \end{itemize}
\end{proof}

\begin{lemma}[Equivalence of polymorphic types]
  \label{lemma:poly-types-equivalence}
  \hfill
  \begin{itemize}
    \item[$-$] For $[[Γ ⊢ ∀pas.iN]]$ and $[[Γ ⊢ ∀pbs.iM]]$,\\ if $[[Γ ⊢ ∀pas.iN ≈ ∀pbs.iM ]]$
    then there exists a bijection $[[μ : {pbs} ∩ fv iM ↔ {pas} ∩ fv iN]]$
    such that $[[ Γ, pas ⊢ iN ≈ [Sub μ] iN ]]$,
    \item[$+$] For $[[Γ ⊢ ∃nas.iP]]$ and $[[Γ ⊢ ∃nbs.iQ]]$,\\  if $[[Γ ⊢ ∃nas.iP ≈ ∃nbs.iQ ]]$
    then there exists a bijection $[[μ : {nbs} ∩ fv iQ ↔ {nas} ∩ fv iP]]$
    such that $[[ Γ, nbs ⊢ iP ≈ [Sub μ] iQ ]]$.
  \end{itemize}
\end{lemma}
\begin{proof}
    \hfill
  \begin{itemize}
    \item[$-$]
    First, by $\alpha$-conversion, we ensure $[[{pas} ∩ fv iM = ∅]]$ and $[[{pbs} ∩ fv iN = ∅]]$.
    By inversion, $[[Γ ⊢ ∀pas.iN ≈ ∀pbs.iM ]]$ implies 
    \begin{enumerate} 
      \item $[[Γ,pbs ⊢ [σ1]iN ≤ iM]]$ for $[[ Γ,pbs ⊢ σ1 : pas ]]$ and 
      \item $[[Γ,pas ⊢ [σ2]iM ≤ iN]]$ for $[[ Γ,pas ⊢ σ2 : pbs ]]$.
    \end{enumerate}
    To apply \cref{lemma:mutual-subst-subtyping}, we weaken 
    and rearrange the contexts, and extend the substitutions to act as identity
    outside of their initial domain:
    \begin{enumerate} 
      \item $[[Γ,pas,pbs ⊢ [σ1]iN ≤ iM]]$ for $[[ Γ,pas,pbs ⊢ σ1 : Γ,pas,pbs ]]$ and 
      \item $[[Γ,pas,pbs ⊢ [σ2]iM ≤ iN]]$ for $[[ Γ,pas,pbs ⊢ σ2 : Γ,pas,pbs ]]$.
    \end{enumerate}
    Then from \cref{lemma:mutual-subst-subtyping}, 
    there exists a bijection $[[μ : fv iM ↔ fv iN]]$ such that 
    $[[Γ,pas,pbs ⊢ σ2 ≈ Sub μ : Ord fv iM]]$ and 
    $[[Γ,pas,pbs ⊢ σ1 ≈ Sub μ-1 : Ord fv iN]]$. 

    Let us show that if we restrict the domain of $[[μ]]$ to 
    $[[pbs]]$, its range will be contained in $[[pas]]$.
    Let us take $[[γ⁺ ∊ {pbs} ∩ fv iM]]$ and 
    assume $[[ [μ]γ⁺]] \notin [[pas]]$.
    Then since $[[ Γ,pbs ⊢ σ1 : pas ]]$, 
    $[[σ1]]$ acts as identity outside of $[[pas]]$, i.e.
    $[[ [σ1][Sub μ]γ⁺ = [Sub μ]γ⁺ ]]$.
    Since
    $[[Γ,pas,pbs ⊢ σ1 ≈ Sub μ-1 : Ord fv iN]]$, 
    application of $[[σ1]]$ is equivalent to application of $[[Sub μ-1]]$,
    then 
    $[[ Γ,pas,pbs ⊢ [Sub μ-1][Sub μ]γ⁺ ≈ [Sub μ]γ⁺ ]]$, i.e.
    $[[Γ,pas,pbs ⊢ γ⁺ ≈ [Sub μ]γ⁺]]$, 
    which means $[[γ⁺ ∊ fv [Sub μ]γ⁺]] \subseteq [[fv iN]]$.
    By assumption, $[[γ⁺ ∊ {pbs} ∩ fv iM]]$, i.e. $[[{pbs} ∩ fv iN]] \neq \emptyset$, hence contradiction.

    By \cref{todo}, 
    $[[Γ,pas,pbs ⊢ σ2 ≈ Sub μ|{pbs} : Ord fv iM]]$ implies
    $[[Γ,pas,pbs ⊢ [σ2]iM ≈ [Sub μ|{pbs}]iM]]$.
    By similar reasoning, $[[Γ,pas,pbs ⊢ [σ1]iN ≈ [Sub μ-1|{pas}]iN]]$.

    This way,
    \begin{align} 
      [[Γ,pas,pbs ⊢ [Sub μ-1|{pas}]iN ≤ iM]] \label{fact:mu-inv-n-sub-m}\\
      [[Γ,pas,pbs ⊢ [Sub μ|{pbs}]iM ≤ iN]] \label{fact:mu-m-subt-n}
    \end{align}

    By applying $[[μ|_{pbs}]]$ to both sides of \ref{fact:mu-inv-n-sub-m} (\cref{todo})
    and contracting $[[μ-1|_{pas} ○ μ|_{pbs}]] = [[μ|_{pbs}-1 ○ μ|_{pbs}]] = [[id]]$,
    we have: $[[Γ,pas,pbs ⊢ iN ≤ [Sub μ|{pbs}]iM]]$, which together with \ref{fact:mu-m-subt-n}
    means $[[Γ,pas,pbs ⊢ iN ≈ [Sub μ|{pbs}]iM]]$, and by strengthening, $[[Γ,pas⊢ iN ≈ [Sub μ|{pbs}]iM]]$.
    Symmetrically, $[[Γ,pbs ⊢ iM ≈ [Sub μ|_{pbs}-1]iN]]$.
    \item{$+$} The proof is symmetric to the proof of the negative case.
  \end{itemize}

\end{proof}


\begin{lemma}[Completeness of equivalence] \label{lemma:equiv-completeness}
  Mutual subtyping implies declarative equivalence.
  Assuming all the types below are well-formed in $[[Γ]]$: 
  \begin{itemize}
  \item[$+$] if $[[Γ ⊢ iP ≈ iQ]]$ then $[[iP ≈ iQ]]$,
  \item[$-$] if $[[Γ ⊢ iN ≈ iM]]$ then $[[iN ≈ iM]]$.
  \end{itemize}
\end{lemma}
\begin{proof}
  \begin{itemize}
    \item[$-$] 
    Induction on the sum of sizes of  $[[iN]]$ and $[[iM]]$. 
    By inversion, $[[Γ ⊢ iN ≈ iM]]$ means $[[Γ ⊢ iN ≤ iM]]$ and $[[Γ ⊢ iM ≤ iN ]]$.
    Let us consider the last rule that forms $[[Γ ⊢ iN ≤ iM]]$:
    \begin{caseof}
      \item \ruleref{\ottdruleDOneNVarLabel} i.e. $[[Γ ⊢ iN ≤ iM]]$ is of the form $[[Γ ⊢ α⁻ ≤ α⁻]]$\\
      Then $[[iN ≈ iM]]$ (i.e. $[[α⁻ ≈ α⁻]]$) holds immediately by \ruleref{\ottdruleEOneNVarLabel}.

      \item \ruleref{\ottdruleDOneShiftULabel} i.e. 
      $[[Γ ⊢ iN ≤ iM]]$ is of the form $[[Γ ⊢ ↑iP ≤ ↑iQ]]$\\
      Then by inversion, $[[Γ ⊢ iP ≈ iQ]]$, 
      and by induction hypothesis, $[[iP ≈ iQ]]$.
      Then $[[iN ≈ iM]]$ (i.e. $[[↑iP ≈ ↑iQ]]$) holds 
      by \ruleref{\ottdruleEOneShiftULabel}.

      \item \ruleref{\ottdruleDOneArrowLabel} i.e. $[[Γ ⊢ iN ≤ iM]]$ is of the form $[[Γ ⊢ iP → iN' ≤ iQ → iM']]$\\
      Then by inversion, $[[Γ ⊢ iP ≥ iQ]]$ and $[[Γ ⊢ iN' ≤ iM']]$.
      Notice that $[[Γ ⊢ iM ≤ iN]]$ is of the form $[[Γ ⊢ iQ → iM' ≤ iP → iN']]$, 
      which by inversion means $[[Γ ⊢ iQ ≥ iP]]$ and $[[Γ ⊢ iM' ≤ iN']]$.

      This way, $[[Γ ⊢ iQ ≈ iP]]$ and $[[Γ ⊢ iM' ≈ iN']]$. 
      Then by induction hypothesis, $[[iQ ≈ iP]]$ and $[[iM' ≈ iN']]$.
      Then $[[iN ≈ iM]]$ (i.e. $[[iP → iN' ≈ iQ → iM']]$) holds by \ruleref{\ottdruleEOneArrowLabel}.

      \item \ruleref{\ottdruleDOneForallLabel} i.e. $[[Γ ⊢ iN ≤ iM]]$ is of the form $[[Γ ⊢ ∀pas.iN' ≤ ∀pbs.iM']]$\\
      Then by \cref{lemma:poly-type-equivalence}, $[[Γ ⊢ ∀pas.iN' ≈ ∀pbs.iM']]$ means that 
      there exists a bijection $[[μ : {pbs} ∩ fv iM' ↔ {pas} ∩ fv iN']]$ such that  
      $[[Γ,pas ⊢ [Sub μ]iM' ≈ iN']]$. 
      
      Notice that the application of bijection $[[μ]]$ to $[[iM']]$ does
      not change its size (which is less than the size of $[[iM]]$), hence the induction hypothesis applies.
      This way, $[[ [Sub μ]iM' ≈ iN']]$ (and by symmetry, $[[iN' ≈ [Sub μ]iM']]$) holds by induction. 
      Then we apply \ruleref{\ottdruleEOneForallLabel} to get $[[∀pas.iN' ≈ ∀pbs.iM']]$, i.e. $[[iN ≈ iM]]$.
    \end{caseof}
      
\item[$+$] The proof is symmetric to the proof of the negative case.
  \end{itemize}
\end{proof}

\begin{corollary}[Normalization is complete w.r.t. subtyping-induced equivalence]
  \label{corollary:nf-complete-wrt-subt-equiv}
  Assuming all the types below are well-formed in $[[Γ]]$:
  \begin{itemize}
    \item [$+$] if $[[Γ ⊢ iP ≈ iQ]]$ then $[[nf(iP) = nf(iQ)]]$,
    \item [$-$] if $[[Γ ⊢ iN ≈ iM]]$ then $[[nf(iN) = nf(iM)]]$.
  \end{itemize}
\end{corollary}  
\begin{proof}
  Immediately from \cref{lemma:equiv-completeness,lemma:normalization-completeness}.
\end{proof}

\begin{lemma}[Algorithmization of subtyping-induced equivalence]
  \label{lemma:subt-equiv-algorithmization}
  Mutual subtyping is equality of normal forms.
 Assuming all the types below are well-formed in $[[Γ]]$:
  \begin{itemize}
    \item [$+$] $[[Γ ⊢ iP ≈ iQ]] \iff [[nf(iP) = nf(iQ)]]$,
    \item [$-$] $[[Γ ⊢ iN ≈ iM]] \iff [[nf(iN) = nf(iM)]]$.
  \end{itemize}
\end{lemma}
\begin{proof}
  Let us prove the positive case, the negative case is symmetric.
  We prove both directions of $\iff$ separately:
  \begin{itemize}
    \item [$\Rightarrow$] exactly \cref{corollary:nf-complete-wrt-subt-equiv};
    \item [$\Leftarrow$] by \cref{lemma:decl-equiv-algorithmization,lemma:equiv-soundness}.
  \end{itemize}
\end{proof}



\subsection{Unification Constraint Merge}
\obsUnifMergeDet*
\begin{proof}
    $[[UC]]$ and $[[UC']]$ both consists of three parts: 
    Entries of $[[UC1]]$ that do not have matching entries in $[[UC2]]$,
    entries of $[[UC2]]$ that do not have matching entries in $[[UC1]]$,
    and the merge of matching entries.

    The parts corresponding to unmatched entries of $[[UC1]]$ and $[[UC2]]$ coincide, 
    since $[[UC1]]$ and $[[UC2]]$ are fixed.
    To show that the merge of matching entries coincide,
    let us take any pair of matching $[[ucE1 ∊ UC1]]$ and $[[ucE2 ∊ UC2]]$
    and consider their shape.
    \begin{caseof}
        \item $[[ucE1]]$ is $[[pua :≈ iQ1]]$ and $[[ucE2]]$ is $[[pua :≈ iQ2]]$
            then the result, if it exists, is always $[[ucE1]]$,
            by inversion of \ruleref{\ottdruleSCMEPEqEqLabel}.
        \item $[[ucE1]]$ is $[[nua :≈ iN1]]$ and $[[ucE2]]$ is $[[nua :≈ iN2]]$
            then analogously, the result, if it exists, is always $[[ucE1]]$,
            by inversion of \ruleref{\ottdruleSCMENEqEqLabel}.
    \end{caseof}
    This way, the third group of entries coincide as well.
\end{proof}

\lemUnifMergeSoundness*
\begin{proof}
    \hfill
    \begin{itemize}
        \item $[[UC1 & UC2]] \subseteq [[UC1]] \cup [[UC2]]$\\
        By definition, 
        $[[UC1 & UC2]]$ consists of three parts:
        entries of $[[UC1]]$ that do not have matching entries of $[[UC2]]$,
        entries of $[[UC2]]$ that do not have matching entries of $[[UC1]]$,
        and the merge of matching entries.

        If $[[ucE]]$ is from the first or the second part, 
        then $[[ucE]] \in [[UC1]] \cup [[UC2]]$ holds immediately.
        If $[[ucE]]$ is from the third part,
        then $[[ucE]]$ is the merge of two matching entries
        $[[ucE1]] \in [[UC1]]$ and $[[ucE2]] \in [[UC2]]$.
        Since $[[UC1]]$ and $[[UC2]]$ are normalized unification , 
        $[[ucE1]]$ and $[[ucE2]]$ have one of the following forms:
        \begin{itemize}
            \item $[[α̂⁺ :≈ iP1]]$ and $[[α̂⁺ :≈ iP2]]$, 
                where $[[iP1]]$ and $[[iP2]]$ are normalized,
                and then since $[[Θ(α̂⁺) ⊢ ucE1 & ucE2 = ucE]]$ exists, 
                \ruleref{\ottdruleSCMEPEqEqLabel} was applied to infer it.
                It means that $[[ucE]] = [[ucE1]] = [[ucE2]]$;
            \item $[[α̂⁻ :≈ iN1]]$ and $[[α̂⁻ :≈ iN2]]$, 
               then symmetrically, 
               $[[Θ(α̂⁻) ⊢ ucE1 & ucE2 = ucE]] = [[ucE1]] = [[ucE2]]$
        \end{itemize}
        In both cases, $[[ucE]] \in [[UC1]] \cup [[UC2]]$.

        \item $[[UC1]] \cup [[UC2]] \subseteq [[UC1 & UC2]]$\\
        Let us take 
        an arbitrary $[[ucE1]] \in [[UC1]]$.
        Then since $[[UC1]]$ is a unification constraint,
         $[[ucE1]]$ has one of the following forms:
        \begin{itemize}
            \item $[[α̂⁺ :≈ iP]]$ where $[[iP]]$ is normalized.
            If $[[α̂⁺]] \notin [[dom(UC2)]]$, then $[[ucE1]] \in [[UC1 & UC2]]$.
            Otherwise, there is a normalized matching
            $[[ucE2]] = [[(α̂⁺ :≈ iP')]] \in [[UC2]]$ and then
            since $[[UC1 & UC2]]$ exists, 
            \ruleref{\ottdruleSCMEPEqEqLabel} was applied to construct
            $[[ucE1 & ucE2]] \in [[UC1 & UC2]]$.
            By inversion of \ruleref{\ottdruleSCMEPEqEqLabel},
            $[[ucE1 & ucE2]] = [[ucE1]]$, and
            $[[nf(iP) = nf(iP')]]$, which since $[[iP]]$
            and $[[iP']]$ are normalized, implies that $[[iP = iP']]$, 
            that is $[[ucE1]] = [[ucE2]] \in [[UC1 & UC2]]$.
            \item $[[α̂⁻ :≈ iN]]$ where $[[iN]]$ is normalized.
            Then symmetrically, $[[ucE1]] = [[ucE2]] \in [[UC1 & UC2]]$.
        \end{itemize}
        Similarly, if we take an arbitrary $[[ucE2]] \in [[UC2]]$,
        then $[[ucE1]] = [[ucE2]] \in [[UC1 & UC2]]$. 
    \end{itemize}
\end{proof}

\corUnifMergeSoundness*
\begin{proof}
    It is clear that since $[[UC = UC1 ∪ UC2]]$ (by \cref{lemma:unif-merge-soundness}),
    and being normalized means that all entries are normalized,
    $[[UC]]$ is a normalized unification constraint.
    Analogously, $[[Θ ⊢ UC]] = [[UC1 ∪ UC2]]$ holds immediately, 
    since $[[Θ ⊢ UC1]]$ and $[[Θ ⊢ UC2]]$.

    Let us take an arbitrary substitution $[[Θ ⊢ uσ : dom(UC)]]$ and assume that 
    $[[ Θ   ⊢ uσ : lift UC ]]$.
    Then $[[ Θ   ⊢ uσ : lift UCi ]]$ holds by definition:
    If $[[ucE]] \in [[lift UCi]] \subseteq [[lift UC1 ∪ lift UC2]] = [[lift UC]]$ then
    $[[Θ(α̂±) ⊢ [uσ]α̂± : ucE]]$ (where $[[ucE]]$ restricts $[[α̂±]]$) holds since $[[Θ ⊢ uσ : dom(UC)]]$.
\end{proof}

\lemUnifEntryMergeCompleteness*
\begin{proof}
    Let us consider the shape of $[[ucE1]]$ and $[[ucE2]]$.
    \begin{caseof}
        \item $[[ucE1]]$ is $[[pua :≈ iQ1]]$ and $[[ucE2]]$ is $[[pua :≈ iQ2]]$.
            Then $[[Γ ⊢ iP : ucE1]]$ means $[[Γ ⊢ iP ≈ iQ1]]$, 
            and $[[Γ ⊢ iP : ucE2]]$ means $[[Γ ⊢ iP ≈ iQ2]]$.
            Then by transitivity of equivalence (\cref{corollary:equivalence-transitivity}),
            $[[Γ ⊢ iQ1 ≈ iQ2]]$, which means $[[nf(iQ1) = nf(iQ2)]]$ by
            \cref{lemma:subt-equiv-algorithmization}.
            Hence, \ruleref{\ottdruleSCMEPEqEqLabel} applies to infer
            $[[Γ ⊢ ucE1 & ucE2 = ucE2]]$, and $[[Γ ⊢ iP : ucE2]]$ holds by assumption.
        \item $[[ucE1]]$ is $[[nua :≈ iN1]]$ and $[[ucE2]]$ is $[[nua :≈ iM2]]$.
            The proof is symmetric.
    \end{caseof}
\end{proof}

\lemUnifMergeCompleteness*
\begin{proof}
    The proof repeats the proof of \cref{lemma:merge-completeness}
    for cases 
    uses \cref{lemma:unif-entry-merge-completeness} instead of \cref{lemma:entry-merge-completeness}.
\end{proof}


\subsection{Unification}
\begin{lemma}[Soundness of Unification] \label{lemma:unification-soundness}
    \hfill
    \begin{itemize}
        \item [$+$] For normalized $[[uP]]$ and $[[iQ]]$ such that 
        $[[Γ ; dom(Θ) ⊢ uP]]$ and $[[Γ ⊢ iQ]]$,\\ 
        if $[[Γ ; Θ ⊨ uP ≈u iQ ⫤ UC]]$ then 
        $[[Θ ⊢ UC : uv uP]]$ and for any normalized $[[uσ]]$ 
        such that $[[ Θ ⊢ uσ : lift UC ]]$, $[[ [uσ]uP = iQ ]]$.

        \item [$-$] For normalized $[[uN]]$ and $[[iM]]$ such that
        $[[Γ ; dom(Θ) ⊢ uN]]$ and $[[Γ ⊢ iM]]$,\\
        if $[[Γ ; Θ ⊨ uN ≈u iM ⫤ UC]]$ then 
        $[[Θ ⊢ UC : uv uN]]$ and for any normalized $[[uσ]]$ such that
        $[[ Θ   ⊢ uσ : lift UC ]]$, $[[ [uσ]uN = iM ]]$.
    \end{itemize}
\end{lemma}
\begin{proof}
    We prove by induction on the derivation of 
    $[[ Γ ; Θ ⊨ uN ≈u iM ⫤ UC ]]$ and mutually $[[Γ ; Θ ⊨ uP ≈u iQ ⫤ UC]]$.
    Let us consider the last rule forming this derivation. 
    \begin{caseof}
        \item \ruleref{\ottdruleUNVarLabel}, then $[[uN]] = [[α⁻]] = [[iM]]$.
        The resulting unification constraint is empty: $[[UC]] = [[·]]$.
        It satisfies $[[Θ ⊢ UC : ·]]$ vacuously, and $[[ [us]α⁻ = α⁻ ]]$, that is $[[ [us]uN = iM ]]$.

        \item \ruleref{\ottdruleUShiftULabel}, then $[[uN]] = [[↑uP]]$ and $[[iM]] = [[↑iQ]]$.
        The algorithm makes a recursive call to $[[Γ ; Θ ⊨ uP ≈u iQ ⫤ UC]]$ returning $[[UC]]$.
        By induction hypothesis, $[[Θ ⊢ UC : uv uP]]$
        and thus, $[[Θ ⊢ UC : uv ↑uP]]$,
        and for any $[[ Θ ⊢ uσ : lift UC ]]$,
        $[[ [uσ]uN ]] = [[ [uσ]↑uP ]] = [[ ↑[uσ]uP ]] = [[ ↑iQ ]] = [[ iM ]]$, as 
        required.

        \item \ruleref{\ottdruleUArrowLabel}, then $[[uN]] = [[uP → uN']]$ and $[[iM]] = [[iQ → iM']]$.
        The algorithm makes two recursive calls to $[[Γ ; Θ ⊨ uP ≈u iQ ⫤ UC1]]$ and
        $[[Γ ; Θ ⊨ uN' ≈u iM' ⫤ UC2]]$ returning $[[Θ ⊢ UC1 & UC2 = UC]]$ as the result.

        It is clear that $[[uP]]$, $[[uN']]$, $[[iQ]]$, and $[[iM']]$ are normalized,
        and that $[[Γ ; dom(Θ) ⊢ uP]]$, $[[Γ ; dom(Θ) ⊢  uN']]$, $[[Γ ⊢ iQ]]$, and $[[Γ ⊢ iM']]$.
        This way, the induction hypothesis is applicable to both recursive calls.

        By applying the induction hypothesis to $[[Γ ; Θ ⊨ uP ≈u iQ ⫤ UC1]]$,
        we have:
        \begin{itemize}
            \item $[[Θ ⊢ UC1 : uv uP]]$,
            \item for any $(Θ  ⊢ lift UC1) ⊢ uσ'$, $[[ [uσ']uP = iQ ]]$.
        \end{itemize}
        By applying it to $[[Γ ; Θ ⊨ uN' ≈u iM' ⫤ UC2]]$, we have:
        \begin{itemize}
            \item $[[Θ ⊢ UC2 : uv uN']]$,
            \item for any $[[ Θ  ⊢ uσ' : lift UC2 ]]$, $[[ [uσ']uN' = iM' ]]$.
        \end{itemize}


        Let us take an arbitrary $[[ Θ ⊢ uσ : lift UC ]]$.
        By the soundness of the constraint merge (\cref{lemma:merge-soundness}), 
        $[[Θ ⊢ lift UC1 & lift UC2 = lift UC]]$ implies
        $[[ Θ   ⊢ uσ : lift UC1  ]]$ and $[[ Θ   ⊢ uσ : lift UC2 ]]$.

        Applying the induction hypothesis to $[[ Θ   ⊢ uσ : lift UC1 ]]$, we have
        $[[ [uσ]uP = iQ ]]$; applying it to $[[ Θ   ⊢ uσ : lift UC2 ]]$, we have
        $[[ [uσ]uN' = iM' ]]$.
        This way, $[[ [uσ]uN ]] = [[ [uσ]uP → [uσ]uN' ]] = [[ iQ → iM' ]] = [[ iM ]]$.

        \item \ruleref{\ottdruleUForallLabel}, then $[[uN]] = [[∀pas.uN']]$ and $[[iM]] = [[∀pas.iM']]$.
        The algorithm makes a recursive call to $[[Γ,pas ; Θ ⊨ uN' ≈u iM' ⫤ UC]]$
        returning $[[UC]]$ as the result.

        The induction hypothesis is applicable: $[[Γ,pas ; dom(Θ) ⊢  uN']]$ and $[[Γ,pas ⊢ iM']]$ hold
        by inversion, and $[[uN']]$ and $[[iM']]$ are normalized, since $[[uN]]$ and $[[iM]]$ are.
        Let us take an arbitrary $[[ Θ   ⊢ uσ : lift UC ]]$.
        By the induction hypothesis, $[[ [uσ]uN' ]] = [[ iM' ]]$. 
        Then $[[ [uσ]uN ]] = [[ [uσ]∀pas.uN' ]] = [[ ∀pas.[uσ]uN' ]] = [[ ∀pas.iM' ]] = [[ iM ]]$.

        \item \ruleref{\ottdruleUNUVarLabel}, then $[[uN]] = [[α̂⁻]]$, $[[â⁻[Δ] ∊ Θ]]$, and $[[Δ ⊢ iM]]$.
        As the result, the algorithm returns $[[UC]] = [[ (â⁻ :≈ iM) ]]$.

        It is clear that $[[α̂⁻[Δ] ⊢ (â⁻ :≈ iM) ]]$, since $[[Δ ⊢ iM]]$, 
        meaning that $[[Θ ⊢ UC]]$.

        Let us take an arbitrary $[[uσ]]$ such that  $[[ Θ   ⊢ uσ : lift UC ]]$.
        Since $[[UC]] = [[ (â⁻ :≈ iM) ]]$, $[[ Θ   ⊢ uσ : lift UC ]]$ implies 
        $[[Θ(â⁻) ⊢ [uσ]â⁻ : (â⁻ :≈ iM) ]]$.
        By inversion of \ruleref{\ottdruleSATSCENEqLabel}, it  means $[[Θ(â⁻) ⊢ [uσ]â⁻ ≈ iM]]$.
        This way, $[[Θ(â⁻) ⊢ [uσ]uN ≈ iM]]$. 
        Notice that $[[uσ]]$ and $[[uN]]$ are normalized, and by \cref{lemma:norm-subst-distr}, 
        so is $[[ [uσ]uN ]]$.
        Since both sides of $[[Θ(â⁻) ⊢ [uσ]uN ≈ iM]]$ are normalized,
        by \cref{lemma:subt-equiv-algorithmization}, we have $[[ [uσ]uN = iM ]]$.

        \item The positive cases are proved symmetrically.
    \end{caseof}
\end{proof}

\begin{lemma}[Completeness of Unification] \label{lemma:unification-completeness}
    \hfill
    \begin{itemize}
        \item [$+$] For normalized $[[uP]]$ and $[[iQ]]$ such that
        $[[Γ ; dom(Θ) ⊢  uP]]$ and $[[Γ ⊢ iQ]]$, 
        for any $[[Θ ⊢ uσ]]$ such that $[[ [uσ]uP = iQ ]]$,
        there exists $[[Γ ; Θ ⊨ uP ≈u iQ ⫤ UC]]$,
        and $[[ Θ   ⊢ uσ : lift UC ]]$.
        
        \item [$-$] For normalized $[[uN]]$ and $[[iM]]$ such that
        $[[Γ ; dom(Θ) ⊢  uN]]$ and $[[Γ ⊢ iM]]$,\\
        for any $[[Θ ⊢ uσ]]$ such that $[[ [uσ]uN = iM ]]$,
        there exists $[[Γ ; Θ ⊨ uN ≈u iM ⫤ UC]]$,
        and $[[ Θ   ⊢ uσ : lift UC ]]$.
   \end{itemize}
\end{lemma}
\begin{proof}
    We prove it by induction on the structure of $[[uP]]$ and mutually, $[[uN]]$.
    \begin{caseof}
        \item $[[uN]] = [[α̂⁻]]$\\
            $[[Γ ; dom(Θ) ⊢  α̂⁻]]$ means that $[[ α̂⁻[Δ] ]] \in [[Θ]]$ for some $[[Δ]]$.

            Let us take an arbitrary $[[Θ ⊢ uσ]]$ such that $[[ [uσ]α̂⁻ = iM ]]$.
            $[[Θ ⊢ uσ]]$ means that $[[Δ ⊢ iM]]$.
            This way, \ruleref{\ottdruleUNUVarLabel} is applicable to infer 
            $[[Γ ; Θ ⊨ â⁻ ≈u iM ⫤ (â⁻ :≈ iM)]]$.
            $[[ Θ   ⊢ uσ : lift (â⁻ :≈ iM) ]]$ holds by \ruleref{\ottdruleSATSCENEqLabel}. 
            
        \item $[[uN]] = [[α⁻]]$\\
            Let us take an arbitrary $[[Θ ⊢ uσ]]$ such that $[[ [uσ]α⁻ = iM ]]$.
            The latter means $[[iM = α⁻]]$.

            Then $[[ [us]α⁻ = iM ]]$ means $[[iM = α⁻]]$.
            This way, \ruleref{\ottdruleUNVarLabel} infers 
            $[[Γ; Θ ⊨ a⁻ ≈u a⁻ ⫤ ·]]$, which is rewritten as $[[Γ; Θ ⊨ uN ≈u iM ⫤ ·]]$, 
            and $[[ Θ   ⊢ uσ : lift · ]]$ holds trivially.

        \item $[[uN]] = [[↑uP]]$\\
            Let us take an arbitrary $[[Θ ⊢ uσ]]$ such that $[[ [uσ]↑uP = iM ]]$.
            The latter means $[[ ↑[uσ]uP = iM ]]$, i.e.
            $[[iM]] = [[↑iQ]]$ for some $[[iQ]]$ and $[[ [uσ]uP = iQ ]]$.

            Let us show that the induction hypothesis is applicable to $[[ [uσ]uP = iQ ]]$.
            Notice that $[[uP]]$ is normalized, since $[[uN]] = [[↑uP]]$ is normalized,
            $[[Γ ; dom(Θ) ⊢  uP]]$ holds by inversion of $[[Γ ; dom(Θ) ⊢  ↑uP]]$, 
            and $[[Γ ⊢ iQ]]$ holds by inversion of $[[Γ ⊢ ↑iQ]]$.

            This way, by the induction hypothesis there exists $[[UC]]$ such that
            $[[Γ ; Θ ⊨ uP ≈u iQ ⫤ UC]]$, and moreover, $[[ Θ   ⊢ uσ : lift UC ]]$.
            
        \item $[[uN]] = [[uP → uN']]$\\
            Let us take an arbitrary $[[Θ ⊢ uσ]]$ such that $[[ [uσ](uP → uN') = iM ]]$.
            The latter means $[[ [uσ]uP → [uσ]uN' = iM ]]$, i.e.
            $[[iM]] = [[iQ → iM']]$ for some $[[iQ]]$ and $[[iM']]$, 
            such that $[[ [uσ]uP = iQ ]]$ and $[[ [uσ]uN' = iM' ]]$.

            Let us show that the induction hypothesis is applicable to 
            $[[ [uσ]uP = iQ ]]$ and to $[[ [uσ]uN' = iM' ]]$:
            \begin{itemize}
                \item $[[uP]]$ and $[[uN']]$ are normalized, since $[[uN]] = [[uP → uN']]$ is normalized
                \item $[[Γ ; dom(Θ) ⊢  uP]]$ and $[[Γ ; dom(Θ) ⊢  uN']]$ follow from the inversion of $[[Γ ; dom(Θ) ⊢  uP → uN']]$,
                \item $[[Γ ⊢ iQ]]$ and $[[Γ ⊢ iM']]$ follow from inversion of $[[Γ ⊢ iQ → iM']]$.
            \end{itemize}

            Then by the induction hypothesis, $[[Γ ; Θ ⊨ uP ≈u iQ ⫤ UC1]]$ and $[[ Θ   ⊢ uσ : lift UC1 ]]$,
            $[[Γ ; Θ ⊨ uN' ≈u iM' ⫤ UC2]]$ and $[[ Θ   ⊢ uσ : lift UC2 ]]$.
            To apply \ruleref{\ottdruleUArrowLabel} and infer the required
            $[[Γ ; Θ ⊨ uN ≈u iM ⫤ UC]]$, we need to show that
            $[[Θ ⊢ UC1 & UC2 = UC]]$ is defined and $[[ Θ   ⊢ uσ : lift UC ]]$.
            It holds by the completeness of the unification constraint merge 
            (\cref{lemma:merge-completeness}):
            \begin{itemize}
                \item $[[Θ ⊢ UC1]]$ and $[[Θ ⊢ UC2]]$ holds by the soundness of unification (\cref{lemma:unification-soundness})
                \item $[[ Θ   ⊢ uσ : lift UC1 ]]$ and $[[ Θ   ⊢ uσ : lift UC2 ]]$ holds as noted above 
            \end{itemize}.

        \item $[[uN]] = [[∀pas.uN']]$\\
            Let us take an arbitrary $[[Θ ⊢ uσ]]$ such that $[[ [uσ]∀pas.uN' = iM ]]$.
            The latter means $[[ ∀pas.[uσ]uN' = iM ]]$, i.e.
            $[[iM]] = [[∀pas.iM']]$ for some $[[iM']]$ such that $[[ [uσ]uN' = iM' ]]$.

            Let us show that the induction hypothesis is applicable to $[[ [uσ]uN' = iM' ]]$.
            Notice that $[[uN']]$ is normalized, since $[[uN]] = [[∀pas.uN']]$ is normalized,
            $[[Γ,pas ; dom(Θ) ⊢  uN']]$ follows from inversion of $[[Γ ; dom(Θ) ⊢  ∀pas.uN']]$,
            $[[Γ,pas ⊢ iM']]$ follows from inversion of $[[Γ ⊢ ∀pas.iM']]$, and
            $[[Θ ⊢ uσ]]$ by assumption. 

            This way, by the induction hypothesis, $[[Γ,pas ; Θ ⊨ uN' ≈u iM' ⫤ UC]]$ exists and 
            moreover, $[[ Θ   ⊢ uσ : lift UC ]]$.
            Hence, \ruleref{\ottdruleUForallLabel} is applicable to infer
            $[[Γ ; Θ ⊨ ∀pas.uN' ≈u ∀pas.iM' ⫤ UC]]$, that is $[[Γ ; Θ ⊨ uN ≈u iM ⫤ UC]]$.

        \item The positive cases are proved symmetrically.
    \end{caseof}
\end{proof}
\

\subsection{Anti-unification}
\begin{lemma}[Soundness of Anti-Unification] \label{lemma:anti-unification-soundness}
    \hfill
    \begin{itemize}
        \item [$+$] 
        \item [$-$] 
    \end{itemize}
\end{lemma}

\begin{lemma}[Completeness of Anti-Unification] \label{lemma:anti-unification-completeness}
    \hfill
    \begin{itemize}
        \item [$+$] 
        \item [$-$] 
    \end{itemize}
\end{lemma}


\subsection{Upper Bounds}
\obsLubDeterministic*
\begin{proof}
  The shape of $[[iP1]]$ and $[[iP2]]$ uniquely determines the rule 
  applied to infer the upper bound.
  By looking at the inference rules,
  it is easy to see that the result of the least upper bound algorithm depends on 
  \begin{itemize}
    \item the inputs of the algorithm (that is $[[iP1]]$, $[[iP2]]$, and $[[Γ]]$),
      which are fixed;
    \item the result of the anti-unification algorithm
      applied to normalized input, which is deterministic
      by \cref{obs:au-deterministic}; 
    \item the result of the recursive call, which is deterministic by the induction hypothesis.
  \end{itemize}
\end{proof}

\lemmaShapeOfSupertypes*
\begin{proof}
  By induction on $[[G ⊢ iP]]$.
  \begin{caseof}
  \item $[[iP]] = [[pb]]$\\
    Immediately from \cref{lemma:var-subt}
  \item $[[iP = ∃nbs.iP']]$\\
    Then if $[[G ⊢ iQ ≥ ∃nbs.iP']]$, then by
    \cref{lemma:quant-rule-decomposition}, $[[G, nbs ⊢ iQ ≥ iP']]$, 
    and $[[fv iQ ∩ {nbs} = ∅]]$ by the convention. The other
    direction holds by \ruleref{\ottdruleDOneExistsLabel}. This way,
    $\{[[iQ]] \mid [[G ⊢ iQ ≥ ∃nbs.iP']] \} = \{[[iQ]] \mid  [[G, nbs ⊢ iQ
    ≥ iP']] \text{ s.t. } [[fv(iQ) ∩ {nbs} = ∅]] \}$. From the induction
    hypothesis, the latter is equal to $\UB([[G, nbs ⊢ iP']])$ not using
    $[[nbs]]$, i.e. $\UB([[G ⊢ ∃nbs.iP']])$.
  \item $[[iP = ↓iM]]$\\
    Then let us consider two subcases upper bounds without outer quantifiers (we
    denote the corresponding set restriction as $|_{\not\exists}$) and upper
    bounds with outer quantifiers ($|_{\exists}$). We prove that for both of
    these groups, the restricted sets are equal.
    % ∃a.P(f a) <=> ∃b∊Im(f).P(b)

    \begin{caseof}
      \item \label{case:sup-shape-down-zero}
      $[[iQ]] \neq [[∃nbs.iQ']]$\\
      Then the last applied rule to infer
      $[[G ⊢ iQ ≥ ↓iM]]$ must be \ruleref{\ottdruleDOneShiftDLabel},
      which means $[[iQ]] = [[↓iM']]$, and by inversion, $[[G ⊢ iM' ≈ iM]]$,
      then by \cref{lemma:equiv-completeness} and
      \ruleref{\ottdruleEOneShiftDLabel}, $[[↓iM' ≈ ↓iM]]$.
      This way, $[[iQ]] = [[↓iM']] \in \{ [[↓iM']] \mid [[↓iM' ≈ ↓iM]] \} = \UB([[Γ⊢↓iM]])|_{\not\exists}$.

      In the other direction,
      $
      \begin{aligned}[t]
        [[↓iM' ≈ ↓iM]] &\Rightarrow [[G ⊢ ↓iM' ≈ ↓iM]]
                       && \text{by \cref{lemma:equiv-soundness,lemma:wf-equiv}}\\
                       &\Rightarrow [[G ⊢ ↓iM' ≥ ↓iM]]
                       && \text{by inversion}
      \end{aligned}
      $
      \item $[[iQ]] = [[∃nbs.iQ']]$ (for non-empty $[[nbs]]$)\\
        Then the last rule applied to infer $[[G ⊢ ∃nbs.iQ' ≥ ↓iM]]$
        must be \ruleref{\ottdruleDOneExistsLabel}.
        Inversion of this rule gives us $[[G ⊢ [iNs/nbs]iQ' ≥ ↓iM]]$
        for some $[[G ⊢ iNi]]$. Notice that $[[ [iNs/nbs]iQ' ]]$ has no outer
        quantifiers. Thus from \cref{case:sup-shape-down-zero},
        $[[ [iNs/nbs]iQ' ≈ ↓iM ]]$, which is only possible if $[[iQ']] = [[↓iM']]$.
        This way, $[[iQ]] = [[∃nbs.↓iM']] \in \UB([[Γ⊢↓iM]])|_{\exists}$ (notice
        that $[[nbs]]$ is not empty).

        In the other direction,

        $
        \begin{aligned}[t]
          [[ [iNs/nbs]↓iM' ≈ ↓iM]] &\Rightarrow [[G ⊢ [iNs/nbs] ↓iM' ≈ ↓iM]]
          && \text{by \cref{lemma:equiv-soundness,lemma:wf-equiv}}\\
                                  &\Rightarrow [[G ⊢ [iNs/nbs]↓iM' ≥ ↓iM]]
         && \text{by inversion}\\
                                  &\Rightarrow [[G ⊢ ∃nbs.↓iM' ≥ ↓iM]] 
         && \text{by \ruleref{\ottdruleDOneExistsLabel}}\\
        \end{aligned}
        $
    \end{caseof}
    
  \end{caseof}
\end{proof}

\lemmaShapeOfNormalizedSupertypes*
\begin{proof}
  By induction on $[[G ⊢ iP]]$.
  \begin{caseof}
  \item $[[iP]] = [[pb]]$\\
    Then from \cref{lemma:shape-of-supertypes},
    $\{[[nf(iQ)]]\ \mid \ [[G ⊢ iQ ≥ pb]] \} = \{[[ nf(∃nas.pb) ]] \ \mid \
    \text{for some }[[nas]]\}  = \{[[pb]]\}$ 
  \item $[[iP = ∃nbs.iP']]$\\
    $
    \begin{aligned}[t]
      & \NFUB([[Γ ⊢ ∃nbs.iP']]) \\
                              &= \NFUB([[Γ, nbs ⊢ iP']]) \text{ not using $[[nbs]]$}\\
                              &= \{ [[nf(iQ)]] \mid [[Γ, nbs ⊢ iQ ≥ iP']]  \}
                                \text{ not using $[[nbs]]$}
                              && \text{by the induction hypothesis}\\
                              &= \{ [[nf(iQ)]] \mid [[Γ, nbs ⊢ iQ ≥ iP']]
                                \text{ s.t. $[[fv iQ]] \cap [[nbs]] = \emptyset$}
                                \}
                             && \text{$[[fv nf(iQ)]] = [[fv iQ]]$ by \cref{lemma:fv-nf}}\\
                              &= \{ [[nf(iQ)]] \mid [[iQ]] \in \UB([[Γ, nbs ⊢ iP']]) \text{ s.t. $[[fv iQ]] \cap [[nbs]] = \emptyset$}
                                \}
                            && \text{by \cref{lemma:shape-of-supertypes}}\\
                              &= \{ [[nf(iQ)]] \mid [[iQ]] \in \UB([[Γ ⊢ ∃nbs.iP']])
                                \}
                              && \text{by the definition of $\UB{}$}\\
                              &= \{ [[nf(iQ)]] \mid [[Γ ⊢ iQ ≥ ∃nbs.iP']]
                                \}
                              && \text{by \cref{lemma:shape-of-supertypes}}\\
    \end{aligned}
    $
  
  \item $[[iP = ↓iM]]$ Let us prove the set equality by two inclusions.
  \begin{itemize}
    \item [$\subseteq$]
      Suppose that $[[Γ ⊢ iQ ≥ ↓iM]]$ and $[[iM]]$ is normalized.

      By \cref{lemma:shape-of-supertypes},
      $[[iQ]] \in \UB([[Γ ⊢ ↓iM]])$.
      Then by definition of $\UB{}$,
      $[[iQ = ∃nas.↓iM']]$ 
      for some $[[nas]]$, $[[iM']]$, and $[[Γ ⊢ σ :{nas}]]$ s.t.  
      $[[ [σ] ↓iM' ≈ ↓iM ]]$.

      We need to show that $[[nf(iQ)]] \in \NFUB([[Γ ⊢ ↓iM]])$.
      Notice that $[[nf(iQ)]] = [[nf(∃nas.↓iM')]] = [[∃nas0.↓iM0]]$, 
      where $[[nf(iM') = iM0]]$ and $[[ord {nas} in iM0 = nas0]]$.

      The belonging of $[[∃nas0.↓iM0]]$ to $\NFUB([[Γ ⊢ ↓iM]])$ means that
      \begin{enumerate}
        \item $[[ord {nas0} in iM0 = nas0]]$ and
        \item that there exists $[[Γ ⊢ σ0 :{nas0}]]$ such that 
          $[[ [σ0] ↓iM0 = ↓iM ]]$.
      \end{enumerate}
      The first requirement holds by \cref{corollary:ord-idemp}.
      To show the second requirement, we construct
      $[[σ0]]$ as $[[nf(σ|fv iM')]]$.
      Let us show the required properties of $[[σ0]]$:
      \begin{enumerate}
        \item $[[Γ ⊢ σ0 :{nas0}]]$. 
          Notice that by \cref{lemma:subst-restr-sig},
          $[[Γ ⊢ σ|fv(iM') :  {nas} ∩ fv(iM')]]$, 
          which we rewrite as 
          $[[Γ ⊢ σ|fv(iM') :{nas0}]]$ 
          (since by \cref{lemma:ord-soundness}
          $[[{nas0} = {nas} ∩ fv iM0]]$ as sets, 
          and $[[fv(iM0) = fv(iM')]]$ by \cref{lemma:fv-nf}).
          Then by \cref{lemma:norm-subst-sig}, 
          $[[Γ ⊢ nf(σ|fv(iM')) :{nas0}]]$,
          that is $[[Γ ⊢ σ0 :{nas0}]]$.
        \item $[[ [σ0] ↓iM0 = ↓iM ]]$.
          $[[ [σ] ↓iM' ≈ ↓iM ]]$ means 
          $[[ [σ|fv(iM')]↓iM' ≈ ↓iM ]]$ by 
          \cref{lemma:subst-restr-fv}.
          Then by \cref{lemma:decl-equiv-algorithmization}, 
          $[[ nf([σ|fv(iM')]↓iM') = nf(↓iM) ]]$, 
          implying $[[ [σ0] ↓iM0 = nf(↓iM) ]]$
          by \cref{lemma:norm-subst-distr},
          and further $[[ [σ0] ↓iM0 = ↓iM ]]$
          by \cref{lemma:norm-idemp} (since $[[↓iM]]$ is normal by assumption).
      \end{enumerate}
    
    \item [$\supseteq$]
      Suppose that a type belongs to $\NFUB([[Γ ⊢ ↓iM]])$ for a normalized 
      $[[↓iM]]$.
      Then it must have shape $[[∃nas0.↓iM0]]$ for some $[[nas0]]$, $[[iM0]]$,
      and $[[Γ ⊢ σ0 :{nas0}]]$ such that 
      $[[ ord {nas0} in iM0 = nas0 ]]$ and $[[ [σ0] ↓iM0 = ↓iM ]]$.
      It suffices to show that 
      \begin{enumerate*}
        \item $[[∃nas0.↓iM0]]$ is normalized itself, and 
        \item $[[Γ ⊢ ∃nas0.↓iM0 ≥ ↓iM]]$.
      \end{enumerate*}

      \begin{enumerate}
        \item By definition, 
          $[[nf(∃nas0.↓iM0) = ∃nas1.↓iM1]]$, 
          where $[[iM1 = nf(iM0)]]$ and \\
          $[[ord {nas0} in iM1 = nas1]]$.
          First, notice that by 
          \cref{lemma:ord-completeness,lemma:normalization-soundness}, 
          $[[ord {nas0} in iM1]] = [[ord {nas0} in iM0]] = [[nas0]]$. 
          This way, $[[nf(∃nas0.↓iM0) = ∃nas0.↓nf(iM0)]]$.
          Second, $[[iM0]]$ is normalized by \cref{lemma:normal-after-subst}, 
          since $[[ [σ0] ↓iM0 = ↓iM ]]$ is normal. 
          As such,\\ $[[nf(∃nas0.↓iM0) = ∃nas0.↓iM0]]$, 
          in other words, $[[∃nas0.↓iM0]]$ is normalized.
        \item $[[Γ ⊢ ∃nas0.↓iM0 ≥ ↓iM]]$ holds immediately by 
          \ruleref{\ottdruleDOneExistsLabel} with the substitution 
          $[[σ0]]$. Notice that $[[Γ ⊢ [σ0]↓iM0 ≥ ↓iM]]$
          follows from $[[ [σ0] ↓iM0 = ↓iM ]]$
          by reflexivity of subtyping (\cref{lemma:subtyping-reflexivity}).
      \end{enumerate}



    \end{itemize}
  \end{caseof}
\end{proof}

\lemUbContextIrrelevant*
\begin{proof}
  We prove both inclusions by structural induction on  
  $[[iP]]$.
  \begin{caseof}
    \item $[[iP]] = [[pb]]$
      Then $\UB([[Γ1 ⊢ pb]]) = \UB([[Γ2 ⊢ pb]]) = 
      \{[[∃nas.pb]] \mid \text{for some }[[nas]]\}$.
      $\NFUB([[Γ1 ⊢ pb]]) = \NFUB([[Γ2 ⊢ pb]]) = \{[[pb]]\}$.
    \item $[[iP]] = [[∃nbs.iP']]$.
      Then $\UB([[Γ1 ⊢ ∃nbs.iP']]) = \UB([[Γ1, nbs ⊢ iP']])$ not using $[[nbs]]$.
      $\UB([[Γ2 ⊢ ∃nbs.iP']]) = \UB([[Γ2, nbs ⊢ iP']])$ not using $[[nbs]]$.
      By the induction hypothesis, $\UB([[Γ1, nbs ⊢ iP']]) = \UB([[Γ2, nbs ⊢ iP']])$,
      and if we restrict these sets to the same domain, they stay equal.
      Analogously, $\NFUB([[Γ1 ⊢ ∃nbs.iP']]) = \NFUB([[Γ2 ⊢ ∃nbs.iP']])$.
    \item $[[iP]] = [[↓iM]]$.
      Suppose that $[[∃nas.↓iM']] \in \UB([[Γ1 ⊢ ↓iM]])$. It means that 
      $[[Γ1, nas ⊢ iM']]$ and there exist $[[Γ1 ⊢ iNs]]$ s.t. 
      $[[ [iNs/nas] ↓iM' ≈ ↓iM ]]$, or in other terms, 
      there exists $[[Γ1 ⊢ σ :{nas}]]$ such that $[[ [σ] ↓iM' ≈ ↓iM ]]$.

      We need to show that $[[∃nas.↓iM']] \in UB([[Γ2 ⊢ ↓iM]])$,  
      in other words, $[[Γ2, nas ⊢ iM']]$ and there exists
      $[[Γ2 ⊢ σ0 :{nas}]]$ such that $[[ [σ0] ↓iM' ≈ ↓iM ]]$.

      First, let us show $[[Γ2, nas ⊢ iM']]$. 
      Notice that $[[ [σ] ↓iM' ≈ ↓iM ]]$ implies $[[ fv([σ]iM') = fv(↓iM) ]]$ 
      by \cref{lemma:equiv-fv}. By \cref{lemma:subst-fv},
      $[[ fv(iM') \ {nas} ⊆ fv([σ]iM') ]]$. This way, 
      $[[ fv(iM') \ {nas} ⊆ fv(iM) ]]$,
      implying $[[ fv(iM') ⊆ fv(iM) ∪ {nas} ]]$.
      By \cref{lemma:wf-soundness}, $[[Γ2 ⊢ ↓iM]]$ implies $[[fv iM ⊆ Γ2]]$,
      hence, $[[fv iM' ⊆ (Γ2, nas)]]$, which by \cref{corollary:wf-ctxt-strengthening}
      means $[[Γ2, nas ⊢ iM']]$.
      
      Second, let us construct the required $[[σ0]]$ in the following way:
      $$
      \begin{cases}
          [[ [σ0]αi⁻ = [σ]αi⁻  ]] & \text{for $[[αi⁻ ∊ {nas} ∩ fv(iM')]]$ }\\
          [[ [σ0]αi⁻ = ∀γ⁺.↑γ⁺ ]] & \text{for $[[αi⁻ ∊ {nas} \ fv(iM')]]$ }\\
          [[ [σ0]γ±  = γ± ]]      & \text{for any other $[[γ±]]$ }\\
      \end{cases}
      $$
      This construction of a substitution coincides with 
      the one from the proof of \cref{lemma:subt-ctxt-irrelevance}.
      This way, for $[[σ0]]$, hold the same properties:
      \begin{enumerate}
        \item $[[ [σ0]iM' = [σ]iM' ]]$,
          which in particular, implies $[[ [σ0]↓iM = [σ]↓iM ]]$,
          and thus, $[[ [σ]↓iM' ≈ ↓iM ]]$ can be rewritten to
          $[[ [σ0]↓iM' ≈ ↓iM ]]$; and
        \item $[[  fv([σ]iM') ⊢ σ0 :{nas}]]$,
          which, as noted above, can be rewritten to 
          $[[  fv(iM) ⊢ σ0 :{nas}]]$,
          and since $[[fv iM ⊆ Γ2]]$, 
          weakened to $[[ Γ2 ⊢ σ0 :{nas}]]$.
      \end{enumerate}

      The proof of $\NFUB([[Γ1 ⊢ ↓iM]]) \subseteq \NFUB([[Γ2 ⊢ ↓iM]])$
      is analogous.
      The differences are:
      \begin{enumerate}
        \item $[[ord {nas} in iM' = nas]]$ holds by assumption, 
        \item $[[ [σ] ↓iM' = ↓iM ]]$ implies $[[ fv([σ]iM') = fv(↓iM) ]]$ by rewriting,
        \item $[[ [σ] ↓iM' = ↓iM ]]$ and $[[ [σ0]↓iM = [σ]↓iM ]]$
          imply $[[ [σ0] ↓iM' = ↓iM ]]$ by rewriting.
      \end{enumerate}
  \end{caseof}
\end{proof}

\lemmaLubSoundness*
\begin{proof}
  Induction on $[[Γ ⊨ iP1 ∨ iP2 = iQ]]$.
  \begin{caseof}
  \item $[[Γ ⊨ pa ∨ pa = pa]]$\\
     Then $[[Γ ⊢ pa]]$ by assumption, and
     $[[Γ ⊢ pa ≥ pa]]$ by \ruleref{\ottdruleDOnePVarLabel}.
   \item $[[Γ ⊨ ∃nas.iP1 ∨ ∃nbs.iP2 = iQ]]$\\
     Then by inversion of $[[Γ ⊢ ∃nas.iPi]]$  and
     weakening, $[[Γ, {nas}, {nbs} ⊢ iPi]]$, hence, the induction
     hypothesis applies to $[[Γ, {nas}, {nbs} ⊨ iP1 ∨ iP2 = iQ]]$. Then
     \begin{itemize}
       \item[(i)] $[[Γ, {nas}, {nbs} ⊢ iQ]]$,
       \item[(ii)] $[[Γ, {nas}, {nbs} ⊢ iQ ≥ iP1]]$,
       \item[(iii)] $[[Γ, {nas}, {nbs} ⊢ iQ ≥ iP2]]$.
     \end{itemize}

     To prove $[[Γ ⊢ iQ]]$, it suffices to show that
     $[[fv(iQ) ∩ (Γ, {nas}, {nbs})]] = [[fv(iQ) ∩ Γ]]$ (and then apply \cref{lemma:wf-ctxt-equiv}).
     The inclusion right-to-left is self-evident. To show
     $[[fv(iQ) ∩ (Γ, {nas}, {nbs})]] \subseteq [[fv(iQ) ∩ Γ]]$, we prove that 
     $[[fv(iQ)]] \subseteq [[Γ]]$.

     $
     \begin{aligned}[t]
       [[fv(iQ)]] &\subseteq [[fv iP1 ∩ fv iP2]]
                    && \text{by \cref{lemma:fv-propagation}}\\
                  &\subseteq [[((Γ, nas) \ {nbs}) ∩ ((Γ, nbs) \ {nas})]]
                    && \text{since } [[Γ ⊢ ∃nas.iP1]],~ [[fv(iP1)]]
                        \subseteq [[(Γ, nas)]] = \\ 
                  & && [[(Γ, nas) \ {nbs}]]
                        \text{(the latter is because by the}\\ 
                  & && \text{Barendregt's convention, $[[(Γ, nas) ∩ {nbs} = ∅]]$)}\\
                  & && \text{similarly, $[[fv(iP2)]] \subseteq [[(Γ, nbs) \ {nas}]]$}\\
                  &\subseteq [[Γ]]
     \end{aligned}
     $

     To show $[[Γ ⊢ iQ ≥ ∃nas.iP1]]$, we apply
     \ruleref{\ottdruleDOneExistsLabel}.
     Then $[[Γ, nas ⊢ iQ ≥ iP1]]$ holds since
     $[[Γ, {nas}, {nbs} ⊢ iQ ≥ iP1]]$ (by the induction hypothesis),
     $[[Γ, nas ⊢ iQ]]$ (by weakening), and $[[Γ, nas ⊢ iP1]]$.

     Judgment $[[Γ ⊢ iQ ≥ ∃nbs.iP2]]$ is proved symmetrically.
  \item $[[Γ ⊨ ↓iN ∨ ↓iM = ∃nas.[nas / ToList Ξ]uP]]$.
    By the inversion, $[[G,· ⊨ nf(↓iN) ≈au nf(↓iM) ⫤ (Ξ, uP, aus1, aus2)]]$.
    Then by the soundness of anti-unification (\cref{lemma:au-soundness}),
    \begin{enumerate}
    \item[(i)] $[[Γ ; Ξ ⊢ uP]]$, then
      by \cref{lemma:var-dealgo-wf},
      \begin{equation} \label{fact:nas-uP-is-wf} [[Γ, nas ⊢ [nas / ToList Ξ]uP]] \end{equation}
    \item[(ii)] $[[Γ ; · ⊢ aus1 : Ξ]]$ and $[[Γ ; · ⊢ aus2 : Ξ]]$.
      Assuming that $[[Ξ]] = [[nub1,..,nubn]]$,
      the antiunification solutions $[[aus1]]$ and $[[aus2]]$ can be
      put explicitly as $[[aus1]] = [[(nub1 :≈ iN1,..,nubn :≈ iNn)]]$,
      and $[[aus2]] = [[(nub1 :≈ iM1,..,nubn :≈ iMn)]]$.
      Then
      \begin{equation}
        \label{fact:aus1-is-compose}
        [[ aus1 ]] = [[ (iNs / nas) ○ (nas / ToList Ξ) ]] 
      \end{equation}
      \begin{equation}
        \label{fact:aus2-is-compose}
        [[ aus2 ]] = [[ (iMs / nas) ○ (nas / ToList Ξ) ]]
      \end{equation}
    \end{enumerate}
  \item[(iii)] $[[ [aus1] uQ = iP1 ]]$ and $[[ [aus2] uQ = iP1 ]]$,
    which, by \ref{fact:aus1-is-compose} and \ref{fact:aus2-is-compose},
    means
    \begin{equation}
      \label{fact:sub-sub-uP-iN}
      [[ [iNs / nas][nas / ToList Ξ]uP = nf(↓iN) ]]
    \end{equation}
    \begin{equation}
      \label{fact:sub-sub-uP-iM}
      [[ [iMs / nas][nas / ToList Ξ]uP = nf(↓iM) ]]
    \end{equation}

    Then $[[Γ ⊢ ∃nas.[nas / ToList Ξ]uP]]$
    follows directly from \ref{fact:nas-uP-is-wf}.

    To show $[[Γ ⊢ ∃nas.[nas / ToList Ξ]uP ≥ ↓iN]]$,
    we apply \ruleref{\ottdruleDOneExistsLabel},
    instantiating $[[nas]]$ with $[[iNs]]$.
    Then $[[Γ ⊢ [iNs / nas][nas / ToList Ξ]uP ≥ ↓iN ]]$ follows
    from \ref{fact:sub-sub-uP-iN} and 
    since $[[Γ ⊢ nf(↓iN) ≥ ↓iN]]$ (by \cref{corollary:nf-sound-wrt-subt-equiv}).

    Analogously, instantiating $[[nas]]$ with $[[iMs]]$,
    gives us $[[Γ ⊢ [iMs / nas][nas / ToList Ξ]uP ≥ ↓iM ]]$
    (from \ref{fact:sub-sub-uP-iM}), and hence,
    $[[Γ ⊢ ∃nas.[nas / ToList Ξ]uP ≥ ↓iM]]$.

  \end{caseof}

\end{proof}

\lemmaLubCompleteness*
\begin{proof}
  Induction on the pair $([[iP1]], [[iP2]])$.
  From \cref{lemma:shape-supertypes-norm},
  $[[iQ]] \in \UB([[Γ ⊢ iP1]]) \cap \UB([[Γ ⊢ iP2]])$.
  Let us consider the cases of what $[[iP1]]$ and $[[iP2]]$ are (i.e. the last
  rules to infer $[[Γ ⊢ iPi]]$).
  \begin{caseof}
    \item $[[iP1]] = [[∃nbs1.iQ1]]$, $[[iP2]] = [[∃nbs2.iQ2]]$, where either
      $[[nbs1]]$ or $[[nbs2]]$ is not empty\\
      \label{case:ub-completeness-exists}

      Then\\
      $
      \begin{aligned}[t]
        [[iQ]] &\in         \UB([[Γ ⊢ ∃nbs1.iQ1]]) \cap \UB([[Γ ⊢ ∃nbs2.iQ2]]) \\
              & \subseteq  \UB([[Γ, nbs1 ⊢ iQ1]]) \cap \UB([[Γ, nbs2 ⊢ iQ2]])
              && \text{definition of $\UB{}$}\\
              & =  \UB([[Γ, {nbs1}, {nbs2} ⊢ iQ1]]) \cap \UB([[Γ, {nbs1}, {nbs2} ⊢ iQ2]])
              && \text{by \cref{observation:ub-context-irrelevant}}\\
              & = \{[[iQ']]\ \mid \ [[Γ, {nbs1}, {nbs2}  ⊢ iQ' ≥ iQ1]] \} \cap
                  \{[[iQ']]\ \mid \ [[Γ, {nbs1}, {nbs2}  ⊢ iQ' ≥ iQ2]] \}
              && \text{by \cref{lemma:shape-of-supertypes}}\\
      \end{aligned}
      $\\
      It means that $[[Γ, {nbs1}, {nbs2} ⊢ iQ ≥ iQ1]]$ and $[[Γ, {nbs1}, {nbs2} ⊢ iQ ≥ iQ2]]$. 
      Then the next step of the algorithm---the recursive call 
      $[[Γ, {nbs1}, {nbs2} ⊨ iQ1 ∨ iQ2 = iQ']]$
      terminates by the induction hypothesis, 
      and moreover, $[[ Γ, {nbs1}, {nbs2} ⊢ iQ ≥ iQ' ]]$.
      This way, the result of the algorithm is $[[iQ']]$, i.e.
      $[[Γ ⊨ iP1 ∨ iP2 = iQ']]$.

      Since both $[[iQ]]$ and $[[iQ']]$ are sound upper bounds,
      $[[Γ ⊢ iQ]]$ and $[[Γ ⊢ iQ']]$, and therefore,
      $[[ Γ, {nbs1}, {nbs2} ⊢ iQ ≥ iQ' ]]$ can be strengthened to
      $[[ Γ ⊢ iQ ≥ iQ' ]]$ by \cref{lemma:subt-ctxt-irrelevance}.

    \item $[[iP1]] = [[pa]]$ and $[[iP2]] = [[↓iN]]$\\
      \label{case:ub-completeness-unmatching}
      Then the set of common upper bounds of $[[↓iN]]$ and $[[pa]]$
      is empty, and thus, $[[iQ]] \in \UB([[Γ ⊢ iP1]]) \cap \UB([[Γ ⊢ iP2]])$
      gives a contradiction:\\
      $
      \begin{aligned}[t]
        [[iQ]] &\in         \UB([[Γ ⊢ pa]]) \cap \UB([[Γ ⊢ ↓iN]]) \\
              & = \{[[∃nas.pa]]\  \mid \cdots \} \cap
                  \{[[∃nbs.↓iM']]\ \mid \cdots \}
              && \text{by the definition of $\UB{}$}\\
              & = \emptyset
              && \text{since $[[pa]] \neq [[↓iM']]$ for any $[[iM']]$}\\
      \end{aligned}
      $
    \item $[[iP1]] = [[↓iN]]$ and $[[iP2]] = [[pa]]$\\
      Symmetric to \cref{case:ub-completeness-unmatching}

    \item $[[iP1]] = [[pa]]$ and $[[iP2]] = [[pb]]$ (where $[[pb]] \neq [[pa]]$)\\
      Similarly to \cref{case:ub-completeness-unmatching},
      the set of common upper bounds is empty, which leads to the contradiction:

      $
      \begin{aligned}[t]
      [[iQ]] &\in         \UB([[Γ ⊢ pa]]) \cap \UB([[Γ ⊢ pb]]) \\
            & = \{[[∃nas.pa]]\  \mid \cdots \} \cap
                \{[[∃nbs.pb]]\ \mid \cdots \}
            && \text{by the definition of $\UB{}$}\\
            & = \emptyset
            && \text{since $[[pa]] \neq [[pb]]$}
      \end{aligned}
      $
    \item $[[iP1]] = [[pa]]$ and $[[iP2]] = [[pa]]$\\
      Then the algorithm terminates in one step (\ruleref{\ottdruleLUBVarLabel})
      and the result is $[[pa]]$, i.e. $[[G ⊨ pa ∨ pa = pa]]$.

      Since $[[iQ]] \in \UB([[Γ ⊢ pa]])$,
      $[[iQ]] = [[∃nas.pa]]$.
      Then $[[Γ ⊢ ∃nas.pa ≥ pa]]$ by \ruleref{\ottdruleDOneExistsLabel}:
      $[[nas]]$ can be instantiated with arbitrary negative types (for example
      $[[∀β⁺.↑β⁺]]$), since the substitution for unused variables does not change the term
      $[[ [iNs/nas]pa]] = [[pa]]$,
      and then $[[Γ ⊢ pa ≥ pa]]$ by \ruleref{\ottdruleDOnePVarLabel}.

    \item $[[iP1]] = [[↓iM1]]$ and $[[iP2]] = [[↓iM2]]$ \label{case:ub-completeness-shift}\\
      Then on the next step, the algorithm tries to anti-unify $[[nf(↓iM1)]]$ and
      $[[nf(↓iM2)]]$. By \cref{lemma:au-completeness}, to show that the
      anti-unification algorithm terminates, it suffices to
      demonstrate that a sound anti-unification solution exists.

      Notice that

      $
      \begin{aligned}[t]
        [[nf(iQ)]] &\in \NFUB([[Γ ⊢ nf(↓iM1)]]) \cap \NFUB([[Γ ⊢ nf(↓iM2)]]) \\
              &= \NFUB([[Γ ⊢ ↓nf(iM1)]]) \cap \NFUB([[Γ ⊢ ↓nf(iM2)]]) \\
              &=           \begin{array}{l}
                              \Set{ [[ ∃nas.↓iM' ]] \ | \begin{array}{l}
                                                          \text{for $[[nas]]$, $[[iM']]$, and $[[iNs]]$ s.t. $[[ord {nas} in iM' = nas]]$,}\\
                                                          \text{$[[G ⊢ iNi]]$, $[[G,nas ⊢ iM']]$,  and $[[ [iNs/nas] ↓iM' = ↓nf(iM1) ]]$}
                                                        \end{array}}\\ \cap\\
                              \Set{ [[ ∃nas.↓iM' ]] \ | \begin{array}{l}
                                                          \text{for $[[nas]]$, $[[iM']]$, and $[[iNs]]$ s.t. $[[ord {nas} in iM' = nas]]$,}\\
                                                          \text{$[[G ⊢ iNs1]]$,
                                                          $[[G ⊢ iNs2]]$, $[[G,nas ⊢ iM']]$,  and $[[ [iNs/nas] ↓iM' = ↓nf(iM2) ]]$}
                                                        \end{array}}
                            \end{array}\\
                &=
                  \Set{ [[ ∃nas.↓iM' ]] \ | \begin{array}{l}
                                              \text{for $[[nas]]$, $[[iM']]$,
                                              $[[iNs1]]$ and $[[iNs2]]$ s.t. $[[ord {nas} in iM' = nas]]$,}\\
                                              \text{$[[G ⊢ iNs1]]$, $[[G ⊢ iNs2]]$, $[[G,nas ⊢ iM']]$,
                                              $[[ [iNs1/nas] ↓iM' = ↓nf(iM1)]]$}\\
                                              \text{, and $[[ [iNs2/nas] ↓iM' = ↓nf(iM2)]]$ }
                                            \end{array}}\\
      \end{aligned}
      $\\
      The fact that the latter set is non-empty means that there exist $[[nas]],
      [[iM']]$, $[[iNs1]]$ and $[[iNs2]]$ such that
      \begin{enumerate}
      \item[(i)] $[[G,nas ⊢ iM']]$ (notice that $[[iM']]$ is normal)
      \item[(ii)] $[[G ⊢ iNs1]]$ and $[[G ⊢ iNs1]]$,
      \item[(iii)] $[[ [iNs1/nas] ↓iM' = ↓nf(iM1)]]$ and $[[ [iNs2/nas] ↓iM' = ↓nf(iM2)]]$
      \end{enumerate}

      For each negative variable $[[α⁻]]$ from $[[nas]]$, let us choose a
      fresh negative anti-unification variable $[[α̂⁻]]$, and denote the
      list of these variables as $[[nuas]]$.
      Let us show that\\ $([[nuas]],~ [[ [nuas/nas]↓iM' ]],~ [[iNs1/nuas]],~ [[iNs2/nuas]])$ is a
      sound anti-unifier of $[[nf(↓iM1)]]$ and $[[nf(↓iM2)]]$ in context $[[Γ]]$:

      \begin{itemize}
        \item $[[nuas]]$ is negative by construction,
        \item $[[Γ ; {nuas} ⊢ [nuas/nas]↓iM']]$ because $[[Γ, nas ⊢ ↓iM']]$ 
        (\cref{lemma:var-algo-wf}),
        \item $[[Γ ; · ⊢ (iNs1/nuas) :{nuas}]]$ because $[[Γ ⊢ iNs1]]$ and
          $[[Γ ; · ⊢ (iNs2/nuas) :{nuas}]]$ because $[[Γ ⊢ iNs2]]$,
        \item $[[ [iNs1/nuas] [nuas/nas] ↓iM' ]] = [[ [iNs1/nas] ↓iM' ]] =
          [[↓nf(iM1)]] = [[nf(↓iM1)]]$.
        \item $[[ [iNs2/nuas] [nuas/nas] ↓iM' ]] = [[ [iNs2/nas] ↓iM' ]] = [[↓nf(iM2)]] = [[nf(↓iM2)]]$.
      \end{itemize}

      Then by the completeness of the anti-unification
      (\cref{lemma:au-completeness}), the anti-unification algorithm
      terminates, so is the Least Upper Bound algorithm invoking it, 
      i.e. $[[iQ']] = [[∃nbs.[nbs / ToList Ξ]uP]]$, where
      $[[(Ξ, uP, aus1, aus2)]]$ is the result of the anti-unification
      of $[[nf(↓iM1)]]$ and $[[nf(↓iM2)]]$ in context $[[Γ]]$.

      Moreover, \cref{lemma:au-completeness} also says that the found anti-unification 
      solution is initial, i.e. there exists $[[aus]]$ such that
      $[[Γ;Ξ ⊢ aus :{nuas}]]$ and $[[ [aus][nuas/nas]↓uM' = uP ]]$.

      Let $[[σ]]$ be a sequential Kleisli composition of the following
      substitutions:
      \begin{enumerate*}
      \item[(i)] $[[nuas/nas]]$,
      \item[(ii)] $[[aus]]$, and
      \item[(iii)] $[[nbs / ToList Ξ]]$.
      \end{enumerate*}
      Notice that $[[Γ, nbs ⊢ σ :{nas}]]$
      and $[[ [σ]↓uM' ]] = [[ [nbs / ToList Ξ][aus][nuas/nas]↓uM' ]] = [[ [nbs /
      ToList Ξ]uP ]]$. In particular, from the reflexivity of subtyping:
      $[[Γ, nbs ⊢ [σ]↓iM' ≥ [nbs / ToList Ξ]uP]]$.

      It allows us to show $[[Γ ⊢ nf(iQ) ≥ iQ']]$, i.e. $[[Γ ⊢ ∃nas.↓iM' ≥
      ∃nbs.[nbs / ToList Ξ]uP]]$, by applying \ruleref{\ottdruleDOneExistsLabel},
      instantiating $[[nas]]$ with respect to $[[σ]]$. Finally, $[[Γ ⊢ iQ ≥ iQ']]$
      by transitively combining $[[Γ ⊢ nf(iQ) ≥ iQ']]$ and $[[Γ ⊢ iQ ≥ nf(iQ)]]$ 
      (holds by \cref{corollary:nf-sound-wrt-subt-equiv} and inversion).
  \end{caseof}
\end{proof}

\subsection{Upgrade}
Let us consider a type $[[iP]]$ well-formed in $[[Γ]]$.
Some of its $[[Γ]]$-supertypes are also well-formed in a smaller context $[[{Δ} ⊆ Γ]]$.
The upgrade is the operation that returns the least of such supertypes.

\begin{observation}[Upgrade is deterministic]
    \label{obs:upgrade-deterministic}
    Assuming $[[iP]]$ is well-formed in $[[Γ ⊆ Δ]],$\\
    if $[[upgrade Γ ⊢ iP to Δ = iQ]]$ and $[[upgrade Γ ⊢ iP to Δ = iQ']]$ are defined 
    then $[[iQ = iQ']]$.
\end{observation}
\begin{proof}
    It follows directly from \cref{obs:lub-deterministic},
    and the convention that the fresh variables are chosen by a fixed deterministic algorithm
    (\cref{sec:fresh-selection}).
\end{proof}

\begin{lemma}[Soundness of Upgrade]\label{lemma:upgrade-soundness}
    Assuming $[[iP]]$ is well-formed in $[[Γ = Δ, pnas]]$,
    if $[[upgrade Γ ⊢ iP to Δ = iQ]]$
    then
    \begin{enumerate}
        \item $[[Δ ⊢ iQ]]$
        \item $[[Γ ⊢ iQ ≥ iP]]$
    \end{enumerate}
\end{lemma}
\begin{proof}
    By inversion, $[[upgrade Γ ⊢ iP to Δ = iQ]]$ means that 
    for fresh $[[pnbs]]$ and $[[pncs]]$,
    $[[Δ, pnbs, pncs ⊨ [pnbs/pnas]iP ∨ [pncs/pnas]iP = iQ]]$.
    Then by the soundness of the least upper bound (\cref{lemma:lub-soundness}),
    \begin{enumerate}
        \item $[[Δ, pnbs, pncs ⊢ iQ]]$, 
        \item $[[Δ, pnbs, pncs ⊢ iQ ≥ [pnbs/pnas]iP]]$, and 
        \item $[[Δ, pnbs, pncs ⊢ iQ ≥ [pncs/pnas]iP]]$.
    \end{enumerate}

    $ 
    \begin{aligned}
        [[fv iQ]] &\subseteq [[fv [pnbs/pnas]iP ∩ fv [pncs/pnas]iP]]
                  &&\text{since by \cref{lemma:fv-propagation}, 
                         $[[fv iQ ⊆ fv [pnbs/pnas]iP]]$ and
                         $[[fv iQ ⊆ fv [pncs/pnas]iP]]$}\\
                  &\subseteq [[ ((fv iP \ {pnas}) ∪ {pnbs}) ∩ ((fv iP \ {pnas}) ∪ {pncs})]]\\
                  &= [[ (fv iP \ {pnas}) ∩ (fv iP \ {pnas}) ]]
                  &&\text{since $[[pnbs]]$ and $[[pncs]]$ are fresh}\\
                  &= [[ fv iP \ {pnas} ]]\\
                  &\subseteq [[ Γ \ {pnas} ]]
                  &&\text{since $[[iP]]$ is well-formed in $[[Γ]]$}\\
                  &\subseteq [[ {Δ} ]]\\
    \end{aligned}
    $\\
    This way, by \cref{lemma:wf-ctxt-equiv}, $[[Δ ⊢ iQ]]$.
    
    Let us apply $[[pnas/pnbs]]$---the inverse of the substitution $[[ pnbs/pnas ]]$ to 
    both sides of $[[Δ, pnbs, pncs ⊢ iQ ≥ [pnbs/pnas]iP]]$ and 
    by \cref{lemma:subst-pres-subt} 
    (since $[[pnbs/pnas]]$ can be specified as 
    $[[Δ,pnbs,pncs ⊢ pnbs/pnas : Δ, pnas, pncs]]$ by \label{lemma:subst-domain-weakening})
    obtain $[[Δ, pnas, pncs ⊢ [pnas/pnbs]iQ ≥ iP]]$.
    Notice that $[[Δ ⊢ iQ]]$ implies that $[[fv iQ ∩ {pnbs} = ∅]]$, 
    then by \cref{corollary:subst-disj}, $[[ [pnas/pnbs]iQ = iQ]]$,
     and thus $[[Δ, pnas, pncs ⊢ iQ ≥ iP]]$.
    By context strengthening, $[[Δ, pnas ⊢ iQ ≥ iP]]$.
\end{proof}

\begin{lemma}[Completeness and Initiality of Upgrade] \label{lemma:upgrade-completeness}
    The upgrade returns the least $[[Γ]]$-supertype of $[[iP]]$ well-formed in $[[Δ]]$.
    Assuming $[[iP]]$ is well-formed in $[[Γ = Δ, pnas]],$\\
    For any $[[iQ']]$ such that 
    \begin{enumerate}
        \item $[[Δ ⊢ iQ']]$ and
        \item $[[Γ ⊢ iQ' ≥ iP]]$,
    \end{enumerate}

    The result of the upgrade algorithm $[[iQ]]$ exists
    ($[[upgrade Γ ⊢ iP to Δ = iQ]]$) and satisfies $[[Δ ⊢ iQ' ≥ iQ]]$.
\end{lemma}
\begin{proof}

    Let us consider fresh (not intersecting with $[[Γ]]$) $[[pnbs]]$ and $[[pncs]]$.

    If we apply substitution $[[pnbs/pnas]]$ to both sides of $[[Δ, pnas ⊢ iQ' ≥ iP]]$,
    we have $[[Δ, pnbs ⊢ [pnbs/pnas]iQ' ≥ [pnbs/pnas]iP]]$, which by  
    \cref{corollary:subst-disj}, since $[[pnas]]$ is disjoint from $[[fv(iQ')]]$
    (because $[[Δ ⊢ iQ']]$), simplifies to $[[Δ, pnbs ⊢ iQ' ≥ [pnbs/pnas]iP]]$.

    Analogously, if we apply substitution $[[pncs/pnas]]$ to both sides of $[[Δ, pnas ⊢ iQ' ≥ iP]]$,
    we have $[[Δ, pncs ⊢ iQ' ≥ [pncs/pnas]iP]]$.

    This way, $[[iQ']]$ is a common supertype of $[[ [pnbs/pnas]iP ]]$ and $[[ [pncs/pnas]iP ]]$ in
    context $[[Δ, pnbs, pncs]]$. It means that we can apply the completeness of the least upper bound
    (\cref{lemma:lub-completeness}):
    \begin{enumerate}
        \item there exists $[[iQ]]$ s.t. $[[Γ ⊨ [pnbs/pnas]iP ∨ [pncs/pnas]iP = iQ]]$ 
        \item $[[Γ ⊢ iQ' ≥ iQ]]$.
    \end{enumerate}
    The former means that the upgrade algorithm terminates and returns $[[iQ]]$.
    The latter means that since both 
    $[[iQ']]$ and $[[iQ]]$ are well-formed in $[[Δ]]$ and $[[Γ]]$,
    by \cref{lemma:subt-ctxt-irrelevance}, $[[Δ ⊢ iQ' ≥ iQ]]$.
\end{proof}




\subsection{Constraint Satisfaction}
\begin{lemma}[Any constraint is satisfiable]
    \label{lemma:constraint-sat}
    Suppose that $[[Θ ⊢ SC]]$
    and $[[Ξ]]$ is a set such that
    $[[dom(SC)]] \subseteq [[Ξ]] \subseteq [[dom(Θ)]]$.
    Then there exists 
    $[[uσ]]$ such that 
    $[[ Θ  ⊢ uσ : Ξ ]]$ and
    $[[ Θ  ⊢ uσ : SC ]]$.
\end{lemma}
\begin{proof}
    Let us define $[[uσ]]$
    on $[[dom(SC)]]$ in the following way: 
    $$
    [[ [uσ]α̂± ]] = 
    \begin{cases}
        [[iP]]      & \text{if $[[(α̂± :≈ iP)]] \in [[SC]]$} \\
        [[iP]]      & \text{if $[[(α̂± :≥ iP)]] \in [[SC]]$} \\
        [[iN]]      & \text{if $[[(α̂± :≈ iN)]] \in [[SC]]$} \\
        [[∃β⁻.↓β⁻]] & \text{if $[[α̂± = α̂⁺]] \in [[Ξ \ dom(SC)]]$} \\
        [[∀β⁺.↑β⁺]] & \text{if $[[α̂± = α̂⁻]] \in [[Ξ \ dom(SC)]]$} \\
    \end{cases}
    $$
    Then $[[ Θ ⊢ uσ : SC ]]$ follows immediately
    from the reflexivity of equivalence and subtyping
    (\cref{lemma:subtyping-reflexivity}) and the corresponding
    rules 
    \ruleref{\ottdruleSATSCEPEqLabel}, 
    \ruleref{\ottdruleSATSCENEqLabel},
    and \ruleref{\ottdruleSATSCESupLabel}.
\end{proof}


\begin{lemma}[Constraint Entry Satiisfaction is Stable under Equivalence]
    \label{lemma:entry-sat-equiv}
    \begin{itemize}
        \item [$-$] If $[[Γ ⊢ iN1 : scE]]$ and $[[Γ ⊢ iN1 ≈ iN2]]$ then $[[Γ ⊢ iN2 : scE]]$.
        \item [$+$] If $[[Γ ⊢ iP1 : scE]]$ and $[[Γ ⊢ iP1 ≈ iP2]]$ then $[[Γ ⊢ iP2 : scE]]$.
    \end{itemize}
\end{lemma}
\begin{proof}
    \begin{itemize}
        \item [$-$] Then $[[scE]]$ has form $[[(α̂⁻ :≈ iM)]]$, 
            and by inversion, $[[Γ ⊢ iN1 ≈ iM]]$.
            Then by transitivity, $[[Γ ⊢ iN2 ≈ iM]]$, 
            meaning $[[Γ ⊢ iN2 : scE]]$.
        \item [$+$] Let us consider what form $[[scE]]$ has.
            \begin{caseof}
                \item $[[scE]] = [[(α̂⁺ :≈ iQ)]]$. Then $[[Γ ⊢ iP1 ≈ iQ]]$,
                    and hence, $[[Γ ⊢ iP2 ≈ iQ]]$ by transitivity.
                    Then $[[Γ ⊢ iP2 : scE]]$.
                \item $[[scE]] = [[(α̂⁺ :≥ iQ)]]$. Then $[[Γ ⊢ iP1 ≥ iQ]]$,
                    and hence, $[[Γ ⊢ iP2 ≥ iQ]]$ by transitivity.
                    Then $[[Γ ⊢ iP2 : scE]]$.
            \end{caseof}
    \end{itemize}
\end{proof}

\begin{corollary}[Constraint Satisfaction is stable under Equivalence]
    \label{corollary:sat-equiv}
    \hfill\\
    If $[[ Θ ⊢ uσ1 : SC ]]$ and $[[Θ ⊢ uσ1 ≈ uσ2 : dom(SC) ]]$ then $[[ Θ ⊢ uσ2 : SC ]]$;\\
    if $[[ Θ ⊢ uσ1 : UC ]]$ and $[[Θ ⊢ uσ1 ≈ uσ2 : dom(SC) ]]$ then $[[ Θ ⊢ uσ2 : UC ]]$.
\end{corollary}

\begin{corollary}[Normalization preserves Constraint Satisfaction]
    \label{corollary:nf-sat}
    \hfill\\
    If $[[Θ ⊢ uσ : SC]]$ then $[[Θ  ⊢ nf(uσ) : SC]]$;\\
    if $[[Θ ⊢ uσ : UC]]$ then $[[Θ  ⊢ nf(uσ) : UC]]$.
\end{corollary}





\subsection{Positive Subtyping}
\begin{lemma}[Soundness of the Positive Subtyping] \label{lemma:pos-subt-soundness}
    If $[[Γ ⊢ Θ]]$, $[[Γ ⊢ iQ]]$, $[[Γ ; dom(Θ) ⊢  uP]]$, and 
    $[[Γ ; Θ ⊨ uP ≥ iQ ⫤ SC]]$,
    then $[[Θ ⊢ SC : uv uP]]$ and
    for any normalized $[[uσ]]$ such that $[[ Θ ⊢ uσ : SC ]]$,
    $[[ Γ ⊢ [uσ]uP ≥ iQ ]]$.
\end{lemma}
\begin{proof} 
    We prove it by induction on $[[Γ ; Θ ⊨ uP ≥ iQ ⫤ SC]]$. 
    Let us consider the last rule to infer this judgment.
    \begin{caseof}
    \item \ruleref{\ottdruleAPUVarLabel} then
        $[[Γ ; Θ ⊨ uP ≥ iQ ⫤ SC]]$ has shape $[[Γ;Θ ⊨ â⁺ ≥ iP' ⫤ (â⁺ :≥ iQ')]]$ where
        $[[â⁺[Δ] ∊ Θ]]$ and $[[upgrade G ⊢ iP' to Δ = iQ']]$.

        Notice that $[[â⁺[Δ] ∊ Θ]]$ and $[[Γ ⊢ Θ]]$ 
        implies $[[Γ = Δ, pnas]]$ for some $[[pnas]]$, hence, the
        soundness of upgrade (\cref{lemma:upgrade-soundness}) is applicable:
        \begin{enumerate}
            \item $[[Δ ⊢ iQ']]$ and
            \item $[[Γ ⊢ iQ' ≥ iP]]$.
        \end{enumerate}

        Since $[[â⁺[Δ] ∊ Θ]]$ and $[[Δ ⊢ iQ']]$, 
        it is clear that $[[ Θ ⊢ (â⁺ :≥ iQ') : â⁺]]$.

        It is left to show that $[[Γ ⊢ [uσ]â⁺ ≥ iP']]$ for any normalized $[[uσ]]$ 
        s.t. $[[ Θ   ⊢ uσ : (â⁺ :≥ iQ') ]]$.
        The latter means that $[[ Θ(â⁺) ⊢ [uσ]â⁺ ≥ iQ' ]]$, i.e. $[[Δ ⊢ [uσ]â⁺ ≥ iQ']]$. 
        By weakening the context to $[[Γ]]$ and combining this judgment
        transitively with $[[Γ ⊢ iQ' ≥ iP]]$, we have $[[Γ ⊢ [uσ]â⁺ ≥ iP]]$,
        as required. 

    \item \label{case:pos-subt-soundness:var} \ruleref{\ottdruleAPVarLabel}  
        then $[[Γ ; Θ ⊨ uP ≥ iQ ⫤ SC]]$ has shape $[[Γ;Θ ⊨ a⁺ ≥ a⁺ ⫤ ·]]$ .
        Then $[[uv a⁺ = ∅]]$, and $[[SC]] = [[·]]$ satisfies $[[Θ ⊢ SC : ·]]$.
        Since $[[uv a⁺ = ∅]]$, application of any substitution $[[uσ]]$ 
        does not change $[[a⁺]]$, i.e. $[[ [uσ] a⁺ = a⁺]]$.
        Therefore, $[[Γ ⊢ [uσ]a⁺ ≥ a⁺]]$ holds by \ruleref{\ottdruleDOneNVarLabel}.

    \item \label{case:pos-subt-soundness:shift} 
        \ruleref{\ottdruleAShiftDLabel} then
        $[[Γ ; Θ ⊨ uP ≥ iQ ⫤ SC]]$ has shape $[[Γ;Θ ⊨ ↓uN ≥ ↓iM ⫤ SC]]$.\\
        Then the next step of the algorithm is the unification of $[[nf(uN)]]$ and $[[nf(iM)]]$,
        and it returns the resulting unification constraint $[[lift UC]] = [[SC]]$ as the result.
        By the soundness of unification (\cref{lemma:unification-soundness}),
        $[[Θ ⊢ SC : uv(uN)]]$ and for any normalized $[[uσ]]$, $[[ Θ ⊢ uσ : SC ]]$
        implies $[[ [uσ]nf(uN) = nf(iM) ]]$, 
        then we rewrite the left-hand side by \cref{lemma:norm-subst-distr}:
        $[[ nf([uσ]uN) = nf(iM) ]]$ and apply \cref{lemma:subt-equiv-algorithmization}:
        $[[Γ ⊢ [uσ]uN ≈ iM]]$, then by \ruleref{\ottdruleDOneShiftULabel},
        $[[Γ ⊢ ↓[uσ]uN ≥ ↓iM]]$.
    
    \item \label{case:pos-subt-soundness:exists}
       \ruleref{\ottdruleAExistsLabel} then
        $[[Γ ; Θ ⊨ uP ≥ iQ ⫤ SC]]$ has shape $[[Γ;Θ ⊨ ∃nas.uP' ≥ ∃nbs.iQ' ⫤ SC]]$ s.t. either 
        $[[nas]]$ or $[[nbs]]$ is not empty.\\
        Then the algorithm creates fresh unification variables $[[â⁻*[Γ,nbs] ]]$, 
        substitutes the old $[[nas]]$ with them in $[[uP']]$, and makes the recursive call:
        $[[G, nbs; Θ, â⁻*[G, nbs] ⊨ [â⁻*/nas] uP' ≥ iQ' ⫤ SC']]$, returning as the result
        $[[SC]] = [[SC' \ {α̂⁻*}]]$.

        Let us take an arbitrary normalized $[[uσ]]$ s.t. $[[ Θ   ⊢ uσ : SC' \ {α̂⁻*} ]]$.
        We wish to show $[[Γ ⊢ [uσ]uP ≥ iQ]]$, i.e. $[[Γ ⊢ ∃nas.[uσ]uP' ≥ ∃nbs.iQ']]$.
        To do that, we apply \ruleref{\ottdruleDOneExistsLabel}, and what is left to show is
        $[[Γ, nbs ⊢ [iNs/nas][uσ]uP' ≥ iQ']]$ for some $[[iNs]]$.
        If we construct a normalized $[[uσ']]$ such that $[[Θ, â⁻*[G, nbs] ⊢ uσ' : SC']]$
        and for some $[[iNs]]$, $[[ [iNs/nas][uσ]uP' ]] = [[ [uσ'][â⁻*/nas]uP' ]]$,
        we can apply the induction hypothesis to 
        $[[Γ, nbs; Θ, â⁻*[G, nbs] ⊨ [â⁻*/nas] uP ≥ iQ ⫤ SC']]$ and infer 
        the required subtyping.

        Let us construct such $[[uσ']]$ by extending $[[uσ]]$ with $[[â⁻*]]$
        mapped to the corresponding types in $[[SC']]$:
        $$
        [[ [uσ']β̂± ]]  = 
            \begin{cases}
               [[ [uσ]β̂± ]] & \text{if } [[β̂±]] \in [[dom(SC') \ {α̂⁻*}]]  \\
               [[ nf(iN) ]] & \text{if } [[β̂±]] \in [[α̂⁻*]] \text{ and } [[(β̂± :≈ iN)]] \in SC' \\
            \end{cases}
        $$

        It is easy to see that $[[uσ']]$ is normalized: it inherits this property from 
        $[[uσ]]$.
        Let us show that $[[Θ, â⁻*[G, nbs] ⊢ uσ' : SC']] $.
        Let us take an arbitrary entry $[[scE]]$ from $[[SC']]$ restricting a variable $[[β̂±]]$.
        Suppose $[[β̂±]] \in [[dom(SC') \ {α̂⁻*}]]$. Then
        $[[ (Θ, â⁻*[G, nbs])(β̂±) ⊢ [uσ']β̂± : scE ]]$ is
        rewritten as $[[ Θ(β̂±) ⊢ [uσ]β̂± : scE ]]$, which holds since $[[ Θ   ⊢ uσ : SC' ]]$.
        Suppose $[[β̂±]] = [[αî⁻]] \in [[α̂⁻*]]$. Then
        $[[scE]] = [[(αî⁻ :≈ iN)]]$ for some $[[iN]]$, 
        $[[ [uσ']αî⁻ ]] = [[ nf(iN) ]]$ by the definition,
        and $[[ Γ, nbs ⊢ nf(iN) : (αî⁻ :≈ iN) ]]$ by \ruleref{\ottdruleSATSCENEqLabel},
        since $[[Γ ⊢ nf(iN) ≈ iN]]$ by \cref{lemma:subt-equiv-algorithmization}.

        Finally, let us show that $[[ [iNs/nas][uσ]uP' ]] = [[ [uσ'][â⁻*/nas]uP' ]]$.
        For $[[iNi]]$, we take the \emph{normalized} type restricting $[[αî⁻]]$ in $[[SC']]$.
        Let us take an arbitrary variable from $[[uP]]$.
        \begin{enumerate}
            \item If this variable is a unification variable $[[β̂±]]$, then
                $[[ [iNs/nas][uσ] β̂± ]] = [[ [uσ]β̂± ]] $, since $[[ Θ   ⊢ uσ : SC' \ {α̂⁻*} ]]$ and 
                $[[ dom(Θ) ∩ {nas} = ∅ ]]$. 

                Notice that $[[β̂±]] \in [[dom(Θ)]]$, which is disjoint from $[[{α̂⁻*}]]$, 
                that is $[[β̂±]] \in [[dom(SC') \ {α̂⁻*}]]$. This way,
                $[[ [uσ'][â⁻*/nas]β̂± ]] = [[  [uσ']β̂± ]] = [[ [uσ]β̂± ]]$ by the definition 
                of $[[uσ']]$,
            \item If this variable is a regular variable $[[β±]] \notin [[nas]]$, then 
                $[[ [iNs/nas][uσ] β± ]] = [[ β± ]] $ and $[[ [uσ'][â⁻*/nas]β± ]] = [[ β± ]]$. 
            \item If this variable is a regular variable $[[αi⁻]] \in [[nas]]$, then 
                $[[ [iNs/nas][uσ] αi⁻ ]] = [[ iNi ]] = [[ nf(iNi) ]]$
                (the latter equality holds since $[[iNi]]$ is normalized)
                and $[[ [uσ'][â⁻*/nas]αi⁻ ]] = [[  [uσ']αî⁻ ]] = [[ nf(iNi) ]]$.
        \end{enumerate}
    \end{caseof}
\end{proof}

\begin{lemma}[Completeness of the Positive Subtyping] \label{lemma:pos-subt-completeness}
    Suppose that $[[Γ ⊢ Θ]]$, $[[Γ ⊢ iQ]]$ and $[[Γ ; dom(Θ) ⊢  uP]]$.
    Then for any $[[Θ ⊢ uσ : uv(uP)]]$ such that $[[ Γ ⊢ [uσ]uP ≥ iQ ]]$,
    there exists $[[Γ; Θ ⊨ uP ≥ iQ ⫤ SC]]$ and moreover, $[[ Θ ⊢ uσ : SC ]]$.
\end{lemma}
\begin{proof}
    Let us prove this lemma by induction on $[[ Γ ⊢ [uσ]uP ≥ iQ ]]$.
    Let us consider the last rule used in the derivation,
    but first, consider the base case for the substitution $[[ [uσ]uP ]]$:
    \begin{caseof}
        \item \label{case:pos-subt-complete-base} $[[uP]] = [[ ∃nbs.α̂⁺ ]]$ 
            (for potentially empty $[[nbs]]$)\\
            Then by assumption, $[[ Γ ⊢ ∃nbs.[uσ]α̂⁺ ≥ iQ ]]$ (where $[[ {nbs} ∩ fv [uσ]α̂⁺ = ∅]]$).
            Let us decompose $[[iQ]]$ as $[[iQ]] = [[∃ncs.iQ0]]$, where $[[iQ0]]$ does
            not start with $[[∃]]$. 

            By inversion, $[[Γ ; dom(Θ) ⊢  ∃nbs.α̂⁺]]$ implies $[[â⁺[Δ] ∊ Θ]]$ for some 
            $[[{Δ} ⊆ {Γ}]]$.

            By \cref{lemma:quant-rule-decomposition} applied twice, 
            $[[ Γ ⊢ ∃nbs.[uσ]α̂⁺ ≥ ∃ncs.iQ0 ]]$ implies
            $[[ Γ,ncs  ⊢ [iNs/nbs][uσ]α̂⁺ ≥ iQ0 ]]$ for some $[[iN]]$, 
            and since $[[ {nbs} ∩ fv([uσ]α̂⁺) ⊆ {nbs} ∩ {Θ(α̂⁺)} ⊆ {nbs} ∩ {Γ} = ∅ ]]$,
            $[[ [iNs/nbs][uσ]α̂⁺ = [uσ]α̂⁺ ]]$, that is $[[ Γ,ncs ⊢ [uσ]α̂⁺ ≥ iQ0]]$.

            When algorithm tires to infer the subtyping 
            $[[Γ; Θ ⊨ ∃nbs.α̂⁺ ≥ ∃ncs.iQ0 ⫤ SC]]$,
            it applies \ruleref{\ottdruleAExistsLabel},
            which reduces the problem to
            $[[Γ, ncs; Θ, β̂⁻*[Γ, ncs] ⊨ [β̂⁻*/nbs] α̂⁺ ≥ iQ0 ⫤ SC]]$, 
            which is equivalent to 
            $[[Γ, ncs; Θ, β̂⁻*[Γ, ncs] ⊨ α̂⁺ ≥ iQ0 ⫤ SC]]$.

            Next, the algorithm tries to apply
            \ruleref{\ottdruleAPUVarLabel}
            and the resulting restriction is $[[SC]] = [[(α̂⁺ :≥ iQ0')]]$ where
            $[[upgrade Γ, ncs ⊢ iQ0 to Δ = iQ0']]$.

            Why does the upgrade procedure terminate?
            Because $[[ [uσ]α̂⁺ ]]$ satisfies the pre-conditions of the completeness of the upgrade
            (\cref{lemma:upgrade-completeness}):
            \begin{enumerate}
                \item $[[Δ ⊢ [uσ]α̂⁺ ]]$ because $[[Θ ⊢ uσ : α̂⁺]]$ and $[[α̂⁺[Δ] ∊ Θ]]$,
                \item $[[ Γ,ncs ⊢ [uσ]α̂⁺ ≥ iQ0]]$ as noted above
            \end{enumerate}

            Moreover, the completeness of upgrade also says that $[[iQ0']]$ is 
            \emph{the least} supertype of $[[iQ0]]$ among types well-formed in $[[Δ]]$, 
            that is $[[Δ ⊢ [uσ]α̂⁺ ≥ iQ0']]$, which means 
            $[[ Θ   ⊢ uσ : (α̂⁺ :≥ iQ0') ]]$, that is $[[ Θ   ⊢ uσ : SC ]]$.

        \item \label{case:pos-subt-complete-pvar}
            $[[ Γ ⊢ [uσ]uP ≥ iQ ]]$ is derived by \ruleref{\ottdruleDOnePVarLabel}\\
            Then $[[iP]] = [[ [uσ]uP ]] = [[ α⁺ ]] = [[iQ]]$, where
            the first equality holds because $[[uP]]$ is not a unification variable:
            it has been covered by \cref{case:pos-subt-complete-base}; and
            the second equality hold because \ruleref{\ottdruleDOnePVarLabel} was applied.

            The algorithm applies \ruleref{\ottdruleAPVarLabel} and 
            infers $[[SC]] = [[·]]$, i.e. $[[Γ;Θ ⊨ a⁺ ≥ a⁺ ⫤ ·]]$.
            Then $[[ Θ   ⊢ uσ : · ]]$ holds trivially.


        \item \label{case:pos-subt-complete-upshift} 
            $[[ Γ ⊢ [uσ]uP ≥ iQ ]]$ 
            is derived by \ruleref{\ottdruleDOneShiftDLabel},
            
            Then $[[ uP ]] = [[ ↓uN ]]$, since the substitution $[[ [uσ]uP ]]$ must preserve the 
            top-level constructor of $[[uP]]\neq [[α̂⁺]]$ (the case $[[uP]] = [[α̂⁺]]$ has been covered
            by \cref{case:pos-subt-complete-base}), and $[[uQ]] = [[ ↓iM ]]$,
            and by inversion, $[[ Γ ⊢ [uσ]uN ≈ iM ]]$.

            Since both types start with $[[↓]]$, 
            the algorithm tries to apply \ruleref{\ottdruleAShiftDLabel}: 
            $[[G;Θ ⊨ ↓uN ≥ ↓iM ⫤ SC]]$. The premise of this rule is the
            unification of $[[nf(uN)]]$ and $[[nf(iM)]]$:
            $[[Γ;Θ ⊨ nf(uN) ≈u nf(iM) ⫤ UC]]$. And the algorithm 
            returns it as a subtyping constraint $[[SC]] = [[lift UC]]$.


            To demonstrate that the unification terminates
            ant $[[uσ]]$ satisfies the resulting constraints, 
            we apply the completeness 
            of the unification algorithm (\cref{lemma:unification-completeness}). 
            In order to do that, we need to provide a substitution unifying  
            $[[nf(uN)]]$ and $[[nf(iM)]]$. 
            Let us show that $[[nf(uσ)]]$ is such a substitution. 

            \begin{itemize}
                \item $[[nf(uN)]]$ and $[[nf(iM)]]$ are normalized 
                \item $[[Γ ; dom(Θ) ⊢  nf(uN)]]$ because $[[Γ ; dom(Θ) ⊢  uN]]$ (\cref{corollary:wf-nf-algo})
                \item $[[Γ ⊢ nf(iM)]]$ because $[[Γ ⊢ iM]]$ (\cref{corollary:wf-nf})
                \item $[[ Θ ⊢ nf(uσ) : uv(uP) ]]$ because $[[Θ ⊢ uσ : uv(uP) ]]$ (\cref{corollary:norm-subst-sig-algo})
                \item $ \begin{aligned}[t]
                        [[ Γ ⊢ [uσ]uN ≈ iM ]] &\Rightarrow [[ [uσ]uN ≈ iM ]]
                                            && \text {by \cref{lemma:equiv-completeness}}\\
                                            &\Rightarrow [[ nf([uσ]uN) = nf(iM) ]]
                                            && \text {by \cref{lemma:normalization-completeness}}\\
                                            &\Rightarrow [[ [nf(uσ)]nf(uN) = nf(iM) ]]
                                            && \text {by \cref{lemma:norm-subst-distr}}\\
                        \end{aligned}
                    $
            \end{itemize}
            By the completeness of the unification,
            $[[Γ ; Θ ⊨ uN ≈u iM ⫤ UC]]$ exists, and
            $[[Θ ⊢ nf(uσ) : lift UC]]$,
            and by \cref{corollary:sat-equiv}, $[[ Θ ⊢ uσ : lift UC ]]$.

        \item \label{case:pos-subt-complete-exists}
            $[[ Γ ⊢ [uσ]uP ≥ iQ ]]$ is derived by \ruleref{\ottdruleDOneExistsLabel}.\\
            We should only consider the case
            when the substitution $[[ [uσ]uP ]]$ results in the existential type 
            $[[∃nas.iP'']]$ (for $[[iP'']] \neq [[∃]]\dots$) by congruence, 
            i.e. $[[uP = ∃nas.uP']]$ (for $[[uP']] \neq [[∃]]\dots$) and $[[ [us]uP' = iP'' ]]$.
            This is because the case when $[[uP = ∃nbs.α̂⁺]]$ has been covered
            (\cref{case:pos-subt-complete-base}), and thus, the substitution $[[uσ]]$ must
            preserve all the outer quantifiers of $[[uP]]$ and does not generate any new ones.

            This way, $[[uP]] = [[∃nas.uP']]$, $[[ [uσ]uP ]] = [[ ∃nas.[uσ]uP' ]]$ 
            (assuming $[[nas]]$ does not intersect with the range of $[[uσ]]$)
            and $[[iQ]] = [[ ∃nbs.iQ' ]]$, where either $[[nas]]$ or $[[nbs]]$ is not empty.

            By inversion, $[[ Γ ⊢ [σ][uσ]uP' ≥ iQ' ]]$ for some $[[Γ, nbs ⊢ σ : nas]]$.
            Since $[[σ]]$ and $[[uσ]]$ have disjoint domains,
            and the range of one does not intersect with the domain of the other,
            they commute, i.e. $[[ Γ, nbs ⊢ [uσ][σ]uP' ≥ iQ' ]]$
            (notice that the tree inferring this judgement is 
            a proper subtree of the tree inferring 
            $[[ Γ ⊢ [uσ]uP ≥ iQ ]]$).

            At the next step, 
            the algorithm creates fresh (disjoint with $[[uv uP']]$) 
            unification variables $[[â⁻*]]$, replaces $[[nas]]$ with them in $[[ uP' ]]$,
            and makes the recursive call:
            $[[Γ, nbs; Θ, â⁻*[G, nbs] ⊨ uP0 ≥ iQ' ⫤ SC1]]$,
            (where $[[uP0]] = [[ [â⁻*/nas]uP' ]]$),
            returning $[[SC1 \ {â⁻*}]]$ as the result.

            To show that the recursive call terminates and that 
            $[[ Θ ⊢ uσ : SC1 \ {â⁻*} ]]$,
            it suffices to build $[[Θ, â⁻*[G, nbs] ⊢ us0 : uv(uP0)]]$---an extension of $[[uσ]]$ with
            $[[{â⁻*} ∩ uv(uP0)]]$ such that $[[Γ, nbs ⊢ [us0]uP0 ≥ iQ]]$.
            Then by the induction hypothesis, $[[Θ, â⁻*[G, nbs] ⊢ us0 : SC1]]$,
            and hence, $[[ Θ  ⊢ uσ : SC1 \ {â⁻*} ]]$, as required.

            Let us construct such a substitution $[[us0]]$:
                \[
                    [[ [uσ0]β̂± ]]  = 
                    \begin{cases}
                    [[ [σ]αi⁻ ]] & \text{if } [[β̂±]] = [[αî⁻]] \in [[{â⁻*} ∩ uv(uP0)]] \\
                    [[ [uσ]β̂± ]] & \text{if } [[β̂±]] \in [[ uv(uP') ]]
                    \end{cases}
            \]
            It is easy to see $[[Θ, â⁻*[Γ, nbs] ⊢ uσ0 : uv(uP0)]]$:
            $[[uv(uP0)]] = [[ uv([â⁻*/nas]uP') ]] = [[{â⁻*} ∩ uv(uP0) ∪ uv(uP')]]$. Then
            \begin{enumerate}
                    \item for $[[αî⁻]] \in [[{â⁻*} ∩ uv(uP0)]]$, $[[ (Θ, â⁻*[Γ, nbs])(αî⁻) ⊢ [uσ0] αî⁻]]$, 
                    i.e. $[[ Γ, nbs ⊢ [σ]αi⁻ ]]$ holds since $[[Γ, nbs ⊢ σ : nas]]$,
                    \item for $[[β̂±]] \in [[ uv(uP') ]] \subseteq [[dom(Θ)]]$, $[[ (Θ, â⁻*[Γ, nbs])(β̂±) ⊢ [uσ0] β̂± ]]$,
                    i.e. $[[Θ(β̂±) ⊢ [uσ] β̂± ]]$ holds since $[[Θ ⊢ uσ : uv(uP)]]$ and $[[β̂±]] \in [[ uv(uP') ]] = [[uv(uP)]]$.
            \end{enumerate}

            Now, let us show that $[[Γ, nbs ⊢ [us0]uP0 ≥ iQ]]$.
            To do that, we notice that $[[ [uσ0]uP0 ]] = [[ [uσ][σ][nas/â⁻*]uP0 ]]$:
            let us consider an arbitrary variable appearing freely in $[[uP0]]$:
            \begin{enumerate}
                \item if this variable is a metavariable $[[αî⁻]] \in [[â⁻*]]$, then
                $[[ [uσ0]αî⁻ ]] = [[ [σ]αi⁻ ]]$ and 
                $[[ [uσ][σ][nas/â⁻*]αî⁻ ]] = [[ [uσ][σ]αi⁻ ]] = [[ [σ]αi⁻ ]]$,
                \item if this variable is a metavariable $[[β̂±]] \in [[ uv(uP0) \ {â⁻*} ]] = [[uv(uP')]]$, then
                $[[ [uσ0]β̂± ]] = [[ [uσ]β̂± ]]$ and $[[ [uσ][σ][nas/â⁻*]β̂± ]] = [[ [uσ][σ]β̂± ]] = [[ [uσ]β̂± ]]$,
                \item if this variable is a regular variable from $[[fv(uP0)]]$, both substitutions do not change it:
                $[[ uσ0 ]]$, $[[ uσ ]]$ and $[[ nas / â⁻* ]]$ act on metavariables, 
                and $[[σ]]$ is defined on $[[nas]]$, however, $[[{nas} ∩ fv(uP0) = ∅]]$.
            \end{enumerate}
            This way, $[[ [uσ0]uP0 ]] = [[ [uσ][σ][nas/â⁻*]uP0 ]] = [[ [uσ][σ]uP' ]]$,
            and thus, $[[ Γ, nbs ⊢ [uσ0]uP0 ≥ iQ' ]]$.
    \end{caseof}
\end{proof}


\subsection{Subtyping Constraint Merge}
\obsEntryMergeDeterministic*
\begin{proof}
    First, notice that the shape of $[[scE1]]$ and $[[scE2]]$
    uniquely determines the rule applied to infer  
    $[[Γ ⊢ scE1 & scE2 = scE]]$,
    which is consequently, the same rule used to 
    infer $[[Γ ⊢ scE1 & scE2 = scE']]$.
    Second, notice that the premises of each rule are deterministic
    on the input:
    the positive subtyping is deterministic by \cref{obs:pos-subt-deterministic},
    and the least upper bound is deterministic by \cref{obs:lub-deterministic}.
\end{proof}

\obsSubtMergeDeterministic*
\begin{proof}
    The proof is analogous to the proof of \cref{obs:unif-merge-deterministic} 
    but uses \cref{obs:entry-merge-deterministic} to show 
    that the merge of the matching constraint entries is fixed.
\end{proof}


\lemEntryMergeSoundness*
\begin{proof}
    Let us consider the rule forming $[[Γ ⊢ scE1 & scE2 = scE]]$.
    \begin{caseof}
        \item \ruleref{\ottdruleSCMEPEqEqLabel}, i.e. 
            $[[Γ ⊢ scE1 & scE2 = scE]]$
            has form $[[Γ ⊢ (pua :≈ iQ) & (pua :≈ iQ') = (pua :≈ iQ)]]$
            and $[[nf(iQ) = nf(iQ')]]$. The latter implies $[[Γ ⊢ iQ ≈ iQ']]$ by
            \cref{lemma:subt-equiv-algorithmization}.
            Then
            \begin{enumerate}
                \item $[[Γ ⊢ scE]]$, i.e. $[[Γ ⊢ pua :≈ iQ]]$ holds by assumption;
                \item by inversion, $[[Γ ⊢ iP : (pua :≈ iQ)]]$ means $[[Γ ⊢ iP ≈ iQ]]$,
                and by transitivity of equivalence (\cref{corollary:equivalence-transitivity}), 
                $[[Γ ⊢ iP ≈ iQ']]$. Thus, $[[Γ ⊢ iP : scE1]]$ and $[[Γ ⊢ iP : scE2]]$ hold
                by \ruleref{\ottdruleSATSCEPEqLabel}.
            \end{enumerate}
        \item \ruleref{\ottdruleSCMENEqEqLabel} the negative case is proved in exactly the same way as the positive one.
        \item \ruleref{\ottdruleSCMESupSupLabel} 
            Then $[[scE1]]$ is $[[pua :≥ iQ1]]$, $[[scE2]]$ is $[[pua :≥ iQ2]]$,
            and $[[scE1 & scE2]] = [[scE]]$ is $[[pua :≥ iQ]]$ where $[[iQ]]$ is the least upper bound of $[[iQ1]]$ and $[[iQ2]]$.
            Then by \cref{lemma:lub-soundness},
            \begin{itemize}
                \item $[[Γ ⊢ iQ]]$,
                \item $[[Γ ⊢ iQ ≥ iQ1]]$,
                \item $[[Γ ⊢ iQ ≥ iQ2]]$.
            \end{itemize}

            Let us show the required properties.
            \begin{itemize}
                \item $[[Γ ⊢ scE]]$ holds from $[[Γ ⊢ iQ]]$,
                \item Assuming $[[Γ ⊢ iP : scE]]$, by inversion, we have $[[Γ ⊢ iP ≥ iQ]]$.
                    Combining it transitively with $[[Γ ⊢ iQ ≥ iQ1]]$, we have $[[Γ ⊢ iP ≥ iQ1]]$.
                    Analogously, $[[Γ ⊢ iP ≥ iQ2]]$.
                    Then $[[Γ ⊢ iP : scE1]]$ and $[[Γ ⊢ iP : scE2]]$ hold by \ruleref{\ottdruleSATSCESupLabel}.
            \end{itemize}

        \item \ruleref{\ottdruleSCMESupEqLabel}
            Then $[[scE1]]$ is $[[pua :≥ iQ1]]$, $[[scE2]]$ is $[[pua :≈ iQ2]]$, 
            where $[[Γ;· ⊨ uQ2 ≥ iQ1 ⫤ ·]]$, and the resulting   
            $[[scE1 & scE2]] = [[scE]]$ is equal to $[[scE2]]$, that is $[[pua :≈ iQ2]]$.
    
            Let us show the required properties.
            \begin{itemize}
                \item By assumption, $[[Γ ⊢ iQ]]$, and hence $[[Γ ⊢ scE]]$.
                \item Since $[[uv(uQ2) = ∅]]$, 
                    $[[Γ;· ⊨ uQ2 ≥ iQ1 ⫤ ·]]$ implies $[[Γ ⊢ iQ2 ≥ iQ1]]$
                    by the soundness of positive subtyping (\cref{lemma:pos-subt-soundness}).
                    Then let us take an arbitrary $[[Γ ⊢ iP]]$ such that $[[Γ ⊢ iP : scE]]$.
                    Since $[[scE2]] = [[scE]]$, $[[Γ ⊢ iP : scE2]]$ holds immediately.
                    
                    By inversion, $[[Γ ⊢ iP : (pua :≈ iQ2)]]$ means $[[Γ ⊢ iP ≈ iQ2]]$, 
                    and then by transitivity of subtyping (\cref{lemma:subtyping-transitivity}),
                    $[[Γ⊢ iP ≥ iQ1]]$.  Then $[[Γ ⊢ iP : scE1]]$ holds by \ruleref{\ottdruleSATSCESupLabel}.
            \end{itemize}
        \item \ruleref{\ottdruleSCMEEqSupLabel} Thee proof is analogous to the previous case.
    \end{caseof}
\end{proof}

\lemMergeSoundness*
\begin{proof}
    By definition, $[[Θ ⊢ SC1 & SC2 = SC]]$ consists of three parts:
    entries of $[[SC1]]$ that do not have matching entries of $[[SC2]]$,
    entries of $[[SC2]]$ that do not have matching entries of $[[SC1]]$,
    and the merge of matching entries.

    Notice that $[[α̂± ∊ Ξ1 \ Ξ2]]$
    if and only if there is an entry $[[scE]]$ in $[[SC1]]$ 
    restricting $[[α̂±]]$, but there is no such entry in $[[SC2]]$.
    Therefore, for any $[[α̂± ∊ Ξ1 \ Ξ2]]$,
    there is an entry $[[scE]]$ in $[[SC]]$ restricting $[[α̂±]]$.
    Notice that $[[Θ(α̂±) ⊢ scE]]$ holds since $[[Θ ⊢ SC1 : Ξ1]]$.

    Analogously, for any $[[β̂± ∊ Ξ2 \ Ξ1]]$,
    there is an entry $[[scE]]$ in $[[SC]]$ restricting $[[β̂±]]$.
    Notice that  $[[Θ(β̂±) ⊢ scE]]$ holds since $[[Θ ⊢ SC2 : Ξ2]]$.

    Finally, for any $[[γ̂± ∊ Ξ1 ∩ Ξ2]]$,
    there is an entry $[[scE1]]$ in $[[SC1]]$ restricting $[[γ̂±]]$
    and an entry $[[scE2]]$ in $[[SC2]]$ restricting $[[γ̂±]]$.
    Since $[[Θ ⊢ SC1 & SC2 = SC]]$ is defined,
    $[[Θ(γ̂±) ⊢ scE1 & scE2 = scE]]$ restricting $[[γ̂±]]$ is
    defined and belongs to $[[SC]]$,
    moreover, $[[Θ(γ̂±) ⊢ scE]]$ by \cref{lemma:entry-merge-soundness}.
    This way, $[[Θ ⊢ SC : Ξ1 ∪ Ξ2]]$.

    Let us show the second property.
    We take an arbitrary $[[uσ]]$ such that $[[Θ ⊢ uσ : Ξ1 ∪ Ξ2]]$ 
    and $[[ Θ ⊢ uσ : SC ]]$.
    To prove $[[ Θ ⊢ uσ : SC1 ]]$, 
    we need to show that for any $[[scE1]] \in [[SC1]]$, 
    restricting $[[α̂±]]$, $[[Θ(α̂±) ⊢ [uσ]α̂± : scE1]]$ holds.

    Let us assume that $[[α̂±]] \notin [[dom(SC2)]]$. It means that $[[SC]] \ni [[scE1]]$, 
    and then since $[[ Θ ⊢ uσ : SC ]]$, $[[Θ(α̂±) ⊢ [uσ]α̂± : scE1]]$. 

    Otherwise, $[[SC2]]$ contains an entry $[[scE2]]$ restricting $[[α̂±]]$,
    and $[[SC]] \ni [[scE]]$ where $[[Θ(α̂±) ⊢ scE1 & scE2 = scE]]$.
    Then since $[[ Θ ⊢ uσ : SC ]]$, $[[Θ(α̂±) ⊢ [uσ]α̂± : scE]]$,
    and by \cref{lemma:entry-merge-soundness}, $[[Θ(α̂±) ⊢ [uσ]α̂± : scE1]]$.

    The proof of $[[ Θ ⊢ uσ : SC2 ]]$ is symmetric.
\end{proof}


\lemEntryMergeCompleteness*
\begin{proof}
    Let us consider the shape of $[[scE1]]$ and $[[scE2]]$.
    \begin{caseof}
        \item $[[scE1]]$ is $[[pua :≈ iQ1]]$ and $[[scE2]]$ is $[[pua :≈ iQ2]]$.
            The proof repeats the corresponding case of \cref{lemma:unif-entry-merge-completeness}
        \item $[[scE1]]$ is $[[pua :≈ iQ1]]$ and $[[scE2]]$ is $[[pua :≥ iQ2]]$.
            Then $[[Γ ⊢ iP : scE1]]$ means $[[Γ ⊢ iP ≈ iQ1]]$, 
            and $[[Γ ⊢ iP : scE2]]$ means $[[Γ ⊢ iP ≥ iQ2]]$.
            Then by transitivity of subtyping, $[[Γ ⊢ iQ1 ≥ iQ2]]$,
            which means $[[Γ ; · ⊨ uQ1 ≥ iQ2 ⫤ ·]]$ by \cref{lemma:pos-subt-completeness}.
            This way, \ruleref{\ottdruleSCMEEqSupLabel} applies to infer
            $[[Γ ⊢ scE1 & scE2 = scE1]]$, and $[[Γ ⊢ iP : scE1]]$ holds by assumption.
        \item $[[scE1]]$ is $[[pua :≥ iQ1]]$ and $[[scE2]]$ is $[[pua :≥ iQ2]]$.
            Then $[[Γ ⊢ iP : scE1]]$ means $[[Γ ⊢ iP ≥ iQ1]]$, 
            and $[[Γ ⊢ iP : scE2]]$ means $[[Γ ⊢ iP ≥ iQ2]]$.
            By the completeness of the least upper bound (\cref{lemma:lub-completeness}), 
            $[[Γ ⊨ iQ1 ∨ iQ2 = iQ]]$, and $[[Γ ⊢ iP ≥ iQ]]$. 
            This way, \ruleref{\ottdruleSCMESupSupLabel} applies to infer
            $[[Γ ⊢ scE1 & scE2 = (pua :≥ iQ)]]$, 
            and $[[Γ ⊢ iP : (pua :≥ iQ)]]$ holds by \ruleref{\ottdruleSATSCESupLabel}.
        \item The negative cases are proved symmetrically.
    \end{caseof}
\end{proof}

\lemMergeCompleteness*
\begin{proof}
    By  definition, $[[SC1 & SC2]]$ is a union of
    \begin{enumerate}
        \item entries of $[[SC1]]$, which do not have matching entries in $[[SC2]]$,
        \item entries of $[[SC2]]$, which do not have matching entries in $[[SC1]]$, and 
        \item the merge of matching entries.
    \end{enumerate}

    This way, to show that $[[Θ ⊢ SC1 & SC2 = SC]]$ is defined, we need to demonstrate that 
    each of these components is defined and satisfies 
    the required property 
    (that the result of $[[uσ]]$ satisfies the corresponding constraint entry).

    It is clear that the first two components of this union exist. 
    Moreover, if $[[scE]]$ is an entry of $[[SCi]]$
    restricting $[[α̂± ∉ dom(SC2)]]$,
    then $[[ Θ ⊢ uσ : SCi ]]$ implies $[[ Θ(α̂±) ⊢ [uσ]α̂± : scE]]$,

    Let us show that the third component exists.  
    Let us take two entries $[[scE1]] \in [[SC1]]$ and $[[scE2]] \in [[SC2]]$ restricting the same variable $[[α̂±]]$.  $[[ Θ   ⊢ uσ : SC1 ]]$ means that $[[Θ(α̂±) ⊢ [uσ]α̂± : scE1]]$ and $[[ Θ   ⊢ uσ : SC2 ]]$ means $[[Θ(α̂±) ⊢ [uσ]α̂± : scE2]]$.
    Then by \cref{lemma:entry-merge-completeness}, $[[Θ(α̂±) ⊢ scE1 & scE2 = scE]]$ is defined and $[[Θ(α̂±) ⊢ [uσ]α̂± : scE]]$.

\end{proof}



\subsection{Negative Subtyping}
\begin{observation}[Negative Algorithmic Subtyping is Deterministic]
    \label{obs:neg-subt-deterministic}
    For fixed $[[Γ]]$, $[[Θ]]$, $[[iM]]$, and $[[uN]]$, 
    if $[[Γ ; Θ ⊨ uN ≤ iM ⫤ SC]]$ 
    and $[[Γ ; Θ ⊨ uN ≤ iM ⫤ SC']]$
    then $[[SC]]$ = $[[SC']]$.
\end{observation}
\begin{proof}
    First, notice that the shape of
    the input uniquely determines the rule applied to infer
    $[[Γ ; Θ ⊨ uN ≤ iM ⫤ SC]]$,
    which is consequently, the same rule used to
    infer $[[Γ ; Θ ⊨ uN ≤ iM ⫤ SC']]$.

    Second, notice that for each of the inference rules, 
    the premises are deterministic on the input.
    Specifically, 
    \begin{itemize}
        \item \ruleref{\ottdruleAShiftULabel} relies on unification,
            which is deterministic by \cref{obs:unif-deterministic};
        \item \ruleref{\ottdruleAForallLabel} relies on
            the choice of fresh algorithmic variables,
            which is deterministic as discussed in \cref{sec:fresh-selection},
            and on the negative subtyping, which is deterministic by
            the induction hypothesis;
        \item \ruleref{\ottdruleAArrowLabel} uses 
            the negative subtyping 
            (deterministic by the induction hypothesis),
            the positive subtyping 
            (\cref{obs:pos-subt-deterministic}),
            and the merge of subtyping constraints
            (\cref{obs:subt-merge-deterministic});
    \end{itemize}
\end{proof}


\begin{lemma}[Soundness of Negative Subtyping] \label{lemma:neg-subt-soundness}
        If $[[Γ ⊢ Θ]]$, $[[Γ ⊢ iM]]$, $[[Γ ; dom(Θ) ⊢ uN]]$ and 
        $[[Γ ; Θ ⊨ uN ≤ iM ⫤ SC]]$, then 
        $[[Θ ⊢ SC : uv(uN)]]$ and 
        for any normalized $[[uσ]]$ such that $[[ Θ  ⊢ uσ : SC ]]$,
        $[[ Γ ⊢ [uσ]uN ≤ iM ]]$.
\end{lemma}
\begin{proof}
    We prove it by induction on $[[Γ ; Θ ⊨ uN ≤ iM ⫤ SC]]$.

    Suppose that $[[uσ]]$ is normalized and $[[ Θ  ⊢ uσ : SC ]]$,
    Let us consider the last rule to infer this judgment. 
    \begin{caseof}
        \item \ruleref{\ottdruleAArrowLabel}. Then $[[Γ ; Θ ⊨ uN ≤ iM ⫤ SC]]$ has shape
        $[[G;Θ ⊨ uP → uN' ≤ iQ → iM' ⫤ SC]]$\\
        On the next step, the the algorithm makes two recursive calls:
        $[[G;Θ ⊨ uP ≥ iQ ⫤ SC1]]$ and $[[G;Θ ⊨ uN' ≤ iM' ⫤ SC2]]$
        and returns $[[Θ ⊢ SC1 & SC2 = SC]]$ as the result.

        By the soundness of constraint merge (\cref{lemma:merge-soundness}),
        $[[ Θ  ⊢ uσ : SC1 ]]$ and $[[ Θ  ⊢ uσ : SC2 ]]$.
        Then by the soundness of positive subtyping (\cref{lemma:pos-subt-soundness}), $[[ Γ ⊢ [uσ]uP ≥ iQ ]]$; 
        and by the induction hypothesis, $[[ Γ ⊢ [uσ]uN' ≤ iM' ]]$.
        This way, by \ruleref{\ottdruleDOneArrowLabel}, $[[Γ ⊢ [uσ](uP → uN') ≤ iQ → iM']]$.

        \item \ruleref{\ottdruleANVarLabel}, and then $[[Γ ; Θ ⊨ uN ≤ iM ⫤ SC]]$ has shape $[[G;Θ ⊨ a⁻ ≤ a⁻ ⫤ ·]]$\\
        This case is symmetric to \cref{case:pos-subt-soundness:var} of \cref{lemma:pos-subt-soundness}.

        \item \ruleref{\ottdruleAShiftULabel}, and then $[[Γ ; Θ ⊨ uN ≤ iM ⫤ SC]]$ has shape $[[G;Θ ⊨ ↑uP ≤ ↑iQ ⫤ SC]]$\\
        This case is symmetric to \cref{case:pos-subt-soundness:shift} of \cref{lemma:pos-subt-soundness}.

        \item \ruleref{\ottdruleAForallLabel}, and then $[[Γ ; Θ ⊨ uN ≤ iM ⫤ SC]]$ has shape
         $[[G;Θ ⊨ ∀pas.uN' ≤ ∀pbs.iM' ⫤ SC]]$ s.t. either $[[pas]]$ or $[[pbs]]$ is not empty\\
        This case is symmetric to \cref{case:pos-subt-soundness:exists} of \cref{lemma:pos-subt-soundness}.

\end{caseof}
\end{proof}

\begin{lemma}[Completeness of the Negative Subtyping]
    \label{lemma:neg-subt-completeness}
    Suppose that $[[Γ ⊢ Θ]]$, $[[Γ ⊢ iM]]$, $[[Γ ; dom(Θ) ⊢ uN]]$,
    and $[[uN]]$ does not contain negative unification variables ($[[α̂⁻]] \notin [[uv uN]]$).
    Then for any $[[Θ ⊢ uσ : uv(uN)]]$ such that $[[Γ ⊢ [uσ]uN ≤ iM]]$,
    there exists $[[Γ ; Θ ⊨ uN ≤ iM ⫤ SC]]$ and moreover, $[[ Θ ⊢ uσ : SC ]]$.
\end{lemma}
\begin{proof}
    We prove it by induction on $[[ Γ ⊢ [uσ]uN ≤ iM ]]$.
    Let us consider the last rule used in the derivation of $[[ Γ ⊢ [uσ]uN ≤ iM ]]$.
    \begin{caseof}
        \item $[[ Γ ⊢ [uσ]uN ≤ iM ]]$ is derived by \ruleref{\ottdruleDOneShiftULabel}\\
            Then $[[ uN ]] = [[ ↑uP ]]$, since
            the substitution $[[ [uσ]uN ]]$ must preserve the 
            top-level constructor of $[[uN]] \neq [[α̂⁻]]$ (since by assumption, $[[α̂⁻]] \notin [[uv uN]]$), 
            and $[[uQ]] = [[ ↓iM ]]$, and by inversion, $[[ Γ ⊢ [uσ]uN ≈ iM ]]$.
            The rest of the proof is symmetric to \cref{case:pos-subt-complete-upshift} of
            \cref{lemma:pos-subt-completeness}: notice that the algorithm does not make a recursive call, 
            and the difference in the induction statement for the positive and 
            the negative case here does not matter.

        \item $[[ Γ ⊢ [uσ]uN ≤ iM ]]$ is derived by \ruleref{\ottdruleDOneArrowLabel}, 
            i.e. $[[ [uσ]uN ]] = [[ [uσ]uP → [uσ]uN' ]]$ and $[[iM]] = [[iQ → iM']]$, 
            and by inversion, $[[ Γ ⊢ [uσ]uP ≥ iQ ]]$ and $[[ Γ ⊢ [uσ]uN' ≤ iM' ]]$.

            The algorithm makes two recursive calls: $[[Γ ; Θ ⊨ uP ≥ iQ ⫤ SC1]]$ and $[[Γ ; Θ ⊨ uN' ≤ iM' ⫤ SC2]]$,
            and then returns $[[Θ ⊢ SC1 & SC2 = SC]]$ as the result.
            Let us show that these recursive calls are successful and the returning constraints 
            are fulfilled by $[[uσ]]$.

            Notice that from the inversion of $[[Γ ⊢ iM]]$, we have: $[[Γ ⊢ iQ]]$ and $[[Γ ⊢ iM']]$;
            from the inversion of $[[Γ ; dom(Θ) ⊢ uN]]$, we have: $[[Γ ; dom( Θ) ⊢  uP]]$ and $[[Γ ; dom( Θ) ⊢  uN']]$;
            and since $[[uN]]$ does not contain negative unification variables,
            $[[uN']]$ does not contain negative unification variables either.

            This way, we can apply the induction hypothesis to $[[Γ ⊢ [uσ]uN' ≤ iM']]$ to 
            obtain $[[Γ ; Θ ⊨ uN' ≤ iM' ⫤ SC2]]$ such that $[[Θ ⊢ SC2 : uv(uN')]]$ and $[[ Θ ⊢ uσ : SC2 ]]$.
            Also, we can apply the completeness of the positive subtyping (\cref{lemma:pos-subt-completeness}) to 
            $[[ Γ ⊢ [uσ]uP ≥ iQ ]]$ to obtain $[[Γ ; Θ ⊨ uP ≥ iQ ⫤ SC1]]$ such that $[[Θ ⊢ SC1 : uv(uP)]]$
            and $[[ Θ ⊢ uσ : SC1 ]]$.

            Finally, we need to show that the merge of $[[SC1]]$ and $[[SC2]]$ is successful and
            satisfies the required properties.
            To do so, we apply the completeness of subtyping constraint merge (\cref{lemma:merge-completeness})
            (notice that $[[Θ ⊢ uσ : uv(uP → uN')]]$ means 
            $[[Θ ⊢ uσ : uv(uP) ∪ uv(uN')]]$).
            This way, $[[Θ ⊢ SC1 & SC2 = SC]]$ is defined and $[[ Θ ⊢ uσ : SC ]]$ holds. 

       \item \label{case:subt-complete-forall}
            $[[ Γ ⊢ [uσ]uN ≤ iM ]]$ is derived by \ruleref{\ottdruleDOneForallLabel}.
            Since $[[uN]]$ does not contain negative unification variables,
            $[[uN]]$ must be of the form $[[∀pas.uN']]$,
            such that $[[ [uσ]uN = ∀pas.[uσ]uN' ]]$ and $[[ [uσ]uN']] \neq [[∀]]\dots$
            (assuming $[[pas]]$ does not intersect with the range of $[[uσ]]$).
            Also, $[[iM]] = [[∀pbs.iM']]$ and either $[[pas]]$ or $[[pbs]]$ is non-empty.

            The rest of the proof is symmetric to \cref{case:pos-subt-complete-exists} of
            \cref{lemma:pos-subt-completeness}.
            To apply the induction hypothesis, we need to show additionally that
            there are no negative unification variables in $[[uN0]] = [[ [â⁺*/pas]uN' ]]$.
            This is because $[[ uv uN0 ⊆ uv uN ∪ {â⁺*} ]]$, and $[[uN]]$ is free of negative
            unification variables by assumption.

       \item $[[ Γ ⊢ [uσ]uN ≤ iM ]]$ is derived by \ruleref{\ottdruleDOneNVarLabel}.\\
            Then $[[iN]] = [[ [uσ]uN ]] = [[ α⁻ ]] = [[iM]]$. 
            Here the first equality holds because $[[uN]]$ is not a unification variable:
            by assumption, $[[uN]]$ is free of negative unification variables.
            The second and the third equations hold because \ruleref{\ottdruleDOneNVarLabel}
            was applied. 

            The rest of the proof is symmetric to \cref{case:pos-subt-complete-pvar} of
            \cref{lemma:pos-subt-completeness}.

    \end{caseof}
\end{proof}




\subsection{Singularity}
\begin{lemma}[Soundness of Entry Singularity]
    \label{lemma:entry-singularity-soundness}
    \begin{itemize}
        \item [$+$] Suppose $[[scE singular with iP]]$ for $[[iP]]$ well-formed in $[[Γ]]$.
            Then $[[ Γ ⊢ iP : scE ]]$
            and for any $[[ Γ ⊢ iP']]$ such that $[[Γ ⊢ iP' : scE]]$, $[[Γ ⊢ iP' ≈ iP]]$;
        \item [$-$] Suppose $[[scE singular with iN]]$ for $[[iN]]$ well-formed in $[[Γ]]$.
            Then $[[ Γ ⊢ iN : scE ]]$
            and for any $[[ Γ ⊢ iN']]$ such that $[[Γ ⊢ iN' : scE]]$, $[[Γ ⊢ iN' ≈ iN]]$.
    \end{itemize}
\end{lemma}
\begin{proof}
    Let us consider how $[[scE singular with iP]]$ or $[[scE singular with iN]]$ is formed.
    \begin{caseof}
        \item \ruleref{\ottdruleSINGNEqLabel}, that is $[[scE]] = [[α̂⁻ :≈ iN0]]$.
            and $[[iN]]$ is $[[nf(iN0)]]$.
            Then $[[Γ ⊢ iN' : scE]]$ means $[[Γ ⊢ iN' ≈ iN0]]$, 
            (by inversion of \ruleref{\ottdruleSATSCENEqLabel}),
            which by transitivity, using \cref{corollary:nf-sound-wrt-subt-equiv},
            means $[[Γ ⊢ iN' ≈ nf(iN0)]]$, 
            as required.
        \item \ruleref{\ottdruleSINGPEqLabel}. This case is symmetric to the previous one.

        \item \ruleref{\ottdruleSINGSupVarLabel}, that is 
            $[[scE]] = [[α̂⁺ :≥ ∃nas.β⁺]]$, and $[[iP = β⁺]]$.

            Since $[[Γ ⊢ β⁺ ≥  ∃nas.β⁺]]$, we have $[[Γ ⊢ β⁺ : scE ]]$, 
            as required.

            Notice that $[[Γ ⊢ iP' : scE]]$ means $[[Γ ⊢ iP' ≥ ∃nas.β⁺]]$.
            Let us show that it implies $[[Γ ⊢ iP' ≈ β⁺]]$.
            By applying \cref{lemma:shape-of-supertypes} once, 
            we have $[[Γ, nas ⊢ iP' ≥ β⁺]]$.
            By applying it again, we notice that
            $[[Γ, nas ⊢ iP' ≥ β⁺]]$ implies $[[iPi = ∃nas'.β⁺]]$.
            Finally, it is easy to see that $[[Γ ⊢ ∃nas'.β⁺ ≈ β⁺]]$

        \item \ruleref{\ottdruleSINGSupShiftLabel},
            that is $[[scE]] = [[α̂⁺ :≥ ∃nbs.↓iN1]]$, 
            where $[[iN1 ≈ nbj]]$, and $[[iP = ∃α⁻.↓α⁻]]$.

            Since $[[Γ ⊢ ∃α⁻.↓α⁻ ≥ ∃nbs.↓iN1]]$ 
            (by \ruleref{\ottdruleDOneExistsLabel}, with substitution $[[iN1 / α⁻]]$),
            we have $[[Γ ⊢ ∃α⁻.↓α⁻ : scE ]]$, as required.

            Notice $[[Γ ⊢ iP' : scE]]$ means $[[Γ ⊢ iP' ≥ ∃nbs.↓iN1]]$.
            Let us show that it implies $[[Γ ⊢ iP' ≈ ∃α⁻.↓α⁻]]$.

            $
            \begin{aligned}[h]
            [[Γ ⊢ iP' ≥ ∃nbs.↓iN1]] &\Rightarrow [[Γ ⊢ nf(iP') ≥ ∃nbs'.↓nf(iN1)]] \text{ where } [[ord {nbs} in iN' = nbs']] 
                                   && \text{by \cref{corollary:nf-pres-subt}} \\
                                   &\Rightarrow [[Γ ⊢ nf(iP') ≥ ∃nbs'.↓nf(nbj)]]  
                                   && \text{by \cref{lemma:normalization-completeness}}\\
                                   &\Rightarrow [[Γ ⊢ nf(iP') ≥ ∃nbs'.↓nbn]]  
                                   && \text{by definition of normalization}\\
                                   &\Rightarrow [[Γ ⊢ nf(iP') ≥ ∃nbj.↓nbj]]  
                                   && \text{since $[[ord {nbs} in nf(iN1)]] = [[nbj]]$}\\
                                   &\Rightarrow [[Γ, nbj ⊢ nf(iP') ≥ ↓nbj]] \text { and } [[nbj ∉ fv(nf(iP'))]]
                                   && \text{by \cref{lemma:shape-supertypes-norm}}\\
            \end{aligned}
            $

            By \cref{lemma:shape-supertypes-norm}, 
            the last subtyping means that $[[nf(iP') = ∃nas.↓iN']]$,
            such that
            \begin{enumerate}
                \item $[[Γ, nbj, nas ⊢ iN']]$
                \item $[[ord {nas} in iN' = nas]]$
                \item for some substitution $[[Γ, nbj ⊢ σ : nas]]$, 
                    $[[ [σ]iN' = nbj ]]$.
            \end{enumerate}
            Since $[[nbj ∉ fv(nf(iP'))]]$,
            the latter means that $[[iN' = na]]$, and then 
            $[[nf(iP') = ∃na.↓na]]$ for some $[[na]]$.
            Finally, notice that all the types of shape
            $[[∃na.↓na]]$ are equal.
   \end{caseof}

\end{proof}



\begin{lemma}[Completeness of Entry Singularity]
    \label{lemma:entry-singularity-completeness}
    \hfill
    \begin{itemize}
        \item [$-$] Suppose that there exists $[[iN]]$ well-formed in $[[Γ]]$ 
            such that for any $[[iN']]$ well-formed in $[[Γ]]$,
            $[[Γ ⊢ iN' : scE]]$ implies $[[Γ ⊢ iN' ≈ iN]]$. 
            Then $[[scE singular with nf(iN)]]$.
        \item [$+$] Suppose that there exists $[[iP]]$ well-formed in $[[Γ]]$ 
            such that for any $[[iP']]$ well-formed in $[[Γ]]$,
            $[[Γ ⊢ iP' : scE]]$ implies $[[Γ ⊢ iP' ≈ iP]]$. 
            Then $[[scE singular with nf(iP)]]$ .
    \end{itemize}
\end{lemma}
\begin{proof}
    \hfill
    \begin{itemize}
        \item [$-$] 
            By \cref{lemma:constraint-sat},
            there exists $[[Γ ⊢ iN' : scE]]$.
            Since $[[iN']]$ is negative, by inversion of
            $[[Γ ⊢ iN' : scE]]$, $[[scE]]$ has shape $[[α̂⁻ :≈ iM]]$, 
            where $[[Γ ⊢ iN' ≈ iM]]$, and transitively, $[[Γ ⊢ iN ≈ iM]]$.
            Then $[[nf(iM) = nf(iN)]]$, 
            and $[[scE singular with nf(iM)]]$ (by \ruleref{\ottdruleSINGNEqLabel})
            is rewritten as $[[scE singular with nf(iN)]]$.
        \item [$+$]
            By \cref{lemma:constraint-sat}, there exists $[[Γ ⊢ iP' : scE]]$, 
            then by assumption, $[[Γ ⊢ iP' ≈ iP]]$,
            which by \cref{lemma:entry-sat-equiv} implies $[[Γ ⊢ iP : scE]]$.

            Let us consider the shape of $[[scE]]$:
            \begin{caseof}
                \item $[[scE]] = [[(α̂⁺ :≈ iQ)]]$ then 
                    inversion of $[[Γ ⊢ iP : scE]]$
                    implies $[[Γ ⊢ iP ≈ iQ]]$, and hence, $[[nf(iP) = nf(iQ)]]$
                    (by \cref{lemma:subt-equiv-algorithmization}).
                    Then $[[scE singular with nf(iQ)]]$, 
                    which holds by \ruleref{\ottdruleSINGPEqLabel}, 
                    is rewritten as $[[scE singular with nf(iP)]]$.

                \item $[[scE]] = [[(α̂⁺ :≥ iQ)]]$.
                    Then the inversion of $[[Γ ⊢ iP : scE]]$ 
                    implies $[[Γ ⊢ iP ≥ iQ]]$.
                    Let us consider the shape of $[[iQ]]$:
                    \begin{caseof}
                        \item $[[iQ]] = [[∃nbs.β⁺]]$ (for potentially empty $[[nbs]]$).
                            Then $[[Γ ⊢ iP ≥ ∃nbs.β⁺]]$ 
                            implies $[[Γ ⊢ iP ≈ β⁺]]$ by 
                            \cref{lemma:shape-of-supertypes}, 
                            as was noted in the proof of 
                            \cref{lemma:entry-singularity-soundness},
                            and hence, $[[nf(iP) = β⁺]]$.

                            Then $[[scE singular with β⁺]]$, which holds by
                            \ruleref{\ottdruleSINGSupVarLabel},
                            can be rewritten as $[[scE singular with nf(iP)]]$.

                        \item $[[iQ]] = [[∃nbs.↓iN]]$ (for potentially empty $[[nbs]]$).
                            Notice that $[[Γ ⊢ ∃γ⁻.↓γ⁻ ≥ ∃nbs.↓iN]]$ 
                            (by \ruleref{\ottdruleDOneExistsLabel}, 
                            with substitution $[[iN / γ⁻]]$), and thus, 
                            $[[Γ ⊢ ∃γ⁻.↓γ⁻ : scE]]$ by \ruleref{\ottdruleSATSCESupLabel}.
                            
                            Then by assumption, $[[Γ ⊢ ∃γ⁻.↓γ⁻ ≈ iP]]$,
                            that is $[[nf(iP) = ∃γ⁻.↓γ⁻]]$.
                            To apply \ruleref{\ottdruleSINGSupShiftLabel}
                            to infer $[[(α̂⁺ :≥ ∃nbs.↓iN) singular with ∃γ⁻.↓γ⁻]]$,
                            it is left to show that $[[iN ≈ nbi]]$ for some $i$.

                            Since $[[Γ ⊢ iQ : scE]]$, by assumption,
                            $[[Γ ⊢ iQ ≈ iP]]$, and by transitivity, 
                            $[[Γ ⊢ iQ ≈ ∃γ⁻.↓γ⁻]]$.
                            It implies
                            $[[nf(∃nbs.↓iN) = ∃γ⁻.↓γ⁻]]$ (by \cref{lemma:subt-equiv-algorithmization}), 
                            which by definition of normalization means
                            $[[∃nbs'.↓nf(iN) = ∃γ⁻.↓γ⁻]]$, where $[[ord {nbs} in iN' = nbs']]$.
                            This way, $[[nbs']]$ is a variable $[[β⁻]]$, and $[[ nf(iN) = β⁻ ]]$.
                            Notice that $[[β⁻]] \in [[nbs']] \subseteq [[nbs]]$ by \cref{lemma:ord-soundness}.
                            This way, $[[iN ≈ β⁻]]$ for $[[β⁻]] \in [[nbs]]$ (by \cref{lemma:subt-equiv-algorithmization}),
                    \end{caseof}
            \end{caseof}
    \end{itemize}
\end{proof}

\begin{lemma}[Soundness of Singularity]
    \label{lemma:singularity-soundness}
    Suppose $[[Θ ⊢ SC]]$, and $[[SC singular with uσ]]$. 
    Then $[[Θ ⊢ uσ : SC]]$ and for any 
    $[[uσ']]$ such that $[[Θ ⊢ uσ' : SC]]$,
    $[[Θ ⊢ uσ' ≈ uσ : dom(SC)]]$.
\end{lemma}
\begin{proof}
    Suppose that $[[Θ ⊢ uσ' : SC]]$.
    It means that for every $[[scE]] \in [[SC]]$ restricting $[[α̂±]]$,
    $[[Θ(α̂±) ⊢ [uσ']α̂± : scE]]$ holds.
    $[[SC singular with uσ]]$ means $[[scE singular with [uσ]α̂± ]]$,
    and hence, by \cref{lemma:entry-singularity-completeness},
    $[[ Θ(α̂±) ⊢ [uσ']α̂± ≈ [uσ]α̂±  ]]$ holds.

    Since the uniqueness holds for every variable from $[[dom(SC)]]$,
    $[[uσ]]$ is equivalent to $[[uσ']]$ on this set.
\end{proof}

\begin{lemma}[Completeness of Singularity]
    \label{lemma:singularity-completeness}
    Suppose there exists $[[Θ ⊢ uσ1]]$ such that
    for any $[[Θ ⊢ uσ]]$, $[[Θ ⊢ uσ : SC]]$ implies $[[Θ ⊢ uσ ≈ uσ1 : varset]]$.
    Then 
    \begin{itemize}
        \item $[[SC|varset singular with uσ0]]$ for some $[[uσ0]]$, and
        \item $[[varset ⊆ dom(SC)]]$.
    \end{itemize} 
\end{lemma}
\begin{proof}
\end{proof}

\subsection{Declarative Typing}
\begin{definition}[Number of prenex quantifiers]
    Let us define $\npq{[[iN]]}$ and $\npq{[[iP]]}$ as the number of prenex quantifiers in these types, i.e.
    \begin{itemize}
        \item [$+$] $\npq{[[∃nas.iP]]} = |[[nas]]|$, if $[[iP ≠ ∃nbs.iP']]$,
        \item [$-$] $\npq{[[∀pas.iN]]} = |[[pas]]|$, if $[[iN ≠ ∀pbs.iN']]$.
    \end{itemize}
\end{definition}

\begin{definition}[Size of a Declarative Judgement]
    \label{def:decl-typing-size}
    For a declarative typing judgment $J$
    let us define a metrics $\size{J}$ as a pair of numbers 
    in the following way:
    \begin{itemize}
        \item [$+$] $\size{[[Γ ; Φ ⊢ v : iP]]} = (\size{[[v]]}, 0)$;
        \item [$-$] $\size{[[Γ ; Φ ⊢ c : iN]]} = (\size{[[c]]}, 0)$;
        \item [$\bullet$] $\size{[[Γ ; Φ ⊢ iN ● args ⇒> iM]]} = 
            (\size{[[args]]}, \npq{[[iN]]})$)
    \end{itemize}
    where $\size{[[v]]}$ or $\size{[[c]]}$ is the size of the 
    syntax tree of the term $[[v]]$ or $[[c]]$
    and $\size{[[args]]}$ is the sum of sizes of the terms in $[[args]]$.
\end{definition}

\begin{definition}[Number of Equivalence Nodes]
    For a tree $T$ inferring
    a declarative typing judgment,
    let us define a function $\eqNodes{T}$
    as the number of nodes in $T$ labeled with \ruleref{\ottdruleDTPEquivLabel} or 
    \ruleref{\ottdruleDTNEquivLabel}.
\end{definition}

\begin{definition}[Metric]
    \label{def:decl-typing-metric}
    For a tree $T$ inferring
    a declarative typing judgment $J$,
    let us define a metric $\metric{T}$
    as a pair $(\size{J}, \eqNodes{T})$.
\end{definition}

\lemmaAppInfEquStable*
\begin{proof}
    By \cref{lemma:equiv-completeness}, 
    $[[Γ ⊢ iN1 ≈ iN2]]$ implies $[[iN1 ≈ iN2]]$.
    Let us prove the required judgement by induction on $[[iN1 ≈ iN2]]$.
    Let us consider the last rule used in the derivation.
    \begin{caseof}
        \item \ruleref{\ottdruleEOneNVarLabel}.
            It means that $[[iN1]]$ is $[[α⁻]]$ and $[[iN2]]$ is $[[α⁻]]$.
            Then the required property coincides with the assumption. 
        \item \ruleref{\ottdruleEOneShiftULabel}. 
            It means that $[[iN1]]$ is $[[↑iP1]]$ and $[[iN2]]$ is $[[↑iP2]]$.
            where $[[iP1 ≈ iP2]]$.

            Then the only rule applicable to infer $[[Γ; Φ ⊢ ↑iP1 ● args ⇒> iM]]$
            is \ruleref{\ottdruleDTEmptyAppLabel},
            meaning that $[[args = ·]]$ and $[[Γ ⊢ ↑iP1 ≈ iM]]$.
            Then by transitivity of equivalence \cref{corollary:equivalence-transitivity},
            $[[Γ ⊢ ↑iP2 ≈ iM]]$, and then \ruleref{\ottdruleDTEmptyAppLabel} is applicable to infer
            $[[Γ; Φ ⊢ ↑iP2 ● · ⇒> iM]]$.
        
        \item \ruleref{\ottdruleEOneArrowLabel}.
            Then we are proving that  
            $[[Γ; Φ ⊢ (iQ1 → iN1) ● v, args ⇒> iM]]$ and $[[iQ1 → iN1 ≈ iQ2 → iN2]]$
            imply $[[Γ; Φ ⊢ (iQ2 → iN2) ● v, args ⇒> iM]]$.
            
            By inversion, $[[(iQ1 → iN1) ≈ (iQ2 → iN2)]]$
            means $[[iQ1 ≈ iQ2]]$ and $[[iN1 ≈ iN2]]$.

            By inversion of $[[Γ; Φ ⊢ (iQ1 → iN1) ● v, args ⇒> iM]]$:
            \begin{enumerate}
                \item $[[Γ ; Φ ⊢ v : iP]]$
                \item $[[Γ ⊢ iQ1 ≥ iP]]$,
                    and then by transitivity \cref{lemma:subtyping-transitivity},
                    $[[Γ ⊢ iQ2 ≥ iP]]$;
                \item $[[Γ ; Φ ⊢ iN1 ● args ⇒> iM]]$, 
                    and then by induction hypothesis, $[[Γ ; Φ ⊢ iN2 ● args ⇒> iM]]$.
            \end{enumerate}

            Since we have $[[Γ ; Φ ⊢ v : iP]]$, $[[Γ ⊢ iQ2 ≥ iP]]$ and 
            $[[Γ ; Φ ⊢ iN2 ● args ⇒> iM]]$, we can apply \ruleref{\ottdruleDTArrowAppLabel}
            to infer $[[Γ; Φ ⊢ (iQ2 → iN2) ● v, args ⇒> iM]]$.

        \item \ruleref{\ottdruleEOneForallLabel}
            Then we are proving that 
            $[[Γ ; Φ ⊢ ∀pas1.iN1' ● args ⇒> iM]]$ and $[[∀pas1.iN1' ≈ ∀pas2.iN2']]$
            imply $[[Γ ; Φ ⊢ ∀pas2.iN2' ● args ⇒> iM]]$.


            By inversion of $[[∀pas1.iN1' ≈ ∀pas2.iN2']]$:
            \begin{enumerate}
                \item $[[{pas2} ∩ fv iN1 = ∅]]$,
                \item there exists a bijection 
                    $[[mu : ({pas2} ∩ fv iN2') ↔ ({pas1} ∩ fv iN1')]]$
                    such that $[[iN1' ≈ [mu] iN2']]$.
            \end{enumerate}

            By inversion of $[[Γ ; Φ ⊢ ∀pas1.iN1' ● args ⇒> iM]]$:
            \begin{enumerate}
                \item $[[Γ ⊢ σ :{pas1}]]$        
                \item $[[Γ ; Φ ⊢ [σ]iN1' ● args ⇒> iM]]$
                \item $[[args ≠ ·]]$
            \end{enumerate}

            Let us construct $[[Γ ⊢ σ0 :{pas2}]]$ in the following way:
            $$
            \begin{cases}
                [[ [σ0]α⁺ =  [σ][mu]α⁺ ]] & \text{if } [[α⁺]] \in [[ {pas2} ∩ fv iN2' ]] \\
                [[ [σ0]α⁺ =  ∃β⁻.↓β⁻ ]] & \text{otherwise (the type does not matter here)} \\
            \end{cases}
            $$

            Then to infer $[[Γ ; Φ ⊢ iN2 ● args ⇒> iM]]$, we 
            apply \ruleref{\ottdruleDTArrowAppLabel} with $[[σ0]]$. 
            Let us show the required premises:
            \begin{enumerate}
                \item $[[Γ ⊢ σ0 :{pas2}]]$ by construction;
                \item $[[args ≠ ·]]$ as noted above;
                \item To show $[[Γ ; Φ ⊢ [σ0]iN2' ● args ⇒> iM]]$,
                Notice that $[[ [σ0]iN2' = [σ][mu]iN2' ]]$   
                and since $[[ [mu]iN2' ≈ iN1' ]]$, $[[ [σ][mu]iN2' ≈ [σ]iN1' ]]$.
                This way, by \cref{lemma:equiv-soundness}, $[[Γ ⊢ [σ]iN1' ≈ [σ0]iN2']]$.
                Then the required judgement holds by the induction hypothesis
                applied to $[[Γ ; Φ ⊢ [σ]iN1' ● args ⇒> iM]]$.
            \end{enumerate}
    \end{caseof}
\end{proof}

\lemmaDeclTypingContextEquiv*
\begin{proof}
    Let us prove it by induction on the $\metric{T_1}$.
    Let us consider the last rule applied in $T_1$ (i.e., its root node).
    \begin{caseof}
        \item \ruleref{\ottdruleDTVarLabel}\\
            Then we are proving 
            that $[[Γ ; Φ1 ⊢ x : iP]]$ implies $[[Γ ; Φ2 ⊢ x : iP]]$.
            By inversion, $[[x : iP ∊ Φ1]]$, and 
            since $[[Γ ⊢ Φ1 ≈ Φ2]]$, $[[x : iP' ∊ Φ2]]$ for some $[[iP']]$ 
            such that $[[Γ ⊢ iP ≈ iP']]$.
            Then we infer $[[Γ ; Φ2 ⊢ x : iP']]$ by \ruleref{\ottdruleDTVarLabel},
            and next, $[[Γ ; Φ2 ⊢ x : iP]]$ by \ruleref{\ottdruleDTPEquivLabel}.

        \item For \ruleref{\ottdruleDTThunkLabel},
              \ruleref{\ottdruleDTPAnnotLabel}, 
              \ruleref{\ottdruleDTTLamLabel},
              \ruleref{\ottdruleDTReturnLabel}, and
              \ruleref{\ottdruleDTNAnnotLabel}
              the proof is analogous. We
              apply the induction hypothesis to the premise of the rule
              to substitute $[[Φ1]]$ for $[[Φ2]]$ in it. 
              The induction is applicable because 
              the metric of the 
              premises is less than the metric of the conclusion:
              the term in the premise is a syntactic subterm of the
              term in the conclusion.

              And after that, we apply the same rule to infer the required judgement.
              
        \item \ruleref{\ottdruleDTPEquivLabel} and \ruleref{\ottdruleDTNEquivLabel}
            In these cases, the induction hypothesis is also applicable to the premise:
            although the first component of the metric 
            is the same for the premise and the conclusion:
            $\size{[[Γ ; Φ ⊢ c : iN']]} = \size{[[Γ ; Φ ⊢ c : iN]]} = \size{[[c]]}$,
            the second component of the metric is less for the premise by one,
            since the equivalence rule was applied to turn the premise tree into
            $T1$.
            Having made this note, we continue the proof in the same way as in the previous case.

        \item \ruleref{\ottdruleDTtLamLabel}
            Then we are proving that 
            $[[Γ ; Φ1 ⊢ λx:iP.c : iP → iN]]$ implies $[[Γ ; Φ2 ⊢ λx:iP.c : iP → iN]]$.
            Analogously to the previous cases, 
            we apply the induction hypothesis to the
            equivalent contexts $[[Γ ⊢ Φ1, x:iP ≈ Φ2, x:iP]]$
            and the premise $[[Γ ; Φ1, x:iP ⊢ c : iN]]$
            to obtain $[[Γ ; Φ2, x:iP ⊢ c : iN]]$.
            Notice that $[[c]]$ is a subterm of $[[λx:iP.c]]$,
            i.e., the metric of the premise tree is less than the metric of the conclusion, 
            and the induction hypothesis is applicable.
            Then we infer $[[Γ ; Φ2 ⊢ λx:iP.c : iP → iN]]$ by \ruleref{\ottdruleDTtLamLabel}.

        \item \ruleref{\ottdruleDTVarLetLabel}
            Then we are proving that 
            $[[Γ ; Φ1 ⊢ let x = v; c : iN]]$ implies $[[Γ ; Φ2 ⊢ let x = v; c : iN]]$.
            First, we apply the induction hypothesis to 
            $[[Γ; Φ1 ⊢ v : iP]]$ to obtain $[[Γ; Φ2 ⊢ v : iP]]$ 
            of the same pure size.
            
            Then we apply the induction hypothesis to
            the equivalent contexts $[[Γ ⊢ Φ1, x:iP ≈ Φ2, x:iP]]$
            and the premise $[[Γ ; Φ1, x:iP ⊢ c : iN]]$ to obtain
            $[[Γ ; Φ2, x:iP ⊢ c : iN]]$.
            Then we infer $[[Γ ; Φ2 ⊢ let x = v; c : iN]]$ by \ruleref{\ottdruleDTVarLetLabel}.

        \item \ruleref{\ottdruleDTAppLetLabel}
            Then we are proving that 
            $[[Γ ; Φ1 ⊢ let x = v(args); c : iN]]$ implies 
            $[[Γ ; Φ2 ⊢ let x = v(args); c : iN]]$.

            We apply the induction hypothesis to each of the premises.
            to rewrite:
            \begin{itemize}
                \item $[[Γ ; Φ1 ⊢ v : ↓iM]]$ into $[[Γ ; Φ2 ⊢ v : ↓iM]]$,
                \item $[[Γ ; Φ1 ⊢ iM ● args ⇒> ↑iQ]]$ into $[[Γ ; Φ2 ⊢ iM ● args ⇒> ↑iQ]]$.
                \item $[[Γ ; Φ1, x:iQ ⊢ c : iN]]$ into $[[Γ ; Φ2, x:iQ ⊢ c : iN]]$
                (notice that $[[Γ ⊢ Φ1, x:iQ ≈ Φ2, x:iQ]]$).
            \end{itemize}

            It is left to show the uniqueness of $[[Γ ; Φ2 ⊢ iM ● args ⇒> ↑iQ]]$.
            Let us assume that this judgement holds for other $[[iQ']]$, 
            i.e.  there exists a tree $T_0$ inferring 
            $[[Γ ; Φ2 ⊢ iM ● args ⇒> ↑iQ']]$.
            Then notice that the induction hypothesis is applicable to
            $T_0$: the first component of the first component of $\metric{T_0}$
            is $S = \sum_{[[v]] \in [[args]]} \size{[[v]]}$, and it is less than
            the corresponding component of $\metric{T_1}$, which is
            $\size{[[let x = v(args); c]]} = 1 + \size{[[v]]} + \size{[[c]]} + S$.
            This way, $[[Γ ; Φ1 ⊢ iM ● args ⇒> ↑iQ']]$ holds by the induction hypothesis,
            but since $[[Γ ; Φ1 ⊢ iM ● args ⇒> ↑iQ uniq]]$, we have $[[Γ ⊢ iQ' ≈ iQ]]$.

            Then we infer $[[Γ ; Φ2 ⊢ let x = v(args); c : iN]]$ by \ruleref{\ottdruleDTAppLetLabel}.

        \item \ruleref{\ottdruleDTAppLetAnnLabel}
            Then we are proving that
            $[[Γ ; Φ1 ⊢ let x:iP = v(args); c : iN]]$ implies
            $[[Γ ; Φ2 ⊢ let x:iP = v(args); c : iN]]$.
        
            As in the previous case, we apply the induction hypothesis to each of the premises
            and rewrite:
            \begin{itemize}
                \item $[[Γ ; Φ1 ⊢ v : ↓iM]]$ into $[[Γ ; Φ2 ⊢ v : ↓iM]]$,
                \item $[[Γ ; Φ1 ⊢ iM ● args ⇒> ↑iQ]]$ into $[[Γ ; Φ2 ⊢ iM ● args ⇒> ↑iQ]]$, 
                    and
                \item $[[Γ ; Φ1, x:iP ⊢ c : iN]]$ into $[[Γ ; Φ2, x:iP ⊢ c : iN]]$
                (notice that $[[Γ ⊢ Φ1, x:iP ≈ Φ2, x:iP]]$).
            \end{itemize}
            
            Notice that $[[Γ ⊢ iP]]$ and $[[Γ ⊢ ↑iQ ≤ ↑iP]]$ 
            do not depend on the variable context, and hold by assumption.
            Then we infer $[[Γ ; Φ2 ⊢ let x:iP = v(args); c : iN]]$ by \ruleref{\ottdruleDTAppLetAnnLabel}.

        \item \ruleref{\ottdruleDTUnpackLabel}, and \ruleref{\ottdruleDTNAnnotLabel}
            are proved in the same way.

        \item \ruleref{\ottdruleDTEmptyAppLabel}
            Then we are proving that 
            $[[Γ ; Φ1 ⊢ iN ● · ⇒> iN']]$ (inferred by \ruleref{\ottdruleDTEmptyAppLabel})
            implies $[[Γ ; Φ2 ⊢ iN ● · ⇒> iN']]$.

            To infer $[[Γ ; Φ2 ⊢ iN ● · ⇒> iN']]$, 
            we apply \ruleref{\ottdruleDTEmptyAppLabel}, noting that 
            $[[Γ ⊢ iN ≈ iN']]$ holds by assumption.

        \item \ruleref{\ottdruleDTArrowAppLabel}
            Then we are proving that 
            $[[Γ ; Φ1 ⊢ iQ → iN ● v, args ⇒> iM]]$ (inferred by \ruleref{\ottdruleDTArrowAppLabel})
            implies $[[Γ ; Φ2 ⊢ iQ → iN ● v, args ⇒> iM]]$.
            And uniqueness of the $[[iM]]$ in the first case implies uniqueness in the second case.

            By induction, we rewrite $[[Γ ; Φ1 ⊢ v : iP]]$ into $[[Γ ; Φ2 ⊢ v : iP]]$, 
            and $[[Γ ; Φ1 ⊢ iN ● args ⇒> iM]]$ into $[[Γ ; Φ2 ⊢ iN ● args ⇒> iM]]$.
            Then we infer $[[Γ ; Φ2 ⊢ iQ → iN ● v, args ⇒> iM]]$ by \ruleref{\ottdruleDTArrowAppLabel}.

            Now, let us show the uniqueness.
            The only rule that can infer $[[Γ ; Φ1 ⊢ iQ → iN ● v, args ⇒> iM]]$
            is \ruleref{\ottdruleDTArrowAppLabel}.
            Then by inversion, 
            uniqueness of $[[Γ ; Φ1 ⊢ iQ → iN ● v, args ⇒> iM]]$ implies
            uniqueness of $[[Γ ; Φ1 ⊢ iN ● args ⇒> iM]]$. By 
            the induction hypothesis, it implies the uniqueness of 
            $[[Γ ; Φ2 ⊢ iN ● args ⇒> iM]]$.


            Suppose that 
            $[[Γ ; Φ2 ⊢ iQ → iN ● v, args ⇒> iM']]$.
            By inversion, $[[Γ ; Φ2 ⊢ iN ● args ⇒> iM']]$, 
            which by uniqueness of $[[Γ ; Φ2 ⊢ iN ● args ⇒> iM]]$ implies
            $[[Γ ⊢ iM ≈ iM']]$.

        \item \ruleref{\ottdruleDTForallAppLabel}
            Then we are proving that
            $[[Γ ; Φ1 ⊢ ∀pas.iN ● args ⇒> iM]]$ (inferred by \ruleref{\ottdruleDTForallAppLabel})
            implies $[[Γ ; Φ2 ⊢ ∀pas.iN ● args ⇒> iM]]$.

            By inversion, we have $[[σ]]$ such that $[[Γ ⊢ σ :{pas}]]$ and
            $[[Γ ; Φ1 ⊢ [σ]iN ● args ⇒> iM]]$ is inferred.
            Let us denote the inference tree as $T_1'$.
            Notice that the induction hypothesis is applicable to $T_1'$:
            $\metric{T_1'} = ((\size{[[args]]}, 0), x)$ is less than 
            $\metric{T_1} = ((\size{[[args]]}, |[[pas]]|), y)$ for any $x$ and $y$,
            since $|[[pas]]| > 0$ by inversion.

            This way, by the induction hypothesis, 
            there exists a tree $T_2'$ inferring
            $[[Γ ; Φ2 ⊢ [σ]iN ● args ⇒> iM]]$.
            Notice that the premises $[[args ≠ ·]]$, $[[Γ ⊢ σ :{pas}]]$,
            and $[[pas ≠ ·]]$ do not depend on the variable context,
            and hold by inversion.
            Then we infer $[[Γ ; Φ2 ⊢ ∀pas.iN ● args ⇒> iM]]$ by \ruleref{\ottdruleDTForallAppLabel}.
    \end{caseof}
\end{proof}

\subsection{Algorithmic Typing}
\lemmaTypingDeterminacy*
\begin{proof}
    We show it by structural induction on the inference tree.
    Notice that the last rule used to infer the judgement is uniquely
    determined by the input, and that each premise
    of each inference rule is deterministic by the corresponding 
    observation.
\end{proof}

Let us extend the declarative typing metric (\cref{def:decl-typing-metric}) 
to the algorithmic typing.

\begin{definition}[Size of an Algorithmic Judgement]
    \label{def:algorithmic-typing-size}
    For an algorithmic typing judgement $J$
    let us define a metrics $\size{J}$ as a pair of numbers 
    in the following way:
    \begin{itemize}
        \item [$+$] $\size{[[Γ ; Φ ⊨ v : iP]]} = (\size{[[v]]}, 0)$;
        \item [$-$] $\size{[[Γ ; Φ ⊨ c : iN]]} = (\size{[[c]]}, 0)$;
        \item [$\bullet$] $\size{[[Γ ; Φ ; Θ ⊨ uN ● args ⇒> uM ⫤ Θ'; SC]]} = 
            (\size{[[args]]}, \npq{[[uN]]})$)
    \end{itemize}
\end{definition}

\begin{definition}[Metric]
    We extend the metric from \cref{def:decl-typing-metric} to the algorithmic typing
    in the following way.
    For a tree $T$ inferring an algorithmic typing judgement $J$, we define 
    $\metric{T}$ as $(\size{J}, 0)$.
\end{definition}

Soundness and completeness are proved by mutual induction on
the metric of the inference tree.

\lemmaTypingSoundness*
\begin{proof}
    We prove it by induction on $\metric{T_1}$, mutually with 
    the completeness of typing (\cref{lemma:typing-soundness}).
    Let us consider the last rule used to infer the derivation.
    \begin{caseof}
        \item \ruleref{\ottdruleATVarLabel}
            We are proving that if $[[Γ; Φ ⊨ x : nf(iP)]]$
            then $[[Γ ⊢ nf(iP)]]$ and $[[Γ; Φ ⊢ x : nf(iP)]]$.

            By inversion, $[[x : iP ∊ Φ]]$.
            Since $[[Γ ⊢ Φ]]$, we have $[[Γ ⊢ iP]]$,
            and by \cref{corollary:wf-nf}, $[[Γ ⊢ nf(iP)]]$.

            By applying \ruleref{\ottdruleDTVarLabel}
            to $[[x : iP ∊ Φ]]$, we infer $[[Γ; Φ ⊢ x : iP]]$.
            Finally, by \ruleref{\ottdruleDTPEquivLabel}, 
            since $[[Γ ⊢ iP ≈ nf(iP)]]$ 
            (\cref{corollary:nf-sound-wrt-subt-equiv}),
            we have $[[Γ; Φ ⊢ x : nf(iP)]]$.

        \item \ruleref{\ottdruleATThunkLabel}
            \label{case:typing-soundness:thunk}

            We are proving that if $[[Γ; Φ ⊨ {c} : ↓iN]]$
            then $[[Γ ⊢ ↓iN]]$ and $[[Γ; Φ ⊢ {c} : ↓iN]]$.

            By inversion of $[[Γ; Φ ⊨ {c} : ↓iN]]$, 
            we have $[[Γ; Φ ⊨ c : iN]]$.
            By the induction hypothesis applied to $[[Γ; Φ ⊨ c : iN]]$,
            we have 
            \begin{enumerate}
                \item $[[Γ ⊢ iN]]$, and hence, $[[Γ ⊢ ↓iN]]$;
                \item $[[Γ; Φ ⊢ c : iN]]$, 
                    which by \ruleref{\ottdruleDTThunkLabel} implies
                    $[[Γ; Φ ⊢ {c} : ↓iN]]$.
            \end{enumerate}

        \item \ruleref{\ottdruleATReturnLabel}
            The proof is symmetric to the previous case
            (\cref{case:typing-soundness:thunk}).

        \item \ruleref{\ottdruleATPAnnotLabel}
            \label{case:typing-soundness:pos-annot}
            We are proving that if $[[Γ; Φ ⊨ (v : iQ) : nf(iQ)]]$
            then $[[Γ ⊢ nf(iQ)]]$ and $[[Γ; Φ ⊢ (v : iQ) : nf(iQ)]]$.

            By inversion of $[[Γ; Φ ⊨ (v : iQ) : nf(iQ)]]$,
            we have:
            \begin{enumerate}
                \item $[[Γ ⊢ (v : iQ)]]$, hence, $[[Γ ⊢ iQ]]$, 
                    and by \cref{corollary:wf-nf}, $[[Γ ⊢ nf(iQ)]]$;
                \item $[[Γ; Φ ⊨ v : iP]]$, 
                    which by the induction hypothesis implies
                    $[[Γ ⊢ iP]]$ and $[[Γ; Φ ⊢ v : iP]]$;
                \item $[[Γ ; · ⊨ uQ ≥ iP ⫤ ·]]$,
                    which by \cref{lemma:pos-subt-soundness} implies
                    $[[Γ ⊢ [·]uQ ≥ iP]]$, that is $[[Γ ⊢ iQ ≥ iP]]$.
            \end{enumerate}

            To infer 
            $[[Γ; Φ ⊢ (v : iQ) : iQ]]$, 
            we apply \ruleref{\ottdruleDTPAnnotLabel} 
            to $[[Γ; Φ ⊢ v : iP]]$ and $[[Γ ⊢ iQ ≥ iP]]$.
            Then by \ruleref{\ottdruleDTPEquivLabel},
            $[[Γ; Φ ⊢ (v : iQ) : nf(iQ)]]$.
  
        \item \ruleref{\ottdruleATNAnnotLabel}
            The proof is symmetric to the previous case
            (\cref{case:typing-soundness:pos-annot}).

        \item \ruleref{\ottdruleATtLamLabel}
            We are proving that if 
            $[[Γ; Φ ⊨ λx:iP . c : nf(iP → iN)]]$
            then 
            $[[Γ ⊢ nf(iP → iN)]]$ and
            $[[Γ; Φ ⊢ λx:iP . c : nf(iP → iN)]]$.


            By inversion of $[[Γ; Φ ⊨ λx:iP . c : nf(iP → iN)]]$,
            we have $[[Γ ⊢ λx:iP . c]]$, 
            which implies $[[Γ ⊢ iP]]$.

            Also by inversion of $[[Γ; Φ ⊨ λx:iP . c : nf(iP → iN)]]$,
            we have $[[Γ; Φ, x:iP ⊨ c : iN]]$, applying induction hypothesis to which gives us:
            \begin{enumerate}
                \item  $[[Γ ⊢ iN]]$, thus $[[Γ ⊢ iP → iN]]$, 
                    and by \cref{corollary:wf-nf}, $[[Γ ⊢ nf(iP → iN)]]$;
                \item $[[Γ; Φ, x:iP ⊢ c : iN]]$, 
                    which by \ruleref{\ottdruleDTtLamLabel} implies
                    $[[Γ; Φ ⊢ λx:iP . c : iP → iN]]$, 
                    and by \ruleref{\ottdruleDTPEquivLabel},
                    $[[Γ; Φ ⊢ λx:iP . c : nf(iP → iN)]]$.
            \end{enumerate}

        \item \ruleref{\ottdruleATTLamLabel}
            We are proving that if
            $[[Γ; Φ ⊨ Λα⁺ . c : nf(∀α⁺.iN)]]$
            then 
            $[[Γ ; Φ ⊢ Λα⁺ . c : nf(∀α⁺.iN)]]$
            and
            $[[Γ ⊢ nf(∀α⁺.iN)]]$.

            
            By inversion of $[[Γ, α⁺ ; Φ ⊨ c : iN]]$,
            we have $[[Γ ⊢ Λα⁺ . c]]$, which implies $[[Γ, α⁺ ⊢ c]]$.

            Also by inversion of 
            $[[Γ, α⁺ ; Φ ⊨ c : iN]]$, 
            we have $[[Γ, α⁺ ; Φ ⊨ c : iN]]$. 
            Obtaining the induction hypothesis to $[[Γ, α⁺ ; Φ ⊨ c : iN]]$,
            we have:
            \begin{enumerate}
                \item $[[Γ, α⁺ ⊢ iN]]$, thus $[[Γ ⊢ ∀α⁺.iN]]$,
                    and by \cref{corollary:wf-nf}, $[[Γ ⊢ nf(∀α⁺.iN)]]$;
                \item $[[Γ, α⁺ ; Φ ⊢ c : iN]]$, 
                    which by \ruleref{\ottdruleDTTLamLabel} implies
                    $[[Γ ; Φ ⊢ Λα⁺ . c : ∀α⁺.iN]]$, 
                    and by \ruleref{\ottdruleDTPEquivLabel},
                    $[[Γ ; Φ ⊢ Λα⁺ . c : nf(∀α⁺.iN)]]$.
            \end{enumerate}
            
        \item \ruleref{\ottdruleATVarLetLabel}
            We are proving that if
            $[[Γ; Φ ⊨ let x = v ; c : iN]]$
            then
            $[[Γ ; Φ ⊢ let x = v ; c : iN]]$ and 
            $[[Γ ⊢ iN]]$.


            By inversion of 
            $[[Γ; Φ ⊨ let x = v ; c : iN]]$,
            we have:
            \begin{enumerate}
                \item $[[Γ ⊢ let x = v ; c]]$, which gives us 
            $[[Γ ⊢ v]]$ and $[[Γ ⊢ c]]$.
                \item $[[Γ; Φ ⊨ v : iP]]$, 
                    which by the induction hypothesis implies
                    $[[Γ ⊢ iP]]$ (and thus, $[[Γ ⊢ Φ, x:iP]]$) 
                    and $[[Γ; Φ ⊢ v : iP]]$; 
                \item $[[Γ; Φ, x:iP ⊨ c : iN]]$, 
                    which by the induction hypothesis implies
                    $[[Γ ⊢ iN]]$ and $[[Γ; Φ, x:iP ⊢ c : iN]]$.
            \end{enumerate}
            This way, 
            $[[Γ; Φ ⊢ let x = v ; c : iN]]$ holds by
            \ruleref{\ottdruleDTVarLetLabel}.

        \item \ruleref{\ottdruleATAppLetAnnLabel}
            We are proving that 
            if
            $[[Γ; Φ ⊨ let x:iP = v(args); c' : iN]]$
            then 
            $[[Γ ; Φ ⊢ let x:iP = v(args); c' : iN]]$ and
            $[[Γ ⊢ iN]]$.

            By inversion, we have:
            \begin{enumerate}
                \item $[[Γ ⊢ iP]]$, hence, $[[Γ ⊢ Φ, x:iP]]$
                \item $[[Γ; Φ ⊨ v : ↓iM]]$
                \item $[[Γ; Φ; · ⊨ uM ● args ⇒> ↑uQ ⫤ Θ; SC1]]$
                \item $[[Γ; Θ ⊨ ↑uQ ≤ ↑iP ⫤ SC2]]$
                \item $[[Θ ⊢ SC1 & SC2 = SC]]$
                \item $[[Γ; Φ, x:iP ⊨ c' : iN]]$
            \end{enumerate}

            By the induction hypothesis applied to $[[Γ; Φ ⊨ v : ↓iM]]$, we have
            $[[Γ; Φ ⊢ v : ↓iM]]$ and $[[Γ ⊢ ↓iM]]$ (and hence, $[[Γ ; dom(Θ) ⊢  uM]]$).

            By the induction hypothesis applied to $[[Γ; Φ, x:iP ⊨ c' : iN]]$, we have
            $[[Γ; Φ, x:iP ⊢ c' : iN]]$ and $[[Γ ⊢ iN]]$. 

            By the induction hypothesis applied to $[[Γ; Φ; · ⊨ uM ● args ⇒> ↑uQ ⫤ Θ; SC1]]$, we have:
            \begin{enumerate}
                \item \label{typing-soundness:theta-wf} $[[Γ ⊢ Θ]]$,
                \item $[[Γ; dom(Θ) ⊢ ↑uQ]]$,
                \item $[[Θ'|uv uM ∪ uv uQ ⊢ SC1]]$,
                    and thus, $[[dom(SC1) ⊆ uv uM ∪ uv uQ]]$.
                \item \label{typing-soundness:SC1-initiality} 
                    for any $[[ Θ' ⊢ uσ : SC1 ]]$, we have $[[ Γ ; Φ ⊢ [uσ]uM ● args ⇒> [uσ]↑uQ ]]$.
            \end{enumerate}

            By soundness of negative subtyping (\cref{lemma:neg-subt-soundness})
            applied to $[[Γ; Θ ⊨ ↑uQ ≤ ↑iP ⫤ SC2]]$, we have
            $[[Θ ⊢ SC2 : uv(↑uQ)]]$, and thus, $[[uv(↑uQ) = dom(SC2)]]$.

            By soundness of constraint merge (\cref{lemma:merge-soundness}),
            $[[dom(SC) = dom(SC1) ∪ dom(SC2)]] \subseteq [[uv uM ∪ uv uQ]]$
            Then by \cref{lemma:constraint-sat},
            let us take $[[uσ]]$ such that
            $[[ Θ ⊢ uσ : uv(uM) ∪ uv(uQ) ]]$ and 
            $[[ Θ ⊢ uσ : SC ]]$.
            By the soundness of constraint merge, 
            $[[ Θ ⊢ uσ : SC1 ]]$ and $[[ Θ  ⊢ uσ : SC2 ]]$,
            and by weakening, $[[ Θ' ⊢ uσ : SC1 ]]$ and $[[ Θ' ⊢ uσ : SC2 ]]$.

            Then as noted above (\ref{typing-soundness:SC1-initiality}),
            $[[ Γ ; Φ ⊢ iM ● args ⇒> [uσ]↑uQ ]]$
            And again, by soundness of negative subtyping (\cref{lemma:neg-subt-soundness})
            applied to $[[Γ; Θ ⊨ ↑uQ ≤ ↑iP ⫤ SC2]]$,
            we have
            $[[Γ ⊢ [uσ]↑uQ ≤ ↑iP]]$.

            To infer $[[Γ ; Φ ⊢ let x : iP = v(args); c' : iN ]]$,
            we apply the corresponding declarative rule \ruleref{\ottdruleDTAppLetAnnLabel}, where
            $[[iQ]]$ is $[[ [uσ]uQ  ]]$. Notice that all the premises were already shown to
            hold above:
            \begin{enumerate}
                \item $[[Γ ⊢ iP]]$ and $[[Γ; Φ ⊢ v : ↓iM]]$ from the assumption,
                \item $[[Γ; Φ ⊢ iM ● args ⇒> ↑[uσ]uQ]]$ holds since $[[ [uσ]↑uQ ]] = [[ ↑[uσ]uQ ]]$,
                \item $[[Γ ⊢ ↑[uσ]uQ ≤ ↑iP]]$ by soundness of negative subtyping,
                \item $[[Γ; Φ, x:iP ⊢ c' : iN]]$ from the the induction hypothesis.
            \end{enumerate}

        \item \ruleref{\ottdruleATAppLetLabel}
            We are proving that if
            $[[Γ; Φ ⊨ let x = v(args); c' : iN]]$
            then
            $[[Γ ; Φ ⊢ let x = v(args); c' : iN]]$ and
            $[[Γ ⊢ iN]]$.

            By the inversion, we have:
            \begin{enumerate}
                \item $[[Γ; Φ ⊨ v : ↓iM]]$, 
                \item $[[Γ ; Φ ; · ⊨ uM ● args ⇒> ↑uQ ⫤ Θ; SC]]$, 
                \item $[[Γ ⊢ uQ SC minby uσ]]$, and
                \item $[[Γ; Φ, x:[uσ]uQ ⊨ c' : iN]]$.
            \end{enumerate}

            By the induction hypothesis applied to $[[Γ; Φ ⊨ v : ↓iM]]$, we have    
            $[[Γ; Φ ⊢ v : ↓iM]]$ and $[[Γ ⊢ ↓iM]]$ (and thus, $[[Γ ; dom(Θ) ⊢ uM]]$).
       
            By the induction hypothesis applied to $[[Γ; Φ, x:[uσ]uQ ⊨ c' : iN]]$, we have
            $[[Γ ⊢ iN]]$ and $[[Γ; Φ, x:[uσ]uQ ⊢ c' : iN]]$.

            By the induction hypothesis applied to 
            $[[Γ ; Φ ; · ⊨ uM ● args ⇒> ↑uQ ⫤ Θ; SC]]$, we have:
            \begin{enumerate}
                \item $[[Γ ⊢ Θ]]$
                \item $[[Γ; dom(Θ) ⊢  ↑uQ]]$
                \item $[[Θ|uv uM ∪ uv uQ ⊢ SC]]$ (and thus, $[[dom(SC) ⊆ uv uM ∪ uv uQ]]$)
                \item for any $[[ Θ ⊢ uσ : SC ]]$, we have $[[ Γ ; Φ ⊢ [uσ]uM ● args ⇒> [uσ]↑uQ ]]$, 
                    which, since  $[[iM]]$ is ground means $[[ Γ ; Φ ⊢ iM ● args ⇒> ↑[uσ]uQ]]$.
            \end{enumerate}

            To infer $[[Γ ; Φ ⊢ let x = v(args) ; c' : iN ]]$, 
            we apply the corresponding 
            declarative rule \ruleref{\ottdruleDTAppLetLabel}.
            Let us show that the premises hold:
            \begin{itemize}
                \item $[[Γ; Φ ⊢ v : ↓iM]]$ holds by the induction hypothesis;
                \item $[[Γ; Φ, x:[uσ]uQ ⊢ c' : iN]]$ also holds by the induction hypothesis, as noted above;
                \item  $[[Γ; Φ ⊢ iM ● args ⇒> ↑[uσ]uQ ]]$ holds, as noted above;
                \item To show the principality of $[[↑[uσ]uQ]]$,
                    we assume that for some other type $[[iR]]$ 
                    holds $[[Γ; Φ ⊢ iM ● args ⇒> ↑iR ]]$,
                    that is $[[Γ; Φ ⊢ [·]uM ● args ⇒> ↑iR ]]$.
                    Then by the completeness of typing 
                    (\cref{lemma:typing-completeness}),
                    there exist $[[uN']]$, $[[Θ']]$, and $[[SC']]$ such that
                    \begin{enumerate}
                        \item $[[ Γ; Φ; · ⊨ uM ● args ⇒> uN' ⫤ Θ'; SC' ]]$ and
                        \item there exists a substitution $[[Θ' ⊢ uσ' : SC']]$ 
                            such that $[[Γ ⊢ [uσ']uN' ≈ ↑iR]]$.
                    \end{enumerate}
                    By determinacy of the typing algorithm (\cref{lemma:typing-determinacy}),
                    $[[ Γ; Φ; · ⊨ uM ● args ⇒> uN' ⫤ Θ'; SC' ]]$,
                    means that $[[SC']]$ is $[[SC]]$, $[[Θ']]$ is $[[Θ]]$, and $[[uN']]$ is
                    $[[↑uQ]]$. 
                    This way, $[[Γ ⊢ [uσ']↑uQ ≈ ↑iR]]$ for substitution 
                    $[[Θ ⊢ uσ' : SC]]$. 
                    To show the principality, it suffices to notice that
                    $[[Γ ⊢ iR ≥ [uσ]uQ]]$ or equivalently $[[Γ ⊢ [uσ']uQ ≥ [uσ]uQ]]$,
                    which holds by the soundness of the minimal instantiation 
                    (\cref{lemma:min-inst-soundness}) since $[[Γ ⊢ uQ SC minby uσ]]$.
            \end{itemize}

        \item \ruleref{\ottdruleATUnpackLabel}
            We are proving that if 
            $[[Γ; Φ ⊨ let∃ (nas, x) = v; c' : iN]]$
            then
            $[[Γ ; Φ ⊢ let∃ (nas, x) = v; c' : iN]]$ and
            $[[Γ ⊢ iN]]$.
            By the inversion, we have:
            \begin{enumerate}
                \item $[[Γ; Φ ⊨ v : ∃nas.iP]]$
                \item $[[Γ, nas ; Φ, x:iP ⊨ c' : iN]]$
                \item $[[Γ ⊢ iN]]$
            \end{enumerate}

            By the induction hypothesis applied to 
            $[[Γ; Φ ⊨ v : ∃nas.iP]]$, we have $[[Γ; Φ ⊢ v : ∃nas.iP]]$
            and $[[∃nas.iP]]$ is normalized.
            By the induction hypothesis applied to
            $[[Γ, nas ; Φ, x:iP ⊨ c' : iN]]$, we have $[[Γ, nas ; Φ, x:iP ⊢ c' : iN]]$.

            To show $[[Γ; Φ ⊢ let∃ (nas, x) = v; c' : iN]]$, we apply the corresponding
            declarative rule \ruleref{\ottdruleDTUnpackLabel}. Let us show that the premises hold:
            \begin{enumerate}
                \item $[[Γ ; Φ ⊢ v : ∃nas.iP]]$ holds by the induction hypothesis, as noted above,
                \item $[[nf(∃nas.iP) = ∃nas.iP]]$ holds since $[[∃nas.iP]]$ is normalized,
                \item $[[Γ, nas ; Φ, x:iP ⊢ c' : iN]]$ also holds by the induction hypothesis,
                \item $[[Γ ⊢ iN]]$ holds by the inversion, as noted above.
            \end{enumerate}

        \item \ruleref{\ottdruleATEmptyAppLabel}
            Then by assumption:
            \begin{itemize}
                \item $[[Γ ⊢ Θ]]$,
                \item $[[Γ; dom(Θ) ⊢  uN]]$ is free from negative algorithmic variables,
                \item $[[Γ; Φ; Θ ⊨ uN ● · ⇒> nf(uN) ⫤ Θ; ·]]$.
            \end{itemize}

            Let us show the required properties: 
            \begin{enumerate}
                \item $[[Γ ⊢ Θ]]$ holds by assumption,
                \item $[[Θ]] \subseteq [[Θ]]$ holds trivially,
                \item $[[nf(uN)]]$ is evidently normalized, 
                    $[[Γ; dom(Θ) ⊢  uN]]$ implies $[[Γ; dom(Θ) ⊢  nf(uN)]]$ by 
                    \cref{corollary:wf-nf-algo},
                    and \cref{lemma:fv-nf} means that $[[nf(uN)]]$ is 
                    inherently free from negative algorithmic variables,
                \item $[[dom(Θ) ∩ uv(nf(uN)) ⊆ uv uN]]$
                    holds since $[[uv(nf(uN)) = uv(uN)]]$,
                \item $[[Θ|uv uN ∪ uv nf(uN) ⊢ ·]]$ holds trivially,
                \item suppose that $[[ Θ ⊢ uσ : uv uN ∪ uv nf(uN) ]]$.
                    To show $[[ Γ ; Φ ⊢ [uσ]uN ● · ⇒> [uσ]nf(uN) ]]$, we apply the corresponding 
                    declarative rule \ruleref{\ottdruleDTEmptyAppLabel}.
                    To show $[[ Γ ⊢ [uσ]uN ≈ [uσ]nf(uN) ]]$,
                    we apply the following sequence:
                    $[[uN ≈ nf(uN)]]$ by 
                    \cref{lemma:normalization-soundness},
                    then $[[ [uσ]uN ≈ [uσ]nf(uN) ]]$
                    by \cref{corollary:subst-pres-decl-equiv},
                    then $[[ Γ ⊢ [uσ]uN ≈ [uσ]nf(uN) ]]$
                    by \cref{lemma:equiv-soundness}. 
            \end{enumerate}

        \item \ruleref{\ottdruleATArrowAppLabel} 
            By assumption:
            \begin{enumerate}
                \item $[[Γ ⊢ Θ]]$,
                \item $[[Γ; dom(Θ) ⊢ uQ → uN]]$ is free from negative algorithmic variables,
                    and hence, so are $[[uQ]]$ and $[[uN]]$,
                \item $[[Γ; Φ; Θ ⊨ uQ → uN ● v , args ⇒> uM ⫤ Θ'; SC]]$, 
                    and by inversion: 
                    \begin{enumerate}
                        \item $[[Γ; Φ ⊨ v : iP]]$,
                            and by the induction hypothesis applied to this judgment,
                            we have $[[Γ; Φ ⊢ v : iP]]$, and $[[Γ ⊢ iP]]$;
                        \item $[[Γ; Θ ⊨ uQ ≥ iP ⫤ SC1]]$,
                            and by the soundness of subtyping:
                            $[[Θ ⊢ SC1 : uv uQ]]$ 
                            (and thus, $[[dom(SC1) = uv uQ]]$),
                            and
                            for any
                            
                            $[[ Θ ⊢ uσ : SC1 ]]$, we have $[[Γ ⊢ [uσ]uQ ≥ iP]]$;
                        \item $[[Γ; Φ; Θ ⊨ uN ● args ⇒> uM ⫤ Θ'; SC2]]$,
                            and by the induction hypothesis applied to this judgment,
                            \begin{enumerate}
                                \item $[[Γ ⊢ Θ']]$,
                                \item $[[Θ ⊆ Θ']]$,
                                \item $[[Γ; dom(Θ') ⊢  uM]]$ is normalized and free from negative algorithmic variables,
                                \item $[[dom(Θ) ∩ uv(uM) ⊆ uv uN]]$,
                                \item $[[Θ'|uv(uM) ∪ uv(uN) ⊢ SC2]]$, and thus, $[[dom(SC2) ⊆ uv(uM) ∪ uv(uN)]]$,
                                \item for any $[[uσ]]$ such that
                                    $[[Θ ⊢ uσ : uv(uM) ∪ uv(uN) ]]$
                                    and 
                                    $[[ Θ' ⊢ uσ : SC2 ]]$, we have 
                                    $[[ Γ ; Φ ⊢ [uσ]uN ● args ⇒> [uσ]uM ]]$;
                            \end{enumerate}
                        \item $[[Θ ⊢ SC1 & SC2 = SC]]$,
                            which by \cref{lemma:merge-soundness} implies
                            $[[dom(SC) = dom(SC1) ∪ dom(SC2)]] \subseteq [[uv uQ ∪ uv uM ∪ uv uN]]$.
                    \end{enumerate}
            \end{enumerate}

            Let us show the required properties:
            \begin{enumerate}
                \item $[[Γ ⊢ Θ']]$ is shown above,
                \item $[[Θ ⊆ Θ']]$ is shown above,
                \item $[[Γ; dom(Θ') ⊢  uM]]$ is normalized and free from negative algorithmic variables, as shown above,
                \item $[[dom(Θ) ∩ uv(uM) ⊆ uv uN ⊆ uv(uQ → uN)]]$
                    (the first inclusion is shown above, the second one is by definition),
                \item To show $[[Θ'|uv(uQ) ∪ uv(uN) ∪ uv(uM) ⊢ SC]]$,
                    first let us notice that $[[uv(uQ) ∪ uv(uN) ∪ uv(uM) ⊆ dom(SC)]]$,
                    as mentioned above.
                    Then we demonstrate 
                    $[[Θ' ⊢ SC]]$:
                    $[[Θ ⊢ SC1]]$ and $[[Θ ⊆ Θ']]$ imply $[[Θ' ⊢ SC1]]$,
                    by the soundness of constraint merge (\cref{lemma:merge-soundness})
                    applied to $[[Θ' ⊢ SC1 & SC2 = SC]]$:
                    \begin{enumerate}
                        \item $[[Θ' ⊢ SC]]$,
                        \item for any $[[ Θ' ⊢ uσ : SC ]]$, $[[ Θ' ⊢ uσ : SCi ]]$ holds;
                    \end{enumerate}
                \item Suppose that 
                    $[[ Θ' ⊢ uσ : uv(uQ) ∪ uv(uN) ∪ uv(uM) ]]$
                    and $[[ Θ' ⊢ uσ : SC ]]$.
                    To show $[[ Γ ; Φ ⊢ [uσ](uQ → uN) ● v , args ⇒> [uσ]uM ]]$, 
                    that is $[[ Γ ; Φ ⊢ [uσ]uQ → [uσ]uN ● v , args ⇒> [uσ]uM ]]$,
                    we apply the corresponding declarative rule \ruleref{\ottdruleDTArrowAppLabel}.
                    Let us show the required premises:
                    \begin{enumerate}
                        \item $[[Γ; Φ ⊢ v : iP]]$ holds as shown above,
                        \item $[[Γ ⊢ [uσ]uQ ≥ iP]]$ holds 
                            since $[[Γ ⊢ [uσ|uv(uQ)]uQ ≥ iP]]$ by the soundness of subtyping 
                            as noted above:
                            since $[[ Θ' ⊢ uσ : SC ]]$ implies $[[ Θ' ⊢ uσ|uv(uQ) : SC1 ]]$,
                            which we strengthen to $[[ Θ ⊢ uσ|uv(uQ) : SC1 ]]$,
                        \item $[[Γ; Φ ⊢ [uσ]uN ● args ⇒> [uσ]uM]]$ holds by the induction hypothesis
                            as shown above,
                            since $[[ Θ' ⊢ uσ : SC ]]$ implies $[[ Θ' ⊢ uσ : SC2 ]]$,
                            and then $[[ Θ' ⊢ uσ | uv(uN) ∪ uv(uM) : SC2 ]]$
                            and $[[ Θ ⊢ uσ | uv(uN) ∪ uv(uM) : uv(uN) ∪ uv(uM) ]]$.
                    \end{enumerate}
            \end{enumerate}

        \item \ruleref{\ottdruleATForallAppLabel}\\
            By assumption:
            \begin{enumerate}
                \item $[[Γ ⊢ Θ]]$,
                \item $[[Γ; dom(Θ) ⊢  ∀pas.uN]]$ is free from negative algorithmic variables,
                \item $[[Γ; Φ; Θ ⊨ ∀pas.uN ● args ⇒> uM ⫤ Θ'; SC]]$, which by inversion means
                    $[[args ≠ ·]]$, $[[pas ≠ ·]]$, and $[[Γ; Φ; Θ, â⁺*[Γ] ⊨ [â⁺*/pas]uN ● args ⇒> uM ⫤ Θ'; SC]]$.
                    It is easy to see that the induction hypothesis is applicable to the latter judgment:
                    \begin{itemize}
                        \item $[[Γ ⊢ Θ, â⁺*[Γ] ]]$ holds by $[[Γ ⊢ Θ]]$, 
                        \item $[[Γ; dom(Θ), â⁺* ⊢ [â⁺*/pas]uN]]$
                            holds since $[[Γ; dom(Θ) ⊢ ∀pas.uN]]$
                            $[[ [â⁺*/pas]uN ]]$ is normalized and free from negative algorithmic variables
                            since so is $[[uN]]$;
                    \end{itemize}
                    This way, by the inductive hypothesis applied to 
                    $[[Γ; Φ; Θ, â⁺*[Γ] ⊨ [â⁺*/pas]uN ● args ⇒> uM ⫤ Θ'; SC]]$,
                    we have:
                    \begin{enumerate}
                        \item $[[Γ ⊢ Θ']]$,
                        \item $[[Θ, â⁺*[Γ] ⊆ Θ']]$,
                        \item $[[Γ; dom(Θ') ⊢ uM]]$ is normalized and free from negative algorithmic variables,
                        \item $[[dom(Θ, â⁺*[Γ]) ∩ uv(uM) ⊆ uv([â⁺*/pas]uN)]]$,
                        \item $[[Θ'|Ξ ∪ uv(uN) ∪ uv(uM) ⊢ SC]]$, where $[[Ξ]]$ denotes $[[uv([â⁺*/pas]uN) ∩ {â⁺*}]]$,
                            that is the algorithmization of the $[[∀]]$-variables that are actually used in $[[uN]]$.
                        \item  \label{typing-soundness:forall-ih}
                            for any 
                            $[[uσ]]$ such that 
                            $[[Θ' ⊢ uσ : Ξ ∪ uv(uN) ∪ uv(uM)]]$ and $[[ Θ' ⊢ uσ : SC ]]$, 
                            we have $[[ Γ ; Φ ⊢ [uσ][â⁺*/pas]uN ● args ⇒> [uσ]uM ]]$.
                    \end{enumerate}
            \end{enumerate}

            Let us show the required properties:
            \begin{enumerate}
                \item $[[Γ ⊢ Θ']]$ is shown above;
                \item $[[Θ ⊆ Θ']]$ since $[[Θ, â⁺*[Γ] ⊆ Θ']]$;
                \item $[[Γ; dom(Θ') ⊢  uM]]$ is normalized and free from negative algorithmic variables, as shown above;
                \item $[[dom(Θ) ∩ uv(uM) ⊆ uv(uN)]]$
                    since 
                    $[[dom(Θ, â⁺*[Γ]) ∩ uv(uM) ⊆ uv([â⁺*/pas]uN)]]$
                    implies 
                    $[[(dom(Θ) ∪ {â⁺*}) ∩ uv(uM) ⊆ uv(uN) ∪ {â⁺*}]]$,
                    thus,
                    $[[dom(Θ) ∩ uv(uM) ⊆ uv(uN) ∪ {â⁺*}]]$,
                    and since $[[dom(Θ)]]$ is disjoint with $[[{â⁺*}]]$,
                    $[[dom(Θ) ∩ uv(uM) ⊆ uv(uN)]]$;

                \item $[[Θ'|uv(uN) ∪ uv(uM) ⊢ SC | uv(uN) ∪ uv(uM)]]$ follows from 
                    $[[Θ'|Ξ ∪ uv(uN) ∪ uv(uM) ⊢ SC]]$ if we restrict both sides to 
                    $[[uv(uN) ∪ uv(uM)]]$.
                \item Let us assume $[[ Θ' ⊢ uσ : uv(uN) ∪ uv(uM) ]]$ and
                    $[[ Θ' ⊢ uσ : SC | uv(uN) ∪ uv(uM) ]]$.
                    Then to show $[[ Γ ; Φ ⊢ [uσ]∀pas.uN ● args ⇒> [uσ]uM ]]$,
                    that is $[[ Γ ; Φ ⊢ ∀pas.[uσ]uN ● args ⇒> [uσ]uM ]]$,
                    we apply the corresponding declarative rule \ruleref{\ottdruleDTForallAppLabel}.
                    To do so, we need to provide a substitution for $[[pas]]$, 
                    i.e. $[[Γ ⊢ σ0 : {pas}]]$ such that
                    $[[Γ; Φ ⊢ [σ0][uσ]uN ● args ⇒> [uσ]uM ]]$.

                    By \cref{lemma:constraint-sat},
                    we construct $[[uσ0]]$ such that
                    $[[Θ' ⊢ uσ0 : {â⁺*}]]$
                    and 
                    $[[Θ' ⊢ uσ0 : SC|{â⁺*}]]$.

                    Then $[[σ0]]$ is defined as 
                    $[[uσ0]] \circ [[uσ|{â⁺*}]] \circ [[â⁺*/pas]]$. 

                    Let us show that the premises of 
                    \ruleref{\ottdruleDTForallAppLabel} hold:
                    \begin{itemize}
                        \item To show $[[Γ ⊢ σ0 :{pas}]]$,
                            let us take $[[αi⁺ ∊ {pas}]]$.
                            If $[[ αî⁺ ∊ uv(uM)]]$
                            then $[[ [σ0]αi⁺ = [uσ]αî⁺ ]]$,
                            and $[[ Θ' ⊢ uσ : uv(uN) ∪ uv(uM) ]]$
                            implies $[[Θ'(α̂⁺) ⊢ [uσ]α̂⁺]]$.
                            Analogously, if $[[ αî⁺ ∊ {â⁺*} \ uv(uM)]]$
                            then $[[ [σ0]αi⁺ = [uσ0]αî⁺ ]]$,
                            and $[[Θ' ⊢ uσ0 : {â⁺*}]]$ implies $[[Θ'(αî⁺) ⊢ [uσ0]αî⁺]]$.
                            In any case, $[[Θ'(αî⁺) ⊢ [σ]αi⁺ ]]$
                            can be weakened to $[[Γ ⊢ [σ0]αi⁺]]$, 
                            since $[[Γ ⊢ Θ']]$.

                        \item Let us show $[[Γ; Φ ⊢ [σ0][uσ]uN ● args ⇒> [uσ]uM ]]$.
                            It suffices to construct $[[uσ1]]$ such that
                            \begin{enumerate}
                                \item $[[Θ' ⊢ uσ1 : Ξ ∪ uv(uN) ∪ uv(uM)]]$,
                                \item $[[Θ' ⊢ uσ1 : SC]]$,
                                \item $[[ [σ0][uσ]uN = [uσ1][â⁺*/pas]uN]]$, and
                                \item $[[ [uσ]uM = [uσ1]uM]]$,
                            \end{enumerate}
                            because then we can apply the induction hypothesis
                            (\ref{typing-soundness:forall-ih}) to
                            $[[uσ1]]$, rewrite the conclusion by 
                            $[[ [uσ1][â⁺*/pas]uN = [σ0][uσ]uN]]$ and
                            $[[ [uσ1]uM = [uσ]uM]]$, and infer the required judgement.

                            Let us take $[[uσ1]] = [[(uσ0 ○ uσ)|Ξ ∪ uv(uN) ∪ uv(uM)]]$, then 
                            \begin{enumerate}
                                \item $[[Θ' ⊢ uσ1 : Ξ ∪ uv(uN) ∪ uv(uM)]]$,
                                    since $[[Θ' ⊢ uσ0 : {â⁺*}]]$ and 
                                    $[[ Θ' ⊢ uσ : uv(uN) ∪ uv(uM) ]]$, 
                                    we have 
                                    $[[Θ' ⊢ uσ0 ○ uσ : {â⁺*} ∪ uv(uN) ∪ uv(uM)]]$, 
                                    which we restrict to $[[Ξ ∪ uv(uN) ∪ uv(uM)]]$.
                                \item $[[Θ' ⊢ uσ1 : SC]]$,
                                    Let us take any constraint $[[scE ∊ SC]]$ restricting 
                                    variable $[[β̂±]]$.
                                    $[[Θ'|Ξ ∪ uv(uN) ∪ uv(uM) ⊢ SC]]$
                                    implies that $[[β̂± ∊ Ξ ∪ uv(uN) ∪ uv(uM)]]$.

                                    If $[[β̂± ∊ uv(uN) ∪ uv(uM)]]$ then 
                                    $[[ [uσ1]β̂± = [uσ]β̂± ]]$.
                                    Additionally,\\ $[[scE ∊ SC | uv(uN) ∪ uv(uM)]]$,
                                    which, since $[[ Θ' ⊢ uσ : SC | uv(uN) ∪ uv(uM) ]]$,
                                    means $[[Θ'(β̂±) ⊢ [uσ]β̂± : scE]]$.

                                    If $[[β̂± ∊ Ξ \ (uv(uN) ∪ uv(uM))]]$ then
                                    $[[ [uσ1]β̂± = [uσ0]β̂± ]]$.
                                    Additionally, $[[scE ∊ SC | {â⁺*}]]$,
                                    which, since $[[ Θ' ⊢ uσ0 : SC | {â⁺*} ]]$,
                                    means $[[Θ'(β̂±) ⊢ [uσ0]β̂± : scE]]$.

                                \item Let us prove $[[ [σ0][uσ]uN = [uσ1][â⁺*/pas]uN]]$
                                    by the following reasoning
                                    $$ 
                                    \begin{aligned}[t] 
                                        [[ [σ0][uσ]uN  ]] 
                                            &= [[ [uσ0][uσ|{â⁺*}][â⁺*/pas][uσ]uN ]] 
                                                && \text{by definition of $[[σ0]]$}\\
                                            &= [[ [uσ0][uσ|{â⁺*}][â⁺*/pas][uσ|uv(uN)]uN ]]
                                                && \text{by \cref{lemma:subst-restr-uv}}\\
                                            &= [[ [uσ0][uσ|{â⁺*}][uσ|uv(uN)][â⁺*/pas]uN ]]
                                                && \text{$[[uv(uN) ∩ {â⁺*} = ∅]]$ and 
                                                    $[[{pas} ∩ Γ = ∅]]$}\\
                                            &= [[ [uσ|{â⁺*}][uσ|uv(uN)][â⁺*/pas]uN ]] 
                                                && \text{$[[ [uσ|{â⁺*}][uσ|uv(uN)][â⁺*/pas]uN ]]$
                                                is ground}\\
                                            &= [[ [uσ|{â⁺*} ∪ uv(uN)][â⁺*/pas]uN ]] \\
                                            &= [[ [uσ|Ξ ∪ uv(uN)][â⁺*/pas]uN ]] 
                                                && \text{by \cref{lemma:subst-restr-uv}:
                                                    $[[ uv([â⁺*/pas]uN) = Ξ ∪ uv(uN)]]$}\\
                                            &= [[ [uσ|Ξ ∪ uv(uN) ∪ uv(uM)][â⁺*/pas]uN ]] 
                                                && \text{also by \cref{lemma:subst-restr-uv}}\\
                                            &= [[ [(uσ0 ○ uσ)|Ξ ∪ uv(uN) ∪ uv(uM)][â⁺*/pas]uN ]]
                                                && \text{$[[ [uσ|Ξ ∪ uv(uN) ∪ uv(uM)][â⁺*/pas]uN]]$ is ground}\\
                                            &= [[ [uσ1][â⁺*/pas]uN ]]
                                                && \text{by definition of $[[uσ1]]$}\\
                                        \end{aligned} 
                                    $$
                                \item $[[ [uσ]uM = [uσ1]uM]]$
                                        By definition of $[[uσ1]]$,
                                        $[[ [uσ1]uM ]]$ is equal to\\
                                        $[[ [(uσ0 ○ uσ)|Ξ ∪ uv(uN) ∪ uv(uM)]uM ]]$,
                                        which by \cref{lemma:subst-restr-uv} is equal to
                                        $[[ [uσ0 ○ uσ]uM ]]$,
                                        that is $[[ [uσ0][uσ]uM ]]$,
                                        and since $[[ [uσ]uM ]]$ is ground, 
                                        $[[ [uσ0][uσ]uM = [uσ]uM ]]$.
                            \end{enumerate}
                        \item $[[pas ≠ ·]]$ and $[[args ≠ ·]]$ hold by assumption.
                    \end{itemize}
            \end{enumerate}
    \end{caseof}
\end{proof}

\lemmaTypingCompleteness*
\begin{proof}
    We prove it by induction on $\metric{T_1}$, mutually with 
    the soundness of typing (\cref{lemma:typing-soundness}).
    Let us consider the last rule applied to infer the derivation.
    \begin{caseof}

        \item \ruleref{\ottdruleDTThunkLabel}\\
            \label{case:typing-completeness-thunk}
            Then we are proving that if 
            $[[Γ; Φ ⊢ {c} : ↓iN]]$ (inferred by \ruleref{\ottdruleDTThunkLabel})
            then $[[Γ; Φ ⊨ {c} : nf(↓iN)]]$.
            By inversion of $[[Γ; Φ ⊢ {c} : ↓iN]]$, we have
            $[[Γ; Φ ⊢ c : iN]]$, which we apply the induction hypothesis to
            to obtain $[[Γ; Φ ⊨ c : nf(iN)]]$.
            Then by \ruleref{\ottdruleATThunkLabel}, we have $[[Γ; Φ ⊨ {c} : ↓nf(iN)]]$.
            It is left to notice that $[[↓nf(iN) = nf(↓iN)]]$.

        \item \ruleref{\ottdruleDTReturnLabel}\\
            The proof is symmetric to the previous case 
            (\cref{case:typing-completeness-thunk}).

        \item \ruleref{\ottdruleDTPAnnotLabel}\\
            \label{case:typing-completeness-pannot}
            Then we are proving that if
            $[[Γ; Φ ⊢ (v : iQ) : iQ]]$ is inferred by \ruleref{\ottdruleDTPAnnotLabel}
            then $[[Γ; Φ ⊨ (v : iQ) : nf(iQ)]]$.
            By inversion, we have:
            \begin{enumerate}
                \item $[[Γ ⊢ iQ]]$;
                \item $[[Γ; Φ ⊢ v : iP]]$, which
                    by the induction hypothesis implies $[[Γ; Φ ⊨ v : nf(iP)]]$;
                \item $[[Γ ⊢ iQ ≥ iP]]$, and by transitivity, $[[Γ ⊢ iQ ≥ nf(iP)]]$;
                    Since $[[iQ]]$ is ground, 
                    we have $[[Γ ; · ⊢ uQ]]$ and $[[Γ ⊢ [·]uQ ≥ nf(iP)]]$.
                    Then by the completeness of subtyping
                    (\cref{lemma:pos-subt-completeness}), we have 
                    $[[Γ ; · ⊨ uQ ≥ nf(iP) ⫤ SC]]$, where $[[· ⊢ SC]]$ 
                    (implying $[[SC]] = [[·]]$).
                    This way, $[[Γ ; · ⊨ uQ ≥ nf(iP) ⫤ ·]]$.
            \end{enumerate}
            Then we can apply \ruleref{\ottdruleATPAnnotLabel} to
            $[[Γ ⊢ iQ]]$, $[[Γ; Φ ⊨ v : nf(iP)]]$ and $[[Γ ; · ⊨ uQ ≥ nf(iP) ⫤ ·]]$
            to infer $[[Γ; Φ ⊨ (v : iQ) : nf(iQ)]]$.

        \item \ruleref{\ottdruleDTNAnnotLabel}\\
            The proof is symmetric to the previous case 
            (\cref{case:typing-completeness-pannot}).

        \item \ruleref{\ottdruleDTtLamLabel}\\
            Then we are proving that if
            $[[Γ ; Φ ⊢ λx:iP . c : iP → iN]]$ is inferred by \ruleref{\ottdruleDTtLamLabel},
            then $[[Γ ; Φ ⊨ λx:iP . c : nf(iP → iN)]]$.

            By inversion of $[[Γ ; Φ ⊢ λx:iP . c : iP → iN]]$, we have
            $[[Γ ⊢ iP]]$ and $[[Γ; Φ, x:iP ⊢ c : iN]]$.
            Then by the induction hypothesis, $[[Γ; Φ, x:iP ⊨ c : nf(iN)]]$.
            By \ruleref{\ottdruleATtLamLabel}, we infer
            $[[Γ; Φ ⊨ λx:iP . c : nf(iP → nf(iN))]]$. 
            By idempotence of normalization (\cref{lemma:norm-idemp}), 
            $[[nf(iP → nf(iN)) = nf(iP → iN)]]$, 
            which concludes the proof for this case.

        \item \ruleref{\ottdruleDTTLamLabel}\\
            Then we are proving that if
            $[[Γ ; Φ ⊢ Λα⁺ . c : ∀α⁺.iN]]$ is inferred by \ruleref{\ottdruleDTTLamLabel},
            then $[[Γ ; Φ ⊨ Λα⁺ . c : nf(∀α⁺.iN)]]$.
            Similar to the previous case, 
            by inversion of $[[Γ ; Φ ⊢ Λα⁺ . c : ∀α⁺.iN]]$, we have
            $[[Γ, α⁺ ; Φ ⊢ c : iN]]$, and then by the induction hypothesis,
            $[[Γ, α⁺ ; Φ ⊨ c : nf(iN)]]$.
            After that, application of \ruleref{\ottdruleATTLamLabel}, 
            gives as $[[Γ ; Φ ⊨ Λα⁺ . c : nf(∀α⁺.nf(iN))]]$.

            It is left to show that $[[nf(∀α⁺.nf(iN)) = nf(∀α⁺.iN)]]$.
            Assume $[[iN = ∀pbs.iM]]$ (where $[[iM]]$ does not start with $\forall$).
            \begin{itemize}
                \item Then by definition, $[[nf(∀α⁺.iN) = nf(∀α⁺,pbs.iM)]] = 
                    [[∀pcs.nf(iM)]]$, where\\ $[[ord {α⁺,pbs} in nf(iM) = pcs]]$.
                \item On the other hand, $[[nf(iN) = ∀pcs'.nf(iM)]]$, 
                    where $[[ord {pbs} in nf(iM) = pcs']]$, and thus, 
                    $[[nf(∀α⁺.nf(iN)) = nf(∀α⁺,pcs'.nf(iM))]] = [[∀pcs''.nf(nf(iM))]]
                    = [[∀pcs''.nf(iM)]]$,
                    where $[[ord {α⁺,pcs'} in nf(nf(iM)) = pcs'']]$. 
            \end{itemize}
            It is left to show that $[[pcs'' = pcs]]$.
            $$ 
            \begin{aligned}[t] 
                [[ pcs'' ]] &= [[ord {α⁺,pcs'} in nf(nf(iM))]] \\
                            &= [[ord {α⁺,pcs'} in nf(iM)]]
                            && \text{by idempotence (\cref{lemma:norm-idemp})}\\
                            &= [[ord {α⁺} ∪ {pbs} ∩ fv nf(iM) in nf(iM)]]
                            && \text{by definition of $[[pcs']]$ and \cref{lemma:ord-soundness}}\\
                            &= [[ord ({α⁺} ∪ {pbs} ∩ fv nf(iM)) ∩ fv nf(iM) in nf(iM)]]
                            && \text{by \cref{corollary:ord-weakening}}\\
                            &= [[ord ({α⁺} ∪ {pbs}) ∩ fv nf(iM) in nf(iM)]]
                            && \text{by set properties}\\
                            &= [[ord {α⁺,pbs} in nf(iM)]]\\
                            &= [[pcs]]
                \end{aligned} 
            $$

        \item \ruleref{\ottdruleDTUnpackLabel}\\
            Then we are proving that if
            $[[Γ ; Φ ⊢ let∃ (nas, x) = v; c : iN]]$ is 
            inferred by \ruleref{\ottdruleDTUnpackLabel},
            then $[[Γ ; Φ ⊨ let∃ (nas, x) = v; c : nf(iN)]]$.

            By inversion of $[[Γ ; Φ ⊢ let∃ (nas, x) = v; c : iN]]$, we have
            \begin{enumerate}
                \item $[[nf(∃nas.iP) = ∃nas.iP]]$,
                \item $[[Γ ; Φ ⊢ v : ∃nas.iP]]$, 
                    which by the induction hypothesis implies 
                    $[[Γ ; Φ ⊨ v : nf(∃nas.iP)]]$, 
                    and hence, $[[Γ ; Φ ⊨ v : ∃nas.iP]]$.
                \item $[[Γ, nas ; Φ, x:iP ⊢ c : iN]]$,
                    and by the induction hypothesis, 
                    $[[Γ, nas ; Φ, x:iP ⊨ c : nf(iN)]]$.
                \item $[[Γ ⊢ iN]]$.
            \end{enumerate}
            
            This way, we can apply \ruleref{\ottdruleATUnpackLabel} to
            infer $[[Γ ; Φ ⊨ let∃ (nas, x) = v; c : nf(iN)]]$.

        \item \ruleref{\ottdruleDTPEquivLabel}\\
            Then we are proving that
            if $[[Γ; Φ ⊢ v : iP']]$ is inferred by \ruleref{\ottdruleDTPEquivLabel},
            then $[[Γ; Φ ⊨ v : nf(iP')]]$.
            By inversion, $[[Γ ; Φ ⊢ v : iP]]$ and $[[Γ ⊢ iP ≈ iP']]$,
            and the metric of the tree inferring $[[Γ ; Φ ⊢ v : iP]]$ is less than the one 
            inferring $[[Γ; Φ ⊢ v : iP']]$.
            Then by the induction hypothesis, $[[Γ; Φ ⊨ v : nf(iP)]]$.

            By \cref{lemma:subt-equiv-algorithmization}
            $[[Γ ⊢ iP ≈ iP']]$ implies $[[nf(iP) = nf(iP')]]$, and thus, 
            $[[Γ; Φ ⊨ v : nf(iP)]]$ can be rewritten to $[[Γ; Φ ⊨ v : nf(iP')]]$.

        \item \ruleref{\ottdruleDTVarLabel}\\
            Then we are proving that
            $[[Γ; Φ ⊢ x : iP]]$
            implies
            $[[Γ; Φ ⊨ x : nf(iP)]]$.
            By inversion of $[[Γ; Φ ⊢ x : iP]]$,
            we have $[[x : iP ∊ Φ ]]$.
            Then \ruleref{\ottdruleATVarLabel} applies to infer
            $[[Γ; Φ ⊨ x : nf(iP)]]$.

        \item \ruleref{\ottdruleDTVarLetLabel}\\
            Then we are proving that
            $[[Γ; Φ ⊢ let x = v(args); c : iN]]$
            implies
            $[[Γ; Φ ⊨ let x = v(args); c : nf(iN)]]$.

            By inversion of
            $[[Γ; Φ ⊢ let x = v(args); c : iN]]$,
            we have
            \begin{enumerate}
                \item $[[Γ; Φ ⊢ v : iP]]$, 
                    and by the induction hypothesis,
                    $[[Γ; Φ ⊨ v : nf(iP)]]$.
                \item $[[Γ; Φ, x:iP ⊢ c : iN]]$,
                    and by \cref{lemma:decl-typing-context-equiv},
                    since $[[Γ ⊢ iP ≈ nf(iP)]]$, we have
                    $[[Γ; Φ, x:nf(iP) ⊢ c : iN]]$.
                    Then by the induction hypothesis, 
                    $[[Γ; Φ, x:nf(iP) ⊨ c : nf(iN)]]$.
            \end{enumerate}

            Together, $[[Γ; Φ ⊨ v : nf(iP)]]$ and $[[Γ; Φ, x:nf(iP) ⊨ c : nf(iN)]]$
            imply $[[Γ; Φ ⊨ let x = v(args); c : nf(iN)]]$ by \ruleref{\ottdruleATVarLetLabel}.

        \item \ruleref{\ottdruleDTAppLetAnnLabel}\\
            Then we prove that 
            $[[Γ ; Φ ⊢ let x:iP = v(args); c : iN]]$
            implies 
            $[[Γ; Φ ⊨ let x:iP = v(args); c : nf(iN)]]$.

            By inversion of 
            $[[Γ ; Φ ⊢ let x:iP = v(args); c : iN]]$,
            we have
            \begin{enumerate}
                \item $[[Γ ⊢ iP]]$
                \item $[[Γ ; Φ ⊢ v : ↓iM]]$ for some ground $[[iM]]$,
                    which by the induction hypothesis means
                    $[[Γ ; Φ ⊨ v : ↓nf(iM)]]$
                \item $[[Γ ; Φ ⊢ iM ● args ⇒> ↑iQ]]$. 
                    By \cref{lemma:app-inf-equ-stable}, since
                    $[[Γ ⊢ iM ≈ nf(iM)]]$, we have
                    $[[Γ ; Φ ⊢ [·]nf(uM) ● args ⇒> ↑iQ]]$, 
                    which by the induction hypothesis means 
                    that there exist normalized 
                    $[[uM']]$, $[[Θ]]$, and $[[SC1]]$ such that
                    (noting that $[[iM]]$ is ground):
                    \begin{enumerate}
                        \item $[[ Γ; Φ; · ⊨ nf(uM) ● args ⇒> uM' ⫤ Θ; SC1 ]]$,
                            where by the soundness, $[[Γ; dom(Θ) ⊢ uM']]$ and $[[Θ ⊢ SC1]]$.
                        \item for any $[[Γ ⊢ iM'']]$ 
                            such that $[[Γ; Φ ⊢ nf(iM) ● args ⇒> iM'']]$
                            there exists $[[uσ]]$ such that 
                            \begin{enumerate}
                                \item $[[ Θ ⊢ uσ : uv uM' ]]$, $[[ Θ ⊢ uσ : SC1 ]]$, and 
                                \item $[[Γ ⊢ [uσ]uM' ≈ iM'']]$,
                            \end{enumerate}
                            In particular, there exists
                            $[[uσ0]]$
                            such that 
                            $[[ Θ ⊢ uσ0 : uv uM']]$,
                            $[[ Θ ⊢ uσ0 : SC1]]$,
                            $[[Γ ⊢ [uσ0]uM' ≈ ↑iQ]]$.
                            Since $[[uM']]$ is normalized and free of negative algorithmic variables,
                            the latter equivalence means
                            $[[uM' = ↑uQ0]]$ for some $[[uQ0]]$, and $[[Γ ⊢ [uσ0]uQ0 ≈ iQ]]$.
                    \end{enumerate}
                \item $[[Γ ⊢ ↑iQ ≤ ↑iP]]$,
                    and by transitivity, since $[[Γ ⊢ [uσ0]↑uQ0 ≈ ↑iQ]]$,
                    we have $[[Γ ⊢ [uσ0]↑uQ0 ≤ ↑iP]]$.
                    
                    Let us apply \cref{lemma:neg-subt-completeness} to 
                    $[[Γ ⊢ [uσ0]↑uQ0  ≤ ↑iP]]$ and obtain
                    $[[Θ ⊢ SC2]]$ such that 
                    \begin{enumerate}
                        \item $[[Γ ; Θ ⊨ ↑uQ0 ≤ ↑iP ⫤ SC2]]$ and
                        \item $[[ Θ   ⊢ uσ0 : SC2 ]]$.
                    \end{enumerate}
                \item $[[Γ; Φ, x:iP ⊢ c : iN]]$, 
                    and by the induction hypothesis,
                    $[[Γ; Φ, x:iP ⊨ c : nf(iN)]]$.
            \end{enumerate}

            To infer $[[Γ; Φ ⊨ let x:iP = v(args); c : nf(iN)]]$,
            we apply the corresponding algorithmic rule 
            \ruleref{\ottdruleATAppLetAnnLabel}.
            Let us show that the premises hold:
            \begin{enumerate}
                \item $[[Γ ⊢ iP]]$,
                \item $[[Γ; Φ ⊨ v : ↓nf(iM)]]$,
                \item $[[Γ; Φ; · ⊨ nf(uM) ● args ⇒> ↑uQ0 ⫤ Θ; SC1]]$, 
                \item $[[Γ; Θ ⊨ ↑uQ0 ≤ ↑iP ⫤ SC2]]$, and
                \item $[[Γ; Φ, x:iP ⊨ c : nf(iN)]]$ hold as noted above;
                \item $[[Θ ⊢ SC1 & SC2 = SC]]$ 
                    is defined by \cref{lemma:merge-completeness},
                    since $[[ Θ   ⊢ uσ0 : SC1 ]]$ and $[[ Θ   ⊢ uσ0 : SC2 ]]$.
            \end{enumerate}

        \item \ruleref{\ottdruleDTAppLetLabel}\\
            By assumption, $[[c]]$ is $[[let x = v(args); c']]$. 
            Then by inversion of
            $[[Γ ; Φ ⊢ let x = v(args); c' : iN]]$: 
            \begin{itemize}
                \item $[[Γ ; Φ ⊢ v : ↓iM]]$, 
                    which by the induction hypothesis means 
                    $[[Γ; Φ ⊨ v : ↓nf(iM)]]$;
                \item $[[Γ ; Φ ⊢ iM ● args ⇒> ↑iQ principal]]$. 
                    Then by \cref{lemma:app-inf-equ-stable}, since 
                    $[[Γ ⊢ iM ≈ nf(iM)]]$, we have
                    $[[Γ ; Φ ⊢ nf(iM) ● args ⇒> ↑iQ]]$
                    and moreover, $[[Γ ; Φ ⊢ nf(iM) ● args ⇒> ↑iQ principal]]$:
                    since for any inference, $[[nf(iM)]]$ can be replaced back with $[[iM]]$,
                    the sets of types $[[iQ']]$ inferred for the applications
                    $[[Γ ; Φ ⊢ nf(iM) ● args ⇒> ↑iQ']]$ and 
                    $[[Γ ; Φ ⊢ iM ● args ⇒> ↑iQ']]$ are the same.
                    Then the induction hypothesis applied to 
                    $[[Γ ; Φ ⊢ [·]nf(uM) ● args ⇒> ↑iQ]]$
                    implies that there exist $[[uM']]$, $[[Θ]]$, and $[[SC]]$ such that
                    (considering $[[iM]]$ is ground):
                    \begin{enumerate}
                        \item $[[ Γ; Φ; · ⊨ nf(uM) ● args ⇒> uM' ⫤ Θ; SC ]]$, 
                            which, by the soundness, implies, in particular
                            that 
                            \begin{enumerate}
                                \item $[[Γ; dom(Θ) ⊢  uM']]$ is normalized and 
                                    free of negative algorithmic variables, 
                                \item $[[Θ|uv(uM') ⊢ SC]]$, which means $[[dom(SC) ⊆ uv(uM')]]$,
                                \item \label{point:typing-completeness:AppLet:ih-sound} 
                                    for any $[[Θ ⊢ uσ : uv uM' ]]$ such that $[[ Θ ⊢ uσ : SC ]]$, 
                                    we have $[[ Γ ; Φ ⊢ nf(iM) ● args ⇒> [uσ]uM' ]]$.
                            \end{enumerate}
                            and
                        \item for any $[[Γ ⊢ iM'']]$
                            such that $[[Γ; Φ ⊢ nf(iM) ● args ⇒> iM'']]$,
                            (and in particular, for $[[Γ ⊢ ↑iQ]]$)
                            there exists $[[uσ1]]$ such that 
                            \begin{enumerate}
                                \item 
                                    $[[ Θ  ⊢ uσ1 : uv uM' ]]$,
                                    $[[ Θ  ⊢ uσ1 : SC ]]$, and 
                                \item $[[Γ ⊢ [uσ1]uM' ≈ iM'']]$, and 
                                    in particular, $[[Γ ⊢ [uσ1]uM' ≈ ↑iQ]]$.
                                    Since $[[uM']]$ is
                                    normalized and free of 
                                    negative algorithmic variables, it means that 
                                    $[[uM' = ↑uP]]$ for some $[[uP]]$ 
                                    ($[[Γ; dom(Θ) ⊢  uP]]$)
                                    that is $[[Γ ⊢ [uσ1]uP ≈ iQ]]$.
                            \end{enumerate}
                    \end{enumerate}
                \item $[[Γ; Φ, x:iQ ⊢ c' : iN]]$
            \end{itemize}

            To infer $[[Γ ; Φ ⊨ let x = v(args); c' : nf(iN)]]$, 
            let us apply the corresponding algorithmic rule 
            (\ruleref{\ottdruleATAppLetLabel}):
            \begin{enumerate}
                \item $[[Γ ; Φ ⊨ v : ↓nf(iM)]]$ holds as noted above;

                \item $[[Γ; Φ ; · ⊨ nf(uM) ● args ⇒> ↑uP ⫤ Θ; SC]]$ holds as noted above;

                \item Let us show that $nf(iQ)$ is the minimal instantiation 
                    of $[[uP]]$ \wrt $[[SC]]$, in other words, 
                    $[[Γ ⊢ uP SC minby uσ]]$ for some $[[uσ]]$ and 
                    $[[ [uσ]uP = nf(iQ) ]]$.
                    By rewriting $[[nf(iQ)]]$ as $[[ nf([uσ1]uP) ]]$,
                    we need to show $[[ [uσ]uP = nf([uσ1]uP) ]]$. 

                    Let us apply the completeness of minimal instantiation
                    (\cref{lemma:min-inst-completeness}). That would give us
                    $[[uσ = nf(uσ1)]]$, which would immediately imply the required
                    equality.  To do that, we need to
                    demonstrate that $[[uσ1]]$ is the minimal instantiation of
                    $[[uP]]$ \wrt $[[SC]]$.
                    In other words, any other
                    substitution respecting $[[SC]]$,
                    instantiate $[[uP]]$ into a \emph{supertype} of $[[iQ]]$.
                    To do that, we apply the principality of $[[iQ]]$:
                    $[[Γ ; Φ ⊢ nf(iM) ● args ⇒> ↑iQ principal]]$:
                    which means that for any 
                    other $[[iQ']]$ such that $[[Γ ; Φ ⊢ nf(iM) ● args ⇒> ↑iQ']]$,
                    we have $[[Γ ⊢ iQ' ≥ iQ]]$.
                    It is left to show that any substitution respecting $[[SC]]$
                    gives us $[[iQ']]$ inferrable for the application $[[Γ ; Φ ⊢ iM ● args ⇒> ↑iQ']]$,
                    which holds by \ref{point:typing-completeness:AppLet:ih-sound}.

                \item To show $[[uv uP = dom(SC)]]$ and 
                    $[[SC singular with uσ0]]$ for some $[[uσ0]]$,
                    we apply \cref{lemma:singularity-completeness}
                    with $[[Ξ = uv uP]] = [[uv(uM')]]$ (as noted above, $[[dom(SC) ⊆ uv(uM')]] = [[Ξ]]$).

                    Now we will show that any substitution satisfying $[[SC]]$ is equivalent to $[[uσ1]]$.
                    As noted in \ref{point:typing-completeness:AppLet:ih-sound},
                    for any substitution $[[Θ ⊢ uσ : Ξ]]$, $[[ Θ ⊢ uσ : SC ]]$ implies 
                    $[[Γ ⊢ [uσ]uM' ≈ ↑iQ]]$,
                    which is rewritten as $[[Γ ⊢ [uσ]uP ≈ iQ]]$.
                   And since $[[Γ ⊢ [uσ1]uP ≈ iQ]]$, 
                    we have $[[Γ ⊢ [uσ]uP ≈ [uσ1]uP]]$,
                    which implies $[[Θ ⊢ uσ ≈ uσ1 : Ξ]]$ by \cref{lemma:subst-equiv-algovar}.

                \item Let us show $[[Γ; Φ, x:[uσ0]uP ⊨ c' : nf(iN)]]$.
                    By the soundness of singularity 
                    (\cref{lemma:singularity-soundness}),
                    we have $[[ Θ ⊢ uσ0 : SC ]]$,
                    which by \ref{point:typing-completeness:AppLet:ih-sound}
                    means $[[Γ ⊢ [uσ0]uM' ≈ ↑iQ]]$,
                    that is $[[Γ ⊢ [uσ0]uP ≈ iQ]]$, 
                    and thus, $[[Γ ⊢ Φ, x:iQ ≈ Φ, x:[uσ0]uP]]$.

                    Then by \cref{lemma:decl-typing-context-equiv},
                    $[[Γ; Φ, x:iQ ⊢ c' : iN]]$ can be rewritten as
                    $[[Γ; Φ, x:[uσ0]uP ⊢ c' : iN]]$.
                    Then by the induction hypothesis applied to it, 
                    $[[Γ; Φ, x:[uσ0]uP ⊨ c' : nf(iN)]]$ holds.
            \end{enumerate}

        \item \ruleref{\ottdruleDTForallAppLabel}\\
            Since $[[uN]]$ cannot be a algorithmic variable,  
            if $[[ [uσ]uN ]]$ starts with $[[∀]]$,
            so does $[[uN]]$. This way,
            $[[uN = ∀pas.uN1]]$.
            Then by assumption:
            \begin{enumerate}
                \item $[[Γ ⊢ Θ]]$
                \item $[[Γ; dom(Θ) ⊢  ∀pas.uN1]]$ is free from negative algorithmic variables, 
                    and then $[[Γ, pas; dom(Θ) ⊢  uN1]]$ is free from negative algorithmic variables too;
                \item $[[Θ ⊢ uσ : uv uN1]]$;
                \item $[[Γ ⊢ iM]]$;
                \item $[[Γ; Φ ⊢ [uσ]∀pas.uN1 ● args ⇒> iM]]$, 
                    \label{point:typing-completeness-forall-app-inversion}
                    that is $[[Γ; Φ ⊢ (∀pas.[uσ]uN1) ● args ⇒> iM]]$.
                    Then by inversion there exists $[[σ]]$ such that 
                    \begin{enumerate}
                        \item $[[Γ ⊢ σ : {pas}]]$;
                        \item $[[args ≠ ·]]$ and $[[pas ≠ ·]]$; and
                        \item $[[Γ ; Φ ⊢ [σ][uσ]uN1 ● args ⇒> iM]]$.
                            \label{point:typing-completeness-forall-app-inversion-2}
                            Notice that $[[σ]]$ and $[[uσ]]$ commute because 
                            the codomain of $[[σ]]$ does not contain
                            algorithmic variables (and thus, does not intersect with 
                            the domain of $[[uσ]]$), and the codomain of $[[uσ]]$ is 
                            $[[Γ]]$ and does not intersect with $[[pas]]$---the domain of $[[σ]]$.

                            Let us take fresh $[[â⁺*]]$ and 
                            construct $[[uN0]]$ = $[[ [â⁺*/pas]uN1 ]]$
                            and $[[Θ, â⁺*[Γ] ⊢ uσ0 : uv(uN0)]]$ defined as
                            $$
                            \begin{cases}
                                [[ [uσ0]αî⁺ = [σ]αi⁺ ]] & \text{for $[[αî⁺]] \in  [[{â⁺*} ∩ uv uN0]]$ }\\
                                [[ [uσ0]β̂± = [uσ]β̂± ]] & \text{for $[[β̂±]] \in [[uv uN1]]$}
                            \end{cases}
                            $$

                            Then it is easy to see that $[[ [uσ0][â⁺*/pas]uN1 = [σ][uσ]uN1 ]]$
                            because this substitution compositions coincide on
                            $[[uv(uN1)]] \cup [[fv(uN1)]]$. 
                            In other words, $[[ [uσ0]uN0 = [σ][uσ]uN1 ]]$.

                            Then let us apply the induction hypothesis
                            to $[[Γ; Φ ⊢ [uσ0]uN0 ● args ⇒> iM]]$ and obtain 
                            $[[uM']]$, $[[Θ']]$, and $[[SC]]$ such that
                            \begin{itemize}
                                \item $[[ Γ; Φ; Θ, â⁺*[Γ] ⊨ uN0 ● args ⇒> uM' ⫤ Θ'; SC ]]$ and
                                \item \label{point:typing-completeness-forall-app-inversion-3}
                                for any $[[Θ, â⁺*[Γ]  ⊢ uσ0 : uv(uN0)]]$ and $[[Γ ⊢ iM]]$
                                    such that $[[Γ; Φ ⊢ [uσ0]uN0 ● args ⇒> iM]]$, 
                                    there exists $[[uσ0']]$ such that 
                                \begin{enumerate}
                                    \item $[[Θ'  ⊢uσ0' : uv(uN0) ∪ uv(uM')]]$, $[[ Θ' ⊢ uσ0'  : SC ]] $,
                                    \item $[[Θ, â⁺*[Γ] ⊢ uσ0' ≈ uσ0 : uv uN0]]$, and 
                                    \item $[[Γ ⊢ [uσ0']uM' ≈ iM]]$.
                                \end{enumerate}
                            \end{itemize}
                    \end{enumerate}
            \end{enumerate}
            Let us take $[[uM']]$, $[[Θ']]$, and $[[SC]]$ from the induction hypothesis
            (\ref{point:typing-completeness-forall-app-inversion-2}) 
            (from $[[SC]]$ we subtract entries restricting $[[{â⁺*}]]$)
            and show they satisfy the required properties 
            \begin{enumerate}
                \item To infer $[[ Γ; Φ; Θ ⊨ ∀pas.uN1 ● args ⇒> uM' ⫤ Θ'; SC \ {â⁺*} ]]$
                    we apply the corresponding algorithmic rule \ruleref{\ottdruleATForallAppLabel}.
                    As noted above, the required premises hold:
                    \begin{enumerate}
                        \item $[[args ≠ ·]]$, $[[pas ≠ ·]]$; and
                        \item $[[Γ; Φ; Θ, â⁺*[Γ] ⊨ [â⁺*/pas]uN1 ● args ⇒> uM' ⫤ Θ'; SC]]$
                            is obtained by unfolding the definition of $[[uN0]]$
                            in $[[ Γ; Φ; Θ, â⁺*[Γ] ⊨ uN0 ● args ⇒> uM' ⫤ Θ'; SC ]]$
                            (\ref{point:typing-completeness-forall-app-inversion-2}).
                    \end{enumerate}
                \item Let us take and arbitrary $[[Θ ⊢ uσ : uv uN1]]$ and $[[Γ ⊢ iM]]$
                    and assume $[[Γ; Φ ⊢ [uσ]∀pas.uN1  ● args ⇒> iM]]$. 
                    Then the same reasoning as in 
                    \ref{point:typing-completeness-forall-app-inversion-2}
                    applies. In particular, we construct 
                    $[[Θ, â⁺*[Γ] ⊢ uσ0 : uv(uN0)]]$ 
                        as an extension of $[[uσ]]$ and obtain 
                    $[[Γ; Φ ⊢ [uσ0]uN0 ● args ⇒> iM]]$.

                    It means we can apply the property inferred from the induction 
                    hypothesis (\ref{point:typing-completeness-forall-app-inversion-3})
                    to obtain $[[uσ0']]$ such that 
                    \begin{enumerate}
                        \item $[[Θ' ⊢ uσ0' : uv(uN0) ∪ uv(uM')]]$ and $[[ Θ'  ⊢ uσ0' : SC ]] $,
                        \item $[[Θ, â⁺*[Γ] ⊢ uσ0' ≈ uσ0 : uv uN0]]$, and 
                        \item $[[Γ ⊢ [uσ0']uM' ≈ iM]]$.
                    \end{enumerate}

                    Let us show that $[[uσ0'|(uv(uN1) ∪ uv(uM'))]]$ 
                    satisfies the required properties.
                    \begin{enumerate}
                        \item $[[Θ' ⊢ uσ0'|(uv(uN1) ∪ uv(uM')): (uv(uN1) ∪ uv(uM'))]]$
                            holds since $[[ Θ' ⊢ uσ0' : uv(uN0) ∪ uv(uM')]]$
                            and $[[uv(uN1) ∪ uv(uM') ⊆ uv(uN0) ∪ uv(uM')]]$;
                            $[[ Θ' ⊢ uσ0'|(uv(uN1) ∪ uv(uM')) : SC \ {â⁺*} ]]$ holds since
                            $[[ Θ' ⊢ uσ0' : SC ]]$,
                            $[[ Θ' ⊢ uσ0' : uv(uN0) ∪ uv(uM') ]]$,
                            and $[[(uv(uN0) ∪ uv(uM')) \ {â⁺*} = uv(uN1) ∪ uv(uM')]]$.

                        \item $[[Γ ⊢ [uσ0']uM' ≈ iM]]$ holds as shown,
                            and hence it holds for $[[uσ0'|(uv(uN1) ∪ uv(uM'))]]$;
                        \item We show $[[Θ ⊢ uσ0' ≈ uσ : uv uN1]]$, from which
                            it follows that it holds for\\ $[[uσ0'|(uv(uN1) ∪ uv(uM'))]]$.
                            Let us take an arbitrary 
                            $[[β̂±]] \in [[dom(Θ)]] \subseteq [[dom(Θ) ∪ {â⁺*}]]$. Then 
                            since $[[Θ, â⁺*[Γ] ⊢ uσ0' ≈ uσ0 : uv uN0]]$, 
                            we have $[[Θ(β̂±) ⊢ [uσ0']β̂±  ≈ [uσ0]β̂± ]]$ and 
                            by definition of $[[uσ0]]$, $[[ [uσ0]β̂±  = [uσ]β̂± ]]$.
                    \end{enumerate}
            \end{enumerate}
           
        \item \ruleref{\ottdruleDTArrowAppLabel}\\
            Since $[[uN]]$ cannot be a algorithmic variable,  
            if the shape of $[[ [uσ]uN ]]$ is an arrow, 
            so is the shape of $[[uN]]$. This way, 
            $[[uN = uQ → uN1]]$.
            Then by assumption:
            \begin{enumerate}
                \item $[[Γ ⊢ Θ]]$;
                \item $[[Γ; dom(Θ) ⊢  uQ → uN1]]$ is free from negative algorithmic variables;
                \item $[[Θ ⊢ uσ : uv uQ ∪ uv uN1]]$;
                \item $[[Γ ⊢ iM]]$;
                \item $[[Γ; Φ ⊢ [uσ](uQ → uN1) ● v, args ⇒> iM]]$, 
                    \label{point:typing-completeness-arrow-app-inversion}
                    that is $[[Γ; Φ ⊢ ([uσ]uQ → [uσ]uN1) ● v, args ⇒> iM]]$,
                    and by inversion:
                    \begin{enumerate}
                        \item $[[Γ; Φ ⊢ v : iP]]$,
                            and by the induction hypothesis, 
                            $[[Γ; Φ ⊨ v : nf(iP)]]$;
                        \item $[[Γ ⊢ [uσ]uQ ≥ iP]]$, 
                            which by transitivity (\cref{lemma:subtyping-transitivity}) means 
                            $[[Γ ⊢ [uσ]uQ ≥ nf(iP)]]$,
                            and then by completeness of subtyping 
                            (\cref{lemma:pos-subt-completeness}),
                            $[[ Γ; Θ ⊨ uQ ≥ nf(iP) ⫤ SC1 ]]$, 
                            for some $[[Θ ⊢ SC1 : uv(uQ)]]$, and moreover, $[[ Θ ⊢ uσ : SC1 ]]$;
                        \item $[[Γ; Φ ⊢ [uσ]uN1 ● args ⇒> iM]]$. 
                            \label{point:completeness-arrow-app-ih}
                            Notice that the induction hypothesis applies to this case:
                            $[[Γ ; dom(Θ) ⊢  uN1]]$ is free from negative algorithmic variables because
                            so is $[[uQ → uN1]]$. This way, there exist 
                            $[[uM']]$, $[[Θ']]$, and $[[SC2]]$ such that 
                            \begin{enumerate}
                                \item $[[ Γ; Φ; Θ ⊨ uN1 ● args ⇒> uM' ⫤ Θ'; SC2 ]]$
                                    and then by the soundness of typing 
                                    (i.e. the induction hypothesis), 
                                    \begin{enumerate}
                                        \item $[[Θ ⊆ Θ']]$
                                        \item $[[Γ; dom(Θ') ⊢  uM']]$
                                        \item $[[dom(Θ) ∩ uv(uM') ⊆ uv uN1]]$
                                        \item $[[Θ'|uv uN1 ∪ uv uM' ⊢ SC2]]$
                                    \end{enumerate}
                                \item  \label{point:new-subdst}
                                    for any $[[Θ ⊢ uσ : uv(uN1)]]$ and $[[Γ ⊢ iM]]$
                                    such that $[[Γ; Φ ⊢ [uσ]uN1 ● args ⇒> iM]]$, 
                                    there exists $[[uσ']]$ such that 
                                    \begin{enumerate}
                                        \item $[[Θ' ⊢ uσ' : uv(uN1) ∪ uv(uM')]]$ and $[[Θ' ⊢ uσ' : SC2]]$,
                                        \item $[[Θ ⊢ uσ' ≈ uσ : uv(uN1)]]$, and 
                                        \item $[[Γ ⊢ [uσ']uM' ≈ iM]]$.
                                    \end{enumerate}
                            \end{enumerate}
                    \end{enumerate}
            \end{enumerate}

            We need to show that there exist $[[uM']]$, $[[Θ']]$, and $[[SC]]$ such that
            $[[ Γ; Φ; Θ ⊨ uQ → uN1 ● v, args ⇒> uM' ⫤ Θ'; SC ]]$ and
            the initiality property holds. 
            We take $[[uM']]$ and $[[Θ']]$ from the induction hypothesis
            (\ref{point:completeness-arrow-app-ih}), and $[[SC]]$
            as a merge of $[[SC1]]$ and $[[SC2]]$.
            To show that $[[Θ' ⊢ SC1 & SC2 = SC]]$ exists,
            we apply \cref{lemma:merge-completeness}.
            To do so, we need to provide 
            a substitution satisfying both 
            $[[SC1]]$ and $[[SC2]]$.

            Notice that $[[dom(SC1) = uv(uQ)]]$ and
            $[[dom(SC2) ⊆ uv uN1 ∪ uv uM']]$.
            This way, it suffices to construct 
            $[[Θ' ⊢ uσ'' : uv(uQ) ∪ uv uN1 ∪ uv uM']]$ such that
            $[[Θ' ⊢ uσ'' : SC1]]$ and $[[Θ' ⊢ uσ'' : SC2]]$.

            By the induction hypothesis (\ref{point:new-subdst}),
            $[[uσ|uv(uN1)]]$
            can be extended to $[[uσ']]$ such that
            \begin{enumerate}
                \item $[[Θ' ⊢ uσ' : uv(uN1) ∪ uv(uM')]]$ and $[[Θ' ⊢ uσ' : SC2]]$,
                \item $[[Θ ⊢ uσ' ≈ uσ: uv(uN1)]]$, and 
                \item $[[Γ ⊢ [uσ']uM' ≈ iM]]$.
            \end{enumerate}
            Let us extend $[[uσ']]$ to $[[uσ'']]$
            defined on $[[uv(uQ) ∪ uv(uN1) ∪ uv(uM')]]$
            with values of $[[uσ]]$ as follows:
            $$
            \begin{cases}
                [[ [uσ'']β̂± = [uσ']β̂± ]] & \text{for $[[β̂±]] \in [[uv(uN1) ∪ uv(uM')]]$}\\
                [[ [uσ'']γ̂± = [uσ]γ̂± ]] & \text{for $[[γ̂±]] \in [[uv(uQ) \ (uv(uN1) ∪ uv(uM'))]]$}
            \end{cases}
            $$

            First, notice that $[[Θ' ⊢ uσ'' ≈ uσ' : uv(uN1) ∪ uv(uM')]]$
            by definition.
            Then since $[[Θ' ⊢ uσ' : SC2]]$ and
            $[[Θ' ⊢ SC2 : uv(uN1) ∪ uv(uM')]]$, 
            we have $[[Θ' ⊢ uσ'' : SC2]]$.

            Second, notice that $[[Θ ⊢ uσ'' ≈ uσ : uv(uN1) ∪ uv(uQ)]]$:
            \begin{itemize}
                \item if $[[γ̂±]] \in [[uv(uQ) \ (uv(uN1) ∪ uv(uM'))]]$
                    then $[[ [uσ'']γ̂± = [uσ]γ̂± ]]$ by definition of $[[uσ'']]$;
                \item if $[[γ̂±]] \in [[uv(uQ) ∩ uv(uN1)]]$
                    then $[[ [uσ'']γ̂± = [uσ']γ̂± ]]$,
                    and $[[Θ ⊢ uσ' ≈ uσ: uv(uN1)]]$, as noted above;
                \item if $[[γ̂±]] \in [[uv(uQ) ∩ uv(uM')]]$
                    then since $[[Γ; dom(Θ) ⊢  uQ]]$, 
                    we have $[[uv(uQ) ⊆ dom(Θ)]]$,
                    implying 
                    $[[γ̂±]] \in [[dom(Θ) ∩ uv(uM') ⊆ uv(uN1)]]$.
                    This way, $[[γ̂±]] \in [[uv(uQ) ∩ uv(uN1)]]$, 
                    and this case is covered by the previous one.
            \end{itemize}
            In particular, $[[Θ ⊢ uσ'' ≈ uσ : uv(uQ)]]$.
            Then since $[[Θ ⊢ uσ : SC1]]$ and $[[Θ ⊢ SC1 : uv(uQ)]]$,
            we have $[[Θ ⊢ uσ'' : SC1]]$.

            This way, $[[uσ']]$ satisfies both $[[SC1]]$ and $[[SC2]]$,
            and by the completeness of constraint merge 
            (\cref{lemma:merge-completeness}),
            $[[Θ' ⊢ SC1 & SC2 = SC]]$ exists.


            Finally, to show the required properties, we take
            $[[uM']]$ and $[[Θ']]$ from the induction hypothesis (\ref{point:new-subdst}), 
            and $[[SC]]$ defined above. Then
            \begin{enumerate}
                \item $[[ Γ; Φ; Θ ⊨ uQ → uN1 ● v,args ⇒> uM' ⫤ Θ'; SC ]]$
                    is inferred by \ruleref{\ottdruleATArrowAppLabel}.
                    As noted above:
                    \begin{enumerate}
                        \item $[[Γ; Φ ⊨ v : nf(iP)]]$,
                        \item $[[Γ; Θ ⊨ uQ ≥ nf(iP) ⫤ SC1]]$,
                        \item $[[Γ; Φ; Θ ⊨ uN1 ● args ⇒> uM' ⫤ Θ'; SC2]]$, and
                        \item $[[Θ' ⊢ SC1 & SC2 = SC]]$.
                    \end{enumerate}
                \item let us take an arbitrary 
                    $[[Θ ⊢ uσ0 : uv uQ ∪ uv uN1]]$;
                    and $[[Γ ⊢ iM0]]$;
                    such that $[[Γ; Φ ⊢ [uσ0](uQ → uN1) ● v,args ⇒> iM0]]$.
                    Then by inversion of 
                    $[[Γ; Φ ⊢ [uσ0]uQ → [uσ0]uN1 ● v, args ⇒> iM0]]$,
                    we have the same properties as in 
                    \ref{point:typing-completeness-arrow-app-inversion}.
                    In particular,
                    \begin{itemize}
                        \item $[[Γ ⊢ [uσ0]uQ ≥ nf(iP)]]$
                            and by the completeness of subtyping 
                            (\cref{lemma:pos-subt-completeness}),
                            $[[ Θ  ⊢ uσ0 : SC1  ]]$.
                        \item $[[Γ; Φ ⊢ [uσ0]uN1 ● args ⇒> iM0]]$. 
                            Then by \ref{point:new-subdst}, 
                            there exists $[[uσ0']]$ such that 
                            \begin{enumerate}
                                    \item $[[Θ' ⊢ uσ0' : uv(uN1) ∪ uv(uM')]]$ and 
                                        $[[Θ' ⊢ uσ0' : SC2]]$,
                                    \item $[[Θ ⊢ uσ0' ≈ uσ0 : uv(uN1)]]$, and 
                                    \item $[[Γ ⊢ [uσ0']uM' ≈ iM0]]$.
                            \end{enumerate}
                    \end{itemize}
                    Let us extend $[[uσ0']]$ to be defined on
                    $[[uv(uQ) ∪ uv(uN1) ∪ uv(uM')]]$
                    with the values of $[[uσ0]]$.
                    We define $[[uσ0'']]$ as follows:
                    $$
                    \begin{cases}
                        [[ [uσ0'']γ̂± = [uσ0']γ̂± ]] & \text{for $[[γ̂±]] \in [[uv(uN1) ∪ uv(uM')]]$}\\
                        [[ [uσ0'']γ̂± = [uσ0]γ̂± ]] & \text{for $[[γ̂±]] \in [[uv(uQ) \ (uv(uN1) ∪ uv(uM'))]]$}
                    \end{cases}
                    $$
                    This way, 
                    \begin{itemize}
                        \item $[[Θ' ⊢ uσ0'' : uv(uQ) ∪ uv(uN1) ∪ uv(uM')]]$,
                        \item $[[Θ' ⊢ uσ0'' : SC]]$,
                            since $[[Θ' ⊢ uσ0'' : SC1]]$ and $[[Θ' ⊢ uσ0'' : SC2]]$,
                            which is proved similarly to
                            $[[Θ' ⊢ uσ'' : SC1]]$ and $[[Θ' ⊢ uσ'' : SC2]]$ above;
                        \item $[[Θ ⊢ uσ0'' ≈ uσ0 : uv(uN1) ∪ uv(uQ)]]$:
                            the proof is analogous to
                            $[[Θ ⊢ uσ'' ≈ uσ : uv(uN1) ∪ uv(uQ)]]$ above.
                        \item $[[Γ ⊢ [uσ0'']uM' ≈ iM0]]$
                            Notice that $[[Θ' ⊢ uσ0'' ≈ uσ0' : uv(uN1) ∪ uv(uM')]]$,
                            which is proved analogously to
                            $[[Θ' ⊢ uσ'' ≈ uσ' : uv(uN1) ∪ uv(uM')]]$ above.
                            Then $[[Γ ⊢ [uσ0']uM' ≈ iM0]]$
                            can be rewritten to $[[Γ ⊢ [uσ0'']uM' ≈ iM0]]$.
                    \end{itemize}
            \end{enumerate}
            


        \item \ruleref{\ottdruleDTEmptyAppLabel}\\
            By assumption: 
            \begin{enumerate}
                \item $[[Γ ⊢ Θ]]$,
                \item $[[Γ ⊢ iN']]$,
                \item $[[Γ; dom(Θ) ⊢ uN]]$ and $[[uN]]$ is free from negative variables,
                \item $[[Θ ⊢ uσ : uv(uN)]]$,
                \item $[[Γ; Φ ⊢ [uσ]uN ● · ⇒> iN' ]]$,
                    and by inversion, $[[Γ ⊢ [uσ]uN ≈ iN']]$.
            \end{enumerate}


            Then we can apply the corresponding algorithmic rule
            \ruleref{\ottdruleATEmptyAppLabel} to infer
            $[[ Γ; Φ; Θ ⊨ uN ● · ⇒> nf(uN) ⫤ Θ; · ]]$.
            Let us show the required properties. 
            Let us take an arbitrary 
            $[[Θ ⊢ uσ0 : uv(uN)]]$ and $[[Γ ⊢ iM]]$
            such that $[[Γ; Φ ⊢ [uσ1]uN ● · ⇒> iM]]$. 
            Then we can take $[[uσ0]]$ as the required substitution:
            \begin{enumerate}
                \item $[[ Θ ⊢ uσ0 : uv(uN) ∪ uv(nf(uN)) ]]$,
                    since $[[uv(nf(uN)) = uv(uN)]]$, 
                    and thus, $[[uv(uN) ∪ uv(nf(uN)) = uv(uN)]]$;
                \item $[[ Θ ⊢ uσ0 : · ]]$ vacuously;
                \item $[[Θ ⊢ uσ0 ≈ uσ0 : uv(uN)]]$ by reflexivity;
                \item Let us show $[[Γ ⊢ [uσ0]nf(uN) ≈ iM]]$.
                    Notice that $[[Γ; Φ ⊢ [uσ0]uN ● · ⇒> iM]]$ can only be inferred by 
                    \ruleref{\ottdruleDTEmptyAppLabel}, and thus, $[[ Γ ⊢ [uσ0]uN ≈ iM ]]$.
                    By \cref{corollary:nf-pres-subt},
                    $[[Γ ⊢ [uσ0]uN ≈ [uσ0]nf(uN)]]$,
                    and then by transitivity, $[[Γ ⊢ [uσ0]nf(uN) ≈ iM]]$,
                    that is $[[Γ ⊢ [uσ0]nf(uN) ≈ iM]]$.
            \end{enumerate}
    \end{caseof}
\end{proof}


\end{document}
