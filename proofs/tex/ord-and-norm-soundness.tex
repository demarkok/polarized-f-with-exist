\begin{lemma}[Soundness of variable ordering]
  \label{lemma:ord-soundness}
  Variable ordering extracts precisely used free variables.
  \begin{itemize}
    \item[$-$] $[[ {ord varset in uN} ]] \equiv [[varset ∩ fv uN]]$ (as sets)
    \item[$+$] $[[ {ord varset in uP} ]] \equiv [[varset ∩ fv uP]]$ (as sets)
  \end{itemize}
\end{lemma}
\begin{proof}
  Straightforward mutual induction on 
  $[[ ord varset in uN = ordVars ]]$ and $[[ ord varset in uP = ordVars ]]$
\end{proof}


\begin{corollary}[Additivity of ordering]
  \label{corollary:ord-additivity}
  Variable ordering is additive (in terms of set union) with respect to its first argument.
  \begin{itemize}
    \item[$-$] $[[ {ord (varset1 ∪ varset2) in iN} ]]
                \equiv
                [[{ord varset1 in iN} ∪ {ord varset2 in iN}]]$ (as sets)
    \item[$+$] $[[{ord (varset1 ∪ varset2) in iP}]]
                \equiv
                [[{ord varset1 in iP} ∪ {ord varset2 in iP}]]$ (as sets)

  \end{itemize}
\end{corollary}

\begin{corollary}[Weakening of ordering]
  \label{corollary:ord-weakening}
  Extending the first argument of the ordering with unused variables does not
  change the result.
  \begin{itemize}
  \item[$-$] $[[ ord (varset ∩ fv iN) in iN ]] = [[ ord varset in iN ]]$
  \item[$+$] $[[ ord (varset ∩ fv iP) in iP ]] = [[ ord varset in iP ]]$
  \end{itemize}
\end{corollary}


% \begin{corollary}
%   \label{corollary:mu-ord}
%   Suppose that $\mu$ is a bijection between two sets of variables
%   $\mu : A \leftrightarrow B$, and $A$ and $B$ are disjoint with $[[varset]]$.
%   \begin{itemize}
%   \item[$-$]
%     If $A$ and $B$ are disjoint with $[[fv iN]]$ then
%     $[[  ord varset in iN ]] = [[ord varset in [mu] iN ]]$
%   \item[$+$]
%     If $A$ and $B$ are disjoint with $[[fv iP]]$ then
%     $[[  ord varset in iP ]] = [[ord varset in [mu] iP ]]$
%   \end{itemize}
% \end{corollary}
% \begin{proof}

% \end{proof}


\begin{lemma}[Soundness of quantifier normalization]
  \label{lemma:normalization-soundness}
  Normalization respects equivalence.

  \begin{itemize}
    \item[$-$] $[[iN ≈ nf(iN)]]$
    \item[$+$] $[[iP ≈ nf(iP)]]$
  \end{itemize}

\end{lemma}
\begin{proof}
  Mutual induction on $[[nf(iN) = iM]]$ and $[[nf(iP) = iQ]]$.
  Let us consider how this judgment is formed:
  \begin{itemize}
    \item{\nameref{\ottdruleNrmNVarLabel} and \nameref{\ottdruleNrmPVarLabel}} by
      the corresponding equivalence rules.
    \item{\nameref{\ottdruleNrmShiftULabel}, \nameref{\ottdruleNrmShiftDLabel},
        and \nameref{\ottdruleNrmArrowLabel}} by the induction hypothesis and
      the corresponding congruent equivalence rules.
    \item{\nameref{\ottdruleNrmForallLabel}} From the induction hypothesis, we
      know that $[[iN ≈ iN']]$. In particular, by \cref{lemma:equiv-fv}, $[[fv
      iN]] \equiv [[fv iN']]$. Then by \cref{lemma:ord-soundness}, $[[pas']]
      \equiv [[pas ∩ fv iN']] \equiv [[pas ∩ fv iN]]$, and thus,
      $[[pas' ∩ fv iN']] \equiv [[pas ∩ fv iN]]$.
      
      To prove $[[∀pas.iN ≈ ∀pas'.iN']]$, it suffices to provide a bijection 
      $\mu : [[pas' ∩ fv iN']] \leftrightarrow [[pas ∩ fv iN]]$ such that
      $[[iN ≈ [mu]iN']]$. Since these sets are equal, we take $\mu = id$.
    \item{\nameref{\ottdruleNrmExistsLabel}} Same as for \nameref{\ottdruleNrmForallLabel}.
  \end{itemize}
\end{proof}

\begin{corollary}
  \label{corollary:fv-nf}
  Free variables are not changed by the normalization
  \begin{itemize}
  \item[$-$] $[[fv iN]] \equiv [[fv nf(iN)]]$
  \item[$+$] $[[fv iP]] \equiv [[fv nf(iP)]]$
  \end{itemize}
\end{corollary}
\begin{proof}
  Immediately from \cref{lemma:normalization-soundness,lemma:equiv-fv}.
\end{proof}
