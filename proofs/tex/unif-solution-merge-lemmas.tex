\begin{lemma} [Soundness of Merge of Unification Solutions]
    \label{lemma:unif-sol-merge-soundness}
    Suppose that $[[us1 : Θ | varset1]]$ and $[[us2 : Θ | varset2]]$ 
    are normalized unification solutions 
    (i.e. $[[us1]]$ and $[[us2]]$ can only have equivalence-shaped restrictions, 
    which types are normalized),
    If $[[us1 & us2]]$ is defined then
    $[[us1 & us2]] = [[us1]] ∪ [[us2]]$, where
\end{lemma}
\begin{proof}
    \hfill
    \begin{itemize}
        \item $[[us1 & us2]] \subseteq [[us1]] \cup [[us2]]$\\
        By definition, 
        $[[us1 & us2]]$ consists of three parts:
        entries of $[[us1]]$ that do not have matching entries of $[[us2]]$,
        entries of $[[us2]]$ that do not have matching entries of $[[us1]]$,
        and the merge of matching entries.

        If $[[usEntry]]$ is from the first or the second part, 
        then $[[usEntry]] \in [[us1]] \cup [[us2]]$. 

        If $[[usEntry]]$ is from the third part,
        then $[[usEntry]]$ is the merge of two matching entries
        $[[usEntry1]] \in [[us1]]$ and $[[usEntry2]] \in [[us2]]$.
        Since $[[us1]]$ and $[[us2]]$ are normalized unification solutions, 
        $[[usEntry1]]$ and $[[usEntry2]]$ have one of the following forms:
        \begin{itemize}
            \item $[[α1̂⁺ :≈ iP1]]$ and $[[α2̂⁺ :≈ iP2]]$, 
                where $[[iP1]]$ and $[[iP2]]$ are normalized,
                and then since $[[usEntry1 & usEntry2]]$ exists, 
                \ruleref{\ottdruleSMEPEqEqLabel} was applied to infer it.
                It means that 
                $[[usEntry]] = ([[usEntry1 & usEntry2]]) = [[usEntry1]] = [[usEntry2]]$;
            \item $[[α1̂⁻ :≈ iN1]]$ and $[[α2̂⁻ :≈ iN2]]$, 
               then symmetrically, 
               $[[usEntry]] = ([[usEntry1 & usEntry2]]) = [[usEntry1]] = [[usEntry2]]$
        \end{itemize}
        In both cases, $[[usEntry]] \in [[us1]] \cup [[us2]]$.
        \item $[[us1]] \cup [[us2]] \subseteq [[us1 & us2]]$\\
        Let us take 
        an arbitrary $[[usEntry]] \in [[us1]]$.
        Then since $[[us1]]$ is a unification solution,
         $[[usEntry]]$ has one of the following forms:
        \begin{itemize}
            \item $[[α̂⁺ :≈ iP]]$ where $[[iP]]$ is normalized.
            If $[[α̂⁺]] \notin [[dom(us2)]]$, then $[[usEntry]] \in [[us1 & us2]]$.
            Otherwise there is a normalized
            $[[usEntry]] = [[(α̂⁺ :≈ iP')]] \in [[us2]]$ and then
            since $[[us1 & us2]]$ exists, 
            \ruleref{\ottdruleSMEPEqEqLabel} was applied to construct
            $[[usEntry & usEntry']] \in [[us1 & us2]]$.
            By inversion of \ruleref{\ottdruleSMEPEqEqLabel},
            $[[usEntry & usEntry']] = [[usEntry]]$, and
            $[[nf(iP) = nf(iP')]]$, which since $[[iP]]$
            and $[[iP']]$ are normalized, implies that $[[iP = iP']]$, 
            that is $[[usEntry]] = [[usEntry']]$.
            This way, $[[usEntry']] = [[usEntry]] \in [[us1 & us2]]$.
            \item $[[α̂⁻ :≈ iN]]$ where $[[iN]]$ is normalized.
            Then symmetrically, $[[usEntry]] \in [[us1 & us2]]$.
        \end{itemize}
        Similarly, if we take an arbitrary $[[usEntry']] \in [[us2]]$,
        then $[[usEntry']] \in [[us1 & us2]]$. In fact, this is 
        the case where normalization is important, since
        \ruleref{\ottdruleSMEPEqEqLabel} returns the left-hand
        operand of $\&$, but as noted above, $[[nf(iP) = nf(iP')]]$
        implies $[[iP = iP']]$ for normalized types.
    \end{itemize}
\end{proof}

\begin{corollary}
    \label{corollary:unif-sol-merge-soundness}
    Suppose that $[[us1 : Θ | varset1]]$ and $[[us2 : Θ | varset2]]$ 
    are normalized unification solutions 
    If $[[us1 & us2]]$ is defined then
    \begin{enumerate}
        \item $[[us1 & us2 : Θ|varset1 ∪ varset2]]$,
        \item $[[us1 & us2]]$ is a normalized unification solution, 
        \item $[[us1 & us2 | varseti]] = [[usi]]$ for $i = 1, 2$,
    \end{enumerate}
\end{corollary}
\begin{proof}
    The first two properties follow immediately from
    \cref{lemma:unif-sol-merge-soundness}.

    To prove the third property, first notice that
    $[[us1 & us2 | varseti]] \supseteq [[usi]]$ follows immediately from 
    $[[us1 & us2]] = [[us1]] \cup [[us2]]$.
    For the other inclusion $[[us1 & us2 | varset1]] \subseteq [[us1]]$,
    let us take an arbitrary $[[usEntry]] \in [[us1 & us2 | varset1]]$.
    Then $[[usEntry]]$ is either from $[[us1]]$ (if it does not have a matching entry in
    $[[us2]]$) or it is a result of merge of two matching entries, 
    but since $[[us1 & us2]]$ exist and $[[us1]]$ and $[[us2]]$ 
    are normalized unification solutions, the matching
    entries of $[[us1]]$ and $[[us2]]$ are equal, and they are equal to their merge.
    In both cases, $[[usEntry]] \in [[us1]]$.
    Similarly, $[[us1 & us2 | varset2]] \subseteq [[us2]]$.
\end{proof}


\begin{lemma} [Completeness of Unification Solution Merge] 
    \label{lemma:unif-sol-merge-completeness}
    Suppose that $[[us]]$ is a unification solution, 
    and $[[varset1]]$, $[[varset2]]$ are sets of variables. 
    Then $[[us | varset1 & us | varset2]]$ is defined and
    equal to $[[us | varset1 ∪ varset2]]$.
\end{lemma}
\begin{proof}
    $[[us | varset1 & us | varset2]]$ is defined as the union of three parts:
    entries of $[[us | varset1]]$ that do not have matching entries of $[[us | varset2]]$,
    entries of $[[us | varset2]]$ that do not have matching entries of $[[us | varset1]]$,
    and the merge of matching entries.
    The first two parts are defined. The merge of matching entries is defined
    by \ruleref{\ottdruleSMEPEqEqLabel}, since the matching entries must be equal
    if they both belong to $[[us]]$.

    It remains to show that $[[us | varset1 & us | varset2]] = [[us | varset1 ∪ varset2]]$.
    It is easy to see that the three parts comprising $[[us | varset1 & us | varset2]]$ 
    correspond to the three parts comprising 
    $[[us | varset1 ∪ varset2]] = [[us | (varset1 \ varset2)]] \cup [[us | (varset2 \ varset1)]]
    \cup [[us | varset1 ∩ varset2]]$. 
\end{proof}
