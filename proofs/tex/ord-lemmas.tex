\begin{observation}[Ordering is deterministic]
  \label{obs:ord-deterministic}
  If $[[ord varset in iN = ordVars1]]$ and $[[ord varset in iN = ordVars2]]$ then $[[ordVars1 = ordVars2]]$.
  If $[[ord varset in iP = ordVars1]]$ and $[[ord varset in iP = ordVars2]]$ then $[[ordVars1 = ordVars2]]$.
  This way, we can use $[[ord varset in iN]]$ and as a function on $[[iN]]$,
  and $[[ord varset in iP]]$ as a function on $[[iP]]$.
\end{observation}
\begin{proof}
  By straightforward induction.
\end{proof}


\begin{definition} [Collision free bijection]
  We say that a bijection $\mu : A \leftrightarrow B$ between sets of
  variables is \textbf{collision free on sets} $P$ and $Q$ if and only if
  \begin{enumerate}
    \item $\mu(P \cap A) \cap Q = \emptyset$
    \item $\mu(Q \cap A) \cap P = \emptyset$
  \end{enumerate}
\end{definition}


\begin{lemma}[Soundness of variable ordering]
  \label{lemma:ord-soundness}
  Variable ordering extracts used free variables.
  \begin{itemize}
    \item[$-$] $[[ {ord varset in iN} ]] = [[varset ∩ fv iN]]$ (as sets)
    \item[$+$] $[[ {ord varset in iP} ]] = [[varset ∩ fv iP]]$ (as sets)
  \end{itemize}
\end{lemma}
\begin{proof}
  Straightforward mutual induction on 
  $[[ ord varset in iN = ordVars ]]$ and $[[ ord varset in iP = ordVars ]]$
\end{proof}


\begin{corollary}[Additivity of ordering]
  \label{corollary:ord-additivity}
  Variable ordering is additive (in terms of set union) with respect to its first argument.
  \begin{itemize}
    \item[$-$] $[[ {ord (varset1 ∪ varset2) in iN} ]]
                \equiv
                [[{ord varset1 in iN} ∪ {ord varset2 in iN}]]$ (as sets)
    \item[$+$] $[[{ord (varset1 ∪ varset2) in iP}]]
                \equiv
                [[{ord varset1 in iP} ∪ {ord varset2 in iP}]]$ (as sets)

  \end{itemize}
\end{corollary}

\begin{corollary}[Weakening of ordering]
  \label{corollary:ord-weakening}
  Extending the first argument of the ordering with unused variables does not
  change the result.
  \begin{itemize}
    \item[$-$] $[[ ord (varset ∩ fv iN) in iN ]] = [[ ord varset in iN ]]$
    \item[$+$] $[[ ord (varset ∩ fv iP) in iP ]] = [[ ord varset in iP ]]$
  \end{itemize}
\end{corollary}

\begin{corollary}[Idempotency of ordering]
  \label{corollary:ord-idemp}
  \hfill
  \begin{itemize}
    \item[$-$] If $[[ ord varset in iN = ordVars ]]$ then 
      $[[ ord {ordVars} in iN = ordVars ]]$,
    \item[$+$] If $[[ ord varset in iP = ordVars ]]$ then 
      $[[ ord {ordVars} in iP = ordVars ]]$;
  \end{itemize}
\end{corollary}
\begin{proof}
  By \cref{lemma:ord-soundness,corollary:ord-weakening}.
\end{proof}
  
\end{proof}

\begin{lemma}[Distributivity of renaming over variable ordering]
  \label{lemma:distr-mu-ord}
  Suppose that $\mu$ is a bijection between two sets of variables
  $\mu : A \leftrightarrow B$.
  
  \begin{itemize}
  \item[$-$]
    If $[[mu]]$ is collision free on $[[varset]]$ and $[[fv iN]]$ then
    $[[ [mu] (ord varset in iN) ]] = [[ord ([mu] varset) in [mu] iN ]]$
  \item[$+$]
    If $[[mu]]$ is collision free on $[[varset]]$ and $[[fv iP]]$ then
    $[[ [mu] (ord varset in iP) ]] = [[ord ([mu] varset) in [mu] iP ]]$
  \end{itemize}
\end{lemma}

\begin{proof}
  Mutual induction on $[[iN]]$ and $[[iP]]$.
  \begin{caseof}
  \item $[[iN]]$ = $[[na]]$ \label{case:distr-mu-ord:var} \\
    let us consider four cases:
    \begin{caseof}
    \item $[[na]] \in A$ and $[[na]] \in [[varset]]$\\ Then
      $
      \begin{aligned}[t] [[ [mu] (ord varset in iN) ]] &= [[ [mu] (ord varset in na)]] \\
                                                             &= [[ [mu] na ]]
                                                             && \text{by \ruleref{\ottdruleOPVarInLabel}}\\
                                                             &= [[nb]]
                                                             && \text{for some $[[nb]] \in B$ (notice that $[[nb]] \in [[ [mu]varset ]]$)} \\
                                                             &= [[ ord [mu]varset in nb ]]
                                                             && \text{by \ruleref{\ottdruleOPVarInLabel},
                                                                because $[[nb]] \in [[ [mu]varset ]]$} \\
                                                             &= [[ord [mu] varset in [mu] na ]]
       \end{aligned}
       $
     \item $[[na]] \notin A$ and $[[na]] \notin [[varset]]$\\
       Notice that
       $[[ [mu] (ord varset in iN) ]] = [[ [mu] (ord varset in na)]] = [[·]]$ by
       \ruleref{\ottdruleOPVarNInLabel}.
       On the other hand, $[[ ord [mu] varset in [mu] na = ord [mu] varset
       in na ]] = [[·]]$ The latter equality is from
       \ruleref{\ottdruleOPVarNInLabel}, because
       $[[mu]]$ is collision free on $[[varset]]$ and $[[fv iN]]$, so
       $[[fv iN]] \ni [[na]] \notin [[mu]](A \cap [[varset]]) \cup
       [[varset]] \supseteq [[ [mu] varset ]]$.
     \item $[[na]] \in A$ but $[[na]] \notin [[varset]]$\\ Then
       $[[ [mu] (ord varset in iN) ]] = [[ [mu] (ord varset in na)]] = [[·]]$
       by \ruleref{\ottdruleOPVarNInLabel}.
       To prove that $[[ ord [mu] varset in [mu] na ]] = [[·]]$, we apply
       \ruleref{\ottdruleOPVarNInLabel}. Let us show that
       $[[ [mu] na ]] \notin [[ [mu] varset ]]$.
       Since $[[ [mu] na ]] = [[mu]]([[na]])$ and
       $[[ [mu] varset ]] \subseteq [[mu]](A \cap [[varset]]) \cup [[varset]]$,
       it suffices to prove 
       $[[mu]]([[na]]) \notin [[mu]](A \cap [[varset]]) \cup [[varset]]$.

       \begin{enumerate}
       \item[(i)] If there is an element $x \in A \cap [[varset]]$ such that
         $[[mu]] x = [[mu]] [[na]]$, then $x = [[na]]$ by bijectivity of
         $[[mu]]$, which contradicts with $[[na]] \notin [[varset]]$. This way, 
         $[[mu]]([[na]]) \notin [[mu]](A \cap [[varset]])$.
       \item[(ii)]
         Since $[[mu]]$ is collision free on $[[varset]]$ and $[[fv iN]]$,
         $[[mu]] (A \cap [[fv iN]]) \ni [[mu]]([[na]]) \notin [[varset]]$.
       \end{enumerate}

     \item $[[na]] \notin A$ but $[[na]] \in [[varset]]$\\
       $[[ ord [mu] varset in [mu] na ]] = [[ ord [mu] varset in na ]] = [[na]]$.
       The latter is by \ruleref{\ottdruleOPVarNInLabel}, because
       $[[na]] = [[ [mu] na ]] \in [[ [mu] varset ]]$ since $[[na]] \in [[varset]]$.
       On the other hand, $[[ [mu] (ord varset in iN) ]] = [[ [mu] (ord varset in na)]] = [[ [mu] na ]] = [[na]]$.
    \end{caseof}
  
  \item $[[iN]] = [[↑iP]]$ \\
    $\begin{aligned}[t]
       [[ [mu] (ord varset in iN) ]] &= [[ [mu] (ord varset in ↑iP) ]] \\
                                     &= [[ [mu] (ord varset in iP) ]]
                                     && \text{by \ruleref{\ottdruleOShiftULabel}}\\
                                     &= [[ ord [mu]varset in [mu]iP ]]
                                     && \text{by the induction hypothesis}\\
                                     &= [[ ord [mu]varset in  ↑[mu]iP ]]
                                     && \text{by \ruleref{\ottdruleOShiftULabel}}\\
                                     &= [[ ord [mu]varset in  [mu]↑iP ]]
                                     && \text{by the definition of substitution}\\
                                     &= [[ ord [mu]varset in  [mu]iN ]]
            \end{aligned}$
          
   \item $[[iN]] = [[iP → iM]]$  \\
     $\begin{aligned}[t]
        [[ [mu] (ord varset in iN) ]] &= [[ [mu] (ord varset in iP → iM) ]] \\
                                      &= [[ [mu] (ordVars1, (ordVars2 \ {ordVars1})) ]]
                                      && \text{where } [[ord varset in iP = ordVars1]] \text{ and } [[ord varset in iM = ordVars2]] \\
                                      &= [[ [mu] ordVars1, [mu](ordVars2 \ {ordVars1}) ]] \\
                                      &= [[ [mu] ordVars1, ([mu]ordVars2 \ [mu]{ordVars1}) ]]
                                      && \text{by induction on $[[ordVars2]]$;
                                         the inductive step is similar to \cref{case:distr-mu-ord:var}.
                                         Notice that $[[mu]]$ is} \\
                                      & && \text{collision free on $[[{ordVars1}]]$ and $[[{ordVars2}]]$
                                           since
                                           $[[{ordVars1}]] \subseteq [[varset]]$ and
                                           $[[{ordVars2}]] \subseteq [[fv iN]]$ }\\
                                      &= [[ [mu] ordVars1, ([mu]ordVars2 \ {[mu]ordVars1}) ]]
      \end{aligned}$ \\
    $\begin{aligned}[t]
       [[  (ord [mu] varset in [mu]iN) ]] &= [[ (ord [mu] varset in [mu]iP → [mu]iM) ]] \\
                                     &= [[ (ordVarsb1, (ordVarsb2 \ {ordVarsb1})) ]]
                                     && \text{where } [[ord [mu] varset in [mu] iP = ordVarsb1]] \text{ and } [[ord [mu] varset in [mu] iM = ordVarsb2]] \\
                                          & && \text{then by the induction
                                               hypothesis,
                                               $[[ordVarsb1]] = [[ [mu] ordVars1 ]]$,
                                               $[[ordVarsb2]] = [[ [mu] ordVars2 ]]$,
                                               }\\
                                     &= [[ [mu] ordVars1, ([mu]ordVars2 \ {[mu]ordVars1}) ]]
     \end{aligned}$
   
   \item $[[iN]] = [[∀ pas.iM]]$ \\
     $\begin{aligned}[t]
          [[ [mu] (ord varset in iN) ]] &= [[ [mu] ord varset in ∀pas.iM]] \\
                                        &= [[ [mu] ord varset in iM]] \\
                                        &= [[ ord [mu] varset in [mu] iM]]
                                        && \text {by the induction hypothesis}\\
     \end{aligned}$ \\
     $ 
     \begin{aligned}[t]
       [[ (ord [mu] varset in [mu] iN) ]] &= [[ ord [mu] varset in [mu] ∀pas.iM ]] \\
                                          &= [[ ord [mu] varset in ∀pas.[mu]iM ]] \\
                                          &= [[ ord [mu] varset in [mu] iM ]] \\
     \end{aligned}
     $
     
  \end{caseof}
\end{proof}

\begin{lemma}[Ordering is not affected by independent substitutions]
  \label{lemma:ord-sigma}
  Suppose that $[[Γ2 ⊢ σ : Γ1]]$, i.e. $[[σ]]$ maps variables from $[[Γ1]]$ into types
  taking free variables from $[[Γ2]]$, and $[[varset]]$ is a set of variables
  disjoint with both $[[Γ1]]$ and $[[Γ2]]$, 
  $[[iN]]$ and $[[iP]]$ are types. Then
    \begin{itemize}
  \item[$-$] $[[ ord varset in [σ]iN ]] = [[ord varset in iN ]]$
  \item[$+$] $[[ ord varset in [σ]iP ]] = [[ord varset in iP ]]$
  \end{itemize}
\end{lemma}
\begin{proof}
  Mutual induction on $[[iN]]$ and $[[iP]]$.
  \begin{caseof}
    \item $[[iN = na]]$ \\
      If $[[na ∉ Γ1]]$ then $[[ [σ]na = na ]]$ and $[[ ord varset in [σ]na ]] = [[ ord varset in na ]]$, 
      as requried.
      If $[[na ∊ Γ1]]$ then $[[na ∉ varset]]$, so $[[ ord varset in na ]] = [[·]]$.
      Moreover, $[[Γ2 ⊢ σ : Γ1]]$ means $[[ fv([σ]na) ⊆ Γ2 ]]$, and thus, 
      as a set, $[[ ord varset in [σ]na ]] = [[varset ∩ fv([σ]na)]] \subseteq [[varset ∩ Γ2]] = [[·]]$.
    \item $[[iN = ∀pas.iM]]$\\
      We can assume $[[{pas} ∩ Γ1 = ∅]]$
      and $[[{pas} ∩ varset = ∅]]$. Then 

      $\begin{aligned}[t]
         [[ ord varset in [σ]iN ]] &= [[ ord varset in [σ]∀pas.iM ]] \\
                                   &= [[ ord varset in ∀pas.[σ]iM ]]\\
                                   &= [[ ord varset in [σ]iM ]]
                                   && \text{by the induction hypothesis}\\
                                   &= [[ ord varset in iM ]]\\
                                   &= [[ ord varset in ∀pas.iM ]]\\
                                   &= [[ ord varset in iN ]]
       \end{aligned}$
    \item $[[iN = ↑iP]]$\\
       $\begin{aligned}[t]
        [[ ord varset in [σ]iN ]] &= [[ ord varset in [σ]↑iP ]] \\
                                   &= [[ ord varset in ↑[σ]iP ]]
                                   && \text{by the definition of substitution}\\
                                   &= [[ ord varset in [σ]iP ]]
                                   && \text{by the induction hypothesis}\\
                                   &= [[ ord varset in iP ]]
                                   && \text{by the definition of substitution}\\
                                   &= [[ ord varset in ↑iP ]]
                                   && \text{by the definition of ordering}\\
                                   &= [[ ord varset in iN ]]
       \end{aligned} $

    \item $[[iN = iP → iM]]$\\
       $ \begin{aligned}
        [[ ord varset in [σ]iN ]] &= [[ ord varset in [σ](iP → iM) ]] \\
                                   &= [[ ord varset in ([σ]iP → [σ]iM) ]]
                                   && \text{by the definition of substitution}\\
                                   &= [[ ord varset in [σ]iP, (ord varset in [σ]iM \ {ord varset in [σ]iP}) ]]
                                   && \text{by the definition of ordering}\\
                                   &= [[ ord varset in iP, (ord varset in iM \ {ord varset in iP}) ]]
                                   && \text{by the induction hypothesis}\\
                                   &= [[ ord varset in iP → iM ]]
                                   && \text{by the definition of ordering}\\
                                   &= [[ ord varset in iN ]]
       \end{aligned} $
    \item The proofs of the positive cases are symmetric.
  \end{caseof}
\end{proof}

\begin{lemma}[Completeness of variable ordering]
  \label{lemma:ord-completeness}
  Variable ordering is invariant under equivalence. For arbitrary $[[varset]]$,
   \begin{itemize}
  \item[$-$] If $[[iN ≈ iM]]$ then $[[ord varset in iN]] = [[ord varset in iM]]$ (as lists)
  \item[$+$] If $[[iP ≈ iQ]]$ then $[[ord varset in iP]] = [[ord varset in iQ]]$ (as lists)
  \end{itemize}
\end{lemma}
\begin{proof}
  Mutual induction on $[[iN ≈ iM]]$ and $[[iP ≈ iQ]]$.
\end{proof}
