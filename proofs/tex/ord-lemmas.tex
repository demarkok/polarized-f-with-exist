\obsOrdDeterministic*
\begin{proof}
  By mutual structural induction on $[[iN]]$ and $[[iP]]$.
  Notice that the shape of the term $[[iN]]$ or $[[iP]]$
  uniquely determines the last used inference rule,
  and all the premises are deterministic on the input.
\end{proof}

\lemOrdSoundness*
\begin{proof}
  We prove it by mutual induction on 
  $[[ ord varset in iN = ordVars ]]$ and $[[ ord varset in iP = ordVars ]]$.
  The only non-trivial cases are 
  \ruleref{\ottdruleOArrowLabel} and 
  \ruleref{\ottdruleOForallLabel}.  
  \begin{caseof}
    \item \ruleref{\ottdruleOArrowLabel}  
      Then the inferred ordering judgement has shape
      $[[ord varset in iP → iN = ordVars1, (ordVars2 \ {ordVars1})]]$
      and by inversion, 
      $[[ord varset in iP = ordVars1]]$   
      and 
      $[[ord varset in iN = ordVars2]]$.

      By definition of free variables, 
      $[[varset ∩ fv iP → iN = varset ∩ fv iP ∪ varset ∩ fv iN]]$,
      and since by the induction hypothesis 
      $[[varset ∩ fv iP = {ordVars1}]]$ and
      $[[varset ∩ fv iN = {ordVars2}]]$,
      we have
      $[[varset ∩ fv iP → iN = {ordVars1} ∪ {ordVars2}]]$.

      On the other hand, 
      as a set $[[{ordVars1} ∪ {ordVars2}]]$
      is equal to $[[ordVars1, (ordVars2 \ {ordVars1})]]$. 
    \item  \ruleref{\ottdruleOForallLabel}.
      Then  the inferred ordering judgement has shape
      $[[ord varset in ∀pas.iN = ordVars]]$,
      and by inversion, 
      $[[varset ∩ {pas} = ∅]]$    
      $[[ord varset in iN = ordVars]]$.
      The latter implies that $[[varset ∩ fv iN = {ordVars}]]$.
      We need to show that $[[varset ∩ fv ∀pas.iN = {ordVars}]]$,
      or equivalently, that
      $[[varset ∩ (fv iN \ {pas}) = varset ∩ fv iN ]]$,
      which holds since $[[varset ∩ {pas} = ∅]]$.
  \end{caseof}
\end{proof}


\corOrdAdditivity*

\lemOrdWeakening*
\begin{proof}
  Mutual structural induction on $[[iN]]$ and $[[iP]]$.

  \begin{caseof}
    \item If $[[iN]]$ is a variable $[[na]]$,
      we notice that $[[na ∊ varset]]$ 
      is equivalent to $[[na ∊ varset ∩ {na}]]$.
    \item If $[[iN]]$ has shape $[[↑iP]]$, then
      the required property holds immediately by the 
      induction hypothesis, since 
      $[[fv(↑iP) = fv(iP)]]$.
    \item If the term has shape $[[iP → iN]]$ then
      \ruleref{\ottdruleOArrowLabel} was applied
      to infer $[[ ord (varset ∩ (fv iP ∪ fv iN)) in iP → iN ]]$
      and $[[ ord varset in iP → iN]]$. 
      By inversion, the result of 
      $[[ ord (varset ∩ (fv iP ∪ fv iN)) in iP → iN ]]$
      depends on 
      $A = [[ ord (varset ∩ (fv iP ∪ fv iN)) in iP]]$
      and 
      $B = [[ ord (varset ∩ (fv iP ∪ fv iN)) in iN]]$.
      The result of
       $[[ ord varset in iP → iN]]$ 
       depends on 
      $X = [[ord varset in iP]]$ and
      $Y = [[ord varset in iN]]$.

      Let us show that $A = B$ and $X = Y$, so the results are equal. 
      By the induction hypothesis and set properties,
      $[[ ord (varset ∩ (fv iP ∪ fv iN)) in iP ]] = 
       [[ ord (varset ∩ (fv iP ∪ fv iN)) ∩ fv(iP) in iP ]] = 
       [[ ord varset ∩ fv(iP) in iP ]] = 
       [[ ord varset in iP ]]$.
      Analogously, 
      $[[ ord (varset ∩ (fv iP ∪ fv iN)) in iN ]]$ $=$
      \\ $[[ ord varset in iN ]]$.
    \item If the term has shape $[[∀pas.iN]]$,
      we can assume that $[[pas]]$ is disjoint
      from $[[varset]]$,
      since we operate on alpha-equivalence classes.
      Then using the induction hypothesis,
      set properties and \ruleref{\ottdruleOForallLabel}: 
      $[[ord varset ∩ (fv(∀pas.iN)) in ∀pas.iN]] =
       [[ord varset ∩ (fv(iN) \ {pas}) in iN]] =
       [[ord varset ∩ (fv(iN) \ {pas}) ∩ fv(iN) in iN]] =
       [[ord varset ∩ fv(iN) in iN]] =
       [[ord varset in iN]]$.
  \end{caseof}
\end{proof}

\corOrdIdemp*
\begin{proof}
  By \cref{lemma:ord-soundness,corollary:ord-weakening}.
\end{proof}
  

Next, we make a set-theoretical observation
that will be useful further.
In general, any injective function (its image)
distributes over the set intersection.
However, for convenience, we allow the bijections
on variables to be applied
\emph{outside of their domains}
(as identities), which may violate
the injectivity. To deal with these cases, 
we define a special notion of
bijections collision-free on certain sets
in such a way that
a bijection that is collision-free on $P$ and $Q$,
distributes over intersection of $P$ and $Q$.

\begin{definition} [Collision-free bijection]
  We say that a bijection $\mu : A \leftrightarrow B$ between sets of
  variables is \textbf{collision-free on sets} $P$ and $Q$ if and only if
  \begin{enumerate}
    \item $\mu(P \cap A) \cap Q = \emptyset$
    \item $\mu(Q \cap A) \cap P = \emptyset$
  \end{enumerate}
\end{definition}

\begin{observation}
  Suppose that $\mu : A \leftrightarrow B$ is a bijection between two sets of variables,
  and $\mu$ is collision-free on $P$ and $Q$.
  Then $\mu(P \cap Q) = \mu(P) \cap \mu(Q)$.
\end{observation}
  
\lemDistrMuOrd*
\begin{proof}
  Mutual induction on $[[iN]]$ and $[[iP]]$.
  \begin{caseof}
  \item $[[iN]]$ = $[[na]]$ \label{case:distr-mu-ord:var} \\
    let us consider four cases:
    \begin{caseof}
    \item $[[na]] \in A$ and $[[na]] \in [[varset]]$. Then
      \begin{align*} [[ [mu] (ord varset in iN) ]] &= [[ [mu] (ord varset in na)]] \\
                                                             &= [[ [mu] na ]]
                                                             && \text{by \ruleref{\ottdruleOPVarInLabel}}\\
                                                             &= [[nb]]
                                                             && \text{for some $[[nb]] \in B$ (notice $[[nb]] \in [[ [mu]varset ]]$)} \\
                                                             &= [[ ord [mu]varset in nb ]]
                                                             && \text{by \ruleref{\ottdruleOPVarInLabel},
                                                                because $[[nb]] \in [[ [mu]varset ]]$} \\
                                                             &= [[ord [mu] varset in [mu] na ]]
       \end{align*}
     \item $[[na]] \notin A$ and $[[na]] \notin [[varset]]$\\
       Notice that
       $[[ [mu] (ord varset in iN) ]] = [[ [mu] (ord varset in na)]] = [[·]]$ by
       \ruleref{\ottdruleOPVarNInLabel}.
       On the other hand, $[[ ord [mu] varset in [mu] na = ord [mu] varset
       in na ]] = [[·]]$ The latter equality is from
       \ruleref{\ottdruleOPVarNInLabel}, because
       $[[mu]]$ is collision-free on $[[varset]]$ and $[[fv iN]]$, so
       $[[fv iN]] \ni [[na]] \notin [[mu]](A \cap [[varset]]) \cup
       [[varset]] \supseteq [[ [mu] varset ]]$.
     \item $[[na]] \in A$ but $[[na]] \notin [[varset]]$\\ Then
       $[[ [mu] (ord varset in iN) ]] = [[ [mu] (ord varset in na)]] = [[·]]$
       by \ruleref{\ottdruleOPVarNInLabel}.
       To prove that\\ $[[ ord [mu] varset in [mu] na ]] = [[·]]$, we apply
       \ruleref{\ottdruleOPVarNInLabel}. Let us show that
       $[[ [mu] na ]] \notin [[ [mu] varset ]]$.
       Since $[[ [mu] na ]] = [[mu]]([[na]])$ and
       $[[ [mu] varset ]] \subseteq [[mu]](A \cap [[varset]]) \cup [[varset]]$,
       it suffices to prove 
       $[[mu]]([[na]]) \notin [[mu]](A \cap [[varset]]) \cup [[varset]]$.

       \begin{enumerate}
       \item[(i)] If there is an element $x \in A \cap [[varset]]$ such that
         $[[mu]] x = [[mu]] [[na]]$, then $x = [[na]]$ by bijectivity of
         $[[mu]]$, which contradicts with $[[na]] \notin [[varset]]$. This way, 
         $[[mu]]([[na]]) \notin [[mu]](A \cap [[varset]])$.
       \item[(ii)]
         Since $[[mu]]$ is collision-free on $[[varset]]$ and $[[fv iN]]$,
         $[[mu]] (A \cap [[fv iN]]) \ni [[mu]]([[na]]) \notin [[varset]]$.
       \end{enumerate}

     \item $[[na]] \notin A$ but $[[na]] \in [[varset]]$\\
       $[[ ord [mu] varset in [mu] na ]] = [[ ord [mu] varset in na ]] = [[na]]$.
       The latter is by \ruleref{\ottdruleOPVarNInLabel}, because
       $[[na]] = [[ [mu] na ]] \in [[ [mu] varset ]]$ since $[[na]] \in [[varset]]$.
       On the other hand, $[[ [mu] (ord varset in iN) ]] = [[ [mu] (ord varset in na)]] = [[ [mu] na ]] = [[na]]$.
    \end{caseof}
  
  \item $[[iN]] = [[↑iP]]$
    \begin{align*}
       [[ [mu] (ord varset in iN) ]] &= [[ [mu] (ord varset in ↑iP) ]] \\
                                     &= [[ [mu] (ord varset in iP) ]]
                                     && \text{by \ruleref{\ottdruleOShiftULabel}}\\
                                     &= [[ ord [mu]varset in [mu]iP ]]
                                     && \text{by the induction hypothesis}\\
                                     &= [[ ord [mu]varset in  ↑[mu]iP ]]
                                     && \text{by \ruleref{\ottdruleOShiftULabel}}\\
                                     &= [[ ord [mu]varset in  [mu]↑iP ]]
                                     && \text{by the definition of substitution}\\
                                     &= [[ ord [mu]varset in  [mu]iN ]]
    \end{align*}
          
   \item $[[iN]] = [[iP → iM]]$
     \begin{align*}
        & [[ [mu] (ord varset in iN) ]] \\ 
        &= [[ [mu] (ord varset in iP → iM) ]] \\
        &= [[ [mu] (ordVars1, (ordVars2 \ {ordVars1})) ]]
          && \text{where } [[ord varset in iP = ordVars1]] \text{ and } [[ord varset in iM = ordVars2]] \\
        &= [[ [mu] ordVars1, [mu](ordVars2 \ {ordVars1}) ]] \\
        &= [[ [mu] ordVars1, ([mu]ordVars2 \ [mu]{ordVars1}) ]]
          && \text{by induction on $[[ordVars2]]$;
                  the inductive step is similar to 
                  \cref{case:distr-mu-ord:var}.}\\
        & && \text{Notice that $[[mu]]$ is collision free on $[[{ordVars1}]]$ and $[[{ordVars2}]]$} \\
        & && \text{since
          $[[{ordVars1}]] \subseteq [[varset]]$ and
          $[[{ordVars2}]] \subseteq [[fv iN]]$ }\\
          &= [[ [mu] ordVars1, ([mu]ordVars2 \ {[mu]ordVars1}) ]]
      \end{align*}
      \hfill\\
      On the other hand,
       $[[  ord [mu] varset in [mu]iN ]] = 
        [[ ord [mu] varset in [mu]iP → [mu]iM ]] = 
        [[ ordVarsb1, (ordVarsb2 \ {ordVarsb1}) ]] = 
        [[ [mu] ordVars1, ([mu]ordVars2 \ {[mu]ordVars1}) ]]$,
        where  $[[ord [mu] varset in [mu] iP = ordVarsb1]]$ 
        and $[[ord [mu] varset in [mu] iM = ordVarsb2]]$, 
        then by the induction hypothesis, 
        $[[ordVarsb1]] = [[ [mu] ordVars1 ]]$, 
        $[[ordVarsb2]] = [[ [mu] ordVars2 ]]$.
   
   \item $[[iN]] = [[∀ pas.iM]]$
     \begin{align*}
          [[ [mu] (ord varset in iN) ]] &= [[ [mu] ord varset in ∀pas.iM]] \\
                                        &= [[ [mu] ord varset in iM]] \\
                                        &= [[ ord [mu] varset in [mu] iM]]
                                        && \text {by the induction hypothesis}\\
     \end{align*}
     \begin{align*}
       [[ (ord [mu] varset in [mu] iN) ]] &= [[ ord [mu] varset in [mu] ∀pas.iM ]] \\
                                          &= [[ ord [mu] varset in ∀pas.[mu]iM ]] \\
                                          &= [[ ord [mu] varset in [mu] iM ]] \\
     \end{align*}
  \end{caseof}
\end{proof}

\lemOrdSigma*
\begin{proof}
  Mutual induction on $[[iN]]$ and $[[iP]]$.
  \begin{caseof}
    \item $[[iN = na]]$ \\
      If $[[na ∉ Γ1]]$ then $[[ [σ]na = na ]]$ and $[[ ord varset in [σ]na ]] = [[ ord varset in na ]]$, 
      as requried.
      If $[[na ∊ Γ1]]$ then $[[na ∉ varset]]$, so $[[ ord varset in na ]] = [[·]]$.
      Moreover, $[[Γ2 ⊢ σ : Γ1]]$ means $[[ fv([σ]na) ⊆ Γ2 ]]$, and thus, 
      as a set, $[[ ord varset in [σ]na ]] = [[varset ∩ fv([σ]na)]] \subseteq [[varset ∩ Γ2]] = [[·]]$.
    \item $[[iN = ∀pas.iM]]$
      We can assume $[[{pas} ∩ Γ1 = ∅]]$
      and $[[{pas} ∩ varset = ∅]]$. Then 
      \begin{align*}[t]
         [[ ord varset in [σ]iN ]] &= [[ ord varset in [σ]∀pas.iM ]] \\
                                   &= [[ ord varset in ∀pas.[σ]iM ]]\\
                                   &= [[ ord varset in [σ]iM ]]
                                   && \text{by the induction hypothesis}\\
                                   &= [[ ord varset in iM ]]\\
                                   &= [[ ord varset in ∀pas.iM ]]\\
                                   &= [[ ord varset in iN ]]
       \end{align*}
    \item $[[iN = ↑iP]]$
       \begin{align*}[t]
        [[ ord varset in [σ]iN ]] &= [[ ord varset in [σ]↑iP ]] \\
                                   &= [[ ord varset in ↑[σ]iP ]]
                                   && \text{by the definition of substitution}\\
                                   &= [[ ord varset in [σ]iP ]]
                                   && \text{by the induction hypothesis}\\
                                   &= [[ ord varset in iP ]]
                                   && \text{by the definition of substitution}\\
                                   &= [[ ord varset in ↑iP ]]
                                   && \text{by the definition of ordering}\\
                                   &= [[ ord varset in iN ]]
       \end{align*}

    \item $[[iN = iP → iM]]$
       \begin{align*}
        [[ ord varset in [σ]iN ]] &= [[ ord varset in [σ](iP → iM) ]] \\
                                   &= [[ ord varset in ([σ]iP → [σ]iM) ]]
                                   && \text{def. of substitution}\\
                                   &= [[ ord varset in [σ]iP]],\\
                                   &\phantom{=} ~ [[(ord varset in [σ]iM \ {ord varset in [σ]iP}) ]]
                                   && \text{def. of ordering}\\
                                   &= [[ ord varset in iP]],\\
                                   &\phantom{=} ~ [[(ord varset in iM \ {ord varset in iP}) ]]
                                   && \text{the induction hypothesis}\\
                                   &= [[ ord varset in iP → iM ]]
                                   && \text{def. of ordering}\\
                                   &= [[ ord varset in iN ]]
       \end{align*}
    \item The proofs of the positive cases are symmetric.
  \end{caseof}
\end{proof}

\lemOrdCompleteness*
\begin{proof}
  Mutual induction on $[[iN ≈ iM]]$ and $[[iP ≈ iQ]]$.
  Let us consider the rule inferring $[[iN ≈ iM]]$. 
  \begin{caseof}
    \item \ruleref{\ottdruleEOneNVarLabel}
    \item \ruleref{\ottdruleEOneShiftULabel}
    \item \ruleref{\ottdruleEOneArrowLabel}
      Then the equivalence has shape $[[iP → iN ≈ iQ → iM]]$,
      and by inversion, $[[iP ≈ iQ]]$ and $[[iN ≈ iM]]$.
      Then by the induction hypothesis,
      $[[ord varset in iP]] = [[ord varset in iQ]]$ 
      and $[[ord varset in iN]] = [[ord varset in iM]]$.
      Since the resulting ordering for $[[iP → iN]]$ and $[[iQ → iM]]$
      depend on the ordering of the corresponding components, 
      which are equal, the results are equal.
    \item \ruleref{\ottdruleEOneForallLabel}
      Then the equivalence has shape $[[∀pas.iN ≈ ∀pbs.iM]]$.
      and by inversion there exists 
      $[[mu : ({pbs} ∩ fv iM) ↔ ({pas} ∩ fv iN)]]$ such that
      \begin{itemize}
        \item $[[{pas} ∩ fv iM = ∅]]$ and 
        \item $[[iN ≈ [mu] iM]]$
      \end{itemize}

      Let us assume that $[[varset]]$ is disjoint from 
      $[[pas]]$ and $[[pbs]]$ 
      (we can always alpha-rename the bound variables).
      Then $[[ord varset in ∀pas.iN = ord varset in iN]]$, 
      $[[ord varset in ∀pas.iM = ord varset in iM]]$
      and by the induction hypothesis,
      $[[ord varset in iN]] = [[ord varset in [mu]iM]]$.
      This way, it suffices tho show  that 
      $[[ord varset in [mu]iM = ord varset in iM]]$.
      It holds by \cref{lemma:ord-sigma} since
      $[[varset]]$ is disjoint form 
      the domain and the codomain of 
      $[[mu : ({pbs} ∩ fv iM) ↔ ({pas} ∩ fv iN)]]$ 
      by assumption.

    \item The positive cases are proved symmetrically.
  \end{caseof}
\end{proof}
