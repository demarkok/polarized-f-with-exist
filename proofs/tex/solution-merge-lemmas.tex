\begin{lemma} \label{lemma:entry-weakening-preorder}
    Given a fixed context $[[Γ]]$, weakening forms a preorder on the set of entries well-formed in $[[Γ]]$.
\end{lemma}
\begin{proof} \hfill 
  \begin{itemize}
    \item Reflexivity: $[[Γ ⊢ usEntry ≈ usEntry]]$ then $[[Γ ⊢ usEntry ⇒ usEntry]]$.
    Let us consider the shape of $[[usEntry]]$. In all cases, there is a rule inferring
     $[[Γ ⊢ usEntry ⇒ usEntry]]$, since $[[Γ ⊢ iP ≈ iQ]]$ implies $[[Γ ⊢ iP ≥ iQ]]$ by inversion.
    \item Transitivity: $[[Γ ⊢ usEntry1 ⇒ usEntry2]]$ and $[[Γ ⊢ usEntry2 ⇒ usEntry3]]$ implies $[[Γ ⊢ usEntry1 ⇒ usEntry3]]$.
    It follows immediately by considering the rules inferring 
    $[[Γ ⊢ usEntry1 ⇒ usEntry2]]$ and $[[Γ ⊢ usEntry2 ⇒ usEntry3]]$, and applying the 
    transitivity of subtyping (\cref{corollary:subtyping-transitivity}).
  \end{itemize}
\end{proof}

\begin{lemma} [Solution Weakening forms a preorder]
    \label{lemma:solution-weakening-preorder}
    Let us consider a set of pairs $([[Θ]], [[us]])$ such that $[[us : Θ]]$.
    Then the relation defined as 
    $([[Θ1]], [[us1]]) \Rightarrow ([[Θ2]], [[us2]])$ iff $[[Θ1]] \supseteq [[Θ2]]$ 
    and $[[Θ2 ⊢ us1 ⇒ us2]]$ forms a preorder.
\end{lemma}
\begin{proof} \hfill
    \begin{itemize}
        \item Reflexivity: $([[Θ]], [[us]]) \Rightarrow ([[Θ]], [[us]])$.\\ It is clear that $[[Θ]] \supseteq [[Θ]]$.
        $[[Θ ⊢ us ⇒ us]]$ holds since for any $[[usEntry]]$ in $[[us]]$ restricting $[[α̂±]]$, $[[Θ(α̂±) ⊢ usEntry ⇒ usEntry]]$  
        by \cref{lemma:entry-weakening-preorder}
        \item Transitivity: If $([[Θ1]], [[us1]]) \Rightarrow ([[Θ2]], [[us2]])$ and $([[Θ2]], [[us2]]) \Rightarrow ([[Θ3]], [[us3]])$
        then $([[Θ1]], [[usEntry1]]) \Rightarrow ([[Θ3]], [[usEntry3]])$.\\
        It is clear that since $[[Θ1]] \subseteq [[Θ2]]$ and $[[Θ2]] \subseteq
        [[Θ3]]$ then $[[Θ1]] \subseteq [[Θ3]]$. Let us consider $[[usEntry3]]
        \in [[us3]]$. Then there exists $[[usEntry2]] \in [[us2]]$ such that
        $[[Θ3 ⊢ usEntry2 ⇒ usEntry3]]$. Also there exists $[[usEntry1]] \in [[us1]]$ 
        such that $[[Θ2 ⊢ usEntry1 ⇒ usEntry2]]$, and by weakening,
        $[[Θ3 ⊢ usEntry1 ⇒ usEntry2]]$. Then by transitivity of solution entry
        weakening (\cref{lemma:entry-weakening-preorder}), $[[Θ3 ⊢ usEntry1 ⇒
        usEntry3]]$.
    \end{itemize}
\end{proof}

\begin{corollary} [Solution Weakening is transitive]
    \label{lemma:weakening-transitivity}
    If $[[Θ ⊢ us1 ⇒ us2]]$ and $[[Θ ⊢ us2 ⇒ us3]]$ then $[[Θ ⊢ us1 ⇒ us3]]$.
\end{corollary}


\begin{lemma} [Soundness of Merge of Unification Solution Entries]
    \label{lemma:unif-sol-ent-merge-soundness}
    For a fixed context $[[Γ]]$,
    suppose that  $[[Γ ⊢ usEntry1]]$ and $[[Γ ⊢ usEntry2]]$,
    where $[[Γ ⊢ usEntryi]]$ is an equivalence-shaped restriction
    (i.e. it has shape $[[α̂⁺ :≈ iP]]$ or $[[α̂⁻ :≈ iN]]$ but not $[[α̂⁺ :≥ iP]]$).
    If $[[usEntry1 & usEntry2]]$ is defined then
    \begin{enumerate}
    \item $[[Γ ⊢ usEntry1 & usEntry2]]$
    \item $[[Γ ⊢ usEntry1 & usEntry2 ⇒ usEntry1]]$
    \item $[[Γ ⊢ usEntry1 & usEntry2 ⇒ usEntry2]]$
    \end{enumerate}
\end{lemma}
\begin{proof}
    Let us consider the rule forming $[[Γ ⊢ usEntry1 & usEntry2 = usEntry]]$.
    \begin{caseof}
        \item \ruleref{\ottdruleSMEPEqEqLabel}, i.e. 
        $[[Γ ⊢ usEntry1 & usEntry2 = usEntry]]$
        has form $[[Γ ⊢ (pua :≈ iP) & (pua :≈ iP') = (pua :≈ iP)]]$
        and $[[nf(iP) = nf(iP')]]$. Then
         \begin{enumerate}
            \item $[[Γ ⊢ usEntry]]$, i.e. $[[Γ ⊢ pua :≈ iP]]$ holds by assumption;
            \item $[[Γ ⊢ (pua :≈ iP) ⇒ (pua :≈ iP)]]$ holds by reflexivity (\cref{lemma:entry-weakening-preorder});
            \item $[[Γ ⊢ (pua :≈ iP) ⇒ (pua :≈ iP')]]$ holds
            because $[[nf(iP) = nf(iP')]]$ implies $[[Γ ⊢ iP ≈ iP']]$
            (by \cref{lemma:subt-equiv-algorithmization}), and thus, 
            \ruleref{\ottdruleSImpEPEqEqLabel} applies.
         \end{enumerate}
        \item \ruleref{\ottdruleSMENEqEqLabel}. The negative case is proved in exactly the same way.
    \end{caseof}
\end{proof}

\begin{lemma} [Soundness of Merge of Unification Solutions]
    \label{lemma:unif-sol-merge-soundness}
    Suppose that $[[us1 : Θ | varset1]]$ and $[[us2 : Θ | varset2]]$ 
    are unification solutions (i.e. $[[us1]]$ and $[[us2]]$ can only have equivalence-shaped restrictions).
    If $[[us1 & us2]]$ is defined then
    \begin{enumerate}
        \item $[[us1 & us2 : Θ|varset1 ∪ varset2]]$,
        \item $[[Θ|varset1 ⊢ us1 & us2 ⇒ us1]]$, and
        \item $[[Θ|varset2 ⊢ us1 & us2 ⇒ us2]]$.
    \end{enumerate}
\end{lemma}
\begin{proof}
    Let us prove the properties separately:
    \begin{enumerate}
        \item $[[us1 & us2 : Θ|varset1 ∪ varset2]]$.
        It suffices to prove the following two properties:
        \begin{itemize}
            \item The set of variables of the entries of $[[us1 & us2]]$ 
            coincides with $[[varset1 ∪ varset2]]$\\
            By definition, $[[us1 & us2]]$ consists of three parts:
            entries of $[[us1]]$ that do not have matching entries of $[[us2]]$,
            entries of $[[us2]]$ that do not have matching entries of $[[us1]]$,
            and the merge of matching entries.
            It means that $[[dom(us1 & us2)]] = [[dom(us1) \ dom(us2) ∪ dom(us2) \ dom(us1) ∪ 
            dom(us1) ∩ dom(us2)]] = [[dom(us1) ∪ dom(us2)]]$, which, since 
            $[[usi : Θ | varseti]]$, is equal to $[[varset1 ∪ varset2]]$. 

            \item Each entry of $[[us1 & us2]]$ restricting $[[α̂±]]$ is well-formed in
            the corresponding context $[[Θ(α̂±)]]$.\\
            Let us consider an arbitrary entry $[[usEntry]]$ of $[[us1 & us2]]$ restricting
            $[[α̂±]]$. Then there are three cases:
            \begin{enumerate}
                \item $[[usEntry]]$ the entry is from $[[us1]]$ and does not have a matching entry 
                in $[[us2]]$, i.e. $[[α̂±]] \in [[dom(us1) \ dom(us2)]]$.
                Then $[[usEntry]]$ is well-formed in $[[Θ | varset1]]$ by assumption.
                \item $[[usEntry]]$ the entry is from $[[us2]]$ and does not
                have a matching entry in $[[us1]]$. This case is symmetric.
                \item $[[usEntry]]$ is the merge of two matching entries $[[usEntry1]] \in [[us1]]$
                 and $[[usEntry2]] \in [[us2]]$ restricting $[[α̂±]]$.
                 Since $[[us1 : Θ | varset1]]$ and $[[us2 : Θ | varset2]]$,
                 $[[α̂±]] \in [[dom(Θ|varset1) ∩ dom(Θ|varset2)]]$, i.e. there 
                 is an entry $[[ α̂±[Γ] ]] \in [[Θ]]$, and 
                 $[[usEntry1]]$ and $[[usEntry2]]$ are well-formed in $[[Γ]]$.
                 Then by \cref{lemma:unif-sol-ent-merge-soundness}, 
                 $[[Γ ⊢ usEntry1 & usEntry2]]$, where $[[Γ]] = [[Θ(α̂±)]]$.
            \end{enumerate}
        \end{itemize}
        \item $[[Θ|varset1 ⊢ us1 & us2 ⇒ us1]]$
        We need to show that for every entry $[[usEntry]]$ from $[[us1]]$,
        there is an entry $[[usEntry']]$ from $[[us1 & us2]]$ such that
        $[[Θ|varset1 ⊢ usEntry' ⇒ usEntry]]$. 
        Let us consider an arbitrary $[[usEntry]]$ from $[[us1]]$ restricting $[[α̂±]]$.
        Then there are two cases:
        \begin{itemize}
            \item $[[usEntry]]$ does not have a matching entry in $[[us2]]$.
            Then $[[usEntry]]$ is also in $[[us1 & us2]]$ and $[[Θ|varset1 ⊢ usEntry ⇒ usEntry]]$ 
            by reflexivity (\cref{lemma:entry-weakening-preorder}).
            \item $[[usEntry]]$ has a matching entry $[[usEntry']]$ in $[[us2]]$.
            Then $[[usEntry & usEntry']]$ is in $[[us1 & us2]]$ and $[[Θ|varset1 ⊢ usEntry & usEntry' ⇒ usEntry]]$ by \cref{lemma:unif-sol-ent-merge-soundness}.
        \end{itemize}
        \item $[[Θ|varset2 ⊢ us1 & us2 ⇒ us2]]$ is proved analogously.
    \end{enumerate}
\end{proof}

\begin{lemma} [Soundness of Solution Entry Merge]
\label{lemma:entry-merge-soundness}
For a fixed context $[[Γ]]$,
suppose that  $[[Γ ⊢ usEntry1]]$ and $[[Γ ⊢ usEntry2]]$. 
If $[[usEntry1 & usEntry2]]$ is defined then
\begin{enumerate}
    \item $[[Γ ⊢ usEntry1 & usEntry2]]$
    \item $[[Γ ⊢ usEntry1 & usEntry2 ⇒ usEntry1]]$
    \item $[[Γ ⊢ usEntry1 & usEntry2 ⇒ usEntry2]]$
\end{enumerate}
\end{lemma}
\begin{proof}
    Let us consider the rule forming $[[Γ ⊢ usEntry1 & usEntry2 = usEntry]]$.
    \begin{caseof}
        \item \ruleref{\ottdruleSMEPEqEqLabel} or \ruleref{\ottdruleSMENEqEqLabel}
        The proof is analogous to the proof of \cref{lemma:unif-sol-ent-merge-soundness}.
        \item \ruleref{\ottdruleSMESupSupLabel} 
        Then $[[usEntry1]]$ is $[[pua :≥ iP1]]$, $[[usEntry2]]$ is $[[pua :≥ iP2]]$,
        and $[[usEntry1 & usEntry2]]$ is $[[pua :≥ iQ]]$ where $[[iQ]]$ is the least upper bound of $[[iP1]]$ and $[[iP2]]$.
        Then by \cref{lemma:lub-soundness},
        \begin{itemize}
            \item $[[Γ ⊢ iQ]]$, and hence $[[Γ ⊢ pua :≥ iQ]]$, that is $[[Γ ⊢ usEntry1 & usEntry2]]$,
            \item $[[Γ ⊢ iQ ≥ iP1]]$, and hence $[[Γ ⊢ pua :≥ iQ ⇒ pua :≥ iP1]]$, that is $[[Γ ⊢ usEntry1 & usEntry2 ⇒ usEntry1]]$,
            \item $[[Γ ⊢ iQ ≥ iQ1]]$, and hence $[[Γ ⊢ usEntry1 & usEntry2 ⇒ usEntry2]]$.
        \end{itemize}
        \item \ruleref{\ottdruleSMESupEqLabel}
        Then $[[usEntry1]]$ is $[[pua :≥ iP]]$, $[[usEntry2]]$ is $[[pua :≈ iQ]]$, 
        where $[[Γ;· ⊨ uQ ≥ iP ⫤ us']]$, and the resulting   
        $[[usEntry1 & usEntry2]]$ is equal to $[[usEntry2]]$, that is $[[pua :≈ iQ]]$.
    
        \begin{itemize}
            \item By assumption, $[[Γ ⊢ iQ]]$, and hence $[[Γ ⊢ pua :≈ iQ]]$, that is $[[Γ ⊢ usEntry1 & usEntry2]]$.
            \item Since $[[uv(uQ) = ∅]]$, 
                $[[Γ;· ⊨ uQ ≥ iP ⫤ us']]$ implies $[[Γ ⊢ iQ ≥ iP]]$
                by the soundness of positive subtyping (\cref{lemma:pos-subt-soundness}), 
                which means $[[Γ ⊢ pua :≈ iQ ⇒ pua :≥ iP]]$, that is $[[Γ ⊢ usEntry1 & usEntry2 ⇒ usEntry1]]$.
            \item  $[[Γ ⊢ pua :≈ iQ ⇒ pua :≈ iQ]]$ by reflexivity
             (\cref{lemma:entry-weakening-preorder}), and hence
              $[[Γ ⊢ usEntry1 & usEntry2 ⇒ usEntry2]]$.
        \end{itemize}
        \item \ruleref{\ottdruleSMEEqSupLabel} Thee proof is analogous to the previous case.
    \end{caseof}
\end{proof}

\begin{lemma} [Soundness of Solution Merge] \label{lemma:merge-soundness}
    Suppose that $[[us1 : Θ1]]$ and $[[us2 : Θ2]]$ 
    and $[[us1 & us2]]$ is defined.
    Then 
    \begin{enumerate}
        \item $[[us1 & us2 : Θ1 ∪ Θ2]]$,
        \item $[[Θ1 ⊢ us1 & us2 ⇒ us1]]$, and
        \item $[[Θ2 ⊢ us1 & us2 ⇒ us2]]$.
    \end{enumerate}
\end{lemma}
\begin{proof}
    The proof repeats the proof of \cref{lemma:unif-sol-merge-soundness},
    but uses \cref{lemma:entry-merge-soundness} instead of 
    \cref{lemma:unif-sol-ent-merge-soundness}.
\end{proof}


\begin{lemma} [Completeness and Initiality of Solution Merge] 
    \label{lemma:merge-completeness}
    Suppose that $[[us1 : Θ1]]$ and $[[us2 : Θ2]]$
    and there exists $[[us]]$
    such that $[[Θ1 ⊢ us1 ⇒ us]]$ and $[[ Θ2 ⊢ us2 ⇒ us]]$.
    Then $[[us1 & us2]]$ is defined and 
    $[[ Θ1 ∪ Θ2 ⊢ us1 & us2 ⇒ us]]$.
\end{lemma}

\begin{lemma} [Solution Weakening is Monotonous]
    \label{lemma:weakening-monotonicity}
    if $[[us1 : Θ]]$ and $[[us2 : Θ]]$ and 
    $[[us1 ⊆ us2]]$ then $[[Θ ⊢ us2 ⇒ us1]]$.
\end{lemma}
