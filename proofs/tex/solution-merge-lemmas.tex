\begin{lemma} \label{lemma:entry-weakening-preorder}
    Given a fixed context $[[Γ]]$, weakening forms a preorder on the set of entries well-formed in $[[Γ]]$.
\end{lemma}
\begin{proof} \hfill 
  \begin{itemize}
    \item Reflexivity: $[[Γ ⊢ usEntry ≈ usEntry]]$ then $[[Γ ⊢ usEntry ⇒ usEntry]]$.
    Let us consider the shape of $[[usEntry]]$. In all cases, there is a rule inferring
     $[[Γ ⊢ usEntry ⇒ usEntry]]$, since $[[Γ ⊢ iP ≈ iQ]]$ implies $[[Γ ⊢ iP ≥ iQ]]$ by inversion.
    \item Transitivity: $[[Γ ⊢ usEntry1 ⇒ usEntry2]]$ and $[[Γ ⊢ usEntry2 ⇒ usEntry3]]$ implies $[[Γ ⊢ usEntry1 ⇒ usEntry3]]$.
    It follows immediately by considering the rules inferring 
    $[[Γ ⊢ usEntry1 ⇒ usEntry2]]$ and $[[Γ ⊢ usEntry2 ⇒ usEntry3]]$, and applying the 
    transitivity of subtyping (\cref{corollary:subtyping-transitivity}).
  \end{itemize}
\end{proof}

\begin{lemma} [Solution Weakening forms a preorder]
    \label{lemma:solution-weakening-preorder}
    Let us consider a set of pairs $([[Θ]], [[us]])$ such that $[[us : Θ]]$.
    Then the relation defined as 
    $([[Θ1]], [[us1]]) \Rightarrow ([[Θ2]], [[us2]])$ iff $[[Θ1]] \supseteq [[Θ2]]$ 
    and $[[Θ2 ⊢ us1 ⇒ us2]]$ forms a preorder.
\end{lemma}
\begin{proof} \hfill
    \begin{itemize}
        \item Reflexivity: $([[Θ]], [[us]]) \Rightarrow ([[Θ]], [[us]])$.\\ It is clear that $[[Θ]] \supseteq [[Θ]]$.
        $[[Θ ⊢ us ⇒ us]]$ holds since for any $[[usEntry]]$ in $[[us]]$ restricting $[[α̂±]]$, $[[Θ(α̂±) ⊢ usEntry ⇒ usEntry]]$  
        by \cref{lemma:entry-weakening-preorder}
        \item Transitivity: If $([[Θ1]], [[us1]]) \Rightarrow ([[Θ2]], [[us2]])$ and $([[Θ2]], [[us2]]) \Rightarrow ([[Θ3]], [[us3]])$
        then $([[Θ1]], [[usEntry1]]) \Rightarrow ([[Θ3]], [[usEntry3]])$.\\
        It is clear that since $[[Θ1]] \subseteq [[Θ2]]$ and $[[Θ2]] \subseteq
        [[Θ3]]$ then $[[Θ1]] \subseteq [[Θ3]]$. Let us consider $[[usEntry3]]
        \in [[us3]]$. Then there exists $[[usEntry2]] \in [[us2]]$ such that
        $[[Θ3 ⊢ usEntry2 ⇒ usEntry3]]$. Also there exists $[[usEntry1]] \in [[us1]]$ 
        such that $[[Θ2 ⊢ usEntry1 ⇒ usEntry2]]$, and by weakening,
        $[[Θ3 ⊢ usEntry1 ⇒ usEntry2]]$. Then by transitivity of solution entry
        weakening (\cref{lemma:entry-weakening-preorder}), $[[Θ3 ⊢ usEntry1 ⇒
        usEntry3]]$.
    \end{itemize}
\end{proof}

\begin{corollary} [Solution Weakening is transitive]
    \label{lemma:weakening-transitivity}
    If $[[Θ ⊢ us1 ⇒ us2]]$ and $[[Θ ⊢ us2 ⇒ us3]]$ then $[[Θ ⊢ us1 ⇒ us3]]$.
\end{corollary}

\begin{lemma} [Solution Weakening is Monotonous]
    \label{lemma:weakening-monotonicity}
    if $[[us1 : Θ1]]$, $[[us2 : Θ2]]$,
    and $[[us1 ⊆ us2]]$ then $[[Θ1 ⊢ us2 ⇒ us1]]$.
\end{lemma}
\begin{proof}
    We need to show that for any $[[usEntry]] \in [[us1]]$,
    there exists $[[usEntry']] \in [[us2]]$ such that 
    $[[Θ1 ⊢ usEntry' ⇒ usEntry]]$.
    Let us consider an arbitrary $[[usEntry]] \in [[us1]] \subseteq [[us2]]$
    restricting $[[α̂±]]$. 
    Notice that $[[Θ1(α̂±) ⊢ usEntry ]]$ since 
    $[[usEntry]] \in [[us1 : Θ1]]$.
    By \cref{lemma:entry-weakening-preorder}, $[[Θ1 ⊢ usEntry ⇒ usEntry]]$. 
\end{proof}

\begin{lemma} [Soundness of Solution Entry Merge]
\label{lemma:entry-merge-soundness}
For a fixed context $[[Γ]]$,
suppose that  $[[Γ ⊢ usEntry1]]$ and $[[Γ ⊢ usEntry2]]$. 
If $[[usEntry1 & usEntry2]]$ is defined then
\begin{enumerate}
    \item $[[Γ ⊢ usEntry1 & usEntry2]]$
    \item $[[Γ ⊢ usEntry1 & usEntry2 ⇒ usEntry1]]$
    \item $[[Γ ⊢ usEntry1 & usEntry2 ⇒ usEntry2]]$
\end{enumerate}
\end{lemma}
\begin{proof}
    Let us consider the rule forming $[[Γ ⊢ usEntry1 & usEntry2 = usEntry]]$.
    \begin{caseof}
        \item \ruleref{\ottdruleSMEPEqEqLabel}, i.e. 
        $[[Γ ⊢ usEntry1 & usEntry2 = usEntry]]$
        has form $[[Γ ⊢ (pua :≈ iP) & (pua :≈ iP') = (pua :≈ iP)]]$
        and $[[nf(iP) = nf(iP')]]$. Then
         \begin{enumerate}
            \item $[[Γ ⊢ usEntry]]$, i.e. $[[Γ ⊢ pua :≈ iP]]$ holds by assumption;
            \item $[[Γ ⊢ (pua :≈ iP) ⇒ (pua :≈ iP)]]$ holds by reflexivity (\cref{lemma:entry-weakening-preorder});
            \item $[[Γ ⊢ (pua :≈ iP) ⇒ (pua :≈ iP')]]$ holds
            because $[[nf(iP) = nf(iP')]]$ implies $[[Γ ⊢ iP ≈ iP']]$
            (by \cref{lemma:subt-equiv-algorithmization}), and thus, 
            \ruleref{\ottdruleSImpEPEqEqLabel} applies.
         \end{enumerate}
        \item \ruleref{\ottdruleSMENEqEqLabel} the negative case is proved in exactly the same way as the positive one.
        \item \ruleref{\ottdruleSMESupSupLabel} 
        Then $[[usEntry1]]$ is $[[pua :≥ iP1]]$, $[[usEntry2]]$ is $[[pua :≥ iP2]]$,
        and $[[usEntry1 & usEntry2]]$ is $[[pua :≥ iQ]]$ where $[[iQ]]$ is the least upper bound of $[[iP1]]$ and $[[iP2]]$.
        Then by \cref{lemma:lub-soundness},
        \begin{itemize}
            \item $[[Γ ⊢ iQ]]$, and hence $[[Γ ⊢ pua :≥ iQ]]$, that is $[[Γ ⊢ usEntry1 & usEntry2]]$,
            \item $[[Γ ⊢ iQ ≥ iP1]]$, and hence $[[Γ ⊢ pua :≥ iQ ⇒ pua :≥ iP1]]$, that is $[[Γ ⊢ usEntry1 & usEntry2 ⇒ usEntry1]]$,
            \item $[[Γ ⊢ iQ ≥ iQ1]]$, and hence $[[Γ ⊢ usEntry1 & usEntry2 ⇒ usEntry2]]$.
        \end{itemize}
        \item \ruleref{\ottdruleSMESupEqLabel}
        Then $[[usEntry1]]$ is $[[pua :≥ iP]]$, $[[usEntry2]]$ is $[[pua :≈ iQ]]$, 
        where $[[Γ;· ⊨ uQ ≥ iP ⫤ us']]$, and the resulting   
        $[[usEntry1 & usEntry2]]$ is equal to $[[usEntry2]]$, that is $[[pua :≈ iQ]]$.
    
        \begin{itemize}
            \item By assumption, $[[Γ ⊢ iQ]]$, and hence $[[Γ ⊢ pua :≈ iQ]]$, that is $[[Γ ⊢ usEntry1 & usEntry2]]$.
            \item Since $[[uv(uQ) = ∅]]$, 
                $[[Γ;· ⊨ uQ ≥ iP ⫤ us']]$ implies $[[Γ ⊢ iQ ≥ iP]]$
                by the soundness of positive subtyping (\cref{lemma:pos-subt-soundness}), 
                which means $[[Γ ⊢ pua :≈ iQ ⇒ pua :≥ iP]]$, that is $[[Γ ⊢ usEntry1 & usEntry2 ⇒ usEntry1]]$.
            \item  $[[Γ ⊢ pua :≈ iQ ⇒ pua :≈ iQ]]$ by reflexivity
             (\cref{lemma:entry-weakening-preorder}), and hence
              $[[Γ ⊢ usEntry1 & usEntry2 ⇒ usEntry2]]$.
        \end{itemize}
        \item \ruleref{\ottdruleSMEEqSupLabel} Thee proof is analogous to the previous case.
    \end{caseof}
\end{proof}

\begin{lemma} [Soundness of Solution Merge] \label{lemma:merge-soundness}
    Suppose that $[[us1 : Θ1]]$ and $[[us2 : Θ2]]$ 
    and $[[us1 & us2]]$ is defined.
    Then 
    \begin{enumerate}
        \item $[[us1 & us2 : Θ1 ∪ Θ2]]$,
        \item $[[Θ1 ⊢ us1 & us2 ⇒ us1]]$, and
        \item $[[Θ2 ⊢ us1 & us2 ⇒ us2]]$.
    \end{enumerate}
\end{lemma}
\begin{proof}

    Let us prove the properties separately:
    \begin{enumerate}
        \item $[[us1 & us2 : Θ|varset1 ∪ varset2]]$.
        It suffices to prove the following two properties:
        \begin{itemize}
            \item The set of variables of the entries of $[[us1 & us2]]$ 
            coincides with $[[varset1 ∪ varset2]]$\\
            By definition, $[[us1 & us2]]$ consists of three parts:
            entries of $[[us1]]$ that do not have matching entries of $[[us2]]$,
            entries of $[[us2]]$ that do not have matching entries of $[[us1]]$,
            and the merge of matching entries.
            It means that $[[dom(us1 & us2)]] = [[dom(us1) \ dom(us2) ∪ dom(us2) \ dom(us1) ∪ 
            dom(us1) ∩ dom(us2)]] = [[dom(us1) ∪ dom(us2)]]$, which, since 
            $[[usi : Θ | varseti]]$, is equal to $[[varset1 ∪ varset2]]$. 

            \item Each entry of $[[us1 & us2]]$ restricting $[[α̂±]]$ is well-formed in
            the corresponding context $[[Θ(α̂±)]]$.\\
            Let us consider an arbitrary entry $[[usEntry]]$ of $[[us1 & us2]]$ restricting
            $[[α̂±]]$. Then there are three cases:
            \begin{enumerate}
                \item $[[usEntry]]$ the entry is from $[[us1]]$ and does not have a matching entry 
                in $[[us2]]$, i.e. $[[α̂±]] \in [[dom(us1) \ dom(us2)]]$.
                Then $[[usEntry]]$ is well-formed in $[[Θ | varset1]]$ by assumption.
                \item $[[usEntry]]$ the entry is from $[[us2]]$ and does not
                have a matching entry in $[[us1]]$. This case is symmetric.
                \item $[[usEntry]]$ is the merge of two matching entries $[[usEntry1]] \in [[us1]]$
                 and $[[usEntry2]] \in [[us2]]$ restricting $[[α̂±]]$.
                 Since $[[us1 : Θ | varset1]]$ and $[[us2 : Θ | varset2]]$,
                 $[[α̂±]] \in [[dom(Θ|varset1) ∩ dom(Θ|varset2)]]$, i.e. there 
                 is an entry $[[ α̂±[Γ] ]] \in [[Θ]]$, and 
                 $[[usEntry1]]$ and $[[usEntry2]]$ are well-formed in $[[Γ]]$.
                 Then by \cref{lemma:entry-merge-soundness}, 
                 $[[Γ ⊢ usEntry1 & usEntry2]]$, where $[[Γ]] = [[Θ(α̂±)]]$.
            \end{enumerate}
        \end{itemize}
        \item $[[Θ|varset1 ⊢ us1 & us2 ⇒ us1]]$
        We need to show that for every entry $[[usEntry]]$ from $[[us1]]$,
        there is an entry $[[usEntry']]$ from $[[us1 & us2]]$ such that
        $[[Θ|varset1 ⊢ usEntry' ⇒ usEntry]]$. 
        Let us consider an arbitrary $[[usEntry]]$ from $[[us1]]$ restricting $[[α̂±]]$.
        Then there are two cases:
        \begin{itemize}
            \item $[[usEntry]]$ does not have a matching entry in $[[us2]]$.
            Then $[[usEntry]]$ is also in $[[us1 & us2]]$ and $[[Θ|varset1 ⊢ usEntry ⇒ usEntry]]$ 
            by reflexivity (\cref{lemma:entry-weakening-preorder}).
            \item $[[usEntry]]$ has a matching entry $[[usEntry']]$ in $[[us2]]$.
            Then $[[usEntry & usEntry']]$ is in $[[us1 & us2]]$ and $[[Θ|varset1 ⊢ usEntry & usEntry' ⇒ usEntry]]$ 
            by \cref{lemma:entry-merge-soundness}.
        \end{itemize}
        \item $[[Θ|varset2 ⊢ us1 & us2 ⇒ us2]]$ is proved analogously.
    \end{enumerate}
\end{proof}


\begin{lemma} [Completeness of Solution Entry Merge]
    \label{lemma:entry-merge-completeness}
    For a fixed context $[[Γ]]$,
    suppose that  $[[Γ ⊢ usEntry1]]$ and $[[Γ ⊢ usEntry2]]$.
    Suppose there exists an entry $[[Γ ⊢ usEntry]]$ such that $[[Γ ⊢ usEntry ⇒ usEntry1]]$ and $[[Γ ⊢ usEntry ⇒ usEntry2]]$.
    Then $[[usEntry1 & usEntry2]]$ is defined and
    $[[Γ ⊢ usEntry ⇒ usEntry1 & usEntry2]]$.
\end{lemma}
\begin{proof}

    Let us consider the shape of $[[usEntry]]$.
    \begin{caseof}

        \item $[[usEntry]]$ is $[[nua :≈ iM]]$
        Then since $[[usEntry1]]$, $[[usEntry2]]$, and $[[usEntry]]$
        restrict the same variable, they have the same polarity. 
        So $[[usEntry1]]$ is $[[nua :≈ iN1]]$, $[[usEntry2]]$ is $[[nua :≈ iN2]]$, 
        Then by inversion of \ruleref{\ottdruleSImpENEqEqLabel},
        \begin{itemize}
            \item $[[Γ ⊢ iM ≈ iN1]]$, and hence $[[nf(iM) = nf(iN1)]]$ by \cref{lemma:subt-equiv-algorithmization},
            \item $[[Γ ⊢ iM ≈ iN2]]$, and hence $[[nf(iM) = nf(iN2)]]$ by \cref{lemma:subt-equiv-algorithmization},
        \end{itemize}
        which implies $[[nf(iN1) = nf(iN2)]]$.
        It means that the merge of $[[usEntry1]]$ and $[[usEntry2]]$ is defined by \ruleref{\ottdruleSMENEqEqLabel}:
        $[[Γ ⊢ (nua :≈ iN1) & (nua :≈ iN2) = (nua :≈ iN1)]]$.

        To show that $[[Γ ⊢ (nua :≈ iM) ⇒ (nua :≈ iN1)]]$, we apply \ruleref{\ottdruleSImpENEqEqLabel}:
        $[[Γ ⊢ iN1 ≈ iM]]$ is obtained from $[[Γ ⊢ iM ≈ iN1]]$ by symmetry of equivalence.

        \item $[[usEntry]]$ is $[[pua :≥ iP]]$.
        \label{case:entry-merge-completeness-sup-sup}
         Then the only rule inferring 
        $[[Γ ⊢ usEntry ⇒ usEntryi]]$ is \ruleref{\ottdruleSImpESupSupLabel}.
        This way, $[[usEntry1]]$ is $[[pua :≥ iP1]]$ and $[[usEntry2]]$ is $[[pua :≥ iP2]]$,
        and $[[iP]]$ is a common supertype of $[[iP1]]$ and $[[iP2]]$, i.e. $[[Γ⊢ iP ≥ iP1]]$ and $[[Γ ⊢ iP ≥ iP2]]$.
        Then by \cref{lemma:lub-completeness}, $[[iP1 ∨ iP2]]$ is defined and $[[Γ ⊢ iP ≥ iP1 ∨ iP2]]$.
        Then $[[usEntry1 & usEntry2]]$, that is $[[pua :≥ iP1 & pua :≥ iP1]]$, is defined as 
        $[[pua :≥ iP1 ∨ iP2]]$ by \ruleref{\ottdruleSMESupSupLabel}.
        Moreover, $[[Γ ⊢ usEntry ⇒ usEntry1 & usEntry2]]$, that is $[[Γ ⊢ pua :≥ iP ⇒ pua :≥ iP1 ∨ iP2]]$,
        holds by \ruleref{\ottdruleSImpESupSupLabel}.

        \item $[[usEntry]]$ is $[[pua :≈ iP]]$. Then let us consider the rules inferring
        $[[Γ ⊢ usEntry ⇒ usEntry1]]$ and $[[Γ ⊢ usEntry ⇒ usEntry2]]$ respectively:
        \begin{caseof}
            \item \ruleref{\ottdruleSImpEPEqEqLabel} and \ruleref{\ottdruleSImpEPEqEqLabel}.
            Then $[[usEntry1]]$ is $[[pua :≈ iP1]]$, $[[usEntry2]]$ is $[[pua :≈ iP2]]$, 
            where $[[Γ ⊢ iP ≈ iP1]]$ and $[[Γ ⊢ iP ≈ iP2]]$.
            Then $[[usEntry1 & usEntry2]]$ is defined as $[[pua :≈ iP1]]$ by \ruleref{\ottdruleSMEPEqEqLabel}, 
            and $[[Γ ⊢ pua :≈ iP ⇒ pua :≈ iP1]]$ holds by \ruleref{\ottdruleSImpEPEqEqLabel},
            since $[[Γ ⊢ iP ≈ iP1]]$.

            \item \ruleref{\ottdruleSImpEPEqEqLabel} and \ruleref{\ottdruleSImpEEqSupLabel}.
            Then $[[usEntry1]]$ is $[[pua :≈ iP1]]$, $[[usEntry2]]$ is $[[pua :≥ iP2]]$, 
            where $[[Γ ⊢ iP ≈ iP1]]$ and $[[Γ ⊢ iP ≥ iP2]]$.

            Then by transitivity of subtyping (\cref{corollary:subtyping-transitivity}),
            $[[Γ ⊢ iP1 ≥ iP2]]$. The completeness of positive subtyping algorithm 
            (\cref{lemma:pos-subt-completeness}) implies that
            $[[Γ;· ⊨ uP1 ≥ iP2 ⫤ us']]$.
            Then $[[usEntry1 & usEntry2]]$ is defined as $[[pua :≈ iP1]]$ by \ruleref{\ottdruleSMEPEqEqLabel}.
            Moreover, $[[Γ ⊢ pua :≈ iP ⇒ pua :≈ iP1]]$ holds by \ruleref{\ottdruleSImpEPEqEqLabel}, 
            as required.

            \item \ruleref{\ottdruleSImpEEqSupLabel} and \ruleref{\ottdruleSImpEPEqEqLabel}.
            This case is symmetric to the previous one.

            \item \ruleref{\ottdruleSImpEEqSupLabel} and \ruleref{\ottdruleSImpEEqSupLabel}.
            Then $[[usEntry1]]$ is $[[pua :≥ iP1]]$, $[[usEntry2]]$ is $[[pua :≥ iP2]]$, 
            where $[[Γ ⊢ iP ≥ iP1]]$ and $[[Γ ⊢ iP ≥ iP2]]$.
            And similarly to \cref{case:entry-merge-completeness-sup-sup},
            $[[usEntry1 & usEntry2]]$ is defined as $[[pua :≥ iP1 ∨ iP2]]$.
            Moreover, $[[Γ ⊢ pua :≈ iP ⇒ pua :≥ iP1 ∨ iP2]]$ holds by \ruleref{\ottdruleSImpEEqSupLabel},
            since $[[Γ ⊢ iP ≥ iP1 ∨ iP2]]$ by completeness of the least upper bound algorithm (\cref{lemma:lub-completeness}).
        \end{caseof}
    \end{caseof}
\end{proof}

\begin{lemma} [Completeness of Solution Merge] 
    \label{lemma:merge-completeness}
    Suppose that $[[us1 : Θ|varset1]]$ and $[[us2 : Θ|varset2]]$
    and there exists $[[us]]$
    such that $[[Θ ⊢ us ⇒ us1]]$ and $[[Θ ⊢ us ⇒ us2]]$.
    Then $[[us1 & us2]]$ is defined and 
    $[[ Θ ⊢ us ⇒ us1 & us2]]$.
\end{lemma}
\begin{proof}
    By  definition, $[[us1 & us2]]$ is a union of entries of $[[us1]]$, which do not have matching entries in $[[us2]]$,
    entries of $[[us2]]$, which do not have matching entries in $[[us1]]$, and the merge of matching entries.
    It is clear that the first two components of this union exist. Let us show that the third component exists, 
    i.e. that for every two entries $[[usEntry1]] \in [[us1]]$ and $[[usEntry2]] \in [[us2]]$
    restricting the same variable $[[α̂±]]$, $[[usEntry1 & usEntry2]]$ is defined.

    $[[Θ|varset1 ⊢ us ⇒ us1]]$ means that there exists $[[usEntry]] \in [[us]]$ restricting $[[α̂±]]$ such that
    $[[Θ(α̂±) ⊢ usEntry ⇒ usEntry1]]$. Similarly, since there can be only one entry in $[[usEntry]]$ restricting 
    the same variable$[[α̂±]]$, $[[Θ2 ⊢ us ⇒ us2]]$ means $[[Θ(α̂±) ⊢ usEntry ⇒ usEntry2]]$.
    Then by \cref{lemma:entry-merge-completeness}, $[[usEntry1 & usEntry2]]$ is defined and 
    $[[Θ(α̂±) ⊢ usEntry ⇒ usEntry1 & usEntry2]]$.

    Let us show: $[[ Θ ⊢ us ⇒ us1 & us2 ]]$, i.e. for every $[[usEntry']] \in [[us1 & us2]]$ restricting 
    $[[α̂±]]$, there exists $[[usEntry]] \in [[us]]$ such that $[[Θ(α̂±) ⊢ usEntry ⇒ usEntry']]$.
    Let us consider three cases:
    \begin{caseof}
        \item $[[usEntry']]$ is from $[[us1]]$ and does not have a matching entry in $[[us2]]$.
        Then $[[Θ|varset1 ⊢ us ⇒ us1]]$ means that there exists $[[usEntry]] \in [[us]]$ such that
        $[[Θ(α̂±) ⊢ usEntry ⇒ usEntry']]$, as required.
        \item $[[usEntry']]$ is from $[[us2]]$ and does not have a matching entry in $[[us1]]$.
        This case is symmetric to the previous one.
        \item $[[usEntry']]$ is the merge of two matching entries $[[usEntry1]] \in [[us1]]$
        and $[[usEntry2]] \in [[us2]]$ restricting $[[α̂±]]$.
        Then as noted previously, $[[Θ(α̂±) ⊢ usEntry ⇒ usEntry1 & usEntry2]] = [[usEntry']]$.
    \end{caseof}
\end{proof}
