\begin{lemma}
  Informally, this lemma says that if $[[Γ, Ord varset1 ⊢ [σ21]iP ≥ iQ]]$
  and $[[Γ, Ord varset2 ⊢ [σ12]iQ ≥ iP]]$ holds for substitutions $[[σ12]]$
  and $[[σ21]]$ then these substitutions are in fact mutually inverse bijections
  between variables $[[varset1]]$ and $[[varset1]]$.

  \begin{itemize}
  \item[$+$]
    For $[[Γ, Ord varset2 ⊢ iP]]$,~ $[[Γ, Ord varset1 ⊢ iQ]]$,~
    $[[Γ, Ord varset2 ⊢ σ12 : Ord varset1]]$,~
    $[[Γ, Ord varset1 ⊢ σ21 : Ord varset2]]$,
    suppose that:
    \begin{enumerate}
      \item $\{ [[coh]] \mid [[α±:coh]] \in [[varset1 ∩ fv iQ]] \}$ = $\{ [[coh]] \mid [[α±:coh]] \in [[varset2 ∩ fv iP]] \}$,
      \item $[[Γ, Ord varset1 ⊢ [σ21]iP ≥ iQ]]$,
      \item $[[Γ, Ord varset2 ⊢ [σ12]iQ ≥ iP]]$.
    \end{enumerate}
    Then there exists a bijection
    $\mu : [[varset1 ∩ fv iQ]] \leftrightarrow [[varset2 ∩ fv iP]]$ such that:
    \begin{enumerate}
    \item $\mu$ preserves cohorts:
      $\mu ([[α±:coh]]) = [[β±:coh]]$ ($[[α±:coh]]$ and $[[β±:coh]]$
      have the same cohort label $\pm [[coh]]$).
    \item $[[Γ, Ord varset2 ⊢ σ12 ≈ Sub μ   : Ord (varset1 ∩ fv iQ) ]]$ (the equivalence is pointwise)
    \item $[[Γ, Ord varset1 ⊢ σ21 ≈ Sub μ-1 : Ord (varset2 ∩ fv iP) ]]$ (the equivalence is pointwise)
    \end{enumerate}

  \item[$-$]
    For $[[Γ, Ord varset2 ⊢ iN]]$,~ $[[Γ, Ord varset1 ⊢ iM]]$,~
    $[[Γ, Ord varset2 ⊢ σ12 : Ord varset1]]$,~
    $[[Γ, Ord varset1 ⊢ σ21 : Ord varset2]]$,
    suppose that:
    \begin{enumerate}
    \item $\{ [[coh]] \mid [[α±:coh]] \in [[varset1 ∩ fv iM]] \}$ = $\{ [[coh]] \mid [[β±:coh]] \in [[varset2 ∩ fv iN]] \}$,
    \item $[[Γ, Ord varset1 ⊢ [σ21]iN ≤ iM]]$,
    \item $[[Γ, Ord varset2 ⊢ [σ12]iM ≤ iN]]$.
    \end{enumerate}
    Then there exists a bijection
    $\mu : [[varset1 ∩ fv iM]] \leftrightarrow [[varset2 ∩ fv iN]]$ such that:
    \begin{enumerate}
    \item $\mu$ preserves cohorts:
      $\mu ([[α±:coh]]) = [[β±:coh]]$ ($[[α±:coh]]$ and $[[β±:coh]]$
      have the same cohort label $\pm [[coh]]$).
    \item $[[Γ, Ord varset2 ⊢ σ12 ≈ Sub μ : Ord (varset1 ∩ fv iM) ]]$ (the equivalence is pointwise)
    \item $[[Γ, Ord varset1 ⊢ σ21 ≈ Sub μ-1 : Ord (varset2 ∩ fv iN)]]$ (the equivalence is pointwise)
    \end{enumerate}
  \end{itemize}
\end{lemma}
\begin{proof}
  Mutual induction on the pair of sizes of inference trees: $[[Γ, Ord varset1 ⊢ [σ21]iN ≤ iM]]$
  and $[[Γ, Ord varset2 ⊢ [σ12]iM ≤ iN]]$ (or the corresponding trees in the
  positive case).

  \begin{caseof}
  \item $[[iN]] = [[∀pds.β⁻:coh]]$, for $[[β⁻:coh]] \in
    [[varset2]]$ and possibly empty $[[pds]]$\\
    \label{case:double-subtyping-subst-happened}
    Then by \cref{todo} \ilyam{lemma} $[[Γ, Ord varset2 ⊢ [σ12]iM ≤ ∀pds.β⁻:coh]]$ means that $[[ [σ12]iM ]] =
    [[∀pcs.β⁻:coh]]$, which (since $[[β⁻:coh]] \notin [[fv iM]]$) is only possible
    when $[[iM]] = [[∀pcs'.α⁻:coh']]$ (where $[[α⁻:coh']] \in [[varset1]]$ and $[[pcs']]$ is possibly empty), and
    $[[σ12]] ([[α⁻:coh']]) = [[∀pcs''.β⁻:coh]]$.
    Notice that by the first condition, the sets of cohort labels of $[[varset1
    ∩ fv iM]]$ and $[[varset2 ∩ fv iN]]$ must be equal, which means $[[coh]] = [[coh']]$.
    Also notice that $[[Γ, Ord varset2 ⊢ ∀pcs''.β⁻:coh ≈ β⁻:coh]]$.

    Then $[[Γ, Ord varset1 ⊢ [σ21]iN ≤ iM]]$ becomes
    $[[Γ, Ord varset1 ⊢ [σ21]∀pds.β⁻:coh ≤ ∀pcs'.α⁻:coh]]$,
    which by \cref{todo} \ilyam{lemma} implies that $[[ [σ21] ∀pds.β⁻:coh]] =
    [[∀pds'.α⁻:coh]]$, and thus,
    $[[σ21]] ([[β⁻:coh]]) = [[∀pds''.α⁻:coh]]$.
    Notice that $[[Γ, Ord varset1 ⊢ ∀pds''.α⁻:coh ≈ α⁻:coh]]$.

    This way, we can take $\mu = [[α⁻:coh]] \mapsto [[β⁻:coh]] $, which by
    construction is a bijection preserving cohorts. Moreover,
    $[[Γ, Ord varset2 ⊢ ∀pcs''.β⁻:coh ≈ β⁻:coh]]$ means
    $[[Γ, Ord varset2 ⊢ [σ12] α⁻:coh ≈ [Sub μ] α⁻:coh]]$ implying
    $[[Γ, Ord varset2 ⊢ σ12 ≈ Sub μ : Ord (varset1 ∩ fv iM) ]]$;
    and
    $[[Γ, Ord varset1 ⊢ ∀pds''.α⁻:coh ≈ α⁻:coh]]$ means
    $[[Γ, Ord varset1 ⊢ [σ21] β⁻:coh ≈ [Sub μ-1] β⁻:coh]]$ implying
    $[[Γ, Ord varset1 ⊢ σ21 ≈ Sub μ-1 : Ord (varset2 ∩ fv iN)]]$.

  \item The last rule to infer $[[Γ, Ord varset1 ⊢ [σ21]iN ≤ iM]]$ was
    \ruleref{\ottdruleDOneNVarLabel}, i.e. $[[ [σ21]iN ]] = [[iM]] = [[γ⁻]]$.\\
    Then the case when $[[γ⁻]] \in [[varset2]]$, has been covered by
    \cref{case:double-subtyping-subst-happened}, so we assume
    that $[[γ⁻]] \notin [[varset2]]$, and thus, $[[iN]] = [[γ⁻]]$.

    Notice that $[[γ⁻]] \notin [[varset1]]$ because otherwise,
    $[[varset1 ∩ fv iN]] \neq \emptyset$,
    which would contradict with $[[Γ, Ord varset2 ⊢ iN]]$.

    Then $[[fv iN ∩ varset2]] = [[{γ⁻} ∩ varset2]] = \emptyset$ and $[[fv iM ∩
    varset1]] = [[{γ⁻} ∩ varset1]] = \emptyset$.
    Hence, we take the empty $\mu : \emptyset \leftrightarrow \emptyset$, which
    vacuously satisfies the required properties.

  \item The last rule to infer $[[Γ, Ord varset1 ⊢ [σ21]iN ≤ iM]]$ was
    \ruleref{\ottdruleDOneShiftULabel}, i.e. $[[ [σ21]iN ]] = [[↑iP]]$, $[[ iM
    ]] = [[↑iQ]]$, and $[[Γ, Ord varset1 ⊢ iP ≈ iQ]]$\\
    Since $[[iN]]$ is not a variable from the domain of $[[σ21]]$ (which has
    been covered by \cref{case:double-subtyping-subst-happened}),
    the substitution applied to $[[iN]]$ must preserve its outer shape.
    Specifically, $[[ [σ21]iN ]] = [[↑iP]]$ means
    $[[ [σ21]iN ]] = [[ [σ21]↑iP']] = [[ ↑[σ21]iP']] =[[↑iP]]$,
    i.e. $[[ iN ]] = [[↑iP']]$ and
    $[[ [σ21]iP']] = [[iP]]$. In particular,
    $[[Γ, Ord varset1 ⊢ iP ≈ iQ]]$ implies
    $[[Γ, Ord varset1 ⊢ iP ≥ iQ]]$, i.e.
    $[[Γ, Ord varset1 ⊢ [σ21]iP' ≥ iQ]]$.

    In addition, $[[Γ, Ord varset2 ⊢ [σ12]iM ≤ iN]]$ becomes
    $[[Γ, Ord varset2 ⊢ ↑[σ12]iQ ≤ ↑iP']]$, which is only inferable by
    \ruleref{\ottdruleDOneShiftULabel}, meaning that
    $[[Γ, Ord varset2 ⊢ [σ12]iQ ≈ iP']]$, and in particular,
    $[[Γ, Ord varset2 ⊢ [σ12]iQ ≥ iP']]$.

    Notice that the tree inferring $[[Γ, Ord varset2 ⊢ [σ12]iQ ≥ iP']]$ is a
    proper subtree of $[[Γ, Ord varset2 ⊢ [σ12]iM ≤ iN]]$. Analogously,
    $[[Γ, Ord varset1 ⊢ [σ21]iP' ≥ iQ]]$ is a proper subtree of
    $[[Γ, Ord varset1 ⊢ [σ21]iN ≤ iM]]$. This way, we apply the induction
    hypothesis to $[[Γ, Ord varset1 ⊢ [σ21]iP' ≥ iQ]]$ and $[[Γ, Ord varset2 ⊢
    [σ12]iQ ≥ iP']]$ (notice that $[[varset1]]$, $[[varset2]]$, and the sets of free variables of
    the types did not change) and obtain exactly what we aimed.

  \item The last rule to infer $[[Γ, Ord varset1 ⊢ [σ21]iN ≤ iM]]$ was
    \ruleref{\ottdruleDOneForallLabel}, i.e. $[[ [σ21]iN ]] = [[∀pcs.iN']]$,
    $[[iM]] = [[∀pds.iM']]$, and $[[Γ, Ord varset1, pds ⊢ [iPs/pcs]iN' ≤ iM']]$
    for  $[[Γ, Ord varset1, pds ⊢ iPi]]$ \\

    Since $[[iN]]$ does not have the shape of $[[∀pds.β⁻:coh]]$, for $[[β⁻:coh]]
    \in [[varset2]]$ (which has been covered by
    \cref{case:double-subtyping-subst-happened}), the substitution applied to
    $[[iN]]$ must preserve its outer shape. Specifically,
    $[[ [σ21]iN ]] = [[∀pcs.iN']]$
    means that $[[ iN ]]$ ``starts with'' $\forall[[pcs]]$, i.e.
    $[[ [σ21]iN ]] = [[ [σ21]∀pcs.iN'' ]] = [[ ∀pcs.[σ21]iN'' ]] = [[ ∀pcs.iN'
    ]]$, where $[[iN]] = [[∀pcs.iN'']]$ and $[[ [σ21]iN'']] = [[iN']]$.

    This way, $[[Γ, Ord varset1, pds ⊢ [iPs/pcs]iN' ≤ iM']]$ becomes
    $[[Γ, Ord varset1, pds ⊢ [iPs/pcs][σ21]iN'' ≤ iM']]$.
    Notice that the tree inferring this judgment is a proper subtree of
    $[[Γ, Ord varset1 ⊢ [σ21]iN ≤ iM]]$.


    On the other hand, $[[Γ, Ord varset2 ⊢ [σ12]iM ≤ iN]]$
    becomes $[[Γ, Ord varset2 ⊢ ∀pds.[σ12]iM' ≤ ∀pcs.iN'']]$
    (where either $[[pds]]$ or $[[pcs]]$ is non-empty),
    which is only inferable by \ruleref{\ottdruleDOneForallLabel},
    meaning that
    $[[Γ, Ord varset2, pcs ⊢ [iQs/pds][σ12]iM' ≤ iN'']]$ for some
    $[[Γ, Ord varset2, pcs ⊢ iQi]]$.
    Notice that the tree inferring this judgment is a proper subtree of
    $[[Γ, Ord varset2 ⊢ [σ12]iM ≤ iN]]$.

    Let us label $[[pcs]]$ and $[[pds]]$ with a sufficiently large cohort label
    $[[mcoh]]$ such that $[[mcoh]] > [[coh]]$ for any $[[α⁺:coh]] \in [[varset1 ∪
    varset2]]$. Then we merge $[[iQs/δ⁺:mcoh*]]$ and $[[σ12]]$ denoting the resulting
    substitution as $[[σ12']]$ $([[(Γ, Ord varset2, γ⁺:mcoh*) ⊢ σ12' : (Ord varset1, δ⁺:mcoh*)]])$.

    What do we mean by merging? The codomains of $[[iQs/δ⁺:mcoh*]]$ and
    $[[σ12]]$ are $[[Γ, Ord varset2,  γ⁺:mcoh*]]$ and $[[Γ, Ord varset2]]$,
    respectively, which are disjoint with both of their domains
    ($[[δ⁺:mcoh*]]$ and $[[Ord varset1]]$, respectively). In turn, the domains
    of $[[iQs/δ⁺:mcoh*]]$ and $[[σ12]]$ are themselves mutually disjoint.
    This way, by merging we mean
    $[[iQs/δ⁺:mcoh*]] \cup [[σ12]]$ (relationally) or
    $[[iQs/δ⁺:mcoh* ○ σ12]]$ or
    $[[σ12 ○ iQs/δ⁺:mcoh*]]$, since these three substitutions are equal.

    Analogously, we merge $[[iPs/γ⁺:mcoh*]]$ and $[[σ21]]$ denoting the
    resulting substitution as $[[σ21']]$ $([[(Γ, Ord varset1, δ⁺:mcoh*) ⊢ σ21' :
    (Ord varset2, γ⁺:mcoh*)]])$.

    We wish to apply the induction hypothesis to
    $[[Γ, Ord varset1, δ⁺:mcoh* ⊢ [σ21']iN'' ≤ iM']]$ and
    $[[Γ, Ord varset2, γ⁺:mcoh* ⊢ [σ12']iM' ≤ iN'']]$.
    To do so, we need to show that the cohort labels of $[[(varset2 ∪ {γ⁺:mcoh*}) ∩ fv
    iN'']]$ coincide with those of $[[(varset1 ∪ {δ⁺:mcoh*}) ∩ fv iM']]$ (as sets).
    \begin{assertion*}
      The set of cohorts of $[[(varset2 ∪ {γ⁺:mcoh*}) ∩ fv iN'']]$
      is equal to the set of cohorts of $[[(varset1 ∪ {δ⁺:mcoh*}) ∩ fv iM']]$.
    \end{assertion*}

    \begin{proof}
      $
      \begin{aligned}[t]
        [[(varset2 ∪ {γ⁺:mcoh*}) ∩ fv iN'']] &= [[(varset2 ∪ {γ⁺:mcoh*}) ∩ (fv
                                               iN ∪ ({γ⁺:mcoh*} ∩ fv iN''))]]\\
                                             &= [[varset2 ∩ fv iN ∪ {γ⁺:mcoh*} ∩ fv iN'']]
                                             && \text{because
                                                $[[varset2]]$ and $[[fv iN]]$ are disjoint with $[[γ⁺:mcoh*]]$}\\
        [[(varset1 ∪ {δ⁺:mcoh*}) ∩ fv iM']]  &= [[varset1 ∩ fv iM ∪ {δ⁺:mcoh*} ∩ fv iM']]
                                             && \text{analogously}\\
      \end{aligned}\\
      $
      Since the cohort labels of $[[varset2 ∩ fv iN]]$ coincide with those of
      $[[varset1 ∩ fv iM]]$ by the assumption, it suffices to prove that the cohort labels of
      $[[{γ⁺:mcoh*} ∩ fv iN'']]$ and of $[[{δ⁺:mcoh*} ∩ fv iM']]$ coincide.
      Note that these sets have the same
      cohorts labels $[[mcoh]]$, i.e. it is required to show that these sets are
      either both empty or both non-empty, i.e.
      $[[{γ⁺:mcoh*} ∩ fv iN'']] = \emptyset \iff [[{δ⁺:mcoh*} ∩ fv iM']] = \emptyset$
      \begin{itemize}
      \item [$(\Leftarrow)$] Suppose that $[[{δ⁺:mcoh*} ∩ fv iM']] = \emptyset$.
        Then by \cref{todo}, since
        $[[Γ, Ord varset1, δ⁺:mcoh* ⊢ [σ21']iN'' ≤ iM']]$, $[[fv [σ21']iN'']]
        \subseteq [[fv iM']]$, implying that $[[{δ⁺:mcoh*} ∩ fv [σ21']iN'']] =
        \emptyset$,
        and by context strengthening \cref{todo}, $[[Γ, Ord varset1, δ⁺:mcoh* ⊢ [σ21']iN'']]$ reduces to
        $[[Γ, Ord varset1 ⊢ [σ21']iN'']]$.


        Let us restrict $[[σ21']]$ to the set of free variables of $[[fv iN'']]$.
        Then the domain and codomain of $[[σ21']]$ are the following:
        $[[Γ, Ord varset1 ⊢ σ21'|fv iN'' : Ord {Ord varset2, γ⁺:mcoh*} ∩ fv iN]]$.
        By \cref{todo}, $[[ [σ21']iN'' ]] = [[ [σ21'|fv iN'']iN'' ]]$,
        and then
        $[[Γ, Ord varset1 ⊢ [σ21']iN'' ≤ iM']]$, can be rewritten as
        $[[Γ, Ord varset1 ⊢ [σ21'|fv iN'']iN'' ≤ iM']]$.

        Let us apply substitution $[[σ12]]$ ($[[Γ, Ord varset2 ⊢ σ12 : Ord varset1]]$)
        to both sides of this judgment to obtain: $[[Γ, Ord varset2 ⊢ [σ12 ○ σ21'|fv iN'']iN'' ≤ [σ12]iM']]$.
        Using the transitivity \cref{todo}, let us compose this subtyping
        judgment with $[[Γ, Ord varset2, γ⁺:mcoh* ⊢ [σ12]iM' ≤ iN'']]$ to
        form
        $[[Γ, Ord varset2, γ⁺:mcoh* ⊢ [σ12 ○ σ21'|fv iN'']iN'' ≤ iN'']]$.

        What is the codomain of $[[σ12 ○ σ21'|fv iN'']]$?
        By composing
        $[[Γ, Ord varset1 ⊢ σ21'|fv iN'' : Ord {Ord varset2, γ⁺:mcoh*} ∩ fv iN]]$
        and
        $[[Γ, Ord varset2 ⊢ σ12 : Ord varset1]]$,
        we have
        $[[Γ, Ord varset2 ⊢ σ12 ○ σ21'|fv iN'' : Ord {Ord varset2, γ⁺:mcoh*} ∩
        fv iN]]$, i.e. $[[σ12 ○ σ21'|fv iN'']]$ is a substitution contracting
        $[[γ⁺:mcoh*]]$.
        Then by \cref{todo},
        $[[Γ, Ord varset2, γ⁺:mcoh* ⊢ [σ12 ○ σ21'|fv iN'']iN'' ≤ iN'']]$
        implies that $[[{γ⁺:mcoh*} ∩ fv iN'']] = \emptyset$.
        \item [$(\Rightarrow)$]
          Analogous to the previous case.
      \end{itemize}
    \end{proof}
    This way, we can apply the induction hypothesis to
    $[[Γ, Ord varset1, δ⁺:mcoh* ⊢ [σ21']iN'' ≤ iM']]$ and
    $[[Γ, Ord varset2, γ⁺:mcoh* ⊢ [σ12']iM' ≤ iN'']]$,
    and obtain
    $\mu : [[(varset1 ∪ {δ⁺:mcoh*}) ∩ fv iM']] \leftrightarrow [[(varset2 ∪ {γ⁺:mcoh*}) ∩ fv iN'']]$ such that:
    \begin{enumerate}
    \item $\mu$ preserves cohorts,
    \item $[[Γ, Ord varset2, γ⁺:mcoh* ⊢ σ12' ≈ Sub μ : Ord (varset1 ∪ {δ⁺:mcoh*}) ∩ fv iM' ]]$, and
    \item $[[Γ, Ord varset1, δ⁺:mcoh* ⊢ σ21' ≈ Sub μ-1 : Ord (varset2 ∪ {γ⁺:mcoh*}) ∩ fv iN'']]$
    \end{enumerate}

    Let us decompose $\mu$ into the union of two bijections: $\mu_1 \sqcup
    \mu_2$, where $\mu_1$ is defined on the variables whose cohort labels are less
    than $[[mcoh]]$, and $\mu_2$ is defined on the variables whose cohort labels
    are equal to $[[mcoh]]$. Notice that since $\mu$ preserves cohorts,
    this union is disjoint: the \emph{range} of $\mu_1$ is the variables with cohorts
    $< [[mcoh]]$ and the \emph{range} of $\mu_2$ is the variables with cohorts
    $= [[mcoh]]$.

    Recalling how $[[mcoh]]$ is chosen, notice that the signatures of $\mu_1$ and
    $\mu_2$ are the following:
    \begin{enumerate}
      \item $\mu_1 : [[varset1 ∩ fv iM']] \leftrightarrow [[varset2 ∩ fv iN'']]$ and
      \item $\mu_2 : [[{δ⁺:mcoh*} ∩ fv iM']] \leftrightarrow [[{γ⁺:mcoh*} ∩ fv iN'']]$.
    \end{enumerate}

    This way, $\mu_1$ is a cohort-preserving bijection with the required
    signature, and what is left to show is the equivalences $[[μ1]] \eqDOne
    [[σ12]]$, and $[[μ1-1]] \eqDOne [[σ21]]$:
    \begin{enumerate}[nosep]
    \item $[[Γ, Ord varset2 ⊢ σ12 ≈ Sub μ1 : Ord (varset1 ∩ fv iM) ]]$\\
      Let us take an arbitrary $[[α±]] \in [[varset1 ∩ fv iM]]$.\\
      $
      \begin{aligned}
        [[ [σ12] α± ]] &= [[ [σ12 ○ iQs/δ⁺:mcoh*] α± ]]
                       && \text{because $[[α±]] \neq [[δi⁺:mcoh]]$} \\
                       &= [[ [σ12'] α± ]]
                       && \text{by the definition of $[[σ12']]$} \\
                       &\eqDOne [[ [Sub μ] α± ]]
                       && \text{since $[[Γ, Ord varset2, γ⁺:mcoh* ⊢ σ12' ≈ Sub μ : Ord ((varset1 ∪ {δ⁺:mcoh*}) ∩ fv iN) ]]$} \\
                       &= [[ [Sub μ1] α± ]]
                       && \text{since $[[μ1]] = [[Sub μ | varset1 ∩ fv iM']]$
                          and  $[[α±]] \in [[varset1 ∩ fv iM]] \subseteq [[varset1 ∩ fv iM']]$} \\
      \end{aligned}
      $\\
      This way, $[[Γ, Ord varset2, γ⁺:mcoh* ⊢ [σ12] α± ≈ [Sub μ1] α±  ]]$, which,
      considering that the codomains of $[[σ12]]$ and $[[μ1]]$ are in $[[Γ, Ord
      varset2]]$, can be strengthen to $[[Γ, Ord varset2 ⊢ [σ12] α± ≈ [Sub μ1] α± ]]$.

    \item $[[Γ, Ord varset1 ⊢ σ21 ≈ Sub μ1-1 : Ord (varset2 ∩ fv iN)]]$ is proved analogously.

    \end{enumerate}

  \item The last rule to infer $[[Γ, Ord varset1 ⊢ [σ21]iN ≤ iM]]$ was
    \ruleref{\ottdruleDOneArrowLabel}, i.e. $[[ [σ21]iN ]] = [[iP → iN']]$, and
    $[[iM]] = [[iQ → iM']]$; then by inverting this rule,
    $[[Γ, Ord varset1 ⊢ iP ≥ iQ]]$ and $[[Γ, Ord varset1 ⊢ iN' ≤ iM']]$.

    $[[ [σ21]iN ]] = [[iP → iN']]$ means that either $[[iN]]$ is a variable
    from the domain of $[[σ21]]$ (which has been covered by
    \cref{case:double-subtyping-subst-happened})
    or $[[ [σ21]iN ]] = [[ [σ21] (iP' → iN'')]] = [[ [σ21] iP' → [σ21] iN'' ]] =
    [[iP → iN']]$, where
    $[[ iN ]] = [[iP' → iN'']]$, $[[ [σ21] iP' ]] = [[iP]]$, and $[[ [σ21] iN'' ]] = [[iN']]$.

    This way, $[[Γ ⊢ iP ≥ iQ]]$ and $[[Γ ⊢ iN' ≤ iM']]$ can be rewritten as
    $[[Γ ⊢ [σ21]iP' ≥ iQ]]$ and $[[Γ ⊢ [σ21] iN'' ≤ iM']]$.

    In addition, $[[Γ, Ord varset2 ⊢ [σ12]iM ≤ iN]]$ becomes
    $[[Γ, Ord varset2 ⊢ [σ12](iQ → iM') ≤ iP' → iN'']]$,
    implying by inversion
    $[[Γ, Ord varset2 ⊢ [σ12]iQ ≥ iP']]$ and
    $[[Γ, Ord varset2 ⊢ [σ12]iM' ≤ iN'']]$.

  \end{caseof}
\end{proof}