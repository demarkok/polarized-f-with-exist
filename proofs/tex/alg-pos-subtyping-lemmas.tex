\begin{lemma}[Soundness of the Positive Subtyping] \label{lemma:pos-subt-soundness}
    If $[[Γ ⊢ Θ]]$, $[[Γ ⊢ iQ]]$, $[[Γ ; dom(Θ) ⊢  uP]]$, and 
    $[[Γ ; Θ ⊨ uP ≥ iQ ⫤ SC]]$,
    then $[[Θ ⊢ SC]]$ and
    for any normalized $[[uσ]]$ such that $[[(Θ  ⊢  SC) ⊢ uσ]]$,
    $[[ Γ ⊢ [uσ]uP ≥ iQ ]]$.
\end{lemma}
\begin{proof} 
    We prove it by induction on $[[Γ ; Θ ⊨ uP ≥ iQ ⫤ SC]]$. 
    Let us consider the last rule to infer this judgment.
    \begin{caseof}
    \item \ruleref{\ottdruleAPUVarLabel} then
    $[[Γ ; Θ ⊨ uP ≥ iQ ⫤ SC]]$ has shape $[[Γ;Θ ⊨ â⁺ ≥ iP' ⫤ (â⁺ :≥ iQ')]]$ where
    $[[â⁺[Δ] ∊ Θ]]$ and $[[upgrade G ⊢ iP' to Δ = iQ']]$.

    Notice that $[[â⁺[Δ] ∊ Θ]]$ and $[[Γ ⊢ Θ]]$ 
    implies $[[Γ = Δ, pnas]]$ for some $[[pnas]]$, hence, the
    soundness of upgrade (\cref{lemma:upgrade-soundness}) is applicable:
    \begin{enumerate}
        \item $[[Δ ⊢ iQ']]$ and
        \item $[[Γ ⊢ iQ' ≥ iP]]$.
    \end{enumerate}

    Since $[[â⁺[Δ] ∊ Θ]]$ and $[[Δ ⊢ iQ']]$, 
    it is clear that $[[ Θ ⊢ (â⁺ :≥ iQ') ]]$.

    It is left to show that $[[Γ ⊢ [uσ]â⁺ ≥ iP']]$ for any normalized $[[uσ]]$ 
    s.t. $[[(Θ  ⊢  (â⁺ :≥ iQ')) ⊢ uσ]]$.
    The latter means that $[[ Θ(â⁺) ⊢ [uσ]â⁺ ≥ iQ' ]]$, i.e. $[[Δ ⊢ [uσ]â⁺ ≥ iQ']]$. 
    By weakening the context to $[[Γ]]$ and combining this judgment
    transitively with $[[Γ ⊢ iQ' ≥ iP]]$, we have $[[Γ ⊢ [uσ]â⁺ ≥ iP]]$,
    as required. 

    \item \label{case:pos-subt-soundness:var} \ruleref{\ottdruleAPVarLabel}  
    then $[[Γ ; Θ ⊨ uP ≥ iQ ⫤ SC]]$ has shape $[[Γ;Θ ⊨ a⁺ ≥ a⁺ ⫤ ·]]$ .
    Then $[[uv a⁺ = ∅]]$, and $[[SC]] = [[·]]$ satisfies $[[Θ ⊢ ·]]$.
    Since $[[uv a⁺ = ∅]]$, application of any substitution $[[uσ]]$ 
     does not change $[[a⁺]]$, i.e. $[[ [uσ] a⁺ = a⁺]]$.
      Therefore, $[[Γ ⊢ [uσ]a⁺ ≥ a⁺]]$ holds by \ruleref{\ottdruleDOneNVarLabel}.

    \item \label{case:pos-subt-soundness:shift} 
     \ruleref{\ottdruleAShiftDLabel} then
    $[[Γ ; Θ ⊨ uP ≥ iQ ⫤ SC]]$ has shape $[[Γ;Θ ⊨ ↓uN ≥ ↓iM ⫤ SC]]$.\\
    Then the next step of the algorithm is the unification of $[[nf(uN)]]$ and $[[nf(iM)]]$,
    and it returns the resulting unification constraint $[[lift UC]] = [[SC]]$ as the result.
    By the soundness of unification (\cref{lemma:unification-soundness}),
    $[[Θ ⊢ SC]]$ and for any normalized $[[uσ]]$, $[[(Θ  ⊢  SC) ⊢ uσ]]$
    implies $[[ [uσ]nf(uN) = nf(iM) ]]$, 
    then we rewrite the left-hand side by \cref{lemma:norm-subst-distr}:
    $[[ nf([uσ]uN) = nf(iM) ]]$ and apply \cref{lemma:subt-equiv-algorithmization}:
    $[[Γ ⊢ [uσ]uN ≈ iM]]$, then by \ruleref{\ottdruleDOneShiftULabel},
    $[[Γ ⊢ ↓[uσ]uN ≥ ↓iM]]$.
    
    \item \label{case:pos-subt-soundness:exists}
       \ruleref{\ottdruleAExistsLabel} then
        $[[Γ ; Θ ⊨ uP ≥ iQ ⫤ SC]]$ has shape $[[Γ;Θ ⊨ ∃nas.uP' ≥ ∃nbs.iQ' ⫤ SC]]$ s.t. either 
        $[[nas]]$ or $[[nbs]]$ is not empty.\\
        Then the algorithm creates fresh unification variables $[[â⁻*[Γ,nbs] ]]$, 
        substitutes the old $[[nas]]$ with them in $[[uP']]$, and makes the recursive call:
        $[[G, nbs; Θ, â⁻*[G, nbs] ⊨ [â⁻*/nas] uP' ≥ iQ' ⫤ SC']]$, returning as the result
        $[[SC]] = [[SC' \ {α̂⁻*}]]$.

        Let us take an arbitrary normalized $[[uσ]]$ s.t. $[[(Θ  ⊢  SC' \ {α̂⁻*}) ⊢ uσ]]$.
        We wish to show $[[Γ ⊢ [uσ]uP ≥ iQ]]$, i.e. $[[Γ ⊢ ∃nas.[uσ]uP' ≥ ∃nbs.iQ']]$.
        To do that, we apply \ruleref{\ottdruleDOneExistsLabel}, and what is left to show is
        $[[Γ, nbs ⊢ [iNs/nas][uσ]uP' ≥ iQ']]$ for some $[[iNs]]$.
        If we construct a normalized $[[uσ']]$ such that $[[(Θ, â⁻*[G, nbs] ⊢ SC') ⊢ uσ']]$
        and for some $[[iNs]]$, $[[ [iNs/nas][uσ]uP' ]] = [[ [uσ'][â⁻*/nas]uP' ]]$,
        we can apply the induction hypothesis to 
        $[[Γ, nbs; Θ, â⁻*[G, nbs] ⊨ [â⁻*/nas] uP ≥ iQ ⫤ SC']]$ and infer 
        the required subtyping.

        Let us construct such $[[uσ']]$ by extending $[[uσ]]$ with $[[â⁻*]]$
        mapped to the corresponding types in $[[SC']]$:
        $$
        [[ [uσ']β̂± ]]  = 
            \begin{cases}
               [[ [uσ]β̂± ]] & \text{if } [[β̂±]] \in [[dom(SC') \ {α̂⁻*}]]  \\
               [[ nf(iN) ]] & \text{if } [[β̂±]] \in [[α̂⁻*]] \text{ and } [[(β̂± :≈ iN)]] \in SC' \\
            \end{cases}
        $$

        It is easy to see that $[[uσ']]$ is normalized: it inherits this property from 
        $[[uσ]]$.
        Let us show that $[[(Θ, â⁻*[G, nbs]  ⊢ SC') ⊢ uσ' ]] $.
        Let us take an arbitrary entry $[[scE]]$ from $[[SC']]$ restricting a variable $[[β̂±]]$.
        Suppose $[[β̂±]] \in [[dom(SC') \ {α̂⁻*}]]$. Then
        $[[ (Θ, â⁻*[G, nbs])(β̂±) ⊢ [uσ']β̂± : scE ]]$ is
        rewritten as $[[ Θ(β̂±) ⊢ [uσ]β̂± : scE ]]$, which holds since $[[(Θ  ⊢  SC') ⊢ uσ]]$.
        Suppose $[[β̂±]] = [[αî⁻]] \in [[α̂⁻*]]$. Then
        $[[scE]] = [[(αî⁻ :≈ iN)]]$ for some $[[iN]]$, 
        $[[ [uσ']αî⁻ ]] = [[ nf(iN) ]]$ by the definition,
        and $[[ Γ, nbs ⊢ nf(iN) : (αî⁻ :≈ iN) ]]$ by \ruleref{\ottdruleSATSCENEqLabel},
        since $[[Γ ⊢ nf(iN) ≈ iN]]$ by \cref{lemma:subt-equiv-algorithmization}.

        Finally, let us show that $[[ [iNs/nas][uσ]uP' ]] = [[ [uσ'][â⁻*/nas]uP' ]]$.
        For $[[iNi]]$, we take the \emph{normalized} type restricting $[[αî⁻]]$ in $[[SC']]$.
        Let us take an arbitrary variable from $[[uP]]$.
        \begin{enumerate}
            \item If this variable is a unification variable $[[β̂±]]$, then
                $[[ [iNs/nas][uσ] β̂± ]] = [[ [uσ]β̂± ]] $, since $[[(Θ  ⊢  SC' \ {α̂⁻*}) ⊢ uσ]]$ and 
                $[[ dom(Θ) ∩ {nas} = ∅ ]]$. 

                Notice that $[[β̂±]] \in [[dom(Θ)]]$, which is disjoint from $[[{α̂⁻*}]]$, 
                that is $[[β̂±]] \in [[dom(SC') \ {α̂⁻*}]]$. This way,
                $[[ [uσ'][â⁻*/nas]β̂± ]] = [[  [uσ']β̂± ]] = [[ [uσ]β̂± ]]$ by the definition 
                of $[[uσ']]$,
            \item If this variable is a regular variable $[[β±]] \notin [[nas]]$, then 
                $[[ [iNs/nas][uσ] β± ]] = [[ β± ]] $ and $[[ [uσ'][â⁻*/nas]β± ]] = [[ β± ]]$. 
            \item If this variable is a regular variable $[[αi⁻]] \in [[nas]]$, then 
                $[[ [iNs/nas][uσ] αi⁻ ]] = [[ iNi ]] = [[ nf(iNi) ]]$
                (the latter equality holds since $[[iNi]]$ is normalized)
                and $[[ [uσ'][â⁻*/nas]αi⁻ ]] = [[  [uσ']αî⁻ ]] = [[ nf(iNi) ]]$.
        \end{enumerate}
    \end{caseof}
\end{proof}

\begin{lemma}[Completeness of the Positive Subtyping] \label{lemma:pos-subt-completeness}
    Suppose that $[[Γ ⊢ Θ]]$, $[[Γ ⊢ iQ]]$ and $[[Γ ; dom(Θ) ⊢  uP]]$.
    Then for any $[[Θ ⊢ uσ]]$ such that $[[ Γ ⊢ [uσ]uP ≥ iQ ]]$,
    there exists $[[Γ; Θ ⊨ uP ≥ iQ ⫤ SC]]$ and moreover, $[[(Θ  ⊢  SC) ⊢ uσ]]$.
\end{lemma}
\begin{proof}
    Let us prove this lemma by induction on $[[ Γ ⊢ [uσ]uP ≥ iQ ]]$.
    Let us consider the last rule used in the derivation,
    but first, consider the base case for the substitution $[[ [uσ]uP ]]$:
    \begin{caseof}
        \item \label{case:pos-subt-complete-base} $[[uP]] = [[ ∃nbs.α̂⁺ ]]$ 
            (for potentially empty $[[nbs]]$)\\
            Then by assumption, $[[ Γ ⊢ ∃nbs.[uσ]α̂⁺ ≥ iQ ]]$ (where $[[ {nbs} ∩ fv [uσ]α̂⁺ = ∅]]$).
            Let us decompose $[[iQ]]$ as $[[iQ]] = [[∃ncs.iQ0]]$, where $[[iQ0]]$ does
            not start with $[[∃]]$. 

            By inversion, $[[Γ ; dom(Θ) ⊢  ∃nbs.α̂⁺]]$ implies $[[â⁺[Δ] ∊ Θ]]$ for some 
            $[[{Δ} ⊆ {Γ}]]$.

            By \cref{lemma:quant-rule-decomposition} applied twice, 
            $[[ Γ ⊢ ∃nbs.[uσ]α̂⁺ ≥ ∃ncs.iQ0 ]]$ implies
            $[[ Γ,ncs  ⊢ [iNs/nbs][uσ]α̂⁺ ≥ iQ0 ]]$ for some $[[iN]]$, 
            and since $[[ {nbs} ∩ fv([uσ]α̂⁺) ⊆ {nbs} ∩ {Θ(α̂⁺)} ⊆ {nbs} ∩ {Γ} = ∅ ]]$,
            $[[ [iNs/nbs][uσ]α̂⁺ = [uσ]α̂⁺ ]]$, that is $[[ Γ,ncs ⊢ [uσ]α̂⁺ ≥ iQ0]]$.

            When algorithm tires to infer the subtyping 
            $[[Γ; Θ ⊨ ∃nbs.α̂⁺ ≥ ∃ncs.iQ0 ⫤ SC]]$,
            it applies \ruleref{\ottdruleAExistsLabel},
            which reduces the problem to
            $[[Γ, ncs; Θ, β̂⁻*[Γ, ncs] ⊨ [β̂⁻*/nbs] α̂⁺ ≥ iQ0 ⫤ SC]]$, 
            which is equivalent to 
            $[[Γ, ncs; Θ, β̂⁻*[Γ, ncs] ⊨ α̂⁺ ≥ iQ0 ⫤ SC]]$.

            Next, the algorithm tries to apply
            \ruleref{\ottdruleAPUVarLabel}
            and the resulting restriction is $[[SC]] = [[(α̂⁺ :≥ iQ0')]]$ where
            $[[upgrade Γ, ncs ⊢ iQ0 to Δ = iQ0']]$.

            Why does the upgrade procedure terminate?
            Because $[[ [uσ]α̂⁺ ]]$ satisfies the pre-conditions of the completeness of the upgrade
            (\cref{lemma:upgrade-completeness}):
            \begin{enumerate}
                \item $[[Δ ⊢ [uσ]α̂⁺ ]]$ because $[[Θ ⊢ uσ]]$ and $[[α̂⁺[Δ] ∊ Θ]]$,
                \item $[[ Γ,ncs ⊢ [uσ]α̂⁺ ≥ iQ0]]$ as noted above
            \end{enumerate}

            Moreover, the completeness of upgrade also says that $[[iQ0']]$ is 
            \emph{the least} supertype of $[[iQ0]]$ among types well-formed in $[[Δ]]$, 
            that is $[[Δ ⊢ [uσ]α̂⁺ ≥ iQ0']]$, which means 
            $[[(Θ  ⊢  (α̂⁺ :≥ iQ0')) ⊢ uσ]]$, that is $[[(Θ  ⊢  SC) ⊢ uσ]]$.

        \item \label{case:pos-subt-complete-pvar}
        $[[ Γ ⊢ [uσ]uP ≥ iQ ]]$ is derived by \ruleref{\ottdruleDOnePVarLabel}\\
        Then $[[iP]] = [[ [uσ]uP ]] = [[ α⁺ ]] = [[iQ]]$, where
        the first equality holds because $[[uP]]$ is not a unification variable:
        it has been covered by \cref{case:pos-subt-complete-base}; and
        the second equality hold because \ruleref{\ottdruleDOnePVarLabel} was applied.

        The algorithm applies \ruleref{\ottdruleAPVarLabel} and 
        infers $[[SC]] = [[·]]$, i.e. $[[Γ;Θ ⊨ a⁺ ≥ a⁺ ⫤ ·]]$.
        Then $[[(Θ  ⊢  ·) ⊢ uσ]]$ holds trivially.


        \item \label{case:pos-subt-complete-upshift} $[[ Γ ⊢ [uσ]uP ≥ iQ ]]$ 
        is derived by \ruleref{\ottdruleDOneShiftDLabel},
        
        Then $[[ uP ]] = [[ ↓uN ]]$, since the substitution $[[ [uσ]uP ]]$ must preserve the 
        top-level constructor of $[[uP]]\neq [[α̂⁺]]$ (the case $[[uP]] = [[α̂⁺]]$ has been covered
        by \cref{case:pos-subt-complete-base}), and $[[uQ]] = [[ ↓iM ]]$,
        and by inversion, $[[ Γ ⊢ [uσ]uN ≈ iM ]]$.

        Since both types start with $[[↓]]$, 
        the algorithm tries to apply \ruleref{\ottdruleAShiftDLabel}: 
        $[[G;Θ ⊨ ↓uN ≥ ↓iM ⫤ SC]]$. The premise of this rule is the
        unification of $[[nf(uN)]]$ and $[[nf(iM)]]$:
        $[[Γ;Θ ⊨ nf(uN) ≈u nf(iM) ⫤ UC]]$. And the algorithm 
        returns it as a subtyping constraint $[[SC]] = [[lift UC]]$.


        To demonstrate that the unification terminates
        ant $[[uσ]]$ satisfies the resulting constraints, 
        we apply the completeness 
        of the unification algorithm (\cref{lemma:unification-completeness}). 
        In order to do that, we need to provide a substitution unifying  
        $[[nf(uN)]]$ and $[[nf(iM)]]$. 
        Let us show that $[[nf(uσ)]]$ is such a substitution. 

        \begin{itemize}
            \item $[[nf(uN)]]$ and $[[nf(iM)]]$ are normalized 
            \item $[[Γ ; dom(Θ) ⊢  nf(uN)]]$ because $[[Γ ; dom(Θ) ⊢  uN]]$ (\cref{corollary:wf-nf-algo})
            \item $[[Γ ⊢ nf(iM)]]$ because $[[Γ ⊢ iM]]$ (\cref{corollary:wf-nf})
            \item $[[ Θ ⊢ nf(uσ) ]]$ because $[[Θ ⊢ uσ ]]$ (\cref{corollary:norm-subst-sig-algo})
            \item $ \begin{aligned}[t]
                    [[ Γ ⊢ [uσ]uN ≈ iM ]] &\Rightarrow [[ [uσ]uN ≈ iM ]]
                                          && \text {by \cref{lemma:equiv-completeness}}\\
                                          &\Rightarrow [[ nf([uσ]uN) = nf(iM) ]]
                                          && \text {by \cref{lemma:normalization-completeness}}\\
                                          &\Rightarrow [[ [nf(uσ)]nf(uN) = nf(iM) ]]
                                          && \text {by \cref{lemma:norm-subst-distr}}\\
                    \end{aligned}
                  $
        \end{itemize}
        Then by the completeness of the unification,
        $[[Γ ; Θ ⊨ uN ≈u iM ⫤ UC]]$ exists, and
        $(Θ  ⊢ lift UC) ⊢ nf(uσ)$.  Then by \cref{corollary:sat-equiv}, $[[(Θ  ⊢  lift UC) ⊢ uσ]]$.

      \item $[[ Γ ⊢ [uσ]uP ≥ iQ ]]$ is derived by \ruleref{\ottdruleDOneExistsLabel}.\\
      We should only consider the case
      when the substitution $[[ [uσ]uP ]]$ results in the existential type 
      $[[∃nas.iP'']]$ (for $[[iP'']] \neq [[∃]]\dots$) by congruence, 
      i.e. $[[uP = ∃nas.uP']]$ (for $[[uP']] \neq [[∃]]\dots$) and $[[ [us]uP' = iP'' ]]$.
      This is because the case when $[[uP = ∃nbs.α̂⁺]]$ has been covered
      (\cref{case:pos-subt-complete-base}), and thus, the substitution $[[uσ]]$ must
      preserve all the outer quantifiers of $[[uP]]$ and does not generate any new ones.

      This way, $[[uP]] = [[∃nas.uP']]$, $[[ [uσ]uP ]] = [[ ∃nas.[uσ]uP' ]]$ 
      (assuming $[[nas]]$ does not intersect with the range of $[[uσ]]$)
      and $[[iQ]] = [[ ∃nbs.iQ' ]]$, where either $[[nas]]$ or $[[nbs]]$ is not empty.

      By inversion, $[[ Γ ⊢ [σ][uσ]uP' ≥ iQ' ]]$ for some $[[Γ, nbs ⊢ σ : nas]]$.
      Since $[[σ]]$ and $[[uσ]]$ have disjoint domains,
      and the range of one does not intersect with the domain of the other,
      they commute, i.e. $[[ Γ, nbs ⊢ [uσ][σ]uP' ≥ iQ' ]]$
      (notice that the tree inferring this judgement is 
      a proper subtree of the tree inferring 
      $[[ Γ ⊢ [uσ]uP ≥ iQ ]]$).

      At the next step, 
      the algorithm creates fresh (disjoint with $[[uv uP']]$) 
      unification variables $[[â⁻*]]$, replaces $[[nas]]$ with them in $[[ uP' ]]$,
      and makes the recursive call:
      $[[Γ, nbs; Θ, â⁻*[G, nbs] ⊨ uP0 ≥ iQ' ⫤ SC1]]$,
      (where $[[uP0]] = [[ [â⁻*/nas]uP' ]]$),
      returning $[[SC1 \ {â⁻*}]]$ as the result.

      To show that the recursive call terminates and that 
      $[[(Θ  ⊢  SC1 \ {â⁻*}) ⊢ uσ]]$,
      it suffices to build $[[Θ, â⁻*[G, nbs] ⊢ us0]]$---an extension of $[[uσ]]$ with
      $[[â⁻*]]$ such that $[[Γ, nbs ⊢ [us0]uP0 ≥ iQ]]$.
      Then by the induction hypothesis, $[[(Θ, â⁻*[G, nbs] ⊢ SC1) ⊢ us0]]$,
      and hence, $[[(Θ  ⊢  SC1 \ {â⁻*}) ⊢ uσ]]$, as required.

      Let us construct such a substitution $[[us0]]$:
        \[
            [[ [uσ0]β̂± ]]  = 
            \begin{cases}
               [[ [σ]αi⁻ ]] & \text{if } [[β̂±]] = [[αî⁻]] \in [[â⁻*]] \\
               [[ [uσ]β̂± ]] & \text{if } [[β̂±]] \in [[ uv(uP') ]]
            \end{cases}
       \]
       It is easy to see $[[Θ, â⁻*[Γ, nbs] ⊢ uσ0]]$:
       \begin{enumerate}
            \item for $[[αî⁻]] \in [[â⁻*]]$, $[[ (Θ, â⁻*[Γ, nbs])(αî⁻) ⊢ [uσ0] αî⁻]]$, 
            i.e. $[[ Γ, nbs ⊢ [σ]αi⁻ ]]$ holds since $[[Γ, nbs ⊢ σ : nas]]$,
            \item for $[[β̂±]] \in [[ uv(uP') ]] \subseteq [[dom(Θ)]]$, $[[ (Θ, â⁻*[Γ, nbs])(β̂±) ⊢ [uσ0] β̂± ]]$,
            i.e. $[[Θ(β̂±) ⊢ [uσ] β̂± ]]$ holds since $[[Θ ⊢ uσ]]$.
       \end{enumerate}

       Now, let us show that $[[Γ, nbs ⊢ [us0]uP0 ≥ iQ]]$.
       To do that, we notice that $[[ [uσ0]uP0 ]] = [[ [uσ][σ][nas/â⁻*]uP0 ]]$:
       let us consider an arbitrary variable appearing freely in $[[uP0]]$:
       \begin{enumerate}
        \item if this variable is a metavariable $[[αî⁻]] \in [[â⁻*]]$, then
        $[[ [uσ0]αî⁻ ]] = [[ [σ]αi⁻ ]]$ and 
        $[[ [uσ][σ][nas/â⁻*]αî⁻ ]] = [[ [uσ][σ]αi⁻ ]] = [[ [σ]αi⁻ ]]$,
        \item if this variable is a metavariable $[[β̂±]] \in [[ uv(uP0) \ {â⁻*} ]] = [[uv(uP')]]$, then
        $[[ [uσ0]β̂± ]] = [[ [uσ]β̂± ]]$ and $[[ [uσ][σ][nas/â⁻*]β̂± ]] = [[ [uσ][σ]β̂± ]] = [[ [uσ]β̂± ]]$,
        \item if this variable is a regular variable from $[[fv(uP0)]]$, both substitutions do not change it:
        $[[ uσ0 ]]$, $[[ uσ ]]$ and $[[ nas / â⁻* ]]$ act on metavariables, 
        and $[[σ]]$ is defined on $[[nas]]$, however, $[[{nas} ∩ fv(uP0) = ∅]]$.
       \end{enumerate}
       This way, $[[ [uσ0]uP0 ]] = [[ [uσ][σ][nas/â⁻*]uP0 ]] = [[ [uσ][σ]uP' ]]$,
       and thus, $[[ Γ, nbs ⊢ [uσ0]uP0 ≥ iQ' ]]$.


    \end{caseof}
\end{proof}
