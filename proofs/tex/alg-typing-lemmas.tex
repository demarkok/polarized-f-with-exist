\begin{lemma}[Soundness of typing] \label{lemma:typing-soundness}
    \hfill
    \begin{itemize}
        \item [$+$] If $[[Γ; Φ ⊨ v : iP]]$ then $[[Γ ⊢ iP]]$ and $[[Γ; Φ ⊢ v : iP]]$
        \item [$-$] If $[[Γ; Φ ⊨ c : iN]]$ then $[[Γ ⊢ iN]]$ and $[[Γ; Φ ⊢ c : iN]]$
        \item  For $[[Γ ⊢ Θ]]$ and $[[Γ; Θ ⊢ uN]]$, if $[[Γ; Φ; Θ ⊨ uN ● args ⇒> uM ⫤ Θ'; SC]]$ then
        \begin{enumerate}
            \item $[[Γ ⊢ Θ']]$
            \item $[[Θ]] \subseteq [[Θ']]$
            \item $[[Γ; Θ' ⊢ uM]]$
            \item $[[Θ' ⊢ SC]]$
            \item for any $[[Θ' ⊢ uσ : SC]]$, we have $[[ Γ ; Φ ⊢ [uσ]uN ● args ⇒> [uσ]uM ]]$
        \end{enumerate}
    \end{itemize}
\end{lemma}
\begin{proof}
    We prove it by induction on the typing derivation.
    Let us consider the last rule used to infer the derivation.
    \begin{caseof}
        \item \ruleref{\ottdruleATVarLabel}
        \item \ruleref{\ottdruleATThunkLabel}
        \item \ruleref{\ottdruleATPAnnotLabel}
        \item \ruleref{\ottdruleATNAnnotLabel}
        \item \ruleref{\ottdruleATtLamLabel}
        \item \ruleref{\ottdruleATTLamLabel}
        \item \ruleref{\ottdruleATReturnLabel}
        \item \ruleref{\ottdruleATVarLetLabel}
        \item \ruleref{\ottdruleATAppLetAnnLabel}
        By inversion, we have:
        \begin{enumerate}
            \item $[[c]]$ is $[[let x : iP = v(args); c']]$
            \item $[[Γ ⊢ iP]]$
            \item $[[Γ; Φ ⊨ v : ↓iM]]$
            \item $[[Γ; Φ; · ⊨ uM ● args ⇒> ↑uQ ⫤ Θ; SC1]]$
            \item $[[Γ; Θ ⊨ ↑uQ ≤ ↑iP ⫤ SC2]]$
            \item $[[Θ ⊢ SC1 & SC2 = SC]]$
            \item $[[Γ; Φ, x:iP ⊨ c' : iN]]$
        \end{enumerate}

        By the soundness of constraint merge (\cref{lemma:merge-soundness}), we have 
        $[[Θ ⊢ SC]]$. Let us take $[[uσ]]$ such that $[[Θ ⊢ uσ : SC]]$
        (it exists by \cref{lemma:substitution-existence}). Notice that by the soundness of 
        constraint merge, $[[Θ ⊢ uσ : SC1]]$ and $[[Θ ⊢ uσ : SC2]]$.

        By the induction hypothesis applied to $[[Γ; Φ ⊨ v : ↓iM]]$, we have
        $[[Γ; Φ ⊢ v : ↓iM]]$ and $[[Γ ⊢ ↓iM]]$ (and hence,$[[Γ ; Θ ⊢ uM]]$).

        By the induction hypothesis applied to I$[[Γ; Φ, x:iP ⊨ c' : iN]]$, we have
        $[[Γ; Φ, x:iP ⊢ c' : iN]]$ and $[[Γ ⊢ iN]]$. 

        By the induction hypothesis applied to $[[Γ; Φ; · ⊨ uM ● args ⇒> ↑uQ ⫤ Θ; SC1]]$, we have:
        \begin{enumerate}
            \item \label{typing-soundness:theta-wf} $[[Γ ⊢ Θ]]$,
            \item $[[Γ; Θ ⊢ ↑uQ]]$,
            \item $[[Θ' ⊢ SC1]]$,
            \item for any $[[Θ' ⊢ uσ : SC1]]$, we have $[[ Γ ; Φ ⊢ [uσ]uM ● args ⇒> [uσ]↑uQ ]]$.
            In particular, it holds for the $[[uσ]]$ chosen above. 
        \end{enumerate}

        By the soundness of negative subtyping (\cref{lemma:neg-subtyping-soundness})
        applied to $[[Γ; Θ ⊨ ↑uQ ≤ ↑iP ⫤ SC]]$, we have $[[Γ ⊢ ↑[uσ]uQ ≤ ↑iP]]$.

        To infer $[[Γ ; Φ ⊢ let x : iP = v(args); c' : iN ]]$,
        we apply the corresponding declarative rule \ruleref{\ottdruleDTAppLetAnnLabel}, where
        $[[iQ]]$ is $[[ [uσ]uQ  ]]$. Notice that all the premises were already shown to
        hold above:
        \begin{enumerate}
            \item $[[Γ ⊢ iP]]$ and $[[Γ; Φ ⊢ v : ↓iM]]$ from the inversion,
            \item $[[Γ; Φ ⊢ iM ● args ⇒> ↑[uσ]uQ]]$ holds since $[[ [uσ]↑uQ ]] = [[ ↑[uσ]uQ ]]$,
            \item $[[Γ ⊢ ↑[uσ]uQ ≤ ↑iP]]$ by the soundness of negative subtyping,
            \item $[[Γ; Φ, x:iP ⊢ c' : iN]]$ from the the induction hypothesis.
        \end{enumerate}

        \item \ruleref{\ottdruleATAppLetLabel}
        By the inversion, we have:
        \begin{enumerate}
            \item $[[c]]$ is $[[let x = v(args) ; c']]$
            \item $[[Γ; Φ ⊨ v : ↓iM]]$ 
            \item $[[Γ ; Φ ; · ⊨ uM ● args ⇒> ↑uQ ⫤ Θ; SC]]$
            \item $[[uv uQ ⊆ dom(SC)]]$
            \item $[[SC|uv(uQ) singular]]$
            \item $[[Γ; Φ, x:iQ ⊨ c' : iN]]$
        \end{enumerate}

        By the induction hypothesis applied to $[[Γ; Φ ⊨ v : ↓iM]]$, we have    
        $[[Γ; Φ ⊢ v : ↓iM]]$ and $[[Γ ⊢ ↓iM]]$ (and thus, $[[Γ ; Θ  ⊢ uM]]$).
       
        By the induction hypothesis applied to $[[Γ; Φ, x:iQ ⊨ c' : iN]]$, we have
        $[[Γ ⊢ iN]]$ and $[[Γ; Φ, x:iQ ⊢ c' : iN]]$.

        By the induction hypothesis applied to 
        $[[Γ ; Φ ; · ⊨ uM ● args ⇒> ↑uQ ⫤ Θ; SC]]$, we have:
        \begin{enumerate}
            \item $[[Γ ⊢ Θ]]$
            \item $[[Γ; Θ ⊢ ↑uQ]]$
            \item $[[Θ ⊢ SC]]$
            \item for any $[[Θ ⊢ uσ : SC]]$, we have $[[ Γ ; Φ ⊢ [uσ]uM ● args ⇒> [uσ]↑uQ ]]$, 
                which, since  $[[iM]]$ is ground means $[[ Γ ; Φ ⊢ iM ● args ⇒> ↑[uσ]uQ]]$
        \end{enumerate}

        To infer $[[Γ ; Φ ⊢ let x = v(args) ; c' : iN ]]$, 
        we apply the corresponding 
        declarative rule \ruleref{\ottdruleDTAppLetLabel}.
        Let us show that the premises hold:
        \begin{itemize}
            \item $[[Γ; Φ ⊢ v : ↓iM]]$ holds by the induction hypothesis;
            \item $[[Γ; Φ, x:iQ ⊢ c' : iN]]$ also holds by the induction hypothesis, as noted above;
            \item Let us take an arbitrary substitution $[[uσ]]$ 
                satisfying $[[Θ ⊢ uσ : SC]]$ (it exists by \cref{lemma:constraint-sat}).
                Then $[[Γ; Φ ⊢ iM ● args ⇒> ↑[uσ]uQ ]]$ holds, as noted above;
            \item To show the uniqueness of $[[↑[uσ]uQ]]$,
                we assume that for some other type $[[iK]]$ 
                holds $[[Γ; Φ ⊢ iM ● args ⇒> iK ]]$
                that is $[[Γ; Φ ⊢ [·]uM ● args ⇒> iK ]]$.
                Then by the completeness of typing 
                (\cref{lemma:typing-completeness}),
                there exist $[[uN']]$, $[[Θ']]$, and $[[SC']]$ such that
                \begin{enumerate}
                    \item $[[ Γ; Φ; · ⊨ uM ● args ⇒> uN' ⫤ Θ'; SC' ]]$ and
                    \item there exists a substitution $[[Θ' ⊢ uσ' : SC']]$ such that
                    $[[Γ ⊢ [uσ']uN' ≈ iK]]$.
                \end{enumerate}
            By the determinicity of the typing algorithm (\cref{lemma:typing-determinicity})
            $[[ Γ; Φ; · ⊨ uM ● args ⇒> uN' ⫤ Θ'; SC' ]]$,
            means that $[[SC']]$ is $[[SC]]$, $[[Θ']]$ is $[[Θ]]$, and $[[uN']]$ is
            $[[↑uQ]]$. 
            This way, $[[Γ ⊢ [uσ']↑uQ ≈ iK]]$ for a substitution 
            $[[Θ ⊢ uσ' : SC]]$. 

            It is left to show that $[[Γ ⊢ [uσ']↑uQ ≈ [uσ]↑uQ]]$, 
            then by transitivity of the equivalence, we will have $[[Γ ⊢ [uσ]↑uQ ≈ iK]]$.
            Since $[[Θ ⊢ uσ : SC|uv(uQ)]]$ and $[[Θ ⊢ uσ' : SC|uv(uQ)]]$, 
            and $[[SC|uv(uQ) singular]]$, we have $[[Θ ⊢ uσ ≈ uσ' : dom(SC|uv(uQ))]]$.
            Then since $[[uv(uQ) ⊆ dom(SC)]]$, we have $[[dom(SC|uv(uQ)) = uv(uQ)]]$.
            This way, $[[Θ ⊢ uσ ≈ uσ' : uv(uQ)]]$, which implies
            $[[Γ ⊢ [uσ']↑uQ ≈ [uσ]↑uQ]]$. 
        \end{itemize}

        \item \ruleref{\ottdruleATUnpackLabel}
        By the inversion, we have:
        \begin{enumerate}
            \item $[[c]]$ is $[[let∃ (α⁻, x) = v; c']]$
            \item $[[Γ; Φ ⊨ v : ∃α⁻.iP]]$
            \item $[[Γ, α⁻ ; Φ, x:iP ⊨ c' : iN]]$
            \item $[[Γ ⊢ iN]]$
        \end{enumerate}

        By the induction hypothesis applied to 
        $[[Γ; Φ ⊨ v : ∃α⁻.iP]]$, we have $[[Γ; Φ ⊢ v : ∃α⁻.iP]]$.
        By the induction hypothesis applied to
        $[[Γ, α⁻ ; Φ, x:iP ⊨ c' : iN]]$, we have $[[Γ, α⁻ ; Φ, x:iP ⊢ c' : iN]]$.

        To show $[[Γ; Φ ⊢ let∃ (α⁻, x) = v; c' : iN]]$, we apply the corresponding
        declarative rule \ruleref{\ottdruleDTUnpackLabel}. Let us show that the premises hold:
        \begin{enumerate}
            \item $[[Γ ; Φ ⊢ v : ∃α⁻.iP]]$ holds by the induction hypothesis, as noted above,
            \item $[[Γ, α⁻ ; Φ, x:iP ⊢ c' : iN]]$ also holds by the induction hypothesis,
            \item $[[Γ ⊢ iN]]$ holds by the inversion, as noted above.
        \end{enumerate}

        \item \ruleref{\ottdruleATEmptyAppLabel}
        Then by assumption:
        \begin{itemize}
            \item $[[Γ ⊢ Θ]]$,
            \item $[[Γ; Θ ⊢ uN]]$,
            \item $[[Γ; Φ; Θ ⊨ uN ● · ⇒> uN ⫤ Θ; ·]]$, which by inversion means that $[[uN ≠ ∀pas.uM]]$.
        \end{itemize}

        Let us show the required properties: 
        \begin{enumerate}
            \item $[[Γ ⊢ Θ]]$ holds by assumption,
            \item $[[Θ]] \subseteq [[Θ]]$ holds trivially,
            \item $[[Γ; Θ ⊢ uN]]$ holds by assumption,
            \item $[[Θ ⊢ ·]]$ holds trivially,
            \item for any $[[Θ ⊢ uσ : ·]]$, we have $[[ Γ ; Φ ⊢ [uσ]uN ● · ⇒> [uσ]uN ]]$.
                To show $[[ Γ ; Φ ⊢ [uσ]uN ● · ⇒> [uσ]uN ]]$, we apply the corresponding 
                declarative rule \ruleref{\ottdruleDTEmptyAppLabel}. 
        \end{enumerate}

   
        \item \ruleref{\ottdruleATArrowAppLabel}\\
        By assumption:
        \begin{enumerate}
            \item $[[Γ ⊢ Θ]]$,
            \item $[[Γ; Θ ⊢ uQ → uN]]$,
            \item $[[Θ ⊢ SC1 & SC2 = SC]]$,
            \item $[[Γ; Φ; Θ ⊨ uQ → uN ● v , args ⇒> uM ⫤ Θ'; SC]]$, 
                and by inversion: 
                \begin{enumerate}
                    \item $[[Γ; Φ ⊨ v : iP]]$,
                        and by the induction hypothesis applied to this judgment,
                        $[[Γ; Φ ⊢ v : iP]]$,
                    \item $[[Γ; Θ ⊨ uQ ≥ iP ⫤ SC1]]$,
                        and by the soundness of subtyping:
                        $[[Θ ⊢ SC]]$ and
                        for any $[[Θ ⊢ uσ : SC1]]$, we have $[[Γ ⊢ [uσ]uQ ≥ iP]]$,
                    \item $[[Γ; Φ; Θ ⊨ uN ● args ⇒> uM ⫤ Θ'; SC2]]$,
                        and by the induction hypothesis applied to this judgment,
                        \begin{enumerate}
                            \item $[[Γ ⊢ Θ']]$,
                            \item $[[Θ ⊆ Θ']]$,
                            \item $[[Γ; Θ' ⊢ uM]]$,
                            \item $[[Θ' ⊢ SC2]]$,
                            \item for any $[[Θ' ⊢ uσ : SC2]]$, we have 
                                $[[ Γ ; Φ ⊢ [uσ]uN ● args ⇒> [uσ]uM ]]$.
                        \end{enumerate}
                \end{enumerate}
        \end{enumerate}

        Let us show the required properties:
        \begin{enumerate}
            \item $[[Γ ⊢ Θ']]$ is shown above,
            \item $[[Θ ⊆ Θ']]$ is shown above,
            \item $[[Γ; Θ' ⊢ uM]]$ is shown above,
            \item $[[Θ' ⊢ SC]]$ holds:  
            $[[Θ ⊢ SC1]]$ implies $[[Θ' ⊢ SC1]]$,
            then we apply the soundness of constraint merge (\cref{lemma:merge-soundness})
            to $[[Θ' ⊢ SC1 & SC2]]$,
            \begin{enumerate}
                \item $[[Θ' ⊢ SC1]]$,
                \item for any $[[Θ' ⊢ uσ : SC]]$, $[[Θ' ⊢ uσ : SCi]]$ holds;
            \end{enumerate}
            \item suppose that $[[Θ' ⊢ uσ : SC]]$. Then to 
                show $[[ Γ ; Φ ⊢ [uσ](uQ → uN) ● v , args ⇒> [uσ]uM ]]$, 
                that is $[[ Γ ; Φ ⊢ [uσ]uQ → [uσ]uN ● v , args ⇒> [uσ]uM ]]$,
                we apply the corresponding declarative rule \ruleref{\ottdruleDTArrowAppLabel}.
                Let us show the required premises:
                \begin{enumerate}
                    \item $[[Γ; Φ ⊢ v : iP]]$ holds as shown above,
                    \item $[[Γ ⊢ [uσ]uQ ≥ iP]]$ holds by the soundness of subtyping 
                        as noted above,
                        since $[[Θ' ⊢ uσ : SC]]$ implies $[[Θ ⊢ uσ : SC1]]$.
                    \item $[[Γ; Φ ⊢ [uσ]uN ● args ⇒> [uσ]uM]]$ holds by the induction hypothesis
                        as shown above,
                        since $[[Θ' ⊢ uσ : SC]]$ implies $[[Θ' ⊢ uσ : SC2]]$.
                \end{enumerate}
        \end{enumerate}

        \item \ruleref{\ottdruleATForallAppLabel}\\
        By assumption:
        \begin{enumerate}
            \item $[[Γ ⊢ Θ]]$,
            \item $[[Γ; Θ ⊢ ∀pas.uN]]$,
            \item $[[Γ; Φ; Θ ⊨ ∀pas.uN ● args ⇒> uM ⫤ Θ'; SC]]$, which by inversion means
                $[[args ≠ ·]]$ and $[[Γ; Φ; Θ, â⁺*[Γ] ⊨ [â⁺*/pas]uN ● args ⇒> uM ⫤ Θ'; SC]]$.
                It is easy to see that the induction hypothesis is applicable to the latter judgment:
                $[[Γ ⊢ Θ, â⁺*[Γ] ]]$ is implied by $[[Γ ⊢ Θ]]$, and $[[Γ; Θ, â⁺*[Γ] ⊢ [â⁺*/pas]uN]]$
                is holds since $[[Γ; Θ ⊢ ∀pas.uN]]$.
                Let us apply the inductive hypothesis to the latter judgment to obtain:
                \begin{enumerate}
                    \item $[[Γ ⊢ Θ']]$,
                    \item $[[Θ, â⁺*[Γ] ⊆ Θ']]$,
                    \item $[[Γ; Θ' ⊢ uM]]$,
                    \item $[[Θ' ⊢ SC]]$,
                    \item for any $[[Θ' ⊢ uσ : SC]]$, we have $[[ Γ ; Φ ⊢ [uσ][â⁺*/pas]uN ● args ⇒> [uσ]uM ]]$.
                \end{enumerate}
        \end{enumerate}

        Let us show the required properties:
        \begin{enumerate}
            \item $[[Γ ⊢ Θ']]$ is shown above,
            \item $[[Θ ⊆ Θ']]$ since $[[Θ, â⁺*[Γ] ⊆ Θ']]$,
            \item $[[Γ; Θ' ⊢ uM]]$ is shown above,
            \item $[[Θ' ⊢ SC]]$ is shown above,
            \item let us assume $[[Θ' ⊢ uσ : SC]]$
            Then to show $[[ Γ ; Φ ⊢ [uσ]∀pas.uN ● args ⇒> [uσ]uM ]]$,
            we apply the corresponding declarative rule \ruleref{\ottdruleDTForallAppLabel}
            with substitution $[[ Γ ⊢ σ : pas ]]$  defined in the following way:
            $[[ [σ]αi⁺ ]] = [[ [uσ]αî⁺ ]]$.

            Let us show that its premises hold:
            \begin{enumerate}
                \item $[[Γ ⊢ σ : pas]]$, i.e.
                $[[ Γ ⊢ [σ] αi⁺ ]]$ holds since $[[ Θ' ⊢ uσ ]]$ and $[[Γ ⊢ Θ']]$;
                \item $[[Γ; Φ ⊢ [σ][uσ]uN ● args ⇒> [uσ]uM ]]$ 
                    holds by rewriting 
                    $[[ Γ ; Φ ⊢ [uσ][â⁺*/pas]uN ● args ⇒> [uσ]uM ]]$
                    using equality $[[ [uσ][â⁺*/pas]uN = [σ][uσ]uN ]]$:
                    \begin{enumerate}
                        \item for $[[ αi⁺ ]] \in [[ pas ]]$, $[[ [uσ][â⁺*/pas] αi⁺ ]] = [[ [uσ]αî⁺ ]] = [[ [σ]αi⁺ ]] = [[ [σ][uσ]αi⁺ ]]$,
                        \item for $[[ β̂± ]] \in [[ dom(uσ) ]]$, $[[ [uσ][â⁺*/pas]β̂±  ]] = [[ [uσ]β̂±  ]] = [[ [σ][uσ]β̂± ]] $, 
                            where the latter equality holds since $[[ {pas} ∩ {Γ} = ∅ ]]$.
                    \end{enumerate}

                \item $[[args ≠ ·]]$ holds by assumption
            \end{enumerate}
        \end{enumerate}
    \end{caseof}
\end{proof}

\begin{lemma} \label{lemma:app-inf-equ-stable}
    If $[[Γ; Φ ⊢ iN1 ● args ⇒> iM]]$ and $[[Γ ⊢ iN1 ≈ iN2]]$ 
    then $[[Γ; Φ ⊢ iN2 ● args ⇒> iM]]$.
\end{lemma}
\begin{proof}
    Induction on $[[Γ; Φ ⊢ iN1 ● args ⇒> iM]]$.
    Let us consider the last rule used in the derivation:
    \begin{caseof}
        \item \ruleref{\ottdruleDTEmptyAppLabel}
            Then $[[args = ·]]$ and by inversion of $[[Γ; Φ ⊢ iN1 ● · ⇒> iM]]$,
            $[[Γ ⊢ iN1 ≈ iM]]$.
            By transitivity \cref{corollary:equivalence-transitivity}, 
            $[[Γ ⊢ iN2 ≈ iM]]$, and then \ruleref{\ottdruleDTEmptyAppLabel} is applicable to infer
            $[[Γ; Φ ⊢ iN2 ● · ⇒> iM]]$.
        \item \ruleref{\ottdruleDTArrowAppLabel}
            Then we are proving that  
            $[[Γ; Φ ⊢ (iQ1 → iN1) ● v, args ⇒> iM]]$ and $[[Γ ⊢ (iQ1 → iN1) ≈ (iQ2 → iN2)]]$
            imply $[[Γ; Φ ⊢ (iQ2 → iN2) ● v, args ⇒> iM]]$.
            
            By inversion, $[[Γ ⊢ (iQ1 → iN1) ≈ (iQ2 → iN2)]]$
            means $[[Γ ⊢ iQ1 ≈ iQ2]]$ and $[[Γ ⊢ iN1 ≈ iN2]]$.

            By inversion of $[[Γ; Φ ⊢ (iQ1 → iN1) ● v, args ⇒> iM]]$:
            \begin{enumerate}
                \item $[[Γ ; Φ ⊢ v : iP]]$
                \item $[[Γ ⊢ iQ1 ≥ iP]]$,
                    and then by transitivity \cref{corollary:subtyping-transitivity},
                    $[[Γ ⊢ iQ2 ≥ iP]]$;
                \item $[[Γ ; Φ ⊢ iN1 ● args ⇒> iM]]$, 
                    and then by induction hypothesis, $[[Γ ; Φ ⊢ iN2 ● args ⇒> iM]]$.
            \end{enumerate}

            Since we have $[[Γ ; Φ ⊢ v : iP]]$, $[[Γ ⊢ iQ2 ≥ iP]]$ and 
            $[[Γ ; Φ ⊢ iN2 ● args ⇒> iM]]$, we can apply \ruleref{\ottdruleDTArrowAppLabel}
            to infer $[[Γ; Φ ⊢ (iQ2 → iN2) ● v, args ⇒> iM]]$.
        \item \ruleref{\ottdruleDTArrowAppLabel}
            Then we are proving that 
            $[[Γ ; Φ ⊢ ∀pas1.iN1' ● args ⇒> iM]]$ and $[[Γ ⊢ ∀pas1.iN1' ≈ iN2]]$
            imply $[[Γ ; Φ ⊢ iN2 ● args ⇒> iM]]$.

            Let us write $[[iN2]]$ as $[[∀pas2.iN2']]$ with potentially empty
            $[[pas2]]$.

            By \cref{lemma:equiv-completeness}, $[[Γ ⊢ ∀pas1.iN1' ≈ ∀pas2.iN2']]$ 
            means $[[∀pas1.iN1' ≈ ∀pas2.iN2']]$, and by inversion: 
            \begin{enumerate}
                \item $[[{pas2} ∩ fv iN1 = ∅]]$,
                \item there exists a bijection 
                    $[[mu : ({pas2} ∩ fv iN2') ↔ ({pas1} ∩ fv iN1')]]$
                    such that $[[iN1' ≈ [mu] iN2']]$.
            \end{enumerate}

            By inversion of $[[Γ ; Φ ⊢ ∀pas1.iN1' ● args ⇒> iM]]$:
            \begin{enumerate}
                \item $[[Γ ⊢ σ : pas1]]$        
                \item $[[Γ ; Φ ⊢ [σ]iN1' ● args ⇒> iM]]$
                \item $[[args ≠ ·]]$
            \end{enumerate}

            Let us construct $[[Γ ⊢ σ0 : pas2]]$ in the following way:
            $$
            \begin{cases}
                [[ [σ0]α⁺ =  [σ][mu]α⁺ ]] & \text{if } [[α⁺]] \in [[ {pas2} ∩ fv iN2' ]] \\
                [[ [σ0]α⁺ =  ∃β⁻.↓β⁻ ]] & \text{otherwise (the type does not matter here)} \\
            \end{cases}
            $$

            Then to infer $[[Γ ; Φ ⊢ iN2 ● args ⇒> iM]]$, we 
            apply \ruleref{\ottdruleDTArrowAppLabel} with $[[σ0]]$. 
            Let us show the required premises:
            \begin{enumerate}
                \item $[[Γ ⊢ σ0 : pas2]]$ by construction;
                \item $[[args ≠ ·]]$ as noted above;
                \item To show $[[Γ ; Φ ⊢ [σ0]iN2' ● args ⇒> iM]]$,
                Notice that $[[ [σ0]iN2' = [σ][mu]iN2' ]]$   
                and since $[[ [mu]iN2' ≈ iN1' ]]$, $[[ [σ][mu]iN2' ≈ [σ]iN1' ]]$.
                This way, by \cref{lemma:equiv-soundness}, $[[Γ ⊢ [σ]iN1' ≈ [σ0]iN2']]$.
                Then the required judgement holds by the induction hypothesis
                applied to $[[Γ ; Φ ⊢ [σ]iN1' ● args ⇒> iM]]$.
            \end{enumerate}
    \end{caseof}
\end{proof}

\begin{lemma}[Completeness of Typing]
    \label{lemma:typing-completeness}
    \hfill
    \begin{itemize}
        \item [$+$] If $[[Γ; Φ ⊢ v : iP]]$ then 
            there exists $[[iP']]$ such that $[[Γ; Φ ⊨ v : iP']]$ and 
            $[[Γ ⊢ iP' ≈ iP]]$;
        \item [$-$] If $[[Γ; Φ ⊢ c : iN]]$ then 
            there exists $[[iN']]$ such that $[[Γ; Φ ⊨ c : iN']]$
            and $[[Γ ⊢ iN' ≈ iN]]$;
        \item [$\bullet$] Suppose that 
            $[[Γ; Φ ⊢ [uσ]uN ● args ⇒> iM]]$ holds for some
            $[[Γ ⊢ Θ]]$,
            $[[Γ; Θ ⊢ uN]]$ (free from negative metavariables, that is $[[α̂⁻]] \notin [[uv uN]]$), 
            $[[Θ ⊢ uσ]]$, and $[[Γ ⊢ iM]]$. Then
            there exist $[[uM']]$, $[[Θ']]$, and $[[SC]]$ such that
            \begin{enumerate}
                \item $[[ Γ; Φ; Θ ⊨ uN ● args ⇒> uM' ⫤ Θ'; SC ]]$ and
                \item for any $[[Θ ⊢ uσ]]$ and $[[Γ ⊢ iM]]$
                    such that $[[Γ; Φ ⊢ [uσ]uN ● args ⇒> iM]]$, 
                    there exists $[[uσ']]$ such that 
                    \begin{enumerate}
                        \item $[[Θ' ⊢ uσ' : SC]]$,
                        \item $[[Θ ⊢ uσ' ≈ uσ : dom(Θ)]]$, and 
                        \item $[[Γ ⊢ [uσ']uM' ≈ iM]]$.
                    \end{enumerate}
            \end{enumerate}
    \end{itemize}
\end{lemma}
\begin{proof}
    By induction on the typing derivation.
    Let us consider the last rule applied to infer the derivation.
    % First, let us consider the case of $[[Γ; Φ ⊢ [uσ]uN ● args ⇒> iM]]$, 
    % when the substitution $[[uσ]]$ can change the outer shape of $[[uN]]$.
    % after that, we consider the last rule applied to infer the derivation
    % (assuming that $[[ uσ ]]$ preserves the outer shape of the type it applied to).
    \begin{caseof}

        \item \ruleref{\ottdruleDTAppLetLabel}\\
            Then by assumption, 
            $[[c]]$ is $[[let x = v(args); c']]$. Then by inversion of
            $[[Γ ; Φ ⊢ let x = v(args); c' : iN]]$: 
            \begin{itemize}
                \item $[[Γ ; Φ ⊢ v : ↓iM]]$, 
                    which by the induction hypothesis means 
                    that there exists $[[iM2]]$ such that
                    $[[Γ; Φ ⊨ v : ↓iM2]]$ and $[[Γ ⊢ ↓iM2 ≈ ↓iM]]$;
                \item $[[Γ ; Φ ⊢ iM ● args ⇒> ↑iQ uniq]]$. 
                    Then by \cref{lemma:app-inf-equ-stable}, since 
                    $[[Γ ⊢ iM ≈ iM2]]$,
                    $[[Γ ; Φ ⊢ iM2 ● args ⇒> ↑iQ]]$.
                    Then by the induction hypothesis applied to 
                    $[[Γ ; Φ ⊢ [·]uM2 ● args ⇒> ↑iQ]]$
                    means that there exist $[[uM']]$, $[[Θ']]$, and $[[SC]]$ such that
                    (considering $[[iM]]$ is ground):
                    \begin{enumerate}
                        \item $[[ Γ; Φ; · ⊨ uM2 ● args ⇒> uM' ⫤ Θ'; SC ]]$ and
                        \item for any  $[[Γ ⊢ iM'']]$
                            such that $[[Γ; Φ ⊢ [·]uM ● args ⇒> iM'']]$,
                            and in particular, for $[[Γ ⊢ ↑iQ]]$,
                            there exists $[[uσ']]$ such that 
                            \begin{enumerate}
                                \item $[[Θ' ⊢ uσ' : SC]]$, and 
                                \item $[[Γ ⊢ [uσ']uM' ≈ ↑iQ]]$.
                            \end{enumerate}
                    \end{enumerate}
                \item $[[Γ; Φ, x:iQ ⊢ c' : iN]]$
                    \label{point:typing-completeness:AppLet:body},
                    which by the induction hypothesis means
                    that there exists $[[iN2]]$ such that
                    $[[Γ; Φ, x:iQ ⊨ c' : iN2]]$ and $[[Γ ⊢ iN2 ≈ iN]]$.
            \end{itemize}

            Let us show that $[[iN2]]$ from \ref{point:typing-completeness:AppLet:body}
            satisfies the required properties.
            $[[Γ ⊢ iN2 ≈ iN]]$ holds as noted above. 
            To infer $[[Γ ; Φ ⊢ let x = v(args); c' : iN2]]$, 
            let us apply the corresponding algorithmic rule (\ruleref{\ottdruleATAppLetLabel})
            with the type of $[[v]]$ being $[[↓iM2]]$:
            \begin{enumerate}
                \item $[[Γ ; Φ ⊨ v : ↓iM2]]$ holds as noted above;

                \item $[[Γ; Φ ; · ⊨ uM2 ● args ⇒> ↑uQ ⫤ Θ; SC]]$

                \item $[[uv uQ ⊆ dom(SC)]]$
                
                \item $[[SC | uv uQ singular]]$ 

                \item $[[Γ; Φ, x:iQ ⊨ c' : iN2]]$ holds as noted above.
            \end{enumerate}









        \item \ruleref{\ottdruleDTForallAppLabel}\\
            Since $[[uN]]$ cannot be a metavariable,  
            if $[[ [uσ]uN ]]$ starts from $[[∀]]$,
            so does $[[uN]]$. This way,
            $[[uN = ∀pas.uN1]]$.
            Then by assumption:
            \begin{enumerate}
                \item $[[Γ; Θ ⊢ ∀pas.uN1]]$ is free from negative metavariables, 
                    and then $[[Γ, pas; Θ ⊢ uN1]]$ is free from negative metavariables;
                \item $[[Θ ⊢ uσ]]$;
                \item $[[Γ ⊢ iM]]$;
                \item $[[Γ; Φ ⊢ [uσ]∀pas.uN1 ● args ⇒> iM]]$, 
                    \label{point:typing-completeness-forall-app-inversion}
                    that is $[[Γ; Φ ⊢ (∀pas.[uσ]uN1) ● args ⇒> iM]]$.
                    Then by inversion there exists $[[σ]]$ such that 
                    \begin{enumerate}
                        \item $[[Γ ⊢ σ : pas]]$;
                        \item $[[args ≠ ·]]$; and
                        \item $[[Γ ; Φ ⊢ [σ][uσ]uN1 ● args ⇒> iM]]$.
                            \label{point:typing-completeness-forall-app-inversion-2}
                            Notice that $[[σ]]$ and $[[uσ]]$ commute because 
                            the codomain of $[[σ]]$ does not contain
                            metavariables (and thus, does not intersect with 
                            the domain of $[[uσ]]$), and the codomain of $[[uσ]]$ is 
                            $[[Γ]]$ and does not intersect with $[[pas]]$---the domain of $[[σ]]$.

                            Let us construct $[[uN0]]$ = $[[ [â⁺*/pas]uN1 ]]$
                            and $[[Θ, â⁺*[Γ] ⊢ uσ0]]$ defined as
                            $$
                            \begin{cases}
                                [[ [uσ0]αî⁺ = [σ]αi⁺ ]] & \text{for $[[αî⁺]] \in  [[â⁺*]]$ }\\
                                [[ [uσ0]β̂± = [uσ]β̂± ]] & \text{for $[[β̂±]] \in [[dom(Θ)]]$}
                            \end{cases}
                            $$

                            Then it is easy to see that $[[ [uσ0][â⁺*/pas]uN1 = [σ][uσ]uN1 ]]$
                            because this substitution compositions coincide on 
                            $[[ {pas} ∪ dom(Θ) ]]$, their domain.
                            In other words, $[[ [uσ0]uN0 = [σ][uσ]uN1 ]]$.

                            Then let us apply the induction hypothesis
                            to $[[Γ; Φ ⊢ [uσ0]uN0 ● args ⇒> iM]]$ and obtain 
                            $[[uM']]$, $[[Θ']]$, and $[[SC]]$ such that
                            \begin{itemize}
                                \item $[[ Γ; Φ; Θ, â⁺*[Γ] ⊨ uN0 ● args ⇒> uM' ⫤ Θ'; SC ]]$ and
                                \item \label{point:typing-completeness-forall-app-inversion-3}
                                for any $[[Θ, â⁺*[Γ]  ⊢ uσ0]]$ and $[[Γ ⊢ iM]]$
                                    such that $[[Γ; Φ ⊢ [uσ0]uN0 ● args ⇒> iM]]$, 
                                    there exists $[[uσ0']]$ such that 
                                \begin{enumerate}
                                    \item $[[Θ' ⊢ uσ0' : SC]]$,
                                    \item $[[Θ, â⁺*[Γ] ⊢ uσ0' ≈ uσ0 : dom(Θ) ∪ {â⁺*}]]$, and 
                                    \item $[[Γ ⊢ [uσ0']uM' ≈ iM]]$.
                                \end{enumerate}
                            \end{itemize}
                    \end{enumerate}
            \end{enumerate}
            Let us take $[[uM']]$, $[[Θ']]$, and $[[SC]]$ from the induction hypothesis
            (\ref{point:typing-completeness-forall-app-inversion-2}) and show they 
            satisfy the required properties.
            \begin{enumerate}
                \item to infer $[[ Γ; Φ; Θ ⊨ ∀pas.uN1 ● args ⇒> uM' ⫤ Θ'; SC ]]$
                    we apply the corresponding algorithmic rule \ruleref{\ottdruleATForallAppLabel},
                    not that the required premises hold, as noted above:
                    \begin{enumerate}
                        \item $[[args ≠ ·]]$, and 
                        \item $[[Γ; Φ; Θ, â⁺*[Γ] ⊨ [â⁺*/pas]uN1 ● args ⇒> uM' ⫤ Θ'; SC]]$
                            can be rewritten as 
                            $[[ Γ; Φ; Θ, â⁺*[Γ] ⊨ uN0 ● args ⇒> uM' ⫤ Θ'; SC ]]$.
                    \end{enumerate}
                \item Let us take and arbitrary $[[Θ ⊢ uσ]]$ and $[[Γ ⊢ iM]]$
                    and assume $[[Γ; Φ ⊢ [uσ]∀pas.uN1  ● args ⇒> iM]]$. 
                    Then the same reasoning as in 
                    \ref{point:typing-completeness-forall-app-inversion-2}
                    applies. In particular, we construct 
                    $[[Θ, â⁺*[Γ] ⊢ uσ0]]$ as an extension of $[[uσ]]$
                    and obtain 
                    $[[Γ; Φ ⊢ [uσ0]uN0 ● args ⇒> iM]]$.

                    It means, we can apply the property inferred from the induction hypothesis 
                    (\ref{point:typing-completeness-forall-app-inversion-3})
                    to obtain $[[uσ0']]$ such that 
                    \begin{enumerate}
                        \item $[[Θ' ⊢ uσ0' : SC]]$,
                        \item $[[Θ, â⁺*[Γ] ⊢ uσ0' ≈ uσ0 : dom(Θ) ∪ {â⁺*}]]$, and 
                        \item $[[Γ ⊢ [uσ0']uM' ≈ iM]]$.
                    \end{enumerate}

                    Let us show that the obtained $[[uσ0']]$ satisfies the required properties.
                    \begin{enumerate}
                        \item $[[Θ' ⊢ uσ0' : SC]]$ holds as shown,
                        \item $[[Γ ⊢ [uσ0']uM' ≈ iM]]$ holds as shown,
                        \item $[[Θ ⊢ uσ0' ≈ uσ : dom(Θ)]]$,
                            holds. Let us take an arbitrary 
                            $[[β̂±]] \in [[dom(Θ)]] \subseteq [[dom(Θ) ∪ {â⁺*}]]$. Then 
                            since $[[Θ, â⁺*[Γ] ⊢ uσ0' ≈ uσ0 : dom(Θ) ∪ {â⁺*}]]$, 
                            we have $[[ [uσ0']β̂±  = [uσ0]β̂± ]]$ and 
                            by definition of $[[uσ0]]$, $[[ [uσ0]β̂±  = [uσ]β̂± ]]$.
                    \end{enumerate}
            \end{enumerate}
            
        \item \ruleref{\ottdruleDTArrowAppLabel}\\
            Since $[[uN]]$ cannot be a metavariable,  
            if the shape of $[[ [uσ]uN ]]$ is an arrow, 
            so is the shape of $[[uN]]$. This way, 
            $[[uN = uQ → uN1]]$.
            Then by assumption:
            \begin{enumerate}
                \item $[[Γ; Θ ⊢ uQ → uN1]]$ is free from negative metavariables;
                \item $[[Θ ⊢ uσ]]$;
                \item $[[Γ ⊢ iM]]$;
                \item $[[Γ; Φ ⊢ [uσ](uQ → uN1) ● v, args ⇒> iM]]$, 
                    \label{point:typing-completeness-arrow-app-inversion}
                    that is $[[Γ; Φ ⊢ ([uσ]uQ → [uσ]uN1) ● v, args ⇒> iM]]$,
                    and by inversion:
                    \begin{enumerate}
                        \item $[[Γ; Φ ⊢ v : iP]]$,
                            and by the induction hypothsis, 
                            $[[Γ; Φ ⊨ v : iP']]$ for some $[[iP']]$
                            such that $[[Γ ⊢ iP' ≈ iP]]$;
                        \item $[[Γ ⊢ [uσ]uQ ≥ iP]]$, 
                            which by transitivity (\cref{lemma:subtyping-transitivity}) means 
                            $[[Γ ⊢ [uσ]uQ ≥ iP']]$,
                            and then by completeness of subtyping 
                            (\cref{lemma:pos-subt-completeness}),
                            $[[ Γ; Θ ⊨ uQ ≥ iP' ⫤ SC1 ]]$, 
                            for some $[[Θ ⊢ SC1]]$, and moreover, $[[Θ ⊢ uσ : SC1]]$;
                        \item $[[Γ; Φ ⊢ [uσ]uN1 ● args ⇒> iM]]$. 
                            \label{point:completeness-arrow-app-ih}
                            Notice that the induction hypothesis is applicable to this case:
                            $[[Γ ; Θ ⊢ uN1]]$ is free from negative metavariables because
                            so is $[[uQ → uN1]]$. This way, there exist 
                            $[[uM']]$, $[[Θ']]$, and $[[SC2]]$ such that 
                            \begin{enumerate}
                                \item $[[ Γ; Φ; Θ ⊨ uN1 ● args ⇒> uM' ⫤ Θ'; SC2 ]]$
                                    and then by the soundness of typing 
                                    (i.e. the induction hypothesis), 
                                    \begin{enumerate}
                                        \item $[[Θ ⊆ Θ']]$
                                        \item $[[Γ; Θ' ⊢ uM']]$
                                    \end{enumerate}
                                \item  \label{point:new-subdst}
                                    for any $[[Θ ⊢ uσ]]$ and $[[Γ ⊢ iM]]$
                                    such that $[[Γ; Φ ⊢ [uσ]uN1 ● args ⇒> iM]]$, 
                                    there exists $[[uσ']]$ such that 
                                    \begin{enumerate}
                                        \item $[[Θ' ⊢ uσ' : SC2]]$,
                                        \item $[[Θ ⊢ uσ' ≈ uσ : dom(Θ)]]$, and 
                                        \item $[[Γ ⊢ [uσ']uM' ≈ iM]]$.
                                    \end{enumerate}
                            \end{enumerate}
                    \end{enumerate}
            \end{enumerate}

            Let us take $[[Θ ⊢ uσ]]$ and $[[iM]]$
            and construct $[[Θ' ⊢ uσ']]$ 
            by the induction hypothesis (\ref{point:new-subdst}).
            Then $[[Θ' ⊢ uσ' : SC2]]$ and $[[Θ' ⊢ uσ' : SC1]]$ 
            holds and since $[[Θ ⊢ uσ : SC1]]$ and $[[Θ ⊢ uσ' ≈ uσ : dom(Θ)]]$.
            Then by the completeness of constraint merge 
            (\cref{lemma:merge-completeness}),
            $[[Θ' ⊢ SC1 & SC2 = SC]]$ exists, $[[Θ' ⊢ SC]]$, and 
            $[[Θ' ⊢ uσ : SC]]$.

            To show the required properties, we take
            $[[uM']]$ and $[[Θ']]$ from the induction hypothesis (\ref{point:new-subdst}), 
            and $[[SC]]$ defined above. Then
            \begin{enumerate}
                \item $[[ Γ; Φ; Θ ⊨ uQ → uN1 ● v,args ⇒> uM' ⫤ Θ'; SC ]]$
                    is inferred by \ruleref{\ottdruleATArrowAppLabel}:
                    \begin{enumerate}
                        \item $[[Γ; Φ ⊨ v : iP']]$ as noted above,
                        \item $[[Γ; Θ ⊨ uQ ≥ iP' ⫤ SC1]]$ as noted above,
                        \item $[[Γ; Φ; Θ ⊨ uN1 ● args ⇒> uM' ⫤ Θ'; SC2]]$ as noted above;
                    \end{enumerate}
                \item let us take an arbitrary $[[Θ ⊢ uσ0]]$ and $[[Γ ⊢ iM0]]$
                    such that $[[Γ; Φ ⊢ [uσ0](uQ → uN1) ● v,args ⇒> iM0]]$.
                    Then by inversion
                    of $[[Γ; Φ ⊢ [uσ0]uQ → [uσ0]uN1 ● v,args ⇒> iM0]]$,
                    we have the same properties as in 
                    \ref{point:typing-completeness-arrow-app-inversion}.
                    In particular,
                    \begin{itemize}
                        \item $[[Γ; Φ ⊢ [uσ0]uN1 ● args ⇒> iM0]]$. 
                            Then by \ref{point:new-subdst}, there exists $[[uσ']]$ such that 
                            \begin{enumerate}
                                \item $[[Θ' ⊢ uσ' : SC2]]$,
                                \item $[[Θ ⊢ uσ' ≈ uσ0 : dom(Θ)]]$, and 
                                \item $[[Γ ⊢ [uσ']uM' ≈ iM0]]$.
                            \end{enumerate}
                        \item $[[Γ ⊢ [uσ0]uQ ≥ iP']]$
                            and by the completeness of subtyping 
                            (\cref{lemma:pos-subt-completeness}),
                            $[[ Θ ⊢ uσ0 : SC1 ]]$.
                    \end{itemize}
                    This way,
                    \begin{itemize}
                        \item $[[Θ ⊢ uσ' ≈ uσ0 : dom(Θ)]]$ holds as noted above;
                        \item $[[Θ' ⊢ uσ' : SC1]]$ holds because $[[Θ ⊢ uσ0 : SC1]]$ and 
                            $[[Θ ⊢ uσ' ≈ uσ0 : dom(Θ)]]$, 
                            and $[[Θ' ⊢ uσ' : SC1]]$ together with $[[Θ' ⊢ uσ' : SC2]]$  
                            implies $[[Θ' ⊢ uσ' : SC]]$ by the completeness of constraint merge 
                            (\cref{lemma:merge-completeness}); and
                        \item $[[Γ ⊢ [uσ']uM' ≈ iM0]]$ holds as noted above.
                    \end{itemize}
            \end{enumerate}

        \item \ruleref{\ottdruleDTEmptyAppLabel}\\
            By assumption: 
            \begin{enumerate}
                \item $[[Γ; Θ ⊢ uN]]$,
                \item $[[Θ ⊢ uσ]]$,
                \item $[[Γ; Φ ⊢ [uσ]uN ● · ⇒> [uσ]uN ]]$.
            \end{enumerate}
            Then we can apply the corresponding algorithmic rule
            \ruleref{\ottdruleATEmptyAppLabel} to infer
            $[[ Γ; Φ; Θ ⊨ uN ● · ⇒> uN ⫤ Θ; · ]]$.
            Let us show the required properties. 
            Let us take an arbitrary 
            $[[Θ ⊢ uσ1]]$ and $[[Γ ⊢ iM]]$
            such that $[[Γ; Φ ⊢ [uσ1]uN ● · ⇒> iM]]$. 
            Then we can take $[[uσ' = uσ1]]$:
            \begin{enumerate}
                \item $[[Θ ⊢ uσ' : ·]]$ holds vacuously,
                \item $[[Θ ⊢ uσ' ≈ uσ1 : dom(Θ)]]$ holds by reflexivity of equivalence,
                \item $[[Γ ⊢ [uσ']uN ≈ iM]]$ or equivalently, 
                    $[[Γ ⊢ [uσ]uN ≈ iM]]$ holds because 
                    $[[Γ; Φ ⊢ [uσ1]uN ● · ⇒> iM]]$ can only be inferred by 
                    \ruleref{\ottdruleDTEmptyAppLabel}, meaning 
                        $[[ [uσ1]uN =  iM ]]$.
            \end{enumerate}
        \item \ruleref{\ottdruleDTVarLabel}\\
        \item \ruleref{\ottdruleDTThunkLabel}\\
        \item \ruleref{\ottdruleDTPAnnotLabel}\\
        \item \ruleref{\ottdruleDTtLamLabel}\\
        \item \ruleref{\ottdruleDTTLamLabel}\\
        \item \ruleref{\ottdruleDTReturnLabel}\\
        \item \ruleref{\ottdruleDTVarLetLabel}\\
        \item \ruleref{\ottdruleDTAppLetAnnLabel}\\
        \item \ruleref{\ottdruleDTUnpackLabel}\\
        \item \ruleref{\ottdruleDTNAnnotLabel}\\
    \end{caseof}
\end{proof}
