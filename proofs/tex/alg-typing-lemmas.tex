\lemmaTypingDeterminicity*
\begin{proof}
    We show it by structural induction on the inference tree.
    Notice that the last rule used to infer the judgement is uniquely
    determined by the input, and that each premise
    of each inference rule is deterministic by the corresponding 
    observation.
\end{proof}

Let us extend the declarative typing metric (\cref{def:decl-typing-metric}) 
to the algorithmic typing.

\begin{definition}[Size of an Algorithmic Judgement]
    \label{def:algorithmic-typing-size}
    For an algorithmic typing judgement $J$
    let us define a metrics $\size{J}$ as a pair of numbers 
    in the following way:
    \begin{itemize}
        \item [$+$] $\size{[[Γ ; Φ ⊨ v : iP]]} = (\size{[[v]]}, 0)$;
        \item [$-$] $\size{[[Γ ; Φ ⊨ c : iN]]} = (\size{[[c]]}, 0)$;
        \item [$\bullet$] $\size{[[Γ ; Φ ; Θ ⊨ uN ● args ⇒> uM ⫤ Θ'; SC]]} = 
            (\size{[[args]]}, \npq{[[uN]]})$)
    \end{itemize}
\end{definition}

\begin{definition}[Metric]
    We extend the metric from \cref{def:decl-typing-metric} to the algorithmic typing
    in the following way.
    For a tree $T$ inferring an algorithmic typing judgement $J$, we define 
    $\size{T}$ as $(\size{J}, 0)$.
\end{definition}

Soundness and completeness are proved by mutual induction on
the metric of the inference tree.

\lemmaTypingSoundness*
\begin{proof}
    We prove it by induction on $\metric{T_1}$, mutually with 
    the completeness of typing (\cref{lemma:typing-soundness}).
    Let us consider the last rule used to infer the derivation.
    \begin{caseof}
        \item \ruleref{\ottdruleATVarLabel}
            We are proving that if $[[Γ; Φ ⊨ x : nf(iP)]]$
            then $[[Γ ⊢ nf(iP)]]$ and $[[Γ; Φ ⊢ x : nf(iP)]]$.

            By inversion, $[[x : iP ∊ Φ]]$.
            Since $[[Γ ⊢ Φ]]$, we have $[[Γ ⊢ iP]]$,
            and by \cref{corollary:wf-nf}, $[[Γ ⊢ nf(iP)]]$.

            By applying \ruleref{\ottdruleDTVarLabel}
            to $[[x : iP ∊ Φ]]$, we infer $[[Γ; Φ ⊢ x : iP]]$.
            Finally, by \ruleref{\ottdruleDTPEquivLabel}, 
            since $[[Γ ⊢ iP ≈ nf(iP)]]$ 
            (\cref{corollary:nf-sound-wrt-subt-equiv}),
            we have $[[Γ; Φ ⊢ x : nf(iP)]]$.

        \item \ruleref{\ottdruleATThunkLabel}
            \label{case:typing-soundness:thunk}

            We are proving that if $[[Γ; Φ ⊨ {c} : ↓iN]]$
            then $[[Γ ⊢ ↓iN]]$ and $[[Γ; Φ ⊢ {c} : ↓iN]]$.

            By inversion of $[[Γ; Φ ⊨ {c} : ↓iN]]$, 
            we have $[[Γ; Φ ⊨ c : iN]]$.
            By the induction hypothesis applied to $[[Γ; Φ ⊨ c : iN]]$,
            we have 
            \begin{enumerate}
                \item $[[Γ ⊢ iN]]$, and hence, $[[Γ ⊢ ↓iN]]$;
                \item $[[Γ; Φ ⊢ c : iN]]$, 
                    which by \ruleref{\ottdruleDTThunkLabel} implies
                    $[[Γ; Φ ⊢ {c} : ↓iN]]$.
            \end{enumerate}

        \item \ruleref{\ottdruleATReturnLabel}
            The proof is symmetric to the previous case
            (\cref{case:typing-soundness:thunk}).

        \item \ruleref{\ottdruleATPAnnotLabel}
            \label{case:typing-soundness:pos-annot}
            We are proving that if $[[Γ; Φ ⊨ (v : iQ) : nf(iQ)]]$
            then $[[Γ ⊢ nf(iQ)]]$ and $[[Γ; Φ ⊢ (v : iQ) : nf(iQ)]]$.

            By inversion of $[[Γ; Φ ⊨ (v : iQ) : nf(iQ)]]$,
            we have:
            \begin{enumerate}
                \item $[[Γ ⊢ (v : iQ)]]$, hence, $[[Γ ⊢ iQ]]$, 
                    and by \cref{corollary:wf-nf}, $[[Γ ⊢ nf(iQ)]]$;
                \item $[[Γ; Φ ⊨ v : iP]]$, 
                    which by the induction hypothesis implies
                    $[[Γ ⊢ iP]]$ and $[[Γ; Φ ⊢ v : iP]]$;
                \item $[[Γ ; · ⊨ uQ ≥ iP ⫤ ·]]$,
                    which by \cref{lemma:pos-subt-soundness} implies
                    $[[Γ ⊢ [·]uQ ≥ iP]]$, that is $[[Γ ⊢ iQ ≥ iP]]$.
            \end{enumerate}

            To infer 
            $[[Γ; Φ ⊢ (v : iQ) : iQ]]$, 
            we apply \ruleref{\ottdruleDTPAnnotLabel} 
            to $[[Γ; Φ ⊢ v : iP]]$ and $[[Γ ⊢ iQ ≥ iP]]$.
            Then by \ruleref{\ottdruleDTPEquivLabel},
            $[[Γ; Φ ⊢ (v : iQ) : nf(iQ)]]$.
  
        \item \ruleref{\ottdruleATNAnnotLabel}
            The proof is symmetric to the previous case
            (\cref{case:typing-soundness:pos-annot}).

        \item \ruleref{\ottdruleATtLamLabel}
            We are proving that if 
            $[[Γ; Φ ⊨ λx:iP . c : nf(iP → iN)]]$
            then 
            $[[Γ ⊢ nf(iP → iN)]]$ and
            $[[Γ; Φ ⊢ λx:iP . c : nf(iP → iN)]]$.


            By inversion of $[[Γ; Φ ⊨ λx:iP . c : nf(iP → iN)]]$,
            we have $[[Γ ⊢ λx:iP . c]]$, 
            which implies $[[Γ ⊢ iP]]$.

            Also by inversion of $[[Γ; Φ ⊨ λx:iP . c : nf(iP → iN)]]$,
            we have $[[Γ; Φ, x:iP ⊨ c : iN]]$, applying induction 
            hypothesis to which gives us:
            \begin{enumerate}
                \item  $[[Γ ⊢ iN]]$, thus $[[Γ ⊢ iP → iN]]$, 
                    and by \cref{corollary:wf-nf}, $[[Γ ⊢ nf(iP → iN)]]$;
                \item $[[Γ; Φ, x:iP ⊢ c : iN]]$, 
                    which by \ruleref{\ottdruleDTtLamLabel} implies
                    $[[Γ; Φ ⊢ λx:iP . c : iP → iN]]$, 
                    and by \ruleref{\ottdruleDTPEquivLabel},
                    $[[Γ; Φ ⊢ λx:iP . c : nf(iP → iN)]]$.
            \end{enumerate}

        \item \ruleref{\ottdruleATTLamLabel}
            We are proving that if
            $[[Γ; Φ ⊨ Λα⁺ . c : nf(∀α⁺.iN)]]$
            then 
            $[[Γ ; Φ ⊢ Λα⁺ . c : nf(∀α⁺.iN)]]$
            and
            $[[Γ ⊢ nf(∀α⁺.iN)]]$.

            
            By inversion of $[[Γ, α⁺ ; Φ ⊨ c : iN]]$,
            we have $[[Γ ⊢ Λα⁺ . c]]$, which implies $[[Γ, α⁺ ⊢ c]]$.

            Also by inversion of 
            $[[Γ, α⁺ ; Φ ⊨ c : iN]]$, 
            we have $[[Γ, α⁺ ; Φ ⊨ c : iN]]$. 
            Obtaining the induction hypothesis to $[[Γ, α⁺ ; Φ ⊨ c : iN]]$,
            we have:
            \begin{enumerate}
                \item $[[Γ, α⁺ ⊢ iN]]$, thus $[[Γ ⊢ ∀α⁺.iN]]$,
                    and by \cref{corollary:wf-nf}, $[[Γ ⊢ nf(∀α⁺.iN)]]$;
                \item $[[Γ, α⁺ ; Φ ⊢ c : iN]]$, 
                    which by \ruleref{\ottdruleDTTLamLabel} implies
                    $[[Γ ; Φ ⊢ Λα⁺ . c : ∀α⁺.iN]]$, 
                    and by \ruleref{\ottdruleDTPEquivLabel},
                    $[[Γ ; Φ ⊢ Λα⁺ . c : nf(∀α⁺.iN)]]$.
            \end{enumerate}
            
        \item \ruleref{\ottdruleATVarLetLabel}
            We are proving that if
            $[[Γ; Φ ⊨ let x = v ; c : iN]]$
            then
            $[[Γ ; Φ ⊢ let x = v ; c : iN]]$ and 
            $[[Γ ⊢ iN]]$.


            By inversion of 
            $[[Γ; Φ ⊨ let x = v ; c : iN]]$,
            we have:
            \begin{enumerate}
                \item $[[Γ ⊢ let x = v ; c]]$, which gives us 
            $[[Γ ⊢ v]]$ and $[[Γ ⊢ c]]$.
                \item $[[Γ; Φ ⊨ v : iP]]$, 
                    which by the induction hypothesis implies
                    $[[Γ ⊢ iP]]$ (and thus, $[[Γ ⊢ Φ, x:iP]]$) 
                    and $[[Γ; Φ ⊢ v : iP]]$; 
                \item $[[Γ; Φ, x:iP ⊨ c : iN]]$, 
                    which by the induction hypothesis implies
                    $[[Γ ⊢ iN]]$ and $[[Γ; Φ, x:iP ⊢ c : iN]]$.
            \end{enumerate}
            This way, 
            $[[Γ; Φ ⊢ let x = v ; c : iN]]$ holds by
            \ruleref{\ottdruleDTVarLetLabel}.

        \item \ruleref{\ottdruleATAppLetAnnLabel}
            We are proving that 
            if
            $[[Γ; Φ ⊨ let x:iP = v(args); c' : iN]]$
            then 
            $[[Γ ; Φ ⊢ let x:iP = v(args); c' : iN]]$ and
            $[[Γ ⊢ iN]]$.

            By inversion, we have:
            \begin{enumerate}
                \item $[[Γ ⊢ iP]]$, hence, $[[Γ ⊢ Φ, x:iP]]$
                \item $[[Γ; Φ ⊨ v : ↓iM]]$
                \item $[[Γ; Φ; · ⊨ uM ● args ⇒> ↑uQ ⫤ Θ; SC1]]$
                \item $[[Γ; Θ ⊨ ↑uQ ≤ ↑iP ⫤ SC2]]$
                \item $[[Θ ⊢ SC1 & SC2 = SC]]$
                \item $[[Γ; Φ, x:iP ⊨ c' : iN]]$
            \end{enumerate}

            By the induction hypothesis applied to $[[Γ; Φ ⊨ v : ↓iM]]$, we have
            $[[Γ; Φ ⊢ v : ↓iM]]$ and $[[Γ ⊢ ↓iM]]$ (and hence, $[[Γ ; dom(Θ) ⊢  uM]]$).

            By the induction hypothesis applied to $[[Γ; Φ, x:iP ⊨ c' : iN]]$, we have
            $[[Γ; Φ, x:iP ⊢ c' : iN]]$ and $[[Γ ⊢ iN]]$. 

            By the induction hypothesis applied to $[[Γ; Φ; · ⊨ uM ● args ⇒> ↑uQ ⫤ Θ; SC1]]$, we have:
            \begin{enumerate}
                \item \label{typing-soundness:theta-wf} $[[Γ ⊢ Θ]]$,
                \item $[[Γ; dom(Θ) ⊢ ↑uQ]]$,
                \item $[[Θ'|uv uM ∪ uv uQ ⊢ SC1]]$,
                    and thus, $[[dom(SC1) ⊆ uv uM ∪ uv uQ]]$.
                \item \label{typing-soundness:SC1-initiality} 
                    for any $[[ Θ' ⊢ uσ : SC1 ]]$, we have $[[ Γ ; Φ ⊢ [uσ]uM ● args ⇒> [uσ]↑uQ ]]$.
            \end{enumerate}

            By soundness of negative subtyping (\cref{lemma:neg-subt-soundness})
            applied to $[[Γ; Θ ⊨ ↑uQ ≤ ↑iP ⫤ SC2]]$, we have
            $[[Θ ⊢ SC2 : uv(↑uQ)]]$, and thus, $[[uv(↑uQ) = dom(SC2)]]$.

            By soundness of constraint merge (\cref{lemma:merge-soundness}),
            $[[dom(SC) = dom(SC1) ∪ dom(SC2)]] \subseteq [[uv uM ∪ uv uQ]]$
            Then by \cref{lemma:constraint-sat},
            let us take $[[uσ]]$ such that
            $[[ Θ ⊢ uσ : uv(uM) ∪ uv(uQ) ]]$ and 
            $[[ Θ ⊢ uσ : SC ]]$.
            By the soundness of constraint merge, 
            $[[ Θ ⊢ uσ : SC1 ]]$ and $[[ Θ  ⊢ uσ : SC2 ]]$,
            and by weakening, $[[ Θ' ⊢ uσ : SC1 ]]$ and $[[ Θ' ⊢ uσ : SC2 ]]$.

            Then as noted above (\ref{typing-soundness:SC1-initiality}),
            $[[ Γ ; Φ ⊢ iM ● args ⇒> [uσ]↑uQ ]]$
            And again, by soundness of negative subtyping (\cref{lemma:neg-subt-soundness})
            applied to $[[Γ; Θ ⊨ ↑uQ ≤ ↑iP ⫤ SC2]]$,
            we have
            $[[Γ ⊢ [uσ]↑uQ ≤ ↑iP]]$.

            To infer $[[Γ ; Φ ⊢ let x : iP = v(args); c' : iN ]]$,
            we apply the corresponding declarative rule \ruleref{\ottdruleDTAppLetAnnLabel}, where
            $[[iQ]]$ is $[[ [uσ]uQ  ]]$. Notice that all the premises were already shown to
            hold above:
            \begin{enumerate}
                \item $[[Γ ⊢ iP]]$ and $[[Γ; Φ ⊢ v : ↓iM]]$ from the assumption,
                \item $[[Γ; Φ ⊢ iM ● args ⇒> ↑[uσ]uQ]]$ holds since $[[ [uσ]↑uQ ]] = [[ ↑[uσ]uQ ]]$,
                \item $[[Γ ⊢ ↑[uσ]uQ ≤ ↑iP]]$ by soundness of negative subtyping,
                \item $[[Γ; Φ, x:iP ⊢ c' : iN]]$ from the the induction hypothesis.
            \end{enumerate}

        \item \ruleref{\ottdruleATAppLetLabel}
            We are proving that if
            $[[Γ; Φ ⊨ let x = v(args); c' : iN]]$
            then
            $[[Γ ; Φ ⊢ let x = v(args); c' : iN]]$ and
            $[[Γ ⊢ iN]]$.

            By the inversion, we have:
            \begin{enumerate}
                \item $[[Γ; Φ ⊨ v : ↓iM]]$ 
                \item $[[Γ ; Φ ; · ⊨ uM ● args ⇒> ↑uQ ⫤ Θ; SC]]$
                \item $[[uv uQ ⊆ dom(SC)]]$
                \item $[[SC|uv(uQ) singular with uσ3]]$
                \item $[[Γ; Φ, x:[uσ3]uQ ⊨ c' : iN]]$
            \end{enumerate}

            By the induction hypothesis applied to $[[Γ; Φ ⊨ v : ↓iM]]$, we have    
            $[[Γ; Φ ⊢ v : ↓iM]]$ and $[[Γ ⊢ ↓iM]]$ (and thus, $[[Γ ; dom(Θ) ⊢  uM]]$).
       
            By the induction hypothesis applied to $[[Γ; Φ, x:[uσ3]uQ ⊨ c' : iN]]$, we have
            $[[Γ ⊢ iN]]$ and $[[Γ; Φ, x:[uσ3]uQ ⊢ c' : iN]]$.

            By the induction hypothesis applied to 
            $[[Γ ; Φ ; · ⊨ uM ● args ⇒> ↑uQ ⫤ Θ; SC]]$, we have:
            \begin{enumerate}
                \item $[[Γ ⊢ Θ]]$
                \item $[[Γ; dom(Θ) ⊢  ↑uQ]]$
                \item $[[Θ|uv uM ∪ uv uQ ⊢ SC]]$ (and thus, $[[dom(SC) ⊆ uv uM ∪ uv uQ]]$)
                \item for any $[[ Θ ⊢ uσ : SC ]]$, we have $[[ Γ ; Φ ⊢ [uσ]uM ● args ⇒> [uσ]↑uQ ]]$, 
                    which, since  $[[iM]]$ is ground means $[[ Γ ; Φ ⊢ iM ● args ⇒> ↑[uσ]uQ]]$.
            \end{enumerate}


            To infer $[[Γ ; Φ ⊢ let x = v(args) ; c' : iN ]]$, 
            we apply the corresponding 
            declarative rule \ruleref{\ottdruleDTAppLetLabel}.
            Let us show that the premises hold:
            \begin{itemize}
                \item $[[Γ; Φ ⊢ v : ↓iM]]$ holds by the induction hypothesis;
                \item $[[Γ; Φ, x:[uσ3]uQ ⊢ c' : iN]]$ also holds by the induction hypothesis, as noted above;
                \item Let us take an arbitrary substitution $[[uσ]]$ 
                    $[[ Θ ⊢ uσ : uv uM ∪ uv uQ ]]$
                    satisfying $[[ Θ ⊢ uσ : SC ]]$
                    (it exists by \cref{lemma:constraint-sat}).
                    Then $[[Γ; Φ ⊢ iM ● args ⇒> ↑[uσ]uQ ]]$ holds, as noted above;
                \item To show the uniqueness of $[[↑[uσ]uQ]]$,
                    we assume that for some other type $[[iK]]$ 
                    holds $[[Γ; Φ ⊢ iM ● args ⇒> iK ]]$,
                    that is $[[Γ; Φ ⊢ [·]uM ● args ⇒> iK ]]$.
                    Then by the completeness of typing 
                    (\cref{lemma:typing-completeness}),
                    there exist $[[uN']]$, $[[Θ']]$, and $[[SC']]$ such that
                    \begin{enumerate}
                        \item $[[ Γ; Φ; · ⊨ uM ● args ⇒> uN' ⫤ Θ'; SC' ]]$ and
                        \item there exists a substitution $[[Θ' ⊢ uσ' : SC']]$ 
                            such that $[[Γ ⊢ [uσ']uN' ≈ iK]]$.
                    \end{enumerate}
                    By determinicity of the typing algorithm (\cref{lemma:typing-determinicity}),
                    $[[ Γ; Φ; · ⊨ uM ● args ⇒> uN' ⫤ Θ'; SC' ]]$,
                    means that $[[SC']]$ is $[[SC]]$, $[[Θ']]$ is $[[Θ]]$, and $[[uN']]$ is
                    $[[↑uQ]]$. 
                    This way, $[[Γ ⊢ [uσ']↑uQ ≈ iK]]$ for substitution 
                    $[[Θ ⊢ uσ' : SC]]$. 

                    It is left to show that $[[Γ ⊢ [uσ']↑uQ ≈ [uσ]↑uQ]]$, 
                    then by transitivity of equivalence, we will have $[[Γ ⊢ [uσ]↑uQ ≈ iK]]$.
                    Since $[[ Θ  ⊢ uσ : SC|uv(uQ) ]]$ and $[[ Θ  ⊢ uσ' : SC|uv(uQ) ]]$, 
                    and $[[SC|uv(uQ) singular with uσ3]]$, we have 
                    $[[Θ ⊢ uσ ≈ uσ3 : dom(SC) ∩ uv(uQ)]]$
                    and $[[Θ ⊢ uσ' ≈ uσ3 : dom(SC) ∩ uv(uQ)]]$.
                    Then since $[[uv(uQ) ⊆ dom(SC)]]$, we have $[[dom(SC) ∩ uv(uQ) = uv(uQ)]]$.
                    This way, by transitivity and symmetry of the equivalence, 
                    $[[Θ ⊢ uσ ≈ uσ' : uv(uQ)]]$, which implies
                    $[[Γ ⊢ [uσ']↑uQ ≈ [uσ]↑uQ]]$. 
            \end{itemize}

        \item \ruleref{\ottdruleATUnpackLabel}
            We are proving that if 
            $[[Γ; Φ ⊨ let∃ (nas, x) = v; c' : iN]]$
            then
            $[[Γ ; Φ ⊢ let∃ (nas, x) = v; c' : iN]]$ and
            $[[Γ ⊢ iN]]$.
            By the inversion, we have:
            \begin{enumerate}
                \item $[[Γ; Φ ⊨ v : ∃nas.iP]]$
                \item $[[Γ, nas ; Φ, x:iP ⊨ c' : iN]]$
                \item $[[Γ ⊢ iN]]$
            \end{enumerate}

            By the induction hypothesis applied to 
            $[[Γ; Φ ⊨ v : ∃nas.iP]]$, we have $[[Γ; Φ ⊢ v : ∃nas.iP]]$
            and $[[∃nas.iP]]$ is normalized.
            By the induction hypothesis applied to
            $[[Γ, nas ; Φ, x:iP ⊨ c' : iN]]$, we have $[[Γ, nas ; Φ, x:iP ⊢ c' : iN]]$.

            To show $[[Γ; Φ ⊢ let∃ (nas, x) = v; c' : iN]]$, we apply the corresponding
            declarative rule \ruleref{\ottdruleDTUnpackLabel}. Let us show that the premises hold:
            \begin{enumerate}
                \item $[[Γ ; Φ ⊢ v : ∃nas.iP]]$ holds by the induction hypothesis, as noted above,
                \item $[[nf(∃nas.iP) = ∃nas.iP]]$ holds since $[[∃nas.iP]]$ is normalized,
                \item $[[Γ, nas ; Φ, x:iP ⊢ c' : iN]]$ also holds by the induction hypothesis,
                \item $[[Γ ⊢ iN]]$ holds by the inversion, as noted above.
            \end{enumerate}

        \item \ruleref{\ottdruleATEmptyAppLabel}
            Then by assumption:
            \begin{itemize}
                \item $[[Γ ⊢ Θ]]$,
                \item $[[Γ; dom(Θ) ⊢  uN]]$ is free from negative algorithmic variables,
                \item $[[Γ; Φ; Θ ⊨ uN ● · ⇒> nf(uN) ⫤ Θ; ·]]$.
            \end{itemize}

            Let us show the required properties: 
            \begin{enumerate}
                \item $[[Γ ⊢ Θ]]$ holds by assumption,
                \item $[[Θ]] \subseteq [[Θ]]$ holds trivially,
                \item $[[nf(uN)]]$ is evidently normalized, 
                    $[[Γ; dom(Θ) ⊢  uN]]$ implies $[[Γ; dom(Θ) ⊢  nf(uN)]]$ by 
                    \cref{corollary:wf-nf-algo},
                    and \cref{lemma:fv-nf} means that $[[nf(uN)]]$ is 
                    inherently free from negative algorithmic variables,
                \item $[[dom(Θ) ∩ uv(nf(uN)) ⊆ uv uN]]$
                    holds since $[[uv(nf(uN)) = uv(uN)]]$,
                \item $[[Θ|uv uN ∪ uv nf(uN) ⊢ ·]]$ holds trivially,
                \item suppose that $[[ Θ ⊢ uσ : uv uN ∪ uv nf(uN) ]]$.
                    To show $[[ Γ ; Φ ⊢ [uσ]uN ● · ⇒> [uσ]nf(uN) ]]$, we apply the corresponding 
                    declarative rule \ruleref{\ottdruleDTEmptyAppLabel}.
                    To show $[[ Γ ⊢ [uσ]uN ≈ [uσ]nf(uN) ]]$,
                    we apply the following sequence:
                    $[[uN ≈ nf(uN)]]$ by 
                    \cref{lemma:normalization-soundness},
                    then $[[ [uσ]uN ≈ [uσ]nf(uN) ]]$
                    by \cref{corollary:subst-pres-decl-equiv},
                    then $[[ Γ ⊢ [uσ]uN ≈ [uσ]nf(uN) ]]$
                    by \cref{lemma:equiv-soundness}. 
            \end{enumerate}

        \item \ruleref{\ottdruleATArrowAppLabel} 
            By assumption:
            \begin{enumerate}
                \item $[[Γ ⊢ Θ]]$,
                \item $[[Γ; dom(Θ) ⊢ uQ → uN]]$ is free from negative algorithmic variables,
                    and hence, so are $[[uQ]]$ and $[[uN]]$,
                \item $[[Γ; Φ; Θ ⊨ uQ → uN ● v , args ⇒> uM ⫤ Θ'; SC]]$, 
                    and by inversion: 
                    \begin{enumerate}
                        \item $[[Γ; Φ ⊨ v : iP]]$,
                            and by the induction hypothesis applied to this judgment,
                            we have $[[Γ; Φ ⊢ v : iP]]$, and $[[Γ ⊢ iP]]$;
                        \item $[[Γ; Θ ⊨ uQ ≥ iP ⫤ SC1]]$,
                            and by the soundness of subtyping:
                            $[[Θ ⊢ SC1 : uv uQ]]$ 
                            (and thus, $[[dom(SC1) = uv uQ]]$),
                            and
                            for any
                            
                            $[[ Θ ⊢ uσ : SC1 ]]$, we have $[[Γ ⊢ [uσ]uQ ≥ iP]]$;
                        \item $[[Γ; Φ; Θ ⊨ uN ● args ⇒> uM ⫤ Θ'; SC2]]$,
                            and by the induction hypothesis applied to this judgment,
                            \begin{enumerate}
                                \item $[[Γ ⊢ Θ']]$,
                                \item $[[Θ ⊆ Θ']]$,
                                \item $[[Γ; dom(Θ') ⊢  uM]]$ is normalized and free from negative algorithmic variables,
                                \item $[[dom(Θ) ∩ uv(uM) ⊆ uv uN]]$,
                                \item $[[Θ'|uv(uM) ∪ uv(uN) ⊢ SC2]]$, and thus, $[[dom(SC2) ⊆ uv(uM) ∪ uv(uN)]]$,
                                \item for any $[[uσ]]$ such that
                                    $[[Θ ⊢ uσ : uv(uM) ∪ uv(uN) ]]$
                                    and 
                                    $[[ Θ' ⊢ uσ : SC2 ]]$, we have 
                                    $[[ Γ ; Φ ⊢ [uσ]uN ● args ⇒> [uσ]uM ]]$;
                            \end{enumerate}
                        \item $[[Θ ⊢ SC1 & SC2 = SC]]$,
                            which by \cref{lemma:merge-soundness} implies
                            $[[dom(SC) = dom(SC1) ∪ dom(SC2)]] \subseteq [[uv uQ ∪ uv uM ∪ uv uN]]$.
                    \end{enumerate}
            \end{enumerate}

            Let us show the required properties:
            \begin{enumerate}
                \item $[[Γ ⊢ Θ']]$ is shown above,
                \item $[[Θ ⊆ Θ']]$ is shown above,
                \item $[[Γ; dom(Θ') ⊢  uM]]$ is normalized and free from negative algorithmic variables, as shown above,
                \item $[[dom(Θ) ∩ uv(uM) ⊆ uv uN ⊆ uv(uQ → uN)]]$
                    (the first inclusion is shown above, the second one is by definition),
                \item To show $[[Θ'|uv(uQ) ∪ uv(uN) ∪ uv(uM) ⊢ SC]]$,
                    first let us notice that $[[uv(uQ) ∪ uv(uN) ∪ uv(uM) ⊆ dom(SC)]]$,
                    as mentioned above.
                    Then we demonstrate 
                    $[[Θ' ⊢ SC]]$:
                    $[[Θ ⊢ SC1]]$ and $[[Θ ⊆ Θ']]$ imply $[[Θ' ⊢ SC1]]$,
                    by the soundness of constraint merge (\cref{lemma:merge-soundness})
                    applied to $[[Θ' ⊢ SC1 & SC2 = SC]]$:
                    \begin{enumerate}
                        \item $[[Θ' ⊢ SC]]$,
                        \item for any $[[ Θ' ⊢ uσ : SC ]]$, $[[ Θ' ⊢ uσ : SCi ]]$ holds;
                    \end{enumerate}
                \item Suppose that 
                    $[[ Θ' ⊢ uσ : uv(uQ) ∪ uv(uN) ∪ uv(uM) ]]$
                    and $[[ Θ' ⊢ uσ : SC ]]$.
                    To show $[[ Γ ; Φ ⊢ [uσ](uQ → uN) ● v , args ⇒> [uσ]uM ]]$, 
                    that is $[[ Γ ; Φ ⊢ [uσ]uQ → [uσ]uN ● v , args ⇒> [uσ]uM ]]$,
                    we apply the corresponding declarative rule \ruleref{\ottdruleDTArrowAppLabel}.
                    Let us show the required premises:
                    \begin{enumerate}
                        \item $[[Γ; Φ ⊢ v : iP]]$ holds as shown above,
                        \item $[[Γ ⊢ [uσ]uQ ≥ iP]]$ holds 
                            since $[[Γ ⊢ [uσ|uv(uQ)]uQ ≥ iP]]$ by the soundness of subtyping 
                            as noted above:
                            since $[[ Θ' ⊢ uσ : SC ]]$ implies $[[ Θ' ⊢ uσ|uv(uQ) : SC1 ]]$,
                            which we strengthen to $[[ Θ ⊢ uσ|uv(uQ) : SC1 ]]$,
                        \item $[[Γ; Φ ⊢ [uσ]uN ● args ⇒> [uσ]uM]]$ holds by the induction hypothesis
                            as shown above,
                            since $[[ Θ' ⊢ uσ : SC ]]$ implies $[[ Θ' ⊢ uσ : SC2 ]]$,
                            and then $[[ Θ' ⊢ uσ | uv(uN) ∪ uv(uM) : SC2 ]]$
                            and $[[ Θ ⊢ uσ | uv(uN) ∪ uv(uM) : uv(uN) ∪ uv(uM) ]]$.
                    \end{enumerate}
            \end{enumerate}

        \item \ruleref{\ottdruleATForallAppLabel}\\
            By assumption:
            \begin{enumerate}
                \item $[[Γ ⊢ Θ]]$,
                \item $[[Γ; dom(Θ) ⊢  ∀pas.uN]]$ is free from negative algorithmic variables,
                \item $[[Γ; Φ; Θ ⊨ ∀pas.uN ● args ⇒> uM ⫤ Θ'; SC]]$, which by inversion means
                    $[[args ≠ ·]]$, $[[pas ≠ ·]]$, and $[[Γ; Φ; Θ, â⁺*[Γ] ⊨ [â⁺*/pas]uN ● args ⇒> uM ⫤ Θ'; SC]]$.
                    It is easy to see that the induction hypothesis is applicable to the latter judgment:
                    \begin{itemize}
                        \item $[[Γ ⊢ Θ, â⁺*[Γ] ]]$ holds by $[[Γ ⊢ Θ]]$, 
                        \item $[[Γ; dom(Θ), â⁺* ⊢ [â⁺*/pas]uN]]$
                            holds since $[[Γ; dom(Θ) ⊢ ∀pas.uN]]$
                            $[[ [â⁺*/pas]uN ]]$ is normalized and free from negative algorithmic variables
                            since so is $[[uN]]$;
                    \end{itemize}
                    This way, by the inductive hypothesis applied to 
                    $[[Γ; Φ; Θ, â⁺*[Γ] ⊨ [â⁺*/pas]uN ● args ⇒> uM ⫤ Θ'; SC]]$,
                    we have:
                    \begin{enumerate}
                        \item $[[Γ ⊢ Θ']]$,
                        \item $[[Θ, â⁺*[Γ] ⊆ Θ']]$,
                        \item $[[Γ; dom(Θ') ⊢ uM]]$ is normalized and free from negative algorithmic variables,
                        \item $[[dom(Θ, â⁺*[Γ]) ∩ uv(uM) ⊆ uv([â⁺*/pas]uN)]]$,
                        \item $[[Θ'|Ξ ∪ uv(uN) ∪ uv(uM) ⊢ SC]]$, where $[[Ξ]]$ denotes $[[uv([â⁺*/pas]uN) ∩ {â⁺*}]]$,
                            that is the algorithmization of the $[[∀]]$-variables that are actually used in $[[uN]]$.
                        \item  \label{typing-soundness:forall-ih}
                            for any 
                            $[[uσ]]$ such that 
                            $[[Θ' ⊢ uσ : Ξ ∪ uv(uN) ∪ uv(uM)]]$ and $[[ Θ' ⊢ uσ : SC ]]$, 
                            we have $[[ Γ ; Φ ⊢ [uσ][â⁺*/pas]uN ● args ⇒> [uσ]uM ]]$.
                    \end{enumerate}
            \end{enumerate}

            Let us show the required properties:
            \begin{enumerate}
                \item $[[Γ ⊢ Θ']]$ is shown above;
                \item $[[Θ ⊆ Θ']]$ since $[[Θ, â⁺*[Γ] ⊆ Θ']]$;
                \item $[[Γ; dom(Θ') ⊢  uM]]$ is normalized and free from negative algorithmic variables, as shown above;
                \item $[[dom(Θ) ∩ uv(uM) ⊆ uv(uN)]]$
                    since 
                    $[[dom(Θ, â⁺*[Γ]) ∩ uv(uM) ⊆ uv([â⁺*/pas]uN)]]$
                    implies 
                    $[[(dom(Θ) ∪ {â⁺*}) ∩ uv(uM) ⊆ uv(uN) ∪ {â⁺*}]]$,
                    thus,
                    $[[dom(Θ) ∩ uv(uM) ⊆ uv(uN) ∪ {â⁺*}]]$,
                    and since $[[dom(Θ)]]$ is disjoint with $[[{â⁺*}]]$,
                    $[[dom(Θ) ∩ uv(uM) ⊆ uv(uN)]]$;

                \item $[[Θ'|uv(uN) ∪ uv(uM) ⊢ SC | uv(uN) ∪ uv(uM)]]$ follows from 
                    $[[Θ'|Ξ ∪ uv(uN) ∪ uv(uM) ⊢ SC]]$ if we restrict both sides to 
                    $[[uv(uN) ∪ uv(uM)]]$.
                \item Let us assume $[[ Θ' ⊢ uσ : uv(uN) ∪ uv(uM) ]]$ and
                    $[[ Θ' ⊢ uσ : SC | uv(uN) ∪ uv(uM) ]]$.
                    Then to show $[[ Γ ; Φ ⊢ [uσ]∀pas.uN ● args ⇒> [uσ]uM ]]$,
                    that is $[[ Γ ; Φ ⊢ ∀pas.[uσ]uN ● args ⇒> [uσ]uM ]]$,
                    we apply the corresponding declarative rule \ruleref{\ottdruleDTForallAppLabel}.
                    To do so, we need to provide a substitution for $[[pas]]$, 
                    i.e. $[[Γ ⊢ σ0 : {pas}]]$ such that
                    $[[Γ; Φ ⊢ [σ0][uσ]uN ● args ⇒> [uσ]uM ]]$.

                    By \cref{lemma:constraint-sat},
                    we construct $[[uσ0]]$ such that
                    $[[Θ' ⊢ uσ0 : {â⁺*}]]$
                    and 
                    $[[Θ' ⊢ uσ0 : SC|{â⁺*}]]$.

                    Then $[[σ0]]$ is defined as 
                    $[[uσ0]] \circ [[uσ|{â⁺*}]] \circ [[â⁺*/pas]]$. 

                    Let us show that the premises of 
                    \ruleref{\ottdruleDTForallAppLabel} hold:
                    \begin{itemize}
                        \item To show $[[Γ ⊢ σ0 :{pas}]]$,
                            let us take $[[αi⁺ ∊ {pas}]]$.
                            If $[[ αî⁺ ∊ uv(uM)]]$
                            then $[[ [σ0]αi⁺ = [uσ]αî⁺ ]]$,
                            and $[[ Θ' ⊢ uσ : uv(uN) ∪ uv(uM) ]]$
                            implies $[[Θ'(α̂⁺) ⊢ [uσ]α̂⁺]]$.
                            Analogously, if $[[ αî⁺ ∊ {â⁺*} \ uv(uM)]]$
                            then $[[ [σ0]αi⁺ = [uσ0]αî⁺ ]]$,
                            and $[[Θ' ⊢ uσ0 : {â⁺*}]]$ implies $[[Θ'(αî⁺) ⊢ [uσ0]αî⁺]]$.
                            In any case, $[[Θ'(αî⁺) ⊢ [σ]αi⁺ ]]$
                            can be weakened to $[[Γ ⊢ [σ0]αi⁺]]$, 
                            since $[[Γ ⊢ Θ']]$.

                        \item Let us show $[[Γ; Φ ⊢ [σ0][uσ]uN ● args ⇒> [uσ]uM ]]$.
                            It suffices to construct $[[uσ1]]$ such that
                            \begin{enumerate}
                                \item $[[Θ' ⊢ uσ1 : Ξ ∪ uv(uN) ∪ uv(uM)]]$,
                                \item $[[Θ' ⊢ uσ1 : SC]]$,
                                \item $[[ [σ0][uσ]uN = [uσ1][â⁺*/pas]uN]]$, and
                                \item $[[ [uσ]uM = [uσ1]uM]]$,
                            \end{enumerate}
                            because then we can apply the induction hypothesis
                            (\ref{typing-soundness:forall-ih}) to
                            $[[uσ1]]$, rewrite the conclusion by 
                            $[[ [uσ1][â⁺*/pas]uN = [σ0][uσ]uN]]$ and
                            $[[ [uσ1]uM = [uσ]uM]]$, and infer the required judgement.

                            Let us take $[[uσ1]] = [[(uσ0 ○ uσ)|Ξ ∪ uv(uN) ∪ uv(uM)]]$, then 
                            \begin{enumerate}
                                \item $[[Θ' ⊢ uσ1 : Ξ ∪ uv(uN) ∪ uv(uM)]]$,
                                    since $[[Θ' ⊢ uσ0 : {â⁺*}]]$ and 
                                    $[[ Θ' ⊢ uσ : uv(uN) ∪ uv(uM) ]]$, 
                                    we have 
                                    $[[Θ' ⊢ uσ0 ○ uσ : {â⁺*} ∪ uv(uN) ∪ uv(uM)]]$, 
                                    which we restrict to $[[Ξ ∪ uv(uN) ∪ uv(uM)]]$.
                                \item $[[Θ' ⊢ uσ1 : SC]]$,
                                    Let us take any constraint $[[scE ∊ SC]]$ restricting 
                                    variable $[[β̂±]]$.
                                    $[[Θ'|Ξ ∪ uv(uN) ∪ uv(uM) ⊢ SC]]$
                                    implies that $[[β̂± ∊ Ξ ∪ uv(uN) ∪ uv(uM)]]$.

                                    If $[[β̂± ∊ uv(uN) ∪ uv(uM)]]$ then 
                                    $[[ [uσ1]β̂± = [uσ]β̂± ]]$.
                                    Additionally,\\ $[[scE ∊ SC | uv(uN) ∪ uv(uM)]]$,
                                    which, since $[[ Θ' ⊢ uσ : SC | uv(uN) ∪ uv(uM) ]]$,
                                    means $[[Θ'(β̂±) ⊢ [uσ]β̂± : scE]]$.

                                    If $[[β̂± ∊ Ξ \ (uv(uN) ∪ uv(uM))]]$ then
                                    $[[ [uσ1]β̂± = [uσ0]β̂± ]]$.
                                    Additionally, $[[scE ∊ SC | {â⁺*}]]$,
                                    which, since $[[ Θ' ⊢ uσ0 : SC | {â⁺*} ]]$,
                                    means $[[Θ'(β̂±) ⊢ [uσ0]β̂± : scE]]$.

                                \item Let us prove $[[ [σ0][uσ]uN = [uσ1][â⁺*/pas]uN]]$
                                    by the following reasoning
                                    $$ 
                                    \begin{aligned}[t] 
                                        [[ [σ0][uσ]uN  ]] 
                                            &= [[ [uσ0][uσ|{â⁺*}][â⁺*/pas][uσ]uN ]] 
                                                && \text{by definition of $[[σ0]]$}\\
                                            &= [[ [uσ0][uσ|{â⁺*}][â⁺*/pas][uσ|uv(uN)]uN ]]
                                                && \text{by \cref{lemma:subst-restr-uv}}\\
                                            &= [[ [uσ0][uσ|{â⁺*}][uσ|uv(uN)][â⁺*/pas]uN ]]
                                                && \text{$[[uv(uN) ∩ {â⁺*} = ∅]]$ and 
                                                    $[[{pas} ∩ Γ = ∅]]$}\\
                                            &= [[ [uσ|{â⁺*}][uσ|uv(uN)][â⁺*/pas]uN ]] 
                                                && \text{$[[ [uσ|{â⁺*}][uσ|uv(uN)][â⁺*/pas]uN ]]$
                                                is ground}\\
                                            &= [[ [uσ|{â⁺*} ∪ uv(uN)][â⁺*/pas]uN ]] \\
                                            &= [[ [uσ|Ξ ∪ uv(uN)][â⁺*/pas]uN ]] 
                                                && \text{by \cref{lemma:subst-restr-uv}:
                                                    $[[ uv([â⁺*/pas]uN) = Ξ ∪ uv(uN)]]$}\\
                                            &= [[ [uσ|Ξ ∪ uv(uN) ∪ uv(uM)][â⁺*/pas]uN ]] 
                                                && \text{also by \cref{lemma:subst-restr-uv}}\\
                                            &= [[ [(uσ0 ○ uσ)|Ξ ∪ uv(uN) ∪ uv(uM)][â⁺*/pas]uN ]]
                                                && \text{$[[ [uσ|Ξ ∪ uv(uN) ∪ uv(uM)][â⁺*/pas]uN]]$ is ground}\\
                                            &= [[ [uσ1][â⁺*/pas]uN ]]
                                                && \text{by definition of $[[uσ1]]$}\\
                                        \end{aligned} 
                                    $$
                                \item $[[ [uσ]uM = [uσ1]uM]]$
                                        By definition of $[[uσ1]]$,
                                        $[[ [uσ1]uM ]]$ is equal to\\
                                        $[[ [(uσ0 ○ uσ)|Ξ ∪ uv(uN) ∪ uv(uM)]uM ]]$,
                                        which by \cref{lemma:subst-restr-uv} is equal to
                                        $[[ [uσ0 ○ uσ]uM ]]$,
                                        that is $[[ [uσ0][uσ]uM ]]$,
                                        and since $[[ [uσ]uM ]]$ is ground, 
                                        $[[ [uσ0][uσ]uM = [uσ]uM ]]$.
                            \end{enumerate}
                        \item $[[pas ≠ ·]]$ and $[[args ≠ ·]]$ hold by assumption.
                    \end{itemize}
            \end{enumerate}
    \end{caseof}
\end{proof}

\lemmaTypingCompleteness*
\begin{proof}
    We prove it by induction on $\metric{T_1}$, mutually with 
    the soundness of typing (\cref{lemma:typing-soundness}).
    Let us consider the last rule applied to infer the derivation.
    \begin{caseof}

        \item \ruleref{\ottdruleDTThunkLabel}\\
            \label{case:typing-completeness-thunk}
            Then we are proving that if 
            $[[Γ; Φ ⊢ {c} : ↓iN]]$ (inferred by \ruleref{\ottdruleDTThunkLabel})
            then $[[Γ; Φ ⊨ {c} : nf(↓iN)]]$.
            By inversion of $[[Γ; Φ ⊢ {c} : ↓iN]]$, we have
            $[[Γ; Φ ⊢ c : iN]]$, which we apply the induction hypothesis to
            to obtain $[[Γ; Φ ⊨ c : nf(iN)]]$.
            Then by \ruleref{\ottdruleATThunkLabel}, we have $[[Γ; Φ ⊨ {c} : ↓nf(iN)]]$.
            It is left to notice that $[[↓nf(iN) = nf(↓iN)]]$.

        \item \ruleref{\ottdruleDTReturnLabel}\\
            The proof is symmetric to the previous case 
            (\cref{case:typing-completeness-thunk}).

        \item \ruleref{\ottdruleDTPAnnotLabel}\\
            \label{case:typing-completeness-pannot}
            Then we are proving that if
            $[[Γ; Φ ⊢ (v : iQ) : iQ]]$ is inferred by \ruleref{\ottdruleDTPAnnotLabel}
            then $[[Γ; Φ ⊨ (v : iQ) : nf(iQ)]]$.
            By inversion, we have:
            \begin{enumerate}
                \item $[[Γ ⊢ iQ]]$;
                \item $[[Γ; Φ ⊢ v : iP]]$, which
                    by the induction hypothesis implies $[[Γ; Φ ⊨ v : nf(iP)]]$;
                \item $[[Γ ⊢ iQ ≥ iP]]$, and by transitivity, $[[Γ ⊢ iQ ≥ nf(iP)]]$;
                    Since $[[iQ]]$ is ground, 
                    we have $[[Γ ; · ⊢ uQ]]$ and $[[Γ ⊢ [·]uQ ≥ nf(iP)]]$.
                    Then by the completeness of subtyping
                    (\cref{lemma:pos-subt-completeness}), we have 
                    $[[Γ ; · ⊨ uQ ≥ nf(iP) ⫤ SC]]$, where $[[· ⊢ SC]]$ 
                    (implying $[[SC]] = [[·]]$).
                    This way, $[[Γ ; · ⊨ uQ ≥ nf(iP) ⫤ ·]]$.
            \end{enumerate}
            Then we can apply \ruleref{\ottdruleATPAnnotLabel} to
            $[[Γ ⊢ iQ]]$, $[[Γ; Φ ⊨ v : nf(iP)]]$ and $[[Γ ; · ⊨ uQ ≥ nf(iP) ⫤ ·]]$
            to infer $[[Γ; Φ ⊨ (v : iQ) : nf(iQ)]]$.

        \item \ruleref{\ottdruleDTNAnnotLabel}\\
            The proof is symmetric to the previous case 
            (\cref{case:typing-completeness-pannot}).

        \item \ruleref{\ottdruleDTtLamLabel}\\
            Then we are proving that if
            $[[Γ ; Φ ⊢ λx:iP . c : iP → iN]]$ is inferred by \ruleref{\ottdruleDTtLamLabel},
            then $[[Γ ; Φ ⊨ λx:iP . c : nf(iP → iN)]]$.

            By inversion of $[[Γ ; Φ ⊢ λx:iP . c : iP → iN]]$, we have
            $[[Γ ⊢ iP]]$ and $[[Γ; Φ, x:iP ⊢ c : iN]]$.
            Then by the induction hypothesis, $[[Γ; Φ, x:iP ⊨ c : nf(iN)]]$.
            By \ruleref{\ottdruleATtLamLabel}, we infer
            $[[Γ; Φ ⊨ λx:iP . c : nf(iP → nf(iN))]]$. 
            By idempotence of normalization (\cref{lemma:norm-idemp}), 
            $[[nf(iP → nf(iN)) = nf(iP → iN)]]$, 
            which concludes the proof for this case.

        \item \ruleref{\ottdruleDTTLamLabel}\\
            Then we are proving that if
            $[[Γ ; Φ ⊢ Λα⁺ . c : ∀α⁺.iN]]$ is inferred by \ruleref{\ottdruleDTTLamLabel},
            then $[[Γ ; Φ ⊨ Λα⁺ . c : nf(∀α⁺.iN)]]$.
            Similar to the previous case, 
            by inversion of $[[Γ ; Φ ⊢ Λα⁺ . c : ∀α⁺.iN]]$, we have
            $[[Γ, α⁺ ; Φ ⊢ c : iN]]$, and then by the induction hypothesis,
            $[[Γ, α⁺ ; Φ ⊨ c : nf(iN)]]$.
            After that, application of \ruleref{\ottdruleATTLamLabel}, 
            gives as $[[Γ ; Φ ⊨ Λα⁺ . c : nf(∀α⁺.nf(iN))]]$.

            It is left to show that $[[nf(∀α⁺.nf(iN)) = nf(∀α⁺.iN)]]$.
            Assume $[[iN = ∀pbs.iM]]$ (where $[[iM]]$ does not start with $\forall$).
            \begin{itemize}
                \item Then by definition, $[[nf(∀α⁺.iN) = nf(∀α⁺,pbs.iM)]] = 
                    [[∀pcs.nf(iM)]]$, where\\ $[[ord {α⁺,pbs} in nf(iM) = pcs]]$.
                \item On the other hand, $[[nf(iN) = ∀pcs'.nf(iM)]]$, 
                    where $[[ord {pbs} in nf(iM) = pcs']]$, and thus, 
                    $[[nf(∀α⁺.nf(iN)) = nf(∀α⁺,pcs'.nf(iM))]] = [[∀pcs''.nf(nf(iM))]]
                    = [[∀pcs''.nf(iM)]]$,
                    where $[[ord {α⁺,pcs'} in nf(nf(iM)) = pcs'']]$. 
            \end{itemize}
            It is left to show that $[[pcs'' = pcs]]$.
            $$ 
            \begin{aligned}[t] 
                [[ pcs'' ]] &= [[ord {α⁺,pcs'} in nf(nf(iM))]] \\
                            &= [[ord {α⁺,pcs'} in nf(iM)]]
                            && \text{by idempotence (\cref{lemma:norm-idemp})}\\
                            &= [[ord {α⁺} ∪ {pbs} ∩ fv nf(iM) in nf(iM)]]
                            && \text{by definition of $[[pcs']]$ and \cref{lemma:ord-soundness}}\\
                            &= [[ord ({α⁺} ∪ {pbs} ∩ fv nf(iM)) ∩ fv nf(iM) in nf(iM)]]
                            && \text{by \cref{corollary:ord-weakening}}\\
                            &= [[ord ({α⁺} ∪ {pbs}) ∩ fv nf(iM) in nf(iM)]]
                            && \text{by set properties}\\
                            &= [[ord {α⁺,pbs} in nf(iM)]]\\
                            &= [[pcs]]
                \end{aligned} 
            $$

        \item \ruleref{\ottdruleDTUnpackLabel}\\
            Then we are proving that if
            $[[Γ ; Φ ⊢ let∃ (nas, x) = v; c : iN]]$ is 
            inferred by \ruleref{\ottdruleDTUnpackLabel},
            then $[[Γ ; Φ ⊨ let∃ (nas, x) = v; c : nf(iN)]]$.

            By inversion of $[[Γ ; Φ ⊢ let∃ (nas, x) = v; c : iN]]$, we have
            \begin{enumerate}
                \item $[[nf(∃nas.iP) = ∃nas.iP]]$,
                \item $[[Γ ; Φ ⊢ v : ∃nas.iP]]$, 
                    which by the induction hypothesis implies 
                    $[[Γ ; Φ ⊨ v : nf(∃nas.iP)]]$, 
                    and hence, $[[Γ ; Φ ⊨ v : ∃nas.iP]]$.
                \item $[[Γ, nas ; Φ, x:iP ⊢ c : iN]]$,
                    and by the induction hypothesis, 
                    $[[Γ, nas ; Φ, x:iP ⊨ c : nf(iN)]]$.
                \item $[[Γ ⊢ iN]]$.
            \end{enumerate}
            
            This way, we can apply \ruleref{\ottdruleATUnpackLabel} to
            infer $[[Γ ; Φ ⊨ let∃ (nas, x) = v; c : nf(iN)]]$.

        \item \ruleref{\ottdruleDTPEquivLabel}\\
            Then we are proving that
            if $[[Γ; Φ ⊢ v : iP']]$ is inferred by \ruleref{\ottdruleDTPEquivLabel},
            then $[[Γ; Φ ⊨ v : nf(iP')]]$.
            By inversion, $[[Γ ; Φ ⊢ v : iP]]$ and $[[Γ ⊢ iP ≈ iP']]$,
            and the metric of the tree inferring $[[Γ ; Φ ⊢ v : iP]]$ is less than the one 
            inferring $[[Γ; Φ ⊢ v : iP']]$.
            Then by the induction hypothesis, $[[Γ; Φ ⊨ v : nf(iP)]]$.

            By \cref{lemma:subt-equiv-algorithmization}
            $[[Γ ⊢ iP ≈ iP']]$ implies $[[nf(iP) = nf(iP')]]$, and thus, 
            $[[Γ; Φ ⊨ v : nf(iP)]]$ can be rewritten to $[[Γ; Φ ⊨ v : nf(iP')]]$.

        \item \ruleref{\ottdruleDTVarLabel}\\
            Then we are proving that
            $[[Γ; Φ ⊢ x : iP]]$
            implies
            $[[Γ; Φ ⊨ x : nf(iP)]]$.
            By inversion of $[[Γ; Φ ⊢ x : iP]]$,
            we have $[[x : iP ∊ Φ ]]$.
            Then \ruleref{\ottdruleATVarLabel} applies to infer
            $[[Γ; Φ ⊨ x : nf(iP)]]$.

        \item \ruleref{\ottdruleDTVarLetLabel}\\
            Then we are proving that
            $[[Γ; Φ ⊢ let x = v(args); c : iN]]$
            implies
            $[[Γ; Φ ⊨ let x = v(args); c : nf(iN)]]$.

            By inversion of
            $[[Γ; Φ ⊢ let x = v(args); c : iN]]$,
            we have
            \begin{enumerate}
                \item $[[Γ; Φ ⊢ v : iP]]$, 
                    and by the induction hypothesis,
                    $[[Γ; Φ ⊨ v : nf(iP)]]$.
                \item $[[Γ; Φ, x:iP ⊢ c : iN]]$,
                    and by \cref{lemma:decl-typing-context-equiv},
                    since $[[Γ ⊢ iP ≈ nf(iP)]]$, we have
                    $[[Γ; Φ, x:nf(iP) ⊢ c : iN]]$.
                    Then by the induction hypothesis, 
                    $[[Γ; Φ, x:nf(iP) ⊨ c : nf(iN)]]$.
            \end{enumerate}

            Together, $[[Γ; Φ ⊨ v : nf(iP)]]$ and $[[Γ; Φ, x:nf(iP) ⊨ c : nf(iN)]]$
            imply $[[Γ; Φ ⊨ let x = v(args); c : nf(iN)]]$ by \ruleref{\ottdruleATVarLetLabel}.

        \item \ruleref{\ottdruleDTAppLetAnnLabel}\\
            Then we are proving that 
            $[[Γ ; Φ ⊢ let x:iP = v(args); c : iN]]$
            implies 
            $[[Γ; Φ ⊨ let x:iP = v(args); c : nf(iN)]]$.

            By inversion of 
            $[[Γ ; Φ ⊢ let x:iP = v(args); c : iN]]$,
            we have
            \begin{enumerate}
                \item $[[Γ ⊢ iP]]$
                \item $[[Γ ; Φ ⊢ v : ↓iM]]$ for some ground $[[iM]]$,
                    which by the induction hypothesis means
                    $[[Γ ; Φ ⊨ v : ↓nf(iM)]]$
                \item $[[Γ ; Φ ⊢ iM ● args ⇒> ↑iQ]]$. 
                    By \cref{lemma:app-inf-equ-stable}, since
                    $[[Γ ⊢ iM ≈ nf(iM)]]$, we have
                    $[[Γ ; Φ ⊢ [·]nf(uM) ● args ⇒> ↑iQ]]$, 
                    which by the induction hypothesis means 
                    that there exist normalized 
                    $[[uM']]$, $[[Θ]]$, and $[[SC1]]$ such that
                    (noting that $[[iM]]$ is ground):
                    \begin{enumerate}
                        \item $[[ Γ; Φ; · ⊨ nf(uM) ● args ⇒> uM' ⫤ Θ; SC1 ]]$,
                            where by the soundness, $[[Γ; dom(Θ) ⊢ uM']]$ and $[[Θ ⊢ SC1]]$.
                        \item for any $[[Γ ⊢ iM'']]$ 
                            such that $[[Γ; Φ ⊢ nf(iM) ● args ⇒> iM'']]$
                            there exists $[[uσ]]$ such that 
                            \begin{enumerate}
                                \item $[[ Θ ⊢ uσ : uv uM' ]]$, $[[ Θ ⊢ uσ : SC1 ]]$, and 
                                \item $[[Γ ⊢ [uσ]uM' ≈ iM'']]$,
                            \end{enumerate}
                            In particular, there exists
                            $[[uσ0]]$
                            such that 
                            $[[ Θ ⊢ uσ0 : uv uM']]$,
                            $[[ Θ ⊢ uσ0 : SC1]]$,
                            $[[Γ ⊢ [uσ0]uM' ≈ ↑iQ]]$.
                            Since $[[uM']]$ is normalized and free of negative algorithmic variables,
                            the latter equivalence means
                            $[[uM' = ↑uQ0]]$ for some $[[uQ0]]$, and $[[Γ ⊢ [uσ0]uQ0 ≈ iQ]]$.
                    \end{enumerate}
                \item $[[Γ ⊢ ↑iQ ≤ ↑iP]]$,
                    and by transitivity, since $[[Γ ⊢ [uσ0]↑uQ0 ≈ ↑iQ]]$,
                    we have $[[Γ ⊢ [uσ0]↑uQ0 ≤ ↑iP]]$.
                    
                    Let us apply \cref{lemma:neg-subt-completeness} to 
                    $[[Γ ⊢ [uσ0]↑uQ0  ≤ ↑iP]]$ and obtain
                    $[[Θ ⊢ SC2]]$ such that 
                    \begin{enumerate}
                        \item $[[Γ ; Θ ⊨ ↑uQ0 ≤ ↑iP ⫤ SC2]]$ and
                        \item $[[ Θ   ⊢ uσ0 : SC2 ]]$.
                    \end{enumerate}
                \item $[[Γ; Φ, x:iP ⊢ c : iN]]$, 
                    and by the induction hypothesis,
                    $[[Γ; Φ, x:iP ⊨ c : nf(iN)]]$.
            \end{enumerate}

            To infer $[[Γ; Φ ⊨ let x:iP = v(args); c : nf(iN)]]$,
            we apply the corresponding algorithmic rule 
            \ruleref{\ottdruleATAppLetAnnLabel}.
            Let us show that the premises hold:
            \begin{enumerate}
                \item $[[Γ ⊢ iP]]$,
                \item $[[Γ; Φ ⊨ v : ↓nf(iM)]]$,
                \item $[[Γ; Φ; · ⊨ nf(uM) ● args ⇒> ↑uQ0 ⫤ Θ; SC1]]$, 
                \item $[[Γ; Θ ⊨ ↑uQ0 ≤ ↑iP ⫤ SC2]]$, and
                \item $[[Γ; Φ, x:iP ⊨ c : nf(iN)]]$ hold as noted above;
                \item $[[Θ ⊢ SC1 & SC2 = SC]]$ 
                    is defined by \cref{lemma:merge-completeness},
                    since $[[ Θ   ⊢ uσ0 : SC1 ]]$ and $[[ Θ   ⊢ uσ0 : SC2 ]]$.
            \end{enumerate}

        \item \ruleref{\ottdruleDTAppLetLabel}\\
            By assumption, $[[c]]$ is $[[let x = v(args); c']]$. 
            Then by inversion of
            $[[Γ ; Φ ⊢ let x = v(args); c' : iN]]$: 
            \begin{itemize}
                \item $[[Γ ; Φ ⊢ v : ↓iM]]$, 
                    which by the induction hypothesis means 
                    $[[Γ; Φ ⊨ v : ↓nf(iM)]]$;
                \item $[[Γ ; Φ ⊢ iM ● args ⇒> ↑iQ uniq]]$. 
                    Then by \cref{lemma:app-inf-equ-stable}, since 
                    $[[Γ ⊢ iM ≈ nf(iM)]]$, we have
                    $[[Γ ; Φ ⊢ nf(iM) ● args ⇒> ↑iQ]]$
                    and moreover, $[[Γ ; Φ ⊢ nf(iM) ● args ⇒> ↑iQ uniq]]$
                    (since symmetrically, $[[nf(iM)]]$ can be replaced back by $[[iM]]$).
                    Then the induction hypothesis applied to 
                    $[[Γ ; Φ ⊢ [·]nf(uM) ● args ⇒> ↑iQ]]$
                    implies that there exist $[[uM']]$, $[[Θ]]$, and $[[SC]]$ such that
                    (considering $[[iM]]$ is ground):
                    \begin{enumerate}
                        \item $[[ Γ; Φ; · ⊨ nf(uM) ● args ⇒> uM' ⫤ Θ; SC ]]$, 
                            which, by the soundness, implies, in particular
                            that 
                            \begin{enumerate}
                                \item $[[Γ; dom(Θ) ⊢  uM']]$ is normalized and 
                                    free of negative algorithmic variables, 
                                \item $[[Θ|uv(uM') ⊢ SC]]$, which means $[[dom(SC) ⊆ uv(uM')]]$,
                                \item \label{point:typing-completeness:AppLet:ih-sound} 
                                    for any $[[Θ ⊢ uσ : uv uM' ]]$ such that $[[ Θ ⊢ uσ : SC ]]$, 
                                    we have $[[ Γ ; Φ ⊢ nf(iM) ● args ⇒> [uσ]uM' ]]$,
                                    which, since $[[Γ ; Φ ⊢ nf(iM) ● args ⇒> ↑iQ uniq]]$,
                                    means $[[Γ ⊢ [uσ]uM' ≈ ↑iQ]]$.
                            \end{enumerate}
                            and
                        \item for any $[[Γ ⊢ iM'']]$
                            such that $[[Γ; Φ ⊢ nf(iM) ● args ⇒> iM'']]$,
                            (and in particular, for $[[Γ ⊢ ↑iQ]]$)
                            there exists $[[uσ1]]$ such that 
                            \begin{enumerate}
                                \item 
                                    $[[ Θ  ⊢ uσ1 : uv uM' ]]$,
                                    $[[ Θ  ⊢ uσ1 : SC ]]$, and 
                                \item $[[Γ ⊢ [uσ1]uM' ≈ iM'']]$, and 
                                    in particular, $[[Γ ⊢ [uσ1]uM' ≈ ↑iQ]]$.
                                    Since $[[uM']]$ is
                                    normalized and free of 
                                    negative algorithmic variables, it means that 
                                    $[[uM' = ↑uP]]$ for some $[[uP]]$ 
                                    ($[[Γ; dom(Θ) ⊢  uP]]$)
                                    that is $[[Γ ⊢ [uσ1]uP ≈ iQ]]$.
                            \end{enumerate}
                    \end{enumerate}
                \item $[[Γ; Φ, x:iQ ⊢ c' : iN]]$
            \end{itemize}

            To infer $[[Γ ; Φ ⊨ let x = v(args); c' : nf(iN)]]$, 
            let us apply the corresponding algorithmic rule 
            (\ruleref{\ottdruleATAppLetLabel}):
            \begin{enumerate}
                \item $[[Γ ; Φ ⊨ v : ↓nf(iM)]]$ holds as noted above;

                \item $[[Γ; Φ ; · ⊨ nf(uM) ● args ⇒> ↑uP ⫤ Θ; SC]]$ holds as noted above;

                \item To show $[[uv uP = dom(SC)]]$ and 
                    $[[SC singular with uσ0]]$ for some $[[uσ0]]$,
                    we apply \cref{lemma:singularity-completeness}
                    with $[[Ξ = uv uP]] = [[uv(uM')]]$ (as noted above, $[[dom(SC) ⊆ uv(uM')]] = [[Ξ]]$).

                    Now we will show that any substitution satisfying $[[SC]]$ is equivalent to $[[uσ1]]$.
                    As noted in \ref{point:typing-completeness:AppLet:ih-sound},
                    for any substitution $[[Θ ⊢ uσ : Ξ]]$, $[[ Θ ⊢ uσ : SC ]]$ implies 
                    $[[Γ ⊢ [uσ]uM' ≈ ↑iQ]]$,
                    which is rewritten as $[[Γ ⊢ [uσ]uP ≈ iQ]]$.
                    And since $[[Γ ⊢ [uσ1]uP ≈ iQ]]$, 
                    we have $[[Γ ⊢ [uσ]uP ≈ [uσ1]uP]]$,
                    which implies $[[Θ ⊢ uσ ≈ uσ1 : Ξ]]$ by \cref{lemma:subst-equiv-algovar}.

                \item Let us show $[[Γ; Φ, x:[uσ0]uP ⊨ c' : nf(iN)]]$.
                    By the soundness of singularity 
                    (\cref{lemma:singularity-soundness}),
                    we have $[[ Θ ⊢ uσ0 : SC ]]$,
                    which by \ref{point:typing-completeness:AppLet:ih-sound}
                    means $[[Γ ⊢ [uσ0]uM' ≈ ↑iQ]]$,
                    that is $[[Γ ⊢ [uσ0]uP ≈ iQ]]$, 
                    and thus, $[[Γ ⊢ Φ, x:iQ ≈ Φ, x:[uσ0]uP]]$.

                    Then by \cref{lemma:decl-typing-context-equiv},
                    $[[Γ; Φ, x:iQ ⊢ c' : iN]]$ can be rewritten as
                    $[[Γ; Φ, x:[uσ0]uP ⊢ c' : iN]]$.
                    Then by the induction hypothesis applied to it, 
                    $[[Γ; Φ, x:[uσ0]uP ⊨ c' : nf(iN)]]$ holds.
            \end{enumerate}

        \item \ruleref{\ottdruleDTForallAppLabel}\\
            Since $[[uN]]$ cannot be a algorithmic variable,  
            if $[[ [uσ]uN ]]$ starts with $[[∀]]$,
            so does $[[uN]]$. This way,
            $[[uN = ∀pas.uN1]]$.
            Then by assumption:
            \begin{enumerate}
                \item $[[Γ ⊢ Θ]]$
                \item $[[Γ; dom(Θ) ⊢  ∀pas.uN1]]$ is free from negative algorithmic variables, 
                    and then $[[Γ, pas; dom(Θ) ⊢  uN1]]$ is free from negative algorithmic variables too;
                \item $[[Θ ⊢ uσ : uv uN1]]$;
                \item $[[Γ ⊢ iM]]$;
                \item $[[Γ; Φ ⊢ [uσ]∀pas.uN1 ● args ⇒> iM]]$, 
                    \label{point:typing-completeness-forall-app-inversion}
                    that is $[[Γ; Φ ⊢ (∀pas.[uσ]uN1) ● args ⇒> iM]]$.
                    Then by inversion there exists $[[σ]]$ such that 
                    \begin{enumerate}
                        \item $[[Γ ⊢ σ : {pas}]]$;
                        \item $[[args ≠ ·]]$ and $[[pas ≠ ·]]$; and
                        \item $[[Γ ; Φ ⊢ [σ][uσ]uN1 ● args ⇒> iM]]$.
                            \label{point:typing-completeness-forall-app-inversion-2}
                            Notice that $[[σ]]$ and $[[uσ]]$ commute because 
                            the codomain of $[[σ]]$ does not contain
                            algorithmic variables (and thus, does not intersect with 
                            the domain of $[[uσ]]$), and the codomain of $[[uσ]]$ is 
                            $[[Γ]]$ and does not intersect with $[[pas]]$---the domain of $[[σ]]$.

                            Let us take fresh $[[â⁺*]]$ and 
                            construct $[[uN0]]$ = $[[ [â⁺*/pas]uN1 ]]$
                            and $[[Θ, â⁺*[Γ] ⊢ uσ0 : uv(uN0)]]$ defined as
                            $$
                            \begin{cases}
                                [[ [uσ0]αî⁺ = [σ]αi⁺ ]] & \text{for $[[αî⁺]] \in  [[{â⁺*} ∩ uv uN0]]$ }\\
                                [[ [uσ0]β̂± = [uσ]β̂± ]] & \text{for $[[β̂±]] \in [[uv uN1]]$}
                            \end{cases}
                            $$

                            Then it is easy to see that $[[ [uσ0][â⁺*/pas]uN1 = [σ][uσ]uN1 ]]$
                            because this substitution compositions coincide on
                            $[[uv(uN1)]] \cup [[fv(uN1)]]$. 
                            In other words, $[[ [uσ0]uN0 = [σ][uσ]uN1 ]]$.

                            Then let us apply the induction hypothesis
                            to $[[Γ; Φ ⊢ [uσ0]uN0 ● args ⇒> iM]]$ and obtain 
                            $[[uM']]$, $[[Θ']]$, and $[[SC]]$ such that
                            \begin{itemize}
                                \item $[[ Γ; Φ; Θ, â⁺*[Γ] ⊨ uN0 ● args ⇒> uM' ⫤ Θ'; SC ]]$ and
                                \item \label{point:typing-completeness-forall-app-inversion-3}
                                for any $[[Θ, â⁺*[Γ]  ⊢ uσ0 : uv(uN0)]]$ and $[[Γ ⊢ iM]]$
                                    such that $[[Γ; Φ ⊢ [uσ0]uN0 ● args ⇒> iM]]$, 
                                    there exists $[[uσ0']]$ such that 
                                \begin{enumerate}
                                    \item $[[Θ'  ⊢uσ0' : uv(uN0) ∪ uv(uM')]]$, $[[ Θ' ⊢ uσ0'  : SC ]] $,
                                    \item $[[Θ, â⁺*[Γ] ⊢ uσ0' ≈ uσ0 : uv uN0]]$, and 
                                    \item $[[Γ ⊢ [uσ0']uM' ≈ iM]]$.
                                \end{enumerate}
                            \end{itemize}
                    \end{enumerate}
            \end{enumerate}
            Let us take $[[uM']]$, $[[Θ']]$, and $[[SC]]$ from the induction hypothesis
            (\ref{point:typing-completeness-forall-app-inversion-2}) 
            (from $[[SC]]$ we subtract entries restricting $[[{â⁺*}]]$)
            and show they satisfy the required properties 
            \begin{enumerate}
                \item To infer $[[ Γ; Φ; Θ ⊨ ∀pas.uN1 ● args ⇒> uM' ⫤ Θ'; SC \ {â⁺*} ]]$
                    we apply the corresponding algorithmic rule \ruleref{\ottdruleATForallAppLabel}.
                    As noted above, the required premises hold:
                    \begin{enumerate}
                        \item $[[args ≠ ·]]$, $[[pas ≠ ·]]$; and
                        \item $[[Γ; Φ; Θ, â⁺*[Γ] ⊨ [â⁺*/pas]uN1 ● args ⇒> uM' ⫤ Θ'; SC]]$
                            is obtained by unfolding the definition of $[[uN0]]$
                            in $[[ Γ; Φ; Θ, â⁺*[Γ] ⊨ uN0 ● args ⇒> uM' ⫤ Θ'; SC ]]$
                            (\ref{point:typing-completeness-forall-app-inversion-2}).
                    \end{enumerate}
                \item Let us take and arbitrary $[[Θ ⊢ uσ : uv uN1]]$ and $[[Γ ⊢ iM]]$
                    and assume $[[Γ; Φ ⊢ [uσ]∀pas.uN1  ● args ⇒> iM]]$. 
                    Then the same reasoning as in 
                    \ref{point:typing-completeness-forall-app-inversion-2}
                    applies. In particular, we construct 
                    $[[Θ, â⁺*[Γ] ⊢ uσ0 : uv(uN0)]]$ 
                        as an extension of $[[uσ]]$ and obtain 
                    $[[Γ; Φ ⊢ [uσ0]uN0 ● args ⇒> iM]]$.

                    It means we can apply the property inferred from the induction 
                    hypothesis (\ref{point:typing-completeness-forall-app-inversion-3})
                    to obtain $[[uσ0']]$ such that 
                    \begin{enumerate}
                        \item $[[Θ' ⊢ uσ0' : uv(uN0) ∪ uv(uM')]]$ and $[[ Θ'  ⊢ uσ0' : SC ]] $,
                        \item $[[Θ, â⁺*[Γ] ⊢ uσ0' ≈ uσ0 : uv uN0]]$, and 
                        \item $[[Γ ⊢ [uσ0']uM' ≈ iM]]$.
                    \end{enumerate}

                    Let us show that $[[uσ0'|(uv(uN1) ∪ uv(uM'))]]$ 
                    satisfies the required properties.
                    \begin{enumerate}
                        \item $[[Θ' ⊢ uσ0'|(uv(uN1) ∪ uv(uM')): (uv(uN1) ∪ uv(uM'))]]$
                            holds since $[[ Θ' ⊢ uσ0' : uv(uN0) ∪ uv(uM')]]$
                            and $[[uv(uN1) ∪ uv(uM') ⊆ uv(uN0) ∪ uv(uM')]]$;
                            $[[ Θ' ⊢ uσ0'|(uv(uN1) ∪ uv(uM')) : SC \ {â⁺*} ]]$ holds since
                            $[[ Θ' ⊢ uσ0' : SC ]]$,
                            $[[ Θ' ⊢ uσ0' : uv(uN0) ∪ uv(uM') ]]$,
                            and $[[(uv(uN0) ∪ uv(uM')) \ {â⁺*} = uv(uN1) ∪ uv(uM')]]$.

                        \item $[[Γ ⊢ [uσ0']uM' ≈ iM]]$ holds as shown,
                            and hence it holds for $[[uσ0'|(uv(uN1) ∪ uv(uM'))]]$;
                        \item We show $[[Θ ⊢ uσ0' ≈ uσ : uv uN1]]$, from which
                            it follows that it holds for\\ $[[uσ0'|(uv(uN1) ∪ uv(uM'))]]$.
                            Let us take an arbitrary 
                            $[[β̂±]] \in [[dom(Θ)]] \subseteq [[dom(Θ) ∪ {â⁺*}]]$. Then 
                            since $[[Θ, â⁺*[Γ] ⊢ uσ0' ≈ uσ0 : uv uN0]]$, 
                            we have $[[Θ(β̂±) ⊢ [uσ0']β̂±  ≈ [uσ0]β̂± ]]$ and 
                            by definition of $[[uσ0]]$, $[[ [uσ0]β̂±  = [uσ]β̂± ]]$.
                    \end{enumerate}
            \end{enumerate}
           
        \item \ruleref{\ottdruleDTArrowAppLabel}\\
            Since $[[uN]]$ cannot be a algorithmic variable,  
            if the shape of $[[ [uσ]uN ]]$ is an arrow, 
            so is the shape of $[[uN]]$. This way, 
            $[[uN = uQ → uN1]]$.
            Then by assumption:
            \begin{enumerate}
                \item $[[Γ ⊢ Θ]]$;
                \item $[[Γ; dom(Θ) ⊢  uQ → uN1]]$ is free from negative algorithmic variables;
                \item $[[Θ ⊢ uσ : uv uQ ∪ uv uN1]]$;
                \item $[[Γ ⊢ iM]]$;
                \item $[[Γ; Φ ⊢ [uσ](uQ → uN1) ● v, args ⇒> iM]]$, 
                    \label{point:typing-completeness-arrow-app-inversion}
                    that is $[[Γ; Φ ⊢ ([uσ]uQ → [uσ]uN1) ● v, args ⇒> iM]]$,
                    and by inversion:
                    \begin{enumerate}
                        \item $[[Γ; Φ ⊢ v : iP]]$,
                            and by the induction hypothesis, 
                            $[[Γ; Φ ⊨ v : nf(iP)]]$;
                        \item $[[Γ ⊢ [uσ]uQ ≥ iP]]$, 
                            which by transitivity (\cref{lemma:subtyping-transitivity}) means 
                            $[[Γ ⊢ [uσ]uQ ≥ nf(iP)]]$,
                            and then by completeness of subtyping 
                            (\cref{lemma:pos-subt-completeness}),
                            $[[ Γ; Θ ⊨ uQ ≥ nf(iP) ⫤ SC1 ]]$, 
                            for some $[[Θ ⊢ SC1 : uv(uQ)]]$, and moreover, $[[ Θ ⊢ uσ : SC1 ]]$;
                        \item $[[Γ; Φ ⊢ [uσ]uN1 ● args ⇒> iM]]$. 
                            \label{point:completeness-arrow-app-ih}
                            Notice that the induction hypothesis is applicable to this case:
                            $[[Γ ; dom(Θ) ⊢  uN1]]$ is free from negative algorithmic variables because
                            so is $[[uQ → uN1]]$. This way, there exist 
                            $[[uM']]$, $[[Θ']]$, and $[[SC2]]$ such that 
                            \begin{enumerate}
                                \item $[[ Γ; Φ; Θ ⊨ uN1 ● args ⇒> uM' ⫤ Θ'; SC2 ]]$
                                    and then by soundness of typing 
                                    (i.e. the induction hypothesis), 
                                    \begin{enumerate}
                                        \item $[[Θ ⊆ Θ']]$
                                        \item $[[Γ; dom(Θ') ⊢  uM']]$
                                        \item $[[dom(Θ) ∩ uv(uM') ⊆ uv uN1]]$
                                        \item $[[Θ'|uv uN1 ∪ uv uM' ⊢ SC2]]$
                                    \end{enumerate}
                                \item  \label{point:new-subdst}
                                    for any $[[Θ ⊢ uσ : uv(uN1)]]$ and $[[Γ ⊢ iM]]$
                                    such that $[[Γ; Φ ⊢ [uσ]uN1 ● args ⇒> iM]]$, 
                                    there exists $[[uσ']]$ such that 
                                    \begin{enumerate}
                                        \item $[[Θ' ⊢ uσ' : uv(uN1) ∪ uv(uM')]]$ and $[[Θ' ⊢ uσ' : SC2]]$,
                                        \item $[[Θ ⊢ uσ' ≈ uσ : uv(uN1)]]$, and 
                                        \item $[[Γ ⊢ [uσ']uM' ≈ iM]]$.
                                    \end{enumerate}
                            \end{enumerate}
                    \end{enumerate}
            \end{enumerate}

            We need to show that there exist $[[uM']]$, $[[Θ']]$, and $[[SC]]$ such that
            $[[ Γ; Φ; Θ ⊨ uQ → uN1 ● v, args ⇒> uM' ⫤ Θ'; SC ]]$ and
            the initiality property holds. 
            We take $[[uM']]$ and $[[Θ']]$ from the induction hypothesis
            (\ref{point:completeness-arrow-app-ih}), and $[[SC]]$
            as a merge of $[[SC1]]$ and $[[SC2]]$.
            To show that $[[Θ' ⊢ SC1 & SC2 = SC]]$ exists,
            we apply \cref{lemma:merge-completeness}.
            To do so, we need to provide 
            a substitution satisfying both 
            $[[SC1]]$ and $[[SC2]]$.

            Notice that $[[dom(SC1) = uv(uQ)]]$ and
            $[[dom(SC2) ⊆ uv uN1 ∪ uv uM']]$.
            This way, it suffices to construct 
            $[[Θ' ⊢ uσ'' : uv(uQ) ∪ uv uN1 ∪ uv uM']]$ such that
            $[[Θ' ⊢ uσ'' : SC1]]$ and $[[Θ' ⊢ uσ'' : SC2]]$.

            By the induction hypothesis (\ref{point:new-subdst}),
            $[[uσ|uv(uN1)]]$
            can be extended to $[[uσ']]$ such that
            \begin{enumerate}
                \item $[[Θ' ⊢ uσ' : uv(uN1) ∪ uv(uM')]]$ and $[[Θ' ⊢ uσ' : SC2]]$,
                \item $[[Θ ⊢ uσ' ≈ uσ: uv(uN1)]]$, and 
                \item $[[Γ ⊢ [uσ']uM' ≈ iM]]$.
            \end{enumerate}
            Let us extend $[[uσ']]$ to $[[uσ'']]$
            defined on $[[uv(uQ) ∪ uv(uN1) ∪ uv(uM')]]$
            with values of $[[uσ]]$ as follows:
            $$
            \begin{cases}
                [[ [uσ'']β̂± = [uσ']β̂± ]] & \text{for $[[β̂±]] \in [[uv(uN1) ∪ uv(uM')]]$}\\
                [[ [uσ'']γ̂± = [uσ]γ̂± ]] & \text{for $[[γ̂±]] \in [[uv(uQ) \ (uv(uN1) ∪ uv(uM'))]]$}
            \end{cases}
            $$

            First, notice that $[[Θ' ⊢ uσ'' ≈ uσ' : uv(uN1) ∪ uv(uM')]]$
            by definition.
            Then since $[[Θ' ⊢ uσ' : SC2]]$ and
            $[[Θ' ⊢ SC2 : uv(uN1) ∪ uv(uM')]]$, 
            we have $[[Θ' ⊢ uσ'' : SC2]]$.

            Second, notice that $[[Θ ⊢ uσ'' ≈ uσ : uv(uN1) ∪ uv(uQ)]]$:
            \begin{itemize}
                \item if $[[γ̂±]] \in [[uv(uQ) \ (uv(uN1) ∪ uv(uM'))]]$
                    then $[[ [uσ'']γ̂± = [uσ]γ̂± ]]$ by definition of $[[uσ'']]$;
                \item if $[[γ̂±]] \in [[uv(uQ) ∩ uv(uN1)]]$
                    then $[[ [uσ'']γ̂± = [uσ']γ̂± ]]$,
                    and $[[Θ ⊢ uσ' ≈ uσ: uv(uN1)]]$, as noted above;
                \item if $[[γ̂±]] \in [[uv(uQ) ∩ uv(uM')]]$
                    then since $[[Γ; dom(Θ) ⊢  uQ]]$, 
                    we have $[[uv(uQ) ⊆ dom(Θ)]]$,
                    implying 
                    $[[γ̂±]] \in [[dom(Θ) ∩ uv(uM') ⊆ uv(uN1)]]$.
                    This way, $[[γ̂±]] \in [[uv(uQ) ∩ uv(uN1)]]$, 
                    and this case is covered by the previous one.
            \end{itemize}
            In particular, $[[Θ ⊢ uσ'' ≈ uσ : uv(uQ)]]$.
            Then since $[[Θ ⊢ uσ : SC1]]$ and $[[Θ ⊢ SC1 : uv(uQ)]]$,
            we have $[[Θ ⊢ uσ'' : SC1]]$.

            This way, $[[uσ']]$ satisfies both $[[SC1]]$ and $[[SC2]]$,
            and by the completeness of constraint merge 
            (\cref{lemma:merge-completeness}),
            $[[Θ' ⊢ SC1 & SC2 = SC]]$ exists.


            Finally, to show the required properties, we take
            $[[uM']]$ and $[[Θ']]$ from the induction hypothesis (\ref{point:new-subdst}), 
            and $[[SC]]$ defined above. Then
            \begin{enumerate}
                \item $[[ Γ; Φ; Θ ⊨ uQ → uN1 ● v,args ⇒> uM' ⫤ Θ'; SC ]]$
                    is inferred by \ruleref{\ottdruleATArrowAppLabel}.
                    As noted above:
                    \begin{enumerate}
                        \item $[[Γ; Φ ⊨ v : nf(iP)]]$,
                        \item $[[Γ; Θ ⊨ uQ ≥ nf(iP) ⫤ SC1]]$,
                        \item $[[Γ; Φ; Θ ⊨ uN1 ● args ⇒> uM' ⫤ Θ'; SC2]]$, and
                        \item $[[Θ' ⊢ SC1 & SC2 = SC]]$.
                    \end{enumerate}
                \item let us take an arbitrary 
                    $[[Θ ⊢ uσ0 : uv uQ ∪ uv uN1]]$;
                    and $[[Γ ⊢ iM0]]$;
                    such that $[[Γ; Φ ⊢ [uσ0](uQ → uN1) ● v,args ⇒> iM0]]$.
                    Then by inversion of 
                    $[[Γ; Φ ⊢ [uσ0]uQ → [uσ0]uN1 ● v, args ⇒> iM0]]$,
                    we have the same properties as in 
                    \ref{point:typing-completeness-arrow-app-inversion}.
                    In particular,
                    \begin{itemize}
                        \item $[[Γ ⊢ [uσ0]uQ ≥ nf(iP)]]$
                            and by the completeness of subtyping 
                            (\cref{lemma:pos-subt-completeness}),
                            $[[ Θ  ⊢ uσ0 : SC1  ]]$.
                        \item $[[Γ; Φ ⊢ [uσ0]uN1 ● args ⇒> iM0]]$. 
                            Then by \ref{point:new-subdst}, 
                            there exists $[[uσ0']]$ such that 
                            \begin{enumerate}
                                    \item $[[Θ' ⊢ uσ0' : uv(uN1) ∪ uv(uM')]]$ and 
                                        $[[Θ' ⊢ uσ0' : SC2]]$,
                                    \item $[[Θ ⊢ uσ0' ≈ uσ0 : uv(uN1)]]$, and 
                                    \item $[[Γ ⊢ [uσ0']uM' ≈ iM0]]$.
                            \end{enumerate}
                    \end{itemize}
                    Let us extend $[[uσ0']]$ to be defined on
                    $[[uv(uQ) ∪ uv(uN1) ∪ uv(uM')]]$
                    with the values of $[[uσ0]]$.
                    We define $[[uσ0'']]$ as follows:
                    $$
                    \begin{cases}
                        [[ [uσ0'']γ̂± = [uσ0']γ̂± ]] & \text{for $[[γ̂±]] \in [[uv(uN1) ∪ uv(uM')]]$}\\
                        [[ [uσ0'']γ̂± = [uσ0]γ̂± ]] & \text{for $[[γ̂±]] \in [[uv(uQ) \ (uv(uN1) ∪ uv(uM'))]]$}
                    \end{cases}
                    $$
                    This way, 
                    \begin{itemize}
                        \item $[[Θ' ⊢ uσ0'' : uv(uQ) ∪ uv(uN1) ∪ uv(uM')]]$,
                        \item $[[Θ' ⊢ uσ0'' : SC]]$,
                            since $[[Θ' ⊢ uσ0'' : SC1]]$ and $[[Θ' ⊢ uσ0'' : SC2]]$,
                            which is proved similarly to
                            $[[Θ' ⊢ uσ'' : SC1]]$ and $[[Θ' ⊢ uσ'' : SC2]]$ above;
                        \item $[[Θ ⊢ uσ0'' ≈ uσ0 : uv(uN1) ∪ uv(uQ)]]$:
                            the proof is analogous to
                            $[[Θ ⊢ uσ'' ≈ uσ : uv(uN1) ∪ uv(uQ)]]$ above.
                        \item $[[Γ ⊢ [uσ0'']uM' ≈ iM0]]$
                            Notice that $[[Θ' ⊢ uσ0'' ≈ uσ0' : uv(uN1) ∪ uv(uM')]]$,
                            which is proved analogously to
                            $[[Θ' ⊢ uσ'' ≈ uσ' : uv(uN1) ∪ uv(uM')]]$ above.
                            Then $[[Γ ⊢ [uσ0']uM' ≈ iM0]]$
                            can be rewritten to $[[Γ ⊢ [uσ0'']uM' ≈ iM0]]$.
                    \end{itemize}
            \end{enumerate}
            


        \item \ruleref{\ottdruleDTEmptyAppLabel}\\
            By assumption: 
            \begin{enumerate}
                \item $[[Γ ⊢ Θ]]$,
                \item $[[Γ ⊢ iN']]$,
                \item $[[Γ; dom(Θ) ⊢ uN]]$ and $[[uN]]$ is free from negative variables,
                \item $[[Θ ⊢ uσ : uv(uN)]]$,
                \item $[[Γ; Φ ⊢ [uσ]uN ● · ⇒> iN' ]]$,
                    and by inversion, $[[Γ ⊢ [uσ]uN ≈ iN']]$.
            \end{enumerate}


            Then we can apply the corresponding algorithmic rule
            \ruleref{\ottdruleATEmptyAppLabel} to infer
            $[[ Γ; Φ; Θ ⊨ uN ● · ⇒> nf(uN) ⫤ Θ; · ]]$.
            Let us show the required properties. 
            Let us take an arbitrary 
            $[[Θ ⊢ uσ0 : uv(uN)]]$ and $[[Γ ⊢ iM]]$
            such that $[[Γ; Φ ⊢ [uσ1]uN ● · ⇒> iM]]$. 
            Then we can take $[[uσ0]]$ as the required substitution:
            \begin{enumerate}
                \item $[[ Θ ⊢ uσ0 : uv(uN) ∪ uv(nf(uN)) ]]$,
                    since $[[uv(nf(uN)) = uv(uN)]]$, 
                    and thus, $[[uv(uN) ∪ uv(nf(uN)) = uv(uN)]]$;
                \item $[[ Θ ⊢ uσ0 : · ]]$ vacuously;
                \item $[[Θ ⊢ uσ0 ≈ uσ0 : uv(uN)]]$ by reflexivity;
                \item Let us show $[[Γ ⊢ [uσ0]nf(uN) ≈ iM]]$.
                    Notice that $[[Γ; Φ ⊢ [uσ0]uN ● · ⇒> iM]]$ can only be inferred by 
                    \ruleref{\ottdruleDTEmptyAppLabel}, and thus, $[[ Γ ⊢ [uσ0]uN ≈ iM ]]$.
                    By \cref{corollary:nf-pres-subt},
                    $[[Γ ⊢ [uσ0]uN ≈ [uσ0]nf(uN)]]$,
                    and then by transitivity, $[[Γ ⊢ [uσ0]nf(uN) ≈ iM]]$,
                    that is $[[Γ ⊢ [uσ0]nf(uN) ≈ iM]]$.
            \end{enumerate}
    \end{caseof}
\end{proof}
