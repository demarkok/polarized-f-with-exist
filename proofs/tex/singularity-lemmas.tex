\lemMinInstSoundness*
\begin{proof}
    We prove it by induction on the inference of $[[Γ ⊢ uP SC minby uσ ]]$.
    Let us consider the last rule used in the inference.
    \begin{caseof}
        \item \ruleref{\ottdruleSINGPUvarLabel}, which means 
            that the inferred judgment is $[[Γ ⊢  α̂⁺ SC minby (nf(iP) / α̂⁺)]]$, 
            and by inversion, $[[(α̂⁺ :≥ iP) ∊ SC]]$. Let us show the required properties:
            \begin{itemize}
                \item $[[Θ ⊢ (nf(iP) / α̂⁺) : uv α̂⁺]]$ holds trivially;
                \item $[[Θ ⊢  (nf(iP) / α̂⁺) : SC]]$ holds since
                    $[[Γ ⊢ nf(iP) : (α̂⁺ :≥ iP)]]$, which is true 
                    since $[[Γ ⊢ nf(iP) ≥ iP]]$ by the soundness of normalization
                    (\cref{lemma:normalization-soundness});
                \item $[[(nf(iP) / α̂⁺)]]$ is normalized trivially;
                \item let us take an arbitrary $[[Θ ⊢ uσ' : {α̂⁺} ]]$
                    respecting $[[SC]]$. Since $[[uσ']]$ respects $[[SC]]$,
                    $[[Γ ⊢ [uσ']α̂⁺ ≥ iP]]$ holds, and then
                    $[[Γ ⊢ [uσ']α̂⁺ ≥ nf(iP)]]$ holds by the soundness of normalization
                    and transitivity of subtyping. Finally,
                    $[[Γ ⊢ [uσ']α̂⁺ ≥ nf(iP)]]$ can be rewritten as
                    $[[Γ ⊢ [uσ']α̂⁺ ≥ [(nf(iP) / α̂⁺)]α̂⁺]]$.
            \end{itemize}
        \item \ruleref{\ottdruleSINGExistsLabel}, which means that 
            the inferred judgment has form $[[Γ ⊢ ∃nas.uP SC minby uσ ]]$,
            and by inversion, $[[Γ, nas ⊢ uP SC minby uσ]]$.
            By applying the induction hypothesis to $[[Γ, nas ⊢ uP SC minby uσ]]$
            we have 
            \begin{itemize}
                \item $[[Θ ⊢ uσ : uv uP]]$, which also means $[[Θ ⊢ uσ : uv ∃nas.uP]]$,
                \item $[[Θ ⊢ uσ : SC]]$,
                \item $[[uσ]]$ is normalized, and
                \item for any other $[[Θ ⊢ uσ' : uv uP ]]$
                    respecting $[[SC]]$ (\ie $[[Θ ⊢ uσ' : SC]]$), we have
                    $[[Γ, nas ⊢ [uσ']uP ≥ [uσ]uP ]]$, which immediately implies 
                    $[[Γ ⊢ [uσ']∃nas.uP ≥ [uσ]∃nas.uP ]]$ (the left-hand side
                    existential variables are instantiated with the corresponding 
                    right-hand side existential variables).
            \end{itemize}
        \item \ruleref{\ottdruleSINGSingLabel}, which means that
            the inferred judgment has form $[[Γ ⊢ uP SC minby uσ]]$,
            and by inversion, $[[uv(uP) ⊆ dom(SC)]]$ and
            $[[SC|uv(uP) singular with uσ]]$. Let us apply the soundness of singularity
            (\cref{lemma:singularity-soundness}) to $[[SC|uv(uP) singular with uσ]]$ to
            obtain the following properties:
            \begin{itemize}
                \item $[[ Θ ⊢ uσ : uv(uP) ∩ dom(SC) ]]$, which also means 
                    $[[Θ ⊢ uσ : uv(uP)]]$,
                \item $[[Θ ⊢ uσ : SC|uv(uP)]]$,
                \item $[[uσ]]$ is normalized, and
                \item for any other $[[Θ ⊢ uσ' : uv(uP) ]]$
                    respecting $[[SC|uv(uP)]]$, we have
                    $[[Θ ⊢ uσ' ≈ uσ : uv(uP)]]$. The latter means that
                    $[[Γ ⊢ [uσ']uP ≈ [uσ]uP]]$, and in particular, 
                    $[[Γ ⊢ [uσ']uP ≥ [uσ]uP]]$.
            \end{itemize}
    \end{caseof}

\end{proof}

\lemMinInstCompleteness*
\begin{proof}
    We prove it by induction on $[[uP]]$.
    \begin{caseof}
        \item \label{case:min-inst-completeness-uvar} 
            $[[uP]] = [[α̂⁺]]$. Suppose that $[[α̂⁺ ∉ dom(SC)]]$.
            Then the instantiation of $[[α̂⁺]]$ is
            not restricted, and thus, any type can instantiate it.
            However, among unrestricted instantiations, there is no minimum:
            any type $[[iP]]$ is \emph{not} a subtype of $[[↓↑iP]]$,
            which contradicts the assumption.
            This way, $[[α̂⁺ ∊ dom(SC)]]$.

            If the entry restricting $[[α̂⁺]]$ in $[[SC]]$ is a \emph{subtyping} entry
            ($[[(α̂⁺ :≥ iQ) ∊ SC]]$), then 
            we apply \ruleref{\ottdruleSINGPUvarLabel} to infer 
            $[[Γ ⊢  α̂⁺ SC minby (nf(iQ) / α̂⁺)]]$.
            It is left to show that $[[ nf(iQ) = nf([uσ]α̂⁺) ]]$.
            Since $[[Θ ⊢ uσ : SC]]$, and $[[(α̂⁺ :≥ iQ) ∊ SC]]$, 
            we know that $[[Γ ⊢ [uσ]α̂⁺ ≥ iQ]]$. 
            On the other hand, let us consider $[[Θ ⊢ uσ' : SC]]$,
            that copies $[[uσ]]$ on $[[dom(SC)]]$ except $[[α̂⁺]]$,
            where it is instantiated with $[[iQ]]$.
            Then $[[Γ ⊢ [uσ']α̂⁺ ≥ [uσ]α̂⁺ ]]$ means $[[Γ ⊢ iQ ≥ [uσ]α̂⁺ ]]$,
            this way, $[[Γ ⊢ iQ ≈ [uσ]α̂⁺ ]]$, which by \cref{lemma:subt-equiv-algorithmization}
            means $[[nf(iQ) = nf([uσ]α̂⁺)]]$.

            If the entry restricting $[[α̂⁺]]$ in $[[SC]]$ is an \emph{equivalence} entry
            ($[[(α̂⁺ :≈ iQ) ∊ SC]]$), then we wish to apply
            \ruleref{\ottdruleSINGSingLabel}.
            The first premise $[[uv(α̂⁺) ⊆ dom(SC)]]$  
            holds by assumption;
            to infer $[[SC|{α̂⁺} singular with uσ0]]$, 
            we apply the completeness of singularity (\cref{lemma:singularity-completeness}).
            It applies because all the substitutions satisfying $[[SC|{α̂⁺}]] = [[(α̂⁺ :≈ iQ)]]$
            are equivalent on $[[{α̂⁺}]]$ by transitivity of equivalence 
            (\cref{corollary:equivalence-transitivity}): 
            the satisfaction of this constraint means that the substitution 
            sends $[[α̂⁺]]$ to $[[iQ]]$ or an equivalent type.
            This way, $[[SC|{α̂⁺} singular with uσ0]]$ for some $[[uσ0]]$,
            which means $[[Γ ⊢ α̂⁺ SC minby uσ0]]$.
            To show that $[[uσ0 = nf(uσ)]]$ notice that 
            Since $[[uσ0]]$ is normalized and equivalent to $[[uσ]]$ on $[[{α̂⁺}]]$, 
            and only has $[[α̂⁺]]$ in its domain
            (by soundness of singularity, \cref{lemma:singularity-soundness}).
            This way, $[[Γ ⊢ α̂⁺ SC minby uσ]]$, as required.
        \item  \label{case:min-inst-completeness-down}
            $[[uP = ↓uN]]$. Then since
            $[[Γ ⊢ ↓[uσ']uN ≥ ↓[uσ]uN ]]$ means
            $[[Γ ⊢ ↓[uσ']uN ≈ ↓[uσ]uN ]]$ by inversion.
            Then by \cref{lemma:subst-equiv}, 
            $[[uσ]]$ is equivalent to any other substitution 
            $[[Θ ⊢ uσ' : uv uN ]]$ satisfying $[[SC|uv uN]]$,
            hence, the completeness of singularity (\cref{lemma:singularity-completeness})
            can be applied to conclude that
            \begin{itemize}
                \item $[[uv(uN) = dom(SC|uv uN)]]$, then $[[uv(uP) ⊆ dom(SC)]]$,
                \item $[[SC|uv uN singular with uσ0]]$ for some (normalized) $[[uσ0]]$.
            \end{itemize}
            It means $[[Γ ⊢ uP SC minby uσ0 ]]$, and then
            since $[[uσ0]]$ is normalized and equivalent to $[[uσ0]]$ on $[[uv(uN)]]$,
            and its domain is $[[uv(uN)]]$, $[[uσ0 = nf(uσ)]]$.
        % \item $[[uP = ∃nas.uP']]$ 
        %     Then we wish to apply the induction hypothesis to show 
        %     $[[Γ, nas ⊢ uP' SC minby nf(uσ)]]$, which would imply 
        %     $[[Γ ⊢ ∃nas.uP' SC minby nf(uσ)]]$ by \ruleref{\ottdruleMINExistsLabel}.
        %     Let us demonstrate that the premises of the induction hypothesis hold. 
        %     \begin{itemize}
        %         \item  $[[Γ, nas ⊢ Θ]]$ holds by weakening;
        %         \item  $[[Γ, nas; dom(Θ) ⊢ uP']]$ holds by inversion of $[[Γ; dom(Θ) ⊢ uP]]$;
        %         \item $[[uv uP' = uv uP]]$, and 
        %                 $[[Γ ⊢ [uσ']∃nas.uP' ≥ [uσ]∃nas.uP' ]]$ implies
        %                 $[[Γ, nas ⊢ [uσ']uP' ≥ [uσ]uP' ]]$ by \cref{lemma:subst-equiv}.
        %     \end{itemize}

        \item $[[uP = ∃nas.β⁺]]$ then 
            as there are no algorithmic variables in $[[uP]]$, 
            $[[nf([uσ]uP) = β⁺]]$,
            and thus, we wish to show that 
            $[[Γ ⊢ ∃nas.β⁺ SC minby ·]]$.
            To do so, we apply \ruleref{\ottdruleSINGExistsLabel},
            and it is left to show that $[[Γ, nas ⊢ β⁺ SC minby · ]]$,
            which holds vacuously by \ruleref{\ottdruleSINGSingLabel}.
        \item $[[uP = ∃nas.α̂⁺]]$ then 
            $[[Γ ⊢ [uσ']∃nas.α̂⁺ ≥ [uσ]∃nas.α̂⁺ ]]$ implies 
            $[[Γ ⊢ ∃nas.[uσ']α̂⁺ ≥ ∃nas.[uσ]α̂⁺ ]]$ implies 
            $[[Γ ⊢ [uσ']α̂⁺ ≥ [uσ]α̂⁺ ]]$.
            It means that $[[uσ]]$ instantiates $[[α̂⁺]]$ to the minimal 
            type among all the instantiations of $[[α̂⁺]]$ respecting $[[SC]]$.
            In other words, we can apply the reasoning from 
            \cref{case:min-inst-completeness-uvar}
            to conclude that $[[Γ ⊢ α̂⁺ SC minby nf(uσ) ]]$.
            And then \ruleref{\ottdruleSINGExistsLabel}
            gives us $[[Γ ⊢ ∃nas.α̂⁺ SC minby nf(uσ) ]]$.
        \item $[[uP = ∃nas.↓uN]]$ then 
            $[[Γ ⊢ [uσ']∃nas.↓uN ≥ ∃nas.↓[uσ]uN ]]$ implies 
            $[[Γ ⊢ ∃nas.↓[uσ']uN ≥ ∃nas.↓[uσ]uN ]]$ implies
            $[[Γ, nas ⊢ ↓[σ0][uσ']uN ≥ ↓[uσ]uN ]]$ for some $[[σ0]]$ implies
            $[[Γ, nas ⊢ [σ0][uσ']uN ≈ [uσ]uN ]]$.
            By \cref{lemma:subst-equiv}, it means in particular
            that $[[uσ']]$ and $[[uσ]]$ are equivalent on 
            $[[uv uN]]$. 
            This way, we can apply the completeness of singularity 
            (\cref{lemma:singularity-completeness}), and continue as in
            \cref{case:min-inst-completeness-down}
            to conclude that $[[Γ, nas ⊢ ↓uN SC minby nf(uσ)]]$.
            Then by \ruleref{\ottdruleSINGExistsLabel},
            we have $[[Γ ⊢ ∃nas.↓uN minby nf(uσ)]]$.
    \end{caseof}
\end{proof}

\obsMinInstDeterministic*
\begin{proof}
    We prove it by induction on $[[Γ ⊢ uP SC minby uσ]]$.
    It is easy to see that each inference rule is deterministic.
\end{proof}

\lemEntrySingularitySoundness*
\begin{proof}
    Let us consider how $[[scE singular with iP]]$ or $[[scE singular with iN]]$ is formed.
    \begin{caseof}
        \item \ruleref{\ottdruleSINGNEqLabel}, that is $[[scE]] = [[α̂⁻ :≈ iN0]]$.
            and $[[iN]]$ is $[[nf(iN0)]]$.
            Then $[[Γ ⊢ iN' : scE]]$ means $[[Γ ⊢ iN' ≈ iN0]]$, 
            (by inversion of \ruleref{\ottdruleSATSCENEqLabel}),
            which by transitivity, using \cref{corollary:nf-sound-wrt-subt-equiv},
            means $[[Γ ⊢ iN' ≈ nf(iN0)]]$, 
            as required.
        \item \ruleref{\ottdruleSINGPEqLabel}. This case is symmetric to the previous one.

        \item \ruleref{\ottdruleSINGSupVarLabel}, that is 
            $[[scE]] = [[α̂⁺ :≥ ∃nas.β⁺]]$, and $[[iP = β⁺]]$.

            Since $[[Γ ⊢ β⁺ ≥  ∃nas.β⁺]]$, we have $[[Γ ⊢ β⁺ : scE ]]$, 
            as required.

            Notice that $[[Γ ⊢ iP' : scE]]$ means $[[Γ ⊢ iP' ≥ ∃nas.β⁺]]$.
            Let us show that it implies $[[Γ ⊢ iP' ≈ β⁺]]$.
            By applying \cref{lemma:shape-of-supertypes} once, 
            we have $[[Γ, nas ⊢ iP' ≥ β⁺]]$.
            By applying it again, we notice that
            $[[Γ, nas ⊢ iP' ≥ β⁺]]$ implies $[[iPi = ∃nas'.β⁺]]$.
            Finally, it is easy to see that $[[Γ ⊢ ∃nas'.β⁺ ≈ β⁺]]$

        \item \ruleref{\ottdruleSINGSupShiftLabel},
            that is $[[scE]] = [[α̂⁺ :≥ ∃nbs.↓iN1]]$, 
            where $[[iN1 ≈ nbj]]$, and $[[iP = ∃α⁻.↓α⁻]]$.

            Since $[[Γ ⊢ ∃α⁻.↓α⁻ ≥ ∃nbs.↓iN1]]$ 
            (by \ruleref{\ottdruleDOneExistsLabel}, with substitution $[[iN1 / α⁻]]$),
            we have $[[Γ ⊢ ∃α⁻.↓α⁻ : scE ]]$, as required.

            Notice $[[Γ ⊢ iP' : scE]]$ means $[[Γ ⊢ iP' ≥ ∃nbs.↓iN1]]$.
            Let us show that it implies $[[Γ ⊢ iP' ≈ ∃α⁻.↓α⁻]]$.
            \begin{align*}
            [[Γ ⊢ iP' ≥ ∃nbs.↓iN1]] &\Rightarrow [[Γ ⊢ nf(iP') ≥ ∃nbs'.↓nf(iN1)]]\\
                                    &\phantom{\Rightarrow} \text{(where $[[ord {nbs} in iN' = nbs']]$)}
                                    && \text{by \cref{corollary:nf-pres-subt}} \\
                                    &\Rightarrow [[Γ ⊢ nf(iP') ≥ ∃nbs'.↓nf(nbj)]]  
                                    && \text{by \cref{lemma:normalization-completeness}}\\
                                    &\Rightarrow [[Γ ⊢ nf(iP') ≥ ∃nbs'.↓nbn]]  
                                    && \text{by definition of normalization}\\
                                    &\Rightarrow [[Γ ⊢ nf(iP') ≥ ∃nbj.↓nbj]]  
                                    && \text{since $[[ord {nbs} in nf(iN1)]] = [[nbj]]$}\\
                                    &\Rightarrow [[Γ, nbj ⊢ nf(iP') ≥ ↓nbj]] \\ 
                                    &\phantom{\Rightarrow} \text { and } [[nbj ∉ fv(nf(iP'))]]
                                    && \text{by \cref{lemma:shape-supertypes-norm}}\\
            \end{align*}
            By \cref{lemma:shape-supertypes-norm}, 
            the last subtyping means that $[[nf(iP') = ∃nas.↓iN']]$,
            such that
            \begin{enumerate}
                \item $[[Γ, nbj, nas ⊢ iN']]$
                \item $[[ord {nas} in iN' = nas]]$
                \item for some substitution $[[Γ, nbj ⊢ σ :{nas}]]$, 
                    $[[ [σ]iN' = nbj ]]$.
            \end{enumerate}
            Since $[[nbj ∉ fv(nf(iP'))]]$,
            the latter means that $[[iN' = na]]$, and then 
            $[[nf(iP') = ∃na.↓na]]$ for some $[[na]]$.
            Finally, notice that all the types of shape
            $[[∃na.↓na]]$ are equal.
   \end{caseof}

\end{proof}



\lemEntrySingularityCompleteness*
\begin{proof}
    \hfill
    \begin{itemize}
        \item [$-$] 
            By \cref{lemma:constraint-sat},
            there exists $[[Γ ⊢ iN' : scE]]$.
            Since $[[iN']]$ is negative, by inversion of
            $[[Γ ⊢ iN' : scE]]$, $[[scE]]$ has shape $[[α̂⁻ :≈ iM]]$, 
            where $[[Γ ⊢ iN' ≈ iM]]$, and transitively, $[[Γ ⊢ iN ≈ iM]]$.
            Then $[[nf(iM) = nf(iN)]]$, 
            and $[[scE singular with nf(iM)]]$ (by \ruleref{\ottdruleSINGNEqLabel})
            is rewritten as $[[scE singular with nf(iN)]]$.
        \item [$+$]
            By \cref{lemma:constraint-sat}, there exists $[[Γ ⊢ iP' : scE]]$, 
            then by assumption, $[[Γ ⊢ iP' ≈ iP]]$,
            which by \cref{lemma:entry-sat-equiv} implies $[[Γ ⊢ iP : scE]]$.

            Let us consider the shape of $[[scE]]$:
            \begin{caseof}
                \item $[[scE]] = [[(α̂⁺ :≈ iQ)]]$ then 
                    inversion of $[[Γ ⊢ iP : scE]]$
                    implies $[[Γ ⊢ iP ≈ iQ]]$, and hence, $[[nf(iP) = nf(iQ)]]$
                    (by \cref{lemma:subt-equiv-algorithmization}).
                    Then $[[scE singular with nf(iQ)]]$, 
                    which holds by \ruleref{\ottdruleSINGPEqLabel}, 
                    is rewritten as $[[scE singular with nf(iP)]]$.

                \item $[[scE]] = [[(α̂⁺ :≥ iQ)]]$.
                    Then the inversion of $[[Γ ⊢ iP : scE]]$ 
                    implies $[[Γ ⊢ iP ≥ iQ]]$.
                    Let us consider the shape of $[[iQ]]$:
                    \begin{caseof}
                        \item $[[iQ]] = [[∃nbs.β⁺]]$ (for potentially empty $[[nbs]]$).
                            Then $[[Γ ⊢ iP ≥ ∃nbs.β⁺]]$ 
                            implies $[[Γ ⊢ iP ≈ β⁺]]$ by 
                            \cref{lemma:shape-of-supertypes}, 
                            as was noted in the proof of 
                            \cref{lemma:entry-singularity-soundness},
                            and hence, $[[nf(iP) = β⁺]]$.

                            Then $[[scE singular with β⁺]]$, which holds by
                            \ruleref{\ottdruleSINGSupVarLabel},
                            can be rewritten as\\ $[[scE singular with nf(iP)]]$.

                        \item $[[iQ]] = [[∃nbs.↓iN]]$ (for potentially empty $[[nbs]]$).
                            Notice that $[[Γ ⊢ ∃γ⁻.↓γ⁻ ≥ ∃nbs.↓iN]]$ 
                            (by \ruleref{\ottdruleDOneExistsLabel}, 
                            with substitution $[[iN / γ⁻]]$), and thus, 
                            $[[Γ ⊢ ∃γ⁻.↓γ⁻ : scE]]$ by \ruleref{\ottdruleSATSCESupLabel}.
                            
                            Then by assumption, $[[Γ ⊢ ∃γ⁻.↓γ⁻ ≈ iP]]$,
                            that is $[[nf(iP) = ∃γ⁻.↓γ⁻]]$.
                            To apply \ruleref{\ottdruleSINGSupShiftLabel}
                            to infer $[[(α̂⁺ :≥ ∃nbs.↓iN) singular with ∃γ⁻.↓γ⁻]]$,
                            it is left to show that $[[iN ≈ nbi]]$ for some $i$.

                            Since $[[Γ ⊢ iQ : scE]]$, by assumption,
                            $[[Γ ⊢ iQ ≈ iP]]$, and by transitivity, 
                            $[[Γ ⊢ iQ ≈ ∃γ⁻.↓γ⁻]]$.
                            It implies
                            $[[nf(∃nbs.↓iN) = ∃γ⁻.↓γ⁻]]$ (by \cref{lemma:subt-equiv-algorithmization}), 
                            which by definition of normalization means
                            $[[∃nbs'.↓nf(iN) = ∃γ⁻.↓γ⁻]]$, where $[[ord {nbs} in iN' = nbs']]$.
                            This way, $[[nbs']]$ is a variable $[[β⁻]]$, and $[[ nf(iN) = β⁻ ]]$.
                            Notice that $[[β⁻]] \in [[nbs']] \subseteq [[nbs]]$ by \cref{lemma:ord-soundness}.
                            This way, $[[iN ≈ β⁻]]$ for $[[β⁻]] \in [[nbs]]$ (by \cref{lemma:subt-equiv-algorithmization}),
                    \end{caseof}
            \end{caseof}
    \end{itemize}
\end{proof}

\lemSingularitySoundness*
\begin{proof}
    Suppose that $[[Θ ⊢ uσ' : SC]]$.
    It means that for every $[[scE]] \in [[SC]]$ restricting $[[α̂±]]$,
    $[[Θ(α̂±) ⊢ [uσ']α̂± : scE]]$ holds.
    $[[SC singular with uσ]]$ means $[[scE singular with [uσ]α̂± ]]$,
    and hence, by \cref{lemma:entry-singularity-completeness},
    $[[ Θ(α̂±) ⊢ [uσ']α̂± ≈ [uσ]α̂±  ]]$ holds.

    Since the uniqueness holds for every variable from $[[dom(SC)]]$,
    $[[uσ]]$ is equivalent to $[[uσ']]$ on this set.
\end{proof}

\obsSingularityDeterministic*
\begin{proof}
    By \cref{lemma:singularity-soundness},
    $[[Θ ⊢ uσ : Ξ]]$ and $[[Θ ⊢ uσ' : Ξ]]$.
    It means that both $[[uσ]]$ and $[[uσ']]$
    act as identity outside of $[[Ξ]]$.

    Moreover, for any $[[α̂± ∊ Ξ]]$,
    $[[Θ ⊢ SC : Ξ]]$ means that 
    there is a unique $[[scE ∊ SC]]$ restricting $[[α̂±]]$.
    Then $[[SC singular with uσ]]$ means 
    that $[[scE singular with [uσ]α̂± ]]$.
    By looking at the inference rules, it is easy to see that
    $[[ [uσ]α̂± ]]$ is uniquely determined by $[[scE]]$, which, 
    Similarly, $[[ [uσ']α̂± ]]$ is also uniquely determined by $[[scE]]$,  
    in the same way, and hence, $[[ [uσ]α̂± = [uσ']α̂± ]]$.
\end{proof}


\lemSingularityCompleteness*
\begin{proof}

    First, let us assume $[[Ξ]] \neq [[dom(SC)]]$. 
    Then there exists $[[α̂± ∊ Ξ \ dom(SC)]]$.
    Let us take $[[Θ ⊢ uσ1 : Ξ]]$ such that 
    any other substitution $[[Θ ⊢ uσ : Ξ]]$ satisfying $[[SC]]$
    is equivalent to $[[uσ1]]$ on $[[Ξ]]$.

    Notice that $[[Θ ⊢ uσ1 : SC]]$: 
    by \cref{lemma:constraint-sat}, there exists $[[uσ']]$
    such that $[[Θ ⊢ uσ' : Ξ]]$ and $[[Θ ⊢ uσ' : SC]]$,
    and by assumption, $[[Θ ⊢ uσ' ≈ uσ1 : Ξ]]$,
    implying  $[[Θ ⊢ uσ' ≈ uσ1 : dom(SC)]]$.

    Let us construct $[[uσ2]]$ such that $[[Θ ⊢ uσ2 : Ξ]]$ as
    follows:
    $$
    \begin{cases}
        [[ [uσ2]β̂± = [uσ1]β̂±  ]] & \text{if } [[β̂± ≠ α̂±]] \\
        [[ [uσ2]α̂± ]] = T & \text{where $T$ is any closed type not equivalent to $[[ [uσ1]α̂± ]]$} \\
    \end{cases}
    $$
    It is easy to see that $[[Θ ⊢ uσ2 : SC]]$ since
    $[[uσ1|dom(SC) = uσ2|dom(SC)]]$, and $[[Θ ⊢ uσ1 : SC]]$.
    However, $[[Θ ⊢ uσ2 ≈ uσ1 : Ξ]]$ does not hold because
    by construction, $[[Θ(α̂±) ⊢ [uσ2]α̂± ≈ [uσ1]α̂±]]$ does not hold. 
    This way, we have a contradiction.

    Second, let us show $[[SC singular with uσ0]]$.
    Let us take arbitrary $[[scE ∊ SC]]$ restricting $[[β̂±]]$.
    We need to show that $[[scE]]$ is singular.
    Notice that $[[Θ ⊢ uσ1 : SC]]$ implies $[[Θ(β̂±) ⊢ [uσ1]β̂±]]$ and $[[Θ(β̂±) ⊢ [uσ1]β̂± : scE ]]$.
    We will show that any other type satisfying $[[scE]]$ is equivalent to $[[ [uσ1]β̂± ]]$,
    then by \cref{lemma:entry-singularity-completeness}, $[[scE singular with [uσ1]β̂± ]]$.
    \begin{itemize}
        \item if $[[β̂±]]$ is positive, 
            let us take any type $[[Θ(β̂±) ⊢ iP']]$
            and assume $[[Θ(β̂±) ⊢ iP' : scE]]$.
            We will show that $[[Θ(β̂±) ⊢ iP' ≈ [uσ1]β̂±]]$,
            which by \cref{lemma:subt-equiv-algorithmization} will
            imply $[[scE singular with nf([uσ1]β̂±)]]$.

            Let us construct $[[uσ2]]$ such that $[[Θ ⊢ uσ2 : Ξ]]$ as
            follows:
            $$
            \begin{cases}
                [[ [uσ2]γ̂± = [uσ1]γ̂±  ]] & \text{if } [[γ̂± ≠ β̂±]] \\
                [[ [uσ2]β̂± = iP' ]]
            \end{cases}
            $$
            It is easy to see that $[[Θ ⊢ uσ2 : SC]]$:
            for $[[scE]]$, $[[Θ(β̂±) ⊢ [uσ2]β̂± : scE ]]$ by construction,
            since $[[Θ(β̂±) ⊢ iP' : scE]]$; for any other $[[scE' ∊ SC]]$
            restricting $[[γ̂±]]$, $[[ [uσ2]γ̂± = [uσ1]γ̂± ]]$, 
            and $[[Θ(γ̂±) ⊢ [uσ1]γ̂± : scE' ]]$ since $[[Θ ⊢ uσ1 : SC]]$.

            Then by assumption, $[[Θ ⊢ uσ2 ≈ uσ1 : Ξ]]$,
            which in particular means $[[Θ(β̂±) ⊢ [uσ2]β̂± ≈ [uσ1]β̂±]]$,
            that is $[[Θ(β̂±) ⊢ iP' ≈ [uσ1]β̂±]]$.
        \item if $[[β̂±]]$ is negative, the proof is analogous.
    \end{itemize}
\end{proof}