\begin{lemma}[Soundness of Entry Singularity]
    \label{lemma:entry-singularity-soundness}
    Suppose $[[scE singular]]$.
    Then 
    \begin{itemize}
        \item [$+$] for any $[[iP1]]$ and $[[iP2]]$ well-formed in $[[Γ]]$,
            if $[[Γ ⊢ iP1 : scE]]$ and $[[Γ ⊢ iP2 : scE]]$ then $[[Γ ⊢ iP1 ≈ iP2]]$;
        \item [$-$] for any $[[iN1]]$ and $[[iN2]]$ well-formed in $[[Γ]]$,
            if $[[Γ ⊢ iN1 : scE]]$ and $[[Γ ⊢ iN2 : scE]]$ then $[[Γ ⊢ iN1 ≈ iN2]]$.
    \end{itemize}
\end{lemma}
\begin{proof}
    Let us consider how $[[scE singular]]$ is formed.
    \begin{caseof}
        \item \ruleref{\ottdruleSINGNEqLabel}, that is $[[scE]] = [[α̂⁻ :≈ iM]]$.
            Then $[[Γ ⊢ iNi : scE]]$ means $[[Γ ⊢ iNi ≈ iM]]$ 
            (by inversion of \ruleref{\ottdruleSATSCENEqLabel}).
            Then by symmetry and transitivity of equivalence (\cref{lemma:subtyping-reflexivity,corollary:equivalence-transitivity}), 
            $[[Γ ⊢ iN1 ≈ iN2]]$.
        \item \ruleref{\ottdruleSINGPEqLabel}. This case is symmetric to the previous one.

        \item \ruleref{\ottdruleSINGSupVarLabel}, that is 
            $[[scE]] = [[α̂⁺ :≥ ∃nas.β⁺]]$.
            Then $[[Γ ⊢ iPi : scE]]$ means $[[Γ ⊢ iPi ≥ ∃nas.β⁺]]$.
        
            Let us show that it implies $[[Γ ⊢ iPi ≈ ∃nas.β⁺]]$.
            By applying \cref{lemma:shape-of-supertypes} once, 
            we have $[[Γ, nas ⊢ iPi ≥ β⁺]]$.
            By applying it again, we notice that
            $[[Γ, nas ⊢ iPi ≥ β⁺]]$ implies $[[iPi = ∃nas'.β⁺]]$.
            Finally, it is easy to see that $[[Γ ⊢ ∃nas'.β⁺ ≈ ∃nas.β⁺]]$

            This way, by symmetry and transitivity of equivalence (\cref{lemma:subtyping-reflexivity,corollary:equivalence-transitivity}),
            $[[Γ ⊢ iP1 ≈ ∃nas.β⁺]]$ and $[[Γ ⊢ iP2 ≈ ∃nas.β⁺]]$ implies $[[Γ ⊢ iP1 ≈ iP2]]$.

        \item \ruleref{\ottdruleSINGSupShiftLabel},
            that is $[[scE]] = [[α̂⁺ :≥ ∃nbs.↓iN]]$, 
            where $[[iN ≈ nbj]]$.
            Then $[[Γ ⊢ iPi : scE]]$ means $[[Γ ⊢ iPi ≥ ∃nbs.↓iN]]$



            Let us show that it implies $[[Γ ⊢ iPi ≈ ∃nbs.↓iN]]$.

            $
            \begin{aligned}[h]
            [[Γ ⊢ iPi ≥ ∃nbs.↓iN]] &\Rightarrow [[Γ ⊢ nf(iPi) ≥ ∃nbs'.↓nf(iN)]] \text{ where } [[ord {nbs} in iN' = nbs']] 
                                   && \text{by \cref{corollary:nf-pres-subt}} \\
                                   &\Rightarrow [[Γ ⊢ nf(iPi) ≥ ∃nbs'.↓nf(nbj)]]  
                                   && \text{by \cref{lemma:normalization-completeness}}\\
                                   &\Rightarrow [[Γ ⊢ nf(iPi) ≥ ∃nbs'.↓nbn]]  
                                   && \text{by definition of normalization}\\
                                   &\Rightarrow [[Γ ⊢ nf(iPi) ≥ ∃nbj.↓nbj]]  
                                   && \text{since $[[ord {nbs} in nf(iN')]] = [[nbj]]$}\\
                                   &\Rightarrow [[Γ, nbj ⊢ nf(iP) ≥ ↓nbj]] \text { and } [[nbj ∉ fv(nf(iPi))]]
                                   && \text{by \cref{lemma:shape-supertypes-norm}}\\
            \end{aligned}
            $

            By \cref{lemma:shape-supertypes-norm}, 
            the last subtyping means that $[[nf(iPi) = ∃nas.↓iN']]$,
            such that
            \begin{enumerate}
                \item $[[Γ, nbj, nas ⊢ iN']]$
                \item $[[ord {nas} in iN' = nas]]$
                \item for some substitution $[[Γ, nbj ⊢ σ : nas]]$, 
                    $[[ [σ]iN' = nbj ]]$.
            \end{enumerate}
            Since $[[nbj ∉ fv(nf(iPi))]]$,
            the latter means that $[[iN' = na]]$, and then 
            $[[nf(iPi) = ∃na.↓na]]$ for some $[[na]]$.
            Finally, notice that all the types of shape
            $[[∃na.↓na]]$ are equal, and hence,
            $[[nf(iP1) = nf(iP2)]]$, which implies
            $[[Γ ⊢ iP1 ≈ iP2]]$ by \cref{lemma:subt-equiv-algorithmization}.
   \end{caseof}

\end{proof}


\begin{lemma}[Completeness of Entry Singularity]
    \label{lemma:entry-singularity-completeness}
    \begin{itemize}
        \item [$-$] Suppose that for any $[[iN1]]$ and $[[iN2]]$ well-formed in $[[Γ]]$
            such that $[[Γ ⊢ iN1 : scE]]$ and $[[Γ ⊢ iN2 : scE]]$,
            $[[Γ ⊢ iN1 ≈ iN2]]$ holds. Then $[[scE singular]]$ holds.
        \item [$+$] Suppose that for any $[[iP1]]$ and $[[iP2]]$ well-formed in $[[Γ]]$
            such that $[[Γ ⊢ iP1 : scE]]$ and $[[Γ ⊢ iP2 : scE]]$,
            $[[Γ ⊢ iP1 ≈ iP2]]$ holds. Then $[[scE singular]]$ holds.
    \end{itemize}
\end{lemma}
\begin{proof}
    \begin{itemize}
        \item [$-$] Since $[[iNi]]$ are negative, by inversion of
            $[[Γ ⊢ iN1 : scE]]$, $[[scE]]$ has shape $[[α̂⁻ :≈ iM]]$.
            Then $[[scE singular]]$ by \ruleref{\ottdruleSINGNEqLabel}.
        \item [$+$]
            Let us consider the shape of $[[scE]]$:
            \begin{caseof}
                \item $[[scE]] = [[(α̂⁺ :≈ iP)]]$ then $[[scE]]$ is 
                    singular by \ruleref{\ottdruleSINGPEqLabel};
                \item $[[scE]] = [[(α̂⁺ :≥ iP)]]$. Let us consider the shape 
                    of the positive type $[[iP]]$:
                    \begin{caseof}
                        \item $[[iP]] = [[∃nbs.β⁺]]$ (for potentially empty $[[nbs]]$) then
                            $[[scE]]$ is singular by \ruleref{\ottdruleSINGSupVarLabel};
                        \item $[[iP]] = [[∃nbs.↓iN]]$ (for potentially empty $[[nbs]]$).
                            Notice since $[[Γ ⊢ iP ≥ iP]]$, $[[Γ ⊢ iP : scE]]$ holds. 

                            Notice that $[[Γ ⊢ ∃γ⁻.↓γ⁻ ≥ ∃nbs.↓iN]]$ 
                            (by \ruleref{\ottdruleDOneExistsLabel}, 
                            with substitution $[[iN / γ⁻]]$), and thus, 
                            $[[Γ ⊢ ∃γ⁻.↓γ⁻ : scE]]$ by \ruleref{\ottdruleSATSCESupLabel}.

                            This way, by assumption, $[[Γ ⊢ ∃γ⁻.↓γ⁻ ≈ ∃nbs.↓iN]]$, which implies
                            $[[nf(∃nbs.↓iN) = ∃γ⁻.↓γ⁻]]$ (by \cref{lemma:subt-equiv-algorithmization}), 
                            which by definition of normalization means
                            $[[∃nbs'.↓nf(iN) = ∃γ⁻.↓γ⁻]]$, where $[[ord {nbs} in iN' = nbs']]$.
                            This way, $[[nbs']]$ is a variable $[[β⁻]]$, and $[[ nf(iN) = β⁻ ]]$.
                            Notice that $[[β⁻]] \in [[nbs']] \subseteq [[nbs]]$ by \cref{lemma:ord-soundness}.

                            This way, $[[Γ ⊢ iN ≈ β⁻]]$ for $[[β⁻]] \in [[nbs]]$ (by \cref{lemma:subt-equiv-algorithmization}),
                            and hence, $[[scE]]$ is singular by \ruleref{\ottdruleSINGSupShiftLabel}.
                    \end{caseof}
            \end{caseof}
    \end{itemize}
\end{proof}

\begin{lemma}[Soundness of Singularity]
    \label{lemma:singularity-soundness}
    Suppose $[[Θ ⊢ SC]]$, and $[[SC singular]]$. 
    Then for any $[[uσ1]]$ and $[[uσ2]]$ such that $[[Θ ⊢ uσ1 : SC]]$ and $[[Θ ⊢ uσ2 : SC]]$,
    $[[Θ ⊢ uσ1 ≈ uσ2 : dom(SC)]]$.
\end{lemma}
\begin{proof}
    Suppose that $[[Θ ⊢ uσ : SC]]$.
    It means that for any $[[scE]] \in [[SC]]$ restricting $[[α̂±]]$,
    $[[Θ(α̂±) ⊢ [uσ]α̂± : scE]]$ holds.
    $[[SC singular]]$ means $[[scE singular]]$,
    and hence, by \cref{lemma:entry-singularity-completeness},
    $[[ [uσ]α̂± ]]$ is unique up-to equivalence.

    Since the uniqueness holds for every variable from $[[dom(SC)]]$,
    $[[uσ]]$ is defined uniquely on this set.
\end{proof}

\begin{lemma}[Completeness of Singularity]
\end{lemma}
