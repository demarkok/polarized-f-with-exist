\begin{lemma}[Soundness of Entry Singularity]
    \label{lemma:entry-singularity-soundness}
    \begin{itemize}
        \item [$+$] Suppose $[[scE singular with iP]]$ for $[[iP]]$ well-formed in $[[Γ]]$.
            Then $[[ Γ ⊢ iP : scE ]]$
            and for any $[[ Γ ⊢ iP']]$ such that $[[Γ ⊢ iP' : scE]]$, $[[Γ ⊢ iP' ≈ iP]]$;
        \item [$-$] Suppose $[[scE singular with iN]]$ for $[[iN]]$ well-formed in $[[Γ]]$.
            Then $[[ Γ ⊢ iN : scE ]]$
            and for any $[[ Γ ⊢ iN']]$ such that $[[Γ ⊢ iN' : scE]]$, $[[Γ ⊢ iN' ≈ iN]]$.
    \end{itemize}
\end{lemma}
\begin{proof}
    Let us consider how $[[scE singular with iP]]$ or $[[scE singular with iN]]$ is formed.
    \begin{caseof}
        \item \ruleref{\ottdruleSINGNEqLabel}, that is $[[scE]] = [[α̂⁻ :≈ iN0]]$.
            and $[[iN]]$ is $[[nf(iN0)]]$.
            Then $[[Γ ⊢ iN' : scE]]$ means $[[Γ ⊢ iN' ≈ iN0]]$, 
            (by inversion of \ruleref{\ottdruleSATSCENEqLabel}),
            which by transitivity, using \cref{corollary:nf-sound-wrt-subt-equiv},
            means $[[Γ ⊢ iN' ≈ nf(iN0)]]$, 
            as required.
        \item \ruleref{\ottdruleSINGPEqLabel}. This case is symmetric to the previous one.

        \item \ruleref{\ottdruleSINGSupVarLabel}, that is 
            $[[scE]] = [[α̂⁺ :≥ ∃nas.β⁺]]$, and $[[iP = β⁺]]$.

            Since $[[Γ ⊢ β⁺ ≥  ∃nas.β⁺]]$, we have $[[Γ ⊢ β⁺ : scE ]]$, 
            as required.

            Notice that $[[Γ ⊢ iP' : scE]]$ means $[[Γ ⊢ iP' ≥ ∃nas.β⁺]]$.
            Let us show that it implies $[[Γ ⊢ iP' ≈ β⁺]]$.
            By applying \cref{lemma:shape-of-supertypes} once, 
            we have $[[Γ, nas ⊢ iP' ≥ β⁺]]$.
            By applying it again, we notice that
            $[[Γ, nas ⊢ iP' ≥ β⁺]]$ implies $[[iPi = ∃nas'.β⁺]]$.
            Finally, it is easy to see that $[[Γ ⊢ ∃nas'.β⁺ ≈ β⁺]]$

        \item \ruleref{\ottdruleSINGSupShiftLabel},
            that is $[[scE]] = [[α̂⁺ :≥ ∃nbs.↓iN1]]$, 
            where $[[iN1 ≈ nbj]]$, and $[[iP = ∃α⁻.↓α⁻]]$.

            Since $[[Γ ⊢ ∃α⁻.↓α⁻ ≥ ∃nbs.↓iN1]]$ 
            (by \ruleref{\ottdruleDOneExistsLabel}, with substitution $[[iN1 / α⁻]]$),
            we have $[[Γ ⊢ ∃α⁻.↓α⁻ : scE ]]$, as required.

            Notice $[[Γ ⊢ iP' : scE]]$ means $[[Γ ⊢ iP' ≥ ∃nbs.↓iN1]]$.
            Let us show that it implies $[[Γ ⊢ iP' ≈ ∃α⁻.↓α⁻]]$.

            $
            \begin{aligned}[h]
            [[Γ ⊢ iP' ≥ ∃nbs.↓iN1]] &\Rightarrow [[Γ ⊢ nf(iP') ≥ ∃nbs'.↓nf(iN1)]] \text{ where } [[ord {nbs} in iN' = nbs']] 
                                   && \text{by \cref{corollary:nf-pres-subt}} \\
                                   &\Rightarrow [[Γ ⊢ nf(iP') ≥ ∃nbs'.↓nf(nbj)]]  
                                   && \text{by \cref{lemma:normalization-completeness}}\\
                                   &\Rightarrow [[Γ ⊢ nf(iP') ≥ ∃nbs'.↓nbn]]  
                                   && \text{by definition of normalization}\\
                                   &\Rightarrow [[Γ ⊢ nf(iP') ≥ ∃nbj.↓nbj]]  
                                   && \text{since $[[ord {nbs} in nf(iN1)]] = [[nbj]]$}\\
                                   &\Rightarrow [[Γ, nbj ⊢ nf(iP') ≥ ↓nbj]] \text { and } [[nbj ∉ fv(nf(iP'))]]
                                   && \text{by \cref{lemma:shape-supertypes-norm}}\\
            \end{aligned}
            $

            By \cref{lemma:shape-supertypes-norm}, 
            the last subtyping means that $[[nf(iP') = ∃nas.↓iN']]$,
            such that
            \begin{enumerate}
                \item $[[Γ, nbj, nas ⊢ iN']]$
                \item $[[ord {nas} in iN' = nas]]$
                \item for some substitution $[[Γ, nbj ⊢ σ : nas]]$, 
                    $[[ [σ]iN' = nbj ]]$.
            \end{enumerate}
            Since $[[nbj ∉ fv(nf(iP'))]]$,
            the latter means that $[[iN' = na]]$, and then 
            $[[nf(iP') = ∃na.↓na]]$ for some $[[na]]$.
            Finally, notice that all the types of shape
            $[[∃na.↓na]]$ are equal.
   \end{caseof}

\end{proof}



\begin{lemma}[Completeness of Entry Singularity]
    \label{lemma:entry-singularity-completeness}
    \hfill
    \begin{itemize}
        \item [$-$] Suppose that there exists $[[iN]]$ well-formed in $[[Γ]]$ 
            such that for any $[[iN']]$ well-formed in $[[Γ]]$,
            $[[Γ ⊢ iN' : scE]]$ implies $[[Γ ⊢ iN' ≈ iN]]$. 
            Then $[[scE singular with nf(iN)]]$.
        \item [$+$] Suppose that there exists $[[iP]]$ well-formed in $[[Γ]]$ 
            such that for any $[[iP']]$ well-formed in $[[Γ]]$,
            $[[Γ ⊢ iP' : scE]]$ implies $[[Γ ⊢ iP' ≈ iP]]$. 
            Then $[[scE singular with nf(iP)]]$ .
    \end{itemize}
\end{lemma}
\begin{proof}
    \hfill
    \begin{itemize}
        \item [$-$] 
            By \cref{lemma:constraint-sat},
            there exists $[[Γ ⊢ iN' : scE]]$.
            Since $[[iN']]$ is negative, by inversion of
            $[[Γ ⊢ iN' : scE]]$, $[[scE]]$ has shape $[[α̂⁻ :≈ iM]]$, 
            where $[[Γ ⊢ iN' ≈ iM]]$, and transitively, $[[Γ ⊢ iN ≈ iM]]$.
            Then $[[nf(iM) = nf(iN)]]$, 
            and $[[scE singular with nf(iM)]]$ (by \ruleref{\ottdruleSINGNEqLabel})
            is rewritten as $[[scE singular with nf(iN)]]$.
        \item [$+$]
            By \cref{lemma:constraint-sat}, there exists $[[Γ ⊢ iP' : scE]]$, 
            then by assumption, $[[Γ ⊢ iP' ≈ iP]]$,
            which by \cref{lemma:entry-sat-equiv} implies $[[Γ ⊢ iP : scE]]$.

            Let us consider the shape of $[[scE]]$:
            \begin{caseof}
                \item $[[scE]] = [[(α̂⁺ :≈ iQ)]]$ then 
                    inversion of $[[Γ ⊢ iP : scE]]$
                    implies $[[Γ ⊢ iP ≈ iQ]]$, and hence, $[[nf(iP) = nf(iQ)]]$
                    (by \cref{lemma:subt-equiv-algorithmization}).
                    Then $[[scE singular with nf(iQ)]]$, 
                    which holds by \ruleref{\ottdruleSINGPEqLabel}, 
                    is rewritten as $[[scE singular with nf(iP)]]$.

                \item $[[scE]] = [[(α̂⁺ :≥ iQ)]]$.
                    Then the inversion of $[[Γ ⊢ iP : scE]]$ 
                    implies $[[Γ ⊢ iP ≥ iQ]]$.
                    Let us consider the shape of $[[iQ]]$:
                    \begin{caseof}
                        \item $[[iQ]] = [[∃nbs.β⁺]]$ (for potentially empty $[[nbs]]$).
                            Then $[[Γ ⊢ iP ≥ ∃nbs.β⁺]]$ 
                            implies $[[Γ ⊢ iP ≈ β⁺]]$ by 
                            \cref{lemma:shape-of-supertypes}, 
                            as was noted in the proof of 
                            \cref{lemma:entry-singularity-soundness},
                            and hence, $[[nf(iP) = β⁺]]$.

                            Then $[[scE singular with β⁺]]$, which holds by
                            \ruleref{\ottdruleSINGSupVarLabel},
                            can be rewritten as $[[scE singular with nf(iP)]]$.

                        \item $[[iQ]] = [[∃nbs.↓iN]]$ (for potentially empty $[[nbs]]$).
                            Notice that $[[Γ ⊢ ∃γ⁻.↓γ⁻ ≥ ∃nbs.↓iN]]$ 
                            (by \ruleref{\ottdruleDOneExistsLabel}, 
                            with substitution $[[iN / γ⁻]]$), and thus, 
                            $[[Γ ⊢ ∃γ⁻.↓γ⁻ : scE]]$ by \ruleref{\ottdruleSATSCESupLabel}.
                            
                            Then by assumption, $[[Γ ⊢ ∃γ⁻.↓γ⁻ ≈ iP]]$,
                            that is $[[nf(iP) = ∃γ⁻.↓γ⁻]]$.
                            To apply \ruleref{\ottdruleSINGSupShiftLabel}
                            to infer $[[(α̂⁺ :≥ ∃nbs.↓iN) singular with ∃γ⁻.↓γ⁻]]$,
                            it is left to show that $[[iN ≈ nbi]]$ for some $i$.

                            Since $[[Γ ⊢ iQ : scE]]$, by assumption,
                            $[[Γ ⊢ iQ ≈ iP]]$, and by transitivity, 
                            $[[Γ ⊢ iQ ≈ ∃γ⁻.↓γ⁻]]$.
                            It implies
                            $[[nf(∃nbs.↓iN) = ∃γ⁻.↓γ⁻]]$ (by \cref{lemma:subt-equiv-algorithmization}), 
                            which by definition of normalization means
                            $[[∃nbs'.↓nf(iN) = ∃γ⁻.↓γ⁻]]$, where $[[ord {nbs} in iN' = nbs']]$.
                            This way, $[[nbs']]$ is a variable $[[β⁻]]$, and $[[ nf(iN) = β⁻ ]]$.
                            Notice that $[[β⁻]] \in [[nbs']] \subseteq [[nbs]]$ by \cref{lemma:ord-soundness}.
                            This way, $[[iN ≈ β⁻]]$ for $[[β⁻]] \in [[nbs]]$ (by \cref{lemma:subt-equiv-algorithmization}),
                    \end{caseof}
            \end{caseof}
    \end{itemize}
\end{proof}

\begin{lemma}[Soundness of Singularity]
    \label{lemma:singularity-soundness}
    Suppose $[[Θ ⊢ SC]]$, and $[[SC singular with uσ]]$. 
    Then $[[(Θ  ⊢  SC) ⊢ uσ]]$ and for any 
    $[[uσ']]$ such that $(Θ  ⊢ SC) ⊢ uσ'$,
    $[[Θ|dom(SC) ⊢ uσ' ≈ uσ]]$.
\end{lemma}
\begin{proof}
    Suppose that $(Θ  ⊢ SC) ⊢ uσ'$.
    It means that for every $[[scE]] \in [[SC]]$ restricting $[[α̂±]]$,
    $[[Θ(α̂±) ⊢ [uσ']α̂± : scE]]$ holds.
    $[[SC singular with uσ]]$ means $[[scE singular with [uσ]α̂± ]]$,
    and hence, by \cref{lemma:entry-singularity-completeness},
    $[[ Θ(α̂±) ⊢ [uσ']α̂± ≈ [uσ]α̂±  ]]$ holds.

    Since the uniqueness holds for every variable from $[[dom(SC)]]$,
    $[[uσ]]$ is equivalent to $[[uσ']]$ on this set.
\end{proof}

\begin{lemma}[Completeness of Singularity]
    \label{lemma:singularity-completeness}
    For a given $[[Θ ⊢ SC]]$
    and $[[varset ⊆ dom(Θ)]]$,
    suppose that all the substitutions satisfying $[[SC]]$ are equivalent,
    in other words, suppose that there exists $[[Θ ⊢ uσ1]]$ such that
    for any $[[Θ ⊢ uσ]]$, $[[(Θ  ⊢  SC) ⊢ uσ]]$ implies $[[Θ|varset ⊢ uσ ≈ uσ1]]$.
    Then 
    \begin{itemize}
        \item $[[SC|varset singular with uσ0]]$ for some $[[uσ0]]$, and
        \item $[[varset ⊆ dom(SC)]]$.
    \end{itemize} 
\end{lemma}
\begin{proof}
    First, let us suppose that $[[varset ⊆ dom(SC)]]$ does not hold.
    It means that there exists $[[ α̂±[Γ] ∊ Θ]]$ and $[[α̂± ∉ dom(SC)]]$

    By \cref{lemma:constraint-sat}, let us take an arbitrary 
    $[[uσ']]$ satisfying $[[SC]]$ (that is $[[(Θ  ⊢  SC) ⊢ uσ']]$).
    Let us extend $[[uσ']]$ to $[[uσ'']]$ so that $[[Θ ⊢ uσ'']]$.

    Although $[[uσ']]$ might not be defined on 
    
    

    such that for any $[[Θ ⊢ uσ]]$, $[[(Θ  ⊢  SC) ⊢ uσ]]$ implies 
    $[[ Γ ⊢ [uσ]α̂± ≈ [uσ1]α̂± ]]$.
    Let us take 


    for any $[[Θ ⊢ uσ]]$, $[[(Θ  ⊢  SC) ⊢ uσ]]$ implies $[[Θ|varset ⊢ uσ ≈ uσ1]]$.


\end{proof}


        \item [$-$] Suppose that there exists $[[iN]]$ well-formed in $[[Γ]]$ 
            such that for any $[[iN']]$ well-formed in $[[Γ]]$,
            $[[Γ ⊢ iN' : scE]]$ implies $[[Γ ⊢ iN' ≈ iN]]$. 
            Then $[[scE singular with nf(iN)]]$.
        \item [$+$] Suppose that there exists $[[iP]]$ well-formed in $[[Γ]]$ 
            such that for any $[[iP']]$ well-formed in $[[Γ]]$,
            $[[Γ ⊢ iP' : scE]]$ implies $[[Γ ⊢ iP' ≈ iP]]$. 
            Then $[[scE singular with nf(iP)]]$ .
