\begin{lemma}
  \label{lemma:equiv-fv}
  Set of free variables is invariant under equivalence.
  \begin{itemize}
  \item[$-$] If $[[iN ≈ iM]]$ then $[[fv iN]] \equiv [[fv iM]]$ (as sets)
  \item[$+$] If $[[iP ≈ iQ]]$ then $[[fv iP]] \equiv [[fv iQ]]$ (as sets)
  \end{itemize}
\end{lemma}
\begin{proof}
  Straightforward mutual induction on $[[iN ≈ iM]]$ and $[[iP ≈ iQ]]$
\end{proof}


\begin{lemma}[Distributivity of renaming over variable ordering]
  \label{lemma:distr-mu-ord}
  Suppose that $\mu$ is a bijection between two sets of variables
  $\mu : A \leftrightarrow B$, and $B$ is disjoint with $[[varset]]$.
  \begin{itemize}
  \item[$-$]
    If $B$ is disjoint with $[[fv iN]]$ then
    $[[ [mu] (ord varset in iN) ]] = [[ord ([mu] varset) in [mu] iN ]]$
  \item[$+$]
    If $B$ is disjoint with $[[fv iP]]$ then
    $[[ [mu] (ord varset in iP) ]] = [[ord ([mu] varset) in [mu] iP ]]$
  \end{itemize}
\end{lemma}

\begin{proof}
  Mutual induction on $[[iN]]$ and $[[iP]]$.
  \begin{caseof}
  \case[foo]{$[[iN]]$ = $[[na]]$}{ 
    let us consider four cases:
    \begin{enumerate}
    \item $[[na]] \in A$ and $[[na]] \in [[varset]]$. Then
      $
      \begin{aligned}[t] [[ [mu] (ord varset in iN) ]] &= [[ [mu] (ord varset in na)]] \\
                                                             &= [[ [mu] na ]]
                                                             && \text{by \ruleref{\ottdruleOPVarInLabel}}\\
                                                             &= [[nb]]
                                                             && \text{for some $[[nb]] \in B$ (notice that $[[nb]] \in [[ [mu]varset ]]$)} \\
                                                             &= [[ ord [mu]varset in nb ]]
                                                             && \text{by \ruleref{\ottdruleOPVarInLabel},
                                                                because $[[nb]] \in [[ [mu]varset ]]$} \\
                                                             &= [[ord [mu] varset in [mu] na ]]
       \end{aligned}
       $
     \item $[[na]] \notin A$ and $[[na]] \notin [[varset]]$.
       Notice that $[[na]] \notin B$, because $B$ is disjoint with $[[fv iN]]$.
       Then $[[ [mu] (ord varset in iN) ]] = [[ [mu] (ord varset in na)]] = [[·]]$ by
       \ruleref{\ottdruleOPVarNInLabel}.
       On the other hand, $[[ ord [mu] varset in [mu] na = ord [mu] varset
       in na ]] = [[·]]$ The latter equality is from
       \ruleref{\ottdruleOPVarNInLabel}, because $[[na]] \notin B \cup
       [[varset]] \supseteq [[ [mu] varset ]]$.
     \item $[[na]] \in A$ but $[[na]] \notin [[varset]]$. Then
       $[[ [mu] (ord varset in iN) ]] = [[ [mu] (ord varset in na)]] = [[·]]$
       by \ruleref{\ottdruleOPVarNInLabel}.
       To prove that $[[ ord [mu] varset in [mu] na ]] = [[·]]$, we apply
       \ruleref{\ottdruleOPVarNInLabel}. Let us show that $ [[ [mu] na ]] \notin
       [[ [mu] varset ]] $. If there is an element $x \in [[varset]]$ such that
       $[[mu]] x = [[mu]] [[na]]$, then $x = [[na]]$ by bijectivity of
       $[[mu]]$, which contradicts with $[[na]] \notin [[varset]]$. On the
       other hand, $[[ [mu] na ]] \in B$, and hence, $ [[ [mu] na ]] \notin [[
       varset ]]$.
     \item $[[na]] \notin A$ but $[[na]] \in [[varset]]$.
       $[[ ord [mu] varset in [mu] na ]] = [[ ord [mu] varset in na ]] = [[na]]$.
       The latter is by \ruleref{\ottdruleOPVarNInLabel}, because
       $[[na]] = [[ [mu] na ]] \in [[ [mu] varset ]]$ since $[[na]] \in [[varset]]$.
       On the other hand, $[[ [mu] (ord varset in iN) ]] = [[ [mu] (ord varset in na)]] = [[ [mu] na ]] = [[na]]$.
    \end{enumerate}
  }
  \case{$[[iN]] = [[↑iP]]$}{
    $\begin{aligned}[t]
       [[ [mu] (ord varset in iN) ]] &= [[ [mu] (ord varset in ↑iP) ]] \\
                                     &= [[ [mu] (ord varset in iP) ]]
                                     && \text{by \ruleref{\ottdruleOShiftULabel}}\\
                                     &= [[ ord [mu]varset in [mu]iP ]]
                                     && \text{by the induction hypothesis}\\
                                     &= [[ ord [mu]varset in  ↑[mu]iP ]]
                                     && \text{by \ruleref{\ottdruleOShiftULabel}}\\
                                     &= [[ ord [mu]varset in  [mu]↑iP ]]
                                     && \text{by the definition of substitution}\\
                                     &= [[ ord [mu]varset in  [mu]iN ]]
            \end{aligned}$
          }
   \case[biz]{$[[iN]] = [[iP → iM]]$}{
     $\begin{aligned}[t]
        [[ [mu] (ord varset in iN) ]] &= [[ [mu] (ord varset in iP → iM) ]] \\
                                      &= [[ [mu] (ordVars1, (ordVars2 \ {ordVars1})) ]]
                                      && \text{where } [[ord varset in iP = ordVars1]] \text{ and } [[ord varset in iM = ordVars2]] \\
                                      &= [[ [mu] ordVars1, [mu](ordVars2 \ {ordVars1}) ]] \\
                                      &= [[ [mu] ordVars1, ([mu]ordVars2 \ [mu]{ordVars1}) ]]
                                      && \text{by induction on $[[ordVars2]]$;
                                         the inductive step is similar to \cref{foo} }\\
                                      &= [[ [mu] ordVars1, ([mu]ordVars2 \ {[mu]ordVars1}) ]]
      \end{aligned}$ \\
    On the other hand,\\
    $\begin{aligned}[t]
       [[  (ord [mu] varset in [mu]iN) ]] &= [[ (ord [mu] varset in [mu]iP → [mu]iM) ]] \\
                                     &= [[ (ordVarsb1, (ordVarsb2 \ {ordVarsb1})) ]]
                                     && \text{where } [[ord [mu] varset in [mu] iP = ordVarsb1]] \text{ and } [[ord [mu] varset in [mu] iM = ordVarsb2]] \\
                                          & && \text{then by the induction
                                               hypothesis,
                                               $[[ordVarsb1]] = [[ [mu] ordVars1 ]]$,
                                               $[[ordVarsb2]] = [[ [mu] ordVars2 ]]$,
                                               }\\
                                     &= [[ [mu] ordVars1, ([mu]ordVars2 \ {[mu]ordVars1}) ]]
     \end{aligned}$
   }
   \case{$[[iN]] = [[∀ pas.iM]]$}{
     $
     \begin{aligned}[t]
          [[ [mu] (ord varset in iN) ]] &= [[ [mu] ord varset in ∀pas.iM]] \\
                                        &= [[ [mu] ord varset in iM]] \\
                                        &= [[ ord [mu] varset in [mu] iM]]
                                        && \text {by the induction hypothesis}\\
     \end{aligned}\\
     $
     On the other hand,\\
     $
     \begin{aligned}[t]
       [[ (ord [mu] varset in [mu] iN) ]] &= [[ ord [mu] varset in [mu] ∀pas.iM ]] \\
                                          &= [[ ord [mu] varset in ∀pas.[mu]iM ]] \\
                                          &= [[ ord [mu] varset in [mu] iM ]] \\
     \end{aligned}
     $
    } 
  \end{caseof}
\end{proof}


\begin{lemma}[Commutativity of normalization and renaming]
  \label{lemma:norm-subst-commute} Normalization of a term commutes with renaming.

  Suppose that $\mu$ is a bijection between two sets of variables
  $\mu : A \leftrightarrow B$. Then
  \begin{itemize}
    \item[$-$] $[[nf([mu]iN)]] = [[ [mu] nf(iN) ]]$
    \item[$+$] $[[nf([mu]iP)]] = [[ [mu] nf(iP) ]]$
  \end{itemize}
  Here equality means alpha-equivalence.
\end{lemma}

\begin{proof}
  Mutual induction on $[[iN]]$ and $[[iP]]$.
  \todo[inline]{Write a little bit about the exists/forall case}
\end{proof}