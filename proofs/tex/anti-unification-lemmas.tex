\begin{observation}[Anti-unification algorithm is deterministic]
    \label{obs:anti-unification-deterministic}
    \hfill\\
\begin{itemize}
    \item [$+$] If $[[Γ ⊨ iP1 ≈au iP2 ⫤ (Ξ, uQ, aus1, aus2)]]$ and 
    $[[Γ ⊨ iP1 ≈au iP2 ⫤ (Ξ', uQ', aus'1, aus'2)]]$,
    then $[[Ξ]] = [[Ξ']]$, $[[uQ]] = [[uQ']]$, $[[aus1]] = [[aus'1]]$, and $[[aus2]] = [[aus'2]]$.
    \item [$-$] If $[[Γ ⊨ iN1 ≈au iN2 ⫤ (Ξ, uM, aus1, aus2)]]$ and
    $[[Γ ⊨ iN1 ≈au iN2 ⫤ (Ξ', uM', aus'1, aus'2)]]$,
    then $[[Ξ]] = [[Ξ']]$, $[[uM]] = [[uM']]$, $[[aus1]] = [[aus'1]]$, and $[[aus2]] = [[aus'2]]$.
\end{itemize}
\end{observation}
\begin{proof}
    By trivial induction on $[[Γ ⊨ iP1 ≈au iP2 ⫤ (Ξ, uQ, aus1, aus2)]]$
    and mutually on $[[Γ ⊨ iN1 ≈au iN2 ⫤ (Ξ, uM, aus1, aus2)]]$.
\end{proof}

\begin{observation} \label{obs:names-defined-by-mapping}
    Names of the anti-unification variables are uniquely defined by
    the types they are mapped to by the resulting substitutions. 

    \begin{itemize}
        \item [$+$]  Assuming $[[iP1]]$ and $[[iP2]]$ are normalized,
        if $[[Γ ⊨ iP1 ≈au iP2 ⫤ (Ξ, uQ, aus1, aus2)]]$
        then for any $[[β̂⁻]] \in [[Ξ]]$,
        $[[β̂⁻]] = [[â⁻_{[aus1]β̂⁻, [aus2]β̂⁻}]]$
        \item [$-$]  Assuming $[[iN1]]$ and $[[iN2]]$ are normalized,
        if $[[Γ ⊨ iN1 ≈au iN2 ⫤ (Ξ, uM, aus1, aus2)]]$
        then for any $[[β̂⁻]] \in [[Ξ]]$,
        $[[β̂⁻]] = [[â⁻_{[aus1]β̂⁻, [aus2]β̂⁻}]]$
    \end{itemize}
\end{observation}
\begin{proof}
    By simple induction on $[[Γ ⊨ iP1 ≈au iP2 ⫤ (Ξ, uQ, aus1, aus2)]]$
    and mutually on $[[Γ ⊨ iN1 ≈au iN2 ⫤ (Ξ, uM, aus1, aus2)]]$.
    Let us consider tha last rule applied to infer this judgment.
    \begin{caseof}
        \item \ruleref{\ottdruleAUPVarLabel} or \ruleref{\ottdruleAUNVarLabel},
        then $[[Ξ]] = [[·]]$, and the property holds vacuously.

        \item \ruleref{\ottdruleAUAULabel}
        Then  $[[Ξ]] = [[â⁻_{iN1, iN2}]]$,
        $[[aus1]] = [[â⁻_{iN1, iN2} :≈ iN1]]$, and $[[aus2]] = [[â⁻_{iN1, iN2} :≈ iN2]]$.
        So the property holds trivially.

        \item \ruleref{ottdruleAUArrowLabel}
        In this case, $[[Ξ]] = [[Ξ' ∪ Ξ'']]$, $[[aus1]] = [[aus'1 ∪ aus''1]]$, and 
        $[[aus2]] = [[aus'2 ∪ aus''2]]$,
        where the property holds for ($[[Ξ']]$, $[[aus'1]]$, $[[aus'2]]$) and 
        ($[[Ξ'']]$, $[[aus''1]]$, $[[aus''2]]$) by the induction hypothesis.
        Then since the union of solutions does not change the types the variables are mapped to,
        the required property holds for $[[Ξ]]$, $[[aus1]]$, and $[[aus2]]$.

        \item For the other rules, the resulting $[[Ξ]]$ is taken from the recursive call
        and the required property holds immediately by the induction hypothesis.
    \end{caseof}
\end{proof}

\begin{lemma}[Soundness of Anti-Unification] \label{lemma:anti-unification-soundness}
    \hfill
    \begin{itemize}
        \item [$+$]  Assuming $[[iP1]]$ and $[[iP2]]$ are normalized,
        if $[[Γ ⊨ iP1 ≈au iP2 ⫤ (Ξ, uQ, aus1, aus2)]]$
        then 
        \begin{enumerate}
            \item $[[Γ ; Ξ ⊢ uQ]]$,
            \item $[[Γ ; · ⊢ ausi : Ξ]]$ for $i \in \{1,2\}$
            are anti-unification solutions, and
            \item $[[ [ausi] uQ = iPi ]]$ for $i \in \{1,2\}$.
        \end{enumerate}

        \item [$-$] Assuming $[[iN1]]$ and $[[iN2]]$ are normalized,
        if $[[Γ ⊨ iN1 ≈au iN2 ⫤ (Ξ, uM, aus1, aus2)]]$
        then
        \begin{enumerate}
            \item $[[Γ ; Ξ ⊢ uM]]$,
            \item $[[Γ ; · ⊢ ausi : Ξ]]$ for $i \in \{1,2\}$
            are anti-unification solutions, and
            \item $[[ [ausi] uM = iNi ]]$ for $i \in \{1,2\}$.
        \end{enumerate}
    \end{itemize}
\end{lemma}
\begin{proof}
    We prove it by induction on 
    $[[Γ ⊨ iN1 ≈au iN2 ⫤ (Ξ, uM, aus1, aus2)]]$
    and mutually, $[[Γ ⊨ iP1 ≈au iP2 ⫤ (Ξ, uQ, aus1, aus2)]]$.
    Let us consider the last rule applied to infer this judgement.
    \begin{caseof}
        \item \ruleref{\ottdruleAUNVarLabel}, then $[[iN1]] = [[α⁻]] = [[iN2]]$,
              $[[Ξ]] = [[·]]$, $[[uM]] = [[α⁻]]$, and $[[aus1]] = [[aus2]] = [[·]]$.
            \begin{enumerate}
                \item $[[Γ ; · ⊢ α⁻]]$ follows from the assumption $[[Γ ⊢ α⁻]]$,
                \item $[[Γ ; · ⊢ · : ·]]$ holds trivially, and
                \item $[[ [·] α⁻ = α⁻ ]]$ holds trivially.
            \end{enumerate}
        \item \label{case:anti-unification-soundness:shiftu}
         \ruleref{\ottdruleAUShiftULabel}, then $[[iN1]] = [[↑iP1]]$,
                $[[iN2]] = [[↑iP2]]$, and the algorithm makes the recursive call:
                $[[Γ ⊨ iP1 ≈au iP2 ⫤ (Ξ, uQ, aus1, aus2)]]$, 
                returning $[[(Ξ, ↑uQ, aus1, aus2)]]$ as the result.

                Since $[[iN1]] = [[↑iP1]]$ and $[[iN2]] = [[↑iP2]]$ are normalized, 
                so are $[[iP1]]$ and $[[iP2]]$, and thus, the induction hypothesis 
                is applicable to $[[Γ ⊨ iP1 ≈au iP2 ⫤ (Ξ, uQ, aus1, aus2)]]$:
                \begin{enumerate}
                    \item $[[Γ ; Ξ ⊢ uQ]]$, and hence, $[[Γ ; Ξ ⊢ ↑uQ]]$,
                    \item $[[Γ ; · ⊢ ausi : Ξ]]$ for $i \in \{1,2\}$, and
                    \item $[[ [ausi] uQ = iPi ]]$ for $i \in \{1,2\}$, and then by 
                    the definition of the substitution, $[[ [ausi] ↑uQ = ↑iPi ]]$ for $i \in \{1,2\}$.
                \end{enumerate}

        \item \ruleref{\ottdruleAUArrowLabel}, then $[[iN1]] = [[iP1 → iN'1]]$,
                $[[iN2]] = [[iP2 → iN'2]]$, and the algorithm makes two recursive calls:
                $[[Γ ⊨ iP1 ≈au iP2 ⫤ (Ξ, uQ, aus1, aus2)]]$ and
                $[[Γ ⊨ iN'1 ≈au iN'2 ⫤ (Ξ', uM, aus'1, aus'2)]]$ and
                and returns $[[(Ξ ∪ Ξ', uQ → uM, aus1 ∪ aus'1, aus2 ∪ aus'2)]]$ as the result.
                
                Notice that the induction hypothesis is applicable to 
                $[[Γ ⊨ iP1 ≈au iP2 ⫤ (Ξ, uQ, aus1, aus2)]]$:
                $[[iP1]]$ and $[[iP2]]$ are normalized, since $[[iN1]] = [[iP1 → iN'1]]$
                and $[[iN2]] = [[iP2 → iN'2]]$ are normalized.
                Similarly, the induction hypothesis is applicable to
                $[[Γ ⊨ iN'1 ≈au iN'2 ⫤ (Ξ', uM, aus'1, aus'2)]]$.

                This way, by the induction hypothesis:
                \begin{enumerate}
                    \item $[[Γ ; Ξ ⊢ uQ]]$ and $[[Γ ; Ξ' ⊢ uM]]$. 
                    Then by weakening (\cref{lemma:wf-weakening-algo}), 
                    $[[Γ ; Ξ ∪ Ξ' ⊢ uQ]]$ and 
                    $[[Γ ; Ξ ∪ Ξ' ⊢ uM]]$, which implies $[[Γ ; Ξ ∪ Ξ' ⊢ uQ → uM]]$;

                    \item $[[Γ ; · ⊢ ausi : Ξ]]$ and $[[Γ ; · ⊢ aus'i : Ξ']]$
                        Then $[[Γ ; · ⊢ ausi ∪ aus'i : Ξ ∪ Ξ']]$
                        are well-defined anti-unification solution.
                        Let us take an arbitrary $[[β̂⁻]] \in [[Ξ ∪ Ξ']]$.
                        If $[[β̂⁻]] \in [[Ξ]]$.
                        then $[[Γ ; · ⊢ ausi : Ξ]]$ implies that $[[ausi]]$, and hence,
                        $[[ausi ∪ aus'i]]$ contains an entry well-formed in $[[Γ]]$.
                        If $[[β̂⁻]] \in [[Ξ']]$, the reasoning is symmetric.

                        $[[ausi ∪ aus'i]]$ is a well-defined anti-unification solution:
                        any anti-unification variable occurs 
                        uniquely $[[ausi ∪ aus'i]]$, since by \cref{obs:names-defined-by-mapping},
                        the name of the variable is in one-to-one correspondence with 
                        the pair of types it is mapped to
                        by $[[aus1]]$ and $[[aus2]]$, 
                        an is in one-to-one correspondence with the pair of types it is mapped to
                        by $[[aus'1]]$ and $[[aus'2]]$ 
                        i.e.  if $[[β̂⁻]] \in [[Ξ ∩ Ξ']]$ then $[[ [aus1]β̂⁻ = [aus'1]β̂⁻ ]]$,
                        and $[[ [aus2]β̂⁻ = [aus'2]β̂⁻ ]]$.

                    \item $[[ [ausi] uQ = iPi ]]$ and $[[ [aus'i] uM = iN'i ]]$.
                    Since $[[ausi ∪ aus'i]]$ restricted to $[[Ξ]]$ is $[[ausi]]$,
                    and $[[ausi ∪ aus'i]]$ restricted to $[[Ξ']]$ is $[[aus'i]]$,
                    we have $[[ [ausi ∪ aus'i] uQ = iPi ]]$ and 
                    $[[ [ausi ∪ aus'i] uM = iN'i ]]$, and thus, 
                    $[[ [ausi ∪ aus'i] uQ → uM = iP1 → iN'1 ]]$
                \end{enumerate}

        \item \ruleref{\ottdruleAUForallLabel}, then $[[iN1]] = [[∀pas.iN'1]]$,
                $[[iN2]] = [[∀pas.iN'2]]$, and the algorithm makes a recursive call:
                $[[Γ ⊨ iN'1 ≈au iN'2 ⫤ (Ξ, uM, aus1, aus2)]]$ and
                returns $[[(Ξ, ∀pas.uM, aus1, aus2)]]$ as the result.

                Similarly to \cref{case:anti-unification-soundness:shiftu}, 
                we apply the induction hypothesis to
                $[[Γ ⊨ iN'1 ≈au iN'2 ⫤ (Ξ, uM, aus1, aus2)]]$ to obtain:
                \begin{enumerate}
                    \item $[[Γ; Ξ ⊢ uM]]$, and hence, $[[Γ ; Ξ ⊢ ∀pas.uM]]$;
                    \item $[[Γ; · ⊢ ausi : Ξ]]$ for $i \in \{1,2\}$, and
                    \item $[[ [ausi] uM = iN'i ]]$ for $i \in \{1,2\}$,
                        and then by the definition of the substitution,
                        $[[ [ausi] ∀pas.uM = ∀pas.iN'i ]]$ for $i \in \{1,2\}$. 
                \end{enumerate}

        \item \ruleref{\ottdruleAUAULabel}, which applies 
        when other rules do not, and $[[G ⊢ iNi]]$,
        returning as the result $[[(Ξ, uM, aus1, aus2)]] = $
        $[[({â⁻_{iN1, iN2}}, â⁻_{iN1, iN2}, (â⁻_{iN1, iN2} :≈ iN1) ,  (â⁻_{iN1, iN2} :≈ iN2))]]$.

        \begin{enumerate}
            \item $[[Γ ; Ξ ⊢ uM]]$ is rewritten as $[[Γ ; { â⁻_{iN1, iN2} } ⊢ â⁻_{iN1, iN2}]]$,
                which holds trivially;
            \item $[[Γ ; · ⊢ ausi : Ξ]]$ is rewritten as $[[Γ ; · ⊢ (â⁻_{iN1, iN2} :≈ iNi) : {â⁻_{iN1, iN2}}]]$,
                which holds since $[[Γ ⊢ iNi]]$ by the premise of the rule;
            \item $[[ [ausi] uM = iNi ]]$ is rewritten as $[[ [â⁻_{iN1, iN2} :≈ iNi] â⁻_{iN1, iN2} = iNi ]]$,
                which holds trivially by the definition of substitution.
        \end{enumerate}

        \item Positive cases are proved symmetrically.
    \end{caseof}
\end{proof}




% \renewcommand{\ottdruleAUPVarName}[0]{(Var$^{+[[≈au]]}$)}
% \renewcommand{\ottdruleAUShiftDName}[0]{($\downarrow^{[[≈au]]}$)}
% \renewcommand{\ottdruleAUExistsName}[0]{($\exists^{[[≈au]]}$)}

% G ⊨ iN1 ≈au iN2 ⫤ ( Ξ , uM , aus1 , aus2 )
% \renewcommand{\ottdruleAUNVarName}[0]{(Var$^{-[[≈au]]}$)}
% \renewcommand{\ottdruleAUShiftUName}[0]{($\uparrow^{[[≈au]]}$)}
% \renewcommand{\ottdruleAUForallName}[0]{($\forall^{[[≈au]]}$)}
% \renewcommand{\ottdruleAUArrowName}[0]{($\rightarrow^{[[≈au]]}$)}
% \renewcommand{\ottdruleAUAUName}[0]{(AU$^{-}$)}


\begin{lemma}[Completeness of Anti-Unification] \label{lemma:anti-unification-completeness}
    \hfill
    \begin{itemize}
        \item [$+$] 
            Assume that $[[iP1]]$ and $[[iP2]]$ are normalized, and
            there exists $[[(Ξ', uQ', aus'1, aus'2)]]$ such that
            \begin{enumerate}
                \item $[[Γ ; Ξ' ⊢ uQ']]$,
                \item $[[Γ ; · ⊢ aus'i : Ξ']]$ for $i \in \{1,2\}$ 
                are anti-unification solutions, and
                \item $[[ [aus'i] uQ' = iPi ]]$ for $i \in \{1,2\}$.
            \end{enumerate}

            Then the anti-unification algorithm terminates, that is there exists
            $[[(Ξ, uQ, aus1, aus2)]]$ such that $[[Γ ⊨ iP1 ≈au iP2 ⫤ (Ξ, uQ, aus1, aus2)]]$

        \item [$-$] 
            Assume that $[[iN1]]$ and $[[iN2]]$ are normalized, and
            there exists $[[(Ξ', uM', aus'1, aus'2)]]$ such that
            \begin{enumerate}
                \item $[[Γ ; Ξ' ⊢ uM']]$,
                \item $[[Γ ; · ⊢ aus'i : Ξ']]$ for $i \in \{1,2\}$,
                are anti-unification solutions, and
                \item $[[ [aus'i] uM' = iNi ]]$ for $i \in \{1,2\}$.
            \end{enumerate}

            Then the anti-unification algorithm succeeds, that is 
            there exists $[[(Ξ, uM, aus1, aus2)]]$ such that
            $[[Γ ⊨ iN1 ≈au iN2 ⫤ (Ξ, uM, aus1, aus2)]]$.
    \end{itemize}
\end{lemma}
\begin{proof}
    We prove it by the induction on $[[uM']]$ and mutually on $[[uQ']]$.
    \begin{caseof}
        \item $[[uM']] = [[â⁻]]$ 
            Then since $[[Γ ; · ⊢ aus'i : Ξ']]$,
            $[[Γ ⊢ [aus'i] uM']] = [[ iNi ]]$. 
            This way, \ruleref{\ottdruleAUAULabel} is always applicable
            if other rules are not.

        \item $[[uM']] = [[α⁻]]$
            Then $[[α⁻]] = [[ [aus'i] α⁻]] = [[ iNi ]]$, which means
            that \ruleref{\ottdruleAUNVarLabel} is applicable.

        \item $[[uM']] = [[↑uQ']]$
            Then $[[ ↑[aus'i]uQ']] = [[ [aus'i]↑uQ']] = [[ iNi ]]$, that is
            $[[iN1]]$ and $[[iN2]]$ have form $[[↑iP1]]$ and $[[↑iP2]]$ respectively.

            Moreover, $[[ [aus'i]uQ' = iPi ]]$, which means that 
            $[[(Ξ', uQ', aus'1, aus'2)]]$ is an anti-unifier of $[[iP1]]$ and $[[iP2]]$.
            Then by the induction hypothesis, there exists $[[(Ξ, uQ, aus1, aus2)]]$ such that
            $[[Γ ⊨ iP1 ≈au iP2 ⫤ (Ξ, uQ, aus1, aus2)]]$, and hence, 
            $[[Γ ⊨ ↑iP1 ≈au ↑iP2 ⫤ (Ξ, ↑uQ, aus1, aus2)]]$ by \ruleref{\ottdruleAUShiftULabel}.
        \item $[[uM']] = [[∀pas.uM'']]$ This case is similar to the previous one:
            we consider $\forall[[pas]]$ as a constructor. 
            Notice that $[[∀pas.[aus'i]uM'']] = [[ [aus'i]∀pas.uM'']] = [[ iNi ]]$, that is
            $[[iN1]]$ and $[[iN2]]$ have form $[[∀pas.iN''1]]$ and $[[∀pas.iN''2]]$ respectively.

            Moreover, $[[ [aus'i]uM'' = iN''i ]]$, which means that
            $[[(Ξ', uM'', aus'1, aus'2)]]$ is an anti-unifier of $[[iN''1]]$ and $[[iN''2]]$.
            Then by the induction hypothesis, there exists $[[(Ξ, uM, aus1, aus2)]]$ such that
            $[[Γ ⊨ iN''1 ≈au iN''2 ⫤ (Ξ, uM, aus1, aus2)]]$, and hence,
            $[[Γ ⊨ ∀pas.iN''1 ≈au ∀pas.iN''2 ⫤ (Ξ, ∀pas.uM, aus1, aus2)]]$ by 
            \ruleref{\ottdruleAUForallLabel}.
        \item $[[uM']] = [[uQ' → uM'']]$
            Then $[[ [aus'i]uQ' → [aus'i]uM'']] = [[ [aus'i](uQ' → uM'')]] = [[ iNi ]]$, that is
            $[[iN1]]$ and $[[iN2]]$ have form $[[uP1 → uN'1]]$ and $[[uP2 → uN'2]]$ respectively.

            Moreover, $[[ [aus'i]uQ' = iPi ]]$ and $[[ [aus'i]uM'' = iN''i ]]$, which means that
            $[[(Ξ', uQ', aus'1, aus'2)]]$ is an anti-unifier of $[[iP1]]$ and $[[iP2]]$,
            and $[[(Ξ', uM'', aus'1, aus'2)]]$ is an anti-unifier of $[[iN''1]]$ and $[[iN''2]]$.
            Then by the induction hypothesis, 
            $[[Γ ⊨ iP1 ≈au iP2 ⫤ (Ξ1, uQ, aus1, aus2)]]$ and 
            $[[Γ ⊨ iN''1 ≈au iN''2 ⫤ (Ξ2, uM, aus3, aus4)]]$ succeed.
            The result of the algorithm is $[[(Ξ1 ∪ Ξ2, uQ → uM, aus1 ∪ aus3, aus2 ∪ aus4)]]$.

        \item $[[uQ']] = [[â⁺]]$ 
            This case if not possible, since $[[Γ ; Ξ' ⊢ uQ']]$ means 
            $[[â⁺]] \in [[Ξ']]$, but $[[Ξ']]$ can only contain negative variables. 

        \item Other positive cases are proved symmetrically to the corresponding negative ones.
    \end{caseof}
\end{proof}

\begin{lemma}[Initiality of Anti-Unification] \label{lemma:anti-unification-initial}
    \hfill
    \begin{itemize}
        \item [$+$] 
            Assume that $[[iP1]]$ and $[[iP2]]$ are normalized, and
            $[[Γ ⊨ iP1 ≈au iP2 ⫤ (Ξ, uQ, aus1, aus2)]]$, 
            then $[[(Ξ, uQ, aus1, aus2)]]$ is more specific than
            any other sound anti-unifier $[[(Ξ', uQ', aus'1, aus'2)]]$, i.e.
            if 
            \begin{enumerate}
                \item $[[Γ ; Ξ' ⊢ uQ']]$,
                \item $[[Γ ; · ⊢ aus'i : Ξ']]$ for $i \in \{1,2\}$
                are anti-unification solutions , and
                \item $[[ [aus'i] uQ' = iPi ]]$ for $i \in \{1,2\}$
            \end{enumerate}
            then there exists $[[ausr]]$ such that
            $[[Γ ; Ξ ⊢ ausr : (Ξ' | uv uQ')]]$ and $[[ [ausr] uQ' = uQ ]]$. 
            Moreover, $[[ [ausr]β̂⁻]]$ can
            be uniquely determined by $[[ [aus'1]β̂⁻ ]]$, $[[ [aus'2]β̂⁻ ]]$, and
            $[[Γ]]$.
        \item [$-$] 
            Assume that $[[iN1]]$ and $[[iN2]]$ are normalized, and
            $[[Γ ⊨ iN1 ≈au iN2 ⫤ (Ξ, uM, aus1, aus2)]]$, 
            then $[[(Ξ, uM, aus1, aus2)]]$ is more specific than
            any other sound anti-unifier $[[(Ξ', uM', aus'1, aus'2)]]$, i.e.
            if
            \begin{enumerate}
                \item $[[Γ ; Ξ' ⊢ uM']]$,
                \item $[[Γ ; · ⊢ aus'i : Ξ']]$ for $i \in \{1,2\}$
                are anti-unification solutions, and
                \item $[[ [aus'i] uM' = iNi ]]$ for $i \in \{1,2\}$
            \end{enumerate}
            then there exists $[[ausr]]$ such that
            $[[Γ ; Ξ ⊢ ausr : (Ξ' | uv uM')]]$ and $[[ [ausr] uM' = uM ]]$.
            Moreover, $[[ [ausr]β̂⁻]]$ can
            be uniquely determined by $[[ [aus'1]β̂⁻ ]]$, $[[ [aus'2]β̂⁻ ]]$, and
            $[[Γ]]$.
    \end{itemize}
\end{lemma}
\begin{proof}
    First, let us assume that $[[uM']]$ is a algorithmic variable $[[α̂⁻]]$. 
    Then we can take $[[ausr]] = [[α̂⁻]] \mapsto [[uM]]$, which satisfies the required properties:
    \begin{itemize}
        \item $[[Γ ; Ξ ⊢ ausr : (Ξ' | uv uM')]]$ holds since 
        $[[Ξ' | uv uM']] = [[{α̂⁻}]]$ and $[[Γ ; Ξ ⊢ uM]]$ by the soundness of anti-unification (\cref{lemma:anti-unification-soundness});
        \item $[[ [ausr] uM' = uM ]]$ holds by construction
        \item $[[ [ausr]α̂⁻]] = [[uM]]$ is the anti-unifier of 
            $[[iN1]] = [[ [aus'1] α̂⁻]]$ and $[[iN2]] = [[ [aus'2] α̂⁻]]$
            in context $[[Γ]]$, and hence, it is uniquely determined by them (\cref{obs:anti-unification-deterministic}).
    \end{itemize}

    Now, we can assume that $[[uM']]$ is not a algorithmic variable. 
    We prove by induction on the derivation of $[[Γ ⊨ iP1 ≈au iP2 ⫤ (Ξ, uQ, aus1, aus2)]]$
    and mutually on the derivation of $[[Γ ⊨ iN1 ≈au iN2 ⫤ (Ξ, uM, aus1, aus2)]]$.

    Since $[[uM']]$ is not a algorithmic variable, 
    the substitution acting on $[[uM']]$ preserves its outer constructor. 
    In other words, 
    $[[ [aus'i] uM' = iNi ]]$ means that $[[uM']]$, 
    $[[iN1]]$ and $[[iN2]]$ have the same outer constructor. 
    Let us consider the algorithmic anti-unification rule corresponding to this constructor, 
    and show that it was successfully applied to anti-unify $[[iN1]]$ and $[[iN2]]$ 
    (or $[[iP1]]$ and $[[iP2]]$).

    \begin{caseof}
        \item \ruleref{\ottdruleAUNVarLabel}, i.e. $[[iN1]] = [[α⁻]] = [[iN2]]$.
        \label{case:anti-unification-initial:nvar}
        This rule is applicable since it has no premises. 
        
        Then $[[Ξ]] = [[·]]$, $[[uM]] = [[α⁻]]$, 
        and $[[aus1]] = [[aus2]] = [[·]]$.
        Since $[[ [aus'i] uM' = iNi ]] = [[α⁻]]$
        and $[[uM']]$ is not a algorithmic variable, $[[uM']] = [[α⁻]]$.
        Then we can take $[[ausr]] = [[·]]$, which satisfies the required properties:
        \begin{itemize}
            \item $[[Γ ; Ξ ⊢ ausr : (Ξ' | uv uM')]]$ holds vacuously since 
            $[[Ξ' | uv uM']] = [[∅]]$; 
            \item $[[ [ausr] uM' = uM ]]$, that is $[[ [·] α⁻ = α⁻ ]]$
            holds by substitution properties;
            \item the unique determination of $[[ [ausr]α̂⁻]]$ for $[[α̂⁻]] \in [[Ξ' | uv uM']] = [[∅]]$ holds vacuously.
        \end{itemize}

        \item \ruleref{\ottdruleAUShiftULabel}, i.e. 
        $[[iN1]] = [[↑iP1]]$ and $[[iN2]] = [[↑iP2]]$.
        \label{case:anti-unification-initial:shiftu}

        Then since $[[ [aus'i] uM' = iNi ]] = [[↑iPi]]$ and $[[uM']]$ is not a algorithmic variable,
        $[[uM']] = [[↑uQ']]$, where $[[ [aus'i] uQ' = iPi ]]$. 
        Let us show that $[[(Ξ', uQ', aus'1, aus'2)]]$ 
        is an anti-unifier of $[[iP1]]$ and $[[iP2]]$.
        \begin{enumerate}
            \item $[[Γ ; Ξ' ⊢ uQ']]$ holds by inversion of $[[Γ ; Ξ' ⊢ ↑uQ']]$;
            \item $[[Γ ; · ⊢ aus'i : Ξ']]$ holds by assumption;
            \item $[[ [aus'i] uQ' = iPi ]]$ holds by assumption.
        \end{enumerate}

        This way, by the completeness of anti-unification 
        (\cref{lemma:anti-unification-completeness}),
        the anti-unification algorithm succeeds on $[[iP1]]$ and $[[iP2]]$:
        $[[Γ ⊨ iP1 ≈au iP2 ⫤ (Ξ, uQ, aus1, aus2)]]$,
        which means that \ruleref{\ottdruleAUShiftULabel} is applicable to infer 
        $[[Γ ⊨ ↑iP1 ≈au ↑iP2 ⫤ (Ξ, ↑uQ, aus1, aus2)]]$.

        Moreover, by the induction hypothesis,
        $[[(Ξ, uQ, aus1, aus2)]]$ is more specific than $[[(Ξ', uQ', aus'1, aus'2)]]$,
        which immediately implies that $[[(Ξ, ↑uQ, aus1, aus2)]]$ is more specific than
        $[[(Ξ', ↑uQ', aus'1, aus'2)]]$ (we keep the same $[[ausr]]$).

        \item \ruleref{\ottdruleAUForallLabel}, i.e. 
        $[[iN1]] = [[∀pas.iN'1]]$ and $[[iN2]] = [[∀pas.iN'2]]$.
        \label{case:anti-unification-initial:forall}
        The proof is symmetric to the previous case.
        Notice that the context $[[Γ]]$ is not changed in \ruleref{\ottdruleAUForallLabel}, 
        as it represents the context in which the anti-unification variables must be instantiated,
        rather than the context forming the types that are being anti-unified.

        \item \ruleref{\ottdruleAUArrowLabel}, i.e.
        $[[iN1]] = [[iP1 → iN'1]]$ and $[[iN2]] = [[iP2 → iN'2]]$.

        Then since $[[ [aus'i] uM' = iNi ]] = [[iPi → iN'i]]$ and $[[uM']]$ is not a algorithmic variable,
        $[[uM']] = [[uQ' → uM'']]$, where $[[ [aus'i] uQ' = iPi ]]$ and $[[ [aus'i] uM'' = iN''i ]]$.

        Let us show that $[[(Ξ', uQ', aus'1, aus'2)]]$
        is an anti-unifier of $[[iP1]]$ and $[[iP2]]$.
        \begin{enumerate}
            \item $[[Γ ; Ξ' ⊢ uQ']]$ holds by inversion of $[[Γ ; Ξ' ⊢ uQ' → uM'']]$;
            \item $[[Γ ; · ⊢ aus'i : Ξ']]$ holds by assumption;
            \item $[[ [aus'i] uQ' = iPi ]]$ holds by assumption.
        \end{enumerate}

        Similarly, $[[(Ξ', uM'', aus'1, aus'2)]]$ is an anti-unifier of $[[iN''1]]$ and $[[iN''2]]$.

        Then by the completeness of anti-unification (\cref{lemma:anti-unification-completeness}),
        the anti-unification algorithm succeeds on $[[iP1]]$ and $[[iP2]]$:
        $[[Γ ⊨ iP1 ≈au iP2 ⫤ (Ξ1, uQ, aus1, aus2)]]$;
        and on $[[iN'1]]$ and $[[iN'2]]$:
        $[[Γ ⊨ iN''1 ≈au iN''2 ⫤ (Ξ2, uM''', aus3, aus4)]]$.
        Notice that $[[aus1 & aus3]]$ and $[[aus2 & aus4]]$ are defined, 
        in other words, for any $[[β̂⁻]] \in [[Ξ1]] \cap [[Ξ2]]$,
        $[[ [aus1] β̂⁻ = [aus2] β̂⁻ ]]$ and $[[ [aus3] β̂⁻ = [aus4] β̂⁻ ]]$,
        which follows immediately from \cref{obs:names-defined-by-mapping}.
        This way, the algorithm proceeds by applying \ruleref{\ottdruleAUArrowLabel} and returns
        $[[(Ξ1 ∪ Ξ2, uQ → uM''', aus1 ∪ aus3, aus2 ∪ aus4)]]$.

        It is left to construct $[[ausr]]$ such that $[[Γ ; Ξ ⊢ ausr : (Ξ' | uv uM')]]$ and $[[ [ausr] uM' = uM ]]$.
        By the induction hypothesis, there exist $[[ausr1]]$ and $[[ausr2]]$ such that
        $[[Γ ; Ξ1 ⊢ ausr1 : (Ξ' | uv uQ')]]$, 
        $[[Γ ; Ξ2 ⊢ ausr2 : (Ξ' | uv uM'')]]$,
        $[[ [ausr1] uQ' = uQ ]]$, and $[[ [ausr2] uM'' = uM''' ]]$.

        Let us show that $[[ausr]] = [[ausr1 ∪ ausr2]]$ satisfies the required properties:
        \begin{itemize}
            \item $[[Γ ; Ξ1 ∪ Ξ2 ⊢ ausr1 ∪ ausr2 : (Ξ' | uv uM')]]$ holds since 
            $[[Ξ' | uv uM']] = [[Ξ' | uv uQ' → uM'']] = [[(Ξ' | uv uQ') ∪ (Ξ' | uv uM'')]]$,
            $[[Γ ; Ξ1 ⊢ ausr1 : (Ξ' | uv uQ')]]$ and $[[Γ ; Ξ2 ⊢ ausr2 : (Ξ' | uv uM'')]]$;
            \item $[[ [ausr] uM' ]] = [[ [ausr] (uQ' → uM'') ]] = [[ [ausr | uv uQ'] uQ' → [ausr | uv uM''] uM'' ]] =
            [[ [ausr1] uQ' → [ausr2] uM'' ]] = [[ uQ → uM''']] = [[ uM ]]$;
            \item Since $[[ [ausr]β̂⁻]]$ is either equal to  $[[ [ausr1]β̂⁻]]$ or $[[ [ausr2]β̂⁻]]$,
            it inherits their property that it is uniquely determined by $[[ [aus'1]β̂⁻]]$, $[[ [aus'2]β̂⁻]]$, and $[[Γ]]$.
        \end{itemize}

        \item $[[iP1]] = [[iP2]] = [[a⁺]]$. This case is symmetric to \cref{case:anti-unification-initial:nvar}.
        \item $[[iP1]] = [[↓iN1]]$ and $[[iP2]] = [[↓iN2]]$. This case is symmetric to \cref{case:anti-unification-initial:shiftu}
        \item $[[iP1]] = [[∃nas.iP'1]]$ and $[[iP2]] = [[∃nas.iP'2]]$. This case is symmetric to \cref{case:anti-unification-initial:forall}
        \end{caseof}
\end{proof}