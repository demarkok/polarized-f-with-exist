\begin{lemma}[Free Variable Propagation] \label{lemma:fv-propagation}
  In the judgments of negative subtyping or positive supertyping,
  free variables propagate left-to-right. For a context $[[Γ]]$,
  \begin{itemize}
    \item $-$ if $[[Γ ⊢ iN ≤ iM]]$ then $[[fv(iN)]] \subseteq [[fv(iM)]]$
    \item $+$ if $[[Γ ⊢ iP ≥ iQ]]$ then $[[fv(iP)]] \subseteq [[fv(iQ)]]$
  \end{itemize}
\end{lemma}
\begin{proof}
  Mutual induction on $[[Γ ⊢ iN ≤ iM]]$ and $[[Γ ⊢ iP ≥ iQ]]$.
  \begin{caseof}
  \item $[[G ⊢ a⁻ ≤ a⁻]]$\\
    It is self-evident that $[[{a⁻} ⊆ {a⁻}]]$.
  \item $[[G ⊢ ↑iP ≤ ↑iQ]]$
    From the inversion (and unfolding $[[G ⊢ iP ≈ iQ]]$ ), we have
    $[[G ⊢ iP ≥ iQ]]$. Then by the induction hypothesis,
    $[[fv(iP)]] \subseteq [[fv(iQ)]]$. The desired 
    inclusion holds, since $[[fv(↑iP)]] = [[fv(iP)]]$ and
    $[[fv(↑iQ)]] = [[fv(iQ)]]$.
  \item $[[G ⊢ iP → iN ≤ iQ → iM]]$
    The induction hypothesis applied to the premises gives:
    $[[fv(iP)]] \subseteq [[fv(iQ)]]$ and
    $[[fv(iN)]] \subseteq [[fv(iM)]]$.
    Then $[[fv(iP → iN)]] = [[fv(iP) ∪ fv(iN)]] \subseteq
    [[fv(iQ) ∪ fv(iM)]] = [[fv(iQ → iM)]]$.

  \item $[[G ⊢ ∀pas.iN ≤ ∀pbs.iM]]$\\
    $
    \begin{aligned}[t]
      [[fv ∀pas.iN ]] &\subseteq [[fv ([iPs/pas] iN) ]] ~\setminus~ [[{pbs}]] 
                      &&   \text{here $[[{pbs}]]$ is excluded by the premise $[[fv iN ∩ {pbs} = ∅]]$}\\
                      &\subseteq [[fv iM]] ~\setminus~ [[{pbs}]]
                      &&   \text{by the induction hypothesis, } [[fv ([iPs/pas] iN) ]] \subseteq [[fv iM]] \\
                      &\subseteq [[fv ∀pbs.iM]]
    \end{aligned}
    $
  \item The positive cases are symmetric.
  \end{caseof}
\end{proof}

\begin{corollary}[Free Variables of mutual subtypes] \label{corollary:fv-mut-sub}
  \hfill
  \begin{itemize}
    \item [$-$] If $[[Γ ⊢ iN ≈ iM]]$ then $[[fv iN]] = [[fv iM]]$, 
    \item [$+$] If $[[Γ ⊢ iP ≈ iQ]]$ then $[[fv iP]] = [[fv iQ]]$
  \end{itemize}
\end{corollary}

\begin{lemma}[Decomposition of quantifier rules]
  \label{lemma:quant-rule-decomposition}
  Assuming that $[[pas]]$, $[[pbs]]$, $[[nas]]$, and $[[nas]]$ are disjoint from $[[Γ]]$,
  \begin{itemize}
    \item [$-_{R}$] $[[Γ ⊢ iN ≤ ∀pbs.iM]]$ holds if and only if $[[Γ, pbs ⊢ iN ≤ iM]]$;
    \item [$+_{R}$] $[[Γ ⊢ iP ≥ ∃nbs.iQ]]$ holds if and only if $[[Γ, nbs ⊢ iP ≥ iQ]]$;
    \item [$-_{L}$] suppose $[[iM]] \neq [[∀]]\dots$
      then $[[Γ ⊢ ∀pas.iN ≤ iM]]$ holds if and only if $[[Γ ⊢ [iPs/pas]iN ≤ iM]]$
      for some $[[Γ ⊢ iPs]]$;
    \item [$+_{L}$] suppose $[[iQ]] \neq [[∃]]\dots$
      then $[[Γ ⊢ ∃nas.iP ≥ iQ]]$ holds if and only if $[[Γ ⊢ [iNs/nas]iP ≥ iQ]]$
      for some $[[Γ ⊢ iNs]]$.
  \end{itemize}
\end{lemma}
\begin{proof}
  \hfill
  \begin{itemize}
    \item [$-_{R}$] Let us prove both directions. 
      \begin{itemize}
        \item [\Rightarrow] Let us assume $[[Γ ⊢ iN ≤ ∀pbs.iM]]$.
          $[[Γ ⊢ iN ≤ ∀pbs.iM]]$.
          Let us decompose $[[iM]]$ as $[[∀pbs'.iM']]$ where $[[iM']]$ does not start with $[[∀]]$, 
          and decompose $[[iN]]$ as $[[∀pas.iN']]$ where $[[iN']]$ does not start with $[[∀]]$.
          If $[[pbs]]$ is empty, then $[[Γ, pbs ⊢ iN ≤ iM]]$ holds by assumption.
          Otherwise, $[[Γ ⊢ ∀pas.iN' ≤ ∀pbs.∀pbs'.iM]]$ is inferred by
          \ruleref{\ottdruleDOneForallLabel}, and by inversion:
          $[[Γ,pbs,pbs' ⊢ [iPs/pas]iN' ≤ iM']]$ for some $[[Γ,pbs,pbs' ⊢ iPs]]$.
          Then again by \ruleref{\ottdruleDOneForallLabel} with the same $[[iPs]]$,
          $[[Γ,pbs ⊢ ∀pas.iN' ≤ ∀pbs'.iM']]$, that is $[[Γ,pbs ⊢ iN ≤ iM]]$.
        \item [\Leftarrow] let us assume $[[Γ, pbs ⊢ iN ≤ iM]]$, and let us decompose 
          $[[iN]]$ as $[[∀pas.iN']]$ where $[[iN']]$ does not start with $[[∀]]$, 
          and $[[iM]]$ as $[[∀pbs'.iM']]$ where $[[iM']]$ does not start with $[[∀]]$.
          if $[[pas]]$ and $[[pbs']]$ are empty then $[[Γ, pbs ⊢ iN ≤ iM]]$
          is turned into $[[Γ ⊢ iN ≤ ∀pbs.iM]]$ by \ruleref{\ottdruleDOneForallLabel}.
          Otherwise, $[[Γ, pbs ⊢ ∀pas.iN' ≤ ∀pbs'.iM']]$ is inferred by
          \ruleref{\ottdruleDOneForallLabel}, that is $[[Γ, pbs, pbs' ⊢ [iPs/pas]iN' ≤ iM']]$
          for some $[[Γ, pbs, pbs' ⊢ iPs]]$.
          Then by \ruleref{\ottdruleDOneForallLabel} again,
          $[[Γ ⊢ ∀pas.iN' ≤ ∀pbs,pbs'.iM']]$, in other words, $[[Γ ⊢ ∀pas.iN' ≤ ∀pbs.∀pbs'.iM']]$, 
          that is $[[Γ ⊢ iN ≤ ∀pbs.iM]]$.
          
      \end{itemize}
    \item [$-_{L}$] Suppose $[[iM]] \neq [[∀]]\dots$. Let us prove both directions.
      \begin{itemize}
        \item [\Rightarrow] Let us assume $[[Γ ⊢ ∀pas.iN ≤ iM]]$.
          then if $[[pas = ·]]$, $[[Γ ⊢ iN ≤ iM]]$ holds immediately.
          Otherwise, let us decompose  $iN$ as $[[∀pas'.iN']]$ where 
          $[[iN']]$ does not start with $[[∀]]$.
          Then $[[Γ ⊢ ∀pas.∀pas'.iN' ≤ iM']]$ is inferred by
          \ruleref{\ottdruleDOneForallLabel},
          and by inversion, 
          there exist $[[Γ ⊢ iPs,iPs']]$ 
          such that $[[Γ ⊢ [iPs/pas][iPs'/pas']iN' ≤ iM']]$ 
          (the decomposition of substitutions is possible since $[[{pas} ∩ {Γ} = ∅]]$).
          Then by \ruleref{\ottdruleDOneForallLabel} again,
          $[[Γ ⊢ ∀pas'.[iPs'/pas']iN' ≤ iM']]$ (notice that $[[ [iPs'/pas']iN' ]]$ cannot
          start with $[[∀]]$).
        \item [\Leftarrow] Let us assume 
          $[[Γ ⊢ [iPs/pas]iN ≤ iM]]$ for some $[[Γ ⊢ iPs]]$.
          let us decompose $iN$ as $[[∀pas'.iN']]$ where $[[iN']]$ does not start with $[[∀]]$.
          Then $[[Γ ⊢ [iPs/pas]∀pas'.iN' ≤ iM']]$ or, equivalently,
          $[[Γ ⊢ ∀pas'.[iPs/pas]iN' ≤ iM']]$ is inferred by \ruleref{\ottdruleDOneForallLabel}
          (notice that $[[ [iPs/pas]iN' ]]$ cannot start with $[[∀]]$).
          By inversion, there exist $[[Γ ⊢ iPs']]$ such that 
          $[[Γ ⊢ [iPs'/pas'][iPs/pas]iN' ≤ iM']]$. Since $[[pas']]$ is disjoint
          from the free variables of $[[iPs]]$ and from $[[pas]]$, the composition of 
          $[[iPs'/pas']]$ and $[[iPs/pas]]$ can be joined into a single substitution
          well-formed in $[[Γ]]$. Then by \ruleref{\ottdruleDOneForallLabel} again,
          $[[Γ ⊢ ∀pas.iN ≤ iM]]$.
      \end{itemize}
      \item [$+$] The positive cases are proved symmetrically.
  \end{itemize}
\end{proof}

\begin{corollary}[Redundant quantifier elimination]
  \label{corollary:red-quant-elim}
  \hfill
  \begin{itemize}
    \item [$-_{L}$] Suppose that $[[ {pas} ∩ fv(iN) = ∅]]$ then 
      $[[Γ ⊢ ∀pas.iN ≤ iM]]$ holds if and only if $[[Γ ⊢ iN ≤ iM]]$;
    \item [$-_{R}$] Suppose that $[[ {pas} ∩ fv(iM) = ∅]]$ then 
      $[[Γ ⊢ iN ≤ ∀pas.iM]]$ holds if and only if $[[Γ ⊢ iN ≤ iM]]$;
    \item [$+_{L}$] Suppose that $[[ {nas} ∩ fv(iP) = ∅]]$ then
      $[[Γ ⊢ ∃nas.iP ≥ iQ]]$ holds if and only if $[[Γ ⊢ iP ≥ iQ]]$.
    \item [$+_{R}$] Suppose that $[[ {nas} ∩ fv(iQ) = ∅]]$ then 
      $[[Γ ⊢ iP ≥ ∃nas.iQ]]$ holds if and only if $[[Γ ⊢ iP ≥ iQ]]$.
  \end{itemize}
\end{lemma}
\begin{proof}
  \begin{itemize}
    \item [$-_{R}$] Suppose that $[[ {pas} ∩ fv(iM) = ∅]]$ then 
      by \cref{lemma:quant-rule-decomposition},
      $[[Γ ⊢ iN ≤ ∀pas.iM]]$ 
      is equivalent to $[[Γ, pas ⊢ iN ≤ iM]]$,
      By \label{lemma:wf-ctxt-equiv},
      since $[[{pas} ∩ fv(iN) = ∅]]$ and $[[{pas} ∩ fv(iM) = ∅]]$,
      $[[Γ, pas ⊢ iN ≤ iM]]$ is equivalent to $[[Γ ⊢ iN ≤ iM]]$.

    \item [$-_{L}$] Suppose that $[[ {pas} ∩ fv(iN) = ∅]]$.
      Let us decompose $[[iM]]$ as $[[∀pbs.iM']]$ 
      where $[[iM']]$ does not start with $[[∀]]$.
      By \cref{lemma:quant-rule-decomposition},
      $[[Γ ⊢ ∀pas.iN ≤ ∀pbs.iM']]$ is equivalent to
      $[[Γ,pbs ⊢ ∀pas.iN ≤ iM']]$, 
      which is equivalent to 
      existence of $[[Γ,pbs ⊢ iPs]]$ such that 
      $[[Γ,pbs ⊢ [iPs/pas]iN ≤ iM']]$.
      Since $[[ [iPs/pas]iN  = iN]]$, the latter is equivalent to 
      $[[Γ,pbs ⊢ iN ≤ iM']]$,
      which is equivalent to $[[Γ ⊢ iN ≤ ∀pbs.iM']]$.
      $[[Γ,pbs ⊢ iPs]]$ can be chosen arbitrary, for example, $[[iPsi]] = [[∃α⁻.↓α⁻]]$.
    \item [$+$] The positive cases are proved symmetrically.
  \end{itemize}
\end{proof}

\begin{lemma}[Subtypes and supertypes of a variable]
  \label{lemma:var-subt}
  Assuming $[[Γ ⊢  α⁻]]$, $[[Γ ⊢ α⁺]]$, $[[Γ ⊢ iN]]$, and $[[Γ ⊢ iP]]$,
  \begin{itemize}
  \item[$+$] if $[[Γ ⊢ iP ≥ ∃nas.α⁺]]$ or $[[Γ ⊢ ∃nas.α⁺ ≥ iP ]]$ then $[[iP]] = [[∃nbs.α⁺]]$ (for some potentially empty $[[nbs]]$)
  \item[$-$] if $[[Γ ⊢ iN ≤ ∀pas.α⁻]]$ or $[[Γ ⊢ ∀pas.α⁻ ≤ iN ]]$ then $[[iN]] = [[∀pbs.α⁻]]$ (for some potentially empty $[[pbs]]$)
  \end{itemize}
\end{lemma}
\begin{proof}
  We prove by induction on the tree
  inferring $[[Γ ⊢ iP ≥ ∃nas.α⁺]]$ or $[[Γ ⊢ ∃nas.α⁺ ≥ iP ]]$ or
  or $[[Γ ⊢ iN ≤ ∀pas.α⁻]]$ or $[[Γ ⊢ ∀pas.α⁻ ≤ iN ]]$.

  Let us consider which one of these judgments is inferred.
  \begin{caseof}
  \item $[[Γ ⊢ iP ≥ ∃nas.α⁺]]$\\
    If the size of the inference tree is $1$ then the only rule that can infer
    it is \ruleref{\ottdruleDOnePVarLabel}, which
    implies that $[[nas]]$ is empty and $[[iP = α⁺]]$.

    If the size of the inference tree is $>1$ then the last rule inferring
    it must be \ruleref{\ottdruleDOneExistsLabel}. By inverting this rule,
    $[[iP = ∃nbs.iP']]$ where $[[iP']]$ does not start with $\exists$ and
    $[[Γ, nas ⊢ [iNs/nbs] iP' ≥ α⁺]]$ for some $[[G, nas ⊢ iNi]]$.

    By the induction hypothesis, $[[ [iNs/nbs] iP' = ∃ncs.α⁺]]$.
    What shape can $[[iP']]$ have?
    As mentioned, it does not start with $\exists$, and it cannot start with
    $\uparrow$ (otherwise, $[[ [iNs/nas] iP' ]]$ would also
    start with $\uparrow$ and would not be equal to $[[∃nbs.α⁺]]$).
    This way, $[[iP']]$ is a \emph{positive} variable. 
    As such, $[[ [iNs/nas] iP' = iP']]$,
    and then $[[iP' = ∃ncs.α⁺]]$ meaning that $[[ncs]]$ is empty and $[[iP' = α⁺]]$.
    This way, $[[iP]] = [[∃nbs.iP']] = [[∃nbs.α⁺]]$, as required.

  \item $[[Γ ⊢ ∃nas.α⁺ ≥ iP]]$\\
    If the size of the inference tree is $1$ then the only rule that can infer
    it is \ruleref{\ottdruleDOnePVarLabel}, which
    implies that $[[nas]]$ is empty and $[[iP = α⁺]]$.

    If the size of the inference tree is $>1$ then the last rule inferring
    it must be \ruleref{\ottdruleDOneExistsLabel}. By inverting this rule,
    $[[iP = ∃nbs.iQ]]$ where $[[G, nbs ⊢ [iNs/nas]α⁺ ≥ iQ]]$ and $[[iQ]]$ 
    does not start with $\exists$.
    Notice that since $[[α⁺]]$ is positive, $[[ [iNs/nas]α⁺ = α⁺]]$, 
    i.e. $[[G, nbs ⊢ α⁺ ≥ iQ]]$.

    By the induction hypothesis, $[[iQ = ∃nbs'.α⁺]]$,
    and since $[[iQ]]$ does not start with $\exists$, $[[nbs']]$ is empty
    This way, $[[iP]] = [[∃nbs.iQ]] = [[∃nbs.α⁺]]$, as required.

  \item The negative cases ($[[Γ ⊢ iN ≤ ∀pas.α⁻]]$ and $[[Γ ⊢ ∀pas.α⁻ ≤ iN ]]$)
    are proved analogously.
  \end{caseof}
\end{proof}

\begin{corollary}[Variables have no proper subtypes and supertypes]
  \label{corollary:vars-no-proper-subtypes}
  Assuming that all mentioned types are well-formed in $[[Γ]]$,
  \begin{align*}
    [[Γ ⊢ iP ≥ α⁺]] ~ &\iff ~ [[iP = ∃nbs.α⁺]]  ~ \iff ~ [[Γ ⊢ iP ≈ α⁺]] ~ \iff ~ [[iP ≈ α⁺]]\\
    [[Γ ⊢ α⁺≥ iP]]  ~ &\iff ~ [[iP = ∃nbs.α⁺]]  ~ \iff ~ [[Γ ⊢ iP ≈ α⁺]] ~ \iff ~ [[iP ≈ α⁺]]\\
    [[Γ ⊢ iN ≤ α⁻]] ~ &\iff ~ [[iN = ∀pbs.α⁻]]  ~ \iff ~ [[Γ ⊢ iN ≈ α⁻]] ~ \iff ~ [[iN ≈ α⁻]]\\
    [[Γ ⊢ α⁻ ≤ iN]] ~ &\iff ~ [[iN = ∀pbs.α⁻]]  ~ \iff ~ [[Γ ⊢ iN ≈ α⁻]] ~ \iff ~ [[iN ≈ α⁻]]\\
  \end{align*}
\end{corollary}
\begin{proof}
  Notice that $[[Γ ⊢ ∃nbs.α⁺ ≈ α⁺]]$ and $[[∃nbs.α⁺ ≈ α⁺]]$ and apply
  \cref{lemma:var-subt}.
\end{proof}

\begin{lemma}[Reflexivity of subtyping] \label{lemma:subtyping-reflexivity}
  Assuming all the types are well-formed in $[[Γ]]$,
  \begin{itemize}
    \item [$-$] $[[Γ ⊢ iN ≤ iN]]$
    \item [$+$] $[[Γ ⊢ iP ≥ iP]]$
  \end{itemize}
\end{lemma}
\begin{proof}
  Let us prove it by the size of $[[iN]]$ and mutually, $[[iP]]$.
  \begin{caseof}
    \item $[[iN]] = [[α⁻]]$\\
      Then $[[Γ ⊢ α⁻ ≤ α⁻]]$ is inferred immediately by \ruleref{\ottdruleDOneNVarLabel}.
    \item $[[iN]] = [[∀pas.iN']]$ where $[[pas]]$ is not empty\\
      First, we rename $[[pas]]$ to fresh $[[pbs]]$ in $[[∀pas.iN']]$ to avoid
      name clashes: $[[∀pas.iN']] = [[∀pbs.[pas/pbs]iN']]$.
      Then to infer $[[Γ ⊢ ∀pas.iN' ≤ ∀pbs.[pas/pbs]iN']]$ we can apply 
      \ruleref{\ottdruleDOneForallLabel}, instantiating $[[pas]]$ with $[[pbs]]$:
      \begin{itemize}
        \item $[[fv iN ∩ {pbs} = ∅ ]]$ by choice of $[[pbs]]$,
        \item $[[G, pbs ⊢ pbi]]$,
        \item $[[G, pbs ⊢ [pbs/pas] iN' ≤ [pbs/pas] iN']]$ by the induction hypothesis,
        since the size of $[[ [pbs/pas]iN' ]]$ is equal to the size of $[[iN']]$,
        which is smaller than the size of $[[iN]] = [[∀pas.iN']]$.
      \end{itemize}
    \item $[[iN]] = [[iP → iM]]$\\
      Then $[[Γ ⊢ iP → iM ≤ iP → iM]]$ is inferred by \ruleref{\ottdruleDOneArrowLabel},
      since $[[Γ ⊢ iP ≥ iP]]$ and $[[Γ ⊢ iM ≤ iM]]$ hold the induction hypothesis. 
    \item $[[iN]] = [[↑iP]]$\\
      Then $[[Γ ⊢ ↑iP ≤ ↑iP]]$ is inferred by \ruleref{\ottdruleDOneShiftULabel},
      since $[[Γ ⊢ iP ≥ iP]]$ holds by the induction hypothesis.
    \item The positive cases are symmetric to the negative ones.
  \end{caseof}
\end{proof}

\begin{lemma}[Substitution preserves subtyipng]
  \label{lemma:subst-pres-subt}
  Suppose that all mentioned types are well-formed in $[[Γ1 , Γ]]$,
  and $[[σ]]$ is a substitution $[[Γ2 ⊢ σ : Γ1]]$.
  \begin{itemize}
    \item $-$ If $[[Γ1 , Γ ⊢ iN ≤ iM]]$ then $[[Γ2 , Γ ⊢ [σ]iN ≤ [σ]iM]]$
    \item $+$ If $[[Γ1 , Γ ⊢ iP ≥ iQ]]$ then $[[Γ2 , Γ ⊢ [σ]iP ≥ [σ]iQ]]$
  \end{itemize}
\end{lemma}
\begin{proof}
  We prove it by induction on the size of the derivation of $[[Γ ⊢ iN ≤ iM]]$
  and mutually, $[[Γ ⊢ iP ≥ iQ]]$. Let us consider the last rule 
  used in the derivation:
  \begin{caseof}
    \item \ruleref{\ottdruleDOneNVarLabel}. Then by inversion, 
      $[[iN = α⁻]]$ and $[[iM = α⁻]]$. By reflexivity of subtyping
      (\cref{lemma:subtyping-reflexivity}),
      we have $[[Γ2, Γ ⊢ [σ]α⁻ ≤ [σ]α⁻]]$, i.e. $[[Γ2, Γ ⊢ [σ]iN ≤ [σ]iM]]$,
      as required.
    \item  \ruleref{\ottdruleDOneForallLabel}. Then by inversion,
      $[[iN = ∀pas.iN']]$, $[[iM = ∀pbs.iM']]$, where $[[pas]]$ or $[[pbs]]$ is not empty.
      Moreover, $[[Γ1 , Γ, pbs ⊢ [iPs/pas]iN' ≤ iM']]$ for some $[[Γ1 , Γ, pbs ⊢ iPs]]$, and 
      $[[fv iN ∩ {pbs} = ∅ ]]$.

      Notice that since the derivation of $[[Γ1, Γ, pbs ⊢ [iPs/pas]iN' ≤ iM']]$ is
      a subderivation of the derivation of $[[Γ ⊢ iN ≤ iM]]$, its size is smaller, 
      and hence, the induction hypothesis applies:
      $[[Γ2 , Γ, pbs ⊢ [σ][iPs/pas]iN' ≤ [σ]iM']]$.

      Notice that by convention, $[[pas]]$ and $[[pbs]]$ are fresh, and thus,  
      $[[ [σ]∀pas.iN' ]] = [[ ∀pas.[σ]iN' ]]$ and $[[ [σ]∀pbs.iM' ]] = [[ ∀pbs.[σ]iM' ]]$, 
      which means that the required $[[Γ2, Γ ⊢ [σ]∀pas.iN' ≤ [σ]∀pbs.iM']]$ is rewritten as
      $[[Γ2 , Γ ⊢ ∀pas.[σ]iN' ≤ ∀pbs.[σ]iM']]$.

      To infer it, we apply \ruleref{\ottdruleDOneForallLabel}, 
      instantiating $[[pai]]$ with $[[ [σ]iPi ]]$:
      \begin{itemize}
        \item $[[fv [σ]iN ∩ {pbs} = ∅ ]]$;
        \item $[[Γ2, Γ,pbs⊢ [σ]iPi]]$, by \cref{lemma:wf-subst} since from the inversion,
          $[[Γ1, Γ, pbs ⊢ iPi]]$;
        \item $[[Γ, pbs ⊢ [ [σ]iPs/pas ][σ]iN' ≤ [σ]iM']]$ holds
          by \cref{lemma:subst-composition}:
          Since $[[pas]]$ is fresh, it is disjoint with the domain and the codomain of $[[σ]]$
          ($[[Γ1]]$ and $[[Γ2]]$), and thus, 
          $[[ [σ][iPs/pas]iN' ]] = [[ [ σ <=< iPs/pas ][σ]iN' ]] = [[ [ [σ]iPs/pas ][σ]iN' ]]$.
          Then $[[Γ2, Γ, pbs ⊢ [σ][iPs/pas]iN' ≤ [σ]iM']]$ holds by the induction hypothesis.
      \end{itemize}

    \item \ruleref{\ottdruleDOneArrowLabel}. Then by inversion,
      $[[iN = iP → iN1]]$, $[[iM = iQ → iM1]]$, $[[Γ ⊢ iP ≥ iQ]]$, and $[[Γ ⊢ iN1 ≤ iM1]]$.
      And by the induction hypothesis, $[[Γ' ⊢ [σ]iP ≥ [σ]iQ]]$ and $[[Γ' ⊢ [σ]iN1 ≤ [σ]iM1]]$.
      Then $[[Γ' ⊢ [σ]iN ≤ [σ]iM]]$, i.e. $[[Γ' ⊢ [σ]iP → [σ]iN1 ≤ [σ]iQ → [σ]iM1]]$,
      is inferred by \ruleref{\ottdruleDOneArrowLabel}.
    \item \ruleref{\ottdruleDOneShiftULabel}. Then by inversion,
      $[[iN = ↑iP]]$, $[[iM = ↑iQ]]$, and $[[Γ ⊢ iP ≈ iQ]]$,
      which by inversion means that $[[Γ ⊢ iP ≥ iQ]]$ and $[[Γ ⊢ iQ ≥ iP]]$.
      Then the induction hypothesis applies, and we have $[[Γ' ⊢ [σ]iP ≥ [σ]iQ]]$
      and $[[Γ' ⊢ [σ]iQ ≥ [σ]iP]]$. 
      Then by sequential application of \ruleref{\ottdruleDOneNDefLabel} 
      and \ruleref{\ottdruleDOneShiftULabel} to these judgments,
      we have $[[Γ' ⊢ ↑[σ]iP ≤ ↑[σ]iQ]]$, i.e.
      $[[Γ' ⊢ [σ]iN ≤ [σ]iM]]$, as required.
    \item The positive cases are proved symmetrically.
  \end{caseof}
\end{proof}

\begin{corollary}[Substitution preserves subtyping induced equivalence]
  \label{corollary:subst-pres-equiv}
  Suppose that $[[Γ ⊢ σ : Γ1]]$. Then
    \begin{itemize}
      \item[$+$] if $[[Γ1 ⊢ iP]]$,~ $[[Γ1 ⊢ iQ]]$,~ and $[[Γ1 ⊢ iP ≈ iQ]]$ ~ 
        then $[[Γ ⊢ [σ]iP ≈ [σ]iQ]]$
      \item[$-$] if $[[Γ1 ⊢ iN]]$,~ $[[Γ1 ⊢ iM]]$,~ and $[[Γ1 ⊢ iN ≈ iM]]$ ~ 
        then $[[Γ ⊢ [σ]iN ≈ [σ]iM]]$
    \end{itemize}
\end{corollary}

\begin{lemma}[Strong transitivity of subtyping] \label{lemma:subtyping-transitivity}
  Assuming all the types are well-formed in $[[Γ]]$,
  \begin{itemize}
    \item $-$ if $[[Γ ⊢ iN ≤ iM1]]$, $[[Γ ⊢ iM2 ≤ iK]]$, and for 
      $[[Γ' ⊢ σ : Γ]]$, $[[ [σ]iM1 = [σ]iM2 ]]$ then $[[Γ' ⊢ [σ]iN ≤ [σ]iK]]$
    \item $+$ if $[[Γ ⊢ iP ≥ iQ1]]$, $[[Γ ⊢ iQ2 ≥ iR]]$, and for
      $[[Γ' ⊢ σ : Γ]]$, $[[ [σ]iQ1 = [σ]iQ2 ]]$ then $[[Γ' ⊢ [σ]iP ≥ [σ]iR]]$
  \end{itemize}
\end{lemma}
\begin{proof}
  We prove it by induction on $\depth{[[Γ ⊢ iN ≤ iM1]]} + \depth{[[Γ ⊢ iM2 ≤ iK]]}$ and mutually, 
  on $\depth{[[Γ ⊢ iP ≥ iQ1]]} + \depth{[[Γ ⊢ iQ2 ≥ iR]]}$.


  First, let us consider the 3 important cases.
  \begin{caseof}
    \item Let us consider the case when $[[iM1 = ∀pbs1.α⁻]]$. 
      Then by \cref{lemma:var-subt},
       $[[Γ ⊢ iN ≤ iM1]]$ means that $[[iN = ∀pas.α⁻]]$. 
      $[[ [σ]iM1 = [σ]iM2 ]]$ means that $[[ ∀pbs1.[σ]α⁻ = [σ]iM2 ]]$.
      Applying $[[σ]]$ to both sides of $[[Γ ⊢ iM2 ≤ iK]]$ (by \cref{lemma:subst-pres-subt}),
      we obtain $[[Γ' ⊢ [σ]iM2 ≤ [σ]iK]]$, that is $[[Γ' ⊢  ∀pbs1.[σ]α⁻ ≤ [σ]iK]]$.
      Since $[[ fv([σ]α⁻) ⊆ {Γ,α⁻} ]]$, it is disjoint from $[[pas]]$ and $[[pbs1]]$,
      This way, by \cref{corollary:red-quant-elim}, 
      $[[Γ' ⊢  ∀pbs1.[σ]α⁻ ≤ [σ]iK]]$ is equivalent to 
      $[[Γ' ⊢  [σ]α⁻ ≤ [σ]iK]]$, which is equivalent to $[[Γ' ⊢  ∀pas.[σ]α⁻ ≤ [σ]iK]]$,
      that is $[[Γ' ⊢  [σ]iN ≤ [σ]iK]]$.
    \item Let us consider the case when $[[iM2 = ∀pbs2.α⁻]]$.
      This case is symmetric to the previous one. Notice that 
      \cref{lemma:var-subt,corollary:red-quant-elim} are agnostic to the 
      side on which the the quantifiers occur, and thus, 
      the proof stays the same. 
    \item Let us decompose the types, by extracting the outer quantifiers:
      \begin{itemize}
        \item $[[iN = ∀pas.iN']]$, where $[[iN']] \neq [[∀]]\dots$,
        \item $[[iM1 = ∀pbs1.iM1']]$, where $[[iM1']] \neq [[∀]]\dots$,
        \item $[[iM2 = ∀pbs2.iM2']]$, where $[[iM2']] \neq [[∀]]\dots$,
        \item $[[iK = ∀pcs.iK']]$, where $[[iK']] \neq [[∀]]\dots$.
      \end{itemize}
      and assume that at least one of $[[pas]]$, $[[pbs1]]$, $[[pbs2]]$, and $[[pcs]]$ is not empty.
      Since $[[ [σ]iM1 = [σ]iM2 ]]$, we have $[[ ∀pbs1.[σ]iM1' = ∀pbs2.[σ]iM2' ]]$,
      and since $[[iMi']]$ are not variables 
      (which was covered by the previous cases) and do not start with $\forall$,
      $[[ [σ]iMi' ]]$ do not start with $\forall$ either,
      which means $[[pbs1]] = [[pbs2]]$ and $[[ [σ]iM1' = [σ]iM2' ]]$.
      Let us rename $[[pbs1]]$ and $[[pbs2]]$ to $[[pbs]]$.
      Then $[[iM1 = ∀pbs.iM1']]$ and $[[iM2 = ∀pbs.iM2']]$.

      By \cref{lemma:quant-rule-decomposition} applied twice
      to $[[Γ ⊢ ∀pas.iN' ≤ ∀pbs.iM1']]$ and to $[[Γ ⊢ ∀pbs.iM2' ≤ ∀pcs.iK']]$,
      we have the following:
      \begin{enumerate}
        \item $[[Γ, pbs ⊢ [iPs/pas]iN' ≤ iM1']]$ for some $[[Γ, pbs ⊢ iPs]]$;
        \item $[[Γ, pcs ⊢ [iQs/pbs]iM2' ≤ iK']]$ for some $[[Γ, pcs ⊢ iQs]]$.
      \end{enumerate}
      And since at least one of 
      $[[pas]]$, $[[pbs]]$, and $[[pcs]]$ is not empty,
      either $[[Γ ⊢ iN ≤ iM1]]$ or $[[Γ ⊢ iM2 ≤ iK]]$ is inferred 
      by \ruleref{\ottdruleDOneForallLabel}, meaning that either 
      $[[Γ, pbs ⊢ [iPs/pas]iN' ≤ iM1']]$ is a proper subderivation of $[[Γ ⊢ iN ≤ iM1]]$ or
      $[[Γ, pcs ⊢ [iQs/pbs]iM2' ≤ iK']]$ is a proper subderivation of $[[Γ ⊢ iM2 ≤ iK]]$.

      Notice that we can weaken and rearrange the contexts without changing the depths of the 
      derivations: $[[Γ, pbs, pcs ⊢ [iPs/pas]iN' ≤ iM1']]$
      and $[[Γ, pbs, pcs ⊢ [iQs/pbs]iM2' ≤ iK']]$. This way, 
      the sum of the depths of these derivations is smaller than the sum of the depths of
      $[[Γ ⊢ iN ≤ iM1]]$ and $[[Γ ⊢ iM2 ≤ iK]]$.
      Let us apply the induction hypothesis to these derivations, 
      with the substitution $[[ Γ', pcs ⊢ σ ○ (iQs/pbs) : Γ, pbs, pcs  ]]$
      (\cref{lemma:subst-domain-weakening}).
      To apply the induction hypothesis, it is left to show that 
      $[[ σ ○ (iQs/pbs) ]]$ unifies $[[iM1']]$ and $[[ [iQs/pbs]iM2']]$:
      $$
      \begin{aligned}[t]
        [[ [σ ○ iQs/pbs]iM1' ]] &= [[ [σ][iQs/pbs]iM1' ]]\\
                                &= [[ [ [σ]iQs/pbs ][σ]iM2' ]]
                                && \text{by \cref{lemma:subst-composition}}\\
                                &= [[ [ [σ]iQs/pbs ][σ]iM2' ]]
                                && \text{Since $[[ [σ]iM1' = [σ]iM2' ]]$}\\
                                &= [[  [σ][iQs/pbs]iM2' ]]
                                && \text{by \cref{lemma:subst-composition}}\\
                                &= [[  [σ][iQs/pbs][iQs/pbs]iM2' ]]
                                && \text{Since $[[Γ, pcs ⊢ iQs]]$, and $[[{(Γ, pcs)} ∩ {pbs} = ∅]]$ }\\
                                &= [[  [σ ○ iQs/pbs][iQs/pbs]iM2' ]]
      \end{aligned}
      $$
      This way the induction hypothesis gives us
      $[[ Γ', pcs ⊢ [σ][iQs/pbs][iPs/pas]iN' ≤  [σ][iQs/pbs]iK' ]]$,
      and since $[[Γ, pcs ⊢ iK']]$, $[[ [iQs/pbs]iK' = iK' ]]$, that is 
      $[[ Γ', pcs ⊢ [σ][iQs/pbs][iPs/pas]iN' ≤  [σ]iK' ]]$.
      Let us rewrite the substitution that we apply to $[[iN']]$:
      $$
      \begin{aligned}[t]
        [[ [σ ○ iQs/pbs ○ iPs/pas]iN' ]] &= [[ [ (σ <=< iQs/pbs) ○ σ ○ iPs/pas]iN' ]]
                                       && \text{by \cref{lemma:subst-composition}}\\
                                       &= [[ [(σ <=< iQs/pbs) ○ (σ <=< iPs/pas) ○ σ] iN' ]]
                                       && \text{by \cref{lemma:subst-composition}}\\
                                       &= [[ [(((σ <=< iQs/pbs) ○ σ) <=< iPs/pas) ○ σ] iN' ]]
                                       && \text{Since $[[fv([σ]iN') ∩ {pbs} = ∅]]$}\\
                                       &= [[ [((σ ○ iQs/pbs) <=< iPs/pas) ○ σ] iN' ]]
                                       && \text{by \cref{lemma:subst-composition}}\\
                                       &= [[ [(σ ○ iQs/pbs) <=< iPs/pas][σ] iN' ]]
      \end{aligned}
      $$
      Notice that $[[(σ ○ iQs/pbs) <=< iPs/pas]]$
      is a substitution that turns $[[pai]]$ into [[ [σ ○ iQs/pbs]iPi ]], 
      where $[[ Γ',pcs ⊢ [σ ○ iQs/pbs]iPi]]$.
      This way, 
      $[[ Γ', pcs ⊢ [(σ ○ iQs/pbs) <=< iPs/pas][σ]iN' ≤  [σ]iK' ]]$
      means $[[Γ ⊢ ∀pas.[σ]iN' ≤ ∀pcs.[σ]iK']]$
      by \cref{lemma:quant-rule-decomposition}, that is
      $[[Γ ⊢ [σ]iN ≤ [σ]iK]]$, as required.
  \end{caseof}

  Now, we can assume that neither $[[Γ ⊢ iN ≤ iM1]]$ nor $[[Γ ⊢ iM2 ≤ iK]]$ 
  is inferred by \ruleref{\ottdruleDOneForallLabel}, and that neither $[[iM1]]$ nor $[[iM2]]$
  is equivalent to a variable.  Because of that, $[[ [σ]iM1 = [σ]iM2 ]]$ means that 
  $[[iM1]]$ and $[[iM2]]$ have the same outer constructor. Let us consider the shape of $[[iM1]]$.

  \begin{caseof}
    \item $[[iM1 = α⁻]]$ this case has been considered;
    \item $[[iM1 = ∀pbs.iM1']]$ this case has been considered;
    \item $[[iM1 = ↑iQ1]]$. Then as noted above, 
      $[[ [σ]iM1 = [σ]iM2 ]]$ means that $[[iM2 = ↑iQ2]]$ and $[[ [σ]iQ1 = [σ]iQ2 ]]$.
      Moreover, $[[Γ ⊢ iN ≤ ↑iQ1]]$ can only be inferred by \ruleref{\ottdruleDOneShiftULabel},
      and thus, $[[iN = ↑iP]]$, and by inversion, $[[Γ ⊢ iP ≥ iQ1]]$ and $[[Γ ⊢ iQ1 ≥ iP]]$.
      Analogously, $[[Γ ⊢ ↑iQ2 ≤ iK]]$ means that $[[iK = ↑iR]]$, $[[Γ ⊢ iQ2 ≥ iR]]$, and $[[Γ ⊢ iR ≥ iQ2]]$.

      Notice that the derivations of $[[Γ ⊢ iP ≥ iQ1]]$ and $[[Γ ⊢ iQ1 ≥ iP]]$ are proper sub-derivations of 
      $[[Γ ⊢ iN ≤ iM1]]$, and the derivations of $[[Γ ⊢ iQ2 ≥ iR]]$ and $[[Γ ⊢ iR ≥ iQ2]]$ are proper sub-derivations of
      $[[Γ ⊢ iM2 ≤ iK]]$. This way, the induction hypothesis is applicable:
      \begin{itemize}
        \item applying the induction hypothesis to $[[Γ ⊢ iP ≥ iQ1]]$ and $[[Γ ⊢ iQ2 ≥ iR]]$ 
          with $[[Γ' ⊢ σ : Γ]]$ unifying $[[iQ1]]$ and $[[iQ2]]$, we obtain $[[Γ' ⊢ [σ]iP ≥ [σ]iR]]$;
        \item applying the induction hypothesis to $[[Γ ⊢ iR ≥ iQ2]]$ and $[[Γ ⊢ iQ1 ≥ iP]]$ 
          with $[[Γ' ⊢ σ : Γ]]$ unifying $[[iQ2]]$ and $[[iQ1]]$, we obtain $[[Γ' ⊢ [σ]iR ≥ [σ]iP]]$.
      \end{itemize}
      This way, by \ruleref{\ottdruleDOneShiftULabel}, $[[Γ' ⊢ [σ]iN ≤ [σ]iK]]$, as required. 

    \item $[[iM1 = iQ1 → iM1']]$. Then as noted above, 
      $[[ [σ]iM1 = [σ]iM2 ]]$ means that $[[iM2 = iQ2 → iM2']]$, $[[ [σ]iQ1 = [σ]iQ2 ]]$, and $[[ [σ]iM1' = [σ]iM2' ]]$.
      Moreover, $[[Γ ⊢ iN ≤ iQ1 → iM1']]$ can only be inferred by \ruleref{\ottdruleDOneArrowLabel},
      and thus, $[[iN = iP → iN']]$, and by inversion, $[[Γ ⊢ iP ≥ iQ1]]$ and $[[Γ ⊢ iN' ≤ iM1']]$.
      Analogously, $[[Γ ⊢ iQ2 → iM2' ≤ iK]]$ means that $[[iK = iR → iK']]$, $[[Γ ⊢ iQ2 ≥ iR]]$, and $[[Γ ⊢ iM2' ≤ iK']]$.

      Notice that the derivations of $[[Γ ⊢ iP ≥ iQ1]]$ and $[[Γ ⊢ iN' ≤ iM1']]$ are proper sub-derivations of
      $[[Γ ⊢ iP → iN' ≤ iQ1 → iM1']]$, and the derivations of $[[Γ ⊢ iQ2 ≥ iR]]$ and $[[Γ ⊢ iM2' ≤ iK']]$ are proper sub-derivations of
      $[[Γ ⊢ iQ2 → iM2' ≤ iR → iK']]$. This way, the induction hypothesis is applicable:
      \begin{itemize}
        \item applying the induction hypothesis to $[[Γ ⊢ iP ≥ iQ1]]$ and $[[Γ ⊢ iQ2 ≥ iR]]$ 
          with $[[Γ' ⊢ σ : Γ]]$ unifying $[[iQ1]]$ and $[[iQ2]]$, we obtain $[[Γ' ⊢ [σ]iP ≥ [σ]iR]]$;
        \item applying the induction hypothesis to $[[Γ ⊢ iN' ≤ iM1']]$ and $[[Γ ⊢ iM2' ≤ iK']]$ 
          with $[[Γ' ⊢ σ : Γ]]$ unifying $[[iM1']]$ and $[[iM2']]$, we obtain $[[Γ' ⊢ [σ]iN' ≤ [σ]iK']]$.
      \end{itemize}
      This way, by \ruleref{\ottdruleDOneArrowLabel}, $[[Γ' ⊢ [σ]iP → [σ]iN' ≤ [σ]iR → [σ]iK']]$,
      that is $[[Γ' ⊢ [σ]iN ≤ [σ]iK]]$, as required.
  \end{caseof}

  After that we consider all the 
  analogous positive cases, and prove them symmetrically.
\end{proof}

\begin{corollary}[Transitivity of subtyping] \label{corollary:subtyping-transitivity}
  Assuming the types are well-formed in $[[Γ]]$,
  \begin{itemize}
    \item[$-$] if $[[Γ ⊢ iN1 ≤ iN2]]$ and $[[Γ ⊢ iN2 ≤ iN3]]$ then $[[Γ ⊢ iN1 ≤ iN3]]$,
    \item[$+$] if $[[Γ ⊢ iP1 ≥ iP2]]$ and $[[Γ ⊢ iP2 ≥ iP3]]$ then $[[Γ ⊢ iP1 ≥ iP3]]$.
  \end{itemize}
\end{corollary}

\begin{corollary}[Transitivity of equivalence] \label{corollary:equivalence-transitivity}
  Assuming the types are well-formed in $[[Γ]]$,
  \begin{itemize}
    \item[$-$] if $[[Γ ⊢ iN1 ≈ iN2]]$ and $[[Γ ⊢ iN2 ≈ iN3]]$ then $[[Γ ⊢ iN1 ≈ iN3]]$,
    \item[$+$] if $[[Γ ⊢ iP1 ≈ iP2]]$ and $[[Γ ⊢ iP2 ≈ iP3]]$ then $[[Γ ⊢ iP1 ≈ iP3]]$.
  \end{itemize}
\end{corollary}









