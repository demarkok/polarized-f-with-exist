\begin{lemma}[Free Variable Propagation] \label{lemma:fv-propagation}
  In the judgments of negative subtyping or positive supertyping,
  free variables propagate left-to-right. For a context $[[Γ]]$,
  \begin{itemize}
    \item $-$ if $[[Γ ⊢ iN ≤ iM]]$ then $[[fv(iN)]] \subseteq [[fv(iM)]]$
    \item $+$ if $[[Γ ⊢ iP ≥ iQ]]$ then $[[fv(iP)]] \subseteq [[fv(iQ)]]$
  \end{itemize}
\end{lemma}
\begin{proof}
  Mutual induction on $[[Γ ⊢ iN ≤ iM]]$ and $[[Γ ⊢ iP ≥ iQ]]$.
  \begin{caseof}
  \item $[[G ⊢ a⁻ ≤ a⁻]]$\\
    It is self-evident that $[[{a⁻} ⊆ {a⁻}]]$.
  \item $[[G ⊢ ↑iP ≤ ↑iQ]]$
    From the inversion (and unfolding $[[G ⊢ iP ≈ iQ]]$ ), we have
    $[[G ⊢ iP ≥ iQ]]$. Then by the induction hypothesis,
    $[[fv(iP)]] \subseteq [[fv(iQ)]]$. The desired inclusion
    inclusion holds, since $[[fv(↑iP)]] = [[fv(iP)]]$ and
    $[[fv(↑iQ)]] = [[fv(iQ)]]$.
  \item $[[G ⊢ iP → iN ≤ iQ → iM]]$
    The induction hypothesis applied to the premises gives:
    $[[fv(iP)]] \subseteq [[fv(iQ)]]$ and
    $[[fv(iN)]] \subseteq [[fv(iM)]]$.
    Then $[[fv(iP → iN)]] = [[fv(iP) ∪ fv(iN)]] \subseteq
    [[fv(iQ) ∪ fv(iM)]] = [[fv(iQ → iM)]]$.

  \item $[[G ⊢ ∀pas.iN ≤ ∀pbs.iM]]$\\
    $
    \begin{aligned}[t]
      [[fv ∀pas.iN ]] &\subseteq [[fv ([iPs/pas] iN) ]] ~\setminus~ [[{pbs}]] 
                      &&   \text{here $[[{pbs}]]$ is excluded by the premise $[[fv iN ∩ {pbs} = ∅]]$}\\
                      &\subseteq [[fv iM]] ~\setminus~ [[{pbs}]]
                      &&   \text{by the induction hypothesis, } [[fv ([iPs/pas] iN) ]] \subseteq [[fv iM]] \\
                      &\subseteq [[fv ∀pbs.iM]]
    \end{aligned}
    $
  \item The positive cases are symmetric.
  \end{caseof}
\end{proof}

\begin{corollary}[Free Variables of mutual subtypes] \label{corollary:fv-mut-sub}
  \hfill
  \begin{itemize}
    \item [$-$] If $[[Γ ⊢ iN ≈ iM]]$ then $[[fv iN]] = [[fv iM]]$, 
    \item [$+$] If $[[Γ ⊢ iP ≈ iQ]]$ then $[[fv iP]] = [[fv iQ]]$
  \end{itemize}
\end{corollary}

\begin{lemma}[Subtypes and supertypes of a variable]
  \label{lemma:var-subt}
  Assuming $[[Γ ⊢  α⁻]]$, $[[Γ ⊢ α⁺]]$, $[[Γ ⊢ iN]]$, and $[[Γ ⊢ iP]]$,
  \begin{itemize}
  \item[$+$] if $[[Γ ⊢ iP ≥ α⁺]]$ or $[[Γ ⊢ α⁺ ≥ iP ]]$ then $[[iP]] = [[∃nas.α⁺]]$ (for some potentially empty $[[nas]]$)
  \item[$-$] if $[[Γ ⊢ iN ≤ α⁻]]$ or $[[Γ ⊢ α⁻ ≤ iN ]]$ then $[[iN]] = [[∀pas.α⁻]]$ (for some potentially empty $[[pas]]$)
  \end{itemize}
\end{lemma}
\begin{proof}
  We prove by induction on the tree
  inferring $[[Γ ⊢ iP ≥ α⁺]]$ or $[[Γ ⊢ α⁺ ≥ iP ]]$ or
  or $[[Γ ⊢ iN ≤ α⁻]]$ or $[[Γ ⊢ α⁻ ≤ iN ]]$.

  Let us consider which of these judgments the tree is inferring.
  \begin{caseof}
  \item $[[Γ ⊢ iP ≥ α⁺]]$\\
    If the size of the inference tree is $1$ then the only rule that can infer
    it is \ruleref{\ottdruleDOnePVarLabel}, which
    implies that $[[iP = α⁺]]$.

    If the size of the inference tree is $>1$ then the last rule inferring
    it must be \ruleref{\ottdruleDOneExistsLabel}. By inverting this rule,
    $[[iP = ∃nas.iP']]$ where $[[iP']]$ does not start with $\exists$ and
    $[[G ⊢ [iNs/nas] iP' ≥ α⁺]]$ for some $[[G, nbs ⊢ iNi]]$.

    By the induction hypothesis, $[[ [iNs/nas] iP' = ∃nbs.α⁺]]$.
    Notice that $[[iP']]$ must be a variable, because
    $[[iP']]$ does not start with $\exists$, nor does it start with
    $\uparrow$ (otherwise, $[[ [iNs/nas] iP' ]]$ would also
    started with $\uparrow$ and would not be equal to $[[∃nbs.α⁺]]$).
    Since $[[iP']]$ is a \emph{positive} variable, $[[ [iNs/nas] iP' = iP']]$,
    and then $[[iP' = ∃nbs.α⁺]]$ means that $[[iP' = α⁺]]$.
    This way, $[[iP]] = [[∃nas.iP']] = [[∃nas.α⁺]]$, as required.

  \item $[[Γ ⊢ α⁺ ≥ iP]]$\\
    If the size of the inference tree is $1$ then the only rule that can infer
    it is \ruleref{\ottdruleDOnePVarLabel}, which
    implies that $[[iP = α⁺]]$.

    If the size of the inference tree is $>1$ then the last rule inferring
    it must be \ruleref{\ottdruleDOneExistsLabel}. By inverting this rule,
    $[[iP = ∃nbs.iQ]]$ where and $[[G, nbs ⊢ α⁺ ≥ iQ]]$.

    By the induction hypothesis, $[[iQ = ∃nbs'.α⁺]]$.
    This way, $[[iP]] = [[∃nbs.iQ]] = [[∃nbs.∃nbs'.α⁺]]$, as required.

  \item The negative cases ($[[Γ ⊢ iN ≤ α⁻]]$ and $[[Γ ⊢ α⁻ ≤ iN ]]$)
    are proved analogously.
  \end{caseof}
\end{proof}

\begin{corollary}[Variables have no proper subtypes and supertypes]
  \label{corollary:vars-no-proper-subtypes}
  Assuming that all mentioned types are well-formed in $[[Γ]]$,
  \begin{align*}
    [[Γ ⊢ iP ≥ α⁺]] ~ &\iff ~ [[iP = ∃nbs.α⁺]]  ~ \iff ~ [[Γ ⊢ iP ≈ α⁺]] ~ \iff ~ [[iP ≈ α⁺]]\\
    [[Γ ⊢ α⁺≥ iP]]  ~ &\iff ~ [[iP = ∃nbs.α⁺]]  ~ \iff ~ [[Γ ⊢ iP ≈ α⁺]] ~ \iff ~ [[iP ≈ α⁺]]\\
    [[Γ ⊢ iN ≤ α⁻]] ~ &\iff ~ [[iN = ∀pbs.α⁻]]  ~ \iff ~ [[Γ ⊢ iN ≈ α⁻]] ~ \iff ~ [[iN ≈ α⁻]]\\
    [[Γ ⊢ α⁻ ≤ iN]] ~ &\iff ~ [[iN = ∀pbs.α⁻]]  ~ \iff ~ [[Γ ⊢ iN ≈ α⁻]] ~ \iff ~ [[iN ≈ α⁻]]\\
  \end{align*}
\end{corollary}
\begin{proof}
  Notice that $[[Γ ⊢ ∃nas.α⁺ ≈ α⁺]]$ and $[[∃nas.α⁺ ≈ α⁺]]$ and apply
  \cref{lemma:var-subt}.
  \ilyam{fix}
\end{proof}

\begin{lemma}[Reflexivity of subtyping] \label{lemma:subtyping-reflexivity}
  Assuming all the types are well-formed in $[[Γ]]$,
  \begin{itemize}
    \item [$-$] $[[Γ ⊢ iN ≤ iN]]$
    \item [$+$] $[[Γ ⊢ iP ≥ iP]]$
  \end{itemize}
\end{lemma}
\begin{proof}
  Let us prove it by the size of $[[iN]]$ and mutually, $[[iP]]$.
  \begin{caseof}
    \item $[[iN]] = [[α⁻]]$\\
      Then $[[Γ ⊢ α⁻ ≤ α⁻]]$ is inferred immediately by \ruleref{\ottdruleDOneNVarLabel}.
    \item $[[iN]] = [[∀pas.iN']]$ where $[[pas]]$ is not empty\\
      First, we rename $[[pas]]$ to fresh $[[pbs]]$ in $[[∀pas.iN']]$ to avoid
      name clashes: $[[∀pas.iN']] = [[∀pbs.[pas/pbs]iN']]$.
      Then to infer $[[Γ ⊢ ∀pas.iN' ≤ ∀pbs.[pas/pbs]iN']]$ we can apply 
      \ruleref{\ottdruleDOneForallLabel}, instantiating $[[pas]]$ with $[[pbs]]$:
      \begin{itemize}
        \item $[[fv iN ∩ {pbs} = ∅ ]]$ by choice of $[[pbs]]$,
        \item $[[G, pbs ⊢ pbi]]$,
        \item $[[G, pbs ⊢ [pbs/pas] iN' ≤ [pbs/pas] iN']]$ by the induction hypothesis,
        since the size of $[[ [pbs/pas]iN' ]]$ is equal to the size of $[[iN']]$,
        which is smaller than the size of $[[iN]] = [[∀pas.iN']]$.
      \end{itemize}
    \item $[[iN]] = [[iP → iM]]$\\
      Then $[[Γ ⊢ iP → iM ≤ iP → iM]]$ is inferred by \ruleref{\ottdruleDOneArrowLabel},
      since $[[Γ ⊢ iP ≥ iP]]$ and $[[Γ ⊢ iM ≤ iM]]$ hold the induction hypothesis. 
    \item $[[iN]] = [[↑iP]]$\\
      Then $[[Γ ⊢ ↑iP ≤ ↑iP]]$ is inferred by \ruleref{\ottdruleDOneShiftULabel},
      since $[[Γ ⊢ iP ≥ iP]]$ holds by the induction hypothesis.
    \item The positive cases are symmetric to the negative ones.
  \end{caseof}
\end{proof}

\begin{lemma}[Substitution preserves subtyipng]
  \label{lemma:subst-pres-subt}
  Assuming that all mentioned types are well-formed in $[[Γ]]$,
  and $[[Γ' ⊢ σ : Γ]]$, where $[[Γ']]$ is disjoint from $[[Γ]]$,
  \begin{itemize}
    \item $-$ If $[[Γ ⊢ iN ≤ iM]]$ then $[[Γ' ⊢ [σ]iN ≤ [σ]iM]]$
    \item $+$ If $[[Γ ⊢ iP ≥ iQ]]$ then $[[Γ' ⊢ [σ]iP ≥ [σ]iQ]]$
  \end{itemize}
\end{lemma}
\begin{proof}
  We prove it by induction on the size of the derivation of $[[Γ ⊢ iN ≤ iM]]$
  and mutually, $[[Γ ⊢ iP ≥ iQ]]$. Let us consider the last rule 
  used in the derivation:
  \begin{caseof}
    \item \ruleref{\ottdruleDOneNVarLabel}. Then by inversion, 
      $[[iN = α⁻]]$ and $[[iM = α⁻]]$. By reflexivity of subtyping
      (\cref{lemma:subtyping-reflexivity}),
      we have $[[Γ' ⊢ [σ]α⁻ ≤ [σ]α⁻]]$, i.e. $[[Γ' ⊢ [σ]iN ≤ [σ]iM]]$,
      as required.
    \item  \ruleref{\ottdruleDOneForallLabel}. Then by inversion,
      $[[iN = ∀pas.iN']]$, $[[iM = ∀pbs.iM']]$, where $[[pas]]$ or $[[pbs]]$ is not empty.
      Moreover, $[[Γ,pbs ⊢ [iPs/pas]iN' ≤ iM']]$ for some $[[Γ, pbs ⊢ iPs]]$, and 
      $[[fv iN ∩ {pbs} = ∅ ]]$.

      Notice that since the derivation of $[[Γ,pbs ⊢ [iPs/pas]iN' ≤ iM']]$ is
      a subderivation of the derivation of $[[Γ ⊢ iN ≤ iM]]$, its size is smaller, 
      and hence, the induction hypothesis applies:
      $[[Γ', pbs ⊢ [σ][iPs/pas]iN' ≤ [σ]iM']]$.

      First, let us assume that $[[{pas} ∩ {Γ'} = ∅]]$ and $[[{pbs} ∩ {Γ'} = ∅]]$
      (otherwise, we rename $[[pas]]$ and $[[pbs]]$ to fresh $[[pas']]$ and $[[pbs']]$).
      Then $[[ [σ]∀pas.iN' ]] = [[ ∀pas.[σ]iN' ]]$ and $[[ [σ]∀pbs.iM' ]] = [[ ∀pbs.[σ]iM' ]]$, 
      which means that the required $[[Γ' ⊢ [σ]∀pas.iN' ≤ [σ]∀pbs.iM']]$ is rewritten as
      $[[Γ' ⊢ ∀pas.[σ]iN' ≤ ∀pbs.[σ]iM']]$.

      To infer it, we apply \ruleref{\ottdruleDOneForallLabel}, 
      instantiating $[[pai]]$ with $[[ [σ]iPi ]]$:
      \begin{itemize}
        \item $[[fv iN ∩ {pbs} = ∅ ]]$ as noted before, from the inversion;
        \item $[[Γ', pbs ⊢ [σ]iPi]]$, by \cref{lemma:wf-subst} since from the inversion,
        $[[Γ, pbs ⊢ iPi]]$;
        \item $[[Γ, pbs ⊢ [ [σ]iPs/pas ][σ]iN' ≤ [σ]iM']]$ holds because 
        $[[ [ [σ]iPs/pas ][σ]iN' ]] = [[ [σ][iPs/pas]iN ]]$ (since $[[{pas} ∩ {Γ} = ∅]]$), 
        and $[[Γ', pbs ⊢ [σ][iPs/pas]iN' ≤ [σ]iM']]$ holds by the induction hypothesis.
      \end{itemize}
    \item \ruleref{\ottdruleDOneArrowLabel}. Then by inversion,
      $[[iN = iP → iN1]]$, $[[iM = iQ → iM1]]$, $[[Γ ⊢ iP ≥ iQ]]$, and $[[Γ ⊢ iN1 ≤ iM1]]$.
      And by the induction hypothesis, $[[Γ' ⊢ [σ]iP ≥ [σ]iQ]]$ and $[[Γ' ⊢ [σ]iN1 ≤ [σ]iM1]]$.
      Then $[[Γ' ⊢ [σ]iN ≤ [σ]iM]]$, i.e. $[[Γ' ⊢ [σ]iP → [σ]iN1 ≤ [σ]iQ → [σ]iM1]]$,
      is inferred by \ruleref{\ottdruleDOneArrowLabel}.
    \item \ruleref{\ottdruleDOneShiftULabel}. Then by inversion,
      $[[iN = ↑iP]]$, $[[iM = ↑iQ]]$, and $[[Γ ⊢ iP ≈ iQ]]$,
      which by inversion means that $[[Γ ⊢ iP ≥ iQ]]$ and $[[Γ ⊢ iQ ≥ iP]]$.
      Then the induction hypothesis applies, and we have $[[Γ' ⊢ [σ]iP ≥ [σ]iQ]]$
      and $[[Γ' ⊢ [σ]iQ ≥ [σ]iP]]$. 
      Then by sequential application of \ruleref{\ottdruleDOneNDefLabel} 
      and \ruleref{\ottdruleDOneShiftULabel} to these judgments,
      we have $[[Γ' ⊢ ↑[σ]iP ≤ ↑[σ]iQ]]$, i.e.
      $[[Γ' ⊢ [σ]iN ≤ [σ]iM]]$, as required.
    \item The positive cases are proved symmetrically.
  \end{caseof}
\end{proof}



\begin{lemma}[Strong transitivity of subtyping] \label{lemma:subtyping-transitivity}
  Assuming all the types are well-formed in $[[Γ]]$,
  \begin{itemize}
    \item $-$ if $[[Γ ⊢ iN ≤ iM1]]$, $[[Γ ⊢ iM2 ≤ iK]]$, and for 
    $[[Γ' ⊢ σ : Γ]]$, $[[ [σ]iM1 = [σ]iM2 ]]$ then $[[Γ' ⊢ [σ]iN ≤ [σ]iK]]$
    \item $+$ if $[[Γ ⊢ iP ≥ iQ1]]$, $[[Γ ⊢ iQ2 ≥ iR]]$, and for
    $[[Γ' ⊢ σ : Γ]]$, $[[ [σ]iQ1 = [σ]iQ2 ]]$ then $[[Γ' ⊢ [σ]iP ≥ [σ]iR]]$
  \end{itemize}
\end{lemma}
\begin{proof}
  We prove it by induction on $\depth{[[Γ ⊢ iN ≤ iM1]]} + \depth{[[Γ ⊢ iM2 ≤ iK]]}$ and mutually, 
  on $\depth{[[Γ ⊢ iP ≥ iQ1]]} + \depth{[[Γ ⊢ iQ2 ≥ iR]]}$.

  \begin{caseof}
    \item Firs, let us consider the case when the last rule applied to infer 
    $[[Γ ⊢ iN ≤ iM1]]$ is \ruleref{\ottdruleDOneNVarLabel}.
    Notice that this case covers the base of the induction: 
    the sum of the depths is minimal when both derivations are inferred by
    the non-recursive rules (i.e. \ruleref{\ottdruleDOneNVarLabel}).

    By inverting the rule, $[[iN = α⁻]]$ and $[[iM1 = α⁻]]$
    Then $[[ [σ]iN ]] = [[ [σ]α⁻ ]] = [[ [σ]iM1 ]] = [[ [σ]iM2 ]]$. 
    And $[[Γ' ⊢ [σ]iM2 ≤ [σ]iK]]$ by \cref{}





    
  \end{caseof}



  
\end{proof}

\begin{corollary}[Transitivity of subtyping] \label{corollary:subtyping-transitivity}
  Assuming the types are well-formed in $[[Γ]]$,
  \begin{itemize}
    \item[$-$] if $[[Γ ⊢ iN1 ≤ iN2]]$ and $[[Γ ⊢ iN2 ≤ iN3]]$ then $[[Γ ⊢ iN1 ≤ iN3]]$,
    \item[$+$] if $[[Γ ⊢ iP1 ≥ iP2]]$ and $[[Γ ⊢ iP2 ≥ iP3]]$ then $[[Γ ⊢ iP1 ≥ iP3]]$.
  \end{itemize}
\end{corollary}


\begin{corollary}[Transitivity of equivalence] \label{corollary:equivalence-transitivity}
  Assuming the types are well-formed in $[[Γ]]$,
  \begin{itemize}
    \item[$-$] if $[[Γ ⊢ iN1 ≈ iN2]]$ and $[[Γ ⊢ iN2 ≈ iN3]]$ then $[[Γ ⊢ iN1 ≈ iN3]]$,
    \item[$+$] if $[[Γ ⊢ iP1 ≈ iP2]]$ and $[[Γ ⊢ iP2 ≈ iP3]]$ then $[[Γ ⊢ iP1 ≈ iP3]]$.
  \end{itemize}
\end{corollary}








