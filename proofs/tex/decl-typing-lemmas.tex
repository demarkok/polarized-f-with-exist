\begin{lemma} \label{lemma:app-inf-equ-stable}
    If $[[Γ; Φ ⊢ iN1 ● args ⇒> iM]]$ and $[[Γ ⊢ iN1 ≈ iN2]]$ 
    then $[[Γ; Φ ⊢ iN2 ● args ⇒> iM]]$.
\end{lemma}
\begin{proof}
    By \cref{lemma:equiv-completeness}, 
    $[[Γ ⊢ iN1 ≈ iN2]]$ implies $[[iN1 ≈ iN2]]$.
    Let us prove the required judgement by induction on $[[iN1 ≈ iN2]]$.
    Let us consider the last rule used in the derivation.
    \begin{caseof}
        \item \ruleref{\ottdruleEOneNVarLabel}.
            It means that $[[iN1]]$ is $[[α⁻]]$ and $[[iN2]]$ is $[[α⁻]]$.
            Then the required property coincides with the assumption. 
        \item \ruleref{\ottdruleEOneShiftULabel}. 
            It means that $[[iN1]]$ is $[[↑iP1]]$ and $[[iN2]]$ is $[[↑iP2]]$.
            where $[[iP1 ≈ iP2]]$.

            Then the only rule applicable to infer $[[Γ; Φ ⊢ ↑iP1 ● args ⇒> iM]]$
            is \ruleref{\ottdruleDTEmptyAppLabel},
            meaning that $[[args = ·]]$ and $[[Γ ⊢ ↑iP1 ≈ iM]]$.
            Then by transitivity of equivalence \cref{corollary:equivalence-transitivity},
            $[[Γ ⊢ ↑iP2 ≈ iM]]$, and then \ruleref{\ottdruleDTEmptyAppLabel} is applicable to infer
            $[[Γ; Φ ⊢ ↑iP2 ● · ⇒> iM]]$.
        
        \item \ruleref{\ottdruleEOneArrowLabel}.
            Then we are proving that  
            $[[Γ; Φ ⊢ (iQ1 → iN1) ● v, args ⇒> iM]]$ and $[[iQ1 → iN1 ≈ iQ2 → iN2]]$
            imply $[[Γ; Φ ⊢ (iQ2 → iN2) ● v, args ⇒> iM]]$.
            
            By inversion, $[[(iQ1 → iN1) ≈ (iQ2 → iN2)]]$
            means $[[iQ1 ≈ iQ2]]$ and $[[iN1 ≈ iN2]]$.

            By inversion of $[[Γ; Φ ⊢ (iQ1 → iN1) ● v, args ⇒> iM]]$:
            \begin{enumerate}
                \item $[[Γ ; Φ ⊢ v : iP]]$
                \item $[[Γ ⊢ iQ1 ≥ iP]]$,
                    and then by transitivity \cref{corollary:subtyping-transitivity},
                    $[[Γ ⊢ iQ2 ≥ iP]]$;
                \item $[[Γ ; Φ ⊢ iN1 ● args ⇒> iM]]$, 
                    and then by induction hypothesis, $[[Γ ; Φ ⊢ iN2 ● args ⇒> iM]]$.
            \end{enumerate}

            Since we have $[[Γ ; Φ ⊢ v : iP]]$, $[[Γ ⊢ iQ2 ≥ iP]]$ and 
            $[[Γ ; Φ ⊢ iN2 ● args ⇒> iM]]$, we can apply \ruleref{\ottdruleDTArrowAppLabel}
            to infer $[[Γ; Φ ⊢ (iQ2 → iN2) ● v, args ⇒> iM]]$.

        \item \ruleref{\ottdruleEOneForallLabel}
            Then we are proving that 
            $[[Γ ; Φ ⊢ ∀pas1.iN1' ● args ⇒> iM]]$ and $[[∀pas1.iN1' ≈ ∀pas2.iN2']]$
            imply $[[Γ ; Φ ⊢ ∀pas2.iN2' ● args ⇒> iM]]$.


            By inversion of $[[∀pas1.iN1' ≈ ∀pas2.iN2']]$:
            \begin{enumerate}
                \item $[[{pas2} ∩ fv iN1 = ∅]]$,
                \item there exists a bijection 
                    $[[mu : ({pas2} ∩ fv iN2') ↔ ({pas1} ∩ fv iN1')]]$
                    such that $[[iN1' ≈ [mu] iN2']]$.
            \end{enumerate}

            By inversion of $[[Γ ; Φ ⊢ ∀pas1.iN1' ● args ⇒> iM]]$:
            \begin{enumerate}
                \item $[[Γ ⊢ σ : pas1]]$        
                \item $[[Γ ; Φ ⊢ [σ]iN1' ● args ⇒> iM]]$
                \item $[[args ≠ ·]]$
            \end{enumerate}

            Let us construct $[[Γ ⊢ σ0 : pas2]]$ in the following way:
            $$
            \begin{cases}
                [[ [σ0]α⁺ =  [σ][mu]α⁺ ]] & \text{if } [[α⁺]] \in [[ {pas2} ∩ fv iN2' ]] \\
                [[ [σ0]α⁺ =  ∃β⁻.↓β⁻ ]] & \text{otherwise (the type does not matter here)} \\
            \end{cases}
            $$

            Then to infer $[[Γ ; Φ ⊢ iN2 ● args ⇒> iM]]$, we 
            apply \ruleref{\ottdruleDTArrowAppLabel} with $[[σ0]]$. 
            Let us show the required premises:
            \begin{enumerate}
                \item $[[Γ ⊢ σ0 : pas2]]$ by construction;
                \item $[[args ≠ ·]]$ as noted above;
                \item To show $[[Γ ; Φ ⊢ [σ0]iN2' ● args ⇒> iM]]$,
                Notice that $[[ [σ0]iN2' = [σ][mu]iN2' ]]$   
                and since $[[ [mu]iN2' ≈ iN1' ]]$, $[[ [σ][mu]iN2' ≈ [σ]iN1' ]]$.
                This way, by \cref{lemma:equiv-soundness}, $[[Γ ⊢ [σ]iN1' ≈ [σ0]iN2']]$.
                Then the required judgement holds by the induction hypothesis
                applied to $[[Γ ; Φ ⊢ [σ]iN1' ● args ⇒> iM]]$.
            \end{enumerate}
    \end{caseof}
\end{proof}

\newcommand{\pureSize}[1]{\ensuremath{\mathsf{pure\_size}(#1)}}
\newcommand{\metric}[1]{\ensuremath{\mathsf{metric}(#1)}}
\newcommand{\eqNodes}[1]{\ensuremath{\mathsf{eq\_nodes}(#1)}}
\newcommand{\size}[1]{\ensuremath{\mathsf{size}(#1)}}
\newcommand{\npq}[1]{\ensuremath{\mathsf{npq}(#1)}}

\begin{definition}[Number of prenex quantifiers]
    Let us define $\npq{[[iN]]}$ and $\npq{[[iP]]}$ as the number of prenex quantifiers in these types, i.e.
    \begin{itemize}
        \item [$+$] $\npq{[[∃nas.iP]]} = |nas|$, if $[[iP ≠ ∃nbs.iP']]$,
        \item [$-$] $\npq{[[∀pas.iN]]} = |pas|$, if $[[iN ≠ ∀pbs.iN']]$.
    \end{itemize}
\end{definition}

\begin{definition}[Size of a Judgement]
    \label{def:decl-typing-size}
    For a declarative typing judgement $J$
    let us define a metrics $\size{J}$ as a pair of numbers 
    in the following way:
    \begin{itemize}
        \item [$+$] $\size{[[Γ ; Φ ⊢ v : iP]]} = (\size{[[v]]}, 0)$;
        \item [$-$] $\size{[[Γ ; Φ ⊢ c : iN]]} = (\size{[[c]]}, 0)$;
        \item [$\bullet$] $\size{[[Γ ; Φ ⊢ iN ● args ⇒> iM]]} = 
            (\size{[[args]]}, \npq{[[iN]]})$)
    \end{itemize}
    where $\size{[[v]]}$ or $\size{[[c]]}$ is the size of the 
    syntax tree of the term $[[v]]$ or $[[c]]$
    and $\size{[[args]]}$ is the sum of sizes of the terms in $[[args]]$.
\end{definition}


\begin{definition}[Number of Equivalence Nodes]
    For a tree $T$ inferring
    a declarative typing judgement,
    let us a function $\eqNodes{T}$
    as the number of nodes in $T$ labeled with \ruleref{\ottdruleDTPEquivLabel} or 
    \ruleref{\ottdruleDTNEquivLabel}.
\end{definition}

\begin{definition}[Metric]
    For a tree $T$ inferring
    a declarative typing judgement $J$,
    let us define a metric $\metric{T}$
    as a pair $(\size{J}, \eqNodes{T})$.
\end{definition}

\begin{lemma}[Declarative typing is preserved under context equivalence]
    Assuming $[[Γ ⊢ Φ1]]$, $[[Γ ⊢ Φ2]]$, and $[[Γ ⊢ Φ1 ≈ Φ2]]$:
    \begin{itemize}
        \item [$+$] 
            for any tree $T_1$ inferring $[[Γ ; Φ1 ⊢ v : iP]]$, 
            there exists a tree $T_2$ inferring $[[Γ ; Φ2 ⊢ v : iP]]$.
        \item [$-$] 
            for any tree $T_1$ inferring $[[Γ ; Φ1 ⊢ c : iN]]$, 
            there exists a tree $T_2$ inferring $[[Γ ; Φ2 ⊢ c : iN]]$.
        \item [$\bullet$] 
            for any tree $T_1$ inferring $[[Γ ; Φ1 ⊢ iN ● args ⇒> iM]]$, 
            there exists a tree $T_2$ inferring $[[Γ ; Φ2 ⊢ iN ● args ⇒> iM]]$.
    \end{itemize}
\end{lemma}
\begin{proof}
    Let us prove it by induction on the $\metric{T_1}$.
    Let us consider the last rule applied in $T_1$ (i.e., its root node).
    \begin{caseof}
        \item \ruleref{\ottdruleDTVarLabel}\\
            Then we are proving 
            that $[[Γ ; Φ1 ⊢ x : iP]]$ implies $[[Γ ; Φ2 ⊢ x : iP]]$.
            By inversion, $[[x : iP ∊ Φ1]]$, and 
            since $[[Γ ⊢ Φ1 ≈ Φ2]]$, $[[x : iP' ∊ Φ2]]$ for some $[[iP']]$ 
            such that $[[Γ ⊢ iP ≈ iP']]$.
            Then we infer $[[Γ ; Φ2 ⊢ x : iP']]$ by \ruleref{\ottdruleDTVarLabel},
            and next, $[[Γ ; Φ2 ⊢ x : iP]]$ by \ruleref{\ottdruleDTPEquivLabel}.

        \item For \ruleref{\ottdruleDTThunkLabel},
              \ruleref{\ottdruleDTPAnnotLabel}, 
              \ruleref{\ottdruleDTTLamLabel},
              \ruleref{\ottdruleDTReturnLabel}, and
              \ruleref{\ottdruleDTNAnnotLabel}
              the proof is analogous. We
              apply the induction hypothesis to the premise of the rule
              to substitute $[[Φ1]]$ for $[[Φ2]]$ in it. 
              The induction is applicable because 
              the metric of the 
              premises is less than the metric of the conclusion:
              the term in the premise is a syntactic subterm of the
              term in the conclusion.

              And after that, we apply the same rule to infer the required judgement.
              
        \item \ruleref{\ottdruleDTPEquivLabel} and \ruleref{\ottdruleDTNEquivLabel}
            In these cases, the induction hypothesis is also applicable to the premise:
            although the first component of the metric 
            is the same for the premise and the conclusion:
            $\size{[[Γ ; Φ ⊢ c : iN']]} = \size{[[Γ ; Φ ⊢ c : iN]]} = \size{[[c]]}$,
            the second component of the metric is less for the premise by one,
            since the equivalence rule was applied to turn the premise tree into
            $T1$.
            Having made this note, we continue the proof in the same way as in the previous case.

        \item \ruleref{\ottdruleDTtLamLabel}
            Then we are proving that 
            $[[Γ ; Φ1 ⊢ λx:iP.c : iP → iN]]$ implies $[[Γ ; Φ2 ⊢ λx:iP.c : iP → iN]]$.
            Analogously to the previous cases, 
            we apply the induction hypothesis to the
            equivalent contexts $[[Γ ⊢ Φ1, x:iP ≈ Φ2, x:iP]]$
            and the premise $[[Γ ; Φ1, x:iP ⊢ c : iN]]$
            to obtain $[[Γ ; Φ2, x:iP ⊢ c : iN]]$.
            Notice that $[[c]]$ is a subterm of $[[λx:iP.c]]$,
            i.e., the metric of the premise tree is less than the metric of the conclusion, 
            and the induction hypothesis is applicable.
            Then we infer $[[Γ ; Φ2 ⊢ λx:iP.c : iP → iN]]$ by \ruleref{\ottdruleDTtLamLabel}.

        \item \ruleref{\ottdruleDTVarLetLabel}
            Then we are proving that 
            $[[Γ ; Φ1 ⊢ let x = v; c : iN]]$ implies $[[Γ ; Φ2 ⊢ let x = v; c : iN]]$.
            First, we apply the induction hypothesis to 
            $[[Γ; Φ1 ⊢ v : iP]]$ to obtain $[[Γ; Φ2 ⊢ v : iP]]$ 
            of the same pure size.
            
            Then we apply the induction hypothesis to
            the equivalent contexts $[[Γ ⊢ Φ1, x:iP ≈ Φ2, x:iP]]$
            and the premise $[[Γ ; Φ1, x:iP ⊢ c : iN]]$ to obtain
            $[[Γ ; Φ2, x:iP ⊢ c : iN]]$.
            Then we infer $[[Γ ; Φ2 ⊢ let x = v; c : iN]]$ by \ruleref{\ottdruleDTVarLetLabel}.

        \item \ruleref{\ottdruleDTAppLetLabel}
            Then we are proving that 
            $[[Γ ; Φ1 ⊢ let x = v(args); c : iN]]$ implies 
            $[[Γ ; Φ2 ⊢ let x = v(args); c : iN]]$.

            We apply the induction hypothesis to each of the premises.
            to rewrite:
            \begin{itemize}
                \item $[[Γ ; Φ1 ⊢ v : ↓iM]]$ into $[[Γ ; Φ2 ⊢ v : ↓iM]]$,
                \item $[[Γ ; Φ1 ⊢ iM ● args ⇒> ↑iQ]]$ into $[[Γ ; Φ2 ⊢ iM ● args ⇒> ↑iQ]]$.
                \item $[[Γ ; Φ1, x:iQ ⊢ c : iN]]$ into $[[Γ ; Φ2, x:iQ ⊢ c : iN]]$
                (notice that $[[Γ ⊢ Φ1, x:iQ ≈ Φ2, x:iQ]]$).
            \end{itemize}

            It is left to show the uniqueness of $[[Γ ; Φ2 ⊢ iM ● args ⇒> ↑iQ]]$.
            Let us assume that this judgement holds for other $[[iQ']]$, 
            i.e.  there exists a tree $T_0$ inferring 
            $[[Γ ; Φ2 ⊢ iM ● args ⇒> ↑iQ']]$.
            Then notice that the induction hypothesis is applicable to
            $T_0$: the first component of the first component of $\metric{T_0}$
            is $S = \sum_{[[v]] \in [[args]]} \size{[[v]]}$, and it is less than
            the corresponding component of $\metric{T_1}$, which is
            $\size{[[let x = v(args); c]]} = 1 + \size{[[v]]} + \size{[[c]]} + S$.
            This way, $[[Γ ; Φ1 ⊢ iM ● args ⇒> ↑iQ']]$ holds by the induction hypothesis,
            but since $[[Γ ; Φ1 ⊢ iM ● args ⇒> ↑iQ uniq]]$, we have $[[Γ ⊢ iQ' ≈ iQ]]$.

            Then we infer $[[Γ ; Φ2 ⊢ let x = v(args); c : iN]]$ by \ruleref{\ottdruleDTAppLetLabel}.

        \item \ruleref{\ottdruleDTAppLetAnnLabel}
            Then we are proving that
            $[[Γ ; Φ1 ⊢ let x:iP = v(args); c : iN]]$ implies
            $[[Γ ; Φ2 ⊢ let x:iP = v(args); c : iN]]$.
        
            As in the previous case, we apply the induction hypothesis to each of the premises
            and rewrite:
            \begin{itemize}
                \item $[[Γ ; Φ1 ⊢ v : ↓iM]]$ into $[[Γ ; Φ2 ⊢ v : ↓iM]]$,
                \item $[[Γ ; Φ1 ⊢ iM ● args ⇒> ↑iQ]]$ into $[[Γ ; Φ2 ⊢ iM ● args ⇒> ↑iQ]]$, 
                    and
                \item $[[Γ ; Φ1, x:iP ⊢ c : iN]]$ into $[[Γ ; Φ2, x:iP ⊢ c : iN]]$
                (notice that $[[Γ ⊢ Φ1, x:iP ≈ Φ2, x:iP]]$).
            \end{itemize}
            
            Notice that $[[Γ ⊢ iP]]$ and $[[Γ ⊢ ↑iQ ≤ ↑iP]]$ 
            do not depend on the variable context, and hold by assumption.
            Then we infer $[[Γ ; Φ2 ⊢ let x:iP = v(args); c : iN]]$ by \ruleref{\ottdruleDTAppLetAnnLabel}.

        \item \ruleref{\ottdruleDTUnpackLabel}, and \ruleref{\ottdruleDTNAnnotLabel}
            are proved in the same way.

        \item \ruleref{\ottdruleDTEmptyAppLabel}
            Then we are proving that 
            $[[Γ ; Φ1 ⊢ iN ● · ⇒> iN']]$ (inferred by \ruleref{\ottdruleDTEmptyAppLabel})
            implies $[[Γ ; Φ2 ⊢ iN ● · ⇒> iN']]$.

            To infer $[[Γ ; Φ2 ⊢ iN ● · ⇒> iN']]$, 
            we apply \ruleref{\ottdruleDTEmptyAppLabel}, noting that 
            $[[Γ ⊢ iN ≈ iN']]$ holds by assumption.

        \item \ruleref{\ottdruleDTArrowAppLabel}
            Then we are proving that 
            $[[Γ ; Φ1 ⊢ iQ → iN ● v, args ⇒> iM]]$ (inferred by \ruleref{\ottdruleDTArrowAppLabel})
            implies $[[Γ ; Φ2 ⊢ iQ → iN ● v, args ⇒> iM]]$.
            And uniqueness of the $[[iM]]$ in the first case implies uniqueness in the second case.

            By induction, we rewrite $[[Γ ; Φ1 ⊢ v : iP]]$ into $[[Γ ; Φ2 ⊢ v : iP]]$, 
            and $[[Γ ; Φ1 ⊢ iN ● args ⇒> iM]]$ into $[[Γ ; Φ2 ⊢ iN ● args ⇒> iM]]$.
            Then we infer $[[Γ ; Φ2 ⊢ iQ → iN ● v, args ⇒> iM]]$ by \ruleref{\ottdruleDTArrowAppLabel}.

            Now, let us show the uniqueness.
            The only rule that can infer $[[Γ ; Φ1 ⊢ iQ → iN ● v, args ⇒> iM]]$
            is \ruleref{\ottdruleDTArrowAppLabel}.
            Then by inversion, 
            uniqueness of $[[Γ ; Φ1 ⊢ iQ → iN ● v, args ⇒> iM]]$ implies
            uniqueness of $[[Γ ; Φ1 ⊢ iN ● args ⇒> iM]]$. By 
            the induction hypothesis, it implies the uniqueness of 
            $[[Γ ; Φ2 ⊢ iN ● args ⇒> iM]]$.


            Suppose that 
            $[[Γ ; Φ2 ⊢ iQ → iN ● v, args ⇒> iM']]$.
            By inversion, $[[Γ ; Φ2 ⊢ iN ● args ⇒> iM']]$, 
            which by uniqueness of $[[Γ ; Φ2 ⊢ iN ● args ⇒> iM]]$ implies
            $[[Γ ⊢ iM ≈ iM']]$.

        \item \ruleref{\ottdruleDTForallAppLabel}
            Then we are proving that
            $[[Γ ; Φ1 ⊢ ∀pas.iN ● args ⇒> iM]]$ (inferred by \ruleref{\ottdruleDTForallAppLabel})
            implies $[[Γ ; Φ2 ⊢ ∀pas.iN ● args ⇒> iM]]$.

            By inversion, we have $[[σ]]$ such that $[[Γ ⊢ σ : pas]]$ and
            $[[Γ ; Φ1 ⊢ [σ]iN ● args ⇒> iM]]$ is inferred.
            Let us denote the inference tree as $T_1'$.
            Notice that the induction hypothesis is applicable to $T_1'$:
            $\metric{T_1'} = ((\size{[[args]]}, 0), x)$ (for some $x$) is less than 
            $\metric{T_1} = ((\size{[[args]]}, |[[pas]]|), y)$ (for some $y$),
            since $|[[pas]]| > 0$ by inversion.

            This way, by the induction hypothesis, 
            there exists a tree $T_2'$ inferring
            $[[Γ ; Φ2 ⊢ [σ]iN ● args ⇒> iM]]$.
            Notice that the premises $[[args ≠ ·]]$, $[[Γ ⊢ σ : pas]]$,
            and $[[pas ≠ ·]]$ do not depend on the variable context,
            and hold by inversion.
            Then we infer $[[Γ ; Φ2 ⊢ ∀pas.iN ● args ⇒> iM]]$ by \ruleref{\ottdruleDTForallAppLabel}.
    \end{caseof}
    
\end{proof}
