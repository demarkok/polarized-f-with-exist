\begin{lemma} [Soundness of Constraint Entry Merge]
    \label{lemma:entry-merge-soundness}
    For a fixed context $[[Γ]]$,
    suppose that  $[[Γ ⊢ scE1]]$ and $[[Γ ⊢ scE2]]$. 
    If $[[Γ ⊢ scE1 & scE2 = scE]]$ is defined then
    \begin{enumerate}
        \item $[[Γ ⊢ scE]]$
        \item For any $[[Γ ⊢ iP]]$, $[[Γ ⊢ iP : scE]]$ implies $[[Γ ⊢ iP : scE1]]$ and $[[Γ ⊢ iP : scE2]]$ 
    \end{enumerate}
\end{lemma}
\begin{proof}
    Let us consider the rule forming $[[Γ ⊢ scE1 & scE2 = scE]]$.
    \begin{caseof}
        \item \ruleref{\ottdruleSCMEPEqEqLabel}, i.e. 
            $[[Γ ⊢ scE1 & scE2 = scE]]$
            has form $[[Γ ⊢ (pua :≈ iQ) & (pua :≈ iQ') = (pua :≈ iQ)]]$
            and $[[nf(iQ) = nf(iQ')]]$. The latter implies $[[Γ ⊢ iQ ≈ iQ']]$ by
            \cref{lemma:subt-equiv-algorithmization}.
            Then
            \begin{enumerate}
                \item $[[Γ ⊢ scE]]$, i.e. $[[Γ ⊢ pua :≈ iQ]]$ holds by assumption;
                \item by inversion, $[[Γ ⊢ iP : (pua :≈ iQ)]]$ means $[[Γ ⊢ iP ≈ iQ]]$,
                and by transitivity of equivalence (\cref{corollary:equivalence-transitivity}), 
                $[[Γ ⊢ iP ≈ iQ']]$. Thus, $[[Γ ⊢ iP : scE1]]$ and $[[Γ ⊢ iP : scE2]]$ hold
                by \ruleref{\ottdruleSATSCEPEqLabel}.
            \end{enumerate}
        \item \ruleref{\ottdruleSCMENEqEqLabel} the negative case is proved in exactly the same way as the positive one.
        \item \ruleref{\ottdruleSCMESupSupLabel} 
            Then $[[scE1]]$ is $[[pua :≥ iQ1]]$, $[[scE2]]$ is $[[pua :≥ iQ2]]$,
            and $[[scE1 & scE2]] = [[scE]]$ is $[[pua :≥ iQ]]$ where $[[iQ]]$ is the least upper bound of $[[iQ1]]$ and $[[iQ2]]$.
            Then by \cref{lemma:lub-soundness},
            \begin{itemize}
                \item $[[Γ ⊢ iQ]]$,
                \item $[[Γ ⊢ iQ ≥ iQ1]]$,
                \item $[[Γ ⊢ iQ ≥ iQ2]]$.
            \end{itemize}

            Let us show the required properties.
            \begin{itemize}
                \item $[[Γ ⊢ scE]]$ holds from $[[Γ ⊢ iQ]]$,
                \item Assuming $[[Γ ⊢ iP : scE]]$, by inversion, we have $[[Γ ⊢ iP ≥ iQ]]$.
                    Combining it transitively with $[[Γ ⊢ iQ ≥ iQ1]]$, we have $[[Γ ⊢ iP ≥ iQ1]]$.
                    Analogously, $[[Γ ⊢ iP ≥ iQ2]]$.
                    Then $[[Γ ⊢ iP : scE1]]$ and $[[Γ ⊢ iP : scE2]]$ hold by \ruleref{\ottdruleSATSCESupLabel}.
            \end{itemize}

        \item \ruleref{\ottdruleSCMESupEqLabel}
            Then $[[scE1]]$ is $[[pua :≥ iQ1]]$, $[[scE2]]$ is $[[pua :≈ iQ2]]$, 
            where $[[Γ;· ⊨ uQ2 ≥ iQ1 ⫤ ·]]$, and the resulting   
            $[[scE1 & scE2]] = [[scE]]$ is equal to $[[scE2]]$, that is $[[pua :≈ iQ2]]$.
    
            Let us show the required properties.
            \begin{itemize}
                \item By assumption, $[[Γ ⊢ iQ]]$, and hence $[[Γ ⊢ scE]]$.
                \item Since $[[uv(uQ2) = ∅]]$, 
                    $[[Γ;· ⊨ uQ2 ≥ iQ1 ⫤ ·]]$ implies $[[Γ ⊢ iQ2 ≥ iQ1]]$
                    by the soundness of positive subtyping (\cref{lemma:pos-subt-soundness}).
                    Then let us take an arbitrary $[[Γ ⊢ iP]]$ such that $[[Γ ⊢ iP : scE]]$.
                    Since $[[scE2]] = [[scE]]$, $[[Γ ⊢ iP : scE2]]$ holds immediately.
                    
                    By inversion, $[[Γ ⊢ iP : (pua :≈ iQ2)]]$ means $[[Γ ⊢ iP ≈ iQ2]]$, 
                    and then by transitivity of subtyping (\cref{lemma:subtyping-transitivity}),
                    $[[Γ⊢ iP ≥ iQ1]]$.  Then $[[Γ ⊢ iP : scE1]]$ holds by \ruleref{\ottdruleSATSCESupLabel}.
            \end{itemize}
        \item \ruleref{\ottdruleSCMEEqSupLabel} Thee proof is analogous to the previous case.
    \end{caseof}
\end{proof}

\begin{lemma} [Soundness of Constraint Merge] \label{lemma:merge-soundness}
    Suppose that $[[Θ ⊢ SC1 : Ξ1]]$ and $[[Θ ⊢ SC2 : Ξ2]]$ 
    and $[[Θ ⊢ SC1 & SC2 = SC]]$ is defined.
    Then 
    \begin{enumerate}
        \item $[[Θ ⊢ SC : Ξ1 ∪ Ξ2]]$,
        \item for any substitution $[[Θ ⊢ uσ : Ξ1 ∪ Ξ2]]$, 
            $[[ Θ ⊢ uσ : SC ]]$
            implies $[[ Θ ⊢ uσ : SC1 ]]$ and $[[ Θ ⊢ uσ : SC2 ]]$.
    \end{enumerate}
\end{lemma}
\begin{proof}
    By definition, $[[Θ ⊢ SC1 & SC2 = SC]]$ consists of three parts:
    entries of $[[SC1]]$ that do not have matching entries of $[[SC2]]$,
    entries of $[[SC2]]$ that do not have matching entries of $[[SC1]]$,
    and the merge of matching entries.

    Notice that $[[α̂± ∊ Ξ1 \ Ξ2]]$
    if and only if there is an entry $[[scE]]$ in $[[SC1]]$ 
    restricting $[[α̂±]]$, but there is no such entry in $[[SC2]]$.
    Therefore, for any $[[α̂± ∊ Ξ1 \ Ξ2]]$,
    there is an entry $[[scE]]$ in $[[SC]]$ restricting $[[α̂±]]$.
    Notice that $[[Θ(α̂±) ⊢ scE]]$ holds since $[[Θ ⊢ SC1 : Ξ1]]$.

    Analogously, for any $[[β̂± ∊ Ξ2 \ Ξ1]]$,
    there is an entry $[[scE]]$ in $[[SC]]$ restricting $[[β̂±]]$.
    Notice that  $[[Θ(β̂±) ⊢ scE]]$ holds since $[[Θ ⊢ SC2 : Ξ2]]$.

    Finally, for any $[[γ̂± ∊ Ξ1 ∩ Ξ2]]$,
    there is an entry $[[scE1]]$ in $[[SC1]]$ restricting $[[γ̂±]]$
    and an entry $[[scE2]]$ in $[[SC2]]$ restricting $[[γ̂±]]$.
    Since $[[Θ ⊢ SC1 & SC2 = SC]]$ is defined,
    $[[Θ(γ̂±) ⊢ scE1 & scE2 = scE]]$ restricting $[[γ̂±]]$ is
    defined and belongs to $[[SC]]$,
    moreover, $[[Θ(γ̂±) ⊢ scE]]$ by \cref{lemma:entry-merge-soundness}.
    This way, $[[Θ ⊢ SC : Ξ1 ∪ Ξ2]]$.

    Let us show the second property.
    We take an arbitrary $[[uσ]]$ such that $[[Θ ⊢ uσ : Ξ1 ∪ Ξ2]]$ 
    and $[[ Θ ⊢ uσ : SC ]]$.
    To prove $[[ Θ ⊢ uσ : SC1 ]]$, 
    we need to show that for any $[[scE1]] \in [[SC1]]$, 
    restricting $[[α̂±]]$, $[[Θ(α̂±) ⊢ [uσ]α̂± : scE1]]$ holds.

    Let us assume that $[[α̂±]] \notin [[dom(SC2)]]$. It means that $[[SC]] \ni [[scE1]]$, 
    and then since $[[ Θ ⊢ uσ : SC ]]$, $[[Θ(α̂±) ⊢ [uσ]α̂± : scE1]]$. 

    Otherwise, $[[SC2]]$ contains an entry $[[scE2]]$ restricting $[[α̂±]]$,
    and $[[SC]] \ni [[scE]]$ where $[[Θ(α̂±) ⊢ scE1 & scE2 = scE]]$.
    Then since $[[ Θ ⊢ uσ : SC ]]$, $[[Θ(α̂±) ⊢ [uσ]α̂± : scE]]$,
    and by \cref{lemma:entry-merge-soundness}, $[[Θ(α̂±) ⊢ [uσ]α̂± : scE1]]$.

    The proof of $[[ Θ ⊢ uσ : SC2 ]]$ is symmetric.
\end{proof}


\begin{lemma} [Completeness of Constraint Entry Merge]
    \label{lemma:entry-merge-completeness}
    For a fixed context $[[Γ]]$,
    suppose that $[[Γ ⊢ scE1]]$ and $[[Γ ⊢ scE2]]$ are matching constraint entries.
    \begin{itemize}
        \item for a type $[[iP]]$ such that $[[Γ ⊢ iP : scE1]]$ and $[[Γ ⊢ iP : scE2]]$,
        $[[Γ ⊢ scE1 & scE2 = scE]]$ is defined and $[[Γ ⊢ iP : scE]]$.
        \item for a type $[[iN]]$ such that $[[Γ ⊢ iN : scE1]]$ and $[[Γ ⊢ iN : scE2]]$,
        $[[Γ ⊢ scE1 & scE2 = scE]]$ is defined and $[[Γ ⊢ iN : scE]]$.
    \end{itemize}
\end{lemma}
\begin{proof}
    Let us consider the shape of $[[scE1]]$ and $[[scE2]]$.
    \begin{caseof}
        \item $[[scE1]]$ is $[[pua :≈ iQ1]]$ and $[[scE2]]$ is $[[pua :≈ iQ2]]$.
            Then $[[Γ ⊢ iP : scE1]]$ means $[[Γ ⊢ iP ≈ iQ1]]$, 
            and $[[Γ ⊢ iP : scE2]]$ means $[[Γ ⊢ iP ≈ iQ2]]$.
            Then by transitivity of equivalence (\cref{corollary:equivalence-transitivity}),
            $[[Γ ⊢ iQ1 ≈ iQ2]]$, which means $[[nf(iQ1) = nf(iQ2)]]$ by
            \cref{lemma:subt-equiv-algorithmization}.
            Hence, \ruleref{\ottdruleSCMEPEqEqLabel} applies to infer
            $[[Γ ⊢ scE1 & scE2 = scE2]]$, and $[[Γ ⊢ iP : scE2]]$ holds by assumption.
        \item $[[scE1]]$ is $[[pua :≈ iQ1]]$ and $[[scE2]]$ is $[[pua :≥ iQ2]]$.
            Then $[[Γ ⊢ iP : scE1]]$ means $[[Γ ⊢ iP ≈ iQ1]]$, 
            and $[[Γ ⊢ iP : scE2]]$ means $[[Γ ⊢ iP ≥ iQ2]]$.
            Then by transitivity of subtyping, $[[Γ ⊢ iQ1 ≥ iQ2]]$,
            which means $[[Γ ; · ⊨ uQ1 ≥ iQ2 ⫤ ·]]$ by \cref{lemma:pos-subt-completeness}.
            This way, \ruleref{\ottdruleSCMEEqSupLabel} applies to infer
            $[[Γ ⊢ scE1 & scE2 = scE1]]$, and $[[Γ ⊢ iP : scE1]]$ holds by assumption.
        \item $[[scE1]]$ is $[[pua :≥ iQ1]]$ and $[[scE2]]$ is $[[pua :≥ iQ2]]$.
            Then $[[Γ ⊢ iP : scE1]]$ means $[[Γ ⊢ iP ≥ iQ1]]$, 
            and $[[Γ ⊢ iP : scE2]]$ means $[[Γ ⊢ iP ≥ iQ2]]$.
            By the completeness of the least upper bound (\cref{lemma:lub-completeness}), 
            $[[Γ ⊨ iQ1 ∨ iQ2 = iQ]]$, and $[[Γ ⊢ iP ≥ iQ]]$. 
            This way, \ruleref{\ottdruleSCMESupSupLabel} applies to infer
            $[[Γ ⊢ scE1 & scE2 = (pua :≥ iQ)]]$, 
            and $[[Γ ⊢ iP : (pua :≥ iQ)]]$ holds by \ruleref{\ottdruleSATSCESupLabel}.
        \item The negative cases are proved symmetrically.
    \end{caseof}
\end{proof}

\begin{lemma} [Completeness of Constraint Merge] 
    \label{lemma:merge-completeness}
    Suppose that $[[Θ ⊢ SC1 : Ξ1]]$ and $[[Θ ⊢ SC2 : Ξ2]]$.
    If there exists a substitution $[[Θ ⊢ uσ : Ξ1 ∪ Ξ2]]$ such that 
    $[[ Θ ⊢ uσ : SC1 ]]$ and $[[ Θ ⊢ uσ : SC2 ]]$
    then $[[Θ ⊢ SC1 & SC2 = SC]]$ is defined,
    and $[[ Θ ⊢ uσ : SC ]]$.
\end{lemma}
\begin{proof}
    By  definition, $[[SC1 & SC2]]$ is a union of
    \begin{enumerate}
        \item entries of $[[SC1]]$, which do not have matching entries in $[[SC2]]$,
        \item entries of $[[SC2]]$, which do not have matching entries in $[[SC1]]$, and 
        \item the merge of matching entries.
    \end{enumerate}

    This way, to show that $[[Θ ⊢ SC1 & SC2 = SC]]$ is defined, we need to demonstrate that 
    each of these components is defined and satisfies 
    the required property 
    (that the result of $[[uσ]]$ satisfies the corresponding constraint entry).

    It is clear that the first two components of this union exist. 
    Moreover, if $[[scE]]$ is an entry of $[[SCi]]$
    restricting $[[α̂± ∉ dom(SC2)]]$,
    then $[[ Θ ⊢ uσ : SCi ]]$ implies $[[ Θ(α̂±) ⊢ [uσ]α̂± : scE]]$,

    Let us show that the third component exists.  
    Let us take two entries $[[scE1]] \in [[SC1]]$ and $[[scE2]] \in [[SC2]]$ restricting the same variable $[[α̂±]]$.  $[[ Θ   ⊢ uσ : SC1 ]]$ means that $[[Θ(α̂±) ⊢ [uσ]α̂± : scE1]]$ and $[[ Θ   ⊢ uσ : SC2 ]]$ means $[[Θ(α̂±) ⊢ [uσ]α̂± : scE2]]$.
    Then by \cref{lemma:entry-merge-completeness}, $[[Θ(α̂±) ⊢ scE1 & scE2 = scE]]$ is defined and $[[Θ(α̂±) ⊢ [uσ]α̂± : scE]]$.

\end{proof}


