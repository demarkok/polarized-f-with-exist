\begin{lemma} [Soundness of Constraint Entry Merge]
\label{lemma:entry-merge-soundness}
For a fixed context $[[Γ]]$,
suppose that  $[[Γ ⊢ scE1]]$ and $[[Γ ⊢ scE2]]$. 
If $[[Γ ⊢ scE1 & scE2 = scE]]$ is defined then
\begin{enumerate}
    \item $[[Γ ⊢ scE]]$
    \item For any $[[Γ ⊢ iP]]$, $[[Γ ⊢ iP : scE]]$ implies $[[Γ ⊢ iP : scE1]]$ and $[[Γ ⊢ iP : scE2]]$ 
\end{enumerate}
\end{lemma}
\begin{proof}
    Let us consider the rule forming $[[Γ ⊢ scE1 & scE2 = scE]]$.
    \begin{caseof}
        \item \ruleref{\ottdruleSCMEPEqEqLabel}, i.e. 
        $[[Γ ⊢ scE1 & scE2 = scE]]$
        has form $[[Γ ⊢ (pua :≈ iQ) & (pua :≈ iQ') = (pua :≈ iQ)]]$
        and $[[nf(iQ) = nf(iQ')]]$. The latter implies $[[Γ ⊢ iQ ≈ iQ']]$ by
        \cref{lemma:subt-equiv-algorithmization}.
        Then
        \begin{enumerate}
            \item $[[Γ ⊢ scE]]$, i.e. $[[Γ ⊢ pua :≈ iQ]]$ holds by assumption;
            \item by inversion, $[[Γ ⊢ iP : (pua :≈ iQ)]]$ means $[[Θ ⊢ iP ≈ iQ]]$,
            and by transitivity of equivalence (\cref{corollary:equivalence-transitivity}), 
            $[[Θ ⊢ iP ≈ iQ']]$. Thus, $[[Γ ⊢ iP : scE1]]$ and $[[Γ ⊢ iP : scE2]]$ hold
            by \ruleref{\ottdruleSATSCEPEqLabel}.
        \end{enumerate}
        \item \ruleref{\ottdruleSCMENEqEqLabel} the negative case is proved in exactly the same way as the positive one.
        \item \ruleref{\ottdruleSCMESupSupLabel} 
        Then $[[scE1]]$ is $[[pua :≥ iQ1]]$, $[[scE2]]$ is $[[pua :≥ iQ2]]$,
        and $[[scE1 & scE2]] = [[scE]]$ is $[[pua :≥ iQ]]$ where $[[iQ]]$ is the least upper bound of $[[iQ1]]$ and $[[iQ2]]$.
        Then by \cref{lemma:lub-soundness},
        \begin{itemize}
            \item $[[Γ ⊢ iQ]]$,
            \item $[[Γ ⊢ iQ ≥ iQ1]]$,
            \item $[[Γ ⊢ iQ ≥ iQ2]]$.
        \end{itemize}

        Let us show the required properties.
        \begin{itemize}
            \item $[[Γ ⊢ scE]]$ holds from $[[Γ ⊢ iQ]]$,
            \item Assuming $[[Γ ⊢ iP : scE]]$, by inversion, we have $[[Γ ⊢ iP ≥ iQ]]$.
                Combining it transitively with $[[Γ ⊢ iQ ≥ iQ1]]$, we have $[[Γ ⊢ iP ≥ iQ1]]$.
                Analogously, $[[Γ ⊢ iP ≥ iQ2]]$.
                Then $[[Γ ⊢ iP : scE1]]$ and $[[Γ ⊢ iP : scE2]]$ hold by \ruleref{\ottdruleSATSCESupLabel}.
        \end{itemize}

        \item \ruleref{\ottdruleSCMESupEqLabel}
        Then $[[scE1]]$ is $[[pua :≥ iQ1]]$, $[[scE2]]$ is $[[pua :≈ iQ2]]$, 
        where $[[Γ;· ⊨ uQ2 ≥ iQ1 ⫤ ·]]$, and the resulting   
        $[[scE1 & scE2]] = [[scE]]$ is equal to $[[scE2]]$, that is $[[pua :≈ iQ2]]$.
    
        Let us show the required properties.
        \begin{itemize}
            \item By assumption, $[[Γ ⊢ iQ]]$, and hence $[[Γ ⊢ scE]]$.
            \item Since $[[uv(uQ2) = ∅]]$, 
                $[[Γ;· ⊨ uQ2 ≥ iQ1 ⫤ ·]]$ implies $[[Γ ⊢ iQ2 ≥ iQ1]]$
                by the soundness of positive subtyping (\cref{lemma:pos-subt-soundness}).
                Then let us take an arbitrary $[[Γ ⊢ iP]]$ such that $[[Γ ⊢ iP : scE]]$.
                Since $[[scE2]] = [[scE]]$, $[[Γ ⊢ iP : scE2]]$ holds immediately.
                
                By inversion, $[[Γ ⊢ iP : (pua :≈ iQ2)]]$ means $[[Γ ⊢ iP ≈ iQ2]]$, 
                and then by transitivity of subtyping (\cref{lemma:subtyping-transitivity}),
                $[[Γ⊢ iP ≥ iQ1]]$.  Then $[[Γ ⊢ iP : scE1]]$ holds by \ruleref{\ottdruleSATSCESupLabel}.
        \end{itemize}
        \item \ruleref{\ottdruleSCMEEqSupLabel} Thee proof is analogous to the previous case.
    \end{caseof}
\end{proof}

\begin{lemma} [Soundness of Constraint Merge] \label{lemma:merge-soundness}
    Suppose that $[[Θ ⊢ SC1]]$ and $[[Θ ⊢ SC2]]$ 
    and $[[Θ ⊢ SC1 & SC2 = SC]]$ is defined.
    Then 
    \begin{enumerate}
        \item $[[Θ ⊢ SC]]$,
        \item for any substitution $[[Θ ⊢ uσ]]$, $[[(Θ  ⊢  SC) ⊢ uσ]]$ implies $[[(Θ  ⊢  SC1) ⊢ uσ]]$ and $[[(Θ  ⊢  SC2) ⊢ uσ]]$.
    \end{enumerate}
\end{lemma}
\begin{proof}
    By definition, $[[SC1 & SC2]] = [[SC]]$ consists of three parts:
    entries of $[[SC1]]$ that do not have matching entries of $[[SC]]$,
    entries of $[[SC2]]$ that do not have matching entries of $[[SC1]]$,
    and the merge of matching entries.

    Let us show $[[Θ ⊢ SC]]$.
    First, let us assume that an entry $[[scE]] \in [[SC]]$ belongs to the first group, 
    i.e. $[[scE]] \in [[SC1]]$.  Let us denote the variable $[[scE]]$ as $[[α̂±]]$. 
    Then $[[Θ(α̂±) ⊢ scE]]$ holds since $[[Θ ⊢ SC1]] \ni [[scE]]$.
    Analogously, if $[[scE]]$ belongs to the second group, then $[[Θ(α̂±) ⊢ scE]]$ holds since $[[Θ ⊢ SC2]] \ni [[scE]]$.
    Finally, if $[[scE]]$ belongs to the third group, then $[[scE]]$ is a merge of two entries $[[Θ(α̂±) ⊢ scE1]]$ 
    and $[[Θ(α̂±) ⊢ scE2]]$.  Then $[[Θ(α̂±) ⊢ scE]]$ holds by \cref{lemma:entry-merge-soundness}.

    Let us show the second property.
    We take an arbitrary $[[uσ]]$ such that $[[Θ ⊢ uσ]]$ and $[[(Θ  ⊢  SC) ⊢ uσ]]$.
    To prove $[[(Θ  ⊢  SC1) ⊢ uσ]]$, we need to show that for any $[[scE1]] \in [[SC1]]$, restricting $[[α̂±]]$,
    $[[Θ(α̂±) ⊢ [uσ]α̂± : scE1]]$ holds.

    Let us assume that $[[α̂±]] \notin [[dom(SC2)]]$. It means that $[[SC]] \ni [[scE1]]$, 
    and then since $[[(Θ  ⊢  SC) ⊢ uσ]]$, $[[Θ(α̂±) ⊢ [uσ]α̂± : scE1]]$. 

    Otherwise, $[[SC2]]$ contains an entry $[[scE2]]$ restricting $[[α̂±]]$,
    and $[[SC]] \ni [[scE]]$ where $[[Θ(α̂±) ⊢ scE1 & scE2 = scE]]$.
    Then since $[[(Θ  ⊢  SC) ⊢ uσ]]$, $[[Θ(α̂±) ⊢ [uσ]α̂± : scE]]$,
    and by \cref{lemma:entry-merge-soundness}, $[[Θ(α̂±) ⊢ [uσ]α̂± : scE1]]$.

    The proof of $[[(Θ  ⊢  SC2) ⊢ uσ]]$ is symmetric.
\end{proof}


\begin{lemma} [Completeness of Constraint Entry Merge]
    \label{lemma:entry-merge-completeness}
    For a fixed context $[[Γ]]$,
    suppose that $[[Γ ⊢ scE1]]$ and $[[Γ ⊢ scE2]]$ are matching constraint entries.
    \begin{itemize}
        \item for a type $[[iP]]$ such that $[[Γ ⊢ iP : scE1]]$ and $[[Γ ⊢ iP : scE2]]$,
        $[[Γ ⊢ scE1 & scE2 = scE]]$ is defined and $[[Γ ⊢ iP : scE]]$.
        \item for a type $[[iN]]$ such that $[[Γ ⊢ iN : scE1]]$ and $[[Γ ⊢ iN : scE2]]$,
        $[[Γ ⊢ scE1 & scE2 = scE]]$ is defined and $[[Γ ⊢ iN : scE]]$.
    \end{itemize}
\end{lemma}
\begin{proof}
    Let us consider the shape of $[[scE1]]$ and $[[scE2]]$.
    \begin{caseof}
        \item $[[scE1]]$ is $[[pua :≈ iQ1]]$ and $[[scE2]]$ is $[[pua :≈ iQ2]]$.
            Then $[[Γ ⊢ iP : scE1]]$ means $[[Γ ⊢ iP ≈ iQ1]]$, 
            and $[[Γ ⊢ iP : scE2]]$ means $[[Γ ⊢ iP ≈ iQ2]]$.
            Then by transitivity of equivalence (\cref{corollary:equivalence-transitivity}),
            $[[Γ ⊢ iQ1 ≈ iQ2]]$, which means $[[nf(iQ1) = nf(iQ2)]]$ by
            \cref{lemma:subt-equiv-algorithmization}.
            Hence, \ruleref{\ottdruleSCMEPEqEqLabel} applies to infer
            $[[Γ ⊢ scE1 & scE2 = scE2]]$, and $[[Γ ⊢ iP : scE2]]$ holds by assumption.
        \item $[[scE1]]$ is $[[pua :≈ iQ1]]$ and $[[scE2]]$ is $[[pua :≥ iQ2]]$.
            Then $[[Γ ⊢ iP : scE1]]$ means $[[Γ ⊢ iP ≈ iQ1]]$, 
            and $[[Γ ⊢ iP : scE2]]$ means $[[Γ ⊢ iP ≥ iQ2]]$.
            Then by transitivity of subtyping, $[[Γ ⊢ iQ1 ≥ iQ2]]$,
            which means $[[Γ ; · ⊨ uQ1 ≥ iQ2 ⫤ ·]]$ by \cref{lemma:pos-subt-completeness}.
            This way, \ruleref{\ottdruleSCMEEqSupLabel} applies to infer
            $[[Γ ⊢ scE1 & scE2 = scE1]]$, and $[[Γ ⊢ iP : scE1]]$ holds by assumption.
        \item $[[scE1]]$ is $[[pua :≥ iQ1]]$ and $[[scE2]]$ is $[[pua :≥ iQ2]]$.
            Then $[[Γ ⊢ iP : scE1]]$ means $[[Γ ⊢ iP ≥ iQ1]]$, 
            and $[[Γ ⊢ iP : scE2]]$ means $[[Γ ⊢ iP ≥ iQ2]]$.
            By the completeness of the least upper bound (\cref{lemma:lub-completeness}), 
            $[[Γ ⊨ iQ1 ∨ iQ2 = iQ]]$, and $[[Γ ⊢ iP ≥ iQ]]$. 
            This way, \ruleref{\ottdruleSCMESupSupLabel} applies to infer
            $[[Γ ⊢ scE1 & scE2 = (pua :≥ iQ)]]$, 
            and $[[Γ ⊢ iP : (pua :≥ iQ)]]$ holds by \ruleref{\ottdruleSATSCESupLabel}.
        \item The negative cases are proved symmetrically.
    \end{caseof}
\end{proof}

\begin{lemma} [Completeness of Constraint Merge] 
    \label{lemma:merge-completeness}
    Suppose that $[[Θ ⊢ SC1]]$ and $[[Θ ⊢ SC2]]$.
    Then for any substitution $[[Θ ⊢ uσ]]$ such that $[[(Θ  ⊢  SC1) ⊢ uσ]]$ and $[[(Θ  ⊢  SC2) ⊢ uσ]]$, 
    $[[Θ ⊢ SC1 & SC2 = SC]]$ is defined.
\end{lemma}
\begin{proof}
    By  definition, $[[SC1 & SC2]]$ is a union of
    \begin{enumerate}
        \item entries of $[[SC1]]$, which do not have matching entries in $[[SC2]]$,
        \item entries of $[[SC2]]$, which do not have matching entries in $[[SC1]]$, and 
        \item the merge of matching entries.
    \end{enumerate}

    This way, to show that $[[Θ ⊢ SC1 & SC2 = SC]]$ is defined, we need to demonstrate that each of these components is defined.

    It is clear that the first two components of this union exist. 
    Let us show that the third component exists.  
    Let us take two entries $[[scE1]] \in [[SC1]]$ and $[[scE2]] \in [[SC2]]$ restricting the same variable $[[α̂±]]$.  $[[(Θ  ⊢  SC1) ⊢ uσ]]$ means that $[[Θ(α̂±) ⊢ [uσ]α̂± : scE1]]$ and $[[(Θ  ⊢  SC2) ⊢ uσ]]$ means $[[Θ(α̂±) ⊢ [uσ]α̂± : scE2]]$.
    Then by \cref{lemma:entry-merge-completeness}, $[[Θ(α̂±) ⊢ scE1 & scE2 = scE]]$ is defined and $[[Θ(α̂±) ⊢ [uσ]α̂± : scE]]$.
\end{proof}

\begin{lemma}[Substitution existence]
    \label{lemma:substitution-existence}
    If $[[Θ ⊢ SC]]$ then there exists $[[Θ ⊢ uσ]]$ such that $[[(Θ  ⊢  SC) ⊢ uσ]]$.
\end{lemma}
\begin{proof}
    
\end{proof}
