\begin{lemma}[Substitution strengthening]
  \label{lemma:subst-restr-fv}
  Restricting the substitution to the free variables of the
  substitution subject does not affect the result.
  Suppose that $[[Γ2 ⊢ σ : Γ1]]$. Then
  \begin{itemize}
  \item[$+$] if $[[Γ1 ⊢ iP]]$ then $[[ [σ]iP ]] = [[ [σ|fv iP]iP ]]$,
  \item[$-$] if $[[Γ1 ⊢ iN]]$ then $[[ [σ]iN ]] = [[ [σ|fv iN]iN ]]$
  \end{itemize}
\end{lemma}
\begin{proof}
  \ilyam{todo}
\end{proof}


\begin{corollary}[Substitution preserves equivalence]
  \label{corollary:subst-pres-equiv}

  Suppose that $[[Γ ⊢ σ : Γ1]]$. Then
  \begin{itemize}
  \item[$+$] if $[[Γ1 ⊢ iP]]$,~ $[[Γ1 ⊢ iQ]]$,~ and $[[Γ1 ⊢ iP ≈ iQ]]$ ~ then $[[Γ ⊢ [σ]iP ≈ [σ]iQ]]$
  \item[$-$] if $[[Γ1 ⊢ iN]]$,~ $[[Γ1 ⊢ iM]]$,~ and $[[Γ1 ⊢ iN ≈ iM]]$ ~ then $[[Γ ⊢ [σ]iN ≈ [σ]iM]]$
  \end{itemize}
\end{corollary}

\begin{lemma}[]
  Suppose that $[[{Γ'} ⊆ {Γ}]]$,
  $[[σ1]]$ and $[[σ2]]$ are substitutions of signature $[[Γ ⊢ σi : Γ']]$.
  Then 
  \begin{enumerate}
    \item [$+$] for a type $[[Γ ⊢ iP]]$, if $[[Γ ⊢ [σ1]iP ≈ [σ2]iP]]$ then 
    $[[Γ ⊢ σ1 ≈ σ2 : fv iP ∩ {Γ'}]]$;
    \item [$-$] for a type $[[Γ ⊢ iN]]$, if $[[Γ ⊢ [σ1]iN ≈ [σ2]iN]]$ then
    $[[Γ ⊢ σ1 ≈ σ2 : fv iN ∩ {Γ'}]]$.
  \end{enumerate}
\end{lemma}
\begin{proof}
  Let us make an additional assumption that $[[σ1]]$, $[[σ2]]$, 
  and the mentioned types are normalized. If they are not,
  we normalize them first.
  
  Notice that the normalization preserves
  the set of free variables (\cref{lemma:fv-nf}),
  well-formedness (\cref{corollary:wf-nf}), 
  and equivalence (\cref{lemma:subt-equiv-algorithmization}), 
  and distributes over substitution (\cref{lemma:norm-subst-distr}). 
  This way, the assumed and desired properties are equivalent to their 
  normalized versions.

  We prove it by induction on the structure of $[[iP]]$ and mutually, $[[iN]]$.
  Let us consider the shape of this type.
  \begin{caseof}
    \item $[[iP]] = [[α⁺]] \in [[Γ']]$.
      Then $[[Γ ⊢ σ1 ≈ σ2 : fv iP ∩ {Γ'}]]$ means $[[Γ ⊢ σ1 ≈ σ2 : {α⁺}]]$, 
      i.e. $[[Γ ⊢ [σ1]α⁺ ≈ [σ2]α⁺ ]]$, which holds by assumption.
    \item $[[iP]] = [[α⁺]] \in [[{Γ} \ {Γ'}]]$.
      Then $[[fv iP ∩ {Γ'}]] = [[∅]]$, 
      so $[[Γ ⊢ σ1 ≈ σ2 : fv iP ∩ {Γ'}]]$ holds vacuously.
    \item $[[iP]] = [[↓iN]]$.
      Then the induction hypothesis is applicable to type $[[iN]]$:
      \begin{enumerate}
        \item $[[iN]]$ is normalized,
        \item $[[Γ ⊢ iN]]$ by inversion of $[[Γ ⊢ ↓iN]]$,
        \item $[[Γ ⊢ [σ1]iN ≈ [σ2]iN]]$ holds by inversion of 
          $[[Γ ⊢ [σ1]↓iN ≈ [σ2]↓iN]]$, i.e. $[[Γ ⊢ ↓[σ1]iN ≈ ↓[σ2]iN]]$.
      \end{enumerate}
      This way, we obtain $[[Γ ⊢ σ1 ≈ σ2 : fv iN ∩ {Γ'}]]$, 
      which implies the required equivalence since 
      $[[fv iP ∩ {Γ'}]] = [[fv ↓iN ∩ {Γ'}]] = [[fv iN ∩ {Γ'}]]$.
    \item $[[iP]] = [[∃nas.iQ]]$
      Then the induction hypothesis is applicable to type $[[iQ]]$ 
      well-formed in context $[[Γ, nas]]$:
      \begin{enumerate}
        \item $[[{Γ'} ⊆ {Γ, nas}]]$ since $[[{Γ'} ⊆ {Γ}]]$,
        \item $[[Γ, nas ⊢ σi : Γ']]$ by weakening,
        \item $[[iQ]]$ is normalized,
        \item $[[Γ, nas ⊢ iQ]]$ by inversion of $[[Γ ⊢ ∃nas.iQ]]$,
        \item Notice that $[[ [σi]∃nas.iQ ]]$ is normalized, and thus, 
          $[[ [σ1]∃nas.iQ ≈ [σ2]∃nas.iQ]]$ implies 
          $[[ [σ1]∃nas.iQ = [σ2]∃nas.iQ ]]$
          (by \cref{lemma:subt-equiv-algorithmization}).).
          This equality means $[[ [σ1]iQ = [σ2]iQ ]]$, 
          which implies $[[Γ ⊢ [σ1]iQ ≈ [σ2]iQ]]$.
      \end{enumerate}
    \item $[[iN]] = [[iP → iM]]$
  \end{caseof}
\end{proof}