\lemmaSubstRestrFv*
\begin{proof}
  First, we strengthen the statement by saying that
  one can restrict the substitution to an arbitrary
  superset of the free variables of the substitution subject:
 \begin{itemize}
    \item[$+$] $[[ [σ]iP ]] = [[ [σ|varset]iP ]]$, for any $[[varset]] \supseteq [[fv iP]]$, and
    \item[$-$] $[[ [σ]iN ]] = [[ [σ|varset]iN ]]$, for any $[[varset]] \supseteq [[fv iN]]$.
  \end{itemize}
  Then the proof is a straightforward induction on $[[iP]]$ and mutually, on $[[iN]]$.
  For the base cases: 
  \begin{caseof}
    \item $[[iN = α⁻]]$\\
      Then $[[ [σ]α⁻ ]] = [[σ|varset]] ([[α⁻]])$ by definition, 
      since $[[α⁻]] \in [[fv α⁻ ⊆ varset]]$.
    \item $[[iN = iP → iM]]$\\
      Then $[[ [σ](iP → iM) ]] = [[ [σ]iP → [σ]iM ]]$ by definition.
      Since $[[fv iP]] \subseteq [[fv(iP → iM)]] \subseteq [[varset]]$, 
      the induction hypothesis is applicable to $[[ [σ]iP ]]$: 
      $[[ [σ]iP ]] = [[ [σ|varset]iP ]]$. 
      Analogously, and $[[ [σ]iM ]] = [[ [σ|varset]iM ]]$.
      Then 
      $[[ [σ](iP → iM) ]] = [[ [σ|varset]iP → [σ|varset]iM ]]
       = [[ [σ|varset](iP → iM) ]]$.
    \item $[[iN = ↑iP]]$ is proved analogously to the previous case.
    \item $[[iN = ∀pas.iM ]]$ (where $[[pas]]$ is not empty)\\
      Then $[[ [σ]∀pas.iM ]] = [[ ∀pas.[σ]iM ]]$ by definition.
      Let us assume $[[pas]]$ are fresh variables, 
      it means that $[[σ]]([[α± ]]) = [[α± ]]$ for any 
      $[[α± ]] \in [[pas]]$, and thus, 
      $[[ [σ|varset] ]] = [[ [σ|(varset ∪ {pas})] ]]$
      immediately from the definition. 

      Since $[[varset ⊆ fv(∀pas.iM)]] = [[fv iM \ {pas}]]$,
      $[[varset ∪ {pas} ⊆ fv(iM)]]$.
      Then by the induction hypothesis, 
      $[[ [σ]iM =  [σ|(varset ∪ {pas})]iM ]]$.
      Finally, 
      $[[ [σ]∀pas.iM ]] = [[ ∀pas.[σ|(varset ∪ {pas})]iM ]] = 
      [[ ∀pas.[σ|varset]iM ]] = [[ [σ|varset]∀pas.iM ]]$.
    \item The positive cases are proves symmetrically. 
  \end{caseof}
\end{proof}

\lemmaSubstRestrSig*
\begin{proof}
  Let us take an arbitrary $[[α± ∊ Γ1 ∩ varset]]$.
  Since $[[α± ∊ Γ1]]$, $[[Γ2 ⊢ [σ]α± ]]$ by the signature of $[[σ]]$.

  Let us take an arbitrary $[[α± ∉ Γ1 ∩ varset]]$.
  If $[[α± ∉ varset]]$ then $[[ [σ|varset]α± ]] = [[α±]]$ by definition
  of restriction. 
  If $[[α± ∊ varset \ Γ1]]$ then $[[ [σ|varset]α± ]] = [[ [σ]α± ]]$ by definition
  and $[[ [σ]α± ]] = [[α±]]$ by the signature of $[[σ]]$.
\end{proof}

\lemmaSubstDisjoint*
\begin{proof}
  Let us take an arbitrary $[[α±]]$.
  If $[[α± ∉ varset]]$ then $[[ [σ|varset]α±]] = [[α±]]$ by definition.

  If $[[α± ∊ varset]]$ then $[[α± ∉ Γ1]]$ by assumption.
  Then $[[ [σ|varset]α± = [σ]α±]]$ by definition of restricted substitution, and
  since $[[Γ2 ⊢ σ : Γ1]]$, we have $[[ [σ]α± = α± ]]$.
\end{proof}

\corollarySubstDisj*

\lemmaSubstRangeWeak*
\begin{proof}
  For any $[[α± ∊ Γ1]]$, 
  $[[Γ2 ⊢ σ : Γ1]]$ gives us  $[[ Γ2 ⊢ [σ]α± ]]$, which can be
  weakened to $[[ Γ'2 ⊢ [σ]α± ]]$ by \cref{lemma:wf-weakening}. 
  This way, $[[Γ'2 ⊢ σ : Γ1]]$.
\end{proof}

\lemmaSubstEquiv*
\begin{proof}
  Let us make an additional assumption that $[[σ1]]$, $[[σ2]]$, 
  and the mentioned types are normalized. If they are not,
  we normalize them first.
  
  Notice that the normalization preserves
  the set of free variables (\cref{lemma:fv-nf}),
  well-formedness (\cref{corollary:wf-nf}), 
  and equivalence (\cref{lemma:subt-equiv-algorithmization}), 
  and distributes over substitution (\cref{lemma:norm-subst-distr}). 
  This way, the assumed and desired properties are equivalent to their 
  normalized versions.

  We prove it by induction on the structure of $[[iP]]$ and mutually, $[[iN]]$.
  Let us consider the shape of this type.
  \begin{caseof}
    \item $[[iP]] = [[α⁺]] \in [[Γ']]$.
      Then $[[Γ ⊢ σ1 ≈ σ2 : fv iP ∩ Γ']]$ means $[[Γ ⊢ σ1 ≈ σ2 : {α⁺}]]$, 
      i.e. $[[Γ ⊢ [σ1]α⁺ ≈ [σ2]α⁺ ]]$, which holds by assumption.
    \item $[[iP]] = [[α⁺]] \in [[Γ \ Γ']]$.
      Then $[[fv iP ∩ Γ']] = [[∅]]$, 
      so $[[Γ ⊢ σ1 ≈ σ2 : fv iP ∩ Γ']]$ holds vacuously.
    \item $[[iP]] = [[↓iN]]$.
      Then the induction hypothesis is applicable to type $[[iN]]$:
      \begin{enumerate}
        \item $[[iN]]$ is normalized,
        \item $[[Γ ⊢ iN]]$ by inversion of $[[Γ ⊢ ↓iN]]$,
        \item $[[Γ ⊢ [σ1]iN ≈ [σ2]iN]]$ holds by inversion of 
          $[[Γ ⊢ [σ1]↓iN ≈ [σ2]↓iN]]$, i.e. $[[Γ ⊢ ↓[σ1]iN ≈ ↓[σ2]iN]]$.
      \end{enumerate}
      This way, we obtain $[[Γ ⊢ σ1 ≈ σ2 : fv iN ∩ Γ']]$, 
      which implies the required equivalence since 
      $[[fv iP ∩ Γ']] = [[fv ↓iN ∩ Γ']] = [[fv iN ∩ Γ']]$.
    \item $[[iP]] = [[∃nas.iQ]]$
      Then the induction hypothesis is applicable to type $[[iQ]]$ 
      well-formed in context $[[Γ, nas]]$:
      \begin{enumerate}
        \item $[[Γ' ⊆ Γ, nas]]$ since $[[Γ' ⊆ Γ]]$,
        \item $[[Γ, nas ⊢ σi : Γ']]$ by weakening,
        \item $[[iQ]]$ is normalized,
        \item $[[Γ, nas ⊢ iQ]]$ by inversion of $[[Γ ⊢ ∃nas.iQ]]$,
        \item Notice that $[[ [σi]∃nas.iQ ]]$ is normalized, and thus, 
          $[[ [σ1]∃nas.iQ ≈ [σ2]∃nas.iQ]]$ implies 
          $[[ [σ1]∃nas.iQ = [σ2]∃nas.iQ ]]$
          (by \cref{lemma:subt-equiv-algorithmization}).).
          This equality means $[[ [σ1]iQ = [σ2]iQ ]]$, 
          which implies $[[Γ ⊢ [σ1]iQ ≈ [σ2]iQ]]$.
      \end{enumerate}
    \item $[[iN]] = [[iP → iM]]$
  \end{caseof}
\end{proof}



\lemmaSubstCompWell*
\begin{proof}
    By definition of composition.
\end{proof}

\lemmaSubstMonadCompWell*
\begin{proof}
  By definition of monadic composition. 
\end{proof}

\lemmaSubstComp*
\begin{proof}
  \hfill
  \begin{enumerate}
    \item Suppose that $[[α± ∊ Γ1]]$ then $[[α± ∉ Γ2]]$ , and thus, 
      $[[ [(σ2 <=< σ1) ○ σ2]α± = [(σ2 <=< σ1)]α± ]] = [[ [σ2][σ1]α± = [(σ2 ○ σ1)]α± ]]$.
    \item Suppose that $[[α± ∉ Γ1]]$ then $[[ [(σ2 ○ σ1)]α± = [σ2]α± ]]$.  Then 
    \begin{enumerate}
      \item if $[[α± ∉ Γ2]]$ then $[[ [σ2]α± = α± ]]$ and 
        $[[ [(σ2 <=< σ1) ○ σ2]α± = [(σ2 <=< σ1)][σ2]α±]] = [[ [σ2 <=< σ1]α± = α± ]]$
      \item if  $[[α± ∊ Γ2]]$ then $[[ Γ2' ⊢ [σ2]α±  ]]$, and hence, 
        $[[ [(σ2 <=< σ1) ○ σ2]α±  = [(σ2 <=< σ1)][σ2]α±]] = [[ [σ2]α± ]]$
        by definition of monadic composition, since 
        none of the free variables of  $[[ [σ2]α± ]]$ is in $[[Γ1]]$.
    \end{enumerate}
  \end{enumerate}
\end{proof}

\corollarySubstCompCumm*
\begin{proof}
  by \cref{lemma:subst-composition},
    $[[ σ2 ○ σ1 ]] = [[ (σ2 <=< σ1) ○ σ2 ]]$.
    Since the codomain of $[[σ1]]$ is $[[Γ'1]]$,
    and it is disjoint with the domain of $[[σ2]]$,
    $[[σ2 <=< σ1]] = [[σ1]]$.
\end{proof}

\lemmaSubstDomWeak*
\begin{proof}
  If the variable $[[α±]]$ is in $[[Γ1]]$ then $[[Γ2 ⊢ [σ]α± ]]$ by assumption,
  and then $[[Γ2, Γ' ⊢ [σ]α± ]]$ by weakening.
  If the variable $[[α±]]$ is in $[[Γ' \ Γ1]]$ then $[[ [σ]α± = α± ]] \in [[Γ' ⊆ Γ2, Γ']]$, 
  and thus, $[[Γ2, Γ' ⊢ α± ]]$.
\end{proof}

\lemmaSubstFv*
\begin{proof}
  We prove it by structural induction on 
  $[[iP]]$ and mutually, on $[[iN]]$.
  \begin{caseof}
    \item $[[iP = α⁺]]$\\
      If $[[α⁺ ∊ Γ1]]$ then $[[Γ2 ⊢ [σ]α⁺]]$, and by 
      \cref{lemma:wf-soundness}, $[[fv([σ]α⁺) ⊆ Γ2]]$.
      $[[fv(α⁺) \ Γ1]] = [[∅]]$, so $[[fv([σ]iP) \ Γ1  ⊆ fv([σ]α⁺)]]$
      vacuously.

      If $[[α⁺ ∉ Γ1]]$ then $[[ [σ]α⁺ = α⁺ ]]$, and 
      $[[fv([σ]α⁺) = {α⁺}]] = [[{α⁺} \ Γ1]]$.
      
    \item $[[iP = ∃nas.iQ]]$\\
      Then we need to show that $[[fv([σ]iP) = fv([σ]iQ) \ {nas}]]$
      is a subset of $[[(fv(iP) \ Γ1) ∪ Γ2]]$ and a superset of
      $[[fv(iP) \ Γ1]]$.
      Notice that $[[fv(iP) = fv(iQ) \ {nas}]]$ by definition.
      This way, we need to show that 
      $[[fv(iQ) \ {nas} \ Γ1 ⊆ fv([σ]iQ) \ {nas} ⊆ (fv(iQ) \ {nas} \ Γ1) ∪ Γ2]]$,
      
      By the induction hypothesis, 
      $[[fv([σ]iQ) ⊆ (fv(iQ) \ Γ1) ∪ Γ2]]$.
      So for the second inclusion, it suffices to show that
      $[[((fv(iQ) \ Γ1) ∪ Γ2) \ {nas} ⊆ (fv(iQ) \ {nas} \ Γ1) ∪ Γ2]]$,
      which holds by set theoretical reasoning. 

      Also by the induction hypothesis,
      $[[fv(iQ) \ Γ1 ⊆ fv([σ]iQ)]]$,
      and thus, by subtracting $[[{nas}]]$ from both sides,
      $[[fv(iQ) \ {nas} \ Γ1 ⊆ fv([σ]iQ) \ {nas}]]$.

    \item The case $[[iN = ∀pas.iM]]$ is proved analogously.
    \item $[[iN = iP → iM]]$\\
      Then $[[fv([σ]iN) = fv([σ]iP) ∪ fv([σ]iM)]]$.
      By the induction hypothesis, 
      \begin{enumerate}
        \item $[[ fv(iP) \ Γ1 ⊆ fv([σ]iP) ⊆ (fv(iP) \ Γ1) ∪ Γ2]]$ and
        \item $[[ fv(iM) \ Γ1 ⊆ fv([σ]iM) ⊆ (fv(iM) \ Γ1) ∪ Γ2]]$.
      \end{enumerate}
      We unite these inclusions vertically and obtain
      $[[ fv(iP) \ Γ1 ∪ fv(iM) \ Γ1 ⊆ fv([σ]iN) ⊆ ((fv(iP) \ Γ1) ∪ Γ2) ∪ ((fv(iM) \ Γ1) ∪ Γ2)]]$,
      which is equivalent to 
      $[[ (fv(iP) ∪ fv(iM)) \ Γ1 ⊆ fv([σ]iN) ⊆ (fv(iP) ∪ fv(iM)) \ Γ1 ∪ Γ2]]$.
      Since $[[fv(iP) ∪ fv(iM) = fv(iN)]]$, 
      $[[fv(iN) \ Γ1 ⊆ fv([σ]iN) ⊆ (fv(iN) \ Γ1) ∪ Γ2]]$.
    \item The cases when $[[iP = ↓iM]]$ and $[[iN = ↑iQ]]$ are proved
      analogously
  \end{caseof}
\end{proof}

\lemmaSubstFvImage*
\begin{proof}
  By mutual induction on $[[iP]]$ and $[[iN]]$.
  The base cases (when $[[iP]]$ or $[[iN]]$ is a variable) 
  are trivial, since then $[[α± ∊ fv(iP)]]$ means 
  $[[α± = iP]]$ (and symmetrically for $[[iN]]$).
  The congruent cases 
  (when the type is formed by $[[↓]]$, $[[↑]]$, or $[[→]]$)
  hold since $[[α±]]$ occurs in type means that it occurs
  in one of its parts, to which we apply the induction hypothesis.

  Let us suppose that the type is $[[∃nas.iQ]]$.
  Then $[[α± ∊ fv(∃nas.iQ)]]$ means
  $[[α± ∊ fv(iQ)]]$ and $[[α± ∉ {nas}]]$.
  Then by the induction hypothesis, 
  $[[ fv([σ]α±) ⊆ fv([σ]iQ) ]]$,
  and it is left to notice that $[[ fv([σ]α±) ∩ {nas} = ∅ ]]$,
  which we can ensure by alpha-equivalence.
\end{proof}
