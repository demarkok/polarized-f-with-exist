\obsNegSubtDeterministic*
\begin{proof}
    First, notice that the shape of
    the input uniquely determines the rule applied to infer
    $[[Γ ; Θ ⊨ uN ≤ iM ⫤ SC]]$,
    which is consequently, the same rule used to
    infer $[[Γ ; Θ ⊨ uN ≤ iM ⫤ SC']]$.

    Second, notice that for each of the inference rules, 
    the premises are deterministic on the input.
    Specifically, 
    \begin{itemize}
        \item \ruleref{\ottdruleAShiftULabel} relies on unification,
            which is deterministic by \cref{obs:unif-deterministic};
        \item \ruleref{\ottdruleAForallLabel} relies on
            the choice of fresh algorithmic variables,
            which is deterministic as discussed in \cref{sec:fresh-selection},
            and on the negative subtyping, which is deterministic by
            the induction hypothesis;
        \item \ruleref{\ottdruleAArrowLabel} uses 
            the negative subtyping 
            (deterministic by the induction hypothesis),
            the positive subtyping 
            (\cref{obs:pos-subt-deterministic}),
            and the merge of subtyping constraints
            (\cref{obs:subt-merge-deterministic});
    \end{itemize}
\end{proof}


\lemNegSubtSoundness*
\begin{proof}
    We prove it by induction on $[[Γ ; Θ ⊨ uN ≤ iM ⫤ SC]]$.

    Suppose that $[[uσ]]$ is normalized and $[[ Θ  ⊢ uσ : SC ]]$,
    Let us consider the last rule to infer this judgment. 
    \begin{caseof}
        \item \ruleref{\ottdruleAArrowLabel}. Then $[[Γ ; Θ ⊨ uN ≤ iM ⫤ SC]]$ has shape
        $[[G;Θ ⊨ uP → uN' ≤ iQ → iM' ⫤ SC]]$\\
        On the next step, the the algorithm makes two recursive calls:
        $[[G;Θ ⊨ uP ≥ iQ ⫤ SC1]]$ and $[[G;Θ ⊨ uN' ≤ iM' ⫤ SC2]]$
        and returns $[[Θ ⊢ SC1 & SC2 = SC]]$ as the result.

        By the soundness of constraint merge (\cref{lemma:merge-soundness}),
        $[[ Θ  ⊢ uσ : SC1 ]]$ and $[[ Θ  ⊢ uσ : SC2 ]]$.
        Then by the soundness of positive subtyping (\cref{lemma:pos-subt-soundness}), $[[ Γ ⊢ [uσ]uP ≥ iQ ]]$; 
        and by the induction hypothesis, $[[ Γ ⊢ [uσ]uN' ≤ iM' ]]$.
        This way, by \ruleref{\ottdruleDOneArrowLabel}, $[[Γ ⊢ [uσ](uP → uN') ≤ iQ → iM']]$.

        \item \ruleref{\ottdruleANVarLabel}, and then $[[Γ ; Θ ⊨ uN ≤ iM ⫤ SC]]$ has shape $[[G;Θ ⊨ a⁻ ≤ a⁻ ⫤ ·]]$\\
        This case is symmetric to \cref{case:pos-subt-soundness:var} of \cref{lemma:pos-subt-soundness}.

        \item \ruleref{\ottdruleAShiftULabel}, and then $[[Γ ; Θ ⊨ uN ≤ iM ⫤ SC]]$ has shape $[[G;Θ ⊨ ↑uP ≤ ↑iQ ⫤ SC]]$\\
        This case is symmetric to \cref{case:pos-subt-soundness:shift} of \cref{lemma:pos-subt-soundness}.

        \item \ruleref{\ottdruleAForallLabel}, and then $[[Γ ; Θ ⊨ uN ≤ iM ⫤ SC]]$ has shape
         $[[G;Θ ⊨ ∀pas.uN' ≤ ∀pbs.iM' ⫤ SC]]$ s.t. either $[[pas]]$ or $[[pbs]]$ is not empty\\
        This case is symmetric to \cref{case:pos-subt-soundness:exists} of \cref{lemma:pos-subt-soundness}.

\end{caseof}
\end{proof}

\lemNegSubtCompleteness*
\begin{proof}
    We prove it by induction on $[[ Γ ⊢ [uσ]uN ≤ iM ]]$.
    Let us consider the last rule used in the derivation of $[[ Γ ⊢ [uσ]uN ≤ iM ]]$.
    \begin{caseof}
        \item $[[ Γ ⊢ [uσ]uN ≤ iM ]]$ is derived by \ruleref{\ottdruleDOneShiftULabel}\\
            Then $[[ uN ]] = [[ ↑uP ]]$, since
            the substitution $[[ [uσ]uN ]]$ must preserve the 
            top-level constructor of $[[uN]] \neq [[α̂⁻]]$ (since by assumption, $[[α̂⁻]] \notin [[uv uN]]$), 
            and $[[uQ]] = [[ ↓iM ]]$, and by inversion, $[[ Γ ⊢ [uσ]uN ≈ iM ]]$.
            The rest of the proof is symmetric to \cref{case:pos-subt-complete-upshift} of
            \cref{lemma:pos-subt-completeness}: notice that the algorithm does not make a recursive call, 
            and the difference in the induction statement for the positive and 
            the negative case here does not matter.

        \item $[[ Γ ⊢ [uσ]uN ≤ iM ]]$ is derived by \ruleref{\ottdruleDOneArrowLabel}, 
            i.e. $[[ [uσ]uN ]] = [[ [uσ]uP → [uσ]uN' ]]$ and $[[iM]] = [[iQ → iM']]$, 
            and by inversion, $[[ Γ ⊢ [uσ]uP ≥ iQ ]]$ and $[[ Γ ⊢ [uσ]uN' ≤ iM' ]]$.

            The algorithm makes two recursive calls: $[[Γ ; Θ ⊨ uP ≥ iQ ⫤ SC1]]$ and $[[Γ ; Θ ⊨ uN' ≤ iM' ⫤ SC2]]$,
            and then returns $[[Θ ⊢ SC1 & SC2 = SC]]$ as the result.
            Let us show that these recursive calls are successful and the returning constraints 
            are fulfilled by $[[uσ]]$.

            Notice that from the inversion of $[[Γ ⊢ iM]]$, we have: $[[Γ ⊢ iQ]]$ and $[[Γ ⊢ iM']]$;
            from the inversion of $[[Γ ; dom(Θ) ⊢ uN]]$, we have: $[[Γ ; dom( Θ) ⊢  uP]]$ and $[[Γ ; dom( Θ) ⊢  uN']]$;
            and since $[[uN]]$ does not contain negative unification variables,
            $[[uN']]$ does not contain negative unification variables either.

            This way, we can apply the induction hypothesis to $[[Γ ⊢ [uσ]uN' ≤ iM']]$ to 
            obtain $[[Γ ; Θ ⊨ uN' ≤ iM' ⫤ SC2]]$ such that $[[Θ ⊢ SC2 : uv(uN')]]$ and $[[ Θ ⊢ uσ : SC2 ]]$.
            Also, we can apply the completeness of the positive subtyping (\cref{lemma:pos-subt-completeness}) to 
            $[[ Γ ⊢ [uσ]uP ≥ iQ ]]$ to obtain $[[Γ ; Θ ⊨ uP ≥ iQ ⫤ SC1]]$ such that $[[Θ ⊢ SC1 : uv(uP)]]$
            and $[[ Θ ⊢ uσ : SC1 ]]$.

            Finally, we need to show that the merge of $[[SC1]]$ and $[[SC2]]$ is successful and
            satisfies the required properties.
            To do so, we apply the completeness of subtyping constraint merge (\cref{lemma:merge-completeness})
            (notice that $[[Θ ⊢ uσ : uv(uP → uN')]]$ means 
            $[[Θ ⊢ uσ : uv(uP) ∪ uv(uN')]]$).
            This way, $[[Θ ⊢ SC1 & SC2 = SC]]$ is defined and $[[ Θ ⊢ uσ : SC ]]$ holds. 

       \item \label{case:subt-complete-forall}
            $[[ Γ ⊢ [uσ]uN ≤ iM ]]$ is derived by \ruleref{\ottdruleDOneForallLabel}.
            Since $[[uN]]$ does not contain negative unification variables,
            $[[uN]]$ must be of the form $[[∀pas.uN']]$,
            such that $[[ [uσ]uN = ∀pas.[uσ]uN' ]]$ and $[[ [uσ]uN']] \neq [[∀]]\dots$
            (assuming $[[pas]]$ does not intersect with the range of $[[uσ]]$).
            Also, $[[iM]] = [[∀pbs.iM']]$ and either $[[pas]]$ or $[[pbs]]$ is non-empty.

            The rest of the proof is symmetric to \cref{case:pos-subt-complete-exists} of
            \cref{lemma:pos-subt-completeness}.
            To apply the induction hypothesis, we need to show additionally that
            there are no negative unification variables in $[[uN0]] = [[ [â⁺*/pas]uN' ]]$.
            This is because $[[ uv uN0 ⊆ uv uN ∪ {â⁺*} ]]$, and $[[uN]]$ is free of negative
            unification variables by assumption.

       \item $[[ Γ ⊢ [uσ]uN ≤ iM ]]$ is derived by \ruleref{\ottdruleDOneNVarLabel}.\\
            Then $[[iN]] = [[ [uσ]uN ]] = [[ α⁻ ]] = [[iM]]$. 
            Here the first equality holds because $[[uN]]$ is not a unification variable:
            by assumption, $[[uN]]$ is free of negative unification variables.
            The second and the third equations hold because \ruleref{\ottdruleDOneNVarLabel}
            was applied. 

            The rest of the proof is symmetric to \cref{case:pos-subt-complete-pvar} of
            \cref{lemma:pos-subt-completeness}.

    \end{caseof}
\end{proof}


