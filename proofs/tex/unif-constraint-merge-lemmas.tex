\obsUnifMergeDet*
\begin{proof}
    $[[UC]]$ and $[[UC']]$ both consists of three parts: 
    Entries of $[[UC1]]$ that do not have matching entries in $[[UC2]]$,
    entries of $[[UC2]]$ that do not have matching entries in $[[UC1]]$,
    and the merge of matching entries.

    The parts corresponding to unmatched entries of $[[UC1]]$ and $[[UC2]]$ coincide, 
    since $[[UC1]]$ and $[[UC2]]$ are fixed.
    To show that the merge of matching entries coincide,
    let us take any pair of matching $[[ucE1 ∊ UC1]]$ and $[[ucE2 ∊ UC2]]$
    and consider their shape.
    \begin{caseof}
        \item $[[ucE1]]$ is $[[pua :≈ iQ1]]$ and $[[ucE2]]$ is $[[pua :≈ iQ2]]$
            then the result, if it exists, is always $[[ucE1]]$,
            by inversion of \ruleref{\ottdruleSCMEPEqEqLabel}.
        \item $[[ucE1]]$ is $[[nua :≈ iN1]]$ and $[[ucE2]]$ is $[[nua :≈ iN2]]$
            then analogously, the result, if it exists, is always $[[ucE1]]$,
            by inversion of \ruleref{\ottdruleSCMENEqEqLabel}.
    \end{caseof}
    This way, the third group of entries coincide as well.
\end{proof}

\lemUnifMergeSoundness*
\begin{proof}
    \hfill
    \begin{itemize}
        \item $[[UC1 & UC2]] \subseteq [[UC1]] \cup [[UC2]]$\\
        By definition, 
        $[[UC1 & UC2]]$ consists of three parts:
        entries of $[[UC1]]$ that do not have matching entries of $[[UC2]]$,
        entries of $[[UC2]]$ that do not have matching entries of $[[UC1]]$,
        and the merge of matching entries.

        If $[[ucE]]$ is from the first or the second part, 
        then $[[ucE]] \in [[UC1]] \cup [[UC2]]$ holds immediately.
        If $[[ucE]]$ is from the third part,
        then $[[ucE]]$ is the merge of two matching entries
        $[[ucE1]] \in [[UC1]]$ and $[[ucE2]] \in [[UC2]]$.
        Since $[[UC1]]$ and $[[UC2]]$ are normalized unification , 
        $[[ucE1]]$ and $[[ucE2]]$ have one of the following forms:
        \begin{itemize}
            \item $[[α̂⁺ :≈ iP1]]$ and $[[α̂⁺ :≈ iP2]]$, 
                where $[[iP1]]$ and $[[iP2]]$ are normalized,
                and then since $[[Θ(α̂⁺) ⊢ ucE1 & ucE2 = ucE]]$ exists, 
                \ruleref{\ottdruleSCMEPEqEqLabel} was applied to infer it.
                It means that $[[ucE]] = [[ucE1]] = [[ucE2]]$;
            \item $[[α̂⁻ :≈ iN1]]$ and $[[α̂⁻ :≈ iN2]]$, 
               then symmetrically, 
               $[[Θ(α̂⁻) ⊢ ucE1 & ucE2 = ucE]] = [[ucE1]] = [[ucE2]]$
        \end{itemize}
        In both cases, $[[ucE]] \in [[UC1]] \cup [[UC2]]$.

        \item $[[UC1]] \cup [[UC2]] \subseteq [[UC1 & UC2]]$\\
        Let us take 
        an arbitrary $[[ucE1]] \in [[UC1]]$.
        Then since $[[UC1]]$ is a unification constraint,
         $[[ucE1]]$ has one of the following forms:
        \begin{itemize}
            \item $[[α̂⁺ :≈ iP]]$ where $[[iP]]$ is normalized.
            If $[[α̂⁺]] \notin [[dom(UC2)]]$, then $[[ucE1]] \in [[UC1 & UC2]]$.
            Otherwise, there is a normalized matching
            $[[ucE2]] = [[(α̂⁺ :≈ iP')]] \in [[UC2]]$ and then
            since $[[UC1 & UC2]]$ exists, 
            \ruleref{\ottdruleSCMEPEqEqLabel} was applied to construct
            $[[ucE1 & ucE2]] \in [[UC1 & UC2]]$.
            By inversion of \ruleref{\ottdruleSCMEPEqEqLabel},
            $[[ucE1 & ucE2]] = [[ucE1]]$, and
            $[[nf(iP) = nf(iP')]]$, which since $[[iP]]$
            and $[[iP']]$ are normalized, implies that $[[iP = iP']]$, 
            that is $[[ucE1]] = [[ucE2]] \in [[UC1 & UC2]]$.
            \item $[[α̂⁻ :≈ iN]]$ where $[[iN]]$ is normalized.
            Then symmetrically, $[[ucE1]] = [[ucE2]] \in [[UC1 & UC2]]$.
        \end{itemize}
        Similarly, if we take an arbitrary $[[ucE2]] \in [[UC2]]$,
        then $[[ucE1]] = [[ucE2]] \in [[UC1 & UC2]]$. 
    \end{itemize}
\end{proof}

\corUnifMergeSoundness*
\begin{proof}
    It is clear that since $[[UC = UC1 ∪ UC2]]$ (by \cref{lemma:unif-merge-soundness}),
    and being normalized means that all entries are normalized,
    $[[UC]]$ is a normalized unification constraint.
    Analogously, $[[Θ ⊢ UC]] = [[UC1 ∪ UC2]]$ holds immediately, 
    since $[[Θ ⊢ UC1]]$ and $[[Θ ⊢ UC2]]$.

    Let us take an arbitrary substitution $[[Θ ⊢ uσ : dom(UC)]]$ and assume that 
    $[[ Θ   ⊢ uσ : lift UC ]]$.
    Then $[[ Θ   ⊢ uσ : lift UCi ]]$ holds by definition:
    If $[[ucE]] \in [[lift UCi]] \subseteq [[lift UC1 ∪ lift UC2]] = [[lift UC]]$ then
    $[[Θ(α̂±) ⊢ [uσ]α̂± : ucE]]$ (where $[[ucE]]$ restricts $[[α̂±]]$) holds since $[[Θ ⊢ uσ : dom(UC)]]$.
\end{proof}

\lemUnifEntryMergeCompleteness*
\begin{proof}
    Let us consider the shape of $[[ucE1]]$ and $[[ucE2]]$.
    \begin{caseof}
        \item $[[ucE1]]$ is $[[pua :≈ iQ1]]$ and $[[ucE2]]$ is $[[pua :≈ iQ2]]$.
            Then $[[Γ ⊢ iP : ucE1]]$ means $[[Γ ⊢ iP ≈ iQ1]]$, 
            and $[[Γ ⊢ iP : ucE2]]$ means $[[Γ ⊢ iP ≈ iQ2]]$.
            Then by transitivity of equivalence (\cref{corollary:equivalence-transitivity}),
            $[[Γ ⊢ iQ1 ≈ iQ2]]$, which means $[[nf(iQ1) = nf(iQ2)]]$ by
            \cref{lemma:subt-equiv-algorithmization}.
            Hence, \ruleref{\ottdruleSCMEPEqEqLabel} applies to infer
            $[[Γ ⊢ ucE1 & ucE2 = ucE2]]$, and $[[Γ ⊢ iP : ucE2]]$ holds by assumption.
        \item $[[ucE1]]$ is $[[nua :≈ iN1]]$ and $[[ucE2]]$ is $[[nua :≈ iM2]]$.
            The proof is symmetric.
    \end{caseof}
\end{proof}

\lemUnifMergeCompleteness*
\begin{proof}
    The proof repeats the proof of \cref{lemma:merge-completeness}
    for cases 
    uses \cref{lemma:unif-entry-merge-completeness} instead of \cref{lemma:entry-merge-completeness}.
\end{proof}
