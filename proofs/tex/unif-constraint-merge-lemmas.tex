\begin{lemma} [Soundness of Unification Constraint Merge]
    \label{lemma:unif-merge-soundness}
    Suppose that $[[Θ ⊢ UC1]]$ and $[[Θ ⊢ UC2]]$ 
    are normalized unification constraints.
    If $[[Θ ⊢ UC1 & UC2 = UC]]$ is defined then
    $[[UC = UC1 ∪ UC2]]$.
\end{lemma}
\begin{proof}
    \hfill
    \begin{itemize}
        \item $[[UC1 & UC2]] \subseteq [[UC1]] \cup [[UC2]]$\\
        By definition, 
        $[[UC1 & UC2]]$ consists of three parts:
        entries of $[[UC1]]$ that do not have matching entries of $[[UC2]]$,
        entries of $[[UC2]]$ that do not have matching entries of $[[UC1]]$,
        and the merge of matching entries.

        If $[[ucE]]$ is from the first or the second part, 
        then $[[ucE]] \in [[UC1]] \cup [[UC2]]$ holds immediately.
        If $[[ucE]]$ is from the third part,
        then $[[ucE]]$ is the merge of two matching entries
        $[[ucE1]] \in [[UC1]]$ and $[[ucE2]] \in [[UC2]]$.
        Since $[[UC1]]$ and $[[UC2]]$ are normalized unification , 
        $[[ucE1]]$ and $[[ucE2]]$ have one of the following forms:
        \begin{itemize}
            \item $[[α̂⁺ :≈ iP1]]$ and $[[α̂⁺ :≈ iP2]]$, 
                where $[[iP1]]$ and $[[iP2]]$ are normalized,
                and then since $[[Θ(α̂⁺) ⊢ ucE1 & ucE2 = ucE]]$ exists, 
                \ruleref{\ottdruleUCMEPEqEqLabel} was applied to infer it.
                It means that $[[ucE]] = [[ucE1]] = [[ucE2]]$;
            \item $[[α̂⁻ :≈ iN1]]$ and $[[α̂⁻ :≈ iN2]]$, 
               then symmetrically, 
               $[[Θ(α̂⁻) ⊢ ucE1 & ucE2 = ucE]] = [[ucE1]] = [[ucE2]]$
        \end{itemize}
        In both cases, $[[ucE]] \in [[UC1]] \cup [[UC2]]$.

        \item $[[UC1]] \cup [[UC2]] \subseteq [[UC1 & UC2]]$\\
        Let us take 
        an arbitrary $[[ucE1]] \in [[UC1]]$.
        Then since $[[UC1]]$ is a unification constraint,
         $[[ucE1]]$ has one of the following forms:
        \begin{itemize}
            \item $[[α̂⁺ :≈ iP]]$ where $[[iP]]$ is normalized.
            If $[[α̂⁺]] \notin [[dom(UC2)]]$, then $[[ucE1]] \in [[UC1 & UC2]]$.
            Otherwise, there is a normalized matching
            $[[ucE2]] = [[(α̂⁺ :≈ iP')]] \in [[UC2]]$ and then
            since $[[UC1 & UC2]]$ exists, 
            \ruleref{\ottdruleUCMEPEqEqLabel} was applied to construct
            $[[ucE1 & ucE2]] \in [[UC1 & UC2]]$.
            By inversion of \ruleref{\ottdruleUCMEPEqEqLabel},
            $[[ucE1 & ucE2]] = [[ucE1]]$, and
            $[[nf(iP) = nf(iP')]]$, which since $[[iP]]$
            and $[[iP']]$ are normalized, implies that $[[iP = iP']]$, 
            that is $[[ucE1]] = [[ucE2]] \in [[UC1 & UC2]]$.
            \item $[[α̂⁻ :≈ iN]]$ where $[[iN]]$ is normalized.
            Then symmetrically, $[[ucE1]] = [[ucE2]] \in [[UC1 & UC2]]$.
        \end{itemize}
        Similarly, if we take an arbitrary $[[ucE2]] \in [[UC2]]$,
        then $[[ucE1]] = [[ucE2]] \in [[UC1 & UC2]]$. 
    \end{itemize}
\end{proof}

\begin{corollary}
    \label{corollary:unif-merge-soundness}
    Suppose that $[[Θ ⊢ UC1]]$ and $[[Θ ⊢ UC2]]$ 
    are normalized unification constraints.
    If $[[Θ ⊢ UC1 & UC2 = UC]]$ is defined then
    \begin{enumerate}
        \item $[[Θ ⊢ UC]]$ is normalized unification constraint,
        \item for any substitution $[[Θ ⊢ uσ]]$, $[[ Θ   ⊢ uσ : lift UC ]]$ implies 
        $[[ Θ   ⊢ uσ : lift UC1 ]]$ and $[[ Θ   ⊢ uσ : lift UC2 ]]$.
    \end{enumerate}
\end{corollary}
\begin{proof}
    It is clear that since $[[UC = UC1 ∪ UC2]]$ (by \cref{lemma:unif-merge-soundness}),
    and being normalized means that all entries are normalized,
    $[[UC]]$ is a normalized unification constraint.
    Analogously, $[[Θ ⊢ UC]] = [[UC1 ∪ UC2]]$ holds immediately, 
    since $[[Θ ⊢ UC1]]$ and $[[Θ ⊢ UC2]]$.

    Let us take an arbitrary substitution $[[Θ ⊢ uσ]]$ and assume that 
    $[[ Θ   ⊢ uσ : lift UC ]]$.
    Then $[[ Θ   ⊢ uσ : lift UCi ]]$ holds by definition:
    If $[[ucE]] \in [[lift UCi]] \subseteq [[lift UC1 ∪ lift UC2]] = [[lift UC]]$.
    So $[[Θ(α̂±) ⊢ [uσ]α̂± : ucE]]$ holds.
\end{proof}


% \begin{lemma} [Completeness of Unification Constraint Merge]
%     \label{lemma:unif-merge-completeness}
%     Suppose that $[[Θ ⊢ UC|varset1]]$ and $[[Θ ⊢ UC|varset2]]$.
%     Then $[[Θ ⊢ UC|varset1 & UC|varset2 = UC']]$ exists and
%     $[[UC' = UC|varset1 ∪ varset2]]$.
% \end{lemma}
% \begin{proof}
%     $[[Θ ⊢ UC|varset1 & UC|varset2 = UC']]$ is defined as the union of three parts:
%     entries of $[[UC|varset1]]$ that do not have matching entries of $[[UC|varset2]]$,
%     entries of $[[UC|varset2]]$ that do not have matching entries of $[[UC|varset1]]$,
%     and the merge of matching entries.
%     The first two parts are defined. The merge of matching entries is defined
%     by \ruleref{\ottdruleUCMEPEqEqLabel}, since the matching entries must be equal
%     if they both belong to $[[UC]]$.

%     It remains to show that $[[UC' = UC|varset1 ∪ varset2]]$.
%     It is easy to see that the three parts comprising $[[UC']]$ 
%     correspond to the three parts comprising 
%     $[[UC|varset1 ∪ varset2]] = [[UC | (varset1 \ varset2) ∪ UC | (varset2 \ varset1) ∪ UC | varset1 ∩ varset2]]$. 
% \end{proof}

\begin{lemma} [Completeness of Unification Constraint Entry Merge]
    \label{lemma:unif-entry-merge-completeness}
    For a fixed context $[[Γ]]$,
    suppose that $[[Γ ⊢ ucE1]]$ and $[[Γ ⊢ ucE2]]$ are matching constraint entries.
    \begin{itemize}
        \item for a type $[[iP]]$ such that $[[Γ ⊢ iP : ucE1]]$ and $[[Γ ⊢ iP : ucE2]]$,
        $[[Γ ⊢ ucE1 & ucE2 = ucE]]$ is defined and $[[Γ ⊢ iP : ucE]]$.
        \item for a type $[[iN]]$ such that $[[Γ ⊢ iN : ucE1]]$ and $[[Γ ⊢ iN : ucE2]]$,
        $[[Γ ⊢ ucE1 & ucE2 = ucE]]$ is defined and $[[Γ ⊢ iN : ucE]]$.
    \end{itemize}
\end{lemma}
\begin{proof}
    The proof repeats the one of \cref{lemma:entry-merge-completeness}
    and is done by the case analysis on the shape of $[[ucE1]]$ and $[[ucE2]]$.
    However, it only needs to consider two cases.
    \begin{caseof}
        \item $[[ucE1]]$ is $[[pua :≈ iQ1]]$ and $[[ucE2]]$ is $[[pua :≈ iQ2]]$.
        \item $[[ucE1]]$ is $[[nua :≈ iN1]]$ and $[[ucE2]]$ is $[[nua :≈ iM2]]$.
    \end{caseof}
    The proof of these cases is based only on \cref{lemma:subt-equiv-algorithmization}
    and \cref{corollary:equivalence-transitivity}, and does not require the properties of 
    the least upper bound or subtyping.
\end{proof}

\begin{lemma} [Completeness of Unification Constraint Merge] 
    \label{lemma:unif-merge-completeness}
    Suppose that $[[Θ ⊢ UC1]]$ and $[[Θ ⊢ UC2]]$.
    Then for any substitution $[[Θ ⊢ uσ]]$ such that $[[ Θ   ⊢ uσ : lift UC1 ]]$ and $[[ Θ   ⊢ uσ : lift UC2 ]]$, 
    \begin{enumerate}
        \item $[[Θ ⊢ UC1 & UC2 = UC]]$ is defined and
        \item $[[ Θ   ⊢ uσ : UC ]]$.
    \end{enumerate}
\end{lemma}
\begin{proof}
    The proof repeats the proof of \cref{lemma:merge-completeness} but 
    uses \cref{lemma:unif-entry-merge-completeness} instead of \cref{lemma:entry-merge-completeness}.
\end{proof}
