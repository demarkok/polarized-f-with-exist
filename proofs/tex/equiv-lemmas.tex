\begin{lemma}[Declarative equivalence is transitive]
  \hfill
  \label{lemma:decl-equiv-transitivity}
  \begin{itemize}
  \item[$+$] if $[[iP1 ≈ iP2]]$ and $[[iP2 ≈ iP3]]$ then $[[iP1 ≈ iP3]]$,
  \item[$-$] if $[[iN1 ≈ iN2]]$ and $[[iN2 ≈ iN3]]$ then $[[iN1 ≈ iN3]]$.
  \end{itemize}
\end{lemma}
\begin{proof}
  \ilyam{should be easy to do by induction since the types are getting smaller}
\end{proof}

\begin{lemma}[Algorithmization of declarative equivalence]
  \label{lemma:decl-equiv-algorithmization}
  Declarative equivalence is equality of normal forms. 
  \begin{itemize}
    \item[$+$] $[[iP ≈ iQ]] \iff [[nf(iP) = nf(iQ)]]$,
    \item[$-$] $[[iN ≈ iM]] \iff [[nf(iN) = nf(iM)]]$.
  \end{itemize}
\end{lemma}
\begin{proof} \hfill
  \begin{itemize}
    \item[$+$] Let us prove both directions separately.
    \begin{itemize}
      \item[$\Rightarrow$] 
        exactly by \cref{lemma:normalization-completeness},
      \item[$\Leftarrow$] 
        from \cref{lemma:normalization-soundness}, we know
        $[[iP ≈ nf(iP)]] = [[nf(iQ) ≈ iQ]]$, then by transitivity (\cref{lemma:decl-equiv-transitivity}),
        $[[iP ≈ iQ]]$.
    \end{itemize}
    \item[$-$] The proof is exactly the same.
  \end{itemize}
\end{proof}

\begin{lemma}[Type well-formedness is invariant under equivalence]
  \label{lemma:wf-equiv}
  Mutual subtyping implies declarative equivalence.
  \begin{itemize}
  \item[$+$] if $[[iP ≈ iQ]]$ then $[[Γ ⊢ iP]] \iff [[Γ ⊢ iQ]]$,
  \item[$-$] if $[[iN ≈ iM]]$ then $[[Γ ⊢ iN]] \iff [[Γ ⊢ iM]]$
  \end{itemize}
\end{lemma}
\begin{proof}
  \ilyam{todo}
\end{proof}

\begin{corollary}[Normalization preserves well-formedness]
  \label{corollary:wf-nf}
  \hfill
  \begin{itemize}
  \item[$+$] $[[Γ ⊢ iP]] \iff [[Γ ⊢ nf(iP)]]$,
  \item[$-$] $[[Γ ⊢ iN]] \iff [[Γ ⊢ nf(iN)]]$
  \end{itemize}
\end{corollary}
\begin{proof}
  Immediately from \cref{lemma:wf-equiv,lemma:normalization-soundness}.
\end{proof}

\begin{corollary}[Normalization preserves well-formedness of substitution]
  \label{corollary:wf-s-nf}
  \hfill \\
   $[[Γ2 ⊢ σ : Γ1]] \iff [[Γ2 ⊢ nf(σ) : Γ1]]$
\end{corollary}

\begin{lemma}[Soundness of equivalence]
  \label{lemma:equiv-soundness}
  Declarative equivalence implies mutual subtyping.
  \begin{itemize}
    \item[$+$] if $[[Γ ⊢ iP]]$, $[[Γ ⊢ iQ]]$, and $[[iP ≈ iQ]]$ then $[[Γ ⊢ iP ≈ iQ]]$,
    \item[$-$] if $[[Γ ⊢ iN]]$, $[[Γ ⊢ iM]]$, and $[[iN ≈ iM]]$ then $[[Γ ⊢ iN ≈ iM]]$.
  \end{itemize}
\end{lemma}
\begin{proof}
  We prove it by mutual induction on $[[iP ≈ iQ]]$ and $[[iN ≈ iM]]$.
  \begin{caseof}
    \item $[[a⁻ ≈ a⁻]]$\\
      Then $[[Γ ⊢ a⁻ ≤ a⁻]]$ by \ruleref{\ottdruleDOneNVarLabel},
      which immediately implies $[[Γ ⊢ a⁻ ≈ a⁻]]$ by \ruleref{\ottdruleDOneNDefLabel}.

    \item $[[↑iP ≈ ↑iQ]]$\\
      Then by inversion of \ruleref{\ottdruleDOneShiftULabel},
      $[[iP ≈ iQ]]$, and from the induction hypothesis, $[[Γ ⊢ iP ≈ iQ]]$,
      and (by symmetry) $[[Γ ⊢ iQ ≈ iP]]$.

      When \ruleref{\ottdruleDOneShiftULabel} is applied to $[[Γ ⊢ iP ≈ iQ]]$,
      it gives us $[[Γ ⊢ ↑iP ≤ ↑iQ]]$; when it is applied to $[[Γ ⊢ iQ ≈ iP]]$,
      we obtain $[[Γ ⊢ ↑iQ ≤ ↑iP]]$. Together, it implies $[[Γ ⊢ ↑iP ≈ ↑iQ]]$.

    \item $[[iP → iN ≈ iQ → iM]]$\\
      Then by inversion of \ruleref{\ottdruleDOneArrowLabel},
      $[[iP ≈ iQ]]$ and $[[iN ≈ iM]]$. By the induction hypothesis,
      $[[Γ ⊢ iP ≈ iQ]]$ and $[[Γ ⊢ iN ≈ iM]]$, which means by inversion:
      \begin{enumerate*}
        \item[(i)] $[[Γ ⊢ iP ≥ iQ]]$,
        \item[(ii)] $[[Γ ⊢ iQ ≥ iP]]$,
        \item[(iii)] $[[Γ ⊢ iN ≤ iM]]$,
        \item[(iv)]  $[[Γ ⊢ iM ≤ iN]]$.
      \end{enumerate*}
      Applying \ruleref{\ottdruleDOneArrowLabel} to (i) and (iii), we obtain
      $[[Γ ⊢ iP → iN ≤ iQ → iM]]$; applying it to (ii) and (iv), we have $[[Γ ⊢
      iQ → iM ≤ iP → iN]]$. Together, it implies $[[Γ ⊢ iP → iN ≈ iQ → iM]]$.
    \item $[[∀pas.iN ≈ ∀pbs.iM]]$\\
      Then by inversion, there exists bijection $[[mu : ({pbs} ∩ fv iM) ↔ ({pas}
      ∩ fv iN)]]$, such that $[[iN ≈ [mu] iM]]$. By the induction hypothesis,
      $[[Γ, pas ⊢ iN ≈ [mu] iM]]$. From \cref{corollary:subst-pres-equiv} and
      the fact that $[[mu]]$ is bijective, we also have
      $[[Γ, pbs ⊢ [mu-1]iN ≈ iM]]$.

      Let us construct a subsitution $[[pas ⊢ iPs/pbs : pbs]]$ by
      extending $[[mu]]$ with arbitrary positive types on $[[{pbs} \ fv iM]]$.

      Notice that $[[ [mu]iM ]] = [[ [iPs/pbs]iM ]]$, and therefore,
      $[[Γ, pas ⊢ iN ≈ [mu] iM]]$ implies $[[Γ, pas ⊢ [iPs/pbs]iM ≤ iN]]$. Then by
      \ruleref{\ottdruleDOneForallLabel}, $[[Γ ⊢ ∀pbs.iM ≤ ∀pas.iN]]$.

      Analogously, we construct the substitution from $[[mu-1]]$, and use it to
      instantiate $[[pas]]$ in the application of
      \ruleref{\ottdruleDOneForallLabel} to infer $[[Γ ⊢ ∀pas.iN ≤ ∀pbs.iM]]$.

      This way, $[[Γ ⊢ ∀pbs.iM ≤ ∀pas.iN]]$ and $[[Γ ⊢ ∀pas.iN ≤ ∀pbs.iM]]$
      gives us $[[Γ ⊢ ∀pbs.iM ≈ ∀pas.iN]]$.

    \item For the cases of the positive types, the proofs are symmetric.
  \end{caseof}
\end{proof}

\begin{corollary}[Normalization is sound w.r.t. subtyping-induced equivalence] \label{corollary:nf-sound-wrt-subt-equiv}
  \hfill
  \begin{itemize}
    \item [$+$] if $[[Γ ⊢ iP]]$ then $[[Γ ⊢ iP ≈ nf(iP)]]$,
    \item [$-$] if $[[Γ ⊢ iN]]$ then $[[Γ ⊢ iN ≈ nf(iN)]]$.
  \end{itemize}
\end{corollary}
\begin{proof}
  Immediately from \cref{lemma:normalization-soundness,corollary:wf-nf,lemma:equiv-soundness}.
\end{proof}

\begin{corollary}[Normalization preserves subtyping] 
  \label{corollary:nf-pres-subt}
  Assuming all the types are well-formed in context $[[Γ]]$,
  \begin{itemize}
    \item [$+$] $[[Γ ⊢ iP ≥ iQ]] \iff [[Γ ⊢ nf(iP) ≥ nf(iQ)]]$,
    \item [$-$] $[[Γ ⊢ iN ≤ iM]] \iff [[Γ ⊢ nf(iN) ≤ nf(iM)]]$.
  \end{itemize}
\end{corollary}
\begin{proof}
  \hfill
  \begin{itemize}
    \item [$+$]  
    \begin{itemize}
      \item [$\Rightarrow$] Let us assume $[[Γ ⊢ iP ≥ iQ]]$.
        By \cref{corollary:nf-sound-wrt-subt-equiv},
        $[[Γ ⊢ iP ≈ nf(iP)]]$ and $[[Γ ⊢ iQ ≈ nf(iQ)]]$, 
        in particular, by inversion, 
        $[[Γ ⊢ nf(iP) ≥ iP]]$ and $[[Γ ⊢ iQ ≥ nf(iQ)]]$.
        Then by the transitivity of subtyping 
        (\cref{corollary:subtyping-transitivity}), 
        $[[Γ ⊢ nf(iP) ≥ nf(iQ)]]$.
      \item [$\Leftarrow$] Let us assume $[[Γ ⊢ nf(iP) ≥ nf(iQ)]]$.
        Also by \cref{corollary:nf-sound-wrt-subt-equiv}
        and inversion, 
        $[[Γ ⊢ iP ≥ nf(iP)]]$ and $[[Γ ⊢ nf(iQ) ≥ iQ]]$.
        Then by the transitivity, $[[Γ ⊢ iP ≥ iQ]]$.
    \end{itemize}
    \item [$-$] The negative case is proved symmetrically.
  \end{itemize}
\end{proof}

\begin{lemma}[Subtyping induced by disjoint substitutions]
  \label{lemma:subt-ind-disj-subst}
  If two disjoint substitutions induce subtyping, they are degenerate (so is the
  subtyping).
  Suppose that $[[Γ ⊢ σ1 : Γ1]]$ and $[[Γ ⊢ σ2 : Γ1]]$,
  where $[[{Γi} ⊆ {Γ}]]$ and $[[{Γ1} ∩ {Γ2}= ∅]]$. Then
  \begin{itemize}
  \item[$-$] assuming $[[Γ ⊢ iN]]$,~
    $[[Γ ⊢ [σ1]iN ≤ [σ2]iN]]$ implies $[[Γ ⊢ σi ≈ id : Ord fv iN]]$
  \item[$+$] assuming $[[Γ ⊢ iP]]$,~
    $[[Γ ⊢ [σ1]iP ≥ [σ2]iP]]$ implies $[[Γ ⊢ σi ≈ id : Ord fv iP]]$
  \end{itemize}
\end{lemma}
\begin{proof}
  Proof by induciton on $[[Γ ⊢ iN]]$ (and mutually on $[[Γ ⊢ iP]]$).
  \begin{caseof}
    \item $[[iN]] = [[α⁻]]$\\
      Then $[[Γ ⊢ [σ1]iN ≤ [σ2]iN]]$ is rewritten as $[[Γ ⊢ [σ1]α⁻ ≤ [σ2]α⁻]]$.
      Let us consider the following cases:
      \begin{caseof}
      \item $[[α⁻ ∉ {Γ1}]]$ and $[[α⁻ ∉ {Γ2}]]$ \label{case:var-not-in-ctxts}\\
        Then $[[Γ ⊢ σi ≈ id : α⁻]]$ holds immediately,
        since $[[ [σi] α⁻]] = [[ [id] α⁻]] = [[α⁻]]$ and
        $[[Γ ⊢ α⁻ ≈ α⁻]]$.
      \item $[[α⁻ ∊ {Γ1}]]$ and $[[α⁻ ∊ {Γ2}]]$\\
        This case is not possible by assumption: $[[{Γ1} ∩ {Γ2}= ∅]]$.
      \item $[[α⁻ ∊ {Γ1}]]$ and $[[α⁻ ∉ {Γ2}]]$\\
        Then we have $[[Γ ⊢ [σ1]α⁻ ≤ α⁻]]$,
        which by \cref{corollary:vars-no-proper-subtypes} means $[[Γ ⊢ [σ1]α⁻ ≈ α⁻]]$,
        and hence, $[[Γ ⊢ σ1 ≈ id : α⁻]]$.

        $[[Γ ⊢ σ2 ≈ id : α⁻]]$ holds since $[[ [σ2]α⁻ ]] = [[α⁻]]$,
        similarly to \cref{case:var-not-in-ctxts}.

      \item $[[α⁻ ∉ {Γ1}]]$ and $[[α⁻ ∊ {Γ2}]]$\\
        Then we have $[[Γ ⊢ α⁻ ≤ [σ2]α⁻]]$,
        which by \cref{corollary:vars-no-proper-subtypes} means $[[Γ ⊢ α⁻ ≈ [σ2]α⁻]]$,
        and hence, $[[Γ ⊢ σ2 ≈ id : α⁻]]$.

        $[[Γ ⊢ σ1 ≈ id : α⁻]]$ holds since $[[ [σ1]α⁻ ]] = [[α⁻]]$,
        similarly to \cref{case:var-not-in-ctxts}.
      \end{caseof}
  \item $[[iN]] = [[∀pas.iM]]$\\
    Then by inversion, $[[Γ, pas ⊢ iM]]$.
    $[[Γ ⊢ [σ1]iN ≤ [σ2]iN]]$ is rewritten as $[[Γ ⊢ [σ1]∀pas.iM ≤ [σ2]∀pas.iM]]$.
    By the congruence of substitution and by the inversion of
    \ruleref{\ottdruleDOneForallLabel}, $[[Γ, pas ⊢ [iQs/pas][σ1]iM ≤ [σ2]iM]]$,
    where $[[Γ, pas ⊢ iQi]]$.
    Let us denote the (Kleisli) composition of $[[σ1]]$ and $[[iQs/pas]]$ as
    $[[σ1']]$, noting that $[[Γ, pas ⊢ σ1' : Γ1, pas]]$,
    and $[[{Γ1, pas} ∩ {Γ2} = ∅]]$.

    Let us apply the induction hypothesis to $[[iM]]$ and the
    substitutions $[[σ1']]$ and $[[σ2]]$ with
    $[[Γ, pas ⊢ [σ1']iM ≤ [σ2]iM]]$ to obtain:
    \begin{align}
      [[Γ, pas ⊢ σ1' ≈ id : Ord fv iM]] \label{fact:subs-proper-sub:forall-ih}\\
      [[Γ, pas ⊢ σ2 ≈ id : Ord fv iM]]  \label{fact:subs-proper-sub:forall-ih2}
    \end{align}

    Then $[[Γ ⊢ σ2 ≈ id : Ord fv ∀pas.iM]]$ holds by strengthening of
    \ref{fact:subs-proper-sub:forall-ih2}:
    for any $[[β±]] \in [[fv ∀pas.iM]] = [[fv iM \ {pas}]]$,
    $[[Γ, pas ⊢ [σ2]β± ≈ β±]]$ is strengthened to $[[Γ ⊢ [σ2]β± ≈ β±]]$, because
    $[[fv [σ2]β±]] = [[fv β±]] = \{[[β±]]\} \subseteq [[{Γ}]]$.

    To show that $[[Γ ⊢ σ1 ≈ id : Ord fv ∀pas.iM]]$, let us take an arbitrary
    $[[β±]] \in [[fv ∀pas.iM]] = [[fv iM \ {pas}]]$.

    $
    \begin{aligned}[t]
      [[β±]] &= [[ [id]β± ]]
             && \text{by definition of $[[id]]$}\\
             &\eqDOne [[ [σ1']β± ]]
             && \text{by \ref{fact:subs-proper-sub:forall-ih}}\\
             &= [[ [iQs/pas][σ1]β±]]
             && \text{by definition of $[[σ1']]$}\\
             &= [[ [σ1]β± ]]
             && \text{because $[[{pas} ∩ fv [σ1]β±]] \subseteq [[{pas} ∩ {Γ}]] = \emptyset$}
    \end{aligned}
    $\\
    This way, $[[Γ ⊢ [σ1]β± ≈ β±]]$ for any $[[β±]] \in [[fv ∀pas.iM]]$ and thus,
    $[[Γ ⊢ σ1 ≈ id : Ord fv ∀pas.iM]]$.

  \item $[[iN]] = [[iP → iM]]$\\
    Then by inversion, $[[Γ ⊢ iP]]$ and $[[Γ ⊢ iM]]$.
    $[[Γ ⊢ [σ1]iN ≤ [σ2]iN]]$ is rewritten as
    $[[Γ ⊢ [σ1](iP → iM) ≤ [σ2](iP → iM)]]$,
    then by congruence of substitution,
    $[[Γ ⊢ [σ1]iP → [σ1]iM ≤ [σ2]iP → [σ2]iM]]$,
    then by inversion
    $[[Γ ⊢ [σ1]iP ≥ [σ2]iP]]$
    and
    $[[Γ ⊢ [σ1]iM ≤ [σ2]iM]]$.

    Applying the induction hypothesis to $[[Γ ⊢ [σ1]iP ≥ [σ2]iP]]$
    and to $[[Γ ⊢ [σ1]iM ≤ [σ2]iM]]$, we obtain (respectively):
    \begin{align}
      &[[Γ ⊢ σi ≈ id : Ord fv iP]] \label{fact:subs-proper-sub:arrow-ih1}\\
      &[[Γ ⊢ σi ≈ id : Ord fv iM]] \label{fact:subs-proper-sub:arrow-ih2}
    \end{align}

    Noting that $[[fv (iP → iM)]] = [[fv iP ∪ fv iM]]$,
    we combine
    \cref{fact:subs-proper-sub:arrow-ih1,fact:subs-proper-sub:arrow-ih2}
    to conclude:
    $[[Γ ⊢ σi ≈ id : Ord fv (iP → iM)]]$.

  \item $[[iN]] = [[↑iP]]$\\
    Then by inversion, $[[Γ ⊢ iP]]$.
    $[[Γ ⊢ [σ1]iN ≤ [σ2]iN]]$ is rewritten as
    $[[Γ ⊢ [σ1]↑iP ≤ [σ2]↑iP]]$,
    then by congruence of substitution and by inversion,
    $[[Γ ⊢ [σ1]iP ≥ [σ2]iP]]$

    Applying the induction hypothesis to $[[Γ ⊢ [σ1]iP ≥ [σ2]iP]]$, we obtain
    $[[Γ ⊢ σi ≈ id : Ord fv iP]]$. Since $[[fv ↑iP]] = [[fv iP]]$, we can
    conclude: $[[Γ ⊢ σi ≈ id : Ord fv ↑iP]]$.
  \item The positive cases are proved symmetrically.
  \end{caseof}
\end{proof}

\begin{corollary}[Substitution cannot induce proper subtypes or supertypes] \label{corollary:subst-proper-subt}
  Assuming all mentioned types are well-formed in $[[Γ]]$ and $[[σ]]$ is a
  substitution $[[Γ ⊢ σ : Γ]]$,
  \begin{align*}
    [[Γ ⊢ [σ]iN ≤ iN]] ~&\Rightarrow~ [[Γ ⊢ [σ]iN ≈ iN]]
                          \text{ and } [[Γ ⊢ σ ≈ id : Ord fv iN]] \\
    [[Γ ⊢ iN ≤ [σ]iN]] ~&\Rightarrow~ [[Γ ⊢ iN ≈ [σ]iN]]
                          \text{ and } [[Γ ⊢ σ ≈ id : Ord fv iN]] \\
    [[Γ ⊢ [σ]iP ≥ iP]] ~&\Rightarrow~ [[Γ ⊢ [σ]iP ≈ iP]]
                          \text{ and } [[Γ ⊢ σ ≈ id : Ord fv iP]] \\
    [[Γ ⊢ iP ≥ [σ]iP]] ~&\Rightarrow~ [[Γ ⊢ iP ≈ [σ]iP]]
                          \text{ and } [[Γ ⊢ σ ≈ id : Ord fv iP]] \\
  \end{align*}
\end{corollary}


\begin{lemma} \label{lemma:mutual-subst-subtyping}
  Asssuming that the mentioned types ($[[iP]]$, $[[iQ]]$, $[[iN]]$, and $[[iM]]$)
  are well-formed in $[[Γ]]$ and that the substitutions ($[[σ1]]$ and $[[σ2]]$) have signature $[[Γ ⊢ σi : Γ]]$,
  \begin{itemize}
  \item[$+$] if $[[Γ ⊢ [σ1] iP ≥ iQ]]$ and $[[Γ ⊢ [σ2] iQ ≥ iP]]$\\
    then there exists a bijection $[[μ : fv iP ↔ fv iQ]]$ such that
    $[[Γ ⊢ σ1 ≈ Sub μ : Ord fv iP]]$ and $[[Γ ⊢ σ2 ≈ Sub μ-1 : Ord fv iQ]]$;
  \item[$-$] if $[[Γ ⊢ [σ1] iN ≤ iM]]$ and $[[Γ ⊢ [σ2] iN ≤ iM]]$\\
    then there exists a bijection $[[μ : fv iN ↔ fv iM]]$ such that
    $[[Γ ⊢ σ1 ≈ Sub μ : Ord fv iN]]$ and $[[Γ ⊢ σ2 ≈ Sub μ-1 : Ord fv iM]]$.
  \end{itemize}
\end{lemma}
\begin{proof}
  \hfill
  \begin{itemize}
  \item[$+$]
    Applying $[[σ2]]$ to both sides of
    $[[Γ ⊢ [σ1] iP ≥ iQ]]$ (by \cref{todo}),
    we have: $[[Γ ⊢ [σ2 ○ σ1] iP ≥ [σ2]iQ]]$.
    Composing it with $[[Γ ⊢ [σ2] iQ ≥ iP]]$ (by transitivity \cref{todo}),
    we have $[[Γ ⊢ [σ2 ○ σ1] iP ≥ iP]]$.
    Then by \cref{corollary:subst-proper-subt},
    $[[Γ ⊢ σ2 ○ σ1 ≈ id : Ord fv iP]]$.

    % Applying $[[σ1]]$ to both sides of
    % $[[Γ ⊢ [σ2]iQ ≥ iP]]$ (by \cref{todo}),
    % we have: $[[Γ ⊢ [σ1 ○ σ2] iQ ≥ [σ1]iP]]$.
    % Composing it with $[[Γ ⊢ [σ1] iP ≥ iQ]]$ (by transitivity \cref{todo}),
    % we have $[[Γ ⊢ [σ1 ○ σ2] iQ ≥ iQ]]$.
    % Then by \cref{corollary:subst-proper-subt},
    By a symmetric argument, we also have:
    $[[Γ ⊢ σ1 ○ σ2 ≈ id : Ord fv iQ]]$.

    Now, we prove that
    $[[Γ ⊢ σ2 ○ σ1 ≈ id : Ord fv iP]]$ and
    $[[Γ ⊢ σ1 ○ σ2 ≈ id : Ord fv iQ]]$
    implies that $[[σ1]]$ and $[[σ1]]$
    are (equivalent to) mutually inverse bijections.

    To do so, it suffices to prove that
    \begin{enumerate}
    \item[(i)] for any $[[α± ∊ fv iP]]$ there exists $[[β± ∊ fv iQ]]$
        such that $[[ Γ ⊢ [σ1] α± ≈ β± ]]$ and
        $[[ Γ ⊢ [σ2] β± ≈ α± ]]$; and
    \item[(ii)] for any $[[β± ∊ fv iQ]]$ there exists $[[α± ∊ fv iP]]$
        such that $[[ Γ ⊢ [σ2] β± ≈ α± ]]$ and
        $[[ Γ ⊢ [σ1] α± ≈ β± ]]$.
    \end{enumerate}
    Then the these correspondences between $[[fv iP]]$ and
    $[[fv iQ]]$ are mutually inverse functions,
    since for any $[[β±]]$ there can be at most one $[[α±]]$
    such that $[[ Γ ⊢ [σ2] β± ≈ α± ]]$ (and vice versa).

    \begin{enumerate}
    \item[(i)] Let us take $[[α± ∊ fv iP]]$.
      \begin{enumerate}
      \item if $[[α±]]$ is positive ($[[α± = α⁺]]$),
        from $[[ Γ ⊢ [σ2][σ1]α⁺ ≈ α⁺ ]]$,
        by \cref{corollary:vars-no-proper-subtypes},
        we have
        $[[ [σ2][σ1]α⁺ = ∃nbs.α⁺ ]]$.

        What shape can $[[ [σ1]α⁺ ]]$ have? It cannot be $[[∃nas.↓iN]]$ (for
        potentially empty $[[nas]]$), because the outer constructor $\downarrow$
        would remain after the substitution $[[σ2]]$, whereas $[[∃nbs.α⁺]]$ does
        not have $[[↓]]$. The only case left is $[[ [σ1]α⁺ = ∃nas.γ⁺ ]]$.

        Notice that $[[Γ ⊢ ∃nas.γ⁺ ≈ γ⁺]]$, meaning that $[[Γ ⊢ [σ1]α⁺ ≈ γ⁺]]$.
        Also notice that $[[ [σ2]∃nas.γ⁺ = ∃nbs.α⁺ ]]$ implies
        $[[Γ ⊢ [σ2]γ⁺ ≈ α⁺]]$.

      \item if $[[α±]]$ is negative ($[[α± = α⁻]]$) from $[[ Γ ⊢ [σ2][σ1]α⁻ ≈ α⁻
        ]]$, by \cref{corollary:vars-no-proper-subtypes}, we have
        $[[ [σ2][σ1]α⁻ = ∀pbs.α⁻ ]]$.

        What shape can $[[ [σ1]α⁻ ]]$ have? It cannot be $[[∀pas.↑iP]]$
        nor $[[∀pas.iP → iM]]$ (for potentially empty $[[pas]]$),
        because the outer constructor ($[[→]]$ or $[[↑]]$), remaining
        after the substitution $[[σ2]]$, is however absent in the resulting
        $[[∀pbs.α⁻]]$. Hence, the only case left is $[[ [σ1]α⁻ = ∀pas.γ⁻ ]]$
        Notice that $[[Γ ⊢ γ⁻ ≈ ∀pas.γ⁻]]$, meaning that $[[Γ ⊢ [σ1]α⁻ ≈ γ⁻]]$.
        Also notice that $[[ [σ2]∀pas.γ⁻ = ∀pbs.α⁻ ]]$ implies
        $[[Γ ⊢ [σ2]γ⁻ ≈ α⁻]]$.
      \end{enumerate}
    \item[(ii)] The proof is symmetric:
      We swap $[[iP]]$ and $[[iQ]]$,
      $[[σ1]]$ and $[[σ2]]$,
      and exploit $[[ Γ ⊢ [σ1][σ2]α± ≈ α± ]]$ instead of
      $[[ Γ ⊢ [σ2][σ1]α± ≈ α± ]]$.

    \end{enumerate}

  \item[$-$] The proof is symmetric to the positive case.
  \end{itemize}
\end{proof}

\begin{lemma}[Equivalence of polymorphic types]
  \label{lemma:poly-types-equivalence}
  \hfill
  \begin{itemize}
    \item[$-$] For $[[Γ ⊢ ∀pas.iN]]$ and $[[Γ ⊢ ∀pbs.iM]]$,\\ if $[[Γ ⊢ ∀pas.iN ≈ ∀pbs.iM ]]$
    then there exists a bijection $[[μ : {pbs} ∩ fv iM ↔ {pas} ∩ fv iN]]$
    such that $[[ Γ, pas ⊢ iN ≈ [Sub μ] iN ]]$,
    \item[$+$] For $[[Γ ⊢ ∃nas.iP]]$ and $[[Γ ⊢ ∃nbs.iQ]]$,\\  if $[[Γ ⊢ ∃nas.iP ≈ ∃nbs.iQ ]]$
    then there exists a bijection $[[μ : {nbs} ∩ fv iQ ↔ {nas} ∩ fv iP]]$
    such that $[[ Γ, nbs ⊢ iP ≈ [Sub μ] iQ ]]$.
  \end{itemize}
\end{lemma}
\begin{proof}
    \hfill
  \begin{itemize}
    \item[$-$]
    First, by $\alpha$-conversion, we ensure $[[{pas} ∩ fv iM = ∅]]$ and $[[{pbs} ∩ fv iN = ∅]]$.
    By inversion, $[[Γ ⊢ ∀pas.iN ≈ ∀pbs.iM ]]$ implies 
    \begin{enumerate} 
      \item $[[Γ,pbs ⊢ [σ1]iN ≤ iM]]$ for $[[ Γ,pbs ⊢ σ1 : pas ]]$ and 
      \item $[[Γ,pas ⊢ [σ2]iM ≤ iN]]$ for $[[ Γ,pas ⊢ σ2 : pbs ]]$.
    \end{enumerate}
    To apply \cref{lemma:mutual-subst-subtyping}, we weaken 
    and rearrange the contexts, and extend the substitutions to act as identity
    outside of their initial domain:
    \begin{enumerate} 
      \item $[[Γ,pas,pbs ⊢ [σ1]iN ≤ iM]]$ for $[[ Γ,pas,pbs ⊢ σ1 : Γ,pas,pbs ]]$ and 
      \item $[[Γ,pas,pbs ⊢ [σ2]iM ≤ iN]]$ for $[[ Γ,pas,pbs ⊢ σ2 : Γ,pas,pbs ]]$.
    \end{enumerate}
    Then from \cref{lemma:mutual-subst-subtyping}, 
    there exists a bijection $[[μ : fv iM ↔ fv iN]]$ such that 
    $[[Γ,pas,pbs ⊢ σ2 ≈ Sub μ : Ord fv iM]]$ and 
    $[[Γ,pas,pbs ⊢ σ1 ≈ Sub μ-1 : Ord fv iN]]$. 

    Let us show that if we restrict the domain of $[[μ]]$ to 
    $[[pbs]]$, its range will be contained in $[[pas]]$.
    Let us take $[[γ⁺ ∊ {pbs} ∩ fv iM]]$ and 
    assume $[[ [μ]γ⁺]] \notin [[pas]]$.
    Then since $[[ Γ,pbs ⊢ σ1 : pas ]]$, 
    $[[σ1]]$ acts as identity outside of $[[pas]]$, i.e.
    $[[ [σ1][Sub μ]γ⁺ = [Sub μ]γ⁺ ]]$.
    Since
    $[[Γ,pas,pbs ⊢ σ1 ≈ Sub μ-1 : Ord fv iN]]$, 
    application of $[[σ1]]$ is equivalent to application of $[[Sub μ-1]]$,
    then 
    $[[ Γ,pas,pbs ⊢ [Sub μ-1][Sub μ]γ⁺ ≈ [Sub μ]γ⁺ ]]$, i.e.
    $[[Γ,pas,pbs ⊢ γ⁺ ≈ [Sub μ]γ⁺]]$, 
    which means $[[γ⁺ ∊ fv [Sub μ]γ⁺]] \subseteq [[fv iN]]$.
    By assumption, $[[γ⁺ ∊ {pbs} ∩ fv iM]]$, i.e. $[[{pbs} ∩ fv iN]] \neq \emptyset$, hence contradiction.

    By \cref{todo}, 
    $[[Γ,pas,pbs ⊢ σ2 ≈ Sub μ|{pbs} : Ord fv iM]]$ implies
    $[[Γ,pas,pbs ⊢ [σ2]iM ≈ [Sub μ|{pbs}]iM]]$.
    By similar reasoning, $[[Γ,pas,pbs ⊢ [σ1]iN ≈ [Sub μ-1|{pas}]iN]]$.

    This way,
    \begin{align} 
      [[Γ,pas,pbs ⊢ [Sub μ-1|{pas}]iN ≤ iM]] \label{fact:mu-inv-n-sub-m}\\
      [[Γ,pas,pbs ⊢ [Sub μ|{pbs}]iM ≤ iN]] \label{fact:mu-m-subt-n}
    \end{align}

    By applying $[[μ|_{pbs}]]$ to both sides of \ref{fact:mu-inv-n-sub-m} (\cref{todo})
    and contracting $[[μ-1|_{pas} ○ μ|_{pbs}]] = [[μ|_{pbs}-1 ○ μ|_{pbs}]] = [[id]]$,
    we have: $[[Γ,pas,pbs ⊢ iN ≤ [Sub μ|{pbs}]iM]]$, which together with \ref{fact:mu-m-subt-n}
    means $[[Γ,pas,pbs ⊢ iN ≈ [Sub μ|{pbs}]iM]]$, and by strengthening, $[[Γ,pas⊢ iN ≈ [Sub μ|{pbs}]iM]]$.
    Symmetrically, $[[Γ,pbs ⊢ iM ≈ [Sub μ|_{pbs}-1]iN]]$.
    \item{$+$} The proof is symmetric to the proof of the negative case.
  \end{itemize}

\end{proof}


\begin{lemma}[Completeness of equivalence] \label{lemma:equiv-completeness}
  Mutual subtyping implies declarative equivalence.
  Assuming all the types below are well-formed in $[[Γ]]$: 
  \begin{itemize}
  \item[$+$] if $[[Γ ⊢ iP ≈ iQ]]$ then $[[iP ≈ iQ]]$,
  \item[$-$] if $[[Γ ⊢ iN ≈ iM]]$ then $[[iN ≈ iM]]$.
  \end{itemize}
\end{lemma}
\begin{proof}
  \begin{itemize}
    \item[$-$] 
    Induction on the sum of sizes of  $[[iN]]$ and $[[iM]]$. 
    By inversion, $[[Γ ⊢ iN ≈ iM]]$ means $[[Γ ⊢ iN ≤ iM]]$ and $[[Γ ⊢ iM ≤ iN ]]$.
    Let us consider the last rule that forms $[[Γ ⊢ iN ≤ iM]]$:
    \begin{caseof}
      \item \ruleref{\ottdruleDOneNVarLabel} i.e. $[[Γ ⊢ iN ≤ iM]]$ is of the form $[[Γ ⊢ α⁻ ≤ α⁻]]$\\
      Then $[[iN ≈ iM]]$ (i.e. $[[α⁻ ≈ α⁻]]$) holds immediately by \ruleref{\ottdruleEOneNVarLabel}.

      \item \ruleref{\ottdruleDOneShiftULabel} i.e. 
      $[[Γ ⊢ iN ≤ iM]]$ is of the form $[[Γ ⊢ ↑iP ≤ ↑iQ]]$\\
      Then by inversion, $[[Γ ⊢ iP ≈ iQ]]$, 
      and by induction hypothesis, $[[iP ≈ iQ]]$.
      Then $[[iN ≈ iM]]$ (i.e. $[[↑iP ≈ ↑iQ]]$) holds 
      by \ruleref{\ottdruleEOneShiftULabel}.

      \item \ruleref{\ottdruleDOneArrowLabel} i.e. $[[Γ ⊢ iN ≤ iM]]$ is of the form $[[Γ ⊢ iP → iN' ≤ iQ → iM']]$\\
      Then by inversion, $[[Γ ⊢ iP ≥ iQ]]$ and $[[Γ ⊢ iN' ≤ iM']]$.
      Notice that $[[Γ ⊢ iM ≤ iN]]$ is of the form $[[Γ ⊢ iQ → iM' ≤ iP → iN']]$, 
      which by inversion means $[[Γ ⊢ iQ ≥ iP]]$ and $[[Γ ⊢ iM' ≤ iN']]$.

      This way, $[[Γ ⊢ iQ ≈ iP]]$ and $[[Γ ⊢ iM' ≈ iN']]$. 
      Then by induction hypothesis, $[[iQ ≈ iP]]$ and $[[iM' ≈ iN']]$.
      Then $[[iN ≈ iM]]$ (i.e. $[[iP → iN' ≈ iQ → iM']]$) holds by \ruleref{\ottdruleEOneArrowLabel}.

      \item \ruleref{\ottdruleDOneForallLabel} i.e. $[[Γ ⊢ iN ≤ iM]]$ is of the form $[[Γ ⊢ ∀pas.iN' ≤ ∀pbs.iM']]$\\
      Then by \cref{lemma:poly-type-equivalence}, $[[Γ ⊢ ∀pas.iN' ≈ ∀pbs.iM']]$ means that 
      there exists a bijection $[[μ : {pbs} ∩ fv iM' ↔ {pas} ∩ fv iN']]$ such that  
      $[[Γ,pas ⊢ [Sub μ]iM' ≈ iN']]$. 
      
      Notice that the application of bijection $[[μ]]$ to $[[iM']]$ does
      not change its size (which is less than the size of $[[iM]]$), hence the induction hypothesis applies.
      This way, $[[ [Sub μ]iM' ≈ iN']]$ (and by symmetry, $[[iN' ≈ [Sub μ]iM']]$) holds by induction. 
      Then we apply \ruleref{\ottdruleEOneForallLabel} to get $[[∀pas.iN' ≈ ∀pbs.iM']]$, i.e. $[[iN ≈ iM]]$.
    \end{caseof}
      
\item[$+$] The proof is symmetric to the proof of the negative case.
  \end{itemize}
\end{proof}

\begin{corollary}[Normalization is complete w.r.t. subtyping-induced equivalence]
  \label{corollary:nf-complete-wrt-subt-equiv}
  Assuming all the types below are well-formed in $[[Γ]]$:
  \begin{itemize}
    \item [$+$] if $[[Γ ⊢ iP ≈ iQ]]$ then $[[nf(iP) = nf(iQ)]]$,
    \item [$-$] if $[[Γ ⊢ iN ≈ iM]]$ then $[[nf(iN) = nf(iM)]]$.
  \end{itemize}
\end{corollary}  
\begin{proof}
  Immediately from \cref{lemma:equiv-completeness,lemma:normalization-completeness}.
\end{proof}

\begin{lemma}[Algorithmization of subtyping-induced equivalence]
  \label{lemma:subt-equiv-algorithmization}
  Mutual subtyping is equality of normal forms.
 Assuming all the types below are well-formed in $[[Γ]]$:
  \begin{itemize}
    \item [$+$] $[[Γ ⊢ iP ≈ iQ]] \iff [[nf(iP) = nf(iQ)]]$,
    \item [$-$] $[[Γ ⊢ iN ≈ iM]] \iff [[nf(iN) = nf(iM)]]$.
  \end{itemize}
\end{lemma}
\begin{proof}
  Let us prove the positive case, the negative case is symmetric.
  We prove both directions of $\iff$ separately:
  \begin{itemize}
    \item [$\Rightarrow$] exactly \cref{corollary:nf-complete-wrt-subt-equiv};
    \item [$\Leftarrow$] by \cref{lemma:decl-equiv-algorithmization,lemma:equiv-soundness}.
  \end{itemize}
\end{proof}

