\usepackage{lscape}
\usepackage{amsmath}
\usepackage{amsthm}
\usepackage{amssymb}
\usepackage{booktabs}
\usepackage{multicol}
\usepackage{supertabular}
\usepackage[inline]{enumitem}
\usepackage{cleveref}
\usepackage{proof}

\usepackage{stackengine}

\usepackage{mathabx}
\usepackage[dvipsnames]{xcolor}
\usepackage{scalerel}

\usepackage{tikz}
\usetikzlibrary{shapes,arrows,positioning}
\usetikzlibrary{shapes.multipart}


\usepackage{todonotes}

\usepackage{enumitem}
\usepackage{xparse}
\usepackage{casenum}

\usepackage{braket}
\usepackage{../quiver}




\setlength{\columnsep}{1cm}
\let\citename\relax
\usepackage[natbib=true, abbreviate=false, dateabbrev=true, isbn=true, doi=false, urldate=comp, url=false, maxbibnames=9, maxcitenames=2, backref=false, backend=biber, style=alphabetic, language=american, arxiv=false]{biblatex}
\addbibresource{../biblio.bib}

\newcommand{\niton}{\not\owns}

\newcommand{\ilyam}[1]{{\color{red} \texttt{Ilya:  #1}}}

\newtheorem{algorithm}{Algorithm}
\newtheorem{definition}{Definition}
\newtheorem*{notation*}{Notation}
\newtheorem{theorem}{Theorem}
\newtheorem*{theorempreview}{Theorem}
\newtheorem{lemma}{Lemma}
\newtheorem{corollary}{Corollary}
\newtheorem{proposition}{Proposition}
\newtheorem{observation}{Observation}
\newtheorem*{assertion*}{Assertion}

% https://tex.stackexchange.com/questions/85033/colored-symbols/85035#85035
\newcommand*{\mathcolor}{}
\def\mathcolor#1#{ \mathcoloraux{#1} }
\newcommand*{\mathcoloraux}[3]{%
  \protect\leavevmode
  \begingroup
  \color#1{#2}#3%
  \endgroup
}

\newcommand{\UB}[0]{\mathsf{UB}}
\newcommand{\NFUB}[0]{\mathsf{NFUB}}



% \newcounter{casenum}

% \newenvironment{caseof}
% {%
%   \par
%   \setlength{\parskip}{6pt}%
%   % \setlength{\parindent}{0pt}%
%   \everypar{\setlength{\hangindent}{17pt}}%
%   \setcounter{casenum}{0}%
% }
% {\par\vskip.5\baselineskip}

% \NewDocumentCommand{\case}{omm}{%
%   % \vskip.5\baselineskip\par%
%   \itemindent\parindent
%   \refstepcounter{casenum}%
%   {\bfseries Case} {\bfseries \arabic{casenum}}%
%   \IfNoValueF{#1}{\label{#1}}%
%   {\bfseries:} #2\\#3 %
% }
