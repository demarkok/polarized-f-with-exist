\usepackage[dvipsnames]{xcolor}
\usepackage{scalerel}


\usepackage{braket}
% \usepackage{hyperref}
\usepackage{mathpartir}

\usepackage{lscape}
\usepackage{amsmath}
\usepackage{amsthm}
\usepackage{booktabs}
\usepackage{multicol}
\usepackage{supertabular}
\usepackage[inline]{enumitem}
\usepackage{cleveref}
\usepackage{proof}
\usepackage{mathtools}

% Show the frame 
% \usepackage{showframe}

\usepackage{newtxmath}
% \usepackage{mathabx}


\usepackage{stackengine}



\usepackage{todonotes}

\usepackage{enumitem}
\usepackage{xparse}

\usepackage{../casenum}
% \usepackage{../quiver}

\usepackage{listings}
\usepackage{xspace}
\usepackage{subcaption}

\usepackage{etoolbox}


% \usepackage{tikz}
\usetikzlibrary{shapes,arrows,arrows.meta,positioning}
\usetikzlibrary{shapes.multipart}

% ⫤
\usepackage{graphicx}
\makeatletter
\providecommand*{\Dashv}{%
  \mathrel{%
    \mathpalette\@Dashv\vDash
  }%
}
\newcommand*{\@Dashv}[2]{%
  \reflectbox{$\m@th#1#2$}%
}
\makeatother




\setlength{\columnsep}{1cm}

\newcommand{\niton}{\not\owns}

\newcommand{\ilyam}[1]{{\color{red} \texttt{Ilya:  #1}}}
\newcommand{\nk}[1]{{\color{purple} \texttt{Neel:  #1}}}

\usepackage{thmtools}
\usepackage{thm-restate}

% https://tex.stackexchange.com/questions/553406/the-label-command-defined-by-cleveref-fails-in-a-thm-restate-environment
\usepackage{xpatch}

% \makeatletter
% \@ifpackageloaded{cleveref}{
%   \xpatchcmd\thmt@restatable
%     {\let\label=\@gobble}
%     {\let\label=\gobbled@cleveref@label}
%     {}{}
  
%   \newcommand\gobbled@cleveref@label[2][]{}
% }{}


\declaretheorem[name=Theorem,refname={theorem,theorems},Refname={Theorem,Theorems}]{theoremRes}

\declaretheorem[name=Corollary,refname={corollary,corollaries},Refname={Corollary,Corollaries}]{corollaryRes}

\declaretheorem[name=Proposition,refname={proposition,propositions},Refname={Proposition,Propositions}]{propositionRes}

\declaretheorem[name=Property,refname={property,properties},Refname={Property,Properties}]{propertyRes}

\declaretheorem[name=Lemma,refname={lemma,lemmas},Refname={Lemma,Lemmas}]{lemmaRes}

\newtheorem{lemma}{Lemma}
\newtheorem*{lemma*}{Lemma}
\newtheorem{corollary}{Corollary}
\newtheorem{theorem}{Theorem}
\newtheorem*{theorem*}{Theorem}
\newtheorem{proposition}{Proposition}
\newtheorem{property}{Property}

\newtheorem{algorithm}{Algorithm}
\newtheorem{definition}{Definition}
\newtheorem*{definition*}{Definition}
\newtheorem*{notation*}{Notation}
\newtheorem*{theorempreview}{Theorem}
\newtheorem{observation}{Observation}
\newtheorem*{assertion*}{Assertion}

\makeatletter
\newcommand{\defeq}{\mathrel{\aban@defeq}}
\newcommand{\aban@defeq}{%
  \vbox{\offinterlineskip\check@mathfonts
    \ialign{\hfil##\hfil\cr
      \fontsize{\ssf@size}{\z@}\normalfont def\cr
      \noalign{\kern1\p@}
      $\m@th=$\cr
      \noalign{\kern-.5\fontdimen22\textfont2}
    }%
  }%
}
\makeatother

\newcommand{\UB}[0]{\mathsf{UB}}
\newcommand{\NFUB}[0]{\mathsf{NFUB}}

\newcommand{\depth}[1]{\ensuremath{\mathsf{depth}(#1)}}
\newcommand{\size}[1]{\ensuremath{\mathsf{size}(#1)}}

\newcommand{\ruleref}[1]{\nameref{#1}}

\newcommand{\fexists}{F$^\pm\exists$\xspace} 
\newcommand{\etc}{\text{e.t.c.}}



\newcommand{\ie}{\text{i.e.,} }
\newcommand{\eg}{\text{e.g.,} }
\newcommand{\wrt}{w.r.t.\xspace}
\newcommand{\stt}{s.t.\xspace}
\newcommand{\aka}{a.k.a.\xspace}
\newcommand{\resp}{resp.\xspace}

\newcommand{\code}[1]{\texttt{#1}}

\newcommand{\cmark}{\ding{51}}%
\newcommand{\xmark}{\ding{55}}%

\newcommand{\CBPV}{\text{Call-By-Push-Value} }

\newcommand{\coq}{\texttt{Coq} }
\newcommand{\agda}{\texttt{Agda} }

\newcommand{\systemf}{\text{System F} }

\newcommand{\pack}{\texttt{pack} }
\newcommand{\unpack}{\texttt{unpack} }

% https://tex.stackexchange.com/questions/85033/colored-symbols/85035#85035
\providecommand*{\mathcolor}{}
\def\mathcolor#1#{\mathcoloraux{#1}}
\newcommand*{\mathcoloraux}[3]{%
  \protect\leavevmode
  \begingroup
  \color#1{#2}#3%
  \endgroup
}

\definecolor{positive}{RGB}{200, 50, 50}
\definecolor{negative}{RGB}{50, 50, 200}


\setlength\multicolsep{0pt}


\newcommand{\pureSize}[1]{\ensuremath{\mathsf{pure\_size}(#1)}}
\newcommand{\metric}[1]{\ensuremath{\mathsf{metric}(#1)}}
\newcommand{\eqNodes}[1]{\ensuremath{\mathsf{eq\_nodes}(#1)}}
\newcommand{\npq}[1]{\ensuremath{\mathsf{npq}(#1)}}


\newcommand{\downharpoonlefttight}{\mathord{\downharpoonright}}
\newcommand{\downharpoonrighttight}{\mathord{\downharpoonleft}}
\newcommand{\upharpoonlefttight}{\mathord{\upharpoonright}}
\newcommand{\upharpoonrighttight}{\mathord{\upharpoonleft}}
