\documentclass[sigplan,natbib=false,screen]{acmart}
\settopmatter{printacmref=false}
\usepackage{../bibliography}
\usepackage{lscape}
\usepackage{amsmath}
\usepackage{amsthm}
\usepackage{amssymb}
\usepackage{booktabs}
\usepackage{multicol}
\usepackage{supertabular}
\usepackage[inline]{enumitem}
\usepackage{cleveref}
\usepackage{proof}

\usepackage{stackengine}

\usepackage{mathabx}
\usepackage[dvipsnames]{xcolor}
\usepackage{scalerel}


\usepackage{todonotes}

\usepackage{enumitem}
\usepackage{xparse}
\usepackage{casenum}

\usepackage{braket}

\newcommand{\niton}{\not\owns}

\newcommand{\ilyam}[1]{{\color{red} \texttt{Ilya:  #1}}}

\newtheorem{definition}{Definition}
\newtheorem{theorem}{Theorem}
\newtheorem{lemma}{Lemma}
\newtheorem{corollary}{Corollary}
\newtheorem{observation}{Observation}
\newtheorem*{assertion*}{Assertion}

% https://tex.stackexchange.com/questions/85033/colored-symbols/85035#85035
\newcommand*{\mathcolor}{}
\def\mathcolor#1#{ \mathcoloraux{#1} }
\newcommand*{\mathcoloraux}[3]{%
  \protect\leavevmode
  \begingroup
  \color#1{#2}#3%
  \endgroup
}

\newcommand{\UB}[0]{\mathsf{UB}}
\newcommand{\NFUB}[0]{\mathsf{NFUB}}


% \newcounter{casenum}

% \newenvironment{caseof}
% {%
%   \par
%   \setlength{\parskip}{6pt}%
%   % \setlength{\parindent}{0pt}%
%   \everypar{\setlength{\hangindent}{17pt}}%
%   \setcounter{casenum}{0}%
% }
% {\par\vskip.5\baselineskip}

% \NewDocumentCommand{\case}{omm}{%
%   % \vskip.5\baselineskip\par%
%   \itemindent\parindent
%   \refstepcounter{casenum}%
%   {\bfseries Case} {\bfseries \arabic{casenum}}%
%   \IfNoValueF{#1}{\label{#1}}%
%   {\bfseries:} #2\\#3 %
% }


\title{Local Type Inference for Polarised System F with Existentials}

\author{Ilya Kaysin}

\begin{document}

\maketitle

Type inference for polymorphic lambda calculi has been extensively
researched, but handling \emph{existential} types remains under-studied.
Our work presents a novel type inference algorithm for a 
variant of System F with impredicative polymorphism and first-class
existential types (F$_{\pm}^{\exists}$).
Describing the algorithm on a low level, we resolve the challenges that 
an implementer would face. In particular, the first-class existentials 
break the standard algorithm invariants, complicate the constraint resolution,
and require anti-unification---a novel for this area technique.

\paragraph{Polarization}
While powerful, the combination of impredicativity and polymorphism 
makes type inference undecidable. To address this, we work in a variant 
of a Call-By-Push-Value system, that stratifies types into positive ($[[iP]]$) 
and negative ($[[iN]]$) kinds. 
The positive types correspond to values (for example, integers, products, 
and existential $[[∃]]$-types),
and the negative types are computations 
(in particular, functions and polymorphic $[[∀]]$-types).
This stratification lets us formulate a typing specification that is decidable.

\paragraph{Algorithm Structure}
The algorithm has four main components: subtyping, unification, anti-unification
and type inference. It introduces novel techniques like mixed-polarity
unification and upgrade-based anti-unification enabled by CBPV's polarized
structure. These differ significantly from traditional Hindley-Milner style
inference. In particular, CBPV's restrictions break some key invariants relied
on by HM algorithms like stability of type variables. This requires mixing
unification and anti-unification in non-standard ways.




\paragraph{Subtyping}
An important component of the algorithm is the subtyping relation.
The polymorphism it induces is driven by instantiation:
$[[∀α⁺.iN]]$ is a subtype of any of the instantiations $[[ [iP/α⁺]iN]]$,
and dually, $[[∃α⁻.iP]]$ is a \emph{super}type of any of the 
instantiations $[[ [iN/α⁻]iP]]$.

When checking for subtyping of $\forall\alpha.P \leqslant Q$
or $\exists\alpha.N \geqslant M$, one must instantiate
$\alpha$ with a type that is not known beforehand. 
Standardly, we handle it by replacing $\alpha$ with a 
\emph{meta}variable $[[α̂⁺]]$, and then we recursively accumulate the 
constraints on it. Typically, the constraints are resolved by unification.
For example, the constraint $[[α̂⁺]]\rightarrow[[iN]]$ is resolved by 
unification to $[[α̂⁺]] = [[iP]]$.


\paragraph{Anti-Unificaiton}
However, in an impredicative system, unification is no longer sufficient.    
For instance, the conjunction of constraints 
$\alpha \geqslant (Int \rightarrow  Int)$ and $\alpha \geqslant (Bool \rightarrow Bool)$
has a solution $\alpha = \exists \beta. \beta \rightarrow \beta$.
This raises the need for \emph{anti-unification}: to find the least upper bound, 
one must find the most concrete pattern that fits both types and then 
quantify over the placeholders by $\exists$.

Our algorithm performs local type inference, instantiating quantifiers based
only on information available at each application site. This provides clear
error messages, modular checking and avoids complex heuristics like those used
by global inference algorithms. The algorithm supports implicit
elimination universal quantifiers ($\forall$) and implicit construction of the 
existential ($\exists$) quantifiers without type annotations. 
We prove the algorithm is sound and complete relative to a declarative specification. 


\paragraph{Extensions}

We also explore extensions to the core algorithm.
One is allowing the quantifiers to be bounded: in particular, 
the universally quantified variables can have 
a lower bound: $\forall \alpha \geqslant P. \cdots$ 
limits the types that can be substituted for $\alpha$.
Another is allowing lambda abstractions to omit type annotations,
which requires bidirectionalization of the type system
and complicates constraint resolution. 

\paragraph{Conclusion}

\end{document}