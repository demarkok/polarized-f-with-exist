
The declarative system serves as a specification of the
type inference algorithm. It consists of two main parts:
the subtyping and the type inference. 

\subsection{Subtyping}
It is represented by a set of inference rules shown in
\cref{fig:declarative-subtyping}.

\begin{figure}[h]
  \begin{multicols}{2}
    \ottdefnDOneNsub{}

    \ottdefnDOnePsup{}
  \end{multicols}
  \hfill\\
  \begin{multicols}{2}
    \ottdefnDOneNeq{}

    \ottdefnDOnePeq{}
  \end{multicols}
  \caption{Declarative Subtyping}
  \label{fig:declarative-subtyping}
\end{figure}

\paragraph{Invariance of Shifts}
An important restriction that we put on the subtyping system is
that the subtyping on shifted types requires their equivalence,
as shown in \ruleref{\ottdruleDOneShiftDLabel} and
\ruleref{\ottdruleDOneShiftULabel}. Relaxing both of these
invariants makes the system equivalent to \systemf, 
and thus, undecidable. 
However, \ruleref{\ottdruleDOneShiftDLabel} might be
relaxed to a covariant form, as we will discuss in \cref{todo}.

As discussed in \cref{todo}, 
the polymorphic rules \ruleref{\ottdruleDOneForallLabel} and 
\ruleref{\ottdruleDOneExistsLabel} are the only non-algorithmic ones.
For convenience of representation, we compose the left-hand side rule and
the right-hand side rule into one, and use substitution $[[σ]]$ to 
represent instantiation.  

The substitution application is defined 
in a standard way, avoiding capture of bound variables,
and preserving the variables that are out of the substitution domain.
The domain and the range of the substitutions are
specified by notation $[[Γ2 ⊢ σ : Γ1]]$.
For instance, the notation $[[Γ, pbs ⊢ σ : {pas}]]$ means that 
$[[σ]]$ maps the variables from $[[pas]]$ to (positive) types
well-formed in $[[Γ, pbs]]$.

\subsection{Type Inference}


\begin{figure}[h]
  \ottdefnDTNInf{}

  \hfill

  \begin{multicols}{2}
  \ottdefnDTPInf{}
  \\
  \ottdefnDTSpinInf{}
  \end{multicols}
  \hfill

  \caption{Declarative Subtyping}
  \label{fig:declarative-inference}
\end{figure}

\subsection{Properties of the Declarative System}

Now, we present selected properties of the declarative system,
which are important for the correctness of the algorithm.

\begin{lemma}
  Variables do not have proper subtypes and supertypes
\end{lemma}

\begin{lemma}
  Free Variable propagation: $[[Γ ⊢ iN ≤ iM]]$ 
  implies $[[fv(iN) ⊆ fv(iM)]]$.
\end{lemma}

\begin{lemma}
  Subtyping is Reflexive and Transitive, and is preserved by substitution
\end{lemma}

\begin{lemma}
  $[[Γ ⊢ iN ≈ iM]]$ is equivalent to $[[nf(iN) ≈ nf(iM)]]$.
\end{lemma}

\begin{lemma}[Characterization of Supertypes]
\end{lemma}