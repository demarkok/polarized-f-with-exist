\subsection{The Language}


The type syntax of \fexists is given in \cref{fig:declarative-types}.
The types of \fexists are stratified into two syntactic 
categories (polarities): positive and negative,  
similarly to the \CBPV system \cite{levy2006:cbpv}.
The negative types represent computations, and the positive types represent values:
\begin{itemize}
\item [$-$] $[[na]]$ is a negative type variable, which can be taken from a context or introduced by $[[∃]]$.
\item [$-$] a function $[[iP → iN]]$ takes a value as input and returns a computation; 
\item [$-$] a polymorphic abstraction $[[∀pas.iN]]$ quantifies a computation over
  a list of positive type variables 
  $[[pas]]$. The polarities are chosen to follow the definition of functions.
\item [$-$] a shift $[[↑iP]]$ allows a value to be used as a computation, 
  which at the term-level corresponds to a pure computation $[[return v]]$.
\item [$+$] $[[pa]]$ is a positive type variable, taken from a context or introduced by $[[∀]]$.
\item [$+$] $[[∃nas.iP]]$, symmetrically to $[[∀]]$, 
  binds negative variables in a positive type $[[iP]]$. 
\item [$+$] a shift $[[↓iN]]$, symmetrically to the up-shift, 
  thunks a computation, which at the term-level corresponds to $[[ {c} ]]$.
\end{itemize}

\begin{figure}[h]
  \begin{multicols}{2}
    \ottgrammartabular{
      \ottiN\ottinterrule
    }

    \ottgrammartabular{
      \ottiP\ottinterrule
    }
    \columnbreak
  \end{multicols}
  \caption{Declarative Types of \fexists}
  \label{fig:declarative-types}
\end{figure}
