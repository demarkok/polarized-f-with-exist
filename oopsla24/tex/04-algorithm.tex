        % - Justify Adding α̂± 
        % - algorithmic type contexts Ξ and constraint context Θ
        % - Describe how wf, nf,... are changed
        % - Constraints -- important part of the algorithm



        % - Dependency Graph


In this section, we present the algorithmization of the system described before
Shadowing the declarative system, the algorithm has two main parts: 
the subtyping and the type inference, which we discuss in this section one after another. 
First, let us discuss the syntax of the algorithmic system.

\begin{description}
        \item[Positive Algorithmic Variables] $[[α̂⁺]]$, $[[β̂⁺]]$, $[[γ̂⁺]]$, \dots 
        \item[Negative Algorithmic Variables] $[[α̂⁻]]$, $[[β̂⁻]]$, $[[γ̂⁻]]$, \dots
        \item[Positive Algorithmic Types] $[[uP]] = \dots \mid [[α̂⁺]]$  
        \item[Negative Algorithmic Types] $[[uN]] = \dots \mid [[α̂⁻]]$
        \item[Algorithmic Type Context] $[[Ξ]] = \text{set of Algorithmic Variables}$
\end{description}

\paragraph{Algorithmic Variables}
Both subtyping and the inference algorithms are represented by sets of inference rules, 
to simplify the soundness and completeness proofs \wrt the declarative specification. 
The terms these rules manipulate we call \emph{algorithmic}. They extend the 
previously defined declarative terms and types by adding \emph{algorithmic type variables}
(\aka existential variables). The algorithmic variables represent unknown types, 
which cannot be inferred immediately but are promised to be instantiated
as the algorithm proceeds.

We denote algorithmic variables as $[[α̂⁺]]$, $[[β̂⁻]]$, \dots to distinguish them from
normal variables $[[α]]$, $[[β]]$, \dots and also to smooth out the transition
from the declarative to the algorithmic system, when we replace the quantified variables $[[pas]]$ 
that we are not able to instantiate with their algorithmic counterpart $[[puas]]$.
The procedure of replacing declarative variables with algorithmic ones we call 
\emph{algorithmization} and denote as $[[ nuas/nas ]]$ and $[[ puas/pas ]]$.

\paragraph{Algorithmic Types}
The syntax of  algorithmic types is the syntax of declarative types extended
with algorithmic type variables: we add positive algorithmic variables $[[α̂⁺]]$ 
to the positive types, and negative algorithmic variables $[[α̂⁻]]$ to the negative types.
Notice that these variables cannot be abstracted by the quantifiers $[[∀]]$ and $[[∃]]$.
We denote the algorithmic types 

\paragraph{Algorithmic Contexts and Well-formedness}
To specify well-formedness, we define algorithmic contexts $[[Ξ]]$ as sets of algorithmic variables.
Then $[[Γ ; Ξ ⊢ uP]]$ and $[[Γ ; Ξ ⊢ uN]]$ represent the well-formedness
judgment of algorithmic terms defined as expected: in addition to the declarative definition
\cref{todo}, we also check that each algorithmic variable is in the context $[[Ξ]]$.

\paragraph


\begin{figure}[h]
  \centering
  \begin{tikzpicture}
    [>={Stealth[scale=2]},node distance=2.4cm,every node/.style={draw,rectangle},every text node part/.style={align=center}]


    % Define nodes
    \node[] (1) {Negative Subtyping\\$[[Γ ; Θ ⊨ uN ≤ iM ⫤ SC]]$\\(\cref{sec:subtyping})};
    \node[below of=1] (3) {Positive Subtyping\\$[[Γ ; Θ ⊨ uP ≥ iQ ⫤ SC]]$\\(\cref{sec:subtyping})};
    \node[left=1cm of 3] (2) {Subtyping Constraint Merge\\$[[Θ ⊢ SC1 & SC2 = SC3]]$\\(\cref{sec:constraint-merge})};
    \node[right=1.4cm of 3] (5) {Unification\\ $[[Γ ; Θ ⊨ uN ≈u iM ⫤ UC]]$\\ $[[Γ ; Θ ⊨ uP ≈u iQ ⫤ UC]]$\\(\cref{sec:unification})};
    \node[below of=3] (4) {Upgrade\\$[[upgrade Γ ⊢ iP to Δ = iQ]]$\\(\cref{sec:lub})};
    \node[below of=2] (6) {Least Upper Bound\\$[[Γ ⊨ iP1 ∨ iP2 = iQ]]$\\(\cref{sec:lub})};
    \node[below of=6] (7) {Anti-Unification\\$[[Γ ⊨ iP1 ≈au iP2 ⫤ ( Ξ , uQ , aus1 , aus2 )]]$\\$[[Γ ⊨ iN1 ≈au iN2 ⫤ ( Ξ , uM , aus1 , aus2 )]]$\\(\cref{sec:antiunification})};
    \node[below of=5] (8) {Unification Constraint Merge\\$[[Θ ⊢ UC1 & UC2 = UC3]]$\\(\cref{sec:constraint-merge})};

    % Define edges
    \draw[->] (1) to (2);
    \draw[->] (1) to (3);
    \draw[->] (1) to (5);
    \draw[->] (2) to (3);
    \draw[->] (2) to (6);
    \draw[->] (3) to (4);
    \draw[->] (3) to (5);
    \draw[->] (4) to (6);
    \draw[->] (5) to (8);
    \draw[->] (6) to (7);
  \end{tikzpicture}  
  \caption{Dependency graph of the subtyping algorithm}
  \label{fig:subtyping-algo}
\end{figure}


%%%%%%%%%%%%%%%%%
%%%%%%%%%%%%%%%%%
\subsection{Subtyping Algorithm}
Hello Subtyping
        % - fig
        % - it takes ... and returns constraint
        % - explain the rules
        % - - ↑/↓ invoke unification up-to-equivalence, 
        %     which is implemented by unification of the normal forms up-to-equality.
        % - - → makes two recursive calls, and thus, it merges the resulting constraints
        % - - ∀/∃ introduce new variables as discussed
        % - - the base case calls the upgrade

    \begin{figure}[h]
        \hfill\\
        \ottdefnANsub{}
        \ottdefnAPsup{}
        \caption{Subtyping Algorithm}
        \label{fig:alg-subtyping}
    \end{figure}

%%%%%%%%%%%%%%%%%
%%%%%%%%%%%%%%%%%
\subsection{Unification}

Hello Unification

\begin{figure}[h]
  \hfill
  \begin{multicols}{2}
  \ottdefnUNUnif{}
  \columnbreak\\
  \ottdefnUPUnif{}
  \end{multicols}
  \caption{Unification Algorithm}
  \label{fig:unification}
\end{figure}

%%%%%%%%%%%%%%%%%
%%%%%%%%%%%%%%%%%
\subsection{Constraint Merge}

\begin{definition} [Matching Entries]
  We call two unification constraint entries 
  or two subtyping constraint entries matching 
  if they are restricting the same unification variable.
\end{definition}

Two matching entries formed in the same context $[[Γ]]$ 
can be merged in the following way:

\begin{figure}[h]
  \ottdefnSCME\\

  \label{fig:merge-entries}
  \caption{Merge Matching Constraint Entries}
\end{figure}

\begin{itemize}
  \item \ruleref{\ottdruleSCMEPEqEqLabel} and \ruleref{\ottdruleSCMENEqEqLabel}
    are symmetric cases. To merge two matching entries restricting
    a variable to be equivalent to certain types, we check
    that these types are equivalent to each other.
    To do so, it suffices to check for \emph{equality} of their normal forms,
    as discussed in \cref{sec:equivalence-normalization}. 
    After that, we return the left-hand entry.

  \item \ruleref{\ottdruleSCMEEqSupLabel} and \ruleref{\ottdruleSCMESupEqLabel}
    are also symmetric. 
    In this case,
    since one of the entries requires the variable to be equal to 
    a type, the resulting entry must also imply that.
    However, for the soundness, it is needed to ensure that
    the equating restriction is stronger than the subtyping restriction.
    For this purpose, the premise invokes the positive subtyping.

  \item \ruleref{\ottdruleSCMESupSupLabel} 
    In this case, we find the least upper bound of the types from the input
    restrictions, 
    and as the output, restrict the variable to be a supertype of the result.
    The least upper bound procedure will be discussed in \cref{sec:lub}.
\end{itemize}

\begin{figure}[h]
  Suppose that $[[Θ ⊢ SC1]]$ and $[[Θ ⊢ SC2]]$.
  Then $[[Θ ⊢ SC1 & SC2 = SC]]$
  defines a set such that $[[ucE]] \in [[SC]]$ iff either
  \begin{itemize}
    \item $[[ucE]] \in [[SC1]]$ and there is no matching $[[ucE']] \in [[SC2]]$; or
    \item $[[ucE]] \in [[SC2]]$ and there is no matching $[[ucE']] \in [[SC1]]$; or
    \item $[[Θ(α̂±) ⊢ ucE1 & ucE2 = ucE]]$ for some $[[ucE1]] \in [[SC1]]$ and $[[ucE2]] \in [[SC2]]$
      such that $[[ucE1]]$ matches with $[[ucE2]]$ restricting variable
      $[[α̂±]]$. 
  \end{itemize}

  \label{fig:merge-subtyping-constraints}
  \caption{Constraint Merge}
\end{figure}

%%%%%%%%%%%%%%%%%
%%%%%%%%%%%%%%%%%
\subsection{Type Upgrade and the Least Upper Bounds}

\begin{figure}[h]
  \ottdefnLUBNsub{}
  \caption{Least Upper Bound Algorithm}
  \label{fig:lub}
\end{figure}

\begin{figure}[h]
    \hfill\\
    \ottdefnLUBUp{}
    \caption{Type Upgrade Algorithm}
    \label{fig:type-upgrade}
\end{figure}

%%%%%%%%%%%%%%%%%
%%%%%%%%%%%%%%%%%
\subsection{Anti-Unification}

\begin{figure}[h]
    \ottdefnAUAUP{}
    \ottdefnAUAUN{}

    \caption{Anti-unification Algorithm}
    \label{fig:anti-unification}
\end{figure}

%%%%%%%%%%%%%%%%%
%%%%%%%%%%%%%%%%%
\subsection{Type Inference}

        % - Overview (Dependency Graph)
        % *** The Inference Algorithm
        % *** Constraint Singularity

\begin{figure}
    \centering
    \begin{subfigure}[b]{0.3\textwidth}
        \centering
        \ottdefnATNInf{}
    \end{subfigure}

    \begin{subfigure}[b]{0.3\textwidth}
        \centering
        \ottdefnATPInf{}
    \end{subfigure}

    \begin{subfigure}[b]{0.3\textwidth}
        \centering
        \ottdefnATSpinInf{}
    \end{subfigure}
    \caption{Algorithmic Type Inferences}
    \label{fig:type-inference}
\end{figure}

