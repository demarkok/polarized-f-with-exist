The central results ensuring the correctness of the inference algorithm are its soundness and completeness with respect to the declarative specification.
The soundness means the algorithm will always
produce a typing \emph{allowed} by the declarative system;
Dually, the completeness says that once a term has some type declaratively, the inference algorithm succeeds. 

The proof of soundness and completeness result is done gradually
for all the subroutines of the algorithm,
following the 
structure of the algorithm 
(\cref{fig:alg-typing-graph,fig:alg-subtyping-graph})
bottom-up. 
\subsection{Normalization}
    The point of type normalization is factoring out non-trivial equivalence 
    by selecting a representative from each equivalence class.
    This way, the correctness of normalization means that
    checking for equivalence of two types is the same as checking for equality of their normal forms.
    \begin{lemma*}[Normalization Correctness]
        Assuming all types are well-formed in $[[Γ]]$, we have
            $[[Γ ⊢ iN ≈ iM]] \iff [[nf(iN) = nf(iM)]]$ and 
            $[[Γ ⊢ iP ≈ iQ]] \iff [[nf(iP) = nf(iQ)]]$.
    \end{lemma*}
    To prove the correctness of normalization, 
    we construct an intermediate relation on types---\emph{declarative equivalence}
    (the notation is $[[iN ≈ iM]]$ and $[[iP ≈ iQ]]$).
    In contrast to $[[Γ ⊢ iN ≈ iM]]$ (which means mutual subtyping), $[[iN ≈ iM]]$ does not depend on subtyping judgments, 
    but explicitly allows quantifier reordering and redundant quantifier removal.
    Then the statement $[[Γ ⊢ iN ≈ iM]] \iff [[nf(iN) = nf(iM)]]$ (as well as its positive counterpart) 
    is split into two steps: $[[Γ ⊢ iN ≈ iM]] \iff [[iN ≈ iM]] \iff [[nf(iN) = nf(iM)]]$.

\subsection{Unification Constraint Merge}


\subsection{Unification}

\begin{lemma*}[Soundness of Unification]
    \label{lemma:unification-soundness}
    \hfill
    \begin{itemize}
        \item [$+$] For normalized $[[uP]]$ and $[[iQ]]$ such that 
        $[[Γ ; dom(Θ) ⊢ uP]]$ and $[[Γ ⊢ iQ]]$,\\ 
        if $[[Γ ; Θ ⊨ uP ≈u iQ ⫤ UC]]$ then 
        $[[Θ ⊢ UC : uv uP]]$ and for any normalized $[[uσ]]$ 
        such that $[[ Θ ⊢ uσ : lift UC ]]$, $[[ [uσ]uP = iQ ]]$.

        \item [$-$] For normalized $[[uN]]$ and $[[iM]]$ such that
        $[[Γ ; dom(Θ) ⊢ uN]]$ and $[[Γ ⊢ iM]]$,\\
        if $[[Γ ; Θ ⊨ uN ≈u iM ⫤ UC]]$ then 
        $[[Θ ⊢ UC : uv uN]]$ and for any normalized $[[uσ]]$ such that
        $[[ Θ   ⊢ uσ : lift UC ]]$, $[[ [uσ]uN = iM ]]$.
    \end{itemize}
\end{lemma*}

\begin{lemma*}[Completeness of Unification]
    \label{lemma:unification-completeness}
    \hfill
    \begin{itemize}
        \item [$+$] For normalized $[[uP]]$ and $[[iQ]]$ such that
        $[[Γ ; dom(Θ) ⊢ uP]]$ and $[[Γ ⊢ iQ]]$, 
        suppose that there exists $[[Θ ⊢ uσ : uv(uP)]]$ such that $[[ [uσ]uP = iQ ]]$,
        then $[[Γ ; Θ ⊨ uP ≈u iQ ⫤ UC]]$ for some $[[UC]]$.
        
        \item [$-$] For normalized $[[uN]]$ and $[[iM]]$ such that
        $[[Γ ; dom(Θ) ⊢  uN]]$ and $[[Γ ⊢ iM]]$,
        suppose that there exists $[[Θ ⊢ uσ : uv(uN)]]$ such that $[[ [uσ]uN = iM ]]$,
        then $[[Γ ; Θ ⊨ uN ≈u iM ⫤ UC]]$ for some $[[UC]]$.
   \end{itemize}
\end{lemma*}

\subsection{Anti-Unification}

\begin{lemma*}[Soundness of Anti-Unification]
    \label{lemma:au-soundness}
    \hfill
    \begin{itemize}
        \item [$+$]  Assuming $[[iP1]]$ and $[[iP2]]$ are normalized,
        if $[[Γ ⊨ iP1 ≈au iP2 ⫤ (Ξ, uQ, aus1, aus2)]]$
        then 
        \begin{enumerate}
            \item $[[Γ ; Ξ ⊢ uQ]]$,
            \item $[[Γ ; · ⊢ ausi : Ξ]]$ for $i \in \{1,2\}$
            are anti-unification substitutions, and
            \item $[[ [ausi] uQ = iPi ]]$ for $i \in \{1,2\}$.
        \end{enumerate}

        \item [$-$] Assuming $[[iN1]]$ and $[[iN2]]$ are normalized,
        if $[[Γ ⊨ iN1 ≈au iN2 ⫤ (Ξ, uM, aus1, aus2)]]$
        then
        \begin{enumerate}
            \item $[[Γ ; Ξ ⊢ uM]]$,
            \item $[[Γ ; · ⊢ ausi : Ξ]]$ for $i \in \{1,2\}$
            are anti-unification substitutions, and
            \item $[[ [ausi] uM = iNi ]]$ for $i \in \{1,2\}$.
        \end{enumerate}
    \end{itemize}
\end{lemma*}

\begin{lemma*}[Completeness of Anti-Unification]
    \label{lemma:au-completeness}
    \hfill
    \begin{itemize}
        \item [$+$] 
            Assume that $[[iP1]]$ and $[[iP2]]$ are normalized, and
            there exists $[[(Ξ', uQ', aus'1, aus'2)]]$ such that
            \begin{enumerate}
                \item $[[Γ ; Ξ' ⊢ uQ']]$,
                \item $[[Γ ; · ⊢ aus'i : Ξ']]$ for $i \in \{1,2\}$ 
                are anti-unification substitutions, and
                \item $[[ [aus'i] uQ' = iPi ]]$ for $i \in \{1,2\}$.
            \end{enumerate}

            Then the anti-unification algorithm terminates, that is there exists
            $[[(Ξ, uQ, aus1, aus2)]]$ such that $[[Γ ⊨ iP1 ≈au iP2 ⫤ (Ξ, uQ, aus1, aus2)]]$

        \item [$-$] 
            Assume that $[[iN1]]$ and $[[iN2]]$ are normalized, and
            there exists $[[(Ξ', uM', aus'1, aus'2)]]$ such that
            \begin{enumerate}
                \item $[[Γ ; Ξ' ⊢ uM']]$,
                \item $[[Γ ; · ⊢ aus'i : Ξ']]$ for $i \in \{1,2\}$,
                are anti-unification substitutions, and
                \item $[[ [aus'i] uM' = iNi ]]$ for $i \in \{1,2\}$.
            \end{enumerate}

            Then the anti-unification algorithm succeeds, that is 
            there exists $[[(Ξ, uM, aus1, aus2)]]$ such that
            $[[Γ ⊨ iN1 ≈au iN2 ⫤ (Ξ, uM, aus1, aus2)]]$.
    \end{itemize}
\end{lemma*}

\begin{lemma*}[Initiality of Anti-Unification]
    \hfill
    \begin{itemize}
        \item [$+$] 
            Assume that $[[iP1]]$ and $[[iP2]]$ are normalized, and
            $[[Γ ⊨ iP1 ≈au iP2 ⫤ (Ξ, uQ, aus1, aus2)]]$, 
            then $[[(Ξ, uQ, aus1, aus2)]]$ is more specific than
            any other sound anti-unifier $[[(Ξ', uQ', aus'1, aus'2)]]$, i.e.
            if 
            \begin{enumerate}
                \item $[[Γ ; Ξ' ⊢ uQ']]$,
                \item $[[Γ ; · ⊢ aus'i : Ξ']]$ for $i \in \{1,2\}$, and
                \item $[[ [aus'i] uQ' = iPi ]]$ for $i \in \{1,2\}$
            \end{enumerate}
            then there exists $[[ausr]]$ such that
            $[[Γ ; Ξ ⊢ ausr : (Ξ' | uv uQ')]]$ and $[[ [ausr] uQ' = uQ ]]$. 
            Moreover, $[[ [ausr]β̂⁻]]$ can
            be uniquely determined by $[[ [aus'1]β̂⁻ ]]$, $[[ [aus'2]β̂⁻ ]]$, and
            $[[Γ]]$.
        \item [$-$] 
            Assume that $[[iN1]]$ and $[[iN2]]$ are normalized, and
            $[[Γ ⊨ iN1 ≈au iN2 ⫤ (Ξ, uM, aus1, aus2)]]$, 
            then $[[(Ξ, uM, aus1, aus2)]]$ is more specific than
            any other sound anti-unifier $[[(Ξ', uM', aus'1, aus'2)]]$, i.e.
            if
            \begin{enumerate}
                \item $[[Γ ; Ξ' ⊢ uM']]$,
                \item $[[Γ ; · ⊢ aus'i : Ξ']]$ for $i \in \{1,2\}$, and
                \item $[[ [aus'i] uM' = iNi ]]$ for $i \in \{1,2\}$
            \end{enumerate}
            then there exists $[[ausr]]$ such that
            $[[Γ ; Ξ ⊢ ausr : (Ξ' | uv uM')]]$ and $[[ [ausr] uM' = uM ]]$.
            Moreover, $[[ [ausr]β̂⁻]]$ can
            be uniquely determined by $[[ [aus'1]β̂⁻ ]]$, $[[ [aus'2]β̂⁻ ]]$, and
            $[[Γ]]$.
    \end{itemize}
\end{lemma*}


\subsection{Least Upper Bound and Upgrade}

    The Least Upper Bound algorithm finds the least type that is a supertype of two given types.
    The \emph{soundness} means that the returned type is indeed a supertype of the given ones;
    the \emph{completeness} means that the algorithm succeeds if the least upper bound exists;
    and the \emph{initiality} means that the returned type is the least among common supertypes. 

    \begin{lemma*}[Least Upper Bound Soundness]
        For types $[[Γ ⊢ iP1]]$, and $[[Γ ⊢ iP2]]$,
        if $[[Γ ⊨ iP1 ∨ iP2 = iQ]]$ then
        \begin{enumerate}
            \item[(i)]  $[[Γ ⊢ iQ]]$
            \item[(ii)] $[[Γ ⊢ iQ ≥ iP1]]$ and $[[Γ ⊢ iQ ≥ iP2]]$
        \end{enumerate}
    \end{lemma*}


    \begin{lemma*}[Least Upper Bound Completeness and Initiality]
        For types $[[Γ ⊢ iP1]]$, $[[Γ ⊢ iP2]]$, and $[[Γ ⊢ iQ]]$
        such that $[[Γ ⊢ iQ ≥ iP1]]$ and $[[Γ ⊢ iQ ≥ iP2]]$,
        there exists $[[iQ']]$ s.t. $[[Γ ⊨ iP1 ∨ iP2 = iQ']]$ 
        and $[[Γ ⊢ iQ ≥ iQ']]$.
    \end{lemma*}

    The key observation that allows us to prove these properties is
    the characterization of positive supertypes. 
    The following lemma justifies the cases of the Least Upper Bound algorithm
    (\cref{sec:lub}). In particular, it establishes
    the correspondence between the upper bounds of shifted types $[[↓iM]]$ and
    \emph{patterns} fitting $[[iM]]$ (represented by existential types), which 
    explains the usage of anti-unification as a way to find a common pattern. 

    \begin{lemma*}[Characterization of Normalized Supertypes]
        \label{lemma:char-supertypes}
        For a normalized positive type $[[iP]]$ well-fordmed in $[[Γ]]$,
        the set of normalized $[[Γ]]$-formed supertypes of $[[iP]]$ is the following:
        \begin{itemize}
            \item if $[[iP]]$ is a variable $[[pb]]$, its only normalized supertype is $[[pb]]$ itself;
            \item if $[[iP]]$ is an existential type $[[ ∃nbs.iP' ]]$ then 
                its $[[Γ]]$-formed supertypes are the $([[G, nbs]])$-formed supertypes of $[[iP']]$ not using $[[nbs]]$;
            \item if $[[iP]]$ is a downshift type $[[↓iM]]$, 
                its supertypes have form $[[∃nas.↓iM']]$ such that there exists
                a $[[Γ]]$-formed instantiation of $[[nas]]$ in $[[↓iM']]$
                that makes $[[↓iM']]$ equal to $[[↓iM]]$, \ie $[[ [iNs/nas] ↓iM' = ↓iM ]]$.
        \end{itemize}
    \end{lemma*}

    Similarly to the Least Upper Bound algorithm, the Upgrade finds the least type among upper bounds
    (this time the set of considered upper bounds consists of supertypes well-formed in a \emph{smaller} context).
    This way, we also use the supertype characterization to prove the following property of the Upgrade algorithm. 

    \begin{lemma*}[Upgrade Soundness]
        Assuming $[[iP]]$ is well-formed in $[[Γ = Δ, pnas]]$,
        if $[[upgrade Γ ⊢ iP to Δ = iQ]]$
        then
        \begin{enumerate*}
            \item $[[Δ ⊢ iQ]]$
            \item $[[Γ ⊢ iQ ≥ iP]]$
        \end{enumerate*}
    \end{lemma*}

    \begin{lemma*}[Upgrade Completeness]
        Assuming $[[iP]]$ is well-formed in $[[Γ = Δ, pnas]]$,
        for any $[[iQ']]$ such that $[[iQ']]$ is a $[[Δ]]$-formed upper bound of $[[iP]]$, \ie
        \begin{enumerate*}
            \item $[[Δ ⊢ iQ']]$ and
            \item $[[Γ ⊢ iQ' ≥ iP]]$,
        \end{enumerate*}
        the result of the upgrade algorithm $[[iQ]]$ exists
        ($[[upgrade Γ ⊢ iP to Δ = iQ]]$) and satisfies $[[Δ ⊢ iQ' ≥ iQ]]$.
    \end{lemma*}

\subsection{Positive Subtyping}


\begin{lemma*}[Soundness of the Positive Subtyping]
    If $[[Γ ⊢ Θ]]$, $[[Γ ⊢ iQ]]$, $[[Γ ; dom(Θ) ⊢  uP]]$, and 
    $[[Γ ; Θ ⊨ uP ≥ iQ ⫤ SC]]$,
    then $[[Θ ⊢ SC : uv uP]]$ and
    for any normalized $[[uσ]]$ such that $[[ Θ ⊢ uσ : SC ]]$,
    $[[ Γ ⊢ [uσ]uP ≥ iQ ]]$.
\end{lemma*}

\begin{lemma*}[Completeness of the Positive Subtyping]
    Suppose that $[[Γ ⊢ Θ]]$, $[[Γ ⊢ iQ]]$ and $[[Γ ; dom(Θ) ⊢  uP]]$.
    Then for any $[[Θ ⊢ uσ : uv(uP)]]$ such that $[[ Γ ⊢ [uσ]uP ≥ iQ ]]$,
    there exists $[[Γ; Θ ⊨ uP ≥ iQ ⫤ SC]]$ and moreover, $[[ Θ ⊢ uσ : SC ]]$.
\end{lemma*}


\subsection{Subtyping Constraint Merge}

\begin{lemma*}[Completeness of Constraint Merge]
    \label{lemma:merge-completeness}
    Suppose that $[[Θ ⊢ SC1 : Ξ1]]$ and $[[Θ ⊢ SC2 : Ξ2]]$.
    If there exists a substitution $[[Θ ⊢ uσ : Ξ1 ∪ Ξ2]]$ such that 
    $[[ Θ ⊢ uσ : SC1 ]]$ and $[[ Θ ⊢ uσ : SC2 ]]$
    then $[[Θ ⊢ SC1 & SC2 = SC]]$ is defined.
\end{lemma*}

\begin{lemma*}[Soundness of Constraint Merge] 
    \label{lemma:merge-soundness}
    Suppose that $[[Θ ⊢ SC1 : Ξ1]]$ and $[[Θ ⊢ SC2 : Ξ2]]$ 
    and $[[Θ ⊢ SC1 & SC2 = SC]]$ is defined.
    Then 
    \begin{enumerate}
        \item $[[Θ ⊢ SC : Ξ1 ∪ Ξ2]]$,
        \item for any substitution $[[Θ ⊢ uσ : Ξ1 ∪ Ξ2]]$, 
            $[[ Θ ⊢ uσ : SC ]]$
            implies $[[ Θ ⊢ uσ : SC1 ]]$ and $[[ Θ ⊢ uσ : SC2 ]]$.
    \end{enumerate}
\end{lemma*}


\subsection{Negative Subtyping}

\begin{lemma*}[Soundness of Negative Subtyping]
    If $[[Γ ⊢ Θ]]$, $[[Γ ⊢ iM]]$, $[[Γ ; dom(Θ) ⊢ uN]]$ and 
    $[[Γ ; Θ ⊨ uN ≤ iM ⫤ SC]]$, then 
    $[[Θ ⊢ SC : uv(uN)]]$ and 
    for any normalized $[[uσ]]$ such that $[[ Θ  ⊢ uσ : SC ]]$,
    $[[ Γ ⊢ [uσ]uN ≤ iM ]]$.
\end{lemma*}

\begin{lemma*}[Completeness of the Negative Subtyping]
    Suppose that $[[Γ ⊢ Θ]]$, $[[Γ ⊢ iM]]$, $[[Γ ; dom(Θ) ⊢ uN]]$,
    and $[[uN]]$ does not contain negative unification variables ($[[α̂⁻]] \notin [[uv uN]]$).
    Then for any $[[Θ ⊢ uσ : uv(uN)]]$ such that $[[Γ ⊢ [uσ]uN ≤ iM]]$,
    there exists $[[Γ ; Θ ⊨ uN ≤ iM ⫤ SC]]$ and moreover, $[[ Θ ⊢ uσ : SC ]]$.
\end{lemma*}

\subsection{Singularity}

\begin{lemma*}[Soundness of Singularity]
    \label{lemma:singularity-soundness}
    Suppose $[[Θ ⊢ SC : Ξ]]$, and $[[SC singular with uσ]]$. 
    Then $[[ Θ ⊢ uσ : Ξ ]]$,
     $[[ Θ ⊢ uσ : SC ]]$ and for any 
    $[[uσ']]$ such that $[[Θ ⊢ uσ : SC]]$,
    $[[Θ ⊢ uσ' ≈ uσ : Ξ]]$.
\end{lemma*}

\begin{lemma*}[Completeness of Singularity]
    \label{lemma:singularity-completeness}
    For a given $[[Θ ⊢ SC]]$,
    suppose that all the substitutions satisfying $[[SC]]$ are equivalent
    on $[[Ξ]] \supseteq [[dom(SC)]]$.
    In other words, suppose that there exists $[[Θ ⊢ uσ1 : Ξ]]$ such that
    for any $[[Θ ⊢ uσ : Ξ]]$, $[[Θ ⊢ uσ : SC]]$ implies 
    $[[Θ ⊢ uσ ≈ uσ1 : Ξ]]$.
    Then 
    \begin{itemize}
        \item $[[SC singular with uσ0]]$ for some $[[uσ0]]$ and
        \item $[[Ξ]] = [[dom(SC)]]$.
    \end{itemize} 
\end{lemma*}


\subsection{Type Inference}

\begin{lemma*}[Soundness of Typing]
    Suppose that $[[Γ ⊢ Φ]]$.
    For an inference tree $T_1$,
    \hfill
    \begin{itemize}
        \item [$+$] If $T_1$ infers $[[Γ; Φ ⊨ v : iP]]$ then $[[Γ ⊢ iP]]$ and $[[Γ; Φ ⊢ v : iP]]$
        \item [$-$] If $T_1$ infers $[[Γ; Φ ⊨ c : iN]]$ then $[[Γ ⊢ iN]]$ and $[[Γ; Φ ⊢ c : iN]]$
        \item  If $T_1$ infers $[[Γ; Φ; Θ ⊨ uN ● args ⇒> uM ⫤ Θ'; SC]]$
                for $[[Γ ⊢ Θ]]$ and $[[Γ; dom(Θ) ⊢  uN]]$ free from negative algorithmic variables, 
                then
            \begin{enumerate}
                \item $[[Γ ⊢ Θ']]$
                \item $[[Θ ⊆ Θ']]$
                \item $[[Γ; dom(Θ') ⊢  uM]]$
                \item $[[dom(Θ) ∩ uv(uM) ⊆ uv uN]]$
                \item $[[uM]]$ is normalized and free from negative algorithmic variables
                \item $[[Θ'|uv uN ∪ uv uM ⊢ SC]]$
                \item for any $[[ Θ' ⊢ uσ : uv uN ∪ uv uM ]]$,
                    $[[ Θ' ⊢ uσ : SC ]]$ implies $[[ Γ ; Φ ⊢ [uσ]uN ● args ⇒> [uσ]uM ]]$
            \end{enumerate}
    \end{itemize}
\end{lemma*}


\begin{lemma*}[Completeness of Typing]
    Suppose that $[[Γ ⊢ Φ]]$.
    For an inference tree $T_1$,
    \begin{itemize}
        \item [$+$] If $T_1$ infers $[[Γ; Φ ⊢ v : iP]]$ then $[[Γ; Φ ⊨ v : nf(iP)]]$        
        \item [$-$] If $T_1$ infers $[[Γ; Φ ⊢ c : iN]]$ then  $[[Γ; Φ ⊨ c : nf(iN)]]$
        \item [$\bullet$] If 
            $T_1$ infers $[[Γ; Φ ⊢ [uσ]uN ● args ⇒> iM]]$
            and 
            \begin{enumerate}
                \item $[[Γ ⊢ Θ]]$, 
                \item $[[Γ ⊢ iM]]$,
                \item $[[Γ; dom(Θ) ⊢ uN]]$ (free from negative algorithmic variables, that is $[[α̂⁻]] \notin [[uv uN]]$), and
                \item $[[Θ ⊢ uσ : uv(uN)]]$,
            \end{enumerate}
            then there exist $[[uM']]$, $[[Θ']]$, and $[[SC]]$ such that
            \begin{enumerate}
                \item $[[ Γ; Φ; Θ ⊨ uN ● args ⇒> uM' ⫤ Θ'; SC ]]$ and
                \item for any $[[Θ ⊢ uσ : uv(uN)]]$ and $[[Γ ⊢ iM]]$
                    such that $[[Γ; Φ ⊢ [uσ]uN ● args ⇒> iM]]$, 
                    there exists $[[uσ']]$ such that 
                    \begin{enumerate}
                        \item $[[Θ' ⊢ uσ' : uv uN ∪ uv uM]]$
                            and $[[Θ' ⊢ uσ' : SC]]$,
                        \item $[[Θ ⊢ uσ' ≈ uσ : uv uN]]$, and 
                        \item $[[Γ ⊢ [uσ']uM' ≈ iM]]$.
                    \end{enumerate}
            \end{enumerate}
    \end{itemize}
\end{lemma*}

\subsection{Determinicity}




