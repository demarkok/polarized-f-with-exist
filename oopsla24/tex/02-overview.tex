\subsection{What types do we infer?}


\subsection{The Language of Types}

The types of \fexists are given in \cref{fig:declarative-types}.
They are stratified into two syntactic 
categories (polarities): positive and negative,  
similarly to the \CBPV system \cite{levy2006:cbpv}.
The negative types represent computations, and the positive types represent values:
\begin{itemize}
\item [$-$] $[[na]]$ is a negative type variable, which can be taken from a context or introduced by $[[∃]]$.
\item [$-$] a function $[[iP → iN]]$ takes a value as input and returns a computation; 
\item [$-$] a polymorphic abstraction $[[∀pas.iN]]$ quantifies a computation over
  a list of positive type variables 
  $[[pas]]$. The polarities are chosen to follow the definition of functions.
\item [$-$] a shift $[[↑iP]]$ allows a value to be used as a computation, 
  which at the term level corresponds to a pure computation $[[return v]]$.
\item [$+$] $[[pa]]$ is a positive type variable, taken from a context or introduced by $[[∀]]$.
\item [$+$] $[[∃nas.iP]]$, symmetrically to $[[∀]]$, 
  binds negative variables in a positive type $[[iP]]$. 
\item [$+$] a shift $[[↓iN]]$, symmetrically to the up-shift, 
  thunks a computation, which at the term level corresponds to $[[ {c} ]]$.
\end{itemize}

\begin{figure}[h]
  \begin{multicols}{2}
    \ottgrammartabular{
      \ottiN\ottinterrule
    }\\
    \ottgrammartabular{
      \ottiP\ottinterrule
    }
    \columnbreak
  \end{multicols}
  \caption{Declarative Types of \fexists}
  \label{fig:declarative-types}
\end{figure}

\paragraph{Definitional Equalities}
For simplicity, we assume that alpha-equivalent terms are equal.
This way, we assume that substitutions do not capture bound variables.
Besides, we equate
$[[∀pas.∀pbs.iN]]$ with $[[∀pas,pbs.iN]]$, 
as well as $[[∃nas.∃nbs.iP]]$ with $[[∃nas,nbs.iP]]$,
and lift these equations transitively and congruently 
to the whole system.

\subsection{The Language of Terms}

In \cref{fig:declarative-terms}, we define the language of terms of 
\fexists. The language combines \systemf with the \CBPV approach.

\begin{itemize}
    \item [$+$] $[[x]]$ denotes a (positive) term variable; 
        \ilyam{why no negatives? Following CBPV}
    \item [$+$] $[[{c}]]$ is a value corresponding to a thunked 
        or suspended computation;
    \item [$\pm$] $[[(c : iN)]]$ and $[[(v : iP)]]$ allow one to annotate 
        positive and negative terms;
    \item [$-$] $[[return v]]$ is a pure computation, returning a value;
    \item [$-$] $[[λ x : iP . c]]$ and $[[Λ pa . c]]$
        are standard lambda abstractions. Notice that we require
        the type annotation for the argument of $[[λ]]$;
    \item [$-$] $[[ let x = v ; c]]$ is a standard let, binding
        a value $[[v]]$ to a variable $[[x]]$ in a computation $[[c]]$;
    \item [$-$] Applicative let forms $[[let x : iP = v ( args ) ; c]]$ and
        $[[let x = v ( args ) ; c]]$ operate similarly to 
        the bind of a monad: they take a suspended computation $[[v]]$,
        apply it to a list of arguments, bind the result 
        (which is expected to be pure) to a variable $[[x]]$,
        and continue with a computation $[[c]]$.
        If the resulting type of the application is unique, 
        one can omit the type annotation, as in the second form:
        it will be inferred by the algorithm;
    \item [$-$] $[[let∃ ( nas , x ) = v ; c]]$
        is the standard unpack of an existential type:
        expecting $[[v]]$ to be an existential type,
        it binds the packed negative types to a list of 
        variables $[[nas]]$, binds the body of the existential
        to $[[x]]$, and continues with a computation $[[c]]$.
\end{itemize}

\paragraph{Missing constructors}
Notice that the language does not have first-class applications: 
their role is played by the applicative let forms, binding 
the result of a \emph{fully applied} function to a variable.
Also notice that the language does not have a type application 
(i.e. the eliminator of $[[∀]]$) and dually, 
it does not have \pack (i.e. the constructor of $[[∃]]$).
This is because the instantiation of polymorphic and existential 
types is inferred by the algorithm.  \ilyam{refer to the extension chapter}

\begin{figure}[h]
  \begin{multicols}{2}
    \ottgrammartabular{
      \ottc\ottinterrule
    }

    \ottgrammartabular{
      \ottv\ottinterrule
    }
  \end{multicols}
  \caption{Declarative Terms of \fexists}
  \label{fig:declarative-terms}
\end{figure}

\subsection{The key ideas of the algorithm}

