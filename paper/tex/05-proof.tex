\label{sec:proofs}
The central results ensuring the correctness of the inference algorithm are its
soundness and completeness with respect to the declarative specification. The
soundness means the algorithm will always produce a typing \emph{allowed} by the
declarative system; dually, the completeness says that once a term has some type
declaratively, the inference algorithm succeeds. 

The formal statements of soundness and completeness are given in the theorems
below. Notice that the theorems also include the soundness and completeness of
\emph{application inference} (labeled as $\bullet$), which is more complex. As
such, let us discuss it in more detail.

Both soundness and completeness of application inference assume that the input
head type $[[uN]]$ is free from \emph{negative} algorithmic variables---it is
achieved by polarization invariants preserved by the inference rules. The
soundness states that the output of the algorithm---$[[uM]]$ and $[[SC]]$---is
viable. Specifically, that the constraint set $[[SC]]$ provides a sufficient set
of restrictions that any substitution $[[uσ]]$ must satisfy to ensure the
\emph{declarative} inference of the output type $[[uM]]$, that is 
$[[ Γ ; Φ ⊢ [uσ]uN ● args ⇒> [uσ]uM ]]$.

The application inference completeness means that if there exists a substitution
$[[uσ]]$ and the resulting type $[[iM]]$ ensuring the declarative inference
$[[Γ; Φ ⊢ [uσ]uN ● args ⇒> iM]]$ then the algorithm succeeds and gives the most
general result $[[uM0]]$ and $[[SC0]]$. The property of `being the most general'
is specified in pt. (\ref{point:mostGeneral}). Intuitively, it means that for
any other solution---substitution $[[uσ]]$ and the resulting type $[[iM]]$, if
it ensures the declarative inference, then $[[uσ]]$ can be extended in a
$[[SC0]]$-complying way to equate $[[uM0]]$ with $[[iM]]$.

\footnotetext[1]{The presented properties hold, 
            but the actual inductive proof requires strengthening 
            of the statement and the corresponding theorem is more involved. 
            See the appendix (\cref{sec:appendix}) for details.}

\begin{theorem*}[Soundness of Typing]
    \label{thm:soundness-typing}
    Suppose that $[[Γ ⊢ Φ]]$. Then\footnotemark[1]
    \hfill
    \begin{itemize}
        \item [$+$] $[[Γ; Φ ⊨ v : iP]]$ implies $[[Γ; Φ ⊢ v : iP]]$,
        \item [$-$] $[[Γ; Φ ⊨ c : iN]]$ implies $[[Γ; Φ ⊢ c : iN]]$,
        \item [$\bullet$] $[[Γ; Φ; Θ ⊨ uN ● args ⇒> uM ⫤ Θ'; SC]]$ implies $[[ Γ ; Φ ⊢ [uσ]uN ● args ⇒> [uσ]uM ]]$, 
            for any instantiation of $[[uσ]]$ satisfying constraints $[[SC]]$.
            All of it under assumptions that $[[Γ ⊢ Θ]]$ and $[[Γ; dom(Θ) ⊢ uN]]$ and that $[[uN]]$ is free from 
            \emph{negative} algorithmic variables.
    \end{itemize}
\end{theorem*}

\begin{theorem*}[Completeness of Typing]
    \label{thm:completeness-typing}
    Suppose that $[[Γ ⊢ Φ]]$. Then\footnotemark[1]
    \begin{itemize}
        \item [$+$] $[[Γ; Φ ⊢ v : iP]]$ implies $[[Γ; Φ ⊨ v : nf(iP)]]$,
        \item [$-$] $[[Γ; Φ ⊢ c : iN]]$ implies $[[Γ; Φ ⊨ c : nf(iN)]]$,
        \item [$\bullet$] If $[[Γ; Φ ⊢ [uσ]uN ● args ⇒> iM]]$
            where 
            \begin{enumerate*}
                \item $[[Γ ⊢ Θ]]$, 
                \item $[[Γ ⊢ iM]]$,
                \item $[[Γ; dom(Θ) ⊢ uN]]$ (free from \emph{negative} algorithmic variables), and
                \item $[[Θ ⊢ uσ : uv(uN)]]$,
            \end{enumerate*}
            then there exist $[[uM0]]$, $[[Θ0]]$, and $[[SC0]]$ such that
            \begin{enumerate}
                \item $[[ Γ; Φ; Θ ⊨ uN ● args ⇒> uM0 ⫤ Θ0; SC0 ]]$ and
                \item \label{point:mostGeneral} for any other $[[uσ]]$ and $[[iM]]$ 
                (where $[[Θ ⊢ uσ : uv(uN)]]$ and $[[Γ ⊢ iM]]$)
                    such that $[[Γ; Φ ⊢ [uσ]uN ● args ⇒> iM]]$, 
                    there exists $[[uσ']]$ such that 
                    \begin{enumerate*}
                        \item $[[Θ0 ⊢ uσ' : uv uN ∪ uv uM0]]$
                            and $[[Θ0 ⊢ uσ' : SC0]]$,
                        \item $[[Θ ⊢ uσ' ≈ uσ : uv uN]]$, and 
                        \item $[[Γ ⊢ [uσ']uM0 ≈ iM]]$.
                    \end{enumerate*}
            \end{enumerate}
    \end{itemize}
\end{theorem*}



The proof of soundness and completeness result is done gradually
for all the subroutines,
following the structure of the algorithm 
(\cref{fig:alg-typing-graph,fig:alg-subtyping-graph})
bottom-up. Next, we discuss the main of these results. 

\subsection{Normalization}
    The point of type normalization is factoring out non-trivial equivalence 
    by selecting a representative from each equivalence class.
    This way, \emph{the correctness of normalization} means that
    checking for equivalence of two types is the same as checking for equality of their normal forms.
    \begin{lemma*}[Normalization correctness]
        Assuming all types are well-formed in $[[Γ]]$, we have
            $[[Γ ⊢ iN ≈ iM]] \iff [[nf(iN) = nf(iM)]]$ and 
            $[[Γ ⊢ iP ≈ iQ]] \iff [[nf(iP) = nf(iQ)]]$.
    \end{lemma*}
    To prove the correctness of normalization, 
    we introduce an \emph{intermediate} relation on types---\emph{declarative equivalence}
    (the notation is $[[iN ≈ iM]]$ and $[[iP ≈ iQ]]$).
    In contrast to $[[Γ ⊢ iN ≈ iM]]$ (which means mutual subtyping), $[[iN ≈ iM]]$ does not depend on subtyping judgments, 
    but explicitly allows quantifier reordering and redundant quantifier removal.
    Then the statement $[[Γ ⊢ iN ≈ iM]] \iff [[nf(iN) = nf(iM)]]$ (as well as its positive counterpart) 
    is split into two steps: $[[Γ ⊢ iN ≈ iM]] \iff [[iN ≈ iM]] \iff [[nf(iN) = nf(iM)]]$.

% \subsection{Unification Constraint Merge}

% \subsection{Unification}

% \begin{lemma*}[Soundness of Unification]
%     \label{lemma:unification-soundness}
%     \hfill
%     \begin{itemize}
%         \item [$+$] For normalized $[[uP]]$ and $[[iQ]]$ such that 
%         $[[Γ ; dom(Θ) ⊢ uP]]$ and $[[Γ ⊢ iQ]]$,\\ 
%         if $[[Γ ; Θ ⊨ uP ≈u iQ ⫤ UC]]$ then 
%         $[[Θ ⊢ UC : uv uP]]$ and for any normalized $[[uσ]]$ 
%         such that $[[ Θ ⊢ uσ : lift UC ]]$, $[[ [uσ]uP = iQ ]]$.

%         \item [$-$] For normalized $[[uN]]$ and $[[iM]]$ such that
%         $[[Γ ; dom(Θ) ⊢ uN]]$ and $[[Γ ⊢ iM]]$,\\
%         if $[[Γ ; Θ ⊨ uN ≈u iM ⫤ UC]]$ then 
%         $[[Θ ⊢ UC : uv uN]]$ and for any normalized $[[uσ]]$ such that
%         $[[ Θ   ⊢ uσ : lift UC ]]$, $[[ [uσ]uN = iM ]]$.
%     \end{itemize}
% \end{lemma*}

% \begin{lemma*}[Completeness of Unification]
%     \label{lemma:unification-completeness}
%     \hfill
%     \begin{itemize}
%         \item [$+$] For normalized $[[uP]]$ and $[[iQ]]$ such that
%         $[[Γ ; dom(Θ) ⊢ uP]]$ and $[[Γ ⊢ iQ]]$, 
%         suppose that there exists $[[Θ ⊢ uσ : uv(uP)]]$ such that $[[ [uσ]uP = iQ ]]$,
%         then $[[Γ ; Θ ⊨ uP ≈u iQ ⫤ UC]]$ for some $[[UC]]$.
        
%         \item [$-$] For normalized $[[uN]]$ and $[[iM]]$ such that
%         $[[Γ ; dom(Θ) ⊢  uN]]$ and $[[Γ ⊢ iM]]$,
%         suppose that there exists $[[Θ ⊢ uσ : uv(uN)]]$ such that $[[ [uσ]uN = iM ]]$,
%         then $[[Γ ; Θ ⊨ uN ≈u iM ⫤ UC]]$ for some $[[UC]]$.
%    \end{itemize}
% \end{lemma*}

\subsection{Anti-Unification}

The anti-unifier of the two types is the most specific pattern that matches
against both of them. In our setting, the anti-unifiers are restricted further:
first, the pattern might only contain `holes' at \emph{negative} positions
(because eventually, the `holes' become variables abstracted by the
existential quantifier); second, the anti-unification is parametrized with a
context $[[Γ]]$, in which the pattern instantiations must be well-formed.

This way, the correctness properties of the anti-unification algorithm are
refined accordingly. The soundness of anti-unification not only ensures that the
resulting pattern matches with the input types, but also that the pattern
instantiations are well-formed in the corresponding context, and that all the
`hole' variables are negative. In turn, completeness states that if there
exists a solution satisfying the soundness criteria, then the algorithm
succeeds.


The correctness properties are formulated by the following lemmas. For brevity,
we only provide the statements for the positive case, since the negative case is
symmetric. 


\begin{lemma*}[Soundness of (positive) anti-unification]
    \label{lemma:au-soundness}
     Assuming $[[iP1]]$ and $[[iP2]]$ are normalized,
    if $[[Γ ⊨ iP1 ≈au iP2 ⫤ (Ξ, uQ, aus1, aus2)]]$
    then 
    \begin{enumerate*}
        \item $[[Γ ; Ξ ⊢ uQ]]$,
        \item $[[Γ ; · ⊢ ausi : Ξ]]$ for $i \in \{1,2\}$
        are anti-unification substitutions (in particular, $[[Ξ]]$ contains only negative algorithmic variables), and
        \item $[[ [ausi] uQ = iPi ]]$ for $i \in \{1,2\}$.
    \end{enumerate*}
\end{lemma*}

\begin{lemma*}[Completeness of (positive) anti-unification]
    \label{lemma:au-completeness}
    Assuming that $[[iP1]]$ and $[[iP2]]$ are normalized terms well-formed in $[[Γ]]$
    and there exist  $[[(Ξ', uQ', aus'1, aus'2)]]$ such that
    \begin{enumerate*}
        \item $[[Γ ; Ξ' ⊢ uQ']]$,
        \item $[[Γ ; · ⊢ aus'i : Ξ']]$ for $i \in \{1,2\}$ 
        are anti-unification substitutions, and
        \item $[[ [aus'i] uQ' = iPi ]]$ for $i \in \{1,2\}$.
    \end{enumerate*}

    Then the anti-unification algorithm terminates, that is there exists
    $[[(Ξ, uQ, aus1, aus2)]]$ such that $[[Γ ⊨ iP1 ≈au iP2 ⫤ (Ξ, uQ, aus1, aus2)]]$.
\end{lemma*}

Notice that for the anti-unification substitution $[[aus]]$ we use a separate
signature specifying the domain and the range. $[[Γ ; Ξ2 ⊢ aus : Ξ1]]$ means
that $[[aus]]$ maps the `holes' (\ie algorithmic variables) from $[[Ξ1]]$
to \emph{algorithmic} types well-formed in $[[Γ]]$ and $[[Ξ2]]$. To put it
formally, $[[Γ ; Ξ2 ⊢ [aus]α̂⁻]]$ for any $[[α̂⁻ ∊ Ξ1]]$.
Although in the formulation of soundness and completeness, the resulting types are declarative
(\ie $[[Ξ2]]$ is always empty), we need the possibility of mapping the
`holes' to types with `holes' to formulate the \emph{initiality} of
anti-unification.

The initiality shows that the anti-unifier that the algorithm provides is the
most specific (or the most `detailed'). Specifically, it means that any other
sound anti-unification solution can be `refined' to the result of the algorithm.  
The `refinement' is represented as an instantiation of the anti-unifier---a 
substitution $[[Γ ; Ξ2 ⊢ ausr : Ξ1]]$ replacing the `holes' from $[[Ξ1]]$ with
types that themselves can contain `holes' from $[[Ξ2]]$.

\begin{lemma*}[Initiality of Anti-Unification]
        Assume that $[[iP1]]$ and $[[iP2]]$ are normalized types well-formed in $[[Γ]]$, 
        and the anti-unification algorithm succeeds: $[[Γ ⊨ iP1 ≈au iP2 ⫤ (Ξ, uQ, aus1, aus2)]]$. 
        Then $[[(Ξ, uQ, aus1, aus2)]]$ is more specific than
        any other sound anti-unifier $[[(Ξ', uQ', aus'1, aus'2)]]$, \ie if
        \begin{enumerate*}
            \item $[[Γ ; Ξ' ⊢ uQ']]$,
            \item $[[Γ ; · ⊢ aus'i : Ξ']]$ for $i \in \{1,2\}$, and
            \item $[[ [aus'i] uQ' = iPi ]]$ for $i \in \{1,2\}$
        \end{enumerate*}
        then there exists a `refining' substitution $[[ausr]]$ such that
        $[[Γ ; Ξ ⊢ ausr : (Ξ' | uv uQ')]]$ and $[[ [ausr] uQ' = uQ ]]$. 
        % Moreover, $[[ [ausr]β̂⁻]]$ is uniquely determined by $[[ [aus'1]β̂⁻ ]]$, 
        % $[[ [aus'2]β̂⁻ ]]$, and $[[Γ]]$.
\end{lemma*}

To prove the correctness properties of the anti-unification algorithm,
one extra observation is essential. The algorithm relies on the fact that
in the resulting tuple $[[(Ξ, uQ, aus1, aus2)]]$,
there are no two different `holes' $[[β̂⁻]]$
mapped to the same pair of types by $[[aus1]]$ and $[[aus2]]$.
This is used to ensure that, for example, 
the anti-unifier of $[[↓↑Int → ↑Int]]$ and 
$[[↓↑Bool → ↑Bool]]$ is $[[↓â⁻ → â⁻]]$, but not
(less specific) $[[↓â⁻ → b̂⁻]]$.
To preserve this property, we arrange the algorithm in such a way 
that the name of the `hole' is determined by the types it is mapped to.
The following lemma specifies this observation.

\begin{lemma*}[Uniqueness of Anti-unification Variable Names]
    Names of the anti-unification variables are uniquely defined by
    the types they are mapped to by the resulting substitutions. 
    Assuming $[[iP1]]$ and $[[iP2]]$ are normalized,
        if $[[Γ ⊨ iP1 ≈au iP2 ⫤ (Ξ, uQ, aus1, aus2)]]$
        then for any $[[β̂⁻]] \in [[Ξ]]$,
        $[[β̂⁻]] = [[â⁻_{[aus1]β̂⁻, [aus2]β̂⁻}]]$.
\end{lemma*}

\subsection{Least Upper Bound and Upgrade}
\label{sec:proof-lub-upgrade}

    The Least Upper Bound algorithm finds the least type that is a supertype of two given types.
    Its \emph{soundness} means that the returned type is indeed a supertype of the given ones;
    the \emph{completeness} means that the algorithm succeeds if the least upper bound exists;
    and the \emph{initiality} means that the returned type is the least among common supertypes. 

    \begin{lemma*}[Least Upper Bound soundness]
        For types $[[Γ ⊢ iP1]]$, and $[[Γ ⊢ iP2]]$,
        if $[[Γ ⊨ iP1 ∨ iP2 = iQ]]$ then
        \begin{enumerate}
            \item[(i)]  $[[Γ ⊢ iQ]]$
            \item[(ii)] $[[Γ ⊢ iQ ≥ iP1]]$ and $[[Γ ⊢ iQ ≥ iP2]]$
        \end{enumerate}
    \end{lemma*}


    \begin{lemma*}[Least Upper Bound completeness and initiality]
        For types $[[Γ ⊢ iP1]]$, $[[Γ ⊢ iP2]]$, and $[[Γ ⊢ iQ]]$
        such that $[[Γ ⊢ iQ ≥ iP1]]$ and $[[Γ ⊢ iQ ≥ iP2]]$,
        there exists $[[iQ']]$ s.t. $[[Γ ⊨ iP1 ∨ iP2 = iQ']]$ 
        and $[[Γ ⊢ iQ ≥ iQ']]$.
    \end{lemma*}

    The key observation that allows us to prove these properties is the
    characterization of positive supertypes. The following lemma justifies the
    case analysis used in the Least Upper Bound algorithm (in \cref{sec:lub}).
    In particular, it establishes the correspondence between the upper bounds of
    shifted types $[[↓iM]]$ and \emph{patterns} fitting $[[iM]]$ (represented by
    existential types), which explains the usage of anti-unification as a way to
    find a common pattern. 

    \begin{lemma*}[Characterization of normalized supertypes]
        \label{lemma:char-supertypes}
        For a normalized positive type $[[iP]]$ well-formed in $[[Γ]]$,
        the set of normalized $[[Γ]]$-formed supertypes of $[[iP]]$ is the following:
        \begin{itemize}
            \item if $[[iP]]$ is a variable $[[pb]]$, its only normalized supertype is $[[pb]]$ itself;
            \item if $[[iP]]$ is an existential type $[[ ∃nbs.iP' ]]$ then 
                its $[[Γ]]$-formed supertypes are the $([[G, nbs]])$-formed supertypes of $[[iP']]$ not using $[[nbs]]$;
            \item if $[[iP]]$ is a downshift type $[[↓iM]]$, 
                its supertypes have form $[[∃nas.↓iM']]$ such that there exists
                a $[[Γ]]$-formed instantiation of $[[nas]]$ in $[[↓iM']]$
                that makes $[[↓iM']]$ equal to $[[↓iM]]$, \ie $[[ [iNs/nas] ↓iM' = ↓iM ]]$.
        \end{itemize}
    \end{lemma*}

    Similarly to the Least Upper Bound algorithm, the Upgrade finds the least type among upper bounds
    (this time the set of considered upper bounds consists of supertypes well-formed in a \emph{smaller} context).
    This way, we also use the supertype characterization to prove the following properties of the Upgrade algorithm. 

    \begin{lemma*}[Upgrade soundness]
        Assuming $[[iP]]$ is well-formed in $[[Γ = Δ, pnas]]$,
        if $[[upgrade Γ ⊢ iP to Δ = iQ]]$
        then
        \begin{enumerate*}
            \item $[[Δ ⊢ iQ]]$ and
            \item $[[Γ ⊢ iQ ≥ iP]]$.
        \end{enumerate*}
    \end{lemma*}

    \begin{lemma*}[Upgrade Completeness]
        Assuming $[[iP]]$ is well-formed in $[[Γ = Δ, pnas]]$,
        for any $[[iQ']]$ such that $[[iQ']]$ is a $[[Δ]]$-formed upper bound of $[[iP]]$, \ie
        \begin{enumerate*}
            \item $[[Δ ⊢ iQ']]$ and
            \item $[[Γ ⊢ iQ' ≥ iP]]$,
        \end{enumerate*}
        there exists $[[iQ]]$ such that
        ($[[upgrade Γ ⊢ iP to Δ = iQ]]$) and $[[Δ ⊢ iQ' ≥ iQ]]$.
    \end{lemma*}

\subsection{Subtyping}
    \label{sec:proof-subtyping}

    As for other properties, the correctness of subtyping means that the
    algorithm produces a valid result (soundness) whenever it exists
    (completeness). In the rules defining the subtyping algorithm
    (\cref{fig:alg-subtyping}), one can see that the positive and negative
    subtyping relations are not \emph{mututally} recursive: negative subtyping
    algorithm uses the positive subtyping, but not vice versa. Because of that,
    the inductive proofs of the \emph{positive} subtyping correctness are done
    independently.

    The soundness of positive subtyping states that the output constraint
    $[[SC]]$ provides a sufficient set of restrictions. In other words, any
    substitution satisfying $[[SC]]$ ensures the desired declarative subtyping:
    $[[Γ ⊢ [uσ]uP ≥ iQ]]$. Notice that the soundness formulation assumes that
    only the left-hand side input type ($[[uP]]$) can contain algorithmic
    variables. This is one of the invariants of the algorithm that significantly
    simplifies the unification and constraint resolution.

\begin{lemma*}[Soundness of Positive Subtyping]
    If $[[Γ ⊢ Θ]]$ and $[[Γ ⊢ iQ]]$ and $[[Γ ; dom(Θ) ⊢  uP]]$ and 
    $[[Γ ; Θ ⊨ uP ≥ iQ ⫤ SC]]$,
    then $[[Θ ⊢ SC : uv uP]]$ and
    for any  $[[uσ]]$ such that $[[ Θ ⊢ uσ : SC ]]$,
    we have $[[ Γ ⊢ [uσ]uP ≥ iQ ]]$.
\end{lemma*}

    The completeness of subtyping says that if the substitution 
    ensuring the declarative subtyping exists, then the subtyping algorithm 
    succeeds (terminates).

\begin{lemma*}[Completeness of Positive Subtyping]
    Suppose that $[[Γ ⊢ Θ]]$, $[[Γ ⊢ iQ]]$ and $[[Γ ; dom(Θ) ⊢  uP]]$.
    Then if there exists $[[uσ]]$ such that $[[Θ ⊢ uσ : uv(uP)]]$ and $[[ Γ ⊢ [uσ]uP ≥ iQ ]]$,
    then the subtyping algorithm succeeds: $[[Γ; Θ ⊨ uP ≥ iQ ⫤ SC]]$.
\end{lemma*}

After the correctness properties of positive subtyping are established,
they are used to prove the correctness of the negative subtyping.
The soundness is formulated symmetrical to the positive case,
however, the completeness requires an additional invariant to be preserved.
The algorithmic input type $[[uN]]$ must be free from \emph{negative} algorithmic variables.
In particular, it ensures that the constraint restricting a \emph{negative} algorithmic
variable will never be generated, and thus, we do not need the 
Greatest Common Subtype procedure to resolve the constraints.

\begin{lemma*}[Completeness of Negative Subtyping]
    Suppose that $[[Γ ⊢ Θ]]$ and $[[Γ ⊢ iM]]$ and $[[Γ ; dom(Θ) ⊢ uN]]$
    and $[[uN]]$ does not contain negative unification variables ($[[α̂⁻]] \notin [[uv uN]]$).
    Then for any $[[Θ ⊢ uσ : uv(uN)]]$ such that $[[Γ ⊢ [uσ]uN ≤ iM]]$,
    the subtyping algorithm succeeds: $[[Γ ; Θ ⊨ uN ≤ iM ⫤ SC]]$.
\end{lemma*}

\subsection{Minimal Instantiation and Singularity}
\label{sec:singularity-proof}

Algorithmic typing relies on the \emph{minimal instantiation} procedure.
For a given positive type $[[uP]]$ and a set of constraints $[[SC]]$
it returns a substitution $[[Θ ⊢ uσ : SC]]$ such that
$[[ [uσ]uP ]]$ is a subtype of all the other instantiations of $[[uP]]$ satisfying $[[SC]]$.
This way, the correctness of minimal instantiation is formulated as the following two lemmas.

\begin{lemma*}[Soundness of Minimal Instantiation]
    \label{lemma:minimal-instantiation-soundness}
    Suppose that $[[Γ ⊢ Θ]]$, $[[Θ ⊢ SC]]$, and $[[Γ; dom(Θ) ⊢ uP]]$.
    Then `$[[Γ ⊢ uP SC minby uσ ]]$' implies that 
    $[[Θ ⊢ uσ : uv uP]]$ is a normalized substitution satisfying $[[SC]]$
    (\ie $[[Θ ⊢ uσ : SC]]$)
    and for any other substitution $[[Θ ⊢ uσ' : uv uP ]]$ satisfying $[[SC]]$,
    we have $[[Γ ⊢ [uσ']uP ≥ [uσ]uP ]]$.
\end{lemma*}

\begin{lemma*}[Completeness of Minimal Instantiation]
    Suppose that $[[Γ ⊢ Θ]]$, $[[Θ ⊢ SC]]$, $[[Γ; dom(Θ) ⊢ uP]]$, and there exists 
    $[[Θ ⊢ uσ : uv uP ]]$ satisfying $[[SC]]$ ($[[Θ ⊢ uσ : SC]]$) 
    such that for any other $[[Θ ⊢ uσ' : uv uP ]]$
    satisfying $[[SC]]$, we have
    $[[Γ ⊢ [uσ']uP ≥ [uσ]uP ]]$.  Then the minimal instantiation procedure  
    succeeds; specifically, $[[Γ ⊢ uP SC minby nf(uσ) ]]$ holds.
\end{lemma*}

The soundness is rather straightforwardly proved by induction on the 
judgment inference tree, relying on the soundness of the singularity procedure.
The proof of completeness is done by induction on the structure of the type $[[uP]]$.
It uses the fact that $[[ Γ ⊢ ∃nas.[uσ']uP ≥ ∃nas.[uσ]uP]]$ implies
$[[ Γ, nas ⊢ [uσ']uP ≥ [uσ]uP]]$.

The soundness of \emph{singularity} states that $[[SC singular with uσ]]$
implies that any substitution satisfying $[[SC]]$ is equivalent to $[[uσ]]$ 
on the domain.  The completeness states that if all the 
$[[SC]]$-compliant substitutions are equivalent, then the singularity procedure succeeds. 

\begin{lemma*}[Soundness of Singularity]
    \label{lemma:singularity-soundness}
    Suppose $[[Θ ⊢ SC : Ξ]]$, and $[[SC singular with uσ]]$. 
    Then $[[ Θ ⊢ uσ : Ξ ]]$,
     $[[ Θ ⊢ uσ : SC ]]$ and for any 
    $[[uσ']]$ such that $[[Θ ⊢ uσ : SC]]$,
    $[[Θ ⊢ uσ' ≈ uσ : Ξ]]$.
\end{lemma*}

\begin{lemma*}[Completeness of Singularity]
    \label{lemma:singularity-completeness}
    For a given set of constraints $[[Θ ⊢ SC]]$,
    suppose that all the substitutions satisfying $[[SC]]$ are equivalent
    on $[[dom(SC)]]$.
    In other words, suppose that there exists $[[Θ ⊢ uσ1 : dom(SC)]]$ such that
    for any $[[Θ ⊢ uσ : dom(SC)]]$, $[[Θ ⊢ uσ : SC]]$ implies 
    $[[Θ ⊢ uσ ≈ uσ1 : dom(SC)]]$.
    Then $[[SC singular with uσ0]]$ for some $[[uσ0]]$.
\end{lemma*}

\subsection{Typing}

Finally, we discuss the proofs of the soundness and completeness of type
inference algorithm that we stated at the beginning of this section.
There are three subtleties that we will cover that are important 
for the proof to go through: the determinacy of the algorithm,
the mutual dependence of the soundness and completeness proofs, and
the non-trivial inductive metric that we use.


\paragraph{Determinacy}
    One of the properties that our proof relies on is the determinacy of the 
    typing algorithm: the output (the inferred type)
    is uniquely determined by the input (the term and the contexts).
    Determinacy is not hard to demonstrate by structural induction: 
    in every algorithmic inference system, only one inference rule can be applied 
    for a given input.  However, we need to prove it for \emph{every} 
    subroutine of the algorithm. Ultimately, it requires 
    the determinacy of such procedures as \emph{generation of fresh variables}, 
    which is easy to ensure, but must be taken into account in the implementation.

\begin{lemma*}[Determinacy of the Typing Algorithm]
    Suppose that $[[Γ ⊢ Φ]]$ and $[[Γ ⊢ Θ]]$. Then 
    \begin{itemize}
        \item [$+$] If $[[Γ; Φ ⊨ v : iP]]$ and $[[Γ; Φ ⊨ v : iP']]$ then $[[iP = iP']]$.
        \item [$-$] If $[[Γ; Φ ⊨ c : iN]]$ and $[[Γ; Φ ⊨ c : iN']]$ then $[[iN = iN']]$.
        \item If $[[Γ; Φ; Θ ⊨ uN ● args ⇒> uM ⫤ Θ'; SC]]$ and 
            $[[Γ; Φ; Θ ⊨ uN ● args ⇒> uM' ⫤ Θ'; SC']]$ then 
            $[[uM = uM']]$, $[[Θ]] = [[Θ']]$, and $[[SC]] =[[SC']]$.  
    \end{itemize}
\end{lemma*}

\paragraph{Mutuality of the Soundness and Completeness Proofs}

Typically in our inductive proofs, the soundness is proven before completeness,
as the completeness requires the premise subtrees to satisfy certain properties
(which can be given by the soundness). However, in the case of the typing algorithm,
the soundness and completeness proofs cannot be separated: the inductive proof
of one requires another, and vice versa. 

The proof of soundness can be viewed as a mapping from an algorithmic tree to a
declarative one. We show that each algorithmic inference rule can be transformed
into the corresponding declarative rule, as long as the premises are transformed
accordingly, and apply the induction principle. 

The soundness requires completeness in the case of
\ruleref{\ottdruleATAppLetLabel}. To prove the soundness, in other words, to
transform this rule into its declarative counterpart
\ruleref{\ottdruleDTAppLetLabel}, one needs to prove the \emph{principality} of
the inferred application type $[[↑[uσ]uQ]]$. In other words, we need to conclude
that any other \emph{declarative} tree infers for the application a supertype of
$[[ [uσ]uQ ]]$. 

The only way to do so is by applying the soundness of minimal instantiation
(see the lemma above). However, first, the declarative
tree must be converted to the corresponding algorithmic one (by completeness!).
This way, the soundness and completeness proofs are mutually dependent.

\paragraph{Inductive Metric}
    The soundness and completeness lemmas are proven by \emph{mutual} induction. 
    Since the declarative and the algorithmic systems do not depend on each other,
    we must introduce a uniform \emph{metric} on which the induction is conducted.
    We define the metric gradually, starting from the
    auxiliary function---the size of a judgment $\size{J}$.

    The size of \emph{declarative} and \emph{algorithmic} judgments is defined as
    a pair: the first component is the syntactic size of the terms used in the judgment. 
    The second component depends on the kind of judgment. For regular
    type inference judgments (such as $[[Γ ; Φ ⊢ v : iP]]$
    or $[[Γ ; Φ ⊨ c : iN]]$), it is always zero. For application inference judgments
    $([[Γ ; Φ ⊢ iN ● args ⇒> iM]]$ or $[[Γ ; Φ ; Θ ⊨ uN ● args ⇒> uM ⫤ Θ'; SC]]$), 
    it is equal to the number of prenex quantifiers of the head type $[[iN]]$.
    We need this adjustment to ensure the monotonicity of the metric, since the rules 
    \ruleref{\ottdruleATForallAppLabel} and \ruleref{\ottdruleDTForallAppLabel}
    only reduce the quantifiers in the head type but they do not change the list of arguments.

\begin{definition*}[Judgement Size]
    \label{def:decl-typing-size}
    For a declarative or an algorithmic typing judgment $J$, we define a metric $\size{J}$ as a pair of integers in the following way:

    \begin{multicols}{2}
    \begin{itemize}
        \item [$+$] $\size{[[Γ ; Φ ⊢ v : iP]]} = (\size{[[v]]}, 0)$;
        \item [$-$] $\size{[[Γ ; Φ ⊢ c : iN]]} = (\size{[[c]]}, 0)$;
        \item [$\bullet$] $\size{[[Γ ; Φ ⊢ iN ● args ⇒> iM]]}=$\\ 
            $(\size{[[args]]}, \npq{[[iN]]})$;
    \end{itemize}
    \columnbreak
    \begin{itemize}[leftmargin=*]
        \item [$+$] $\size{[[Γ ; Φ ⊨ v : iP]]} = (\size{[[v]]}, 0)$;
        \item [$-$] $\size{[[Γ ; Φ ⊨ c : iN]]} = (\size{[[c]]}, 0)$;
        \item [$\bullet$] $\size{[[Γ ; Φ ; Θ ⊨ uN ● args ⇒> uM ⫤ Θ'; SC]]} =$\\ 
             $(\size{[[args]]}, \npq{[[uN]]})$.
    \end{itemize}
    \end{multicols}

    Here $\size{[[v]]}$ and $\size{[[c]]}$ is the size of the 
    syntax tree of the term,
    and $\size{[[args]]}$ is the sum of sizes of the terms in $[[args]]$;
    and $\npq{[[iN]]}$ and $\npq{[[iP]]}$ represent the number of 
    prenex quantifiers, \ie
        \begin{itemize} \centering
            \item [$+$] $\npq{[[∃nas.iP]]} = |[[nas]]|$, if $[[iP ≠ ∃nbs.iP']]$,
            \item [$-$] $\npq{[[∀pas.iN]]} = |[[pas]]|$, if $[[iN ≠ ∀pbs.iN']]$.
        \end{itemize}
\end{definition*}

Notice that for \emph{algorithmic} inference system, $\size{J}$ decreases in
all the inductive steps, \ie for each inference rule, the size of the
premise judgments is strictly less than the size of the conclusion.
However, the \emph{declarative} inference system has rules
\ruleref{\ottdruleDTNEquivLabel} and \ruleref{\ottdruleDTPEquivLabel}, that
`step to' an equivalent type, and thus, technically, might keep the judgment
unchanged altogether.

To deal with this issue, we introduce the metric on the entire \emph{inference
trees} rather than on judgments, and plug into this metric the parameter that
certainly decreases in rules \ruleref{\ottdruleDTNEquivLabel} and
\ruleref{\ottdruleDTPEquivLabel}---the number of such nodes in the inference
tree. We denote this number as $\eqNodes{T}$. Then the final metric is defined
as a pair in the following way.

\begin{definition*}[Inference Tree Metric]
    For a tree $T$, inferring a declarative or an algorithmic judgement $J$, we define $\metric{T}$ as follows:
        \[
        \metric{T} = \begin{cases}
        (\size{J}, \eqNodes{T}) & \text{if } J \text{ represents a declarative judgement}, \\
        (\size{J}, 0) & \text{if } J \text{ represents an algorithmic judgement}.
        \end{cases}
        \]
    Here $\eqNodes{T}$ is the number of nodes in $T$ labeled with \ruleref{\ottdruleDTPEquivLabel} or \ruleref{\ottdruleDTNEquivLabel}.
\end{definition*}

This metric is suitable for mutual induction on the soundness and completeness
of the typing algorithm.  First, it is monotonous \wrt the inference rules,
and this way, we can always apply the induction hypothesis to premises 
of each rule. Second, the induction hypothesis is powerful enough,
so we can use the completeness of the algorithm in the soundness proof,
where required. For instance, to prove the soundness
of typing in case of $[[Γ; Φ ⊨ let x = v(args); c' : iN]]$,
we can assume, that  \emph{completeness}
holds for a term of shape $[[Γ; Φ ⊢ iM ● args ⇒> iK ]]$, since
$\size{args} < \size{let x = v(args); c'}$.
This is exactly what allows us to 
deal with the case of \ruleref{\ottdruleATAppLetLabel},
because then we can conclude that 
the inferred type (of a declarative judgment $[[Γ; Φ ⊢ iM ● args ⇒> ↑[uσ]uQ
]]$ constructed by the induction hypothesis) is unique.

